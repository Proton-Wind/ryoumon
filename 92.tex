\hakosyokika
\item
    \begin{mawarikomi}(10pt,0){200pt}{\input{./fig/fig092.tex}}
        図のように,一定波長の平面波の水面波を,波面と平行に並んだ間隔5\sftanni{cm}の2つのスリット$\mathrm{S_1}$および$\mathrm{S_2}$を通して干渉させた。$\mathrm{S_1}$を通りm$\mathrm{S_1}$と$\mathrm{S_2}$を結ぶ直線$\mathrm{S_1T}$に沿って水面の動きを調べたところ,波が弱めあって,水位がほとんど変化しない場所が2つだけ見つかった。そのうち,$\mathrm{S_1}$から遠い方を$\mathrm{A_1}$,$\mathrm{S_1}$から遠い方を$\mathrm{A_2}$とすると,$\mathrm{S_1}$から$\mathrm{A_1}$までの距離は$12$\sftanni{cm}であった。
        \begin{enumerate}
            \item 距離$\mathrm{S_1A_1}$と$\mathrm{S_2A_1}$の差は,波長の何倍か。また$\mathrm{S_1A_2}$と$\mathrm{S_2A_2}$との差は,波長の何倍か。
            \item この水面波の波長は何\sftanni{cm}か。
            \item 水面には強めあいの線(双曲線や直線)が何本生じているか。
            \item 次に,スリット$\mathrm{S_1}$は固定したまま,$\mathrm{S_2}$を動かし,$\mathrm{S_1S_2}$の間隔を広げていった。このとき,直線$\mathrm{S_1}$での,水位がほとんど変化しない点の個数は増すか,減るか。また,$\mathrm{A_1}$は$\mathrm{S_1}$に近づくか遠ざかるか。
        \end{enumerate}
    \end{mawarikomi}