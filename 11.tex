\hakosyokika
\item
    \begin{mawarikomi}{120pt}{\begin{zahyou*}[ul=4.5mm](-1,13)(-1,13)
	\def\A{(11,0.5)}
	\def\AD{(11,0)}
	\def\BD{(-3,0)}
	\def\B{(-3,0.5)}
	\def\T{(0,12)}
	\def\TL{(-0.5,12)}
	\def\BSW{(-0.5,-0.5)}
	\def\RD{(12,-0.5)}
	\def\R{(12,0)}
	\def\O{(0,0)}
	\def\C{(4,1)}
	\def\D{(5,1)}
	\def\E{(5,0.5)}
	\def\F{(4,0.5)}
	\Kaiten\AD\A{-38.3}\AA
	\Kaiten\AD\BD{-38.3}\BDB
	\Kaiten\AD\B{-38.3}\BB
	\Kaiten\AD\C{-38.3}\CC
	\Kaiten\AD\D{-38.3}\DD
	\Kaiten\AD\E{-38.3}\EE
	\Kaiten\AD\F{-38.3}\FF
	% \Drawline{\BD\B\A\AD\BD}
	\Drawline{\AD\BDB\BB\AA\AD}
	\Drawline{\CC\DD\EE\FF\CC}
	\Drawline{\T\O\R}
	% \Drawline{\C\D\E\F\C}
	\Nuritubusi*<0.22>{\TL\T\O\R\RD\BSW\TL}
	% \Nuritubusi[1]{\C\D\E\F\C}
	\Nuritubusi[1]{\CC\DD\EE\FF\CC}
	\Put\BB[ne]{B}
	\Put\CC[ne]{P}
	\Put\AA[ne]{A}
	\Kakukigou\BDB\AD\O<hankei=2>[w]{$\theta $}

\end{zahyou*}
}
    粗い水平な床となめらかで鉛直な壁に,質量$M$,長さ$\ell $の一様な棒ABを,床から角$\theta $だけ傾けて立てかけた。そして棒の中点に質量$m$の小物体Pを置いたところ,棒の表面が粗いため,Pは棒の上で静止し,棒も静止したままであった。A点で棒が床から受ける摩擦力の大きさは\Hako である。ただし,重力加速度の大きさを$g$とする。\\
    ~~また,棒と床との静止摩擦係数を$\mu $とすると,棒が静止していることから$\mu \geqq$\Hako の条件が成り立っている。Pの位置を少しずつ変えていくと,A点からの距離が$x$の位置に置いたとき棒がすべらずに静止する限界となった。$x=$\Hako である。
    \end{mawarikomi}