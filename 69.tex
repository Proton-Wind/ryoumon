\hakosyokika
\item
    \begin{mawarikomi}(10pt,0){80pt}{{\small
\begin{zahyou*}[ul=6mm](0,4)(-1,14)
    \tenretu*{
        A(4,0);
        B(4,6);
        C(0,6);
        D(0,0);
        E(3,0.5);
        F(3,4);
        G(1,4);
        H(1,0.5);
        I(3,2.5);
        J(1,2.5);
        K(3,9.5);
        L(3,13);
        M(1,13);
        N(1,9.5);
        O(3,10);
        P(1,10);
        Q(4,13);
        R(0,13);
        S(0,9);
        T(4,9);
        U(1.5,13);
        V(1.5,10);
        W(2,6);
        X(2,4);
        Y(1.5,4);
        Z(1.5,2.5);
    }
    \Nuritubusi[0.3]{\A\B\C\D\A}
    \Nuritubusi[0.3]{\Q\R\S\T\Q}
    \Nuritubusi[0]{\I\F\G\J\I}
    \Nuritubusi[0]{\L\M\P\O\L}
    \Drawlines{\B\C;\I\J;\O\P;\B\C;\Q\R}
    {\thicklines
        \Drawlines{\E\F\G\H;\K\L\M\N}
    }
    \HenKo<henkotype=parallel
    ,henkoH=1ex
    ,yazirusi=b>{\V}{\U}{$d$}

    \HenKo<henkotype=parallel
    ,henkoH=1ex
    ,yazirusi=b>{\X}{\W}{$h$}

    \HenKo<henkotype=parallel
    ,henkoH=1ex
    ,putoption={(3.2mm,0)}
    ,yazirusi=b>{\Z}{\Y}{$\bunsuu{1}{2}d$}
    
    \put(1.6,8.5){図1}
    \put(1.6,-0.5){図2}
    \put(0.6,6.5){$P_0$}
    \put(0.6,13.5){$P_0$}
    \put(0.2,11){$\rho $}
    \put(0.2,4.5){$\rho $}
    \put(3.2,11.5){$M$}
    \put(3.2,2){$M$}
    % \drawline(0.7,3)(1.3,3)
    % \Put\BB(20pt,0pt)[b]{気体}
    % \Put\N(0pt,0pt)[b]{B}
    % {\thicklines
    % \Drawlines{\G\B}
    % }
    % \Put\B(0,-15pt)[b]{$3p$}
    % \Put\A(0,10pt)[b]{1モル}
    % \Put\B(0,10pt)[b]{2モル}
    % \Put\K(-3pt,0pt)[b]{K}
\end{zahyou*}}
}
        断面積$S$,質量$M$の一端を閉じた円筒が,開口部を下にし,上端は水面に一致して鉛直に静止している。円筒には鉛直下向きに外力が加えられている。円筒の内部には気体が入っており,円筒の上端から内部の水面までの距離を$d$とする。円筒の厚さ,内部の気体の質量,水の蒸発は無視する。大気圧を$P_0$,水の密度を$\rho $,重力加速度の大きさを$g$とする。
        \begin{Enumerate}
            \item 外力の大きさを求めよ。
        \end{Enumerate}
        ~~次に円筒を深さ$h$の位置まで沈めると,外力を加えなくても円筒は静止した(図2)。このときの内部の気体の高さは$\bunsuu{1}{2}d$であった。
        \begin{Enumerate*}
            \item 円筒の質量$M$を$P_0$,$\rho $,$S$,$d$,$g$の中から必要なものを用いて表せ。
            \item 円筒内の気体の変化は等温変化とみなせるものとする。$h$を$P_0$,$\rho $,$S$,$d$,$g$の中から必要なものを用いて表せ。
            \item 円筒内の気体の変化が断熱変化とみなせる場合を考える。円筒が静止できる深さ$h$は(3)で求めた値より大きいか,小さいか。
        \end{Enumerate*}
    \end{mawarikomi}