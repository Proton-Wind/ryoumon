\hakosyokika
\item
    \begin{mawarikomi}{150pt}{\begin{zahyou*}[ul=4.5mm](-1,13)(-1,10)

	\def\T{(2,10)}
	\def\TD{(2,0)}
	\def\TL{(1.8,10)}
	\def\TDL{(1.8,0)}
	\def\A{(2,2)}
	\def\AD{(2,1.8)}
	\def\B{(12,2)}
	\def\BD{(12,1.8)}
	\kandk\A{90}\B{150}\C
	\def\M{(12,0)}
	\Drawline{\B\A}
	\Drawline{\T\TD}
	\Drawline{\B\M}
	\Drawline{\AD\A\B\BD\AD}
	\Drawline{\A\B\C\A}
	% \scriptsize
	\HenKo[0]<henkoH=3ex>\B\C{糸}
	\Put\A[w]{A}
	\Put\B[e]{B}
	\Put\C[w]{C}
	\Put\M(0pt,-8pt)[t]{M}
	\En*[1]\M{0.3}
	\En\M{0.3}
	\Kakukigou\C\B\A<hankei=2>[w]{$30\Deg$}
	\Nuritubusi*<0.22>{\TD\T\TL\TDL\TD}

\end{zahyou*}
}
    長さが$\ell $で質量が$M$の一様な棒ABをA端を鉛直な粗い壁面に押し当て,B端を糸で結び,糸の他端をC点に固定する。
    B端に質量$M$のおもりMをつり下げた状態で,棒はA点で壁に垂直になっている。糸BCと棒ABのなす角度は$30\Deg$であり,重力加速度の大きさを$g$とする。\\
    ~~Aまわりの力のモーメントのつり合いより,糸の張力は\Hako である。また,A点での垂直抗力は\Hako であり,静止摩擦力は\Hako である。\\
    ~~Mをつり下げる位置をB点からAの方にゆっくりと移動していくと,MがB点から$x$離れたPの位置に来たとき棒のA端がすべり始めた。壁面と棒の間の静止摩擦係数を$\mu $壁面の垂直抗力を$N$とすると,棒がすべり出す直前では,棒のBまわりでの力のモーメントのつり合いから,糸の張力は$N$,$M$,$\ell $,$x$,$g$,$\mu $を用いて表すと,$\bunsuu{1}{2}Mg\ell + \Hako =0$となる。この式と水平方向での力のつり合いから,糸の張力は$M$,$\ell $,$x$,$g$,$\mu $を用いて\Hako と表される。そして,PB間の距離$x$は$\ell $,$\mu $を用いて表すと,\Hako である。
    \end{mawarikomi}