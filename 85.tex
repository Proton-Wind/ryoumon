\hakosyokika
\item
    \begin{mawarikomi}(10pt,0){160pt}{%WinTpicVersion4.32a
{\unitlength 0.1in%
\begin{picture}(20.2756,15.1083)(3.9370,-17.0768)%
% LINE 1 0 3 0 Black White  
% 2 400 800 2400 800
% 
\special{pn 13}%
\special{pa 394 787}%
\special{pa 2362 787}%
\special{fp}%
% LINE 1 0 3 0 Black White  
% 2 2400 1600 400 1600
% 
\special{pn 13}%
\special{pa 2362 1575}%
\special{pa 394 1575}%
\special{fp}%
% LINE 2 0 3 0 Black White  
% 2 400 1600 1780 800
% 
\special{pn 8}%
\special{pa 394 1575}%
\special{pa 1752 787}%
\special{fp}%
% LINE 2 0 3 0 Black White  
% 2 580 800 1180 200
% 
\special{pn 8}%
\special{pa 571 787}%
\special{pa 1161 197}%
\special{fp}%
% LINE 2 0 3 0 Black White  
% 2 980 800 1580 200
% 
\special{pn 8}%
\special{pa 965 787}%
\special{pa 1555 197}%
\special{fp}%
% LINE 2 0 3 0 Black White  
% 2 1380 800 1980 200
% 
\special{pn 8}%
\special{pa 1358 787}%
\special{pa 1949 197}%
\special{fp}%
% LINE 2 0 3 0 Black White  
% 2 2180 800 2380 600
% 
\special{pn 8}%
\special{pa 2146 787}%
\special{pa 2343 591}%
\special{fp}%
% LINE 2 0 3 0 Black White  
% 2 1780 800 2380 200
% 
\special{pn 8}%
\special{pa 1752 787}%
\special{pa 2343 197}%
\special{fp}%
% LINE 2 0 3 0 Black White  
% 4 1380 800 400 1370 1000 800 400 1150
% 
\special{pn 8}%
\special{pa 1358 787}%
\special{pa 394 1348}%
\special{fp}%
\special{pa 984 787}%
\special{pa 394 1132}%
\special{fp}%
% LINE 2 0 3 0 Black White  
% 2 600 800 400 920
% 
\special{pn 8}%
\special{pa 591 787}%
\special{pa 394 906}%
\special{fp}%
% LINE 2 0 3 0 Black White  
% 2 800 1600 2180 800
% 
\special{pn 8}%
\special{pa 787 1575}%
\special{pa 2146 787}%
\special{fp}%
% LINE 2 0 3 0 Black White  
% 4 1200 1600 2400 910 1850 1290 1850 1290
% 
\special{pn 8}%
\special{pa 1181 1575}%
\special{pa 2362 896}%
\special{fp}%
\special{pa 1821 1270}%
\special{pa 1821 1270}%
\special{fp}%
% LINE 2 0 3 0 Black White  
% 2 1600 1600 2400 1140
% 
\special{pn 8}%
\special{pa 1575 1575}%
\special{pa 2362 1122}%
\special{fp}%
% LINE 2 0 3 0 Black White  
% 2 2000 1600 2400 1370
% 
\special{pn 8}%
\special{pa 1969 1575}%
\special{pa 2362 1348}%
\special{fp}%
% CIRCLE 2 0 3 0 Black White  
% 4 1380 800 1580 800 2380 800 1980 200
% 
\special{pn 8}%
\special{ar 1358 787 197 197 5.4977871 6.2831853}%
% STR 2 0 3 0 Black White  
% 4 1610 600 1610 700 2 0 0 0
% 45\Deg
\put(15.8465,-6.8898){\makebox(0,0)[lb]{45\Deg}}%
% CIRCLE 2 0 3 0 Black White  
% 4 1380 800 1580 800 780 800 370 1380
% 
\special{pn 8}%
\special{ar 1358 787 197 197 2.6203166 3.1415927}%
% STR 2 0 3 0 Black White  
% 4 970 860 970 960 2 0 0 0
% 30\Deg
\put(9.5472,-9.4488){\makebox(0,0)[lb]{30\Deg}}%
% STR 2 0 3 0 Black White  
% 4 2490 700 2490 800 5 0 0 0
% A
\put(24.5079,-7.8740){\makebox(0,0){A}}%
% STR 2 0 3 0 Black White  
% 4 2490 1500 2490 1600 5 0 0 0
% B
\put(24.5079,-15.7480){\makebox(0,0){B}}%
% STR 2 0 3 0 Black White  
% 4 2180 300 2180 400 5 0 1 0
% 媒質1
\put(21.4567,-3.9370){\makebox(0,0){{\colorbox[named]{White}{媒質1}}}}%
% STR 2 0 3 0 Black White  
% 4 2180 1100 2180 1200 5 0 1 0
% 媒質2
\put(21.4567,-11.8110){\makebox(0,0){{\colorbox[named]{White}{媒質2}}}}%
% STR 2 0 3 0 Black White  
% 4 2180 1700 2180 1800 5 0 1 0
% 媒質3
\put(21.4567,-17.7165){\makebox(0,0){{\colorbox[named]{White}{媒質3}}}}%
\end{picture}}%
}
    図のように,平行な境界面A,Bで接した3種の媒質1,2,3がある。媒質1から入射した平面波の一部が屈折して媒質2へ入っていく。
    図中の平行線は入射波と屈折波の波面を表している。
     媒質1における波の波長は$2.0$\sftanni{cm},振動数は$25$\sftanni{Hz}である。
        \begin{enumerate}
            \item 媒質1に対する媒質2の屈折率はいくらか。
            \item 媒質1の中での波の速さは何\sftanni{cm/s}か。
            \item 媒質2の中での,波の波長は\sftanni{cm}か。振動数は何\sftanni{Hz}か。速さは何\sftanni{cm/s}か。
            \item 媒質1に対する媒質3の屈折率は0.80であった。媒質2に対する媒質3の屈折率はいくらか。
            \item 境界面Bで反射された波は,媒質2を通って,その一部が媒質1へもどる。そのときの屈折角は何度か。
        \end{enumerate}
    \end{mawarikomi}