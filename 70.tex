\hakosyokika
\item
    \begin{mawarikomi}(10pt,0){90pt}{{\small
\begin{zahyou*}[ul=6mm](0,5)(0,8)
    \tenretu*{
        A(1,0.5);
        B(5,0.5);
        C(5,1);
        D(1,1);
        E(2.5,1);
        F(3.5,1);
        G(3.5,1.5);
        H(2.5,1.5);
        I(3.1,3.8);
        J(2.9,3.8);
        K(3.1,4.03);
        L(2.9,4.03);
        M(3,6);
        N(1.66,4.5);
        O(4.34,4.5);
    }
    \Nuritubusi*{\E\F\G\H\E}
    \Nuritubusi[0.3]{\A\B\C\D\A}
    \En\M{2}
    \Nuritubusi[0]{\J\I\K\L\J}
    % \Drawlines{\K\I;\L\J;\N\H;\O\G;\C\D}
    \Drawlines{\N\H;\O\G;\K\I;\L\J;\E\F\G\H\E;\C\D}

    \put(4.5,2.5){外気}
    \put(4.0,1.5){$P_0$,$T_0$,$\rho _0$}
    \Put\M{$V$}
\end{zahyou*}}
}
        \begin{enumerate}
            \item 気体の圧力$P$,密度$\rho$,絶対温度$T$の間には,状態方程式より,$P=a \rho T$の関係が成り立つ。定数$\alpha $を気体定数$R$と1モルの気体の質量$m_0$で表せ。
            \item 熱気球がある。風船部の体積は$V$\tanni{m^3}であり,風船部内の空気(内部空気)を除いた全体の質量は$M$\tanni{kg}である。内部空気の圧力は外気圧に等しく,温度は自由に調節できる。地表での外気の圧力を$P_0$\tanni{Pa},気温を$T_0$\tanni{K},密度を$\rho _0$\tanni{kg/m^3}とする。
            \begin{enumerate}
                \item 内部空気を加熱していくと,気球は地表に静止したまま,温度が$T$\tanni{K}となった。内部空気の密度$\rho $\tanni{kg/m^3}を求めよ。
                \item 内部空気をさらに加熱し,温度が$T_1$\tanni{K}より高くなると,気球は地表より浮上する。$T_1$\tanni{K}を求めよ。
                \item 気球が浮上した後,内部空気の温度を$\alpha T_0$\tanni{K}$(\alpha >1)$としたところ,気球はある高度で静止した。そこでの外気の圧力は$\beta P_0$\tanni{Pa}であった。内部空気の密度$\rho '$\tanni{kg/m^3},および外気の密度$\rho '_0$\tanni{kg/m^3}を求めよ。
            \end{enumerate}
        \end{enumerate}
    \end{mawarikomi}