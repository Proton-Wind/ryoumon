\hakosyokika
\item
    \begin{mawarikomi}(10pt,0pt){180pt}{
        %WinTpicVersion4.32a
{\unitlength 0.1in%
\begin{picture}(20.0000,10.1400)(2.0000,-11.0000)%
% LINE 2 0 3 0 Black White  
% 2 200 800 700 1100
% 
\special{pn 8}%
\special{pa 200 800}%
\special{pa 700 1100}%
\special{fp}%
% LINE 2 0 3 0 Black White  
% 2 200 800 700 1100
% 
\special{pn 8}%
\special{pa 200 800}%
\special{pa 700 1100}%
\special{fp}%
% LINE 2 0 3 0 Black White  
% 2 700 1100 2200 1100
% 
\special{pn 8}%
\special{pa 700 1100}%
\special{pa 2200 1100}%
\special{fp}%
% LINE 2 0 3 0 Black White  
% 2 820 900 450 675
% 
\special{pn 8}%
\special{pa 820 900}%
\special{pa 450 675}%
\special{fp}%
% POLYGON 2 0 2 0 Black White  
% 5 565 720 520 745 720 860 762 834 565 720
% 
\special{pn 0}%
\special{sh 0}%
\special{pa 565 720}%
\special{pa 520 745}%
\special{pa 720 860}%
\special{pa 762 834}%
\special{pa 565 720}%
\special{ip}%
\special{pn 8}%
\special{pa 565 720}%
\special{pa 520 745}%
\special{pa 720 860}%
\special{pa 762 834}%
\special{pa 565 720}%
\special{pa 520 745}%
\special{fp}%
% LINE 2 0 3 0 Black White  
% 2 820 900 1920 400
% 
\special{pn 8}%
\special{pa 820 900}%
\special{pa 1920 400}%
\special{fp}%
% LINE 2 0 3 0 Black White  
% 2 820 900 1220 900
% 
\special{pn 8}%
\special{pa 820 900}%
\special{pa 1220 900}%
\special{fp}%
% CIRCLE 2 0 3 0 Black White  
% 4 820 900 1020 900 1920 900 1920 400
% 
\special{pn 8}%
\special{ar 820 900 200 200 5.8565578 6.2831853}%
% STR 2 0 3 0 Black White  
% 4 1135 780 1135 830 5 0 0 0
% $\theta$
\put(11.3500,-8.3000){\makebox(0,0){$\theta$}}%
% LINE 2 0 3 0 Black White  
% 2 460 675 1560 175
% 
\special{pn 8}%
\special{pa 460 675}%
\special{pa 1560 175}%
\special{fp}%
% POLYGON 2 0 2 0 Black White  
% 5 1126 351 1554 586 1554 606 1126 366 1126 351
% 
\special{pn 0}%
\special{sh 0}%
\special{pa 1126 351}%
\special{pa 1554 586}%
\special{pa 1554 606}%
\special{pa 1126 366}%
\special{pa 1126 351}%
\special{ip}%
\special{pn 8}%
\special{pa 1126 351}%
\special{pa 1554 586}%
\special{pa 1554 606}%
\special{pa 1126 366}%
\special{pa 1126 351}%
\special{pa 1554 586}%
\special{fp}%
% POLYGON 2 0 2 0 Black White  
% 6 1553 585 1553 585 1553 605 1593 588 1593 568 1553 585
% 
\special{pn 0}%
\special{sh 0}%
\special{pa 1553 585}%
\special{pa 1553 585}%
\special{pa 1553 605}%
\special{pa 1593 588}%
\special{pa 1593 568}%
\special{pa 1553 585}%
\special{ip}%
\special{pn 8}%
\special{pa 1553 585}%
\special{pa 1553 605}%
\special{pa 1593 588}%
\special{pa 1593 568}%
\special{pa 1553 585}%
\special{fp}%
% POLYGON 2 0 2 0 Black White  
% 5 1126 352 1553 586 1593 569 1161 338 1126 352
% 
\special{pn 0}%
\special{sh 0}%
\special{pa 1126 352}%
\special{pa 1553 586}%
\special{pa 1593 569}%
\special{pa 1161 338}%
\special{pa 1126 352}%
\special{ip}%
\special{pn 8}%
\special{pa 1126 352}%
\special{pa 1553 586}%
\special{pa 1593 569}%
\special{pa 1161 338}%
\special{pa 1126 352}%
\special{pa 1553 586}%
\special{fp}%
% VECTOR 2 0 3 0 Black White  
% 4 1600 320 1427 233 1599 320 1836 438
% 
\special{pn 8}%
\special{pa 1600 320}%
\special{pa 1427 233}%
\special{fp}%
\special{sh 1}%
\special{pa 1427 233}%
\special{pa 1478 281}%
\special{pa 1475 257}%
\special{pa 1496 245}%
\special{pa 1427 233}%
\special{fp}%
\special{pa 1599 320}%
\special{pa 1836 438}%
\special{fp}%
\special{sh 1}%
\special{pa 1836 438}%
\special{pa 1785 390}%
\special{pa 1788 414}%
\special{pa 1767 426}%
\special{pa 1836 438}%
\special{fp}%
% STR 2 0 3 0 Black White  
% 4 1617 317 1617 327 5 0 1 0
% $L$
\put(16.1700,-3.2700){\makebox(0,0){{\colorbox[named]{White}{$L$}}}}%
% VECTOR 0 0 3 0 Black White  
% 2 797 407 797 727
% 
\special{pn 20}%
\special{pa 797 407}%
\special{pa 797 727}%
\special{fp}%
\special{sh 1}%
\special{pa 797 727}%
\special{pa 817 660}%
\special{pa 797 674}%
\special{pa 777 660}%
\special{pa 797 727}%
\special{fp}%
% VECTOR 0 0 3 0 Black White  
% 2 1076 246 1076 566
% 
\special{pn 20}%
\special{pa 1076 246}%
\special{pa 1076 566}%
\special{fp}%
\special{sh 1}%
\special{pa 1076 566}%
\special{pa 1096 499}%
\special{pa 1076 513}%
\special{pa 1056 499}%
\special{pa 1076 566}%
\special{fp}%
% VECTOR 0 0 3 0 Black White  
% 2 1356 86 1356 406
% 
\special{pn 20}%
\special{pa 1356 86}%
\special{pa 1356 406}%
\special{fp}%
\special{sh 1}%
\special{pa 1356 406}%
\special{pa 1376 339}%
\special{pa 1356 353}%
\special{pa 1336 339}%
\special{pa 1356 406}%
\special{fp}%
% STR 2 0 3 0 Black White  
% 4 956 266 956 286 5 0 0 0
% $B$
\put(9.5600,-2.8600){\makebox(0,0){$B$}}%
% STR 2 0 3 0 Black White  
% 4 430 636 430 656 3 0 0 0
% a
\put(4.3000,-6.5600){\makebox(0,0)[rb]{a}}%
% STR 2 0 3 0 Black White  
% 4 828 892 828 912 1 0 0 0
% b
\put(8.2800,-9.1200){\makebox(0,0)[lt]{b}}%
\end{picture}}%

    }
    金属棒を間隔$L$\tanni{m}で平行に並べてレールをつくり,水平面に対して傾斜角$\theta $で設置した。レールの下端a,bは抵抗値$R$\tanni{\Omega }の抵抗でつないである。一様な磁束密度$B$\tanni{T}の磁場を鉛直下向きにかけた状態で,質量$m$\tanni{kg}の金属棒をレールの上に水平に静かに置いた。棒とレールとの摩擦と電気抵抗は無視でき,棒はレールに対して,常に直交しているものとする。重力加速度の大きさを$g$\tanni{m/s^2}とする。
        \begin{enumerate}
            \item 棒が滑り下りるとき,抵抗に流れる電流の向きは図中のa$\rightarrow$bとb$\rightarrow a$のどちらか。
            \item 棒が速さ$v$\tanni{m/s}で動いているとき,抵抗に流れる電流の大きさを求めよ。
            \item 十分に時間が経過した後の棒の速さを求めよ。このとき,棒はレール上にあるものとする。
            \item (3)の状態のとき,抵抗で単位時間あたり発生する熱エネルギー$Q$と重力が棒にする仕事率$P$との比$\bunsuu{Q}{P}$を求めよ。
        \end{enumerate}
    \end{mawarikomi}