\hakosyokika
\item
    \begin{mawarikomi}(10pt,0pt){180pt}{
        \input{./fig/fig129.tex}
    }
    金属棒を間隔$L$\tanni{m}で平行に並べてレールをつくり,水平面に対して傾斜角$\theta $で設置した。レールの下端a,bは抵抗値$R$\tanni{\Omega }の抵抗でつないである。一様な磁束密度$B$\tanni{T}の磁場を鉛直下向きにかけた状態で,質量$m$\tanni{kg}の金属棒をレールの上に水平に静かに置いた。棒とレールとの摩擦と電気抵抗は無視でき,棒はレールに対して,常に直交しているものとする。重力加速度の大きさを$g$\tanni{m/s^2}とする。
        \begin{enumerate}
            \item 棒が滑り下りるとき,抵抗に流れる電流の向きは図中のa$\rightarrow$bとb$\rightarrow a$のどちらか。
            \item 棒が速さ$v$\tanni{m/s}で動いているとき,抵抗に流れる電流の大きさを求めよ。
            \item 十分に時間が経過した後の棒の速さを求めよ。このとき,棒はレール上にあるものとする。
            \item (3)の状態のとき,抵抗で単位時間あたり発生する熱エネルギー$Q$と重力が棒にする仕事率$P$との比$\bunsuu{Q}{P}$を求めよ。
        \end{enumerate}
    \end{mawarikomi}