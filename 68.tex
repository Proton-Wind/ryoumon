\hakosyokika
\item
    \begin{mawarikomi}(10pt,0){160pt}{{\small
\begin{zahyou*}[ul=4mm](0,15)(0,6)
    \def\Fx{0.2*(cos(T)+T/3.5)+7.2}
	\def\Fy{-0.3*sin(T)+2.2}
	\BGurafu\Fx\Fy{$pi}{34*$pi}
    \def\A{(1.5,0)}
    \def\B{(14,0)}
    \def\C{(14,5)}
    \def\D{(13.5,5)}
    \def\E{(13.5,0.5)}
    \def\F{(1.5,0.5)}
    \def\G{(2,0.5)}
    \def\H{(10,0.5)}
    \def\I{(10,1)}
    \def\J{(2.5,1)}
    \def\K{(2.5,3.5)}
    \def\L{(10,3.5)}
    \def\M{(10,4)}
    \def\N{(2,4)}
    \def\O{(6.75,1)}
    \def\P{(7.25,1)}
    \def\Q{(7.25,3.5)}
    \def\R{(6.75,3.5)}
    \def\S{(1,1.5)}
    \def\T{(3,1.5)}
    \def\U{(3,4.5)}
    \def\V{(2,4.5)}
    \def\W{(2.2,5)}
    \def\X{(1.2,4.5)}
    \def\Y{(1,4.5)}
    \def\SU{(1,2.8)}
    \def\YD{(1,3)}
    \def\UD{(3,3)}
    \def\TU{(3,2)}
    \def\AA{(2.75,2)}
    \def\BB{(3.25,2)}
    \def\CC{(3.25,3)}
    \def\DD{(2.75,3)}
    \def\ED{(13.5,0)}
    \Nuritubusi[1]{\O\P\Q\R\O}
    \Nuritubusi*{\A\ED\E\F\A}
    \Nuritubusi*{\ED\B\C\D\ED}
    \Drawlines{\D\E\F;\H\I\J\K\L\M\N\G;\SU\S\T\TU;\UD\U\V;\W\X\Y\YD;\AA\BB\CC\DD\AA}
    {\thicklines
        \drawline(0.5,2.8)(0.9,2.8)
    }
    \drawline(0,3)(0.8,3)
    \Put\BB(20pt,0pt)[b]{気体}
    % \Put\N(0pt,0pt)[b]{B}
    % {\thicklines
    % \Drawlines{\G\B}
    % }
    % \Put\B(0,-15pt)[b]{$3p$}
    % \Put\A(0,10pt)[b]{1モル}
    % \Put\B(0,10pt)[b]{2モル}
    % \Put\K(-3pt,0pt)[b]{K}
\end{zahyou*}}
}
    $n$モルの単原子分子からなる理想気体が,水平なばね振り子(ばね定数は$k$\tanni{N/m})につながれた断面積$S$\tanni{m^2}のピストンによってシリンダー(床に固定)内に封入されている。
    ピストン,シリンダーはともに断熱材でつくられており,ピストンはなめらかに動くものとする。さて,ヒーターにより気体を熱したところ,気体はゆっくりと膨張し,加熱前の体積$V_0$\tanni{m^3}の2倍になった。
    加熱前のばねは自然の長さであり,気体定数を$R$\tanni{J/(mol\cdot K)},大気圧を$P_0$\tanni{Pa}とする。    
        \begin{enumerate}
            \item 加熱前の気体の温度を求めよ。
            \item 加熱し始めてからピストンが移動した距離を$x$\tanni{m}として,そのときの気体の圧力$P$\tanni{Pa}を$x$の関数として表せ。また,$P$を気体の体積$V$の関数として表せ。
            \item 2倍の体積になったときの気体の温度を求めよ。
            \item 2倍の体積になるまでに気体がした仕事を求めよ。
            \item 気体に加えた熱量を求めよ。
        \end{enumerate}
    \end{mawarikomi}