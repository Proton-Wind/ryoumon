\item
        \begin{mawarikomi}{80pt}{%WinTpicVersion4.32a
{\unitlength 0.1in%
\begin{picture}(15.3500,25.9000)(2.7500,-29.0000)%
% LINE 2 0 3 0 Black White  
% 2 610 400 1410 400
% 
\special{pn 8}%
\special{pa 610 400}%
\special{pa 1410 400}%
\special{fp}%
% LINE 3 0 3 0 Black White  
% 28 1310 310 1220 400 1250 310 1160 400 1190 310 1100 400 1130 310 1040 400 1070 310 980 400 1010 310 920 400 950 310 860 400 890 310 800 400 830 310 740 400 770 310 680 400 710 310 620 400 650 310 610 350 1370 310 1280 400 1410 330 1340 400
% 
\special{pn 4}%
\special{pa 1310 310}%
\special{pa 1220 400}%
\special{fp}%
\special{pa 1250 310}%
\special{pa 1160 400}%
\special{fp}%
\special{pa 1190 310}%
\special{pa 1100 400}%
\special{fp}%
\special{pa 1130 310}%
\special{pa 1040 400}%
\special{fp}%
\special{pa 1070 310}%
\special{pa 980 400}%
\special{fp}%
\special{pa 1010 310}%
\special{pa 920 400}%
\special{fp}%
\special{pa 950 310}%
\special{pa 860 400}%
\special{fp}%
\special{pa 890 310}%
\special{pa 800 400}%
\special{fp}%
\special{pa 830 310}%
\special{pa 740 400}%
\special{fp}%
\special{pa 770 310}%
\special{pa 680 400}%
\special{fp}%
\special{pa 710 310}%
\special{pa 620 400}%
\special{fp}%
\special{pa 650 310}%
\special{pa 610 350}%
\special{fp}%
\special{pa 1370 310}%
\special{pa 1280 400}%
\special{fp}%
\special{pa 1410 330}%
\special{pa 1340 400}%
\special{fp}%
% LINE 2 0 3 0 Black White  
% 2 1010 400 1010 1000
% 
\special{pn 8}%
\special{pa 1010 400}%
\special{pa 1010 1000}%
\special{fp}%
% DOT 0 0 3 0 Black White  
% 1 1010 1000
% 
\special{pn 4}%
\special{sh 1}%
\special{ar 1010 1000 16 16 0 6.2831853}%
% CIRCLE 2 0 3 0 Black White  
% 4 1010 1000 1410 1000 1410 1000 1410 1000
% 
\special{pn 8}%
\special{ar 1010 1000 400 400 0.0000000 6.2831853}%
% LINE 2 0 3 0 Black White  
% 4 610 1000 610 2000 710 2100 710 2100
% 
\special{pn 8}%
\special{pa 610 1000}%
\special{pa 610 2000}%
\special{fp}%
\special{pa 710 2100}%
\special{pa 710 2100}%
\special{fp}%
% CIRCLE 2 0 1 0 Black Black  
% 4 610 2100 710 2100 710 2100 710 2100
% 
\special{sh 0.300}%
\special{ia 610 2100 100 100 0.0000000 6.2831853}%
\special{pn 8}%
\special{ar 610 2100 100 100 0.0000000 6.2831853}%
% LINE 2 0 3 0 Black White  
% 2 610 2200 610 2800
% 
\special{pn 8}%
\special{pa 610 2200}%
\special{pa 610 2800}%
\special{fp}%
% LINE 2 0 3 0 Black White  
% 2 410 2800 810 2800
% 
\special{pn 8}%
\special{pa 410 2800}%
\special{pa 810 2800}%
\special{fp}%
% LINE 3 0 3 0 Black White  
% 16 500 2800 410 2890 560 2800 460 2900 620 2800 520 2900 680 2800 580 2900 740 2800 640 2900 800 2800 700 2900 810 2850 760 2900 440 2800 410 2830
% 
\special{pn 4}%
\special{pa 500 2800}%
\special{pa 410 2890}%
\special{fp}%
\special{pa 560 2800}%
\special{pa 460 2900}%
\special{fp}%
\special{pa 620 2800}%
\special{pa 520 2900}%
\special{fp}%
\special{pa 680 2800}%
\special{pa 580 2900}%
\special{fp}%
\special{pa 740 2800}%
\special{pa 640 2900}%
\special{fp}%
\special{pa 800 2800}%
\special{pa 700 2900}%
\special{fp}%
\special{pa 810 2850}%
\special{pa 760 2900}%
\special{fp}%
\special{pa 440 2800}%
\special{pa 410 2830}%
\special{fp}%
% LINE 2 0 3 0 Black White  
% 14 1410 1000 1410 1800 1410 1800 1010 2400 1010 2400 1810 2400 1810 2400 1410 1800 1010 2400 1010 2500 1010 2500 1810 2500 1810 2500 1810 2400
% 
\special{pn 8}%
\special{pa 1410 1000}%
\special{pa 1410 1800}%
\special{fp}%
\special{pa 1410 1800}%
\special{pa 1010 2400}%
\special{fp}%
\special{pa 1010 2400}%
\special{pa 1810 2400}%
\special{fp}%
\special{pa 1810 2400}%
\special{pa 1410 1800}%
\special{fp}%
\special{pa 1010 2400}%
\special{pa 1010 2500}%
\special{fp}%
\special{pa 1010 2500}%
\special{pa 1810 2500}%
\special{fp}%
\special{pa 1810 2500}%
\special{pa 1810 2400}%
\special{fp}%
% BOX 2 0 3 0 Black White  
% 2 1310 2400 1510 2100
% 
\special{pn 8}%
\special{pa 1310 2400}%
\special{pa 1510 2400}%
\special{pa 1510 2100}%
\special{pa 1310 2100}%
\special{pa 1310 2400}%
\special{pa 1510 2400}%
\special{fp}%
% LINE 3 0 3 0 Black White  
% 14 1510 2270 1380 2400 1510 2210 1320 2400 1510 2150 1310 2350 1500 2100 1310 2290 1440 2100 1310 2230 1380 2100 1310 2170 1510 2330 1440 2400
% 
\special{pn 4}%
\special{pa 1510 2270}%
\special{pa 1380 2400}%
\special{fp}%
\special{pa 1510 2210}%
\special{pa 1320 2400}%
\special{fp}%
\special{pa 1510 2150}%
\special{pa 1310 2350}%
\special{fp}%
\special{pa 1500 2100}%
\special{pa 1310 2290}%
\special{fp}%
\special{pa 1440 2100}%
\special{pa 1310 2230}%
\special{fp}%
\special{pa 1380 2100}%
\special{pa 1310 2170}%
\special{fp}%
\special{pa 1510 2330}%
\special{pa 1440 2400}%
\special{fp}%
% STR 2 0 3 0 Black White  
% 4 1410 1900 1410 2000 5 0 0 0
% B
\put(14.1000,-20.0000){\makebox(0,0){B}}%
% STR 2 0 3 0 Black White  
% 4 1610 2190 1610 2290 5 0 0 0
% $M$
\put(16.1000,-22.9000){\makebox(0,0){$M$}}%
% STR 2 0 3 0 Black White  
% 4 1710 2460 1710 2560 5 0 0 0
% $m$
\put(17.1000,-25.6000){\makebox(0,0){$m$}}%
% STR 2 0 3 0 Black White  
% 4 1620 1860 1620 1960 5 0 0 0
% 糸
\put(16.2000,-19.6000){\makebox(0,0){糸}}%
% STR 2 0 3 0 Black White  
% 4 1230 2490 1230 2590 5 0 0 0
% 板
\put(12.3000,-25.9000){\makebox(0,0){板}}%
% STR 2 0 3 0 Black White  
% 4 1520 1330 1520 1430 5 0 0 0
% $\alpha $
\put(15.2000,-14.3000){\makebox(0,0){$\alpha $}}%
% STR 2 0 3 0 Black White  
% 4 510 1330 510 1430 5 0 0 0
% $\alpha $
\put(5.1000,-14.3000){\makebox(0,0){$\alpha $}}%
% STR 2 0 3 0 Black White  
% 4 510 2330 510 2430 5 0 0 0
% $\beta $
\put(5.1000,-24.3000){\makebox(0,0){$\beta $}}%
% STR 2 0 3 0 Black White  
% 4 420 1840 420 1940 5 0 0 0
% A
\put(4.2000,-19.4000){\makebox(0,0){A}}%
% STR 2 0 3 0 Black White  
% 4 420 2130 420 2230 5 0 0 0
% $m$
\put(4.2000,-22.3000){\makebox(0,0){$m$}}%
% STR 2 0 3 0 Black White  
% 4 910 2770 910 2870 5 0 0 0
% 床
\put(9.1000,-28.7000){\makebox(0,0){床}}%
% STR 2 0 3 0 Black White  
% 4 1090 420 1090 520 5 0 0 0
% $\gamma $
\put(10.9000,-5.2000){\makebox(0,0){$\gamma $}}%
\end{picture}}%
}
            天井から糸$\gamma $でつるされた定滑車に糸$\alpha$をかけ,
            左には質量$m$の物体Aを,右には質量$m$の板をつるす。Aと床の間を糸$\beta $で結び,板上には質量$m$の物体Bを置く。滑車は滑らかで,質量は無視でき,重力加速度の大きさを$g$とする。
            \begin{enumerate}
                \item 糸$\alpha$,$\beta $,$\gamma $の張力はそれぞれいくらか。
                \item 糸$\beta $を切ると,全体が動き出した。
                    \begin{enumerate}[(ア)]
                        \item Aの加速度はいくらか。また,Aが距離$h$だけ上がるのにかかる時間はいくらか。
                        \item 糸$\gamma$の張力はいくらか。
                        \item Bが板を押している力はいくらか。
                    \end{enumerate}
            \end{enumerate}     
        \end{mawarikomi}
