\item
    \begin{mawarikomi}{150pt}{\begin{zahyou*}[ul=4mm](-1,13)(-1,6)
\def\O{(4,4)}
\def\A{(0,4)}
\def\AR{(0.2,4)}
\def\AD{(0,0)}
\def\B{(4,0)}
\def\CD{(7.464,0)}
\def\D{(9.196,3.5)}
\def\DD{(9.196,0)}
\def\E{(11.8,0)}
\def\F{(-1,0)}
\def\G{(13,0)}
\def\H{(13,-0.5)}
\def\I{(-1,-0.5)}
\def\vvec{(1,1.732)}
\def\Gx{(4*cos(T)+4)}
\def\Gy{(4*sin(T)+4)}
\BGurafu\Gx\Gy{3.1415}{5.759}
\Drawline{\F\G}

\Candk\O{4}\O{330}\CC\C
\Nuritubusi*{\A\AD\CD\C\A}
\BNuri[0]\Gx\Gy{3.1415}{5.8}
\Enko\O{4}{180}{330}

\Drawline{\C\CD}
\Drawline{\A\AD}
\Put\C[se]{C}
\Put\C{\Yasen\vvec}
\Put\A[w]{A}
\Put\O[n]{O}
\Put\D[n]{D}
\HenKo<henkoH=0.001ex,henkotype=parallel,yazirusi=b>\DD\D{$h$}
\Hasen{\O\A}
\Hasen{\O\C}
\Hasen{\O\B}
\Tyokkakukigou\A\O\B
\Kakukigou\B\O\C(0pt,-10pt)[l]{60\Deg}
\Put\B(0pt,-10pt){B}
\Put\E[ne]{E}
\def\Fx{6.261*0.5*T+7.464}
\def\Fy{6.261*1.732*0.5*T-4.9*T*T+2}
\BGurafu(0.05)(0.02)\Fx\Fy{0}{1.4}

\En*[0]\AR{0.2}
\En\AR{0.2}
\Put\AR(3pt,5pt)[l]{$m$}
\Nuritubusi*{\F\G\H\I\F}
\HenKo<henkoH=0.001ex,henkotype=parallel,yazirusi=b>\O\C{$r$}

\end{zahyou*}
}
        半径$r$の円弧の形をした滑らかなすべり台ABCが,水平な床にB点で接して固定されている。
        中心をOとする円弧ABCは鉛直な平面内にあり,$\angle $AOB=90\Deg ,$\angle $BOC=60\Deg
        である。A点に静止していた質量$m$の小球が,すべり台をすべり落ちてB点を通り,C点ですべり台から飛び出す。
        そののち,最高点Dに達し,再び落下してE点において床と衝突する。重力加速度を$g$とする。
        \begin{enumerate}
            \item 小球のB点での速さ$v_\mathrm{B}$を求めよ。また,C点での速さ$v_\mathrm{C}$を求めよ。
            \item AC間で,小球にはたらく重力のした仕事と垂直抗力のした仕事をそれぞれ求めよ。
            \item D点での小球の速さ$v_\mathrm{D}$とD点の高さ$h$を求め,それぞれ$r$,$g$を用いて表せ。
            \item E点で床に衝突するときの速さ$v_\mathrm{E}$を求め,$r$,$g$を用いて表せ。
        \end{enumerate}
    \end{mawarikomi}