\hakosyokika
\item
    \begin{mawarikomi}(10pt,0pt){150pt}{
        \hfill%WinTpicVersion4.32a
{\unitlength 0.1in%
\begin{picture}(12.0000,12.5200)(4.3800,-13.2900)%
% BOX 2 0 3 0 Black White  
% 2 1000 400 1500 900
% 
\special{pn 8}%
\special{pa 1000 400}%
\special{pa 1500 400}%
\special{pa 1500 900}%
\special{pa 1000 900}%
\special{pa 1000 400}%
\special{pa 1500 400}%
\special{fp}%
% STR 2 0 3 0 Black White  
% 4 1500 350 1500 400 2 0 0 0
% C
\put(15.0000,-4.0000){\makebox(0,0)[lb]{C}}%
% STR 2 0 3 0 Black White  
% 4 1500 850 1500 900 1 0 0 0
% B
\put(15.0000,-9.0000){\makebox(0,0)[lt]{B}}%
% STR 2 0 3 0 Black White  
% 4 1000 850 1000 900 4 0 0 0
% A
\put(10.0000,-9.0000){\makebox(0,0)[rt]{A}}%
% STR 2 0 3 0 Black White  
% 4 1000 350 1000 400 3 0 0 0
% D
\put(10.0000,-4.0000){\makebox(0,0)[rb]{D}}%
% STR 2 0 3 0 Black White  
% 4 1250 600 1250 650 5 0 0 0
% $\odot$
\put(12.5000,-6.5000){\makebox(0,0){$\odot$}}%
% STR 2 0 3 0 Black White  
% 4 1250 100 1250 150 5 0 0 0
% $\odot$
\put(12.5000,-1.5000){\makebox(0,0){$\odot$}}%
% STR 2 0 3 0 Black White  
% 4 1850 100 1850 150 5 0 0 0
% $\odot$
\put(18.5000,-1.5000){\makebox(0,0){$\odot$}}%
% STR 2 0 3 0 Black White  
% 4 1850 600 1850 650 5 0 0 0
% $\odot$
\put(18.5000,-6.5000){\makebox(0,0){$\odot$}}%
% STR 2 0 3 0 Black White  
% 4 1850 1100 1850 1150 5 0 0 0
% $\odot$
\put(18.5000,-11.5000){\makebox(0,0){$\odot$}}%
% STR 2 0 3 0 Black White  
% 4 1250 1100 1250 1150 5 0 0 0
% $\odot$
\put(12.5000,-11.5000){\makebox(0,0){$\odot$}}%
% STR 2 0 3 0 Black White  
% 4 650 100 650 150 5 0 0 0
% $\odot$
\put(6.5000,-1.5000){\makebox(0,0){$\odot$}}%
% STR 2 0 3 0 Black White  
% 4 650 600 650 650 5 0 0 0
% $\odot$
\put(6.5000,-6.5000){\makebox(0,0){$\odot$}}%
% STR 2 0 3 0 Black White  
% 4 650 1100 650 1150 5 0 0 0
% $\odot$
\put(6.5000,-11.5000){\makebox(0,0){$\odot$}}%
% STR 2 0 3 0 Black White  
% 4 1253 1352 1253 1402 5 0 0 0
% 図1
\put(12.5300,-14.0200){\makebox(0,0){図1}}%
\end{picture}}%
\hfill~\\
        \input{./fig/fig131_1.tex}
    }
    図1のように、紙面に垂直で裏から表に向かう磁場中に,一辺の長さ$L$の正方形のコイルABCDが紙面内に置かれている。コイルを通る磁場は一様で,その磁束密度の大きさ$B$が図2のように時間$t$とともに変化した。コイルの電気抵抗を$R$とする。
        \begin{enumerate}
            \item 時間帯I$(0\leqq t \leqq 2t_0)$について,
                \begin{enumerate}[(ア)]
                    \item コイルを貫く磁束$\varPhi$を,時間$t$の関数として表せ。
                    \item コイルに生じる誘導起電力の大きさ$V_0$を$B_0$,$L$,$t_0$を用いて表せ。
                    \item コイルを流れる電流の大きさ$I_0$を求め,$B_0$,$L$,$t_0$,$R$を用いて表せ。
                    \item この時間内にコイルに生じるジュール熱$Q_J$を求め,$B_0$,$L$,$t_0$,$R$を用いて表せ。
                \end{enumerate}
            \item コイルを流れる誘導電流$I$の時間変化をグラフに描け。A$\rightarrow$Bの向きの電流を正とし,目盛には$I_0$を用いてよい。
        \end{enumerate}
    \end{mawarikomi}