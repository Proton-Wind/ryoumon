\hakosyokika
\item
    \begin{mawarikomi}(10pt,0pt){150pt}{
        \hfill%WinTpicVersion4.32a
{\unitlength 0.1in%
\begin{picture}(12.0000,12.5200)(4.3800,-13.2900)%
% BOX 2 0 3 0 Black White  
% 2 1000 400 1500 900
% 
\special{pn 8}%
\special{pa 1000 400}%
\special{pa 1500 400}%
\special{pa 1500 900}%
\special{pa 1000 900}%
\special{pa 1000 400}%
\special{pa 1500 400}%
\special{fp}%
% STR 2 0 3 0 Black White  
% 4 1500 350 1500 400 2 0 0 0
% C
\put(15.0000,-4.0000){\makebox(0,0)[lb]{C}}%
% STR 2 0 3 0 Black White  
% 4 1500 850 1500 900 1 0 0 0
% B
\put(15.0000,-9.0000){\makebox(0,0)[lt]{B}}%
% STR 2 0 3 0 Black White  
% 4 1000 850 1000 900 4 0 0 0
% A
\put(10.0000,-9.0000){\makebox(0,0)[rt]{A}}%
% STR 2 0 3 0 Black White  
% 4 1000 350 1000 400 3 0 0 0
% D
\put(10.0000,-4.0000){\makebox(0,0)[rb]{D}}%
% STR 2 0 3 0 Black White  
% 4 1250 600 1250 650 5 0 0 0
% $\odot$
\put(12.5000,-6.5000){\makebox(0,0){$\odot$}}%
% STR 2 0 3 0 Black White  
% 4 1250 100 1250 150 5 0 0 0
% $\odot$
\put(12.5000,-1.5000){\makebox(0,0){$\odot$}}%
% STR 2 0 3 0 Black White  
% 4 1850 100 1850 150 5 0 0 0
% $\odot$
\put(18.5000,-1.5000){\makebox(0,0){$\odot$}}%
% STR 2 0 3 0 Black White  
% 4 1850 600 1850 650 5 0 0 0
% $\odot$
\put(18.5000,-6.5000){\makebox(0,0){$\odot$}}%
% STR 2 0 3 0 Black White  
% 4 1850 1100 1850 1150 5 0 0 0
% $\odot$
\put(18.5000,-11.5000){\makebox(0,0){$\odot$}}%
% STR 2 0 3 0 Black White  
% 4 1250 1100 1250 1150 5 0 0 0
% $\odot$
\put(12.5000,-11.5000){\makebox(0,0){$\odot$}}%
% STR 2 0 3 0 Black White  
% 4 650 100 650 150 5 0 0 0
% $\odot$
\put(6.5000,-1.5000){\makebox(0,0){$\odot$}}%
% STR 2 0 3 0 Black White  
% 4 650 600 650 650 5 0 0 0
% $\odot$
\put(6.5000,-6.5000){\makebox(0,0){$\odot$}}%
% STR 2 0 3 0 Black White  
% 4 650 1100 650 1150 5 0 0 0
% $\odot$
\put(6.5000,-11.5000){\makebox(0,0){$\odot$}}%
% STR 2 0 3 0 Black White  
% 4 1253 1352 1253 1402 5 0 0 0
% 図1
\put(12.5300,-14.0200){\makebox(0,0){図1}}%
\end{picture}}%
\hfill~\\
        %WinTpicVersion4.32a
{\unitlength 0.1in%
\begin{picture}(20.7300,13.4600)(0.2700,-13.6400)%
% STR 2 0 3 0 Black White  
% 4 1250 1390 1250 1440 5 0 0 0
% 図2
\put(12.5000,-14.4000){\makebox(0,0){図2}}%
% VECTOR 2 0 3 0 Black White  
% 2 300 900 300 100
% 
\special{pn 8}%
\special{pa 300 900}%
\special{pa 300 100}%
\special{fp}%
\special{sh 1}%
\special{pa 300 100}%
\special{pa 280 167}%
\special{pa 300 153}%
\special{pa 320 167}%
\special{pa 300 100}%
\special{fp}%
% VECTOR 2 0 3 0 Black White  
% 2 300 900 2100 900
% 
\special{pn 8}%
\special{pa 300 900}%
\special{pa 2100 900}%
\special{fp}%
\special{sh 1}%
\special{pa 2100 900}%
\special{pa 2033 880}%
\special{pa 2047 900}%
\special{pa 2033 920}%
\special{pa 2100 900}%
\special{fp}%
% LINE 2 0 3 0 Black White  
% 2 700 900 700 850
% 
\special{pn 8}%
\special{pa 700 900}%
\special{pa 700 850}%
\special{fp}%
% LINE 2 0 3 0 Black White  
% 2 1100 900 1100 850
% 
\special{pn 8}%
\special{pa 1100 900}%
\special{pa 1100 850}%
\special{fp}%
% LINE 2 0 3 0 Black White  
% 2 1500 900 1500 850
% 
\special{pn 8}%
\special{pa 1500 900}%
\special{pa 1500 850}%
\special{fp}%
% LINE 2 0 3 0 Black White  
% 2 1900 900 1900 850
% 
\special{pn 8}%
\special{pa 1900 900}%
\special{pa 1900 850}%
\special{fp}%
% LINE 1 0 3 0 Black White  
% 6 300 900 1100 400 1100 400 1500 400 1500 400 1900 900
% 
\special{pn 13}%
\special{pa 300 900}%
\special{pa 1100 400}%
\special{fp}%
\special{pa 1100 400}%
\special{pa 1500 400}%
\special{fp}%
\special{pa 1500 400}%
\special{pa 1900 900}%
\special{fp}%
% LINE 2 1 3 0 Black White  
% 2 1100 400 300 400
% 
\special{pn 8}%
\special{pa 1100 400}%
\special{pa 300 400}%
\special{da 0.015}%
% LINE 2 1 3 0 Black White  
% 2 1100 400 1100 900
% 
\special{pn 8}%
\special{pa 1100 400}%
\special{pa 1100 900}%
\special{da 0.015}%
% STR 2 0 3 0 Black White  
% 4 300 870 300 920 1 0 0 0
% 0
\put(3.0000,-9.2000){\makebox(0,0)[lt]{0}}%
% STR 2 0 3 0 Black White  
% 4 700 930 700 980 5 0 0 0
% $t_0$
\put(7.0000,-9.8000){\makebox(0,0){$t_0$}}%
% STR 2 0 3 0 Black White  
% 4 1100 930 1100 980 5 0 0 0
% $2t_0$
\put(11.0000,-9.8000){\makebox(0,0){$2t_0$}}%
% STR 2 0 3 0 Black White  
% 4 1500 930 1500 980 5 0 0 0
% $3t_0$
\put(15.0000,-9.8000){\makebox(0,0){$3t_0$}}%
% STR 2 0 3 0 Black White  
% 4 1900 930 1900 980 5 0 0 0
% $4t_0$
\put(19.0000,-9.8000){\makebox(0,0){$4t_0$}}%
% STR 2 0 3 0 Black White  
% 4 200 350 200 400 5 0 0 0
% $B_0$
\put(2.0000,-4.0000){\makebox(0,0){$B_0$}}%
% STR 2 0 3 0 Black White  
% 4 2100 750 2100 800 5 0 0 0
% $t$
\put(21.0000,-8.0000){\makebox(0,0){$t$}}%
% STR 2 0 3 0 Black White  
% 4 330 120 330 170 2 0 0 0
% $B$
\put(3.3000,-1.7000){\makebox(0,0)[lb]{$B$}}%
% STR 2 0 3 0 Black White  
% 4 275 850 275 900 3 0 0 0
% 0
\put(2.7500,-9.0000){\makebox(0,0)[rb]{0}}%
% LINE 2 1 3 0 Black White  
% 2 300 1080 300 1280
% 
\special{pn 8}%
\special{pa 300 1080}%
\special{pa 300 1280}%
\special{da 0.015}%
% LINE 2 1 3 0 Black White  
% 2 1100 1080 1100 1280
% 
\special{pn 8}%
\special{pa 1100 1080}%
\special{pa 1100 1280}%
\special{da 0.015}%
% LINE 2 1 3 0 Black White  
% 2 1500 1080 1500 1280
% 
\special{pn 8}%
\special{pa 1500 1080}%
\special{pa 1500 1280}%
\special{da 0.015}%
% LINE 2 1 3 0 Black White  
% 2 1900 1080 1900 1280
% 
\special{pn 8}%
\special{pa 1900 1080}%
\special{pa 1900 1280}%
\special{da 0.015}%
% LINE 2 1 3 0 Black White  
% 2 1500 896 1500 400
% 
\special{pn 8}%
\special{pa 1500 896}%
\special{pa 1500 400}%
\special{da 0.015}%
% VECTOR 2 0 3 0 Black White  
% 12 682 1160 302 1160 702 1160 1102 1160 1302 1160 1102 1160 1302 1160 1502 1160 1702 1160 1502 1160 1702 1160 1902 1160
% 
\special{pn 8}%
\special{pa 682 1160}%
\special{pa 302 1160}%
\special{fp}%
\special{sh 1}%
\special{pa 302 1160}%
\special{pa 369 1180}%
\special{pa 355 1160}%
\special{pa 369 1140}%
\special{pa 302 1160}%
\special{fp}%
\special{pa 702 1160}%
\special{pa 1102 1160}%
\special{fp}%
\special{sh 1}%
\special{pa 1102 1160}%
\special{pa 1035 1140}%
\special{pa 1049 1160}%
\special{pa 1035 1180}%
\special{pa 1102 1160}%
\special{fp}%
\special{pa 1302 1160}%
\special{pa 1102 1160}%
\special{fp}%
\special{sh 1}%
\special{pa 1102 1160}%
\special{pa 1169 1180}%
\special{pa 1155 1160}%
\special{pa 1169 1140}%
\special{pa 1102 1160}%
\special{fp}%
\special{pa 1302 1160}%
\special{pa 1502 1160}%
\special{fp}%
\special{sh 1}%
\special{pa 1502 1160}%
\special{pa 1435 1140}%
\special{pa 1449 1160}%
\special{pa 1435 1180}%
\special{pa 1502 1160}%
\special{fp}%
\special{pa 1702 1160}%
\special{pa 1502 1160}%
\special{fp}%
\special{sh 1}%
\special{pa 1502 1160}%
\special{pa 1569 1180}%
\special{pa 1555 1160}%
\special{pa 1569 1140}%
\special{pa 1502 1160}%
\special{fp}%
\special{pa 1702 1160}%
\special{pa 1902 1160}%
\special{fp}%
\special{sh 1}%
\special{pa 1902 1160}%
\special{pa 1835 1140}%
\special{pa 1849 1160}%
\special{pa 1835 1180}%
\special{pa 1902 1160}%
\special{fp}%
% STR 2 0 3 0 Black White  
% 4 690 1140 690 1160 5 0 1 0
% I
\put(6.9000,-11.6000){\makebox(0,0){{\colorbox[named]{White}{I}}}}%
% STR 2 0 3 0 Black White  
% 4 1300 1140 1300 1160 5 0 1 0
% I\kern-1ptI
\put(13.0000,-11.6000){\makebox(0,0){{\colorbox[named]{White}{I\kern-1ptI}}}}%
% STR 2 0 3 0 Black White  
% 4 1700 1140 1700 1160 5 0 1 0
% I\kern-1ptI\kern-1ptI
\put(17.0000,-11.6000){\makebox(0,0){{\colorbox[named]{White}{I\kern-1ptI\kern-1ptI}}}}%
\end{picture}}%

    }
    図1のように、紙面に垂直で裏から表に向かう磁場中に,一辺の長さ$L$の正方形のコイルABCDが紙面内に置かれている。コイルを通る磁場は一様で,その磁束密度の大きさ$B$が図2のように時間$t$とともに変化した。コイルの電気抵抗を$R$とする。
        \begin{enumerate}
            \item 時間帯I$(0\leqq t \leqq 2t_0)$について,
                \begin{enumerate}[(ア)]
                    \item コイルを貫く磁束$\varPhi$を,時間$t$の関数として表せ。
                    \item コイルに生じる誘導起電力の大きさ$V_0$を$B_0$,$L$,$t_0$を用いて表せ。
                    \item コイルを流れる電流の大きさ$I_0$を求め,$B_0$,$L$,$t_0$,$R$を用いて表せ。
                    \item この時間内にコイルに生じるジュール熱$Q_J$を求め,$B_0$,$L$,$t_0$,$R$を用いて表せ。
                \end{enumerate}
            \item コイルを流れる誘導電流$I$の時間変化をグラフに描け。A$\rightarrow$Bの向きの電流を正とし,目盛には$I_0$を用いてよい。
        \end{enumerate}
    \end{mawarikomi}