\hakosyokika
\item
    \begin{mawarikomi}(10pt,0pt){150pt}{
        %%% C:/vpn/KeTCindy/fig/fig134.tex 
%%% Generator=fig134.cdy 
{\unitlength=1cm%
\begin{picture}%
(6,6)(-3,-3)%
\special{pn 8}%
%
\special{pa  -925   366}\special{pa  -650   366}%
\special{fp}%
\special{pn 16}%
\special{pa  -843   421}\special{pa  -732   421}%
\special{fp}%
\special{pn 8}%
\special{pa  -787   787}\special{pa  -787   421}%
\special{fp}%
\special{pa  -787    -0}\special{pa  -787   366}%
\special{fp}%
\special{pa  -856  -531}\special{pa  -856  -256}\special{pa  -719  -256}\special{pa  -719  -531}%
\special{pa  -856  -531}%
\special{fp}%
\special{pa  -787    -0}\special{pa  -787  -256}%
\special{fp}%
\special{pa  -787  -787}\special{pa  -787  -531}%
\special{fp}%
\special{pa   -69  -138}\special{pa   -69   138}\special{pa    69   138}\special{pa    69  -138}%
\special{pa   -69  -138}%
\special{fp}%
\special{pa     0   787}\special{pa     0   138}%
\special{fp}%
\special{pa     0  -787}\special{pa     0  -138}%
\special{fp}%
\special{pa   787  -236}\special{pa   795  -236}\special{pa   802  -236}\special{pa   810  -235}%
\special{pa   817  -234}\special{pa   824  -233}\special{pa   831  -232}\special{pa   838  -231}%
\special{pa   844  -229}\special{pa   851  -227}\special{pa   857  -225}\special{pa   863  -223}%
\special{pa   868  -220}\special{pa   874  -218}\special{pa   878  -215}\special{pa   883  -212}%
\special{pa   887  -209}\special{pa   891  -206}\special{pa   894  -202}\special{pa   897  -199}%
\special{pa   900  -195}\special{pa   902  -192}\special{pa   903  -188}\special{pa   905  -185}%
\special{pa   905  -181}\special{pa   906  -177}\special{pa   905  -173}\special{pa   905  -170}%
\special{pa   903  -166}\special{pa   902  -162}\special{pa   900  -159}\special{pa   897  -155}%
\special{pa   894  -152}\special{pa   891  -149}\special{pa   887  -146}\special{pa   883  -142}%
\special{pa   878  -140}\special{pa   874  -137}\special{pa   868  -134}\special{pa   863  -132}%
\special{pa   857  -129}\special{pa   851  -127}\special{pa   844  -125}\special{pa   838  -124}%
\special{pa   831  -122}\special{pa   824  -121}\special{pa   817  -120}\special{pa   810  -119}%
\special{pa   802  -119}\special{pa   795  -118}\special{pa   787  -118}\special{pa   795  -118}%
\special{pa   802  -118}\special{pa   810  -117}\special{pa   817  -116}\special{pa   824  -115}%
\special{pa   831  -114}\special{pa   838  -112}\special{pa   844  -111}\special{pa   851  -109}%
\special{pa   857  -107}\special{pa   863  -105}\special{pa   868  -102}\special{pa   874   -99}%
\special{pa   878   -97}\special{pa   883   -94}\special{pa   887   -91}\special{pa   891   -88}%
\special{pa   894   -84}\special{pa   897   -81}\special{pa   900   -77}\special{pa   902   -74}%
\special{pa   903   -70}\special{pa   905   -66}\special{pa   905   -63}\special{pa   906   -59}%
\special{pa   905   -55}\special{pa   905   -52}\special{pa   903   -48}\special{pa   902   -44}%
\special{pa   900   -41}\special{pa   897   -37}\special{pa   894   -34}\special{pa   891   -31}%
\special{pa   887   -27}\special{pa   883   -24}\special{pa   878   -21}\special{pa   874   -19}%
\special{pa   868   -16}\special{pa   863   -14}\special{pa   857   -11}\special{pa   851    -9}%
\special{pa   844    -7}\special{pa   838    -6}\special{pa   831    -4}\special{pa   824    -3}%
\special{pa   817    -2}\special{pa   810    -1}\special{pa   802    -0}\special{pa   795    -0}%
\special{pa   787    -0}\special{pa   795     0}\special{pa   802     0}\special{pa   810     1}%
\special{pa   817     2}\special{pa   824     3}\special{pa   831     4}\special{pa   838     6}%
\special{pa   844     7}\special{pa   851     9}\special{pa   857    11}\special{pa   863    14}%
\special{pa   868    16}\special{pa   874    19}\special{pa   878    21}\special{pa   883    24}%
\special{pa   887    27}\special{pa   891    31}\special{pa   894    34}\special{pa   897    37}%
\special{pa   900    41}\special{pa   902    44}\special{pa   903    48}\special{pa   905    52}%
\special{pa   905    55}\special{pa   906    59}\special{pa   905    63}\special{pa   905    66}%
\special{pa   903    70}\special{pa   902    74}\special{pa   900    77}\special{pa   897    81}%
\special{pa   894    84}\special{pa   891    88}\special{pa   887    91}\special{pa   883    94}%
\special{pa   878    97}\special{pa   874    99}\special{pa   868   102}\special{pa   863   105}%
\special{pa   857   107}\special{pa   851   109}\special{pa   844   111}\special{pa   838   112}%
\special{pa   831   114}\special{pa   824   115}\special{pa   817   116}\special{pa   810   117}%
\special{pa   802   118}\special{pa   795   118}\special{pa   787   118}\special{pa   795   118}%
\special{pa   802   119}\special{pa   810   119}\special{pa   817   120}\special{pa   824   121}%
\special{pa   831   122}\special{pa   838   124}\special{pa   844   125}\special{pa   851   127}%
\special{pa   857   129}\special{pa   863   132}\special{pa   868   134}\special{pa   874   137}%
\special{pa   878   140}\special{pa   883   142}\special{pa   887   146}\special{pa   891   149}%
\special{pa   894   152}\special{pa   897   155}\special{pa   900   159}\special{pa   902   162}%
\special{pa   903   166}\special{pa   905   170}\special{pa   905   173}\special{pa   906   177}%
\special{pa   905   181}\special{pa   905   185}\special{pa   903   188}\special{pa   902   192}%
\special{pa   900   195}\special{pa   897   199}\special{pa   894   202}\special{pa   891   206}%
\special{pa   887   209}\special{pa   883   212}\special{pa   878   215}\special{pa   874   218}%
\special{pa   868   220}\special{pa   863   223}\special{pa   857   225}\special{pa   851   227}%
\special{pa   844   229}\special{pa   838   231}\special{pa   831   232}\special{pa   824   233}%
\special{pa   817   234}\special{pa   810   235}\special{pa   802   236}\special{pa   795   236}%
\special{pa   787   236}%
\special{fp}%
\special{pa   787  -787}\special{pa   787  -236}%
\special{fp}%
\special{pa   787   787}\special{pa   787   236}%
\special{fp}%
\special{pa  -551  -787}\special{pa  -278  -945}%
\special{fp}%
\special{pa  -787  -787}\special{pa  -551  -787}%
\special{fp}%
\special{pa     0  -787}\special{pa  -236  -787}%
\special{fp}%
\special{pa     0  -787}\special{pa   787  -787}%
\special{fp}%
\special{pa  -787   787}\special{pa   787   787}%
\special{fp}%
\settowidth{\Width}{$E$}\setlength{\Width}{-1\Width}%
\settoheight{\Height}{$E$}\settodepth{\Depth}{$E$}\setlength{\Height}{-0.5\Height}\setlength{\Depth}{0.5\Depth}\addtolength{\Height}{\Depth}%
\put( -2.590, -1.000){\hspace*{\Width}\raisebox{\Height}{$E$}}%
%
\settowidth{\Width}{$r$}\setlength{\Width}{-1\Width}%
\settoheight{\Height}{$r$}\settodepth{\Depth}{$r$}\setlength{\Height}{-0.5\Height}\setlength{\Depth}{0.5\Depth}\addtolength{\Height}{\Depth}%
\put( -2.475,  1.000){\hspace*{\Width}\raisebox{\Height}{$r$}}%
%
\settowidth{\Width}{$R$}\setlength{\Width}{-1\Width}%
\settoheight{\Height}{$R$}\settodepth{\Depth}{$R$}\setlength{\Height}{-0.5\Height}\setlength{\Depth}{0.5\Depth}\addtolength{\Height}{\Depth}%
\put( -0.475,  0.000){\hspace*{\Width}\raisebox{\Height}{$R$}}%
%
\settowidth{\Width}{$L$}\setlength{\Width}{-1\Width}%
\settoheight{\Height}{$L$}\settodepth{\Depth}{$L$}\setlength{\Height}{-0.5\Height}\setlength{\Depth}{0.5\Depth}\addtolength{\Height}{\Depth}%
\put(  2.625,  0.000){\hspace*{\Width}\raisebox{\Height}{$L$}}%
%
\settowidth{\Width}{$\mathrm{S}$}\setlength{\Width}{-0.5\Width}%
\settoheight{\Height}{$\mathrm{S}$}\settodepth{\Depth}{$\mathrm{S}$}\setlength{\Height}{\Depth}%
\put( -1.000,  2.350){\hspace*{\Width}\raisebox{\Height}{$\mathrm{S}$}}%
%
\special{pa 15 -787}\special{pa 15 -789}\special{pa 14 -791}\special{pa 14 -793}\special{pa 13 -795}%
\special{pa 12 -796}\special{pa 11 -798}\special{pa 10 -799}\special{pa 8 -800}\special{pa 6 -801}%
\special{pa 5 -802}\special{pa 3 -802}\special{pa 1 -802}\special{pa -1 -802}\special{pa -3 -802}%
\special{pa -5 -802}\special{pa -6 -801}\special{pa -8 -800}\special{pa -10 -799}%
\special{pa -11 -798}\special{pa -12 -796}\special{pa -13 -795}\special{pa -14 -793}%
\special{pa -14 -791}\special{pa -15 -789}\special{pa -15 -787}\special{pa -15 -786}%
\special{pa -14 -784}\special{pa -14 -782}\special{pa -13 -780}\special{pa -12 -779}%
\special{pa -11 -777}\special{pa -10 -776}\special{pa -8 -775}\special{pa -6 -774}%
\special{pa -5 -773}\special{pa -3 -773}\special{pa -1 -772}\special{pa 1 -772}\special{pa 3 -773}%
\special{pa 5 -773}\special{pa 6 -774}\special{pa 8 -775}\special{pa 10 -776}\special{pa 11 -777}%
\special{pa 12 -779}\special{pa 13 -780}\special{pa 14 -782}\special{pa 14 -784}\special{pa 15 -786}%
\special{pa 15 -787}\special{pa 15 -787}\special{sh 1}\special{ip}%
\special{pa    15  -787}\special{pa    15  -789}\special{pa    14  -791}\special{pa    14  -793}%
\special{pa    13  -795}\special{pa    12  -796}\special{pa    11  -798}\special{pa    10  -799}%
\special{pa     8  -800}\special{pa     6  -801}\special{pa     5  -802}\special{pa     3  -802}%
\special{pa     1  -802}\special{pa    -1  -802}\special{pa    -3  -802}\special{pa    -5  -802}%
\special{pa    -6  -801}\special{pa    -8  -800}\special{pa   -10  -799}\special{pa   -11  -798}%
\special{pa   -12  -796}\special{pa   -13  -795}\special{pa   -14  -793}\special{pa   -14  -791}%
\special{pa   -15  -789}\special{pa   -15  -787}\special{pa   -15  -786}\special{pa   -14  -784}%
\special{pa   -14  -782}\special{pa   -13  -780}\special{pa   -12  -779}\special{pa   -11  -777}%
\special{pa   -10  -776}\special{pa    -8  -775}\special{pa    -6  -774}\special{pa    -5  -773}%
\special{pa    -3  -773}\special{pa    -1  -772}\special{pa     1  -772}\special{pa     3  -773}%
\special{pa     5  -773}\special{pa     6  -774}\special{pa     8  -775}\special{pa    10  -776}%
\special{pa    11  -777}\special{pa    12  -779}\special{pa    13  -780}\special{pa    14  -782}%
\special{pa    14  -784}\special{pa    15  -786}\special{pa    15  -787}%
\special{fp}%
\special{pa 15 787}\special{pa 15 786}\special{pa 14 784}\special{pa 14 782}\special{pa 13 780}%
\special{pa 12 779}\special{pa 11 777}\special{pa 10 776}\special{pa 8 775}\special{pa 6 774}%
\special{pa 5 773}\special{pa 3 773}\special{pa 1 772}\special{pa -1 772}\special{pa -3 773}%
\special{pa -5 773}\special{pa -6 774}\special{pa -8 775}\special{pa -10 776}\special{pa -11 777}%
\special{pa -12 779}\special{pa -13 780}\special{pa -14 782}\special{pa -14 784}\special{pa -15 786}%
\special{pa -15 787}\special{pa -15 789}\special{pa -14 791}\special{pa -14 793}\special{pa -13 795}%
\special{pa -12 796}\special{pa -11 798}\special{pa -10 799}\special{pa -8 800}\special{pa -6 801}%
\special{pa -5 802}\special{pa -3 802}\special{pa -1 802}\special{pa 1 802}\special{pa 3 802}%
\special{pa 5 802}\special{pa 6 801}\special{pa 8 800}\special{pa 10 799}\special{pa 11 798}%
\special{pa 12 796}\special{pa 13 795}\special{pa 14 793}\special{pa 14 791}\special{pa 15 789}%
\special{pa 15 787}\special{pa 15 787}\special{sh 1}\special{ip}%
\special{pa    15   787}\special{pa    15   786}\special{pa    14   784}\special{pa    14   782}%
\special{pa    13   780}\special{pa    12   779}\special{pa    11   777}\special{pa    10   776}%
\special{pa     8   775}\special{pa     6   774}\special{pa     5   773}\special{pa     3   773}%
\special{pa     1   772}\special{pa    -1   772}\special{pa    -3   773}\special{pa    -5   773}%
\special{pa    -6   774}\special{pa    -8   775}\special{pa   -10   776}\special{pa   -11   777}%
\special{pa   -12   779}\special{pa   -13   780}\special{pa   -14   782}\special{pa   -14   784}%
\special{pa   -15   786}\special{pa   -15   787}\special{pa   -15   789}\special{pa   -14   791}%
\special{pa   -14   793}\special{pa   -13   795}\special{pa   -12   796}\special{pa   -11   798}%
\special{pa   -10   799}\special{pa    -8   800}\special{pa    -6   801}\special{pa    -5   802}%
\special{pa    -3   802}\special{pa    -1   802}\special{pa     1   802}\special{pa     3   802}%
\special{pa     5   802}\special{pa     6   801}\special{pa     8   800}\special{pa    10   799}%
\special{pa    11   798}\special{pa    12   796}\special{pa    13   795}\special{pa    14   793}%
\special{pa    14   791}\special{pa    15   789}\special{pa    15   787}%
\special{fp}%
\special{pa 802 -787}\special{pa 802 -789}\special{pa 802 -791}\special{pa 801 -793}%
\special{pa 801 -795}\special{pa 800 -796}\special{pa 798 -798}\special{pa 797 -799}%
\special{pa 795 -800}\special{pa 794 -801}\special{pa 792 -802}\special{pa 790 -802}%
\special{pa 788 -802}\special{pa 786 -802}\special{pa 785 -802}\special{pa 783 -802}%
\special{pa 781 -801}\special{pa 779 -800}\special{pa 778 -799}\special{pa 776 -798}%
\special{pa 775 -796}\special{pa 774 -795}\special{pa 773 -793}\special{pa 773 -791}%
\special{pa 773 -789}\special{pa 772 -787}\special{pa 773 -786}\special{pa 773 -784}%
\special{pa 773 -782}\special{pa 774 -780}\special{pa 775 -779}\special{pa 776 -777}%
\special{pa 778 -776}\special{pa 779 -775}\special{pa 781 -774}\special{pa 783 -773}%
\special{pa 785 -773}\special{pa 786 -772}\special{pa 788 -772}\special{pa 790 -773}%
\special{pa 792 -773}\special{pa 794 -774}\special{pa 795 -775}\special{pa 797 -776}%
\special{pa 798 -777}\special{pa 800 -779}\special{pa 801 -780}\special{pa 801 -782}%
\special{pa 802 -784}\special{pa 802 -786}\special{pa 802 -787}\special{pa 802 -787}%
\special{sh 1}\special{ip}%
\special{pa   802  -787}\special{pa   802  -789}\special{pa   802  -791}\special{pa   801  -793}%
\special{pa   801  -795}\special{pa   800  -796}\special{pa   798  -798}\special{pa   797  -799}%
\special{pa   795  -800}\special{pa   794  -801}\special{pa   792  -802}\special{pa   790  -802}%
\special{pa   788  -802}\special{pa   786  -802}\special{pa   785  -802}\special{pa   783  -802}%
\special{pa   781  -801}\special{pa   779  -800}\special{pa   778  -799}\special{pa   776  -798}%
\special{pa   775  -796}\special{pa   774  -795}\special{pa   773  -793}\special{pa   773  -791}%
\special{pa   773  -789}\special{pa   772  -787}\special{pa   773  -786}\special{pa   773  -784}%
\special{pa   773  -782}\special{pa   774  -780}\special{pa   775  -779}\special{pa   776  -777}%
\special{pa   778  -776}\special{pa   779  -775}\special{pa   781  -774}\special{pa   783  -773}%
\special{pa   785  -773}\special{pa   786  -772}\special{pa   788  -772}\special{pa   790  -773}%
\special{pa   792  -773}\special{pa   794  -774}\special{pa   795  -775}\special{pa   797  -776}%
\special{pa   798  -777}\special{pa   800  -779}\special{pa   801  -780}\special{pa   801  -782}%
\special{pa   802  -784}\special{pa   802  -786}\special{pa   802  -787}%
\special{fp}%
\special{pa 802 787}\special{pa 802 786}\special{pa 802 784}\special{pa 801 782}\special{pa 801 780}%
\special{pa 800 779}\special{pa 798 777}\special{pa 797 776}\special{pa 795 775}\special{pa 794 774}%
\special{pa 792 773}\special{pa 790 773}\special{pa 788 772}\special{pa 786 772}\special{pa 785 773}%
\special{pa 783 773}\special{pa 781 774}\special{pa 779 775}\special{pa 778 776}\special{pa 776 777}%
\special{pa 775 779}\special{pa 774 780}\special{pa 773 782}\special{pa 773 784}\special{pa 773 786}%
\special{pa 772 787}\special{pa 773 789}\special{pa 773 791}\special{pa 773 793}\special{pa 774 795}%
\special{pa 775 796}\special{pa 776 798}\special{pa 778 799}\special{pa 779 800}\special{pa 781 801}%
\special{pa 783 802}\special{pa 785 802}\special{pa 786 802}\special{pa 788 802}\special{pa 790 802}%
\special{pa 792 802}\special{pa 794 801}\special{pa 795 800}\special{pa 797 799}\special{pa 798 798}%
\special{pa 800 796}\special{pa 801 795}\special{pa 801 793}\special{pa 802 791}\special{pa 802 789}%
\special{pa 802 787}\special{pa 802 787}\special{sh 1}\special{ip}%
\special{pa   802   787}\special{pa   802   786}\special{pa   802   784}\special{pa   801   782}%
\special{pa   801   780}\special{pa   800   779}\special{pa   798   777}\special{pa   797   776}%
\special{pa   795   775}\special{pa   794   774}\special{pa   792   773}\special{pa   790   773}%
\special{pa   788   772}\special{pa   786   772}\special{pa   785   773}\special{pa   783   773}%
\special{pa   781   774}\special{pa   779   775}\special{pa   778   776}\special{pa   776   777}%
\special{pa   775   779}\special{pa   774   780}\special{pa   773   782}\special{pa   773   784}%
\special{pa   773   786}\special{pa   772   787}\special{pa   773   789}\special{pa   773   791}%
\special{pa   773   793}\special{pa   774   795}\special{pa   775   796}\special{pa   776   798}%
\special{pa   778   799}\special{pa   779   800}\special{pa   781   801}\special{pa   783   802}%
\special{pa   785   802}\special{pa   786   802}\special{pa   788   802}\special{pa   790   802}%
\special{pa   792   802}\special{pa   794   801}\special{pa   795   800}\special{pa   797   799}%
\special{pa   798   798}\special{pa   800   796}\special{pa   801   795}\special{pa   801   793}%
\special{pa   802   791}\special{pa   802   789}\special{pa   802   787}%
\special{fp}%
\settowidth{\Width}{b}\setlength{\Width}{0\Width}%
\settoheight{\Height}{b}\settodepth{\Depth}{b}\setlength{\Height}{\Depth}%
\put(  2.050,  2.050){\hspace*{\Width}\raisebox{\Height}{b}}%
%
\settowidth{\Width}{a}\setlength{\Width}{0\Width}%
\settoheight{\Height}{a}\settodepth{\Depth}{a}\setlength{\Height}{-\Height}%
\put(  2.050, -2.050){\hspace*{\Width}\raisebox{\Height}{a}}%
%
\end{picture}}%
    }
    抵抗値が$R$\tanni{\Omega }と$r$\tanni{\Omega }の抵抗,自己インダクタンス$L$\tanni{H}のコイル,起電力$E$\tanni{V}の電池およびスイッチSからなる回路がある。電池の内部抵抗とコイルの抵抗は無視できるものとする。
        \begin{enumerate}
            \item Sを閉じた直後に電池を流れる電流$I_1$\tanni{A}を求めよ。また,そのときaに対するbの電位$V_1$\tanni{V}を求めよ。
            \item Sを閉じてから十分に時間がたったときに電池を流れる電流$I_2$\tanni{A}を求めよ。
            \item この後Sを開く。その直後に$R$に流れる電流$I_3$\tanni{A}と,aに対するbの電位$V_3$\tanni{V}を求めよ。また,Sを開いてから十分に時間が経過する間に,抵抗$R$で発生するジュール熱$W$\tanni{J}を求めよ。
            \end{enumerate}
    \end{mawarikomi}