\hakosyokika
\item
    \begin{mawarikomi}(10pt,0){210pt}{{\small
\begin{zahyou}[
    ul=5mm
    ,yokozikukigou=$t \times 10^{-2}$\tanni{s}
    ,tatezikukigou=$y$\tanni{mm}
    ,yokozikuhaiti={[ne]}
    ,tatezikuhaiti={[n]}
    ,xscale=1
    ,yscale=1](0,12)(-3,3)
    \def\Fx{2*sin(2*$pi *X/8)}
    {\thicklines
    \YGurafu\Fx\xmin\xmax
    }
    \zahyouMemori[g]<dx=2,dy=1>
\end{zahyou}}
}
        ある媒質内を$x$軸の正方向に速さ100\sftanni{cm/s}で進行している正弦波の縦波がある。波がないときに$x$にあった媒質が,波がやってきたときに$y$だけ変位したとする。(変位$y$は$+x$方向を正にとる。)$x=0$の媒質が時間$t$と共に図で示されるように変化している。波は$t=0$よりもずっと以前から存在しているとする。
        \begin{enumerate}
            \item この波の周期,振動数,波長はいくらか。
            \item この時刻$t=0$での波形を示す$y-x$グラフを$0\leqq x \leqq 12$\tanni{cm}の範囲で描け。(横波表示)
            \item $x=4$\tanni{cm}の媒質の振動グラフを上図に点線で記入せよ。
            \item $x=0$の点の負の$x$方向の速さが最大になる時刻は図の示された時間内でいつか。
            \item $t=0$で,媒質の密度が最大の点は$0\leqq x \leqq 12$\tanni{cm}の範囲内でどこか。
            \item $x=4$\tanni{cm}で,媒質の密度が最大になる時刻は図に示された時間内でいつか。
        \end{enumerate}
    \end{mawarikomi}