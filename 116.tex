\hakosyokika
\item
    \begin{mawarikomi}(20pt,0pt){150pt}{
        %%% C:/vpn/KeTCindy/fig/fig116.tex 
%%% Generator=fig116.cdy 
{\unitlength=1cm%
\begin{picture}%
(5,4)(-2.5,-2)%
\special{pn 8}%
%
\special{pa    30   689}\special{pa    30   492}%
\special{fp}%
\special{pn 16}%
\special{pa    69   630}\special{pa    69   551}%
\special{fp}%
\special{pn 8}%
\special{pa   787   591}\special{pa    69   591}%
\special{fp}%
\special{pa  -689   591}\special{pa    30   591}%
\special{fp}%
\special{pa   374  -492}\special{pa   374  -689}%
\special{fp}%
\special{pn 16}%
\special{pa   413  -551}\special{pa   413  -630}%
\special{fp}%
\special{pn 8}%
\special{pa   591  -591}\special{pa   413  -591}%
\special{fp}%
\special{pa   197  -591}\special{pa   374  -591}%
\special{fp}%
\special{pa   -49  -640}\special{pa  -246  -640}\special{pa  -246  -541}\special{pa   -49  -541}%
\special{pa   -49  -640}%
\special{fp}%
\special{pa  -492  -591}\special{pa  -246  -591}%
\special{fp}%
\special{pa   197  -591}\special{pa   -49  -591}%
\special{fp}%
\special{pa   148    49}\special{pa   -49    49}\special{pa   -49   148}\special{pa   148   148}%
\special{pa   148    49}%
\special{fp}%
\special{pa  -492    98}\special{pa   -49    98}%
\special{fp}%
\special{pa   591    98}\special{pa   148    98}%
\special{fp}%
\special{pa   837   295}\special{pa   837    98}\special{pa   738    98}\special{pa   738   295}%
\special{pa   837   295}%
\special{fp}%
\special{pa   787  -197}\special{pa   787    98}%
\special{fp}%
\special{pa   787   591}\special{pa   787   295}%
\special{fp}%
\special{pa  -689   591}\special{pa  -689  -197}%
\special{fp}%
\special{pa  -689  -197}\special{pa  -492  -197}%
\special{fp}%
\special{pa   591  -591}\special{pa   591    98}%
\special{fp}%
\special{pa  -492  -591}\special{pa  -492    98}%
\special{fp}%
\special{pa   591  -197}\special{pa   787  -197}%
\special{fp}%
\special{pa -485 -197}\special{pa -485 -198}\special{pa -485 -199}\special{pa -485 -200}%
\special{pa -486 -200}\special{pa -486 -201}\special{pa -487 -202}\special{pa -487 -203}%
\special{pa -488 -203}\special{pa -489 -204}\special{pa -490 -204}\special{pa -491 -204}%
\special{pa -492 -204}\special{pa -493 -204}\special{pa -494 -204}\special{pa -494 -204}%
\special{pa -495 -204}\special{pa -496 -203}\special{pa -497 -203}\special{pa -498 -202}%
\special{pa -498 -201}\special{pa -499 -200}\special{pa -499 -200}\special{pa -499 -199}%
\special{pa -500 -198}\special{pa -500 -197}\special{pa -500 -196}\special{pa -499 -195}%
\special{pa -499 -194}\special{pa -499 -193}\special{pa -498 -192}\special{pa -498 -192}%
\special{pa -497 -191}\special{pa -496 -191}\special{pa -495 -190}\special{pa -494 -190}%
\special{pa -494 -190}\special{pa -493 -189}\special{pa -492 -189}\special{pa -491 -190}%
\special{pa -490 -190}\special{pa -489 -190}\special{pa -488 -191}\special{pa -487 -191}%
\special{pa -487 -192}\special{pa -486 -192}\special{pa -486 -193}\special{pa -485 -194}%
\special{pa -485 -195}\special{pa -485 -196}\special{pa -485 -197}\special{pa -485 -197}%
\special{sh 1}\special{ip}%
\special{pa  -485  -197}\special{pa  -485  -198}\special{pa  -485  -199}\special{pa  -485  -200}%
\special{pa  -486  -200}\special{pa  -486  -201}\special{pa  -487  -202}\special{pa  -487  -203}%
\special{pa  -488  -203}\special{pa  -489  -204}\special{pa  -490  -204}\special{pa  -491  -204}%
\special{pa  -492  -204}\special{pa  -493  -204}\special{pa  -494  -204}\special{pa  -494  -204}%
\special{pa  -495  -204}\special{pa  -496  -203}\special{pa  -497  -203}\special{pa  -498  -202}%
\special{pa  -498  -201}\special{pa  -499  -200}\special{pa  -499  -200}\special{pa  -499  -199}%
\special{pa  -500  -198}\special{pa  -500  -197}\special{pa  -500  -196}\special{pa  -499  -195}%
\special{pa  -499  -194}\special{pa  -499  -193}\special{pa  -498  -192}\special{pa  -498  -192}%
\special{pa  -497  -191}\special{pa  -496  -191}\special{pa  -495  -190}\special{pa  -494  -190}%
\special{pa  -494  -190}\special{pa  -493  -189}\special{pa  -492  -189}\special{pa  -491  -190}%
\special{pa  -490  -190}\special{pa  -489  -190}\special{pa  -488  -191}\special{pa  -487  -191}%
\special{pa  -487  -192}\special{pa  -486  -192}\special{pa  -486  -193}\special{pa  -485  -194}%
\special{pa  -485  -195}\special{pa  -485  -196}\special{pa  -485  -197}%
\special{fp}%
\special{pa 598 -197}\special{pa 598 -198}\special{pa 598 -199}\special{pa 598 -200}%
\special{pa 597 -200}\special{pa 597 -201}\special{pa 596 -202}\special{pa 595 -203}%
\special{pa 595 -203}\special{pa 594 -204}\special{pa 593 -204}\special{pa 592 -204}%
\special{pa 591 -204}\special{pa 590 -204}\special{pa 589 -204}\special{pa 588 -204}%
\special{pa 587 -204}\special{pa 587 -203}\special{pa 586 -203}\special{pa 585 -202}%
\special{pa 584 -201}\special{pa 584 -200}\special{pa 584 -200}\special{pa 583 -199}%
\special{pa 583 -198}\special{pa 583 -197}\special{pa 583 -196}\special{pa 583 -195}%
\special{pa 584 -194}\special{pa 584 -193}\special{pa 584 -192}\special{pa 585 -192}%
\special{pa 586 -191}\special{pa 587 -191}\special{pa 587 -190}\special{pa 588 -190}%
\special{pa 589 -190}\special{pa 590 -189}\special{pa 591 -189}\special{pa 592 -190}%
\special{pa 593 -190}\special{pa 594 -190}\special{pa 595 -191}\special{pa 595 -191}%
\special{pa 596 -192}\special{pa 597 -192}\special{pa 597 -193}\special{pa 598 -194}%
\special{pa 598 -195}\special{pa 598 -196}\special{pa 598 -197}\special{pa 598 -197}%
\special{sh 1}\special{ip}%
\special{pa   598  -197}\special{pa   598  -198}\special{pa   598  -199}\special{pa   598  -200}%
\special{pa   597  -200}\special{pa   597  -201}\special{pa   596  -202}\special{pa   595  -203}%
\special{pa   595  -203}\special{pa   594  -204}\special{pa   593  -204}\special{pa   592  -204}%
\special{pa   591  -204}\special{pa   590  -204}\special{pa   589  -204}\special{pa   588  -204}%
\special{pa   587  -204}\special{pa   587  -203}\special{pa   586  -203}\special{pa   585  -202}%
\special{pa   584  -201}\special{pa   584  -200}\special{pa   584  -200}\special{pa   583  -199}%
\special{pa   583  -198}\special{pa   583  -197}\special{pa   583  -196}\special{pa   583  -195}%
\special{pa   584  -194}\special{pa   584  -193}\special{pa   584  -192}\special{pa   585  -192}%
\special{pa   586  -191}\special{pa   587  -191}\special{pa   587  -190}\special{pa   588  -190}%
\special{pa   589  -190}\special{pa   590  -189}\special{pa   591  -189}\special{pa   592  -190}%
\special{pa   593  -190}\special{pa   594  -190}\special{pa   595  -191}\special{pa   595  -191}%
\special{pa   596  -192}\special{pa   597  -192}\special{pa   597  -193}\special{pa   598  -194}%
\special{pa   598  -195}\special{pa   598  -196}\special{pa   598  -197}%
\special{fp}%
\settowidth{\Width}{100{\sf V}}\setlength{\Width}{0\Width}%
\settoheight{\Height}{100{\sf V}}\settodepth{\Depth}{100{\sf V}}\setlength{\Height}{-\Height}%
\put(  0.170, -1.850){\hspace*{\Width}\raisebox{\Height}{100{\sf V}}}%
%
\settowidth{\Width}{$\mathrm{E_1}$}\setlength{\Width}{-1\Width}%
\settoheight{\Height}{$\mathrm{E_1}$}\settodepth{\Depth}{$\mathrm{E_1}$}\setlength{\Height}{-\Height}%
\put( -0.030, -1.850){\hspace*{\Width}\raisebox{\Height}{$\mathrm{E_1}$}}%
%
\settowidth{\Width}{30{\sf V}}\setlength{\Width}{-0.5\Width}%
\settoheight{\Height}{30{\sf V}}\settodepth{\Depth}{30{\sf V}}\setlength{\Height}{-\Height}%
\put(  1.000,  1.150){\hspace*{\Width}\raisebox{\Height}{30{\sf V}}}%
%
\settowidth{\Width}{$\mathrm{E_2}$}\setlength{\Width}{0\Width}%
\settoheight{\Height}{$\mathrm{E_2}$}\settodepth{\Depth}{$\mathrm{E_2}$}\setlength{\Height}{\Depth}%
\put(  1.150,  1.700){\hspace*{\Width}\raisebox{\Height}{$\mathrm{E_2}$}}%
%
\settowidth{\Width}{$\mathrm{R_1}$}\setlength{\Width}{-1\Width}%
\settoheight{\Height}{$\mathrm{R_1}$}\settodepth{\Depth}{$\mathrm{R_1}$}\setlength{\Height}{\Depth}%
\put( -0.680,  1.800){\hspace*{\Width}\raisebox{\Height}{$\mathrm{R_1}$}}%
%
\settowidth{\Width}{$I_2$}\setlength{\Width}{-0.5\Width}%
\settoheight{\Height}{$I_2$}\settodepth{\Depth}{$I_2$}\setlength{\Height}{\Depth}%
\put( -0.380,  1.800){\hspace*{\Width}\raisebox{\Height}{$I_2$}}%
%
\settowidth{\Width}{$I_1$}\setlength{\Width}{-0.5\Width}%
\settoheight{\Height}{$I_1$}\settodepth{\Depth}{$I_1$}\setlength{\Height}{-\Height}%
\put( -1.000, -1.680){\hspace*{\Width}\raisebox{\Height}{$I_1$}}%
%
\settowidth{\Width}{$\mathrm{R_2}$}\setlength{\Width}{-0.5\Width}%
\settoheight{\Height}{$\mathrm{R_2}$}\settodepth{\Depth}{$\mathrm{R_2}$}\setlength{\Height}{\Depth}%
\put(  0.130,  0.050){\hspace*{\Width}\raisebox{\Height}{$\mathrm{R_2}$}}%
%
\settowidth{\Width}{$\mathrm{15{\sf \Omega }}$}\setlength{\Width}{-0.5\Width}%
\settoheight{\Height}{$\mathrm{15{\sf \Omega }}$}\settodepth{\Depth}{$\mathrm{15{\sf \Omega }}$}\setlength{\Height}{-\Height}%
\put( -0.380,  1.200){\hspace*{\Width}\raisebox{\Height}{$\mathrm{15{\sf \Omega }}$}}%
%
\settowidth{\Width}{$\mathrm{20{\sf \Omega }}$}\setlength{\Width}{-0.5\Width}%
\settoheight{\Height}{$\mathrm{20{\sf \Omega }}$}\settodepth{\Depth}{$\mathrm{20{\sf \Omega }}$}\setlength{\Height}{-\Height}%
\put(  0.130, -0.550){\hspace*{\Width}\raisebox{\Height}{$\mathrm{20{\sf \Omega }}$}}%
%
\settowidth{\Width}{$\mathrm{R_3}$}\setlength{\Width}{0\Width}%
\settoheight{\Height}{$\mathrm{R_3}$}\settodepth{\Depth}{$\mathrm{R_3}$}\setlength{\Height}{\Depth}%
\put(  2.300, -0.200){\hspace*{\Width}\raisebox{\Height}{$\mathrm{R_3}$}}%
%
\settowidth{\Width}{$\mathrm{8{\sf \Omega }}$}\setlength{\Width}{0\Width}%
\settoheight{\Height}{$\mathrm{8{\sf \Omega }}$}\settodepth{\Depth}{$\mathrm{8{\sf \Omega }}$}\setlength{\Height}{-\Height}%
\put(  2.300, -0.800){\hspace*{\Width}\raisebox{\Height}{$\mathrm{8{\sf \Omega }}$}}%
%
\special{pa -209 -681}\special{pa -246 -669}\special{pa -209 -657}\special{pa -216 -669}%
\special{pa -209 -681}\special{pa -209 -681}\special{sh 1}\special{ip}%
\special{pn 1}%
\special{pa  -209  -681}\special{pa  -246  -669}\special{pa  -209  -657}\special{pa  -216  -669}%
\special{pa  -209  -681}%
\special{fp}%
\special{pn 8}%
\special{pa   -49  -669}\special{pa  -216  -669}%
\special{fp}%
\special{pa -456 612}\special{pa -493 624}\special{pa -456 636}\special{pa -463 624}%
\special{pa -456 612}\special{pa -456 612}\special{sh 1}\special{ip}%
\special{pn 1}%
\special{pa  -456   612}\special{pa  -493   624}\special{pa  -456   636}\special{pa  -463   624}%
\special{pa  -456   612}%
\special{fp}%
\special{pn 8}%
\special{pa  -296   624}\special{pa  -463   624}%
\special{fp}%
\end{picture}}%
    }
    図の回路で,$\mathrm{E_1}$と$\mathrm{E_2}$は起電力100Vと30Vの
    電池,$\mathrm{R_1}$,$\mathrm{R_2}$,$\mathrm{R_3}$はそれぞれ
    15\sftanni{\Omega },20\sftanni{\Omega },8\sftanni{\Omega }の
    抵抗である。電池の内部抵抗は無視できるものとする。
        \begin{enumerate}
            \item $\mathrm{E_1}$と$\mathrm{R_1}$を流れる電流を,
            図の矢印の向きに,$I_1$\tanni{A},$I_2$\tanni{A}とする。
            次の閉回路についてキルヒホッフの法則を示せ。
                \begin{enumerate}
                    \item $\mathrm{E_1}\rightarrow \mathrm{R_2}\rightarrow \mathrm{R_3}\rightarrow E_1$
                    \item $\mathrm{E_1}\rightarrow \mathrm{R_1}\rightarrow \mathrm{E_2}\rightarrow R_3\rightarrow \mathrm{E_1}$
                \end{enumerate}
            \item $\mathrm{E_1}$,$\mathrm{E_2}$を流れる電流の強さはそれぞれいくらか。また,$\mathrm{E_2}$を流れる電流の向きは,図の左・右どちら向きか。
            \item 3つの抵抗での消費電力の和$P$はいくらか。
            \item $\mathrm{E_1}$の供給電力$Q$はいくらか。
            \item $Q$がPと一致しない理由を簡潔に呼べよ。
        \end{enumerate}
    \end{mawarikomi}