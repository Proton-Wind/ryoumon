\hakosyokika
\item
    \begin{mawarikomi}(10pt,0){260pt}{%WinTpicVersion4.32a
{\unitlength 0.1in%
\begin{picture}(41.0039,13.9469)(2.6969,-16.9390)%
% LINE 2 0 3 0 Black White  
% 4 1000 600 920 800 1000 600 1080 800
% 
\special{pn 8}%
\special{pa 984 591}%
\special{pa 906 787}%
\special{fp}%
\special{pa 984 591}%
\special{pa 1063 787}%
\special{fp}%
% ELLIPSE 2 0 3 1 Black White  
% 4 1000 800 1080 820 880 800 1120 800
% 
\special{pn 8}%
\special{ar 984 787 79 20 6.2831853 3.1415927}%
% LINE 2 0 3 2 Black White  
% 2 980 860 900 900
% 
\special{pn 8}%
\special{pa 965 846}%
\special{pa 886 886}%
\special{fp}%
% LINE 2 0 3 3 Black White  
% 2 838 962 870 885
% 
\special{pn 8}%
\special{pa 825 947}%
\special{pa 856 871}%
\special{fp}%
% SPLINE 2 0 3 4 Black White  
% 41 870 885 871 885 873 884 875 884 877 884 878 884 881 885 882 885 885 886 887 888 889 889 891 891 893 892 895 894 897 896 899 899 900 901 902 904 904 906 905 909 907 912 908 915 909 917 910 921 911 923 912 926 912 929 913 932 913 935 913 937 913 940 913 943 912 945 912 947 911 949 910 951 909 953 908 954 907 956 906 956 904 957
% 
\special{pn 8}%
\special{pa 856 871}%
\special{pa 883 882}%
\special{pa 898 910}%
\special{pa 894 940}%
\special{pa 890 942}%
\special{fp}%
% LINE 2 0 3 5 Black White  
% 2 841 961 905 955
% 
\special{pn 8}%
\special{pa 828 946}%
\special{pa 891 940}%
\special{fp}%
% SPLINE 2 0 3 6 Black White  
% 41 982 859 983 859 983 859 984 859 984 859 985 859 985 860 986 860 987 861 987 861 988 862 989 863 989 864 990 865 991 866 991 867 992 868 992 870 993 871 993 872 994 873 994 875 994 876 995 877 995 879 995 880 995 881 995 882 995 883 995 885 995 886 995 887 995 888 995 889 995 889 994 890 994 891 994 891 993 891 993 892 992 892
% 
\special{pn 8}%
\special{pa 967 845}%
\special{pa 979 869}%
\special{pa 976 878}%
\special{fp}%
% LINE 2 0 3 7 Black White  
% 2 992 892 910 932
% 
\special{pn 8}%
\special{pa 976 878}%
\special{pa 896 917}%
\special{fp}%
% LINE 2 0 3 8 Black White  
% 2 962 818 962 868
% 
\special{pn 8}%
\special{pa 947 805}%
\special{pa 947 854}%
\special{fp}%
% LINE 2 0 3 9 Black White  
% 4 962 908 962 1188 1042 1188 1042 818
% 
\special{pn 8}%
\special{pa 947 894}%
\special{pa 947 1169}%
\special{fp}%
\special{pa 1026 1169}%
\special{pa 1026 805}%
\special{fp}%
% ELLIPSE 2 0 3 10 Black White  
% 4 1002 1184 1042 1208 922 1184 1082 1184
% 
\special{pn 8}%
\special{ar 986 1165 39 24 6.2831853 3.1415927}%
% LINE 2 0 3 0 Black White  
% 4 400 1010 960 1010 1040 1010 4440 1010
% 
\special{pn 8}%
\special{pa 394 994}%
\special{pa 945 994}%
\special{fp}%
\special{pa 1024 994}%
\special{pa 4370 994}%
\special{fp}%
% VECTOR 2 0 3 0 Black White  
% 2 3490 960 4100 1060
% 
\special{pn 8}%
\special{pa 3435 945}%
\special{pa 4035 1043}%
\special{fp}%
\special{sh 1}%
\special{pa 4035 1043}%
\special{pa 3973 1013}%
\special{pa 3983 1034}%
\special{pa 3968 1052}%
\special{pa 4035 1043}%
\special{fp}%
% STR 2 0 3 0 Black White  
% 4 4150 1020 4150 1120 5 0 0 0
% $x$
\put(40.8465,-11.0236){\makebox(0,0){$x$}}%
% VECTOR 2 0 3 0 Black White  
% 2 3780 1520 3780 320
% 
\special{pn 8}%
\special{pa 3720 1496}%
\special{pa 3720 315}%
\special{fp}%
\special{sh 1}%
\special{pa 3720 315}%
\special{pa 3701 381}%
\special{pa 3720 367}%
\special{pa 3740 381}%
\special{pa 3720 315}%
\special{fp}%
% STR 2 0 3 0 Black White  
% 4 3880 280 3880 380 5 0 0 0
% $y$
\put(38.1890,-3.7402){\makebox(0,0){$y$}}%
% LINE 2 0 3 0 Black White  
% 2 3650 600 3650 1400
% 
\special{pn 8}%
\special{pa 3593 591}%
\special{pa 3593 1378}%
\special{fp}%
% LINE 2 0 3 0 Black White  
% 6 3650 1400 3940 1600 3940 1600 3940 490 3940 490 3650 600
% 
\special{pn 8}%
\special{pa 3593 1378}%
\special{pa 3878 1575}%
\special{fp}%
\special{pa 3878 1575}%
\special{pa 3878 482}%
\special{fp}%
\special{pa 3878 482}%
\special{pa 3593 591}%
\special{fp}%
% ELLIPSE 2 0 3 0 Black White  
% 4 2740 1007 3340 2207 1940 207 1940 1807
% 
\special{pn 8}%
\special{ar 2697 991 591 1181 2.6779450 3.6052403}%
% ELLIPSE 2 0 3 0 Black White  
% 4 1666 1007 2266 2207 2466 1807 2466 207
% 
\special{pn 8}%
\special{ar 1640 991 591 1181 5.8195377 0.4636476}%
% STR 2 0 3 0 Black White  
% 4 2360 423 2360 473 2 0 0 0
% 凸レンズ
\put(23.2283,-4.6555){\makebox(0,0)[lb]{凸レンズ}}%
% DOT 0 0 3 0 Black White  
% 1 3000 1008
% 
\special{pn 4}%
\special{sh 1}%
\special{ar 2953 992 16 16 0 6.2831853}%
% DOT 0 0 3 0 Black White  
% 1 1400 1008
% 
\special{pn 4}%
\special{sh 1}%
\special{ar 1378 992 16 16 0 6.2831853}%
% STR 2 0 3 0 Black White  
% 4 3000 1074 3000 1094 5 0 0 0
% F'
\put(29.5276,-10.7677){\makebox(0,0){F'}}%
% STR 2 0 3 0 Black White  
% 4 1400 1074 1400 1094 5 0 0 0
% F
\put(13.7795,-10.7677){\makebox(0,0){F}}%
% STR 2 0 3 0 Black White  
% 4 400 1074 400 1094 5 0 0 0
% 光軸
\put(3.9370,-10.7677){\makebox(0,0){光軸}}%
% STR 2 0 3 0 Black White  
% 4 846 1200 846 1220 1 0 0 0
% 2本の矢印形の
\put(8.3268,-12.0079){\makebox(0,0)[lt]{2本の矢印形の}}%
% STR 2 0 3 0 Black White  
% 4 846 1316 846 1336 1 0 0 0
% 光源
\put(8.3268,-13.1496){\makebox(0,0)[lt]{光源}}%
% STR 2 0 3 0 Black White  
% 4 3560 1706 3560 1756 2 0 0 0
% スクリーン
\put(35.0394,-17.2835){\makebox(0,0)[lb]{スクリーン}}%
% LINE 2 0 3 0 Black White  
% 2 2337 1721 2439 1567
% 
\special{pn 8}%
\special{pa 2300 1694}%
\special{pa 2401 1542}%
\special{fp}%
% SPLINE 2 0 3 1 Black White  
% 41 2413 1605 2415 1607 2418 1610 2420 1612 2422 1613 2424 1615 2425 1618 2428 1620 2429 1622 2432 1624 2433 1627 2435 1629 2437 1631 2439 1634 2441 1636 2442 1639 2444 1641 2445 1643 2447 1645 2448 1648 2450 1650 2452 1653 2453 1656 2454 1658 2456 1661 2457 1664 2458 1666 2460 1669 2461 1672 2462 1675 2463 1678 2464 1680 2466 1683 2466 1686 2467 1689 2468 1692 2469 1694 2469 1697 2470 1699 2471 1703 2471 1706
% 
\special{pn 8}%
\special{pa 2375 1580}%
\special{pa 2396 1602}%
\special{pa 2413 1627}%
\special{pa 2427 1656}%
\special{pa 2432 1679}%
\special{fp}%
% LINE 2 0 3 2 Black White  
% 2 2337 1721 2484 1705
% 
\special{pn 8}%
\special{pa 2300 1694}%
\special{pa 2445 1678}%
\special{fp}%
% SPLINE 2 0 3 3 Black White  
% 41 2454 1709 2452 1708 2450 1705 2449 1703 2447 1701 2445 1700 2444 1697 2442 1696 2440 1695 2439 1692 2437 1691 2436 1688 2434 1687 2432 1684 2430 1682 2429 1680 2428 1677 2426 1676 2425 1673 2424 1670 2423 1669 2421 1666 2419 1665 2418 1662 2417 1659 2416 1657 2415 1655 2413 1652 2412 1650 2411 1648 2410 1646 2409 1643 2407 1640 2407 1638 2406 1636 2405 1633 2404 1631 2403 1628 2402 1626 2401 1623 2401 1621
% 
\special{pn 8}%
\special{pa 2415 1682}%
\special{pa 2396 1660}%
\special{pa 2380 1634}%
\special{pa 2366 1606}%
\special{pa 2363 1595}%
\special{fp}%
% SPLINE 2 0 0 4 Black Black  
% 41 2448 1649 2450 1654 2452 1658 2454 1663 2456 1668 2457 1670 2459 1674 2459 1676 2459 1678 2459 1681 2458 1681 2457 1682 2456 1681 2454 1680 2452 1678 2449 1675 2447 1672 2444 1669 2441 1665 2439 1661 2436 1657 2433 1652 2431 1648 2429 1643 2427 1640 2426 1636 2425 1633 2424 1630 2424 1627 2424 1625 2425 1625 2426 1624 2428 1626 2430 1627 2432 1628 2434 1631 2436 1634 2440 1638 2442 1641 2445 1645 2448 1649
% 
\special{sh 0.400}%
\special{pn 0}%
\special{pa 2409 1623}%
\special{pa 2420 1653}%
\special{pa 2401 1636}%
\special{pa 2387 1607}%
\special{pa 2403 1613}%
\special{pa 2409 1623}%
\special{fp}%
\special{pn 8}%
\special{pa 2409 1623}%
\special{pa 2420 1653}%
\special{pa 2401 1636}%
\special{pa 2387 1607}%
\special{pa 2403 1613}%
\special{pa 2409 1623}%
\special{fp}%
% CIRCLE 2 0 2 5 Black White  
% 4 2422 1636 2429 1634 2429 1634 2429 1634
% 
\special{sh 0}%
\special{ia 2384 1610 7 7 0.0000000 6.2831853}%
\special{pn 8}%
\special{ar 2384 1610 7 7 0.0000000 6.2831853}%
% STR 2 0 3 0 Black White  
% 4 2506 1786 2506 1806 2 0 0 0
% 観測者
\put(24.6654,-17.7756){\makebox(0,0)[lb]{観測者}}%
\end{picture}}%
}
    矢印を組み合わせた形の光源を凸レンズの光軸上に配置したところ,
    スクリーン上に実像ができた。スクリーンは光軸に対して垂直であり,
    F,F'はレンズの焦点である。スクリーンと光軸の交点を原点にして,
    水平方向に$x$軸をとり,レンズ側から見て右向きを正とし,鉛直方向に
    $y$軸をとり上向きを正とする。光源の太い矢印は$y$軸の正の向き,
    細い矢印は$x$軸正の向きを向いている。このとき,観測者がレンズ側から見ると,
    スクリーン上の像は次の\Hako である。
    \end{mawarikomi}
    \begin{center}
        %WinTpicVersion4.32a
{\unitlength 0.1in%
\begin{picture}(49.0945,11.8406)(2.4606,-15.1870)%
% BOX 2 0 1 0 Black Black  
% 2 700 732 1066 1092
% 
\special{pn 0}%
\special{sh 0.200}%
\special{pa 689 720}%
\special{pa 1049 720}%
\special{pa 1049 1075}%
\special{pa 689 1075}%
\special{pa 689 720}%
\special{ip}%
\special{pn 8}%
\special{pa 689 720}%
\special{pa 1049 720}%
\special{pa 1049 1075}%
\special{pa 689 1075}%
\special{pa 689 720}%
\special{pa 1049 720}%
\special{fp}%
% BOX 2 0 1 0 Black Black  
% 2 700 1092 1066 1452
% 
\special{pn 0}%
\special{sh 0.200}%
\special{pa 689 1075}%
\special{pa 1049 1075}%
\special{pa 1049 1429}%
\special{pa 689 1429}%
\special{pa 689 1075}%
\special{ip}%
\special{pn 8}%
\special{pa 689 1075}%
\special{pa 1049 1075}%
\special{pa 1049 1429}%
\special{pa 689 1429}%
\special{pa 689 1075}%
\special{pa 1049 1075}%
\special{fp}%
% BOX 2 0 1 0 Black Black  
% 2 334 1092 700 1452
% 
\special{pn 0}%
\special{sh 0.200}%
\special{pa 329 1075}%
\special{pa 689 1075}%
\special{pa 689 1429}%
\special{pa 329 1429}%
\special{pa 329 1075}%
\special{ip}%
\special{pn 8}%
\special{pa 329 1075}%
\special{pa 689 1075}%
\special{pa 689 1429}%
\special{pa 329 1429}%
\special{pa 329 1075}%
\special{pa 689 1075}%
\special{fp}%
% BOX 2 0 1 0 Black Black  
% 2 334 733 700 1093
% 
\special{pn 0}%
\special{sh 0.200}%
\special{pa 329 721}%
\special{pa 689 721}%
\special{pa 689 1076}%
\special{pa 329 1076}%
\special{pa 329 721}%
\special{ip}%
\special{pn 8}%
\special{pa 329 721}%
\special{pa 689 721}%
\special{pa 689 1076}%
\special{pa 329 1076}%
\special{pa 329 721}%
\special{pa 689 721}%
\special{fp}%
% STR 2 0 3 0 Black White  
% 4 1231 1074 1231 1092 5 0 0 0
% $x$
\put(12.1161,-10.7480){\makebox(0,0){$x$}}%
% STR 2 0 3 0 Black White  
% 4 700 588 700 606 5 0 0 0
% $y$
\put(6.8898,-5.9646){\makebox(0,0){$y$}}%
% VECTOR 2 0 3 0 Black Black  
% 2 288 1093 1188 1093
% 
\special{pn 8}%
\special{pa 283 1076}%
\special{pa 1169 1076}%
\special{fp}%
\special{sh 1}%
\special{pa 1169 1076}%
\special{pa 1103 1056}%
\special{pa 1117 1076}%
\special{pa 1103 1095}%
\special{pa 1169 1076}%
\special{fp}%
% VECTOR 2 0 3 0 Black Black  
% 2 700 1543 700 643
% 
\special{pn 8}%
\special{pa 689 1519}%
\special{pa 689 633}%
\special{fp}%
\special{sh 1}%
\special{pa 689 633}%
\special{pa 669 699}%
\special{pa 689 685}%
\special{pa 709 699}%
\special{pa 689 633}%
\special{fp}%
% POLYGON 2 0 2 0 Black White  
% 16 700 792 588 942 644 942 644 1017 494 1017 494 998 456 1036 494 1074 494 1055 644 1055 644 1430 756 1430 756 942 794 942 794 942 700 792
% 
\special{pn 0}%
\special{sh 0}%
\special{pa 689 780}%
\special{pa 579 927}%
\special{pa 634 927}%
\special{pa 634 1001}%
\special{pa 486 1001}%
\special{pa 486 982}%
\special{pa 449 1020}%
\special{pa 486 1057}%
\special{pa 486 1038}%
\special{pa 634 1038}%
\special{pa 634 1407}%
\special{pa 744 1407}%
\special{pa 744 927}%
\special{pa 781 927}%
\special{pa 781 927}%
\special{pa 689 780}%
\special{ip}%
\special{pn 8}%
\special{pa 689 780}%
\special{pa 579 927}%
\special{pa 634 927}%
\special{pa 634 1001}%
\special{pa 486 1001}%
\special{pa 486 982}%
\special{pa 449 1020}%
\special{pa 486 1057}%
\special{pa 486 1038}%
\special{pa 634 1038}%
\special{pa 634 1407}%
\special{pa 744 1407}%
\special{pa 744 927}%
\special{pa 781 927}%
\special{pa 689 780}%
\special{pa 579 927}%
\special{fp}%
% BOX 2 0 1 0 Black Black  
% 2 2050 732 2416 1092
% 
\special{pn 0}%
\special{sh 0.200}%
\special{pa 2018 720}%
\special{pa 2378 720}%
\special{pa 2378 1075}%
\special{pa 2018 1075}%
\special{pa 2018 720}%
\special{ip}%
\special{pn 8}%
\special{pa 2018 720}%
\special{pa 2378 720}%
\special{pa 2378 1075}%
\special{pa 2018 1075}%
\special{pa 2018 720}%
\special{pa 2378 720}%
\special{fp}%
% BOX 2 0 1 0 Black Black  
% 2 2050 1092 2416 1452
% 
\special{pn 0}%
\special{sh 0.200}%
\special{pa 2018 1075}%
\special{pa 2378 1075}%
\special{pa 2378 1429}%
\special{pa 2018 1429}%
\special{pa 2018 1075}%
\special{ip}%
\special{pn 8}%
\special{pa 2018 1075}%
\special{pa 2378 1075}%
\special{pa 2378 1429}%
\special{pa 2018 1429}%
\special{pa 2018 1075}%
\special{pa 2378 1075}%
\special{fp}%
% BOX 2 0 1 0 Black Black  
% 2 1684 1092 2050 1452
% 
\special{pn 0}%
\special{sh 0.200}%
\special{pa 1657 1075}%
\special{pa 2018 1075}%
\special{pa 2018 1429}%
\special{pa 1657 1429}%
\special{pa 1657 1075}%
\special{ip}%
\special{pn 8}%
\special{pa 1657 1075}%
\special{pa 2018 1075}%
\special{pa 2018 1429}%
\special{pa 1657 1429}%
\special{pa 1657 1075}%
\special{pa 2018 1075}%
\special{fp}%
% BOX 2 0 1 0 Black Black  
% 2 1684 733 2050 1093
% 
\special{pn 0}%
\special{sh 0.200}%
\special{pa 1657 721}%
\special{pa 2018 721}%
\special{pa 2018 1076}%
\special{pa 1657 1076}%
\special{pa 1657 721}%
\special{ip}%
\special{pn 8}%
\special{pa 1657 721}%
\special{pa 2018 721}%
\special{pa 2018 1076}%
\special{pa 1657 1076}%
\special{pa 1657 721}%
\special{pa 2018 721}%
\special{fp}%
% STR 2 0 3 0 Black White  
% 4 2581 1074 2581 1092 5 0 0 0
% $x$
\put(25.4035,-10.7480){\makebox(0,0){$x$}}%
% STR 2 0 3 0 Black White  
% 4 2050 588 2050 606 5 0 0 0
% $y$
\put(20.1772,-5.9646){\makebox(0,0){$y$}}%
% VECTOR 2 0 3 0 Black Black  
% 2 1638 1093 2538 1093
% 
\special{pn 8}%
\special{pa 1612 1076}%
\special{pa 2498 1076}%
\special{fp}%
\special{sh 1}%
\special{pa 2498 1076}%
\special{pa 2432 1056}%
\special{pa 2446 1076}%
\special{pa 2432 1095}%
\special{pa 2498 1076}%
\special{fp}%
% VECTOR 2 0 3 0 Black Black  
% 2 2050 1543 2050 643
% 
\special{pn 8}%
\special{pa 2018 1519}%
\special{pa 2018 633}%
\special{fp}%
\special{sh 1}%
\special{pa 2018 633}%
\special{pa 1998 699}%
\special{pa 2018 685}%
\special{pa 2037 699}%
\special{pa 2018 633}%
\special{fp}%
% POLYGON 2 0 2 0 Black White  
% 16 2050 792 2162 942 2106 942 2106 1017 2256 1017 2256 998 2294 1036 2256 1074 2256 1055 2106 1055 2106 1430 1994 1430 1994 942 1956 942 1956 942 2050 792
% 
\special{pn 0}%
\special{sh 0}%
\special{pa 2018 780}%
\special{pa 2128 927}%
\special{pa 2073 927}%
\special{pa 2073 1001}%
\special{pa 2220 1001}%
\special{pa 2220 982}%
\special{pa 2258 1020}%
\special{pa 2220 1057}%
\special{pa 2220 1038}%
\special{pa 2073 1038}%
\special{pa 2073 1407}%
\special{pa 1963 1407}%
\special{pa 1963 927}%
\special{pa 1925 927}%
\special{pa 1925 927}%
\special{pa 2018 780}%
\special{ip}%
\special{pn 8}%
\special{pa 2018 780}%
\special{pa 2128 927}%
\special{pa 2073 927}%
\special{pa 2073 1001}%
\special{pa 2220 1001}%
\special{pa 2220 982}%
\special{pa 2258 1020}%
\special{pa 2220 1057}%
\special{pa 2220 1038}%
\special{pa 2073 1038}%
\special{pa 2073 1407}%
\special{pa 1963 1407}%
\special{pa 1963 927}%
\special{pa 1925 927}%
\special{pa 2018 780}%
\special{pa 2128 927}%
\special{fp}%
% BOX 2 0 1 0 Black Black  
% 2 3400 732 3766 1092
% 
\special{pn 0}%
\special{sh 0.200}%
\special{pa 3346 720}%
\special{pa 3707 720}%
\special{pa 3707 1075}%
\special{pa 3346 1075}%
\special{pa 3346 720}%
\special{ip}%
\special{pn 8}%
\special{pa 3346 720}%
\special{pa 3707 720}%
\special{pa 3707 1075}%
\special{pa 3346 1075}%
\special{pa 3346 720}%
\special{pa 3707 720}%
\special{fp}%
% BOX 2 0 1 0 Black Black  
% 2 3400 1092 3766 1452
% 
\special{pn 0}%
\special{sh 0.200}%
\special{pa 3346 1075}%
\special{pa 3707 1075}%
\special{pa 3707 1429}%
\special{pa 3346 1429}%
\special{pa 3346 1075}%
\special{ip}%
\special{pn 8}%
\special{pa 3346 1075}%
\special{pa 3707 1075}%
\special{pa 3707 1429}%
\special{pa 3346 1429}%
\special{pa 3346 1075}%
\special{pa 3707 1075}%
\special{fp}%
% BOX 2 0 1 0 Black Black  
% 2 3034 1092 3400 1452
% 
\special{pn 0}%
\special{sh 0.200}%
\special{pa 2986 1075}%
\special{pa 3346 1075}%
\special{pa 3346 1429}%
\special{pa 2986 1429}%
\special{pa 2986 1075}%
\special{ip}%
\special{pn 8}%
\special{pa 2986 1075}%
\special{pa 3346 1075}%
\special{pa 3346 1429}%
\special{pa 2986 1429}%
\special{pa 2986 1075}%
\special{pa 3346 1075}%
\special{fp}%
% BOX 2 0 1 0 Black Black  
% 2 3034 733 3400 1093
% 
\special{pn 0}%
\special{sh 0.200}%
\special{pa 2986 721}%
\special{pa 3346 721}%
\special{pa 3346 1076}%
\special{pa 2986 1076}%
\special{pa 2986 721}%
\special{ip}%
\special{pn 8}%
\special{pa 2986 721}%
\special{pa 3346 721}%
\special{pa 3346 1076}%
\special{pa 2986 1076}%
\special{pa 2986 721}%
\special{pa 3346 721}%
\special{fp}%
% STR 2 0 3 0 Black White  
% 4 3931 1074 3931 1092 5 0 0 0
% $x$
\put(38.6909,-10.7480){\makebox(0,0){$x$}}%
% STR 2 0 3 0 Black White  
% 4 3400 588 3400 606 5 0 0 0
% $y$
\put(33.4646,-5.9646){\makebox(0,0){$y$}}%
% VECTOR 2 0 3 0 Black Black  
% 2 2988 1093 3888 1093
% 
\special{pn 8}%
\special{pa 2941 1076}%
\special{pa 3827 1076}%
\special{fp}%
\special{sh 1}%
\special{pa 3827 1076}%
\special{pa 3761 1056}%
\special{pa 3775 1076}%
\special{pa 3761 1095}%
\special{pa 3827 1076}%
\special{fp}%
% VECTOR 2 0 3 0 Black Black  
% 2 3400 1543 3400 643
% 
\special{pn 8}%
\special{pa 3346 1519}%
\special{pa 3346 633}%
\special{fp}%
\special{sh 1}%
\special{pa 3346 633}%
\special{pa 3327 699}%
\special{pa 3346 685}%
\special{pa 3366 699}%
\special{pa 3346 633}%
\special{fp}%
% POLYGON 2 0 2 0 Black White  
% 16 3400 1430 3288 1280 3344 1280 3344 1205 3194 1205 3194 1224 3156 1186 3194 1148 3194 1167 3344 1167 3344 792 3456 792 3456 1280 3494 1280 3494 1280 3400 1430
% 
\special{pn 0}%
\special{sh 0}%
\special{pa 3346 1407}%
\special{pa 3236 1260}%
\special{pa 3291 1260}%
\special{pa 3291 1186}%
\special{pa 3144 1186}%
\special{pa 3144 1205}%
\special{pa 3106 1167}%
\special{pa 3144 1130}%
\special{pa 3144 1149}%
\special{pa 3291 1149}%
\special{pa 3291 780}%
\special{pa 3402 780}%
\special{pa 3402 1260}%
\special{pa 3439 1260}%
\special{pa 3439 1260}%
\special{pa 3346 1407}%
\special{ip}%
\special{pn 8}%
\special{pa 3346 1407}%
\special{pa 3236 1260}%
\special{pa 3291 1260}%
\special{pa 3291 1186}%
\special{pa 3144 1186}%
\special{pa 3144 1205}%
\special{pa 3106 1167}%
\special{pa 3144 1130}%
\special{pa 3144 1149}%
\special{pa 3291 1149}%
\special{pa 3291 780}%
\special{pa 3402 780}%
\special{pa 3402 1260}%
\special{pa 3439 1260}%
\special{pa 3346 1407}%
\special{pa 3236 1260}%
\special{fp}%
% BOX 2 0 1 0 Black Black  
% 2 4750 732 5116 1092
% 
\special{pn 0}%
\special{sh 0.200}%
\special{pa 4675 720}%
\special{pa 5035 720}%
\special{pa 5035 1075}%
\special{pa 4675 1075}%
\special{pa 4675 720}%
\special{ip}%
\special{pn 8}%
\special{pa 4675 720}%
\special{pa 5035 720}%
\special{pa 5035 1075}%
\special{pa 4675 1075}%
\special{pa 4675 720}%
\special{pa 5035 720}%
\special{fp}%
% BOX 2 0 1 0 Black Black  
% 2 4750 1092 5116 1452
% 
\special{pn 0}%
\special{sh 0.200}%
\special{pa 4675 1075}%
\special{pa 5035 1075}%
\special{pa 5035 1429}%
\special{pa 4675 1429}%
\special{pa 4675 1075}%
\special{ip}%
\special{pn 8}%
\special{pa 4675 1075}%
\special{pa 5035 1075}%
\special{pa 5035 1429}%
\special{pa 4675 1429}%
\special{pa 4675 1075}%
\special{pa 5035 1075}%
\special{fp}%
% BOX 2 0 1 0 Black Black  
% 2 4384 1092 4750 1452
% 
\special{pn 0}%
\special{sh 0.200}%
\special{pa 4315 1075}%
\special{pa 4675 1075}%
\special{pa 4675 1429}%
\special{pa 4315 1429}%
\special{pa 4315 1075}%
\special{ip}%
\special{pn 8}%
\special{pa 4315 1075}%
\special{pa 4675 1075}%
\special{pa 4675 1429}%
\special{pa 4315 1429}%
\special{pa 4315 1075}%
\special{pa 4675 1075}%
\special{fp}%
% BOX 2 0 1 0 Black Black  
% 2 4384 733 4750 1093
% 
\special{pn 0}%
\special{sh 0.200}%
\special{pa 4315 721}%
\special{pa 4675 721}%
\special{pa 4675 1076}%
\special{pa 4315 1076}%
\special{pa 4315 721}%
\special{ip}%
\special{pn 8}%
\special{pa 4315 721}%
\special{pa 4675 721}%
\special{pa 4675 1076}%
\special{pa 4315 1076}%
\special{pa 4315 721}%
\special{pa 4675 721}%
\special{fp}%
% STR 2 0 3 0 Black White  
% 4 5281 1074 5281 1092 5 0 0 0
% $x$
\put(51.9783,-10.7480){\makebox(0,0){$x$}}%
% STR 2 0 3 0 Black White  
% 4 4750 588 4750 606 5 0 0 0
% $y$
\put(46.7520,-5.9646){\makebox(0,0){$y$}}%
% VECTOR 2 0 3 0 Black Black  
% 2 4338 1093 5238 1093
% 
\special{pn 8}%
\special{pa 4270 1076}%
\special{pa 5156 1076}%
\special{fp}%
\special{sh 1}%
\special{pa 5156 1076}%
\special{pa 5090 1056}%
\special{pa 5103 1076}%
\special{pa 5090 1095}%
\special{pa 5156 1076}%
\special{fp}%
% VECTOR 2 0 3 0 Black Black  
% 2 4750 1543 4750 643
% 
\special{pn 8}%
\special{pa 4675 1519}%
\special{pa 4675 633}%
\special{fp}%
\special{sh 1}%
\special{pa 4675 633}%
\special{pa 4656 699}%
\special{pa 4675 685}%
\special{pa 4695 699}%
\special{pa 4675 633}%
\special{fp}%
% POLYGON 2 0 2 0 Black White  
% 16 4750 1430 4862 1280 4806 1280 4806 1205 4956 1205 4956 1224 4994 1186 4956 1148 4956 1167 4806 1167 4806 792 4694 792 4694 1280 4656 1280 4656 1280 4750 1430
% 
\special{pn 0}%
\special{sh 0}%
\special{pa 4675 1407}%
\special{pa 4785 1260}%
\special{pa 4730 1260}%
\special{pa 4730 1186}%
\special{pa 4878 1186}%
\special{pa 4878 1205}%
\special{pa 4915 1167}%
\special{pa 4878 1130}%
\special{pa 4878 1149}%
\special{pa 4730 1149}%
\special{pa 4730 780}%
\special{pa 4620 780}%
\special{pa 4620 1260}%
\special{pa 4583 1260}%
\special{pa 4583 1260}%
\special{pa 4675 1407}%
\special{ip}%
\special{pn 8}%
\special{pa 4675 1407}%
\special{pa 4785 1260}%
\special{pa 4730 1260}%
\special{pa 4730 1186}%
\special{pa 4878 1186}%
\special{pa 4878 1205}%
\special{pa 4915 1167}%
\special{pa 4878 1130}%
\special{pa 4878 1149}%
\special{pa 4730 1149}%
\special{pa 4730 780}%
\special{pa 4620 780}%
\special{pa 4620 1260}%
\special{pa 4583 1260}%
\special{pa 4675 1407}%
\special{pa 4785 1260}%
\special{fp}%
% STR 2 0 3 0 Black White  
% 4 250 432 250 470 2 0 0 0
% \maru{1}
\put(2.4606,-4.6260){\makebox(0,0)[lb]{\maru{1}}}%
% STR 2 0 3 0 Black White  
% 4 1590 440 1590 478 2 0 0 0
% \maru{2}
\put(15.6496,-4.7047){\makebox(0,0)[lb]{\maru{2}}}%
% STR 2 0 3 0 Black White  
% 4 2940 432 2940 470 2 0 0 0
% \maru{3}
\put(28.9370,-4.6260){\makebox(0,0)[lb]{\maru{3}}}%
% STR 2 0 3 0 Black White  
% 4 4290 432 4290 470 2 0 0 0
% \maru{4}
\put(42.2244,-4.6260){\makebox(0,0)[lb]{\maru{4}}}%
\end{picture}}%

    \end{center}
     光源とスクリーンの距離が100\sftanni{cm}で,実像の倍率が1だったので,
    レンズの焦点距離は\Hako \sftanni{cm}である。\\
     次にレンズの中心より上半分に黒い紙を貼った。スクリーン上の像はどのようになるか。
    次のうちから選べ。\Hako 
        \begin{edaenumerate}[m]
            \item $y>0$の部分が見えなくなった。
            \item 全体が暗くなった。
            \item $y<0$の部分が見えなくなった。
            \item 何も見えなくなった。
        \end{edaenumerate}
