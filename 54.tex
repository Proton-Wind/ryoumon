\hakosyokika
\item 質量110\sftanni{g}の銅製の熱量計に水50\sftanni{g}を入れて温度を測ると20℃であった。そこへ80℃の高温の水30\sftanni{g}を加えたところ,全体の温度が40℃になった。水の比熱を4.2\sftanni{J/(g\cdot K)}とし,外部との熱の出入りはないものとする。
\begin{enumerate}
    \item 高温の水が失った熱量$Q_0$は\Hako \sftanni{J}となる。
    \item 熱量計の熱容量$C_M$は\Hako \sftanni{J/K}となる。
    \item (2)より銅の比熱$c_1$は\Hako \sftanni{J/(g\cdot K)}となる。
    \item 全体の温度を40℃から50℃にしたい場合,80℃の高温の水をさらに\Hako \sftanni{g}加えればよいことになる。
    \item 全体の温度が50℃となった(4)の状態で,さらにこの中へ,100℃に加熱された質量400\sftanni{g}の金属球を入れたとき,全体の温度が60℃となった。この金属球の比熱$c_2$は\Hako \sftanni{J/(g\cdot K)}となる。
\end{enumerate}