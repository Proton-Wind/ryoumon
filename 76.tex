\hakosyokika
\item
    \begin{mawarikomi}(10pt,0){210pt}{%WinTpicVersion4.32a
{\unitlength 0.1in%
\begin{picture}(23.3500,11.0000)(3.8300,-13.8000)%
% LINE 2 0 3 0 Black White  
% 6 788 600 788 550 788 550 538 550 538 600 788 600
% 
\special{pn 8}%
\special{pa 788 600}%
\special{pa 788 550}%
\special{fp}%
\special{pa 788 550}%
\special{pa 538 550}%
\special{fp}%
\special{pa 538 600}%
\special{pa 788 600}%
\special{fp}%
% CIRCLE 2 0 3 1 Black White  
% 4 538 650 588 650 538 550 538 750
% 
\special{pn 8}%
\special{ar 538 650 50 50 1.5707963 4.7123890}%
% CIRCLE 2 0 3 2 Black White  
% 4 538 650 638 650 538 550 538 800
% 
\special{pn 8}%
\special{ar 538 650 100 100 1.5707963 4.7123890}%
% LINE 2 0 3 3 Black White  
% 6 538 700 788 700 788 700 788 750 788 750 538 750
% 
\special{pn 8}%
\special{pa 538 700}%
\special{pa 788 700}%
\special{fp}%
\special{pa 788 700}%
\special{pa 788 750}%
\special{fp}%
\special{pa 788 750}%
\special{pa 538 750}%
\special{fp}%
% BOX 2 5 2 4 Black White  
% 2 90 625 453 675
% 
\special{pn 0}%
\special{sh 0}%
\special{pa 90 625}%
\special{pa 453 625}%
\special{pa 453 675}%
\special{pa 90 675}%
\special{pa 90 625}%
\special{ip}%
\special{pn 8}%
\special{pa 90 625}%
\special{pa 453 625}%
\special{pa 453 675}%
\special{pa 90 675}%
\special{pa 90 625}%
\special{ip}%
% LINE 2 0 3 5 Black White  
% 6 440 675 90 675 90 675 90 625 90 625 440 625
% 
\special{pn 8}%
\special{pa 440 675}%
\special{pa 90 675}%
\special{fp}%
\special{pa 90 675}%
\special{pa 90 625}%
\special{fp}%
\special{pa 90 625}%
\special{pa 440 625}%
\special{fp}%
% FUNC 2 0 3 0 Black White  
% 9 788 450 2588 650 788 550 1388 550 788 450 788 450 2588 650 0 2 1 0
% sin(x)
\special{pn 8}%
\special{pa 785 552}%
\special{pa 795 546}%
\special{pa 800 544}%
\special{pa 805 541}%
\special{pa 810 539}%
\special{pa 820 533}%
\special{pa 825 531}%
\special{pa 830 528}%
\special{pa 835 526}%
\special{pa 840 523}%
\special{pa 845 521}%
\special{pa 850 518}%
\special{pa 855 516}%
\special{pa 860 513}%
\special{pa 865 511}%
\special{pa 870 508}%
\special{pa 880 504}%
\special{pa 885 501}%
\special{pa 900 495}%
\special{pa 905 492}%
\special{pa 935 480}%
\special{pa 940 479}%
\special{pa 955 473}%
\special{pa 960 472}%
\special{pa 970 468}%
\special{pa 980 466}%
\special{pa 985 464}%
\special{pa 995 462}%
\special{pa 1000 460}%
\special{pa 1025 455}%
\special{pa 1030 455}%
\special{pa 1040 453}%
\special{pa 1045 453}%
\special{pa 1055 451}%
\special{pa 1065 451}%
\special{pa 1070 450}%
\special{pa 1105 450}%
\special{pa 1110 451}%
\special{pa 1120 451}%
\special{pa 1125 452}%
\special{pa 1130 452}%
\special{pa 1140 454}%
\special{pa 1145 454}%
\special{pa 1180 461}%
\special{pa 1185 463}%
\special{pa 1195 465}%
\special{pa 1200 467}%
\special{pa 1205 468}%
\special{pa 1210 470}%
\special{pa 1215 471}%
\special{pa 1225 475}%
\special{pa 1230 476}%
\special{pa 1280 496}%
\special{pa 1285 499}%
\special{pa 1295 503}%
\special{pa 1300 506}%
\special{pa 1310 510}%
\special{pa 1315 513}%
\special{pa 1320 515}%
\special{pa 1325 518}%
\special{pa 1330 520}%
\special{pa 1335 523}%
\special{pa 1340 525}%
\special{pa 1345 528}%
\special{pa 1350 530}%
\special{pa 1355 533}%
\special{pa 1360 535}%
\special{pa 1370 541}%
\special{pa 1375 543}%
\special{pa 1380 546}%
\special{pa 1385 548}%
\special{pa 1395 554}%
\special{pa 1400 556}%
\special{pa 1405 559}%
\special{pa 1410 561}%
\special{pa 1420 567}%
\special{pa 1425 569}%
\special{pa 1430 572}%
\special{pa 1435 574}%
\special{pa 1440 577}%
\special{pa 1445 579}%
\special{pa 1450 582}%
\special{pa 1455 584}%
\special{pa 1460 587}%
\special{pa 1465 589}%
\special{pa 1470 592}%
\special{pa 1480 596}%
\special{pa 1485 599}%
\special{pa 1500 605}%
\special{pa 1505 608}%
\special{pa 1535 620}%
\special{pa 1540 621}%
\special{pa 1555 627}%
\special{pa 1560 628}%
\special{pa 1570 632}%
\special{pa 1580 634}%
\special{pa 1585 636}%
\special{pa 1595 638}%
\special{pa 1600 640}%
\special{pa 1625 645}%
\special{pa 1630 645}%
\special{pa 1640 647}%
\special{pa 1645 647}%
\special{pa 1655 649}%
\special{pa 1665 649}%
\special{pa 1670 650}%
\special{pa 1705 650}%
\special{pa 1710 649}%
\special{pa 1720 649}%
\special{pa 1725 648}%
\special{pa 1730 648}%
\special{pa 1740 646}%
\special{pa 1745 646}%
\special{pa 1780 639}%
\special{pa 1785 637}%
\special{pa 1795 635}%
\special{pa 1800 633}%
\special{pa 1805 632}%
\special{pa 1810 630}%
\special{pa 1815 629}%
\special{pa 1825 625}%
\special{pa 1830 624}%
\special{pa 1880 604}%
\special{pa 1885 601}%
\special{pa 1895 597}%
\special{pa 1900 594}%
\special{pa 1910 590}%
\special{pa 1915 587}%
\special{pa 1920 585}%
\special{pa 1925 582}%
\special{pa 1930 580}%
\special{pa 1935 577}%
\special{pa 1940 575}%
\special{pa 1945 572}%
\special{pa 1950 570}%
\special{pa 1955 567}%
\special{pa 1960 565}%
\special{pa 1970 559}%
\special{pa 1975 557}%
\special{pa 1980 554}%
\special{pa 1985 552}%
\special{pa 1995 546}%
\special{pa 2000 544}%
\special{pa 2005 541}%
\special{pa 2010 539}%
\special{pa 2020 533}%
\special{pa 2025 531}%
\special{pa 2030 528}%
\special{pa 2035 526}%
\special{pa 2040 523}%
\special{pa 2045 521}%
\special{pa 2050 518}%
\special{pa 2055 516}%
\special{pa 2060 513}%
\special{pa 2065 511}%
\special{pa 2070 508}%
\special{pa 2080 504}%
\special{pa 2085 501}%
\special{pa 2100 495}%
\special{pa 2105 492}%
\special{pa 2135 480}%
\special{pa 2140 479}%
\special{pa 2155 473}%
\special{pa 2160 472}%
\special{pa 2170 468}%
\special{pa 2180 466}%
\special{pa 2185 464}%
\special{pa 2195 462}%
\special{pa 2200 460}%
\special{pa 2225 455}%
\special{pa 2230 455}%
\special{pa 2240 453}%
\special{pa 2245 453}%
\special{pa 2255 451}%
\special{pa 2265 451}%
\special{pa 2270 450}%
\special{pa 2305 450}%
\special{pa 2310 451}%
\special{pa 2320 451}%
\special{pa 2325 452}%
\special{pa 2330 452}%
\special{pa 2340 454}%
\special{pa 2345 454}%
\special{pa 2380 461}%
\special{pa 2385 463}%
\special{pa 2395 465}%
\special{pa 2400 467}%
\special{pa 2405 468}%
\special{pa 2410 470}%
\special{pa 2415 471}%
\special{pa 2425 475}%
\special{pa 2430 476}%
\special{pa 2480 496}%
\special{pa 2485 499}%
\special{pa 2495 503}%
\special{pa 2500 506}%
\special{pa 2510 510}%
\special{pa 2515 513}%
\special{pa 2520 515}%
\special{pa 2525 518}%
\special{pa 2530 520}%
\special{pa 2535 523}%
\special{pa 2540 525}%
\special{pa 2545 528}%
\special{pa 2550 530}%
\special{pa 2555 533}%
\special{pa 2560 535}%
\special{pa 2570 541}%
\special{pa 2575 543}%
\special{pa 2580 546}%
\special{pa 2585 548}%
\special{fp}%
% FUNC 2 2 3 0 Black White  
% 10 788 450 2588 650 788 550 1388 550 788 450 788 450 2588 650 0 2 1 0 0 0
% -sin(x)
\special{pn 8}%
\special{pn 8}%
\special{pa 785 548}%
\special{pa 792 552}%
\special{fp}%
\special{pa 807 560}%
\special{pa 814 564}%
\special{fp}%
\special{pa 829 572}%
\special{pa 837 575}%
\special{fp}%
\special{pa 852 583}%
\special{pa 859 586}%
\special{fp}%
\special{pa 874 594}%
\special{pa 882 597}%
\special{fp}%
\special{pa 897 604}%
\special{pa 905 608}%
\special{fp}%
\special{pa 920 614}%
\special{pa 928 617}%
\special{fp}%
\special{pa 944 623}%
\special{pa 952 626}%
\special{fp}%
\special{pa 968 631}%
\special{pa 976 633}%
\special{fp}%
\special{pa 992 637}%
\special{pa 1000 640}%
\special{fp}%
\special{pa 1016 643}%
\special{pa 1024 645}%
\special{fp}%
\special{pa 1041 647}%
\special{pa 1049 648}%
\special{fp}%
\special{pa 1066 649}%
\special{pa 1074 650}%
\special{fp}%
\special{pa 1091 650}%
\special{pa 1099 650}%
\special{fp}%
\special{pa 1117 649}%
\special{pa 1124 648}%
\special{fp}%
\special{pa 1141 646}%
\special{pa 1149 645}%
\special{fp}%
\special{pa 1166 642}%
\special{pa 1174 640}%
\special{fp}%
\special{pa 1191 636}%
\special{pa 1198 634}%
\special{fp}%
\special{pa 1215 629}%
\special{pa 1222 626}%
\special{fp}%
\special{pa 1238 621}%
\special{pa 1246 618}%
\special{fp}%
\special{pa 1262 611}%
\special{pa 1269 608}%
\special{fp}%
\special{pa 1285 601}%
\special{pa 1292 598}%
\special{fp}%
\special{pa 1308 591}%
\special{pa 1315 587}%
\special{fp}%
\special{pa 1330 580}%
\special{pa 1337 576}%
\special{fp}%
\special{pa 1353 568}%
\special{pa 1360 565}%
\special{fp}%
\special{pa 1375 557}%
\special{pa 1382 553}%
\special{fp}%
\special{pa 1397 545}%
\special{pa 1404 541}%
\special{fp}%
\special{pa 1419 533}%
\special{pa 1427 530}%
\special{fp}%
\special{pa 1442 522}%
\special{pa 1449 519}%
\special{fp}%
\special{pa 1465 511}%
\special{pa 1472 507}%
\special{fp}%
\special{pa 1487 500}%
\special{pa 1495 497}%
\special{fp}%
\special{pa 1510 490}%
\special{pa 1518 487}%
\special{fp}%
\special{pa 1534 481}%
\special{pa 1541 478}%
\special{fp}%
\special{pa 1557 473}%
\special{pa 1565 470}%
\special{fp}%
\special{pa 1581 465}%
\special{pa 1589 463}%
\special{fp}%
\special{pa 1606 459}%
\special{pa 1614 457}%
\special{fp}%
\special{pa 1630 455}%
\special{pa 1638 453}%
\special{fp}%
\special{pa 1655 451}%
\special{pa 1663 451}%
\special{fp}%
\special{pa 1680 450}%
\special{pa 1688 450}%
\special{fp}%
\special{pa 1706 450}%
\special{pa 1713 451}%
\special{fp}%
\special{pa 1731 452}%
\special{pa 1738 454}%
\special{fp}%
\special{pa 1755 456}%
\special{pa 1763 458}%
\special{fp}%
\special{pa 1780 461}%
\special{pa 1788 464}%
\special{fp}%
\special{pa 1804 468}%
\special{pa 1812 470}%
\special{fp}%
\special{pa 1828 476}%
\special{pa 1836 478}%
\special{fp}%
\special{pa 1852 485}%
\special{pa 1859 488}%
\special{fp}%
\special{pa 1875 494}%
\special{pa 1882 497}%
\special{fp}%
\special{pa 1898 505}%
\special{pa 1905 508}%
\special{fp}%
\special{pa 1920 515}%
\special{pa 1928 519}%
\special{fp}%
\special{pa 1943 527}%
\special{pa 1950 530}%
\special{fp}%
\special{pa 1965 538}%
\special{pa 1972 542}%
\special{fp}%
\special{pa 1988 550}%
\special{pa 1995 554}%
\special{fp}%
\special{pa 2010 561}%
\special{pa 2017 565}%
\special{fp}%
\special{pa 2032 573}%
\special{pa 2039 577}%
\special{fp}%
\special{pa 2055 584}%
\special{pa 2062 588}%
\special{fp}%
\special{pa 2077 595}%
\special{pa 2084 599}%
\special{fp}%
\special{pa 2100 605}%
\special{pa 2107 609}%
\special{fp}%
\special{pa 2123 615}%
\special{pa 2131 618}%
\special{fp}%
\special{pa 2147 624}%
\special{pa 2154 627}%
\special{fp}%
\special{pa 2171 632}%
\special{pa 2178 634}%
\special{fp}%
\special{pa 2195 638}%
\special{pa 2203 641}%
\special{fp}%
\special{pa 2219 644}%
\special{pa 2227 645}%
\special{fp}%
\special{pa 2244 647}%
\special{pa 2252 648}%
\special{fp}%
\special{pa 2269 650}%
\special{pa 2277 650}%
\special{fp}%
\special{pa 2294 650}%
\special{pa 2303 650}%
\special{fp}%
\special{pa 2320 649}%
\special{pa 2328 648}%
\special{fp}%
\special{pa 2344 646}%
\special{pa 2352 645}%
\special{fp}%
\special{pa 2369 641}%
\special{pa 2377 640}%
\special{fp}%
\special{pa 2394 635}%
\special{pa 2401 633}%
\special{fp}%
\special{pa 2418 628}%
\special{pa 2425 625}%
\special{fp}%
\special{pa 2441 619}%
\special{pa 2449 616}%
\special{fp}%
\special{pa 2465 610}%
\special{pa 2472 607}%
\special{fp}%
\special{pa 2488 600}%
\special{pa 2495 597}%
\special{fp}%
\special{pa 2511 590}%
\special{pa 2518 586}%
\special{fp}%
\special{pa 2533 578}%
\special{pa 2540 575}%
\special{fp}%
\special{pa 2555 567}%
\special{pa 2563 563}%
\special{fp}%
\special{pa 2578 555}%
\special{pa 2585 552}%
\special{fp}%
% CIRCLE 2 0 2 0 Black White  
% 4 2558 590 2618 590 2618 590 2618 590
% 
\special{sh 0}%
\special{ia 2558 590 60 60 0.0000000 6.2831853}%
\special{pn 8}%
\special{ar 2558 590 60 60 0.0000000 6.2831853}%
% DOT 1 0 3 0 Black White  
% 1 2558 590
% 
\special{pn 4}%
\special{sh 1}%
\special{ar 2558 590 10 10 0 6.2831853}%
% LINE 2 0 3 0 Black White  
% 2 2618 590 2618 790
% 
\special{pn 8}%
\special{pa 2618 590}%
\special{pa 2618 790}%
\special{fp}%
% CIRCLE 2 0 3 0 Black White  
% 4 2618 840 2668 840 2268 840 2618 640
% 
\special{pn 8}%
\special{ar 2618 840 50 50 4.7123890 3.1415927}%
% CIRCLE 2 0 3 0 Black White  
% 4 2618 920 2668 920 2618 1120 2418 920
% 
\special{pn 8}%
\special{ar 2618 920 50 50 3.1415927 1.5707963}%
% LINE 2 0 3 0 Black White  
% 2 2618 960 2618 1080
% 
\special{pn 8}%
\special{pa 2618 960}%
\special{pa 2618 1080}%
\special{fp}%
% BOX 2 0 3 0 Black White  
% 2 2518 1080 2718 1180
% 
\special{pn 8}%
\special{pa 2518 1080}%
\special{pa 2718 1080}%
\special{pa 2718 1180}%
\special{pa 2518 1180}%
\special{pa 2518 1080}%
\special{pa 2718 1080}%
\special{fp}%
% LINE 2 0 3 0 Black White  
% 2 2618 1180 2618 1280
% 
\special{pn 8}%
\special{pa 2618 1180}%
\special{pa 2618 1280}%
\special{fp}%
% CIRCLE 2 0 3 0 Black White  
% 4 2618 1330 2668 1330 2268 1330 2618 1130
% 
\special{pn 8}%
\special{ar 2618 1330 50 50 4.7123890 3.1415927}%
% STR 2 0 3 0 Black White  
% 4 418 380 418 480 5 0 0 0
% A
\put(4.1800,-4.8000){\makebox(0,0){A}}%
% LINE 2 0 3 0 Black White  
% 2 788 480 788 280
% 
\special{pn 8}%
\special{pa 788 480}%
\special{pa 788 280}%
\special{fp}%
% LINE 2 0 3 0 Black White  
% 2 2598 280 2598 480
% 
\special{pn 8}%
\special{pa 2598 280}%
\special{pa 2598 480}%
\special{fp}%
% VECTOR 2 0 3 0 Black White  
% 2 1498 370 788 370
% 
\special{pn 8}%
\special{pa 1498 370}%
\special{pa 788 370}%
\special{fp}%
\special{sh 1}%
\special{pa 788 370}%
\special{pa 855 390}%
\special{pa 841 370}%
\special{pa 855 350}%
\special{pa 788 370}%
\special{fp}%
% VECTOR 2 0 3 0 Black White  
% 2 1788 370 2598 370
% 
\special{pn 8}%
\special{pa 1788 370}%
\special{pa 2598 370}%
\special{fp}%
\special{sh 1}%
\special{pa 2598 370}%
\special{pa 2531 350}%
\special{pa 2545 370}%
\special{pa 2531 390}%
\special{pa 2598 370}%
\special{fp}%
% STR 2 0 3 0 Black White  
% 4 1638 270 1638 370 5 0 0 0
% $\ell $
\put(16.3800,-3.7000){\makebox(0,0){$\ell $}}%
% STR 2 0 3 0 Black White  
% 4 2358 1030 2358 1130 5 0 0 0
% $m$
\put(23.5800,-11.3000){\makebox(0,0){$m$}}%
\end{picture}}%
}
    音さAに減の左端を固定し,水平に移動することのできる滑車を通して右端におもりをつるし,弦の長さを変えられるようにした装置がある。
    弦の線密度を$\rho $,おもりの質量を$m$とし,重力加速度の大きさを$g$とする。\\
    ~~いま,弦の長さを$\ell $とし,音さを振動させると,図のように腹が3個ある定在波を生じた。これから音さの振動数$f$は\Hako と分かる。
    次に,おもりの下にもう1つ,質量$M$のおもりを下げたら,同じ音さによって今度はちょうど腹が2個の定在波ができた。これから$M$は$m$の\Hako 倍である。\\
    ~~今度はAとほんの少し振動数の違う別の音さBをとりつけて,図と同じく質量$m$のおもりだけで実験した。このとき,同じく腹が3個の定在波を作るためには,滑車を動かして
    弦の長さを少しだけ長くしなければならなかった。そして,音さA,Bを同時に鳴らすと毎秒$n$回のうなりが聞こえた。これからBの振動数は$f$,$n$を用いて\Hako と表される。
    また弦の長さは,弦を伝わる波の速さ$v$,$\ell $,$n$を用いて\Hako だけ長くされたことが分かる。
    \end{mawarikomi}