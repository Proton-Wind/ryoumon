\hakosyokika
\item
    \begin{mawarikomi}(10pt,0){210pt}{{\small
\begin{zahyou}[
    ul=8mm
    ,yokozikukigou=$x$\tanni{m}
    ,tatezikukigou=$y$\tanni{m}
    ,yokozikuhaiti={[se]}
    ,tatezikuhaiti={[n]}
    ,xscale=1
    ,yscale=10](-2,5.5)(-0.15,0.15)
    \def\Fx{0.1*cos(2*$pi *X/4)}
    \tenretu*{
        A(5,0.15);
        B(5,-0.15);
        C(5.2,-0.15);
        D(5.2,0.15);
    }
    \Nuritubusi[0.3]{\A\B\C\D\A}
    \Drawlines{\A\B}
    {\thicklines
    \YGurafu\Fx\xmin{5}
    }
    \put(5.2,0.01){P}
    \zahyouMemori<dx=1,dy=0.1>
\end{zahyou}}
}
        図は縦波を表すグラフである。$x$軸は媒質のつり合いの位置を$y$軸は左右への媒質の変位(右方向を正)を表す。波は右へ速さ$2$\sftanni{m/s}で進み,波の先端が自由端P($x=5$\sftanni{m}の位置)に達した時刻を$t=0$\sftanni{s}とする。
        \begin{enumerate}
            \item 波の周期はいくらか。
            \item $t=0$\sftanni{s}において,媒質の密度が最も疎である点の$x$座標を図の範囲で答えよ。
            \item 右方向の媒質の速度を正として時刻$t=0$\sftanni{s}において,各位置における媒質の速度$u$の概略を,図の範囲内でグラフに描け。
            \item
                \begin{enumerate}
                    \item この波が自由端Pで反射して,反射波の先端が点$x=0$\sftanni{m}に達する時刻を求めよ。
                    \item その時刻において,図に示す各位置での変化をグラフに描け。
                    \item $x=0$\sftanni{m}における媒質変位の時間変化を$0\leqq t \leqq 4.5$\sftanni{s}の範囲でグラフを描け。
                \end{enumerate}
            \item Pが固定端の場合について,前問{\bf イ},{\bf ウ}のグラフを描け。
        \end{enumerate}
    \end{mawarikomi}