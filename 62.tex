\hakosyokika
\item
    \begin{mawarikomi}(20pt,0){150pt}{{\unitlength3mm\small
% \Drawaxisfalse
\begin{zahyou}[%
    % yokozikuhaiti{[s]}
    % ,tatezikuhaiti{[e]}
    yokozikukigou={体積\tanni{m^3}}
    ,tatezikukigou={圧力\tanni{Pa}}
    ,gentenkigou={0}
    ,hidariyohaku=0.5
    ,tatezikuhaiti={[e]}
    ](0,14)(0,14)
\small
\def\A{(4,4)}
\def\B{(4,12)}
\def\C{(12,4)}
\def\Fx{48/X}
\kuromaru{\A;\B;\C}
\Put\A[sw]{A}
\Put\B[n]{B}
\Put\C[se]{C}
\Put\A[syaei=xy,xlabel=V,ylabel=P]{}
\Put\B[syaei=y,ylabel=3P]{}
% \Put\C[syaei=x,xlabel=2V_1]{}
% \put(2.5,\YB){\yasen(-0.2,0)}
% \put(2,\BCY){\yasen(0.2,-1.3)}
% \put(5,\YC){\yasen(0.2,0)}
% \calcval{(\YC+\YE)*0.5}\CDY
% \put(\XD,\CDY){\yasen(0,0.5)}
% \put(4.9,\AEY){\yasen(-0.2,0.8)}
{\thicklines
\Drawlines{\C\A\B}
\YGurafu\Fx{4}{12}
\changeArrowHeadSize<0.333>{2}
\put(4,8){\yasen(0,0.1)}
\put(6.928,6.928){\yasen(0.1,-0.1)}
\put(8,4){\yasen(-0.1,0)}
}
\end{zahyou}}
}
        単原子分子理想気体の状態を図に示すようなA$\rightarrow$B$\rightarrow$C$\rightarrow$Aの経路に沿って,ゆっくり変化させた。B$\rightarrow$Cは等温変化である。
        \begin{enumerate}
            \item A$\rightarrow$Bの過程における気体の内部エネルギーの変化は\Hako \tanni{J}である。
            \item B$\rightarrow$Cの過程において気体が吸収した熱量が$Q$\tanni{J}であるとすると,気体の内部エネルギーの変化は\Hako であり,気体が外部にした仕事は\Hako \tanni{J}である。
            \item C$\rightarrow$Aの過程で気体が受けた仕事は\Hako \tanni{J}である。
            \item A$\rightarrow$B$\rightarrow$C$\rightarrow$Aの1サイクルで気体がした正味の仕事は\Hako \tanni{J}であり,このサイクルの熱効率は\Hako である。
        \end{enumerate}
    \end{mawarikomi}