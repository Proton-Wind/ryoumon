\hakosyokika
\item
    \begin{mawarikomi}(20pt,0){150pt}{\unitlength5mm\small
\Drawaxisfalse
\begin{Zahyou}(0,8)(0,8)(0,8)
    \small
    \def\O{(0,0,0)}
    \def\A{(5,0,0)}
    \def\C{(0,5,0)}
    \def\D{(0,0,5)}
    \def\DD{(8,0,0)}
    \def\OX{(6,0,0)}
    \def\OY{(8,2,0)}
    \def\OZ{(8,0,2)}
    \def\W{(2.5,5,2.5)}
    \iiiAddvec\A\C\B
    \iiiAddvec\B\D\F
    \iiiAddvec\C\D\G
    \iiiAddvec\A\D\E
    \iiiNuritubusi{\F\B\C\G\F}
    \Kakutyuu{ABC}{O}{D}
    \iiiHenKo\D\E{$L$}
    \iiiHenKo\E\A{$L$}
    \iiiHenKo\A\B{$L$}
    \iiiArrowLine\DD\OX
    \iiiArrowLine\DD\OY
    \iiiArrowLine\DD\OZ
    \iiiPut\OY{$x$}
    \iiiPut\OX{$y$}
    \iiiPut\OZ{$z$}
    \iiiPut\W{W}
\end{Zahyou}}
        辺の長さ$L$の立方容器内の理想気体について考える。ある分子(質量$m$)の速度の$x$成分を$v_x$とすると,1回の弾性衝突によりこの分子が$x$軸に垂直な壁Wに与える力積は,\Hako である。この分子は時間$t$の間にWと\Hako 回衝突するから,この間にWに与える力積は\Hako である。したがって,容器内の全分子$N$個についての$v_x^2$の平均値$\overline{v_x^2}$を用いると,全分子がWに与える力は\Hako となる。
        また分子運動はどの方向についても同等であるから,$\overline{v_x^2}$は$v^2$の平均値$\overline{v^2}$で書き換えられる。このようにして圧力$P$は$\overline{v^2}$を用いて$P=$\Hako となる。一方,この理想気体の状態方程式として$P$と$T$の間には,気体定数$R$,アボガドロ定数$N_\mathrm{A}$を用いて\Hako の関係式が成り立つので,分子の運動エネルギーの平均値$\bunsuu{1}{2}m\overline{v^2}$は$T$を用いて\Hako と表せる。そして,この理想気体が単原子分子からなるとすると,内部エネルギー$U$は$T$を用いて$U=$\Hako と表せる。
    \end{mawarikomi}