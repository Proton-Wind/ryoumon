\hakosyokika
\item
    \begin{mawarikomi}{80pt}{\begin{zahyou*}[ul=5mm](0,5)(0,7)
    \drawline(0,0)(0,6)(5,6)(5,0)(0,0)    
    \def\P{(2,3)}
    \def\PU{(2,6)}
    \def\PD{(2,0)}
    \def\QD{(0,1.5)}
    \def\H{(4,4)}
    \def\E{(2.5,6)}
    \def\vvec{(0,1.2)}

    \small
    \Drawline{\PU\P}
    \En*[1]\P{0.2}
    \En\P{0.2}
    \Put\P[se]{小球}
    \Put\E{\Yasen\vvec}
    \HenKo<henkotype=parallel,
    yazirusi=b,
    henkosideb=0,
    henkosidet=1.5>\P\PD{$h$}

	\drawline(4,4)(4,1.8)
	\drawline(4,3.5)(3.5,2.5)
	\drawline(4,3.5)(4.5,2.5)
	\drawline(4,1.8)(3.5,0)
	\drawline(4,1.8)(4.5,0)
	% \drawline(10,3)(10,0)
    \En*[0]\H{0.3}
    \En\H{0.3}
\end{zahyou*}
}
        質量$m$の小球が,エレベーターの天井から糸でつるされており,床からの高さは$h$である。エレベーター(中の人を含む)の質量は$M$であり,重力加速度の大きさを$g$とする。このエレベーターを,鉛直上方へ一定の大きさの力で引き上げるときの運動について考える。上昇加速度の大きさを$a$とする。
        \begin{enumerate}
            \item エレベーターを引き上げる力$F$はいくらか。
            \item 小球をつるしている糸の張力$T$はエレベーターが静止している場合と比べて,何倍になるか。
            \item 次に,力の大きさ$F$を変えないで,小球をつるしている糸を静かに切ったところ,エレベーターの上昇加速度の大きさが$b$に変わった。$b$はいくらか。$a$,$M$,$m$,$g$を用いて答えよ。
            \item このとき,エレベーターの中の人が小球の運動を観測すると,小球に働いている力(合力)の大きさはいくらか。答えには$b$を用いてよい。
            \item 糸が切れてから,小球がエレベーターの床に達するまでの時間$t$はいくらか。答えには$b$を用いてよい。
        \end{enumerate}
    \end{mawarikomi}