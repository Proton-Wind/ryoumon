\hakosyokika
\item
    \begin{mawarikomi}{50pt}{\begin{zahyou*}[ul=6mm](-3,3)(-5,2)
    \small
    \def\Fy{-0.2*(cos(T)+T/4.5)-0.07}
	\def\Fx{-0.5*sin(T)}
	\BGurafu\Fx\Fy{$pi}{20*$pi}
    \drawline(-0.8,2)(-0.8,-3.5)(-0.7,-3.5)(-0.7,2)(-0.8,2)
    \drawline(0.8,2)(0.8,-3.5)(0.7,-3.5)(0.7,2)(0.8,2)
    \def\A{(-0.65,0)}
    \def\B{(0.65,0)}
    \def\C{(0.65,0.5)}
    \def\D{(-0.65,0.5)}
    \def\E{(-0.65,-3.5)}
    \def\F{(0.65,-3.5)}
    \def\G{(0.65,-3)}
    \def\H{(-0.65,-3)}
    \def\O{(0,0)}
    \def\OL{(-0.8,0)}
    \def\P{(0,-1)}
    \def\PU{(-0.8,1)}
    \Put\P(-28pt,0)[r]{P}
    \Nuritubusi*{\A\B\C\D\A}
    \Nuritubusi[0]{\E\F\G\H\E}
    \Nuritubusi*{\E\F\G\H\E}
    \Drawline{\A\B\C\D\A}
    \Drawline{\E\F\G\H\E}
    {\thicklines
    \drawline(-1.5,-3.5)(1.5,-3.5)
    }
    {\footnotesize \fboxsep=0pt
    \Put\A(8pt,1.5pt)[lb]{\colorbox{white}{A}}
    \Put\E(8pt,1.5pt)[lb]{\colorbox{white}{B}}
    }
    \HenKo<henkotype=parallel,
    henkoH=6ex,
    yazirusi=b,
    henkosideb=0,
    henkosidet=1.1>\O\P{$a$}
    \HenKo<henkotype=parallel,
    henkoH=2.5ex,
    yazirusi=b,
    henkosideb=0,
    henkosidet=1.1>\PU\OL{$a$}
    \Put\OL(-8mm,0)[l]{O}
    \drawline(-2.5,0)(-2.3,0)
    \put(-2.4,1.5){\yasen(0,-3.5)}
    \put(-2.55,-2.3){$x$}
\end{zahyou*}
}
        軽いばねの両端に同じ質量$m$の物体AとBを取り付け,なめらかな円筒状のガードでばねが鉛直に保たれるようにして,Bを床の上に置いたところ,ばねが自然長より$a$だけ縮んだ位置OでAは静止した。重力加速度の大きさを$g$とする。
        \begin{enumerate}
            \item ばねのばね定数はいくらか。また,床がBから受ける力の大きさはいくらか。Bに作用する力のつり合いより求めよ。
            \item AをO点よりさらに$a$だけ下のP点まで押し下げて,静かに放したところAは振動した。
                \begin{enumerate}
                    \item 振動中のAの速さの最大値はいくらか。
                    \item O点を原点とし,鉛直下向きを正とする$x$軸をとると,Aの変位$x$は放してからの時間$t$とともにどのように変わるか。$x$を$t$の関数として表せ。
                \end{enumerate}
            \item はじめAをO点より押し下げる距離を$b$にして運動させたとき,Aの振動中にBが床から離れて上方に動き出さないためには,$b$の値はどれだけ以下でなければならないか。
        \end{enumerate}
    \end{mawarikomi}