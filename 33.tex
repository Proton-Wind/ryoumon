\hakosyokika
\item
    \begin{mawarikomi}{180pt}{\begin{zahyou*}[ul=5mm](-1,11)(-1,5)
    \def\Fx{0.12*(X-5)**2}
    \def\Gx{0}
    \def\P{(0,0.35)}
    \def\vvec{(3,0)}
    \small
    \YGurafu\Fx{5}{10}
    \YNurii\Fx\Gx{5}{10}
    \Put\P{\Yasen\vvec}
    \put(1.5,0.8){$v_0$}
    \put(-0.5,0.8){$m$}
    \put(10.2,1.5){$M$}
    \put(7.5,2){台}
    \put(10.5,-0.8){床}
    \put(-0.2,-0.8){P}
    \En*\P{0.35}
    \En\P{0.35}
	\drawline(-1,0)(11,0)
	\drawline(10,3)(10,0)
\end{zahyou*}
}
        水平でなめらかな床の上に,質量$m$の小物体Pとなめらかな曲面をもつ質量$M$の台が静止していた。Pに速さ$v_0$を与え,台に向かって動かした。Pが台に達すると,Pは曲面を上り,再び床面上を動いた。曲面の左端は床になだらかにつながっており,重力加速度の大きさを$g$とする。
        \begin{enumerate}
            \item Pが台上の最高点に達したとき,
            \begin{enumerate}
                \item 台の速さはいくらか。
                \item 最高点の床面からの高さ$h$はいくらか。
            \end{enumerate}
            \item Pが再び床面上に達した後の,台の速さはいくらか。
        \end{enumerate}
    \end{mawarikomi}