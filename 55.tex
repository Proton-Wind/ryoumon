\hakosyokika
\item
    \begin{mawarikomi}{100pt}{\begin{zahyou*}[ul=6mm](-3,3)(0,7)
    \small
    \def\A{(-3,7)}
    \def\B{(-3,0)}
    \def\C{(3,0)}
    \def\D{(3,7)}
    \def\E{(2.9,7)}
    \def\F{(2.9,0.1)}
    \def\G{(-2.9,0.1)}
    \def\GU{(-2.9,7)}
    \def\H{(-2.9,3.8)}
    \def\I{(2.9,3.8)}
    \def\P{(0,6)}
    \En*\P{0.5}
    \En\P{0.5}
    \Nuritubusi{\H\I\F\G\H}
    \Drawline{\A\B\C\D\E\F\G\GU\A}
    \Drawline{\H\I}
    \Put\P(-10pt,10pt)[r]{A}
    \Put\P(10pt,0pt)[l]{$T_1$}
    \put(-2,2){水}
    \put(1.6,2.2){$T_2$}
    \put(1.6,0.5){$M$}
    % \put(-3.5,0.2){$v_0$}
    % \put(-4.3,1.7){$v$}
    % \put(4.1,-0.6){$V$}
\end{zahyou*}
}
空所\hako{ア}には「小さく」か「大きく」を,\hako{イ}と\hako{ウ}には2桁の小数値を入れよ。\\
    ~~アルミニウムの比熱が0.90\sftanni{J/(g\cdot K)}であることを確認する実験をしたい。温度$T_1=42.0$℃,質量100\sftanni{g}のアルミニウム球Aを,温度$T_2=20.0$℃,質量$M$\tanni{g}の水の中に入れ,Aと水が同じ温度になった時の$T_3$\tanni{℃}を測定する。水の質量$M$が\Hako なるほど,温度上昇$T_3-T_2$が小さくなる。\\
    ~~温度上昇$T_3-T_2$が1.0℃になるようにするためには,$M=$ \Hako $\times 10^2$\sftanni{g}としなければならない。ただし,水の比熱は$4.2$\sftanni{J/(g\cdot K)}であり,熱はAと水の間だけで移動する。続いて,Aと水全体に$9.9\times 10^3$\sftanni{J}の熱量を加えると,温度はさらに\Hako ℃上昇する。
    \end{mawarikomi}