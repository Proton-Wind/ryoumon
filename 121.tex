\hakosyokika
\item
    \begin{mawarikomi}(20pt,0pt){150pt}{
        %%% C:/vpn/vpn/KeTCindy/fig/fig121.tex 
%%% Generator=fig121.cdy 
{\unitlength=1cm%
\begin{picture}%
(5.09,5.37)(-2.59,-2.87)%
\special{pn 8}%
%
\special{pa  -925   366}\special{pa  -650   366}%
\special{fp}%
\special{pn 16}%
\special{pa  -843   421}\special{pa  -732   421}%
\special{fp}%
\special{pn 8}%
\special{pa  -787   787}\special{pa  -787   421}%
\special{fp}%
\special{pa  -787    -0}\special{pa  -787   366}%
\special{fp}%
\special{pa  -787  -283}\special{pa  -898  -474}%
\special{fp}%
\special{pa  -787    -0}\special{pa  -787  -283}%
\special{fp}%
\special{pa  -787  -787}\special{pa  -787  -504}%
\special{fp}%
{%
\color[cmyk]{0,0,0,0}%
\special{pa -772 -283}\special{pa -773 -285}\special{pa -773 -287}\special{pa -773 -289}%
\special{pa -774 -291}\special{pa -775 -292}\special{pa -776 -294}\special{pa -778 -295}%
\special{pa -779 -296}\special{pa -781 -297}\special{pa -783 -298}\special{pa -785 -298}%
\special{pa -786 -298}\special{pa -788 -298}\special{pa -790 -298}\special{pa -792 -298}%
\special{pa -794 -297}\special{pa -795 -296}\special{pa -797 -295}\special{pa -798 -294}%
\special{pa -800 -292}\special{pa -801 -291}\special{pa -801 -289}\special{pa -802 -287}%
\special{pa -802 -285}\special{pa -802 -283}\special{pa -802 -282}\special{pa -802 -280}%
\special{pa -801 -278}\special{pa -801 -276}\special{pa -800 -275}\special{pa -798 -273}%
\special{pa -797 -272}\special{pa -795 -271}\special{pa -794 -270}\special{pa -792 -269}%
\special{pa -790 -269}\special{pa -788 -269}\special{pa -786 -269}\special{pa -785 -269}%
\special{pa -783 -269}\special{pa -781 -270}\special{pa -779 -271}\special{pa -778 -272}%
\special{pa -776 -273}\special{pa -775 -275}\special{pa -774 -276}\special{pa -773 -278}%
\special{pa -773 -280}\special{pa -773 -282}\special{pa -772 -283}\special{pa -772 -283}%
\special{sh 1}\special{ip}%
}%
\special{pa  -772  -283}\special{pa  -773  -285}\special{pa  -773  -287}\special{pa  -773  -289}%
\special{pa  -774  -291}\special{pa  -775  -292}\special{pa  -776  -294}\special{pa  -778  -295}%
\special{pa  -779  -296}\special{pa  -781  -297}\special{pa  -783  -298}\special{pa  -785  -298}%
\special{pa  -786  -298}\special{pa  -788  -298}\special{pa  -790  -298}\special{pa  -792  -298}%
\special{pa  -794  -297}\special{pa  -795  -296}\special{pa  -797  -295}\special{pa  -798  -294}%
\special{pa  -800  -292}\special{pa  -801  -291}\special{pa  -801  -289}\special{pa  -802  -287}%
\special{pa  -802  -285}\special{pa  -802  -283}\special{pa  -802  -282}\special{pa  -802  -280}%
\special{pa  -801  -278}\special{pa  -801  -276}\special{pa  -800  -275}\special{pa  -798  -273}%
\special{pa  -797  -272}\special{pa  -795  -271}\special{pa  -794  -270}\special{pa  -792  -269}%
\special{pa  -790  -269}\special{pa  -788  -269}\special{pa  -786  -269}\special{pa  -785  -269}%
\special{pa  -783  -269}\special{pa  -781  -270}\special{pa  -779  -271}\special{pa  -778  -272}%
\special{pa  -776  -273}\special{pa  -775  -275}\special{pa  -774  -276}\special{pa  -773  -278}%
\special{pa  -773  -280}\special{pa  -773  -282}\special{pa  -772  -283}%
\special{fp}%
{%
\color[cmyk]{0,0,0,0}%
\special{pa -772 -504}\special{pa -773 -506}\special{pa -773 -508}\special{pa -773 -509}%
\special{pa -774 -511}\special{pa -775 -513}\special{pa -776 -514}\special{pa -778 -515}%
\special{pa -779 -517}\special{pa -781 -517}\special{pa -783 -518}\special{pa -785 -519}%
\special{pa -786 -519}\special{pa -788 -519}\special{pa -790 -519}\special{pa -792 -518}%
\special{pa -794 -517}\special{pa -795 -517}\special{pa -797 -515}\special{pa -798 -514}%
\special{pa -800 -513}\special{pa -801 -511}\special{pa -801 -509}\special{pa -802 -508}%
\special{pa -802 -506}\special{pa -802 -504}\special{pa -802 -502}\special{pa -802 -500}%
\special{pa -801 -498}\special{pa -801 -497}\special{pa -800 -495}\special{pa -798 -494}%
\special{pa -797 -492}\special{pa -795 -491}\special{pa -794 -490}\special{pa -792 -490}%
\special{pa -790 -489}\special{pa -788 -489}\special{pa -786 -489}\special{pa -785 -489}%
\special{pa -783 -490}\special{pa -781 -490}\special{pa -779 -491}\special{pa -778 -492}%
\special{pa -776 -494}\special{pa -775 -495}\special{pa -774 -497}\special{pa -773 -498}%
\special{pa -773 -500}\special{pa -773 -502}\special{pa -772 -504}\special{pa -772 -504}%
\special{sh 1}\special{ip}%
}%
\special{pa  -772  -504}\special{pa  -773  -506}\special{pa  -773  -508}\special{pa  -773  -509}%
\special{pa  -774  -511}\special{pa  -775  -513}\special{pa  -776  -514}\special{pa  -778  -515}%
\special{pa  -779  -517}\special{pa  -781  -517}\special{pa  -783  -518}\special{pa  -785  -519}%
\special{pa  -786  -519}\special{pa  -788  -519}\special{pa  -790  -519}\special{pa  -792  -518}%
\special{pa  -794  -517}\special{pa  -795  -517}\special{pa  -797  -515}\special{pa  -798  -514}%
\special{pa  -800  -513}\special{pa  -801  -511}\special{pa  -801  -509}\special{pa  -802  -508}%
\special{pa  -802  -506}\special{pa  -802  -504}\special{pa  -802  -502}\special{pa  -802  -500}%
\special{pa  -801  -498}\special{pa  -801  -497}\special{pa  -800  -495}\special{pa  -798  -494}%
\special{pa  -797  -492}\special{pa  -795  -491}\special{pa  -794  -490}\special{pa  -792  -490}%
\special{pa  -790  -489}\special{pa  -788  -489}\special{pa  -786  -489}\special{pa  -785  -489}%
\special{pa  -783  -490}\special{pa  -781  -490}\special{pa  -779  -491}\special{pa  -778  -492}%
\special{pa  -776  -494}\special{pa  -775  -495}\special{pa  -774  -497}\special{pa  -773  -498}%
\special{pa  -773  -500}\special{pa  -773  -502}\special{pa  -772  -504}%
\special{fp}%
\special{pa   283    -0}\special{pa   474  -110}%
\special{fp}%
\special{pa     0    -0}\special{pa   283    -0}%
\special{fp}%
\special{pa   787    -0}\special{pa   504    -0}%
\special{fp}%
{%
\color[cmyk]{0,0,0,0}%
\special{pa 298 0}\special{pa 298 -2}\special{pa 298 -4}\special{pa 297 -6}\special{pa 297 -7}%
\special{pa 296 -9}\special{pa 294 -10}\special{pa 293 -12}\special{pa 291 -13}\special{pa 290 -14}%
\special{pa 288 -14}\special{pa 286 -15}\special{pa 284 -15}\special{pa 283 -15}\special{pa 281 -15}%
\special{pa 279 -14}\special{pa 277 -14}\special{pa 275 -13}\special{pa 274 -12}\special{pa 273 -10}%
\special{pa 271 -9}\special{pa 270 -7}\special{pa 270 -6}\special{pa 269 -4}\special{pa 269 -2}%
\special{pa 269 0}\special{pa 269 2}\special{pa 269 4}\special{pa 270 6}\special{pa 270 7}%
\special{pa 271 9}\special{pa 273 10}\special{pa 274 12}\special{pa 275 13}\special{pa 277 14}%
\special{pa 279 14}\special{pa 281 15}\special{pa 283 15}\special{pa 284 15}\special{pa 286 15}%
\special{pa 288 14}\special{pa 290 14}\special{pa 291 13}\special{pa 293 12}\special{pa 294 10}%
\special{pa 296 9}\special{pa 297 7}\special{pa 297 6}\special{pa 298 4}\special{pa 298 2}%
\special{pa 298 0}\special{pa 298 0}\special{sh 1}\special{ip}%
}%
\special{pa   298    -0}\special{pa   298    -2}\special{pa   298    -4}\special{pa   297    -6}%
\special{pa   297    -7}\special{pa   296    -9}\special{pa   294   -10}\special{pa   293   -12}%
\special{pa   291   -13}\special{pa   290   -14}\special{pa   288   -14}\special{pa   286   -15}%
\special{pa   284   -15}\special{pa   283   -15}\special{pa   281   -15}\special{pa   279   -14}%
\special{pa   277   -14}\special{pa   275   -13}\special{pa   274   -12}\special{pa   273   -10}%
\special{pa   271    -9}\special{pa   270    -7}\special{pa   270    -6}\special{pa   269    -4}%
\special{pa   269    -2}\special{pa   269     0}\special{pa   269     2}\special{pa   269     4}%
\special{pa   270     6}\special{pa   270     7}\special{pa   271     9}\special{pa   273    10}%
\special{pa   274    12}\special{pa   275    13}\special{pa   277    14}\special{pa   279    14}%
\special{pa   281    15}\special{pa   283    15}\special{pa   284    15}\special{pa   286    15}%
\special{pa   288    14}\special{pa   290    14}\special{pa   291    13}\special{pa   293    12}%
\special{pa   294    10}\special{pa   296     9}\special{pa   297     7}\special{pa   297     6}%
\special{pa   298     4}\special{pa   298     2}\special{pa   298     0}%
\special{fp}%
{%
\color[cmyk]{0,0,0,0}%
\special{pa 519 0}\special{pa 519 -2}\special{pa 518 -4}\special{pa 518 -6}\special{pa 517 -7}%
\special{pa 516 -9}\special{pa 515 -10}\special{pa 513 -12}\special{pa 512 -13}\special{pa 510 -14}%
\special{pa 509 -14}\special{pa 507 -15}\special{pa 505 -15}\special{pa 503 -15}\special{pa 501 -15}%
\special{pa 499 -14}\special{pa 498 -14}\special{pa 496 -13}\special{pa 494 -12}\special{pa 493 -10}%
\special{pa 492 -9}\special{pa 491 -7}\special{pa 490 -6}\special{pa 489 -4}\special{pa 489 -2}%
\special{pa 489 0}\special{pa 489 2}\special{pa 489 4}\special{pa 490 6}\special{pa 491 7}%
\special{pa 492 9}\special{pa 493 10}\special{pa 494 12}\special{pa 496 13}\special{pa 498 14}%
\special{pa 499 14}\special{pa 501 15}\special{pa 503 15}\special{pa 505 15}\special{pa 507 15}%
\special{pa 509 14}\special{pa 510 14}\special{pa 512 13}\special{pa 513 12}\special{pa 515 10}%
\special{pa 516 9}\special{pa 517 7}\special{pa 518 6}\special{pa 518 4}\special{pa 519 2}%
\special{pa 519 0}\special{pa 519 0}\special{sh 1}\special{ip}%
}%
\special{pa   519    -0}\special{pa   519    -2}\special{pa   518    -4}\special{pa   518    -6}%
\special{pa   517    -7}\special{pa   516    -9}\special{pa   515   -10}\special{pa   513   -12}%
\special{pa   512   -13}\special{pa   510   -14}\special{pa   509   -14}\special{pa   507   -15}%
\special{pa   505   -15}\special{pa   503   -15}\special{pa   501   -15}\special{pa   499   -14}%
\special{pa   498   -14}\special{pa   496   -13}\special{pa   494   -12}\special{pa   493   -10}%
\special{pa   492    -9}\special{pa   491    -7}\special{pa   490    -6}\special{pa   489    -4}%
\special{pa   489    -2}\special{pa   489     0}\special{pa   489     2}\special{pa   489     4}%
\special{pa   490     6}\special{pa   491     7}\special{pa   492     9}\special{pa   493    10}%
\special{pa   494    12}\special{pa   496    13}\special{pa   498    14}\special{pa   499    14}%
\special{pa   501    15}\special{pa   503    15}\special{pa   505    15}\special{pa   507    15}%
\special{pa   509    14}\special{pa   510    14}\special{pa   512    13}\special{pa   513    12}%
\special{pa   515    10}\special{pa   516     9}\special{pa   517     7}\special{pa   518     6}%
\special{pa   518     4}\special{pa   519     2}\special{pa   519     0}%
\special{fp}%
\special{pa    69  -256}\special{pa    69  -531}\special{pa   -69  -531}\special{pa   -69  -256}%
\special{pa    69  -256}%
\special{fp}%
\special{pa     0  -787}\special{pa     0  -531}%
\special{fp}%
\special{pa     0    -0}\special{pa     0  -256}%
\special{fp}%
\special{pa    69   531}\special{pa    69   256}\special{pa   -69   256}\special{pa   -69   531}%
\special{pa    69   531}%
\special{fp}%
\special{pa     0    -0}\special{pa     0   256}%
\special{fp}%
\special{pa     0   787}\special{pa     0   531}%
\special{fp}%
\special{pa   925  -352}\special{pa   650  -352}%
\special{fp}%
\special{pa   925  -435}\special{pa   650  -435}%
\special{fp}%
\special{pa   787  -787}\special{pa   787  -435}%
\special{fp}%
\special{pa   787    -0}\special{pa   787  -352}%
\special{fp}%
\special{pa   925   435}\special{pa   650   435}%
\special{fp}%
\special{pa   925   352}\special{pa   650   352}%
\special{fp}%
\special{pa   787    -0}\special{pa   787   352}%
\special{fp}%
\special{pa   787   787}\special{pa   787   435}%
\special{fp}%
\special{pa  -787   787}\special{pa   787   787}%
\special{fp}%
\special{pa  -787  -787}\special{pa   787  -787}%
\special{fp}%
\settowidth{\Width}{$\mathrm{E}$}\setlength{\Width}{-0.5\Width}%
\settoheight{\Height}{$\mathrm{E}$}\settodepth{\Depth}{$\mathrm{E}$}\setlength{\Height}{-0.5\Height}\setlength{\Depth}{0.5\Depth}\addtolength{\Height}{\Depth}%
\put( -2.490, -1.000){\hspace*{\Width}\raisebox{\Height}{$\mathrm{E}$}}%
%
\settowidth{\Width}{$\mathrm{S_1}$}\setlength{\Width}{-1\Width}%
\settoheight{\Height}{$\mathrm{S_1}$}\settodepth{\Depth}{$\mathrm{S_1}$}\setlength{\Height}{-\Height}%
\put( -2.250,  0.850){\hspace*{\Width}\raisebox{\Height}{$\mathrm{S_1}$}}%
%
\settowidth{\Width}{$\mathrm{S_2}$}\setlength{\Width}{-1\Width}%
\settoheight{\Height}{$\mathrm{S_2}$}\settodepth{\Depth}{$\mathrm{S_2}$}\setlength{\Height}{\Depth}%
\put(  0.750,  0.250){\hspace*{\Width}\raisebox{\Height}{$\mathrm{S_2}$}}%
%
\settowidth{\Width}{$\mathrm{R_1}$}\setlength{\Width}{-1\Width}%
\settoheight{\Height}{$\mathrm{R_1}$}\settodepth{\Depth}{$\mathrm{R_1}$}\setlength{\Height}{-0.5\Height}\setlength{\Depth}{0.5\Depth}\addtolength{\Height}{\Depth}%
\put( -0.300,  1.000){\hspace*{\Width}\raisebox{\Height}{$\mathrm{R_1}$}}%
%
\settowidth{\Width}{$\mathrm{R_2}$}\setlength{\Width}{-1\Width}%
\settoheight{\Height}{$\mathrm{R_2}$}\settodepth{\Depth}{$\mathrm{R_2}$}\setlength{\Height}{-0.5\Height}\setlength{\Depth}{0.5\Depth}\addtolength{\Height}{\Depth}%
\put( -0.300, -1.000){\hspace*{\Width}\raisebox{\Height}{$\mathrm{R_2}$}}%
%
\settowidth{\Width}{$\mathrm{C_1}$}\setlength{\Width}{-1\Width}%
\settoheight{\Height}{$\mathrm{C_1}$}\settodepth{\Depth}{$\mathrm{C_1}$}\setlength{\Height}{-\Height}%
\put(  1.540,  0.950){\hspace*{\Width}\raisebox{\Height}{$\mathrm{C_1}$}}%
%
\settowidth{\Width}{$\mathrm{C_2}$}\setlength{\Width}{-1\Width}%
\settoheight{\Height}{$\mathrm{C_2}$}\settodepth{\Depth}{$\mathrm{C_2}$}\setlength{\Height}{-\Height}%
\put(  1.540, -1.050){\hspace*{\Width}\raisebox{\Height}{$\mathrm{C_2}$}}%
%
\settowidth{\Width}{A}\setlength{\Width}{0\Width}%
\settoheight{\Height}{A}\settodepth{\Depth}{A}\setlength{\Height}{-0.5\Height}\setlength{\Depth}{0.5\Depth}\addtolength{\Height}{\Depth}%
\put(  2.150,  0.000){\hspace*{\Width}\raisebox{\Height}{A}}%
%
\settowidth{\Width}{B}\setlength{\Width}{-1\Width}%
\settoheight{\Height}{B}\settodepth{\Depth}{B}\setlength{\Height}{-0.5\Height}\setlength{\Depth}{0.5\Depth}\addtolength{\Height}{\Depth}%
\put( -0.150,  0.000){\hspace*{\Width}\raisebox{\Height}{B}}%
%
\special{pa 15 -787}\special{pa 15 -789}\special{pa 14 -791}\special{pa 14 -793}\special{pa 13 -795}%
\special{pa 12 -796}\special{pa 11 -798}\special{pa 10 -799}\special{pa 8 -800}\special{pa 6 -801}%
\special{pa 5 -802}\special{pa 3 -802}\special{pa 1 -802}\special{pa -1 -802}\special{pa -3 -802}%
\special{pa -5 -802}\special{pa -6 -801}\special{pa -8 -800}\special{pa -10 -799}%
\special{pa -11 -798}\special{pa -12 -796}\special{pa -13 -795}\special{pa -14 -793}%
\special{pa -14 -791}\special{pa -15 -789}\special{pa -15 -787}\special{pa -15 -786}%
\special{pa -14 -784}\special{pa -14 -782}\special{pa -13 -780}\special{pa -12 -779}%
\special{pa -11 -777}\special{pa -10 -776}\special{pa -8 -775}\special{pa -6 -774}%
\special{pa -5 -773}\special{pa -3 -773}\special{pa -1 -772}\special{pa 1 -772}\special{pa 3 -773}%
\special{pa 5 -773}\special{pa 6 -774}\special{pa 8 -775}\special{pa 10 -776}\special{pa 11 -777}%
\special{pa 12 -779}\special{pa 13 -780}\special{pa 14 -782}\special{pa 14 -784}\special{pa 15 -786}%
\special{pa 15 -787}\special{pa 15 -787}\special{sh 1}\special{ip}%
\special{pa    15  -787}\special{pa    15  -789}\special{pa    14  -791}\special{pa    14  -793}%
\special{pa    13  -795}\special{pa    12  -796}\special{pa    11  -798}\special{pa    10  -799}%
\special{pa     8  -800}\special{pa     6  -801}\special{pa     5  -802}\special{pa     3  -802}%
\special{pa     1  -802}\special{pa    -1  -802}\special{pa    -3  -802}\special{pa    -5  -802}%
\special{pa    -6  -801}\special{pa    -8  -800}\special{pa   -10  -799}\special{pa   -11  -798}%
\special{pa   -12  -796}\special{pa   -13  -795}\special{pa   -14  -793}\special{pa   -14  -791}%
\special{pa   -15  -789}\special{pa   -15  -787}\special{pa   -15  -786}\special{pa   -14  -784}%
\special{pa   -14  -782}\special{pa   -13  -780}\special{pa   -12  -779}\special{pa   -11  -777}%
\special{pa   -10  -776}\special{pa    -8  -775}\special{pa    -6  -774}\special{pa    -5  -773}%
\special{pa    -3  -773}\special{pa    -1  -772}\special{pa     1  -772}\special{pa     3  -773}%
\special{pa     5  -773}\special{pa     6  -774}\special{pa     8  -775}\special{pa    10  -776}%
\special{pa    11  -777}\special{pa    12  -779}\special{pa    13  -780}\special{pa    14  -782}%
\special{pa    14  -784}\special{pa    15  -786}\special{pa    15  -787}%
\special{fp}%
\special{pa 15 787}\special{pa 15 786}\special{pa 14 784}\special{pa 14 782}\special{pa 13 780}%
\special{pa 12 779}\special{pa 11 777}\special{pa 10 776}\special{pa 8 775}\special{pa 6 774}%
\special{pa 5 773}\special{pa 3 773}\special{pa 1 772}\special{pa -1 772}\special{pa -3 773}%
\special{pa -5 773}\special{pa -6 774}\special{pa -8 775}\special{pa -10 776}\special{pa -11 777}%
\special{pa -12 779}\special{pa -13 780}\special{pa -14 782}\special{pa -14 784}\special{pa -15 786}%
\special{pa -15 787}\special{pa -15 789}\special{pa -14 791}\special{pa -14 793}\special{pa -13 795}%
\special{pa -12 796}\special{pa -11 798}\special{pa -10 799}\special{pa -8 800}\special{pa -6 801}%
\special{pa -5 802}\special{pa -3 802}\special{pa -1 802}\special{pa 1 802}\special{pa 3 802}%
\special{pa 5 802}\special{pa 6 801}\special{pa 8 800}\special{pa 10 799}\special{pa 11 798}%
\special{pa 12 796}\special{pa 13 795}\special{pa 14 793}\special{pa 14 791}\special{pa 15 789}%
\special{pa 15 787}\special{pa 15 787}\special{sh 1}\special{ip}%
\special{pa    15   787}\special{pa    15   786}\special{pa    14   784}\special{pa    14   782}%
\special{pa    13   780}\special{pa    12   779}\special{pa    11   777}\special{pa    10   776}%
\special{pa     8   775}\special{pa     6   774}\special{pa     5   773}\special{pa     3   773}%
\special{pa     1   772}\special{pa    -1   772}\special{pa    -3   773}\special{pa    -5   773}%
\special{pa    -6   774}\special{pa    -8   775}\special{pa   -10   776}\special{pa   -11   777}%
\special{pa   -12   779}\special{pa   -13   780}\special{pa   -14   782}\special{pa   -14   784}%
\special{pa   -15   786}\special{pa   -15   787}\special{pa   -15   789}\special{pa   -14   791}%
\special{pa   -14   793}\special{pa   -13   795}\special{pa   -12   796}\special{pa   -11   798}%
\special{pa   -10   799}\special{pa    -8   800}\special{pa    -6   801}\special{pa    -5   802}%
\special{pa    -3   802}\special{pa    -1   802}\special{pa     1   802}\special{pa     3   802}%
\special{pa     5   802}\special{pa     6   801}\special{pa     8   800}\special{pa    10   799}%
\special{pa    11   798}\special{pa    12   796}\special{pa    13   795}\special{pa    14   793}%
\special{pa    14   791}\special{pa    15   789}\special{pa    15   787}%
\special{fp}%
\special{pa 15 0}\special{pa 15 -2}\special{pa 14 -4}\special{pa 14 -6}\special{pa 13 -7}%
\special{pa 12 -9}\special{pa 11 -10}\special{pa 10 -12}\special{pa 8 -13}\special{pa 6 -14}%
\special{pa 5 -14}\special{pa 3 -15}\special{pa 1 -15}\special{pa -1 -15}\special{pa -3 -15}%
\special{pa -5 -14}\special{pa -6 -14}\special{pa -8 -13}\special{pa -10 -12}\special{pa -11 -10}%
\special{pa -12 -9}\special{pa -13 -7}\special{pa -14 -6}\special{pa -14 -4}\special{pa -15 -2}%
\special{pa -15 0}\special{pa -15 2}\special{pa -14 4}\special{pa -14 6}\special{pa -13 7}%
\special{pa -12 9}\special{pa -11 10}\special{pa -10 12}\special{pa -8 13}\special{pa -6 14}%
\special{pa -5 14}\special{pa -3 15}\special{pa -1 15}\special{pa 1 15}\special{pa 3 15}%
\special{pa 5 14}\special{pa 6 14}\special{pa 8 13}\special{pa 10 12}\special{pa 11 10}%
\special{pa 12 9}\special{pa 13 7}\special{pa 14 6}\special{pa 14 4}\special{pa 15 2}%
\special{pa 15 0}\special{pa 15 0}\special{sh 1}\special{ip}%
\special{pa    15    -0}\special{pa    15    -2}\special{pa    14    -4}\special{pa    14    -6}%
\special{pa    13    -7}\special{pa    12    -9}\special{pa    11   -10}\special{pa    10   -12}%
\special{pa     8   -13}\special{pa     6   -14}\special{pa     5   -14}\special{pa     3   -15}%
\special{pa     1   -15}\special{pa    -1   -15}\special{pa    -3   -15}\special{pa    -5   -14}%
\special{pa    -6   -14}\special{pa    -8   -13}\special{pa   -10   -12}\special{pa   -11   -10}%
\special{pa   -12    -9}\special{pa   -13    -7}\special{pa   -14    -6}\special{pa   -14    -4}%
\special{pa   -15    -2}\special{pa   -15     0}\special{pa   -15     2}\special{pa   -14     4}%
\special{pa   -14     6}\special{pa   -13     7}\special{pa   -12     9}\special{pa   -11    10}%
\special{pa   -10    12}\special{pa    -8    13}\special{pa    -6    14}\special{pa    -5    14}%
\special{pa    -3    15}\special{pa    -1    15}\special{pa     1    15}\special{pa     3    15}%
\special{pa     5    14}\special{pa     6    14}\special{pa     8    13}\special{pa    10    12}%
\special{pa    11    10}\special{pa    12     9}\special{pa    13     7}\special{pa    14     6}%
\special{pa    14     4}\special{pa    15     2}\special{pa    15     0}%
\special{fp}%
\special{pa 802 0}\special{pa 802 -2}\special{pa 802 -4}\special{pa 801 -6}\special{pa 801 -7}%
\special{pa 800 -9}\special{pa 798 -10}\special{pa 797 -12}\special{pa 795 -13}\special{pa 794 -14}%
\special{pa 792 -14}\special{pa 790 -15}\special{pa 788 -15}\special{pa 786 -15}\special{pa 785 -15}%
\special{pa 783 -14}\special{pa 781 -14}\special{pa 779 -13}\special{pa 778 -12}\special{pa 776 -10}%
\special{pa 775 -9}\special{pa 774 -7}\special{pa 773 -6}\special{pa 773 -4}\special{pa 773 -2}%
\special{pa 772 0}\special{pa 773 2}\special{pa 773 4}\special{pa 773 6}\special{pa 774 7}%
\special{pa 775 9}\special{pa 776 10}\special{pa 778 12}\special{pa 779 13}\special{pa 781 14}%
\special{pa 783 14}\special{pa 785 15}\special{pa 786 15}\special{pa 788 15}\special{pa 790 15}%
\special{pa 792 14}\special{pa 794 14}\special{pa 795 13}\special{pa 797 12}\special{pa 798 10}%
\special{pa 800 9}\special{pa 801 7}\special{pa 801 6}\special{pa 802 4}\special{pa 802 2}%
\special{pa 802 0}\special{pa 802 0}\special{sh 1}\special{ip}%
\special{pa   802    -0}\special{pa   802    -2}\special{pa   802    -4}\special{pa   801    -6}%
\special{pa   801    -7}\special{pa   800    -9}\special{pa   798   -10}\special{pa   797   -12}%
\special{pa   795   -13}\special{pa   794   -14}\special{pa   792   -14}\special{pa   790   -15}%
\special{pa   788   -15}\special{pa   786   -15}\special{pa   785   -15}\special{pa   783   -14}%
\special{pa   781   -14}\special{pa   779   -13}\special{pa   778   -12}\special{pa   776   -10}%
\special{pa   775    -9}\special{pa   774    -7}\special{pa   773    -6}\special{pa   773    -4}%
\special{pa   773    -2}\special{pa   772     0}\special{pa   773     2}\special{pa   773     4}%
\special{pa   773     6}\special{pa   774     7}\special{pa   775     9}\special{pa   776    10}%
\special{pa   778    12}\special{pa   779    13}\special{pa   781    14}\special{pa   783    14}%
\special{pa   785    15}\special{pa   786    15}\special{pa   788    15}\special{pa   790    15}%
\special{pa   792    14}\special{pa   794    14}\special{pa   795    13}\special{pa   797    12}%
\special{pa   798    10}\special{pa   800     9}\special{pa   801     7}\special{pa   801     6}%
\special{pa   802     4}\special{pa   802     2}\special{pa   802     0}%
\special{fp}%
\special{pa 15 787}\special{pa 15 786}\special{pa 14 784}\special{pa 14 782}\special{pa 13 780}%
\special{pa 12 779}\special{pa 11 777}\special{pa 10 776}\special{pa 8 775}\special{pa 6 774}%
\special{pa 5 773}\special{pa 3 773}\special{pa 1 772}\special{pa -1 772}\special{pa -3 773}%
\special{pa -5 773}\special{pa -6 774}\special{pa -8 775}\special{pa -10 776}\special{pa -11 777}%
\special{pa -12 779}\special{pa -13 780}\special{pa -14 782}\special{pa -14 784}\special{pa -15 786}%
\special{pa -15 787}\special{pa -15 789}\special{pa -14 791}\special{pa -14 793}\special{pa -13 795}%
\special{pa -12 796}\special{pa -11 798}\special{pa -10 799}\special{pa -8 800}\special{pa -6 801}%
\special{pa -5 802}\special{pa -3 802}\special{pa -1 802}\special{pa 1 802}\special{pa 3 802}%
\special{pa 5 802}\special{pa 6 801}\special{pa 8 800}\special{pa 10 799}\special{pa 11 798}%
\special{pa 12 796}\special{pa 13 795}\special{pa 14 793}\special{pa 14 791}\special{pa 15 789}%
\special{pa 15 787}\special{pa 15 787}\special{sh 1}\special{ip}%
\special{pa    15   787}\special{pa    15   786}\special{pa    14   784}\special{pa    14   782}%
\special{pa    13   780}\special{pa    12   779}\special{pa    11   777}\special{pa    10   776}%
\special{pa     8   775}\special{pa     6   774}\special{pa     5   773}\special{pa     3   773}%
\special{pa     1   772}\special{pa    -1   772}\special{pa    -3   773}\special{pa    -5   773}%
\special{pa    -6   774}\special{pa    -8   775}\special{pa   -10   776}\special{pa   -11   777}%
\special{pa   -12   779}\special{pa   -13   780}\special{pa   -14   782}\special{pa   -14   784}%
\special{pa   -15   786}\special{pa   -15   787}\special{pa   -15   789}\special{pa   -14   791}%
\special{pa   -14   793}\special{pa   -13   795}\special{pa   -12   796}\special{pa   -11   798}%
\special{pa   -10   799}\special{pa    -8   800}\special{pa    -6   801}\special{pa    -5   802}%
\special{pa    -3   802}\special{pa    -1   802}\special{pa     1   802}\special{pa     3   802}%
\special{pa     5   802}\special{pa     6   801}\special{pa     8   800}\special{pa    10   799}%
\special{pa    11   798}\special{pa    12   796}\special{pa    13   795}\special{pa    14   793}%
\special{pa    14   791}\special{pa    15   789}\special{pa    15   787}%
\special{fp}%
\special{pa     0   787}\special{pa     0   984}%
\special{fp}%
\special{pa   -98   984}\special{pa    98   984}%
\special{fp}%
\special{pa   -59  1024}\special{pa    59  1024}%
\special{fp}%
\special{pa   -20  1063}\special{pa    20  1063}%
\special{fp}%
\end{picture}}%
    }
    内部抵抗が無視できる直流電源E,電気抵抗$\mathrm{R_1}$,$\mathrm{R_2}$の抵抗,コンデンサー$\mathrm{C_1}$,$\mathrm{C_2}$,およびスイッチ$\mathrm{S_1}$,$\mathrm{S_2}$からなる回路がある。Eの起電力は100\sftanni{V},$\mathrm{R_1}$,$\mathrm{R_2}$の抵抗値はそれぞれ,$20$\sftanni{\Omega },$30$\sftanni{\Omega }であり,$\mathrm{C_1}$,$\mathrm{C_2}$の電気容量は$20$\sftanni{\mu F},$30$\sftanni{\mu F}である。はじめ$\mathrm{S_1}$,$\mathrm{S_2}$は共に開いていて,$\mathrm{C_1}$,$\mathrm{C_2}$には電荷は蓄えられていないものとする。
        \begin{enumerate}
            \item $\mathrm{S_1}$を閉じて十分時間が経過した後に,
                \begin{enumerate}[ア.]
                    \item Eを流れる電流は何\sftanni{A}か。
                    \item $\mathrm{C_1}$に蓄えられる電荷は何\sftanni{\mu C}か。
                \end{enumerate}
            \item $\mathrm{S_1}$を閉じたまま$\mathrm{S_2}$を閉じる。$\mathrm{S_2}$を閉じてから十分時間が経過するまでに$\mathrm{S_2}$を通過する正電荷は何\sftanni{\mu C}か。また,どちら向きに通過するか。
            \item $\mathrm{S_2}$を開き,続いて$\mathrm{S_1}$を開いてから十分長い時間が経過した後には,
                \begin{enumerate}[ア.]
                    \item 点Aの電位は何\sftanni{V}か。アース点の電位を$0$\sftanni{V}とする。
                    \item $\mathrm{C_1}$および$\mathrm{C_2}$に蓄えられている電荷はそれぞれ何\sftanni{\mu C}か。
                    \item $\mathrm{S_1}$を開いたのち,抵抗で生じたジュール熱は何\sftanni{J}か。
                \end{enumerate}
        \end{enumerate}
    \end{mawarikomi}