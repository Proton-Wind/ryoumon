\hakosyokika
\item
    \begin{mawarikomi}(20pt,0){150pt}{{\unitlength6mm\small
\begin{zahyou*}[ul=5mm](0,9)(0,5)
    \def\A{(2,2)}
    \def\B{(7.5,2)}
    \def\C{(4.25,2.8)}
    \def\D{(5.25,2.8)}
    \def\E{(4.75,2.8)}
    \def\F{(4.75,1.5)}
    \def\G{(3.95,1.8)}
    \def\H{(5.55,1.8)}
    \def\I{(5.55,2.2)}
    \def\J{(3.95,2.2)}
    \def\K{(5,3)}
    \En*[0]\A{2}
    \En*[0]\B{2}
    \Nuritubusi[0]{\G\H\I\J\G}
    \Drawlines{\I\J;\G\H}
    {\thicklines
        \Drawlines{\C\D;\E\F}
    }
    \Put\A(-20pt,-5pt)[b]{A}
    \Put\B(20pt,-5pt)[b]{B}
    \Put\A(0,-15pt)[b]{$p$}
    \Put\B(0,-15pt)[b]{$3p$}
    \Put\A(0,10pt)[b]{1モル}
    \Put\B(0,10pt)[b]{2モル}
    \Put\K(-3pt,0pt)[b]{K}
\end{zahyou*}}
}
        それぞれの容器が$V$\tanni{m^3}の2つの容器A,BがコックKを取り付けた細い管で結ばれている。はじめ,コックKは閉じられており,それぞれ1モルと2モルの単原子分子理想気体が入っている。A,Bの気体の圧力はそれぞれ,$p$\tanni{Pa},$3p$\tanni{Pa}であった。気体定数を$R$\tanni{J/(mol\cdot K)}として,\karaHako に適する数値を答えよ。
        \begin{enumerate}
            \item Aの気体の温度を$T_\mathrm{A}=T$\tanni{K}とすると,Bの気体の温度は$T_\mathrm{B}=\Hako \times T$\tanni{K}である。またA,Bの気体の内部エネルギーの和は,$\Hako \times RT$\tanni{J}である。
            \item 容器の壁を通して熱の出入りがないようにして,コックKを開いてA,Bの気体を混合した。このときの混合気体の温度は$\Hako \times T$\tanni{K}となり,圧力は$\Hako \times p$\tanni{Pa}となる。
            \item こののち,Kを開いたまま,A,Bの気体の温度を調整し,はじめの温度$T_\mathrm{A}$,$T_\mathrm{B}$にもどした。このとき,A内の気体の量は\Hako \tanni{mol}となり,圧力は$\Hako \times p$\tanni{Pa}となる。
        \end{enumerate}
    \end{mawarikomi}