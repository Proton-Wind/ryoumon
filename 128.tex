\hakosyokika
\item
    \begin{mawarikomi}(10pt,0pt){120pt}{
        \input{./fig/fig128.tex}
    }
    鉛直上向きで磁束密度$B$の一様な磁場中に,十分に長い2本の導線ab,cdが水平に間隔$\ell$だけ隔てて置かれている。ac間には,抵抗値$R$の抵抗と起電力$E$の電池が接続されている。導線ab,cd間には,これらと垂直になるように質量$M$の導線Lが置かれている。はじめ,Lは固定されており,Lと導線ab,cd間の静止摩擦係数を$\mu $,動摩擦係数を$\mu '$とする。重力加速度の大きさは$g$である。$R$以外の抵抗や回路を流れる電流がつくる磁場は無視できるとする。
        \begin{enumerate}
            \item Lが固定されているとき,Lが磁場から受ける力の大きさと向きを答えよ。
            \item 固定が外されたとき,Lは滑り出した。$\mu $が満たす条件を求めよ。
            \item Lが滑り出したのち,十分時間が経過すると,Lの速度は一定になる。このときの電流の大きさ$I_0$と,速さ$v_0$を求めよ。
            \item Lが一定速度で運動しているときに,摩擦によって単位時間当たりに発生する熱量$Q$を求めよ。
            \item Lが一定速度で運動しているとき,電池のエネルギーはどのように消費されるか。20字程度で述べよ。
        \end{enumerate}
    \end{mawarikomi}