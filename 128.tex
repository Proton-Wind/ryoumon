\hakosyokika
\item
    \begin{mawarikomi}(10pt,0pt){120pt}{
        %WinTpicVersion4.32a
{\unitlength 0.1in%
\begin{picture}(13.1800,12.4300)(4.0000,-14.5500)%
% LINE 2 0 3 0 Black White  
% 2 1700 400 500 400
% 
\special{pn 8}%
\special{pa 1700 400}%
\special{pa 500 400}%
\special{fp}%
% LINE 2 0 3 0 Black White  
% 2 500 1400 1700 1400
% 
\special{pn 8}%
\special{pa 500 1400}%
\special{pa 1700 1400}%
\special{fp}%
% LINE 2 0 3 0 Black White  
% 2 500 1150 500 1400
% 
\special{pn 8}%
\special{pa 500 1150}%
\special{pa 500 1400}%
\special{fp}%
% LINE 2 0 3 0 Black White  
% 2 500 400 500 1100
% 
\special{pn 8}%
\special{pa 500 400}%
\special{pa 500 1100}%
\special{fp}%
% LINE 1 0 3 0 Black White  
% 2 450 1100 550 1100
% 
\special{pn 13}%
\special{pa 450 1100}%
\special{pa 550 1100}%
\special{fp}%
% LINE 2 0 3 0 Black White  
% 2 600 1150 400 1150
% 
\special{pn 8}%
\special{pa 600 1150}%
\special{pa 400 1150}%
\special{fp}%
% BOX 2 0 2 0 Black White  
% 2 450 600 550 900
% 
\special{pn 0}%
\special{sh 0}%
\special{pa 450 600}%
\special{pa 550 600}%
\special{pa 550 900}%
\special{pa 450 900}%
\special{pa 450 600}%
\special{ip}%
\special{pn 8}%
\special{pa 450 600}%
\special{pa 550 600}%
\special{pa 550 900}%
\special{pa 450 900}%
\special{pa 450 600}%
\special{pa 550 600}%
\special{fp}%
% STR 2 0 3 0 Black White  
% 4 450 305 450 355 5 0 0 0
% a
\put(4.5000,-3.5500){\makebox(0,0){a}}%
% STR 2 0 3 0 Black White  
% 4 450 1405 450 1455 5 0 0 0
% c
\put(4.5000,-14.5500){\makebox(0,0){c}}%
% STR 2 0 3 0 Black White  
% 4 1750 1405 1750 1455 5 0 0 0
% d
\put(17.5000,-14.5500){\makebox(0,0){d}}%
% STR 2 0 3 0 Black White  
% 4 1750 305 1750 355 5 0 0 0
% b
\put(17.5000,-3.5500){\makebox(0,0){b}}%
% LINE 0 0 3 0 Black White  
% 2 1250 355 1250 1455
% 
\special{pn 20}%
\special{pa 1250 355}%
\special{pa 1250 1455}%
\special{fp}%
% STR 2 0 3 0 Black White  
% 4 1250 235 1250 285 5 0 0 0
% L
\put(12.5000,-2.8500){\makebox(0,0){L}}%
% VECTOR 2 0 3 0 Black White  
% 4 1500 800 1500 400 1500 1000 1500 1400
% 
\special{pn 8}%
\special{pa 1500 800}%
\special{pa 1500 400}%
\special{fp}%
\special{sh 1}%
\special{pa 1500 400}%
\special{pa 1480 467}%
\special{pa 1500 453}%
\special{pa 1520 467}%
\special{pa 1500 400}%
\special{fp}%
\special{pa 1500 1000}%
\special{pa 1500 1400}%
\special{fp}%
\special{sh 1}%
\special{pa 1500 1400}%
\special{pa 1520 1333}%
\special{pa 1500 1347}%
\special{pa 1480 1333}%
\special{pa 1500 1400}%
\special{fp}%
% STR 2 0 3 0 Black White  
% 4 1500 850 1500 900 5 0 0 0
% $\ell$
\put(15.0000,-9.0000){\makebox(0,0){$\ell$}}%
% STR 2 0 3 0 Black White  
% 4 800 750 800 800 5 0 0 0
% $\odot$
\put(8.0000,-8.0000){\makebox(0,0){$\odot$}}%
% STR 2 0 3 0 Black White  
% 4 890 850 890 900 2 0 0 0
% $B$
\put(8.9000,-9.0000){\makebox(0,0)[lb]{$B$}}%
\end{picture}}%

    }
    鉛直上向きで磁束密度$B$の一様な磁場中に,十分に長い2本の導線ab,cdが水平に間隔$\ell$だけ隔てて置かれている。ac間には,抵抗値$R$の抵抗と起電力$E$の電池が接続されている。導線ab,cd間には,これらと垂直になるように質量$M$の導線Lが置かれている。はじめ,Lは固定されており,Lと導線ab,cd間の静止摩擦係数を$\mu $,動摩擦係数を$\mu '$とする。重力加速度の大きさは$g$である。$R$以外の抵抗や回路を流れる電流がつくる磁場は無視できるとする。
        \begin{enumerate}
            \item Lが固定されているとき,Lが磁場から受ける力の大きさと向きを答えよ。
            \item 固定が外されたとき,Lは滑り出した。$\mu $が満たす条件を求めよ。
            \item Lが滑り出したのち,十分時間が経過すると,Lの速度は一定になる。このときの電流の大きさ$I_0$と,速さ$v_0$を求めよ。
            \item Lが一定速度で運動しているときに,摩擦によって単位時間当たりに発生する熱量$Q$を求めよ。
            \item Lが一定速度で運動しているとき,電池のエネルギーはどのように消費されるか。20字程度で述べよ。
        \end{enumerate}
    \end{mawarikomi}