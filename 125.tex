\hakosyokika
\item
    \begin{mawarikomi}(20pt,0pt){80pt}{
        %WinTpicVersion4.32a
{\unitlength 0.1in%
\begin{picture}(9.2126,17.6181)(10.0787,-23.3760)%
% ELLIPSE 2 0 3 0 Black White  
% 4 1192 656 1360 712 1360 712 1360 712
% 
\special{pn 8}%
\special{ar 1173 646 165 55 0.0000000 6.2831853}%
% ELLIPSE 2 0 3 1 Black White  
% 4 1192 2448 1360 2504 1024 2448 1360 2448
% 
\special{pn 8}%
\special{ar 1173 2409 165 55 6.2831853 3.1415927}%
% ELLIPSE 2 1 3 2 Black White  
% 4 1192 2448 1360 2504 1360 2448 1024 2448
% 
\special{pn 8}%
\special{pn 8}%
\special{pa 1008 2409}%
\special{pa 1008 2406}%
\special{pa 1009 2406}%
\special{pa 1009 2403}%
\special{pa 1010 2402}%
\special{pa 1010 2400}%
\special{pa 1011 2400}%
\special{pa 1012 2397}%
\special{pa 1013 2397}%
\special{pa 1013 2396}%
\special{pa 1014 2396}%
\special{pa 1014 2395}%
\special{pa 1016 2393}%
\special{pa 1016 2392}%
\special{pa 1017 2392}%
\special{pa 1020 2389}%
\special{fp}%
\special{pa 1038 2378}%
\special{pa 1038 2377}%
\special{pa 1040 2377}%
\special{pa 1040 2376}%
\special{pa 1045 2375}%
\special{pa 1045 2374}%
\special{pa 1050 2373}%
\special{pa 1050 2372}%
\special{pa 1053 2372}%
\special{pa 1053 2371}%
\special{pa 1056 2371}%
\special{pa 1056 2370}%
\special{pa 1059 2370}%
\special{pa 1059 2370}%
\special{fp}%
\special{pa 1082 2363}%
\special{pa 1086 2363}%
\special{pa 1086 2362}%
\special{pa 1091 2362}%
\special{pa 1091 2361}%
\special{pa 1095 2361}%
\special{pa 1096 2360}%
\special{pa 1101 2360}%
\special{pa 1102 2359}%
\special{pa 1107 2359}%
\special{fp}%
\special{pa 1133 2356}%
\special{pa 1135 2356}%
\special{pa 1136 2355}%
\special{pa 1151 2355}%
\special{pa 1152 2354}%
\special{pa 1159 2354}%
\special{fp}%
\special{pa 1187 2354}%
\special{pa 1195 2354}%
\special{pa 1196 2355}%
\special{pa 1211 2355}%
\special{pa 1212 2356}%
\special{pa 1214 2356}%
\special{fp}%
\special{pa 1240 2359}%
\special{pa 1244 2359}%
\special{pa 1245 2360}%
\special{pa 1256 2361}%
\special{pa 1256 2362}%
\special{pa 1261 2362}%
\special{pa 1261 2363}%
\special{pa 1265 2363}%
\special{fp}%
\special{pa 1287 2370}%
\special{pa 1290 2370}%
\special{pa 1290 2371}%
\special{pa 1293 2371}%
\special{pa 1293 2372}%
\special{pa 1296 2372}%
\special{pa 1296 2373}%
\special{pa 1301 2374}%
\special{pa 1301 2375}%
\special{pa 1306 2376}%
\special{pa 1306 2377}%
\special{pa 1308 2377}%
\special{pa 1308 2378}%
\special{pa 1308 2378}%
\special{fp}%
\special{pa 1328 2390}%
\special{pa 1328 2391}%
\special{pa 1331 2392}%
\special{pa 1331 2393}%
\special{pa 1333 2395}%
\special{pa 1333 2396}%
\special{pa 1334 2396}%
\special{pa 1334 2397}%
\special{pa 1335 2397}%
\special{pa 1335 2399}%
\special{pa 1336 2399}%
\special{pa 1336 2400}%
\special{pa 1337 2400}%
\special{pa 1337 2402}%
\special{pa 1338 2403}%
\special{pa 1338 2406}%
\special{pa 1339 2406}%
\special{pa 1339 2409}%
\special{fp}%
% LINE 2 0 3 3 Black White  
% 4 1024 2448 1024 656 1360 656 1360 2448
% 
\special{pn 8}%
\special{pa 1008 2409}%
\special{pa 1008 646}%
\special{fp}%
\special{pa 1339 646}%
\special{pa 1339 2409}%
\special{fp}%
% LINE 2 0 3 4 Black White  
% 2 968 656 800 656
% 
\special{pn 8}%
\special{pa 953 646}%
\special{pa 787 646}%
\special{fp}%
% VECTOR 2 0 3 5 Black White  
% 4 884 1440 884 656 884 1608 884 2448
% 
\special{pn 8}%
\special{pa 870 1417}%
\special{pa 870 646}%
\special{fp}%
\special{sh 1}%
\special{pa 870 646}%
\special{pa 850 712}%
\special{pa 870 698}%
\special{pa 890 712}%
\special{pa 870 646}%
\special{fp}%
\special{pa 870 1583}%
\special{pa 870 2409}%
\special{fp}%
\special{sh 1}%
\special{pa 870 2409}%
\special{pa 890 2344}%
\special{pa 870 2357}%
\special{pa 850 2344}%
\special{pa 870 2409}%
\special{fp}%
% LINE 2 0 3 6 Black White  
% 2 968 2448 800 2448
% 
\special{pn 8}%
\special{pa 953 2409}%
\special{pa 787 2409}%
\special{fp}%
% STR 2 0 3 7 Black White  
% 4 884 1496 884 1524 5 0 0 0
% $\ell$
\put(8.7008,-15.0000){\makebox(0,0){$\ell$}}%
% VECTOR 2 0 3 8 Black White  
% 4 1360 1524 2200 1524 1360 1524 1920 1076
% 
\special{pn 8}%
\special{pa 1339 1500}%
\special{pa 2165 1500}%
\special{fp}%
\special{sh 1}%
\special{pa 2165 1500}%
\special{pa 2099 1480}%
\special{pa 2113 1500}%
\special{pa 2099 1520}%
\special{pa 2165 1500}%
\special{fp}%
\special{pa 1339 1500}%
\special{pa 1890 1059}%
\special{fp}%
\special{sh 1}%
\special{pa 1890 1059}%
\special{pa 1826 1085}%
\special{pa 1848 1092}%
\special{pa 1850 1115}%
\special{pa 1890 1059}%
\special{fp}%
% LINE 2 0 3 9 Black White  
% 4 1472 1434 1472 1311 1472 1311 1360 1401
% 
\special{pn 8}%
\special{pa 1449 1411}%
\special{pa 1449 1290}%
\special{fp}%
\special{pa 1449 1290}%
\special{pa 1339 1379}%
\special{fp}%
% LINE 2 0 3 10 Black White  
% 4 1472 1434 1598 1434 1598 1434 1494 1524
% 
\special{pn 8}%
\special{pa 1449 1411}%
\special{pa 1573 1411}%
\special{fp}%
\special{pa 1573 1411}%
\special{pa 1470 1500}%
\special{fp}%
% STR 2 0 3 0 Black White  
% 4 1930 1580 1930 1680 2 0 0 0
% $v$
\put(18.9961,-16.5354){\makebox(0,0)[lb]{$v$}}%
% STR 2 0 3 0 Black White  
% 4 1960 1010 1960 1110 2 0 0 0
% $B$
\put(19.2913,-10.9252){\makebox(0,0)[lb]{$B$}}%
% STR 2 0 3 0 Black White  
% 4 1500 1620 1500 1720 5 0 0 0
% 導
\put(14.7638,-16.9291){\makebox(0,0){導}}%
% STR 2 0 3 0 Black White  
% 4 1500 1770 1500 1870 5 0 0 0
% 体
\put(14.7638,-18.4055){\makebox(0,0){体}}%
% STR 2 0 3 0 Black White  
% 4 1500 1910 1500 2010 5 0 0 0
% 棒
\put(14.7638,-19.7835){\makebox(0,0){棒}}%
% STR 2 0 3 0 Black White  
% 4 1470 550 1470 650 5 0 0 0
% C
\put(14.4685,-6.3976){\makebox(0,0){C}}%
% STR 2 0 3 0 Black White  
% 4 1470 2340 1470 2440 5 0 0 0
% D
\put(14.4685,-24.0157){\makebox(0,0){D}}%
\end{picture}}%

    }
    誘導起電力の発生メカニズムを考えてみよう。図に示すように,磁束密度$B$\tanni{T}の一様な磁場に垂直に置かれた長さ$\ell$\tanni{m}の1本の導体棒CDを,棒自身にも磁場にも直角方向に速度$v$\tanni{m/s}で動かしてみる。\\
    ~~導体中の,電荷$-e$\tanni{C}の自由電子波,棒とともに速度$v$\tanni{m/s}で動くから,磁場によるローレンツ力を受ける。その大きさは$f$=\Hako \tanni{N}で,その向きは\Hako である。その結果,棒の\Hako 端は負に帯電し,\Hako 端は正に帯電する。これらの電荷により,棒CD中には強さ$E$\tanni{V/m}の一様な電場が生じる。残りの自由電子はこの電場から電気力も受けることになる。その大きさは$F=$\Hako \tanni{N}で,その向きはローレンツ力と\Hako 向きである。したがって,電子の移動はやがて終わり,電場の強さは$E=$\Hako \tanni{V/m}となる。そして,棒の\Hako 端は\Hako 端より電位が$V=$\Hako \tanni{V}だけ高い。こうして,導体棒CDは,電位の高いほうがプラスとなるような起電力$V$\tanni{V}の電池と同等に考えることができる。これを誘導起電力とよぶ。
    \end{mawarikomi}