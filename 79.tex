\hakosyokika
\item
    \begin{enumerate}
        \item ある時,音源から出た音の波面Hが,静止している人に向かう。音速を$V$とし,音源が近づく速さを$v(<V)$とすると,時間$t$後の音源と波面Hとの距離は\Hako である。音源の振動数を$f_0$とすると,この距離の間に\Hako 個の波が入っているので,音波の波長は\Hako となる。その結果,人には振動数が\Hako の音として聞こえる。
        \item 静止している音源に向かって人が速さ$u(<V)$で近づいていくときにも類似の現象が起こる。このとき,単位時間に人は\Hako の距離の間に含まれる音波が到達するから,人は振動数\Hako の音として聞く。
        \item 音源が速さ$v$で,人が速さ$u$で互いに近づくときには,人には振動数\Hako の音が聞こえる。
        \item 一定の速さ$w(<V)$の風が吹いているとする。風と同じ向きに音源が$v$で進むとき,前方で静止している人には振動数\Hako の音が聞こえる。
    \end{enumerate}