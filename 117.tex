\hakosyokika
\item
    \begin{mawarikomi}(20pt,0pt){150pt}{
        %%% C:/vpn/vpn/KeTCindy/fig/fig117_1.tex 
%%% Generator=fig117_1.cdy 
{\unitlength=1.5cm%
\begin{picture}%
(4.25,3.25)(-0.5,-0.75)%
\special{pn 8}%
%
\settowidth{\Width}{$0.2$}\setlength{\Width}{-0.5\Width}%
\settoheight{\Height}{$0.2$}\settodepth{\Depth}{$0.2$}\setlength{\Height}{-\Height}%
\put(  0.500, -0.100){\hspace*{\Width}\raisebox{\Height}{$0.2$}}%
%
\special{pa   295    -0}\special{pa   295    59}%
\special{fp}%
\settowidth{\Width}{$0.4$}\setlength{\Width}{-0.5\Width}%
\settoheight{\Height}{$0.4$}\settodepth{\Depth}{$0.4$}\setlength{\Height}{-\Height}%
\put(  1.000, -0.100){\hspace*{\Width}\raisebox{\Height}{$0.4$}}%
%
\special{pa   591    -0}\special{pa   591    59}%
\special{fp}%
\settowidth{\Width}{$0.6$}\setlength{\Width}{-0.5\Width}%
\settoheight{\Height}{$0.6$}\settodepth{\Depth}{$0.6$}\setlength{\Height}{-\Height}%
\put(  1.500, -0.100){\hspace*{\Width}\raisebox{\Height}{$0.6$}}%
%
\special{pa   886    -0}\special{pa   886    59}%
\special{fp}%
\settowidth{\Width}{$0.8$}\setlength{\Width}{-0.5\Width}%
\settoheight{\Height}{$0.8$}\settodepth{\Depth}{$0.8$}\setlength{\Height}{-\Height}%
\put(  2.000, -0.100){\hspace*{\Width}\raisebox{\Height}{$0.8$}}%
%
\special{pa  1181    -0}\special{pa  1181    59}%
\special{fp}%
\settowidth{\Width}{$1.0$}\setlength{\Width}{-0.5\Width}%
\settoheight{\Height}{$1.0$}\settodepth{\Depth}{$1.0$}\setlength{\Height}{-\Height}%
\put(  2.500, -0.100){\hspace*{\Width}\raisebox{\Height}{$1.0$}}%
%
\special{pa  1476    -0}\special{pa  1476    59}%
\special{fp}%
\settowidth{\Width}{$1.2$}\setlength{\Width}{-0.5\Width}%
\settoheight{\Height}{$1.2$}\settodepth{\Depth}{$1.2$}\setlength{\Height}{-\Height}%
\put(  3.000, -0.100){\hspace*{\Width}\raisebox{\Height}{$1.2$}}%
%
\special{pa  1772    -0}\special{pa  1772    59}%
\special{fp}%
\settowidth{\Width}{$0$}\setlength{\Width}{-1\Width}%
\settoheight{\Height}{$0$}\settodepth{\Depth}{$0$}\setlength{\Height}{-0.5\Height}\setlength{\Depth}{0.5\Depth}\addtolength{\Height}{\Depth}%
\put( -0.100,  0.000){\hspace*{\Width}\raisebox{\Height}{$0$}}%
%
\special{pa     0    -0}\special{pa   -59    -0}%
\special{fp}%
\settowidth{\Width}{$0.5$}\setlength{\Width}{-1\Width}%
\settoheight{\Height}{$0.5$}\settodepth{\Depth}{$0.5$}\setlength{\Height}{-0.5\Height}\setlength{\Depth}{0.5\Depth}\addtolength{\Height}{\Depth}%
\put( -0.100,  0.500){\hspace*{\Width}\raisebox{\Height}{$0.5$}}%
%
\special{pa     0  -295}\special{pa   -59  -295}%
\special{fp}%
\settowidth{\Width}{$1.0$}\setlength{\Width}{-1\Width}%
\settoheight{\Height}{$1.0$}\settodepth{\Depth}{$1.0$}\setlength{\Height}{-0.5\Height}\setlength{\Depth}{0.5\Depth}\addtolength{\Height}{\Depth}%
\put( -0.100,  1.000){\hspace*{\Width}\raisebox{\Height}{$1.0$}}%
%
\special{pa     0  -591}\special{pa   -59  -591}%
\special{fp}%
\settowidth{\Width}{$1.5$}\setlength{\Width}{-1\Width}%
\settoheight{\Height}{$1.5$}\settodepth{\Depth}{$1.5$}\setlength{\Height}{-0.5\Height}\setlength{\Depth}{0.5\Depth}\addtolength{\Height}{\Depth}%
\put( -0.100,  1.500){\hspace*{\Width}\raisebox{\Height}{$1.5$}}%
%
\special{pa     0  -886}\special{pa   -59  -886}%
\special{fp}%
\settowidth{\Width}{$2.0$}\setlength{\Width}{-1\Width}%
\settoheight{\Height}{$2.0$}\settodepth{\Depth}{$2.0$}\setlength{\Height}{-0.5\Height}\setlength{\Depth}{0.5\Depth}\addtolength{\Height}{\Depth}%
\put( -0.100,  2.000){\hspace*{\Width}\raisebox{\Height}{$2.0$}}%
%
\special{pa     0 -1181}\special{pa   -59 -1181}%
\special{fp}%
\special{pa 295 0}\special{pa 295 -38}\special{fp}\special{pa 295 -76}\special{pa 295 -114}\special{fp}%
\special{pa 295 -152}\special{pa 295 -190}\special{fp}\special{pa 295 -228}\special{pa 295 -266}\special{fp}%
\special{pa 295 -304}\special{pa 295 -342}\special{fp}\special{pa 295 -380}\special{pa 295 -418}\special{fp}%
\special{pa 295 -456}\special{pa 295 -494}\special{fp}\special{pa 295 -531}\special{pa 295 -569}\special{fp}%
\special{pa 295 -607}\special{pa 295 -645}\special{fp}\special{pa 295 -683}\special{pa 295 -721}\special{fp}%
\special{pa 295 -759}\special{pa 295 -797}\special{fp}\special{pa 295 -835}\special{pa 295 -873}\special{fp}%
\special{pa 295 -911}\special{pa 295 -949}\special{fp}\special{pa 295 -987}\special{pa 295 -1025}\special{fp}%
\special{pa 295 -1063}\special{pa 295 -1101}\special{fp}\special{pa 295 -1139}\special{pa 295 -1177}\special{fp}%
\special{pa 295 -1215}\special{pa 295 -1253}\special{fp}\special{pa 295 -1291}\special{pa 295 -1329}\special{fp}%
%
%
\special{pa 591 0}\special{pa 591 -38}\special{fp}\special{pa 591 -76}\special{pa 591 -114}\special{fp}%
\special{pa 591 -152}\special{pa 591 -190}\special{fp}\special{pa 591 -228}\special{pa 591 -266}\special{fp}%
\special{pa 591 -304}\special{pa 591 -342}\special{fp}\special{pa 591 -380}\special{pa 591 -418}\special{fp}%
\special{pa 591 -456}\special{pa 591 -494}\special{fp}\special{pa 591 -531}\special{pa 591 -569}\special{fp}%
\special{pa 591 -607}\special{pa 591 -645}\special{fp}\special{pa 591 -683}\special{pa 591 -721}\special{fp}%
\special{pa 591 -759}\special{pa 591 -797}\special{fp}\special{pa 591 -835}\special{pa 591 -873}\special{fp}%
\special{pa 591 -911}\special{pa 591 -949}\special{fp}\special{pa 591 -987}\special{pa 591 -1025}\special{fp}%
\special{pa 591 -1063}\special{pa 591 -1101}\special{fp}\special{pa 591 -1139}\special{pa 591 -1177}\special{fp}%
\special{pa 591 -1215}\special{pa 591 -1253}\special{fp}\special{pa 591 -1291}\special{pa 591 -1329}\special{fp}%
%
%
\special{pa 886 0}\special{pa 886 -38}\special{fp}\special{pa 886 -76}\special{pa 886 -114}\special{fp}%
\special{pa 886 -152}\special{pa 886 -190}\special{fp}\special{pa 886 -228}\special{pa 886 -266}\special{fp}%
\special{pa 886 -304}\special{pa 886 -342}\special{fp}\special{pa 886 -380}\special{pa 886 -418}\special{fp}%
\special{pa 886 -456}\special{pa 886 -494}\special{fp}\special{pa 886 -531}\special{pa 886 -569}\special{fp}%
\special{pa 886 -607}\special{pa 886 -645}\special{fp}\special{pa 886 -683}\special{pa 886 -721}\special{fp}%
\special{pa 886 -759}\special{pa 886 -797}\special{fp}\special{pa 886 -835}\special{pa 886 -873}\special{fp}%
\special{pa 886 -911}\special{pa 886 -949}\special{fp}\special{pa 886 -987}\special{pa 886 -1025}\special{fp}%
\special{pa 886 -1063}\special{pa 886 -1101}\special{fp}\special{pa 886 -1139}\special{pa 886 -1177}\special{fp}%
\special{pa 886 -1215}\special{pa 886 -1253}\special{fp}\special{pa 886 -1291}\special{pa 886 -1329}\special{fp}%
%
%
\special{pa 1181 0}\special{pa 1181 -38}\special{fp}\special{pa 1181 -76}\special{pa 1181 -114}\special{fp}%
\special{pa 1181 -152}\special{pa 1181 -190}\special{fp}\special{pa 1181 -228}\special{pa 1181 -266}\special{fp}%
\special{pa 1181 -304}\special{pa 1181 -342}\special{fp}\special{pa 1181 -380}\special{pa 1181 -418}\special{fp}%
\special{pa 1181 -456}\special{pa 1181 -494}\special{fp}\special{pa 1181 -531}\special{pa 1181 -569}\special{fp}%
\special{pa 1181 -607}\special{pa 1181 -645}\special{fp}\special{pa 1181 -683}\special{pa 1181 -721}\special{fp}%
\special{pa 1181 -759}\special{pa 1181 -797}\special{fp}\special{pa 1181 -835}\special{pa 1181 -873}\special{fp}%
\special{pa 1181 -911}\special{pa 1181 -949}\special{fp}\special{pa 1181 -987}\special{pa 1181 -1025}\special{fp}%
\special{pa 1181 -1063}\special{pa 1181 -1101}\special{fp}\special{pa 1181 -1139}\special{pa 1181 -1177}\special{fp}%
\special{pa 1181 -1215}\special{pa 1181 -1253}\special{fp}\special{pa 1181 -1291}\special{pa 1181 -1329}\special{fp}%
%
%
\special{pa 1476 0}\special{pa 1476 -38}\special{fp}\special{pa 1476 -76}\special{pa 1476 -114}\special{fp}%
\special{pa 1476 -152}\special{pa 1476 -190}\special{fp}\special{pa 1476 -228}\special{pa 1476 -266}\special{fp}%
\special{pa 1476 -304}\special{pa 1476 -342}\special{fp}\special{pa 1476 -380}\special{pa 1476 -418}\special{fp}%
\special{pa 1476 -456}\special{pa 1476 -494}\special{fp}\special{pa 1476 -531}\special{pa 1476 -569}\special{fp}%
\special{pa 1476 -607}\special{pa 1476 -645}\special{fp}\special{pa 1476 -683}\special{pa 1476 -721}\special{fp}%
\special{pa 1476 -759}\special{pa 1476 -797}\special{fp}\special{pa 1476 -835}\special{pa 1476 -873}\special{fp}%
\special{pa 1476 -911}\special{pa 1476 -949}\special{fp}\special{pa 1476 -987}\special{pa 1476 -1025}\special{fp}%
\special{pa 1476 -1063}\special{pa 1476 -1101}\special{fp}\special{pa 1476 -1139}\special{pa 1476 -1177}\special{fp}%
\special{pa 1476 -1215}\special{pa 1476 -1253}\special{fp}\special{pa 1476 -1291}\special{pa 1476 -1329}\special{fp}%
%
%
\special{pa 1772 0}\special{pa 1772 -38}\special{fp}\special{pa 1772 -76}\special{pa 1772 -114}\special{fp}%
\special{pa 1772 -152}\special{pa 1772 -190}\special{fp}\special{pa 1772 -228}\special{pa 1772 -266}\special{fp}%
\special{pa 1772 -304}\special{pa 1772 -342}\special{fp}\special{pa 1772 -380}\special{pa 1772 -418}\special{fp}%
\special{pa 1772 -456}\special{pa 1772 -494}\special{fp}\special{pa 1772 -531}\special{pa 1772 -569}\special{fp}%
\special{pa 1772 -607}\special{pa 1772 -645}\special{fp}\special{pa 1772 -683}\special{pa 1772 -721}\special{fp}%
\special{pa 1772 -759}\special{pa 1772 -797}\special{fp}\special{pa 1772 -835}\special{pa 1772 -873}\special{fp}%
\special{pa 1772 -911}\special{pa 1772 -949}\special{fp}\special{pa 1772 -987}\special{pa 1772 -1025}\special{fp}%
\special{pa 1772 -1063}\special{pa 1772 -1101}\special{fp}\special{pa 1772 -1139}\special{pa 1772 -1177}\special{fp}%
\special{pa 1772 -1215}\special{pa 1772 -1253}\special{fp}\special{pa 1772 -1291}\special{pa 1772 -1329}\special{fp}%
%
%
\special{pa 0 -295}\special{pa 39 -295}\special{fp}\special{pa 78 -295}\special{pa 118 -295}\special{fp}%
\special{pa 157 -295}\special{pa 196 -295}\special{fp}\special{pa 235 -295}\special{pa 274 -295}\special{fp}%
\special{pa 313 -295}\special{pa 353 -295}\special{fp}\special{pa 392 -295}\special{pa 431 -295}\special{fp}%
\special{pa 470 -295}\special{pa 509 -295}\special{fp}\special{pa 548 -295}\special{pa 588 -295}\special{fp}%
\special{pa 627 -295}\special{pa 666 -295}\special{fp}\special{pa 705 -295}\special{pa 744 -295}\special{fp}%
\special{pa 783 -295}\special{pa 823 -295}\special{fp}\special{pa 862 -295}\special{pa 901 -295}\special{fp}%
\special{pa 940 -295}\special{pa 979 -295}\special{fp}\special{pa 1018 -295}\special{pa 1058 -295}\special{fp}%
\special{pa 1097 -295}\special{pa 1136 -295}\special{fp}\special{pa 1175 -295}\special{pa 1214 -295}\special{fp}%
\special{pa 1253 -295}\special{pa 1293 -295}\special{fp}\special{pa 1332 -295}\special{pa 1371 -295}\special{fp}%
\special{pa 1410 -295}\special{pa 1449 -295}\special{fp}\special{pa 1488 -295}\special{pa 1528 -295}\special{fp}%
\special{pa 1567 -295}\special{pa 1606 -295}\special{fp}\special{pa 1645 -295}\special{pa 1684 -295}\special{fp}%
\special{pa 1723 -295}\special{pa 1763 -295}\special{fp}\special{pa 1802 -295}\special{pa 1841 -295}\special{fp}%
\special{pa 1880 -295}\special{pa 1919 -295}\special{fp}%
%
\special{pa 0 -591}\special{pa 39 -591}\special{fp}\special{pa 78 -591}\special{pa 118 -591}\special{fp}%
\special{pa 157 -591}\special{pa 196 -591}\special{fp}\special{pa 235 -591}\special{pa 274 -591}\special{fp}%
\special{pa 313 -591}\special{pa 353 -591}\special{fp}\special{pa 392 -591}\special{pa 431 -591}\special{fp}%
\special{pa 470 -591}\special{pa 509 -591}\special{fp}\special{pa 548 -591}\special{pa 588 -591}\special{fp}%
\special{pa 627 -591}\special{pa 666 -591}\special{fp}\special{pa 705 -591}\special{pa 744 -591}\special{fp}%
\special{pa 783 -591}\special{pa 823 -591}\special{fp}\special{pa 862 -591}\special{pa 901 -591}\special{fp}%
\special{pa 940 -591}\special{pa 979 -591}\special{fp}\special{pa 1018 -591}\special{pa 1058 -591}\special{fp}%
\special{pa 1097 -591}\special{pa 1136 -591}\special{fp}\special{pa 1175 -591}\special{pa 1214 -591}\special{fp}%
\special{pa 1253 -591}\special{pa 1293 -591}\special{fp}\special{pa 1332 -591}\special{pa 1371 -591}\special{fp}%
\special{pa 1410 -591}\special{pa 1449 -591}\special{fp}\special{pa 1488 -591}\special{pa 1528 -591}\special{fp}%
\special{pa 1567 -591}\special{pa 1606 -591}\special{fp}\special{pa 1645 -591}\special{pa 1684 -591}\special{fp}%
\special{pa 1723 -591}\special{pa 1763 -591}\special{fp}\special{pa 1802 -591}\special{pa 1841 -591}\special{fp}%
\special{pa 1880 -591}\special{pa 1919 -591}\special{fp}%
%
\special{pa 0 -886}\special{pa 39 -886}\special{fp}\special{pa 78 -886}\special{pa 118 -886}\special{fp}%
\special{pa 157 -886}\special{pa 196 -886}\special{fp}\special{pa 235 -886}\special{pa 274 -886}\special{fp}%
\special{pa 313 -886}\special{pa 353 -886}\special{fp}\special{pa 392 -886}\special{pa 431 -886}\special{fp}%
\special{pa 470 -886}\special{pa 509 -886}\special{fp}\special{pa 548 -886}\special{pa 588 -886}\special{fp}%
\special{pa 627 -886}\special{pa 666 -886}\special{fp}\special{pa 705 -886}\special{pa 744 -886}\special{fp}%
\special{pa 783 -886}\special{pa 823 -886}\special{fp}\special{pa 862 -886}\special{pa 901 -886}\special{fp}%
\special{pa 940 -886}\special{pa 979 -886}\special{fp}\special{pa 1018 -886}\special{pa 1058 -886}\special{fp}%
\special{pa 1097 -886}\special{pa 1136 -886}\special{fp}\special{pa 1175 -886}\special{pa 1214 -886}\special{fp}%
\special{pa 1253 -886}\special{pa 1293 -886}\special{fp}\special{pa 1332 -886}\special{pa 1371 -886}\special{fp}%
\special{pa 1410 -886}\special{pa 1449 -886}\special{fp}\special{pa 1488 -886}\special{pa 1528 -886}\special{fp}%
\special{pa 1567 -886}\special{pa 1606 -886}\special{fp}\special{pa 1645 -886}\special{pa 1684 -886}\special{fp}%
\special{pa 1723 -886}\special{pa 1763 -886}\special{fp}\special{pa 1802 -886}\special{pa 1841 -886}\special{fp}%
\special{pa 1880 -886}\special{pa 1919 -886}\special{fp}%
%
\special{pa 0 -1181}\special{pa 39 -1181}\special{fp}\special{pa 78 -1181}\special{pa 118 -1181}\special{fp}%
\special{pa 157 -1181}\special{pa 196 -1181}\special{fp}\special{pa 235 -1181}\special{pa 274 -1181}\special{fp}%
\special{pa 313 -1181}\special{pa 353 -1181}\special{fp}\special{pa 392 -1181}\special{pa 431 -1181}\special{fp}%
\special{pa 470 -1181}\special{pa 509 -1181}\special{fp}\special{pa 548 -1181}\special{pa 588 -1181}\special{fp}%
\special{pa 627 -1181}\special{pa 666 -1181}\special{fp}\special{pa 705 -1181}\special{pa 744 -1181}\special{fp}%
\special{pa 783 -1181}\special{pa 823 -1181}\special{fp}\special{pa 862 -1181}\special{pa 901 -1181}\special{fp}%
\special{pa 940 -1181}\special{pa 979 -1181}\special{fp}\special{pa 1018 -1181}\special{pa 1058 -1181}\special{fp}%
\special{pa 1097 -1181}\special{pa 1136 -1181}\special{fp}\special{pa 1175 -1181}\special{pa 1214 -1181}\special{fp}%
\special{pa 1253 -1181}\special{pa 1293 -1181}\special{fp}\special{pa 1332 -1181}\special{pa 1371 -1181}\special{fp}%
\special{pa 1410 -1181}\special{pa 1449 -1181}\special{fp}\special{pa 1488 -1181}\special{pa 1528 -1181}\special{fp}%
\special{pa 1567 -1181}\special{pa 1606 -1181}\special{fp}\special{pa 1645 -1181}\special{pa 1684 -1181}\special{fp}%
\special{pa 1723 -1181}\special{pa 1763 -1181}\special{fp}\special{pa 1802 -1181}\special{pa 1841 -1181}\special{fp}%
\special{pa 1880 -1181}\special{pa 1919 -1181}\special{fp}%
%
\special{pa 1844 24}\special{pa 1919 0}\special{pa 1844 -24}\special{pa 1859 0}\special{pa 1844 24}%
\special{pa 1844 24}\special{sh 1}\special{ip}%
\special{pn 1}%
\special{pa  1844    24}\special{pa  1919    -0}\special{pa  1844   -24}\special{pa  1859    -0}%
\special{pa  1844    24}%
\special{fp}%
\special{pn 8}%
\special{pa     0    -0}\special{pa  1859    -0}%
\special{fp}%
\special{pa 24 -1254}\special{pa 0 -1329}\special{pa -24 -1254}\special{pa 0 -1269}%
\special{pa 24 -1254}\special{pa 24 -1254}\special{sh 1}\special{ip}%
\special{pn 1}%
\special{pa    24 -1254}\special{pa     0 -1329}\special{pa   -24 -1254}\special{pa     0 -1269}%
\special{pa    24 -1254}%
\special{fp}%
\special{pn 8}%
\special{pa     0    -0}\special{pa     0 -1269}%
\special{fp}%
\special{pa   179  -909}\special{pa  1887  -567}%
\special{fp}%
\settowidth{\Width}{$I\mathrm{〔A〕}$}\setlength{\Width}{0\Width}%
\settoheight{\Height}{$I\mathrm{〔A〕}$}\settodepth{\Depth}{$I\mathrm{〔A〕}$}\setlength{\Height}{-\Height}%
\put(  3.317, -0.067){\hspace*{\Width}\raisebox{\Height}{$I\mathrm{〔A〕}$}}%
%
\settowidth{\Width}{$V\mathrm{〔V〕}$}\setlength{\Width}{-1\Width}%
\settoheight{\Height}{$V\mathrm{〔V〕}$}\settodepth{\Depth}{$V\mathrm{〔V〕}$}\setlength{\Height}{-0.5\Height}\setlength{\Depth}{0.5\Depth}\addtolength{\Height}{\Depth}%
\put( -0.067,  2.250){\hspace*{\Width}\raisebox{\Height}{$V\mathrm{〔V〕}$}}%
%
\settowidth{\Width}{$図1$}\setlength{\Width}{-0.5\Width}%
\settoheight{\Height}{$図1$}\settodepth{\Depth}{$図1$}\setlength{\Height}{-0.5\Height}\setlength{\Depth}{0.5\Depth}\addtolength{\Height}{\Depth}%
\put(  1.550, -0.500){\hspace*{\Width}\raisebox{\Height}{$図1$}}%
%
\end{picture}}%
        %%% C:/vpn/KeTCindy/fig/fig117_2.tex 
%%% Generator=fig117_2.cdy 
{\unitlength=1.5cm%
\begin{picture}%
(4.25,2.5)(-0.5,0)%
\special{pn 8}%
%
\special{pa   812 -1107}\special{pa   517 -1107}\special{pa   517  -960}\special{pa   812  -960}%
\special{pa   812 -1107}%
\special{fp}%
\special{pa   295 -1033}\special{pa   517 -1033}%
\special{fp}%
\special{pa  1033 -1033}\special{pa   812 -1033}%
\special{fp}%
\special{pa  1550 -1107}\special{pa  1255 -1107}\special{pa  1255  -960}\special{pa  1550  -960}%
\special{pa  1550 -1107}%
\special{fp}%
\special{pa  1033 -1033}\special{pa  1255 -1033}%
\special{fp}%
\special{pa  1772 -1033}\special{pa  1550 -1033}%
\special{fp}%
\special{pa  1181  -517}\special{pa   886  -517}\special{pa   886  -369}\special{pa  1181  -369}%
\special{pa  1181  -517}%
\special{fp}%
\special{pa   295  -443}\special{pa   886  -443}%
\special{fp}%
\special{pa  1772  -443}\special{pa  1181  -443}%
\special{fp}%
\special{pa   295 -1033}\special{pa   295  -443}%
\special{fp}%
\special{pa  1772 -1033}\special{pa  1772  -443}%
\special{fp}%
\special{pa     0  -738}\special{pa   295  -738}%
\special{fp}%
\special{pa  2067  -738}\special{pa  1772  -738}%
\special{fp}%
\settowidth{\Width}{A}\setlength{\Width}{-0.5\Width}%
\settoheight{\Height}{A}\settodepth{\Depth}{A}\setlength{\Height}{\Depth}%
\put(  0.000,  1.317){\hspace*{\Width}\raisebox{\Height}{A}}%
%
\settowidth{\Width}{B}\setlength{\Width}{-0.5\Width}%
\settoheight{\Height}{B}\settodepth{\Depth}{B}\setlength{\Height}{\Depth}%
\put(  3.500,  1.317){\hspace*{\Width}\raisebox{\Height}{B}}%
%
\settowidth{\Width}{$\mathrm{10{\sf \Omega }}$}\setlength{\Width}{-0.5\Width}%
\settoheight{\Height}{$\mathrm{10{\sf \Omega }}$}\settodepth{\Depth}{$\mathrm{10{\sf \Omega }}$}\setlength{\Height}{\Depth}%
\put(  1.120,  2.007){\hspace*{\Width}\raisebox{\Height}{$\mathrm{10{\sf \Omega }}$}}%
%
\settowidth{\Width}{$R$}\setlength{\Width}{-0.5\Width}%
\settoheight{\Height}{$R$}\settodepth{\Depth}{$R$}\setlength{\Height}{\Depth}%
\put(  2.380,  2.007){\hspace*{\Width}\raisebox{\Height}{$R$}}%
%
\settowidth{\Width}{$\mathrm{12{\sf \Omega }}$}\setlength{\Width}{-0.5\Width}%
\settoheight{\Height}{$\mathrm{12{\sf \Omega }}$}\settodepth{\Depth}{$\mathrm{12{\sf \Omega }}$}\setlength{\Height}{-\Height}%
\put(  1.750,  0.500){\hspace*{\Width}\raisebox{\Height}{$\mathrm{12{\sf \Omega }}$}}%
%
{%
\color[cmyk]{0,0,0,0}%
\special{pa 22 -738}\special{pa 22 -741}\special{pa 22 -744}\special{pa 21 -746}\special{pa 20 -749}%
\special{pa 18 -751}\special{pa 16 -754}\special{pa 14 -755}\special{pa 12 -757}\special{pa 10 -758}%
\special{pa 7 -760}\special{pa 4 -760}\special{pa 1 -761}\special{pa -1 -761}\special{pa -4 -760}%
\special{pa -7 -760}\special{pa -10 -758}\special{pa -12 -757}\special{pa -14 -755}%
\special{pa -16 -754}\special{pa -18 -751}\special{pa -20 -749}\special{pa -21 -746}%
\special{pa -22 -744}\special{pa -22 -741}\special{pa -22 -738}\special{pa -22 -735}%
\special{pa -22 -733}\special{pa -21 -730}\special{pa -20 -727}\special{pa -18 -725}%
\special{pa -16 -723}\special{pa -14 -721}\special{pa -12 -719}\special{pa -10 -718}%
\special{pa -7 -717}\special{pa -4 -716}\special{pa -1 -716}\special{pa 1 -716}\special{pa 4 -716}%
\special{pa 7 -717}\special{pa 10 -718}\special{pa 12 -719}\special{pa 14 -721}\special{pa 16 -723}%
\special{pa 18 -725}\special{pa 20 -727}\special{pa 21 -730}\special{pa 22 -733}\special{pa 22 -735}%
\special{pa 22 -738}\special{pa 22 -738}\special{sh 1}\special{ip}%
}%
\special{pa    22  -738}\special{pa    22  -741}\special{pa    22  -744}\special{pa    21  -746}%
\special{pa    20  -749}\special{pa    18  -751}\special{pa    16  -754}\special{pa    14  -755}%
\special{pa    12  -757}\special{pa    10  -758}\special{pa     7  -760}\special{pa     4  -760}%
\special{pa     1  -761}\special{pa    -1  -761}\special{pa    -4  -760}\special{pa    -7  -760}%
\special{pa   -10  -758}\special{pa   -12  -757}\special{pa   -14  -755}\special{pa   -16  -754}%
\special{pa   -18  -751}\special{pa   -20  -749}\special{pa   -21  -746}\special{pa   -22  -744}%
\special{pa   -22  -741}\special{pa   -22  -738}\special{pa   -22  -735}\special{pa   -22  -733}%
\special{pa   -21  -730}\special{pa   -20  -727}\special{pa   -18  -725}\special{pa   -16  -723}%
\special{pa   -14  -721}\special{pa   -12  -719}\special{pa   -10  -718}\special{pa    -7  -717}%
\special{pa    -4  -716}\special{pa    -1  -716}\special{pa     1  -716}\special{pa     4  -716}%
\special{pa     7  -717}\special{pa    10  -718}\special{pa    12  -719}\special{pa    14  -721}%
\special{pa    16  -723}\special{pa    18  -725}\special{pa    20  -727}\special{pa    21  -730}%
\special{pa    22  -733}\special{pa    22  -735}\special{pa    22  -738}%
\special{fp}%
{%
\color[cmyk]{0,0,0,0}%
\special{pa 2089 -738}\special{pa 2089 -741}\special{pa 2089 -744}\special{pa 2088 -746}%
\special{pa 2087 -749}\special{pa 2085 -751}\special{pa 2083 -754}\special{pa 2081 -755}%
\special{pa 2079 -757}\special{pa 2076 -758}\special{pa 2074 -760}\special{pa 2071 -760}%
\special{pa 2068 -761}\special{pa 2066 -761}\special{pa 2063 -760}\special{pa 2060 -760}%
\special{pa 2057 -758}\special{pa 2055 -757}\special{pa 2053 -755}\special{pa 2051 -754}%
\special{pa 2049 -751}\special{pa 2047 -749}\special{pa 2046 -746}\special{pa 2045 -744}%
\special{pa 2045 -741}\special{pa 2044 -738}\special{pa 2045 -735}\special{pa 2045 -733}%
\special{pa 2046 -730}\special{pa 2047 -727}\special{pa 2049 -725}\special{pa 2051 -723}%
\special{pa 2053 -721}\special{pa 2055 -719}\special{pa 2057 -718}\special{pa 2060 -717}%
\special{pa 2063 -716}\special{pa 2066 -716}\special{pa 2068 -716}\special{pa 2071 -716}%
\special{pa 2074 -717}\special{pa 2076 -718}\special{pa 2079 -719}\special{pa 2081 -721}%
\special{pa 2083 -723}\special{pa 2085 -725}\special{pa 2087 -727}\special{pa 2088 -730}%
\special{pa 2089 -733}\special{pa 2089 -735}\special{pa 2089 -738}\special{pa 2089 -738}%
\special{sh 1}\special{ip}%
}%
\special{pa  2089  -738}\special{pa  2089  -741}\special{pa  2089  -744}\special{pa  2088  -746}%
\special{pa  2087  -749}\special{pa  2085  -751}\special{pa  2083  -754}\special{pa  2081  -755}%
\special{pa  2079  -757}\special{pa  2076  -758}\special{pa  2074  -760}\special{pa  2071  -760}%
\special{pa  2068  -761}\special{pa  2066  -761}\special{pa  2063  -760}\special{pa  2060  -760}%
\special{pa  2057  -758}\special{pa  2055  -757}\special{pa  2053  -755}\special{pa  2051  -754}%
\special{pa  2049  -751}\special{pa  2047  -749}\special{pa  2046  -746}\special{pa  2045  -744}%
\special{pa  2045  -741}\special{pa  2044  -738}\special{pa  2045  -735}\special{pa  2045  -733}%
\special{pa  2046  -730}\special{pa  2047  -727}\special{pa  2049  -725}\special{pa  2051  -723}%
\special{pa  2053  -721}\special{pa  2055  -719}\special{pa  2057  -718}\special{pa  2060  -717}%
\special{pa  2063  -716}\special{pa  2066  -716}\special{pa  2068  -716}\special{pa  2071  -716}%
\special{pa  2074  -717}\special{pa  2076  -718}\special{pa  2079  -719}\special{pa  2081  -721}%
\special{pa  2083  -723}\special{pa  2085  -725}\special{pa  2087  -727}\special{pa  2088  -730}%
\special{pa  2089  -733}\special{pa  2089  -735}\special{pa  2089  -738}%
\special{fp}%
\special{pa 306 -738}\special{pa 306 -740}\special{pa 306 -741}\special{pa 306 -742}%
\special{pa 305 -744}\special{pa 304 -745}\special{pa 303 -746}\special{pa 302 -747}%
\special{pa 301 -748}\special{pa 300 -748}\special{pa 299 -749}\special{pa 297 -749}%
\special{pa 296 -749}\special{pa 295 -749}\special{pa 293 -749}\special{pa 292 -749}%
\special{pa 290 -748}\special{pa 289 -748}\special{pa 288 -747}\special{pa 287 -746}%
\special{pa 286 -745}\special{pa 285 -744}\special{pa 285 -742}\special{pa 284 -741}%
\special{pa 284 -740}\special{pa 284 -738}\special{pa 284 -737}\special{pa 284 -735}%
\special{pa 285 -734}\special{pa 285 -733}\special{pa 286 -732}\special{pa 287 -731}%
\special{pa 288 -730}\special{pa 289 -729}\special{pa 290 -728}\special{pa 292 -728}%
\special{pa 293 -727}\special{pa 295 -727}\special{pa 296 -727}\special{pa 297 -727}%
\special{pa 299 -728}\special{pa 300 -728}\special{pa 301 -729}\special{pa 302 -730}%
\special{pa 303 -731}\special{pa 304 -732}\special{pa 305 -733}\special{pa 306 -734}%
\special{pa 306 -735}\special{pa 306 -737}\special{pa 306 -738}\special{pa 306 -738}%
\special{sh 1}\special{ip}%
\special{pa   306  -738}\special{pa   306  -740}\special{pa   306  -741}\special{pa   306  -742}%
\special{pa   305  -744}\special{pa   304  -745}\special{pa   303  -746}\special{pa   302  -747}%
\special{pa   301  -748}\special{pa   300  -748}\special{pa   299  -749}\special{pa   297  -749}%
\special{pa   296  -749}\special{pa   295  -749}\special{pa   293  -749}\special{pa   292  -749}%
\special{pa   290  -748}\special{pa   289  -748}\special{pa   288  -747}\special{pa   287  -746}%
\special{pa   286  -745}\special{pa   285  -744}\special{pa   285  -742}\special{pa   284  -741}%
\special{pa   284  -740}\special{pa   284  -738}\special{pa   284  -737}\special{pa   284  -735}%
\special{pa   285  -734}\special{pa   285  -733}\special{pa   286  -732}\special{pa   287  -731}%
\special{pa   288  -730}\special{pa   289  -729}\special{pa   290  -728}\special{pa   292  -728}%
\special{pa   293  -727}\special{pa   295  -727}\special{pa   296  -727}\special{pa   297  -727}%
\special{pa   299  -728}\special{pa   300  -728}\special{pa   301  -729}\special{pa   302  -730}%
\special{pa   303  -731}\special{pa   304  -732}\special{pa   305  -733}\special{pa   306  -734}%
\special{pa   306  -735}\special{pa   306  -737}\special{pa   306  -738}%
\special{fp}%
\special{pa 1783 -738}\special{pa 1783 -740}\special{pa 1783 -741}\special{pa 1782 -742}%
\special{pa 1781 -744}\special{pa 1781 -745}\special{pa 1780 -746}\special{pa 1779 -747}%
\special{pa 1778 -748}\special{pa 1776 -748}\special{pa 1775 -749}\special{pa 1774 -749}%
\special{pa 1772 -749}\special{pa 1771 -749}\special{pa 1770 -749}\special{pa 1768 -749}%
\special{pa 1767 -748}\special{pa 1766 -748}\special{pa 1765 -747}\special{pa 1763 -746}%
\special{pa 1763 -745}\special{pa 1762 -744}\special{pa 1761 -742}\special{pa 1761 -741}%
\special{pa 1761 -740}\special{pa 1760 -738}\special{pa 1761 -737}\special{pa 1761 -735}%
\special{pa 1761 -734}\special{pa 1762 -733}\special{pa 1763 -732}\special{pa 1763 -731}%
\special{pa 1765 -730}\special{pa 1766 -729}\special{pa 1767 -728}\special{pa 1768 -728}%
\special{pa 1770 -727}\special{pa 1771 -727}\special{pa 1772 -727}\special{pa 1774 -727}%
\special{pa 1775 -728}\special{pa 1776 -728}\special{pa 1778 -729}\special{pa 1779 -730}%
\special{pa 1780 -731}\special{pa 1781 -732}\special{pa 1781 -733}\special{pa 1782 -734}%
\special{pa 1783 -735}\special{pa 1783 -737}\special{pa 1783 -738}\special{pa 1783 -738}%
\special{sh 1}\special{ip}%
\special{pa  1783  -738}\special{pa  1783  -740}\special{pa  1783  -741}\special{pa  1782  -742}%
\special{pa  1781  -744}\special{pa  1781  -745}\special{pa  1780  -746}\special{pa  1779  -747}%
\special{pa  1778  -748}\special{pa  1776  -748}\special{pa  1775  -749}\special{pa  1774  -749}%
\special{pa  1772  -749}\special{pa  1771  -749}\special{pa  1770  -749}\special{pa  1768  -749}%
\special{pa  1767  -748}\special{pa  1766  -748}\special{pa  1765  -747}\special{pa  1763  -746}%
\special{pa  1763  -745}\special{pa  1762  -744}\special{pa  1761  -742}\special{pa  1761  -741}%
\special{pa  1761  -740}\special{pa  1760  -738}\special{pa  1761  -737}\special{pa  1761  -735}%
\special{pa  1761  -734}\special{pa  1762  -733}\special{pa  1763  -732}\special{pa  1763  -731}%
\special{pa  1765  -730}\special{pa  1766  -729}\special{pa  1767  -728}\special{pa  1768  -728}%
\special{pa  1770  -727}\special{pa  1771  -727}\special{pa  1772  -727}\special{pa  1774  -727}%
\special{pa  1775  -728}\special{pa  1776  -728}\special{pa  1778  -729}\special{pa  1779  -730}%
\special{pa  1780  -731}\special{pa  1781  -732}\special{pa  1781  -733}\special{pa  1782  -734}%
\special{pa  1783  -735}\special{pa  1783  -737}\special{pa  1783  -738}%
\special{fp}%
\settowidth{\Width}{図2}\setlength{\Width}{-0.5\Width}%
\settoheight{\Height}{図2}\settodepth{\Depth}{図2}\setlength{\Height}{-0.5\Height}\setlength{\Depth}{0.5\Depth}\addtolength{\Height}{\Depth}%
\put(  1.550,  0.050){\hspace*{\Width}\raisebox{\Height}{図2}}%
%
\end{picture}}%
    }
    乾電池Kの両端に可変抵抗を接続し,抵抗値を変えて回路を流れる電流$I$\tanni{A}と乾電池Kの両極間の電位差$V$\tanni{V}を測定したところ,図1のような直線のグラフが得られた。
        \begin{enumerate}
            \item 乾電池Kの起電力$E$を求めよ。
            \item 乾電池Kの内部抵抗$r$を求めよ。
            \item 回路を流れる電流が$1.0$\sftanni{A}のとき,乾電池Kの両極間の電位差を求めよ。
            \item $R$\tanni{\Omega }の抵抗を含む図2のような回路がある。乾電池Kを端子A,Bに接続したところ,$0.20$\sftanni{A}の電流がAを流れた。AB間の3つの抵抗での消費電力の和を求めよ。また,$R$を求めよ。
        \end{enumerate}
    \end{mawarikomi}