\item
    \begin{mawarikomi}{150pt}{
        %%% C:/vpn/vpn/KeTCindy/fig/fig137.tex 
%%% Generator=fig137.cdy 
{\unitlength=1cm%
\begin{picture}%
(7,5)(-3.5,-2.5)%
\special{pn 8}%
%
\special{pa -1181   -39}\special{pa  -787   -39}%
\special{fp}%
\special{pn 16}%
\special{pa -1063    39}\special{pa  -906    39}%
\special{fp}%
\special{pn 8}%
\special{pa  -984   591}\special{pa  -984    39}%
\special{fp}%
\special{pa  -984  -591}\special{pa  -984   -39}%
\special{fp}%
\special{pa  -650  -591}\special{pa  -377  -748}%
\special{fp}%
\special{pa  -984  -591}\special{pa  -650  -591}%
\special{fp}%
\special{pa     0  -591}\special{pa  -335  -591}%
\special{fp}%
{%
\color[cmyk]{0,0,0,0}%
\special{pa -635 -591}\special{pa -635 -592}\special{pa -635 -594}\special{pa -636 -596}%
\special{pa -636 -598}\special{pa -638 -599}\special{pa -639 -601}\special{pa -640 -602}%
\special{pa -642 -603}\special{pa -643 -604}\special{pa -645 -605}\special{pa -647 -605}%
\special{pa -649 -605}\special{pa -651 -605}\special{pa -652 -605}\special{pa -654 -605}%
\special{pa -656 -604}\special{pa -658 -603}\special{pa -659 -602}\special{pa -661 -601}%
\special{pa -662 -599}\special{pa -663 -598}\special{pa -664 -596}\special{pa -664 -594}%
\special{pa -664 -592}\special{pa -665 -591}\special{pa -664 -589}\special{pa -664 -587}%
\special{pa -664 -585}\special{pa -663 -583}\special{pa -662 -582}\special{pa -661 -580}%
\special{pa -659 -579}\special{pa -658 -578}\special{pa -656 -577}\special{pa -654 -576}%
\special{pa -652 -576}\special{pa -651 -576}\special{pa -649 -576}\special{pa -647 -576}%
\special{pa -645 -576}\special{pa -643 -577}\special{pa -642 -578}\special{pa -640 -579}%
\special{pa -639 -580}\special{pa -638 -582}\special{pa -636 -583}\special{pa -636 -585}%
\special{pa -635 -587}\special{pa -635 -589}\special{pa -635 -591}\special{pa -635 -591}%
\special{sh 1}\special{ip}%
}%
\special{pa  -635  -591}\special{pa  -635  -592}\special{pa  -635  -594}\special{pa  -636  -596}%
\special{pa  -636  -598}\special{pa  -638  -599}\special{pa  -639  -601}\special{pa  -640  -602}%
\special{pa  -642  -603}\special{pa  -643  -604}\special{pa  -645  -605}\special{pa  -647  -605}%
\special{pa  -649  -605}\special{pa  -651  -605}\special{pa  -652  -605}\special{pa  -654  -605}%
\special{pa  -656  -604}\special{pa  -658  -603}\special{pa  -659  -602}\special{pa  -661  -601}%
\special{pa  -662  -599}\special{pa  -663  -598}\special{pa  -664  -596}\special{pa  -664  -594}%
\special{pa  -664  -592}\special{pa  -665  -591}\special{pa  -664  -589}\special{pa  -664  -587}%
\special{pa  -664  -585}\special{pa  -663  -583}\special{pa  -662  -582}\special{pa  -661  -580}%
\special{pa  -659  -579}\special{pa  -658  -578}\special{pa  -656  -577}\special{pa  -654  -576}%
\special{pa  -652  -576}\special{pa  -651  -576}\special{pa  -649  -576}\special{pa  -647  -576}%
\special{pa  -645  -576}\special{pa  -643  -577}\special{pa  -642  -578}\special{pa  -640  -579}%
\special{pa  -639  -580}\special{pa  -638  -582}\special{pa  -636  -583}\special{pa  -636  -585}%
\special{pa  -635  -587}\special{pa  -635  -589}\special{pa  -635  -591}%
\special{fp}%
{%
\color[cmyk]{0,0,0,0}%
\special{pa -320 -591}\special{pa -320 -592}\special{pa -320 -594}\special{pa -321 -596}%
\special{pa -322 -598}\special{pa -323 -599}\special{pa -324 -601}\special{pa -325 -602}%
\special{pa -327 -603}\special{pa -328 -604}\special{pa -330 -605}\special{pa -332 -605}%
\special{pa -334 -605}\special{pa -336 -605}\special{pa -337 -605}\special{pa -339 -605}%
\special{pa -341 -604}\special{pa -343 -603}\special{pa -344 -602}\special{pa -346 -601}%
\special{pa -347 -599}\special{pa -348 -598}\special{pa -349 -596}\special{pa -349 -594}%
\special{pa -349 -592}\special{pa -350 -591}\special{pa -349 -589}\special{pa -349 -587}%
\special{pa -349 -585}\special{pa -348 -583}\special{pa -347 -582}\special{pa -346 -580}%
\special{pa -344 -579}\special{pa -343 -578}\special{pa -341 -577}\special{pa -339 -576}%
\special{pa -337 -576}\special{pa -336 -576}\special{pa -334 -576}\special{pa -332 -576}%
\special{pa -330 -576}\special{pa -328 -577}\special{pa -327 -578}\special{pa -325 -579}%
\special{pa -324 -580}\special{pa -323 -582}\special{pa -322 -583}\special{pa -321 -585}%
\special{pa -320 -587}\special{pa -320 -589}\special{pa -320 -591}\special{pa -320 -591}%
\special{sh 1}\special{ip}%
}%
\special{pa  -320  -591}\special{pa  -320  -592}\special{pa  -320  -594}\special{pa  -321  -596}%
\special{pa  -322  -598}\special{pa  -323  -599}\special{pa  -324  -601}\special{pa  -325  -602}%
\special{pa  -327  -603}\special{pa  -328  -604}\special{pa  -330  -605}\special{pa  -332  -605}%
\special{pa  -334  -605}\special{pa  -336  -605}\special{pa  -337  -605}\special{pa  -339  -605}%
\special{pa  -341  -604}\special{pa  -343  -603}\special{pa  -344  -602}\special{pa  -346  -601}%
\special{pa  -347  -599}\special{pa  -348  -598}\special{pa  -349  -596}\special{pa  -349  -594}%
\special{pa  -349  -592}\special{pa  -350  -591}\special{pa  -349  -589}\special{pa  -349  -587}%
\special{pa  -349  -585}\special{pa  -348  -583}\special{pa  -347  -582}\special{pa  -346  -580}%
\special{pa  -344  -579}\special{pa  -343  -578}\special{pa  -341  -577}\special{pa  -339  -576}%
\special{pa  -337  -576}\special{pa  -336  -576}\special{pa  -334  -576}\special{pa  -332  -576}%
\special{pa  -330  -576}\special{pa  -328  -577}\special{pa  -327  -578}\special{pa  -325  -579}%
\special{pa  -324  -580}\special{pa  -323  -582}\special{pa  -322  -583}\special{pa  -321  -585}%
\special{pa  -320  -587}\special{pa  -320  -589}\special{pa  -320  -591}%
\special{fp}%
\special{pa   335  -591}\special{pa   607  -748}%
\special{fp}%
\special{pa     0  -591}\special{pa   335  -591}%
\special{fp}%
\special{pa   984  -591}\special{pa   650  -591}%
\special{fp}%
{%
\color[cmyk]{0,0,0,0}%
\special{pa 350 -591}\special{pa 349 -592}\special{pa 349 -594}\special{pa 349 -596}%
\special{pa 348 -598}\special{pa 347 -599}\special{pa 346 -601}\special{pa 344 -602}%
\special{pa 343 -603}\special{pa 341 -604}\special{pa 339 -605}\special{pa 337 -605}%
\special{pa 336 -605}\special{pa 334 -605}\special{pa 332 -605}\special{pa 330 -605}%
\special{pa 328 -604}\special{pa 327 -603}\special{pa 325 -602}\special{pa 324 -601}%
\special{pa 323 -599}\special{pa 322 -598}\special{pa 321 -596}\special{pa 320 -594}%
\special{pa 320 -592}\special{pa 320 -591}\special{pa 320 -589}\special{pa 320 -587}%
\special{pa 321 -585}\special{pa 322 -583}\special{pa 323 -582}\special{pa 324 -580}%
\special{pa 325 -579}\special{pa 327 -578}\special{pa 328 -577}\special{pa 330 -576}%
\special{pa 332 -576}\special{pa 334 -576}\special{pa 336 -576}\special{pa 337 -576}%
\special{pa 339 -576}\special{pa 341 -577}\special{pa 343 -578}\special{pa 344 -579}%
\special{pa 346 -580}\special{pa 347 -582}\special{pa 348 -583}\special{pa 349 -585}%
\special{pa 349 -587}\special{pa 349 -589}\special{pa 350 -591}\special{pa 350 -591}%
\special{sh 1}\special{ip}%
}%
\special{pa   350  -591}\special{pa   349  -592}\special{pa   349  -594}\special{pa   349  -596}%
\special{pa   348  -598}\special{pa   347  -599}\special{pa   346  -601}\special{pa   344  -602}%
\special{pa   343  -603}\special{pa   341  -604}\special{pa   339  -605}\special{pa   337  -605}%
\special{pa   336  -605}\special{pa   334  -605}\special{pa   332  -605}\special{pa   330  -605}%
\special{pa   328  -604}\special{pa   327  -603}\special{pa   325  -602}\special{pa   324  -601}%
\special{pa   323  -599}\special{pa   322  -598}\special{pa   321  -596}\special{pa   320  -594}%
\special{pa   320  -592}\special{pa   320  -591}\special{pa   320  -589}\special{pa   320  -587}%
\special{pa   321  -585}\special{pa   322  -583}\special{pa   323  -582}\special{pa   324  -580}%
\special{pa   325  -579}\special{pa   327  -578}\special{pa   328  -577}\special{pa   330  -576}%
\special{pa   332  -576}\special{pa   334  -576}\special{pa   336  -576}\special{pa   337  -576}%
\special{pa   339  -576}\special{pa   341  -577}\special{pa   343  -578}\special{pa   344  -579}%
\special{pa   346  -580}\special{pa   347  -582}\special{pa   348  -583}\special{pa   349  -585}%
\special{pa   349  -587}\special{pa   349  -589}\special{pa   350  -591}%
\special{fp}%
{%
\color[cmyk]{0,0,0,0}%
\special{pa 665 -591}\special{pa 664 -592}\special{pa 664 -594}\special{pa 664 -596}%
\special{pa 663 -598}\special{pa 662 -599}\special{pa 661 -601}\special{pa 659 -602}%
\special{pa 658 -603}\special{pa 656 -604}\special{pa 654 -605}\special{pa 652 -605}%
\special{pa 651 -605}\special{pa 649 -605}\special{pa 647 -605}\special{pa 645 -605}%
\special{pa 643 -604}\special{pa 642 -603}\special{pa 640 -602}\special{pa 639 -601}%
\special{pa 638 -599}\special{pa 636 -598}\special{pa 636 -596}\special{pa 635 -594}%
\special{pa 635 -592}\special{pa 635 -591}\special{pa 635 -589}\special{pa 635 -587}%
\special{pa 636 -585}\special{pa 636 -583}\special{pa 638 -582}\special{pa 639 -580}%
\special{pa 640 -579}\special{pa 642 -578}\special{pa 643 -577}\special{pa 645 -576}%
\special{pa 647 -576}\special{pa 649 -576}\special{pa 651 -576}\special{pa 652 -576}%
\special{pa 654 -576}\special{pa 656 -577}\special{pa 658 -578}\special{pa 659 -579}%
\special{pa 661 -580}\special{pa 662 -582}\special{pa 663 -583}\special{pa 664 -585}%
\special{pa 664 -587}\special{pa 664 -589}\special{pa 665 -591}\special{pa 665 -591}%
\special{sh 1}\special{ip}%
}%
\special{pa   665  -591}\special{pa   664  -592}\special{pa   664  -594}\special{pa   664  -596}%
\special{pa   663  -598}\special{pa   662  -599}\special{pa   661  -601}\special{pa   659  -602}%
\special{pa   658  -603}\special{pa   656  -604}\special{pa   654  -605}\special{pa   652  -605}%
\special{pa   651  -605}\special{pa   649  -605}\special{pa   647  -605}\special{pa   645  -605}%
\special{pa   643  -604}\special{pa   642  -603}\special{pa   640  -602}\special{pa   639  -601}%
\special{pa   638  -599}\special{pa   636  -598}\special{pa   636  -596}\special{pa   635  -594}%
\special{pa   635  -592}\special{pa   635  -591}\special{pa   635  -589}\special{pa   635  -587}%
\special{pa   636  -585}\special{pa   636  -583}\special{pa   638  -582}\special{pa   639  -580}%
\special{pa   640  -579}\special{pa   642  -578}\special{pa   643  -577}\special{pa   645  -576}%
\special{pa   647  -576}\special{pa   649  -576}\special{pa   651  -576}\special{pa   652  -576}%
\special{pa   654  -576}\special{pa   656  -577}\special{pa   658  -578}\special{pa   659  -579}%
\special{pa   661  -580}\special{pa   662  -582}\special{pa   663  -583}\special{pa   664  -585}%
\special{pa   664  -587}\special{pa   664  -589}\special{pa   665  -591}%
\special{fp}%
\special{pa  -295   492}\special{pa  -689   492}\special{pa  -689   689}\special{pa  -295   689}%
\special{pa  -295   492}%
\special{fp}%
\special{pa  -984   591}\special{pa  -689   591}%
\special{fp}%
\special{pa     0   591}\special{pa  -295   591}%
\special{fp}%
\special{pa  -197   -59}\special{pa   197   -59}%
\special{fp}%
\special{pa  -197    59}\special{pa   197    59}%
\special{fp}%
\special{pa     0   591}\special{pa     0    59}%
\special{fp}%
\special{pa     0  -591}\special{pa     0   -59}%
\special{fp}%
\special{pa   984  -157}\special{pa   989  -157}\special{pa   994  -157}\special{pa   999  -157}%
\special{pa  1004  -156}\special{pa  1009  -156}\special{pa  1013  -155}\special{pa  1018  -154}%
\special{pa  1022  -153}\special{pa  1026  -151}\special{pa  1031  -150}\special{pa  1034  -148}%
\special{pa  1038  -147}\special{pa  1042  -145}\special{pa  1045  -143}\special{pa  1048  -141}%
\special{pa  1051  -139}\special{pa  1053  -137}\special{pa  1055  -135}\special{pa  1057  -133}%
\special{pa  1059  -130}\special{pa  1061  -128}\special{pa  1062  -125}\special{pa  1062  -123}%
\special{pa  1063  -121}\special{pa  1063  -118}\special{pa  1063  -116}\special{pa  1062  -113}%
\special{pa  1062  -111}\special{pa  1061  -108}\special{pa  1059  -106}\special{pa  1057  -104}%
\special{pa  1055  -101}\special{pa  1053   -99}\special{pa  1051   -97}\special{pa  1048   -95}%
\special{pa  1045   -93}\special{pa  1042   -91}\special{pa  1038   -89}\special{pa  1034   -88}%
\special{pa  1031   -86}\special{pa  1026   -85}\special{pa  1022   -84}\special{pa  1018   -82}%
\special{pa  1013   -82}\special{pa  1009   -81}\special{pa  1004   -80}\special{pa   999   -79}%
\special{pa   994   -79}\special{pa   989   -79}\special{pa   984   -79}\special{pa   989   -79}%
\special{pa   994   -78}\special{pa   999   -78}\special{pa  1004   -78}\special{pa  1009   -77}%
\special{pa  1013   -76}\special{pa  1018   -75}\special{pa  1022   -74}\special{pa  1026   -73}%
\special{pa  1031   -71}\special{pa  1034   -70}\special{pa  1038   -68}\special{pa  1042   -66}%
\special{pa  1045   -64}\special{pa  1048   -63}\special{pa  1051   -60}\special{pa  1053   -58}%
\special{pa  1055   -56}\special{pa  1057   -54}\special{pa  1059   -52}\special{pa  1061   -49}%
\special{pa  1062   -47}\special{pa  1062   -44}\special{pa  1063   -42}\special{pa  1063   -39}%
\special{pa  1063   -37}\special{pa  1062   -34}\special{pa  1062   -32}\special{pa  1061   -30}%
\special{pa  1059   -27}\special{pa  1057   -25}\special{pa  1055   -23}\special{pa  1053   -20}%
\special{pa  1051   -18}\special{pa  1048   -16}\special{pa  1045   -14}\special{pa  1042   -12}%
\special{pa  1038   -11}\special{pa  1034    -9}\special{pa  1031    -8}\special{pa  1026    -6}%
\special{pa  1022    -5}\special{pa  1018    -4}\special{pa  1013    -3}\special{pa  1009    -2}%
\special{pa  1004    -1}\special{pa   999    -1}\special{pa   994    -0}\special{pa   989    -0}%
\special{pa   984    -0}\special{pa   989     0}\special{pa   994     0}\special{pa   999     1}%
\special{pa  1004     1}\special{pa  1009     2}\special{pa  1013     3}\special{pa  1018     4}%
\special{pa  1022     5}\special{pa  1026     6}\special{pa  1031     8}\special{pa  1034     9}%
\special{pa  1038    11}\special{pa  1042    12}\special{pa  1045    14}\special{pa  1048    16}%
\special{pa  1051    18}\special{pa  1053    20}\special{pa  1055    23}\special{pa  1057    25}%
\special{pa  1059    27}\special{pa  1061    30}\special{pa  1062    32}\special{pa  1062    34}%
\special{pa  1063    37}\special{pa  1063    39}\special{pa  1063    42}\special{pa  1062    44}%
\special{pa  1062    47}\special{pa  1061    49}\special{pa  1059    52}\special{pa  1057    54}%
\special{pa  1055    56}\special{pa  1053    58}\special{pa  1051    60}\special{pa  1048    63}%
\special{pa  1045    64}\special{pa  1042    66}\special{pa  1038    68}\special{pa  1034    70}%
\special{pa  1031    71}\special{pa  1026    73}\special{pa  1022    74}\special{pa  1018    75}%
\special{pa  1013    76}\special{pa  1009    77}\special{pa  1004    78}\special{pa   999    78}%
\special{pa   994    78}\special{pa   989    79}\special{pa   984    79}\special{pa   989    79}%
\special{pa   994    79}\special{pa   999    79}\special{pa  1004    80}\special{pa  1009    81}%
\special{pa  1013    82}\special{pa  1018    82}\special{pa  1022    84}\special{pa  1026    85}%
\special{pa  1031    86}\special{pa  1034    88}\special{pa  1038    89}\special{pa  1042    91}%
\special{pa  1045    93}\special{pa  1048    95}\special{pa  1051    97}\special{pa  1053    99}%
\special{pa  1055   101}\special{pa  1057   104}\special{pa  1059   106}\special{pa  1061   108}%
\special{pa  1062   111}\special{pa  1062   113}\special{pa  1063   116}\special{pa  1063   118}%
\special{pa  1063   121}\special{pa  1062   123}\special{pa  1062   125}\special{pa  1061   128}%
\special{pa  1059   130}\special{pa  1057   133}\special{pa  1055   135}\special{pa  1053   137}%
\special{pa  1051   139}\special{pa  1048   141}\special{pa  1045   143}\special{pa  1042   145}%
\special{pa  1038   147}\special{pa  1034   148}\special{pa  1031   150}\special{pa  1026   151}%
\special{pa  1022   153}\special{pa  1018   154}\special{pa  1013   155}\special{pa  1009   156}%
\special{pa  1004   156}\special{pa   999   157}\special{pa   994   157}\special{pa   989   157}%
\special{pa   984   157}%
\special{fp}%
\special{pa   984  -591}\special{pa   984  -157}%
\special{fp}%
\special{pa   984   591}\special{pa   984   157}%
\special{fp}%
\special{pa     0   591}\special{pa   984   591}%
\special{fp}%
\settowidth{\Width}{$R$}\setlength{\Width}{-0.5\Width}%
\settoheight{\Height}{$R$}\settodepth{\Depth}{$R$}\setlength{\Height}{\Depth}%
\put( -1.250, -0.950){\hspace*{\Width}\raisebox{\Height}{$R$}}%
%
\settowidth{\Width}{$C$}\setlength{\Width}{-1\Width}%
\settoheight{\Height}{$C$}\settodepth{\Depth}{$C$}\setlength{\Height}{-0.5\Height}\setlength{\Depth}{0.5\Depth}\addtolength{\Height}{\Depth}%
\put( -0.750,  0.000){\hspace*{\Width}\raisebox{\Height}{$C$}}%
%
\settowidth{\Width}{$L$}\setlength{\Width}{-1\Width}%
\settoheight{\Height}{$L$}\settodepth{\Depth}{$L$}\setlength{\Height}{-0.5\Height}\setlength{\Depth}{0.5\Depth}\addtolength{\Height}{\Depth}%
\put(  2.350,  0.000){\hspace*{\Width}\raisebox{\Height}{$L$}}%
%
\settowidth{\Width}{$V$}\setlength{\Width}{-0.5\Width}%
\settoheight{\Height}{$V$}\settodepth{\Depth}{$V$}\setlength{\Height}{\Depth}%
\put( -3.200,  0.150){\hspace*{\Width}\raisebox{\Height}{$V$}}%
%
\settowidth{\Width}{S$_1$}\setlength{\Width}{-1\Width}%
\settoheight{\Height}{S$_1$}\settodepth{\Depth}{S$_1$}\setlength{\Height}{\Depth}%
\put( -1.450,  1.650){\hspace*{\Width}\raisebox{\Height}{S$_1$}}%
%
\settowidth{\Width}{S$_2$}\setlength{\Width}{-1\Width}%
\settoheight{\Height}{S$_2$}\settodepth{\Depth}{S$_2$}\setlength{\Height}{\Depth}%
\put(  1.050,  1.650){\hspace*{\Width}\raisebox{\Height}{S$_2$}}%
%
\settowidth{\Width}{A}\setlength{\Width}{-0.5\Width}%
\settoheight{\Height}{A}\settodepth{\Depth}{A}\setlength{\Height}{-0.5\Height}\setlength{\Depth}{0.5\Depth}\addtolength{\Height}{\Depth}%
\put(  0.500,  0.500){\hspace*{\Width}\raisebox{\Height}{A}}%
%
\settowidth{\Width}{B}\setlength{\Width}{-0.5\Width}%
\settoheight{\Height}{B}\settodepth{\Depth}{B}\setlength{\Height}{-0.5\Height}\setlength{\Depth}{0.5\Depth}\addtolength{\Height}{\Depth}%
\put(  0.500, -0.500){\hspace*{\Width}\raisebox{\Height}{B}}%
%
\special{pa 15 -591}\special{pa 15 -592}\special{pa 14 -594}\special{pa 14 -596}\special{pa 13 -598}%
\special{pa 12 -599}\special{pa 11 -601}\special{pa 10 -602}\special{pa 8 -603}\special{pa 6 -604}%
\special{pa 5 -605}\special{pa 3 -605}\special{pa 1 -605}\special{pa -1 -605}\special{pa -3 -605}%
\special{pa -5 -605}\special{pa -6 -604}\special{pa -8 -603}\special{pa -10 -602}%
\special{pa -11 -601}\special{pa -12 -599}\special{pa -13 -598}\special{pa -14 -596}%
\special{pa -14 -594}\special{pa -15 -592}\special{pa -15 -591}\special{pa -15 -589}%
\special{pa -14 -587}\special{pa -14 -585}\special{pa -13 -583}\special{pa -12 -582}%
\special{pa -11 -580}\special{pa -10 -579}\special{pa -8 -578}\special{pa -6 -577}%
\special{pa -5 -576}\special{pa -3 -576}\special{pa -1 -576}\special{pa 1 -576}\special{pa 3 -576}%
\special{pa 5 -576}\special{pa 6 -577}\special{pa 8 -578}\special{pa 10 -579}\special{pa 11 -580}%
\special{pa 12 -582}\special{pa 13 -583}\special{pa 14 -585}\special{pa 14 -587}\special{pa 15 -589}%
\special{pa 15 -591}\special{pa 15 -591}\special{sh 1}\special{ip}%
\special{pa    15  -591}\special{pa    15  -592}\special{pa    14  -594}\special{pa    14  -596}%
\special{pa    13  -598}\special{pa    12  -599}\special{pa    11  -601}\special{pa    10  -602}%
\special{pa     8  -603}\special{pa     6  -604}\special{pa     5  -605}\special{pa     3  -605}%
\special{pa     1  -605}\special{pa    -1  -605}\special{pa    -3  -605}\special{pa    -5  -605}%
\special{pa    -6  -604}\special{pa    -8  -603}\special{pa   -10  -602}\special{pa   -11  -601}%
\special{pa   -12  -599}\special{pa   -13  -598}\special{pa   -14  -596}\special{pa   -14  -594}%
\special{pa   -15  -592}\special{pa   -15  -591}\special{pa   -15  -589}\special{pa   -14  -587}%
\special{pa   -14  -585}\special{pa   -13  -583}\special{pa   -12  -582}\special{pa   -11  -580}%
\special{pa   -10  -579}\special{pa    -8  -578}\special{pa    -6  -577}\special{pa    -5  -576}%
\special{pa    -3  -576}\special{pa    -1  -576}\special{pa     1  -576}\special{pa     3  -576}%
\special{pa     5  -576}\special{pa     6  -577}\special{pa     8  -578}\special{pa    10  -579}%
\special{pa    11  -580}\special{pa    12  -582}\special{pa    13  -583}\special{pa    14  -585}%
\special{pa    14  -587}\special{pa    15  -589}\special{pa    15  -591}%
\special{fp}%
\special{pa 15 591}\special{pa 15 589}\special{pa 14 587}\special{pa 14 585}\special{pa 13 583}%
\special{pa 12 582}\special{pa 11 580}\special{pa 10 579}\special{pa 8 578}\special{pa 6 577}%
\special{pa 5 576}\special{pa 3 576}\special{pa 1 576}\special{pa -1 576}\special{pa -3 576}%
\special{pa -5 576}\special{pa -6 577}\special{pa -8 578}\special{pa -10 579}\special{pa -11 580}%
\special{pa -12 582}\special{pa -13 583}\special{pa -14 585}\special{pa -14 587}\special{pa -15 589}%
\special{pa -15 591}\special{pa -15 592}\special{pa -14 594}\special{pa -14 596}\special{pa -13 598}%
\special{pa -12 599}\special{pa -11 601}\special{pa -10 602}\special{pa -8 603}\special{pa -6 604}%
\special{pa -5 605}\special{pa -3 605}\special{pa -1 605}\special{pa 1 605}\special{pa 3 605}%
\special{pa 5 605}\special{pa 6 604}\special{pa 8 603}\special{pa 10 602}\special{pa 11 601}%
\special{pa 12 599}\special{pa 13 598}\special{pa 14 596}\special{pa 14 594}\special{pa 15 592}%
\special{pa 15 591}\special{pa 15 591}\special{sh 1}\special{ip}%
\special{pa    15   591}\special{pa    15   589}\special{pa    14   587}\special{pa    14   585}%
\special{pa    13   583}\special{pa    12   582}\special{pa    11   580}\special{pa    10   579}%
\special{pa     8   578}\special{pa     6   577}\special{pa     5   576}\special{pa     3   576}%
\special{pa     1   576}\special{pa    -1   576}\special{pa    -3   576}\special{pa    -5   576}%
\special{pa    -6   577}\special{pa    -8   578}\special{pa   -10   579}\special{pa   -11   580}%
\special{pa   -12   582}\special{pa   -13   583}\special{pa   -14   585}\special{pa   -14   587}%
\special{pa   -15   589}\special{pa   -15   591}\special{pa   -15   592}\special{pa   -14   594}%
\special{pa   -14   596}\special{pa   -13   598}\special{pa   -12   599}\special{pa   -11   601}%
\special{pa   -10   602}\special{pa    -8   603}\special{pa    -6   604}\special{pa    -5   605}%
\special{pa    -3   605}\special{pa    -1   605}\special{pa     1   605}\special{pa     3   605}%
\special{pa     5   605}\special{pa     6   604}\special{pa     8   603}\special{pa    10   602}%
\special{pa    11   601}\special{pa    12   599}\special{pa    13   598}\special{pa    14   596}%
\special{pa    14   594}\special{pa    15   592}\special{pa    15   591}%
\special{fp}%
\end{picture}}%
    }
    電池(起電力$V$), 抵抗(抵抗値$R$), コンデンサー(容量$C$), コイル(自己インダクタンス$L$), スイッチS$_1$, S$_2$からなる回路があり,最初S$_1$, S$_2$は開いている。電池やコイルなどの内部抵抗は無視する。
    \begin{enumerate}
        \item S$_1$を閉じる。
        \begin{enumerate}[(ア)]
            \item 閉じた直後に抵抗に流れる電流$I_0$を求めよ。
            \item 電流が$I$ ($0 \leqq I \leqq I_0$)になったとき,コンデンサーに蓄えられた電気量$q$を求めよ。
            \item 十分時間が経過した後,コンデンサーに蓄えられる電気量$Q$を求めよ。
        \end{enumerate}
        \item S$_1$を閉じて十分時間が経過した後,S$_1$を開き,次にS$_2$を閉じる。
        \begin{enumerate}[(ア)]
            \item 回路を流れる振動電流$i$の最大値$i_\mathrm{m}$を求めよ。
            \item S$_2$を閉じた直後からの$i$の時間変化を図示せよ。ただし,$i$は時計回りの向きを正とする。
            \item S$_2$を閉じてから,コンデンサーの下側極板Bの電荷が正で最大となるまでにかかる時間を求めよ。
        \end{enumerate}
    \end{enumerate}
    \end{mawarikomi}
