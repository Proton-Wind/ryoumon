\hakosyokika
\item
    \begin{mawarikomi}{150pt}{\begin{zahyou*}[ul=5mm](-6,4)(-1,7)
    \small
    \def\O{(0,6)}
    \def\OD{(0,3)}
    \def\B{(0,0)}
    \def\M{(-5,\yii)}
    \Kaiten\OD\B{140}\C
    \Kaiten\OD\B{90}\CD
    \Kaiten\O\B{-90}\A
    \Drawline{\O\A}
    \Drawline{\O\B}
    \Drawline{\OD\C}
    \Hasen{\OD\CD}
    \Enko<hasen=[0.2][0.2]>\O{6}{180}{270}
    \Enko<hasen=[0.4][0.4]>\OD{3}{270}{360}
    \Enko<hasen=[0.4][0.4]>\OD{3}{0}{50}
    \Kakukigou[b]\CD\OD\C<Hankei=5mm>(0.8pt,2pt)[l]{$\theta _0$}
    \Kuromaru\OD
    \HenKo<henkotype=parallel,
    % henkoH=11ex,
    yazirusi=b,
    henkosideb=0.5,
    henkosidet=1.5>\O\A{$\ell $}
    \HenKo<henkotype=parallel,
    % henkoH=11ex,
    yazirusi=b,
    henkosideb=0.5,
    henkosidet=1.5>\O\OD{$\bunsuu{\ell }{2}$}
    \Put\B(0pt,-5pt)[t]{B}
    \Put\C[ne]{C}
    \Put\O[ne]{O}
    \Put\OD[se]{P}
    \Put\A[w]{A}
    \En*\A{0.15}
    \En\A{0.15}
    \En*\B{0.15}
    \En\B{0.15}
    \En*\C{0.15}
    \En\C{0.15}
\end{zahyou*}
}
        質量$m$の質点をつけた長さ$\ell $の位置の端を点Oにとめ,糸をぴんと張り質点が点Oと同じ高さの点Aにくるようにした。質点を静かに放すと,OAを含む鉛直面(紙面)内で運動する。細いなめらかな棒が点Oから鉛直下方$\bunsuu{\ell }{2}$の距離にある点Pで,この鉛直面と垂直に交わるように固定されている。重力加速度の大きさを$g$とする。
        \begin{enumerate}
            \item 質点が点Oの鉛直下方にある点Bを通過するときの速さ$v_0$を求めよ。
            \item 質点が点Bを通過する直前の糸の張力$T_1$と,通過した直後の張力$T_2$を求めよ。
            \item 質点が点Cにきたとき,糸がゆるみ始めた。その時の速さ$v$を求めよ。また,PCが水平となす角を$\theta _0$として$\sin{\theta _0}$を求めよ。
            \item その後,質点は点Cからどれだけの高さまで上がるか。
            \item 点Aで質点に鉛直下向きの初速を与えれば,質点は点Oに達する。必要な初速$u$を求めよ。
        \end{enumerate}
    \end{mawarikomi}