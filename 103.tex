\hakosyokika
\item
    \begin{mawarikomi}(20pt,0pt){150pt}{%%% C:/Users/yuich/ketcindy2025Apr19/fig/fig103.tex 
%%% Generator=fig103.cdy 
{\unitlength=0.7cm%
\begin{picture}%
(10,10)(-5,-5)%
\special{pn 8}%
%
\special{pa 201 -802}\special{pa 276 -827}\special{pa 201 -851}\special{pa 216 -827}%
\special{pa 201 -802}\special{pa 201 -802}\special{sh 1}\special{ip}%
\special{pn 1}%
\special{pa   201  -802}\special{pa   276  -827}\special{pa   201  -851}\special{pa   216  -827}%
\special{pa   201  -802}%
\special{fp}%
\special{pn 8}%
\special{pa  -276  -827}\special{pa   216  -827}%
\special{fp}%
\special{pa -816 551}\special{pa -816 550}\special{pa -817 549}\special{pa -817 547}%
\special{pa -818 546}\special{pa -818 545}\special{pa -819 544}\special{pa -820 543}%
\special{pa -821 542}\special{pa -822 542}\special{pa -824 541}\special{pa -825 541}%
\special{pa -826 541}\special{pa -827 541}\special{pa -829 541}\special{pa -830 541}%
\special{pa -831 542}\special{pa -832 542}\special{pa -833 543}\special{pa -834 544}%
\special{pa -835 545}\special{pa -836 546}\special{pa -837 547}\special{pa -837 549}%
\special{pa -837 550}\special{pa -837 551}\special{pa -837 552}\special{pa -837 554}%
\special{pa -837 555}\special{pa -836 556}\special{pa -835 557}\special{pa -834 558}%
\special{pa -833 559}\special{pa -832 560}\special{pa -831 561}\special{pa -830 561}%
\special{pa -829 561}\special{pa -827 562}\special{pa -826 562}\special{pa -825 561}%
\special{pa -824 561}\special{pa -822 561}\special{pa -821 560}\special{pa -820 559}%
\special{pa -819 558}\special{pa -818 557}\special{pa -818 556}\special{pa -817 555}%
\special{pa -817 554}\special{pa -816 552}\special{pa -816 551}\special{pa -816 551}%
\special{sh 1}\special{ip}%
\special{pa  -816   551}\special{pa  -816   550}\special{pa  -817   549}\special{pa  -817   547}%
\special{pa  -818   546}\special{pa  -818   545}\special{pa  -819   544}\special{pa  -820   543}%
\special{pa  -821   542}\special{pa  -822   542}\special{pa  -824   541}\special{pa  -825   541}%
\special{pa  -826   541}\special{pa  -827   541}\special{pa  -829   541}\special{pa  -830   541}%
\special{pa  -831   542}\special{pa  -832   542}\special{pa  -833   543}\special{pa  -834   544}%
\special{pa  -835   545}\special{pa  -836   546}\special{pa  -837   547}\special{pa  -837   549}%
\special{pa  -837   550}\special{pa  -837   551}\special{pa  -837   552}\special{pa  -837   554}%
\special{pa  -837   555}\special{pa  -836   556}\special{pa  -835   557}\special{pa  -834   558}%
\special{pa  -833   559}\special{pa  -832   560}\special{pa  -831   561}\special{pa  -830   561}%
\special{pa  -829   561}\special{pa  -827   562}\special{pa  -826   562}\special{pa  -825   561}%
\special{pa  -824   561}\special{pa  -822   561}\special{pa  -821   560}\special{pa  -820   559}%
\special{pa  -819   558}\special{pa  -818   557}\special{pa  -818   556}\special{pa  -817   555}%
\special{pa  -817   554}\special{pa  -816   552}\special{pa  -816   551}%
\special{fp}%
\special{pa 837 551}\special{pa 837 550}\special{pa 837 549}\special{pa 837 547}\special{pa 836 546}%
\special{pa 835 545}\special{pa 834 544}\special{pa 833 543}\special{pa 832 542}\special{pa 831 542}%
\special{pa 830 541}\special{pa 829 541}\special{pa 827 541}\special{pa 826 541}\special{pa 825 541}%
\special{pa 824 541}\special{pa 822 542}\special{pa 821 542}\special{pa 820 543}\special{pa 819 544}%
\special{pa 818 545}\special{pa 818 546}\special{pa 817 547}\special{pa 817 549}\special{pa 816 550}%
\special{pa 816 551}\special{pa 816 552}\special{pa 817 554}\special{pa 817 555}\special{pa 818 556}%
\special{pa 818 557}\special{pa 819 558}\special{pa 820 559}\special{pa 821 560}\special{pa 822 561}%
\special{pa 824 561}\special{pa 825 561}\special{pa 826 562}\special{pa 827 562}\special{pa 829 561}%
\special{pa 830 561}\special{pa 831 561}\special{pa 832 560}\special{pa 833 559}\special{pa 834 558}%
\special{pa 835 557}\special{pa 836 556}\special{pa 837 555}\special{pa 837 554}\special{pa 837 552}%
\special{pa 837 551}\special{pa 837 551}\special{sh 1}\special{ip}%
\special{pa   837   551}\special{pa   837   550}\special{pa   837   549}\special{pa   837   547}%
\special{pa   836   546}\special{pa   835   545}\special{pa   834   544}\special{pa   833   543}%
\special{pa   832   542}\special{pa   831   542}\special{pa   830   541}\special{pa   829   541}%
\special{pa   827   541}\special{pa   826   541}\special{pa   825   541}\special{pa   824   541}%
\special{pa   822   542}\special{pa   821   542}\special{pa   820   543}\special{pa   819   544}%
\special{pa   818   545}\special{pa   818   546}\special{pa   817   547}\special{pa   817   549}%
\special{pa   816   550}\special{pa   816   551}\special{pa   816   552}\special{pa   817   554}%
\special{pa   817   555}\special{pa   818   556}\special{pa   818   557}\special{pa   819   558}%
\special{pa   820   559}\special{pa   821   560}\special{pa   822   561}\special{pa   824   561}%
\special{pa   825   561}\special{pa   826   562}\special{pa   827   562}\special{pa   829   561}%
\special{pa   830   561}\special{pa   831   561}\special{pa   832   560}\special{pa   833   559}%
\special{pa   834   558}\special{pa   835   557}\special{pa   836   556}\special{pa   837   555}%
\special{pa   837   554}\special{pa   837   552}\special{pa   837   551}%
\special{fp}%
\special{pa 837 -551}\special{pa 837 -552}\special{pa 837 -554}\special{pa 837 -555}%
\special{pa 836 -556}\special{pa 835 -557}\special{pa 834 -558}\special{pa 833 -559}%
\special{pa 832 -560}\special{pa 831 -561}\special{pa 830 -561}\special{pa 829 -561}%
\special{pa 827 -562}\special{pa 826 -562}\special{pa 825 -561}\special{pa 824 -561}%
\special{pa 822 -561}\special{pa 821 -560}\special{pa 820 -559}\special{pa 819 -558}%
\special{pa 818 -557}\special{pa 818 -556}\special{pa 817 -555}\special{pa 817 -554}%
\special{pa 816 -552}\special{pa 816 -551}\special{pa 816 -550}\special{pa 817 -549}%
\special{pa 817 -547}\special{pa 818 -546}\special{pa 818 -545}\special{pa 819 -544}%
\special{pa 820 -543}\special{pa 821 -542}\special{pa 822 -542}\special{pa 824 -541}%
\special{pa 825 -541}\special{pa 826 -541}\special{pa 827 -541}\special{pa 829 -541}%
\special{pa 830 -541}\special{pa 831 -542}\special{pa 832 -542}\special{pa 833 -543}%
\special{pa 834 -544}\special{pa 835 -545}\special{pa 836 -546}\special{pa 837 -547}%
\special{pa 837 -549}\special{pa 837 -550}\special{pa 837 -551}\special{pa 837 -551}%
\special{sh 1}\special{ip}%
\special{pa   837  -551}\special{pa   837  -552}\special{pa   837  -554}\special{pa   837  -555}%
\special{pa   836  -556}\special{pa   835  -557}\special{pa   834  -558}\special{pa   833  -559}%
\special{pa   832  -560}\special{pa   831  -561}\special{pa   830  -561}\special{pa   829  -561}%
\special{pa   827  -562}\special{pa   826  -562}\special{pa   825  -561}\special{pa   824  -561}%
\special{pa   822  -561}\special{pa   821  -560}\special{pa   820  -559}\special{pa   819  -558}%
\special{pa   818  -557}\special{pa   818  -556}\special{pa   817  -555}\special{pa   817  -554}%
\special{pa   816  -552}\special{pa   816  -551}\special{pa   816  -550}\special{pa   817  -549}%
\special{pa   817  -547}\special{pa   818  -546}\special{pa   818  -545}\special{pa   819  -544}%
\special{pa   820  -543}\special{pa   821  -542}\special{pa   822  -542}\special{pa   824  -541}%
\special{pa   825  -541}\special{pa   826  -541}\special{pa   827  -541}\special{pa   829  -541}%
\special{pa   830  -541}\special{pa   831  -542}\special{pa   832  -542}\special{pa   833  -543}%
\special{pa   834  -544}\special{pa   835  -545}\special{pa   836  -546}\special{pa   837  -547}%
\special{pa   837  -549}\special{pa   837  -550}\special{pa   837  -551}%
\special{fp}%
\special{pa -827 1102}\special{pa -827 1024}\special{fp}\special{pa -827 984}\special{pa -827 906}\special{fp}%
\special{pa -827 866}\special{pa -827 787}\special{fp}\special{pa -827 748}\special{pa -827 669}\special{fp}%
\special{pa -827 630}\special{pa -827 551}\special{fp}\special{pa -827 512}\special{pa -827 433}\special{fp}%
\special{pa -827 394}\special{pa -827 315}\special{fp}\special{pa -827 276}\special{pa -827 197}\special{fp}%
\special{pa -827 157}\special{pa -827 79}\special{fp}\special{pa -827 39}\special{pa -827 -39}\special{fp}%
\special{pa -827 -79}\special{pa -827 -157}\special{fp}\special{pa -827 -197}\special{pa -827 -276}\special{fp}%
\special{pa -827 -315}\special{pa -827 -394}\special{fp}\special{pa -827 -433}\special{pa -827 -512}\special{fp}%
\special{pa -827 -551}\special{pa -827 -630}\special{fp}\special{pa -827 -669}\special{pa -827 -748}\special{fp}%
\special{pa -827 -787}\special{pa -827 -866}\special{fp}\special{pa -827 -906}\special{pa -827 -984}\special{fp}%
\special{pa -827 -1024}\special{pa -827 -1102}\special{fp}%
%
\special{pa 827 1102}\special{pa 827 1024}\special{fp}\special{pa 827 984}\special{pa 827 906}\special{fp}%
\special{pa 827 866}\special{pa 827 787}\special{fp}\special{pa 827 748}\special{pa 827 669}\special{fp}%
\special{pa 827 630}\special{pa 827 551}\special{fp}\special{pa 827 512}\special{pa 827 433}\special{fp}%
\special{pa 827 394}\special{pa 827 315}\special{fp}\special{pa 827 276}\special{pa 827 197}\special{fp}%
\special{pa 827 157}\special{pa 827 79}\special{fp}\special{pa 827 39}\special{pa 827 -39}\special{fp}%
\special{pa 827 -79}\special{pa 827 -157}\special{fp}\special{pa 827 -197}\special{pa 827 -276}\special{fp}%
\special{pa 827 -315}\special{pa 827 -394}\special{fp}\special{pa 827 -433}\special{pa 827 -512}\special{fp}%
\special{pa 827 -551}\special{pa 827 -630}\special{fp}\special{pa 827 -669}\special{pa 827 -748}\special{fp}%
\special{pa 827 -787}\special{pa 827 -866}\special{fp}\special{pa 827 -906}\special{pa 827 -984}\special{fp}%
\special{pa 827 -1024}\special{pa 827 -1102}\special{fp}%
%
\special{pa 827 -551}\special{pa 794 -529}\special{fp}\special{pa 762 -508}\special{pa 729 -486}\special{fp}%
\special{pa 696 -464}\special{pa 664 -443}\special{fp}\special{pa 631 -421}\special{pa 599 -399}\special{fp}%
\special{pa 566 -377}\special{pa 534 -356}\special{fp}\special{pa 501 -334}\special{pa 468 -312}\special{fp}%
\special{pa 436 -291}\special{pa 403 -269}\special{fp}\special{pa 371 -247}\special{pa 338 -225}\special{fp}%
\special{pa 306 -204}\special{pa 273 -182}\special{fp}\special{pa 240 -160}\special{pa 208 -139}\special{fp}%
\special{pa 175 -117}\special{pa 143 -95}\special{fp}\special{pa 110 -73}\special{pa 78 -52}\special{fp}%
\special{pa 45 -30}\special{pa 12 -8}\special{fp}\special{pa -20 13}\special{pa -53 35}\special{fp}%
\special{pa -85 57}\special{pa -118 79}\special{fp}\special{pa -150 100}\special{pa -183 122}\special{fp}%
\special{pa -216 144}\special{pa -248 165}\special{fp}\special{pa -281 187}\special{pa -313 209}\special{fp}%
\special{pa -346 231}\special{pa -378 252}\special{fp}\special{pa -411 274}\special{pa -444 296}\special{fp}%
\special{pa -476 317}\special{pa -509 339}\special{fp}\special{pa -541 361}\special{pa -574 383}\special{fp}%
\special{pa -606 404}\special{pa -639 426}\special{fp}\special{pa -672 448}\special{pa -704 469}\special{fp}%
\special{pa -737 491}\special{pa -769 513}\special{fp}\special{pa -802 535}\special{pa -827 551}\special{pa -817 551}\special{fp}%
\special{pa -778 551}\special{pa -739 551}\special{fp}\special{pa -700 551}\special{pa -661 551}\special{fp}%
\special{pa -622 551}\special{pa -583 551}\special{fp}\special{pa -543 551}\special{pa -504 551}\special{fp}%
\special{pa -465 551}\special{pa -426 551}\special{fp}\special{pa -387 551}\special{pa -348 551}\special{fp}%
\special{pa -309 551}\special{pa -269 551}\special{fp}\special{pa -230 551}\special{pa -191 551}\special{fp}%
\special{pa -152 551}\special{pa -113 551}\special{fp}\special{pa -74 551}\special{pa -35 551}\special{fp}%
\special{pa 5 551}\special{pa 44 551}\special{fp}\special{pa 83 551}\special{pa 122 551}\special{fp}%
\special{pa 161 551}\special{pa 200 551}\special{fp}\special{pa 240 551}\special{pa 279 551}\special{fp}%
\special{pa 318 551}\special{pa 357 551}\special{fp}\special{pa 396 551}\special{pa 435 551}\special{fp}%
\special{pa 474 551}\special{pa 514 551}\special{fp}\special{pa 553 551}\special{pa 592 551}\special{fp}%
\special{pa 631 551}\special{pa 670 551}\special{fp}\special{pa 709 551}\special{pa 748 551}\special{fp}%
\special{pa 788 551}\special{pa 827 551}\special{fp}%
%
\settowidth{\Width}{$E$}\setlength{\Width}{0\Width}%
\settoheight{\Height}{$E$}\settodepth{\Depth}{$E$}\setlength{\Height}{-0.5\Height}\setlength{\Depth}{0.5\Depth}\addtolength{\Height}{\Depth}%
\put(  1.214,  3.000){\hspace*{\Width}\raisebox{\Height}{$E$}}%
%
\special{pa  -551   551}\special{pa  -553   517}\special{pa  -560   483}\special{pa  -571   450}%
\special{pa  -585   418}\special{pa  -598   399}%
\special{fp}%
\settowidth{\Width}{$\theta$}\setlength{\Width}{-0.5\Width}%
\settoheight{\Height}{$\theta$}\settodepth{\Depth}{$\theta$}\setlength{\Height}{-0.5\Height}\setlength{\Depth}{0.5\Depth}\addtolength{\Height}{\Depth}%
\put( -1.620, -1.580){\hspace*{\Width}\raisebox{\Height}{$\theta$}}%
%
\settowidth{\Width}{A}\setlength{\Width}{-0.5\Width}%
\settoheight{\Height}{A}\settodepth{\Depth}{A}\setlength{\Height}{-\Height}%
\put( -3.000, -4.214){\hspace*{\Width}\raisebox{\Height}{A}}%
%
\settowidth{\Width}{B}\setlength{\Width}{-0.5\Width}%
\settoheight{\Height}{B}\settodepth{\Depth}{B}\setlength{\Height}{-\Height}%
\put(  3.000, -4.214){\hspace*{\Width}\raisebox{\Height}{B}}%
%
\settowidth{\Width}{P}\setlength{\Width}{-1\Width}%
\settoheight{\Height}{P}\settodepth{\Depth}{P}\setlength{\Height}{-0.5\Height}\setlength{\Depth}{0.5\Depth}\addtolength{\Height}{\Depth}%
\put( -3.214, -2.000){\hspace*{\Width}\raisebox{\Height}{P}}%
%
\settowidth{\Width}{Q}\setlength{\Width}{0\Width}%
\settoheight{\Height}{Q}\settodepth{\Depth}{Q}\setlength{\Height}{-0.5\Height}\setlength{\Depth}{0.5\Depth}\addtolength{\Height}{\Depth}%
\put(  3.214, -2.000){\hspace*{\Width}\raisebox{\Height}{Q}}%
%
\settowidth{\Width}{R}\setlength{\Width}{0\Width}%
\settoheight{\Height}{R}\settodepth{\Depth}{R}\setlength{\Height}{-0.5\Height}\setlength{\Depth}{0.5\Depth}\addtolength{\Height}{\Depth}%
\put(  3.214,  2.000){\hspace*{\Width}\raisebox{\Height}{R}}%
%
\special{pa   827  -551}\special{pa   809  -548}\special{pa   792  -544}\special{pa   775  -540}%
\special{pa   758  -537}\special{pa   741  -533}\special{pa   723  -529}\special{pa   706  -525}%
\special{pa   689  -520}\special{pa   672  -516}\special{pa   655  -511}\special{pa   638  -507}%
\special{pa   621  -502}\special{pa   604  -497}\special{pa   587  -492}\special{pa   570  -487}%
\special{pa   553  -482}\special{pa   536  -477}\special{pa   520  -471}\special{pa   503  -466}%
\special{pa   486  -460}\special{pa   470  -454}\special{pa   453  -448}\special{pa   436  -442}%
\special{pa   420  -436}\special{pa   403  -430}\special{pa   387  -424}\special{pa   370  -417}%
\special{pa   354  -411}\special{pa   338  -404}\special{pa   321  -397}\special{pa   305  -390}%
\special{pa   289  -383}\special{pa   273  -376}\special{pa   257  -369}\special{pa   241  -362}%
\special{pa   225  -354}\special{pa   209  -347}\special{pa   193  -339}\special{pa   177  -331}%
\special{pa   161  -323}\special{pa   145  -315}\special{pa   130  -307}\special{pa   114  -299}%
\special{pa    99  -291}\special{pa    83  -282}\special{pa    68  -274}\special{pa    52  -265}%
\special{pa    37  -256}\special{pa    22  -248}\special{pa     6  -239}%
\special{fp}%
\special{pa  -223   -86}\special{pa  -237   -75}\special{pa  -251   -65}\special{pa  -265   -54}%
\special{pa  -279   -43}\special{pa  -293   -32}\special{pa  -306   -21}\special{pa  -320   -10}%
\special{pa  -334     2}\special{pa  -347    13}\special{pa  -360    24}\special{pa  -374    36}%
\special{pa  -387    48}\special{pa  -400    59}\special{pa  -413    71}\special{pa  -426    83}%
\special{pa  -439    95}\special{pa  -452   107}\special{pa  -465   119}\special{pa  -478   131}%
\special{pa  -490   144}\special{pa  -503   156}\special{pa  -515   169}\special{pa  -528   181}%
\special{pa  -540   194}\special{pa  -552   207}\special{pa  -564   220}\special{pa  -576   233}%
\special{pa  -588   246}\special{pa  -600   259}\special{pa  -612   272}\special{pa  -623   285}%
\special{pa  -635   298}\special{pa  -646   312}\special{pa  -658   325}\special{pa  -669   339}%
\special{pa  -680   353}\special{pa  -691   366}\special{pa  -702   380}\special{pa  -713   394}%
\special{pa  -724   408}\special{pa  -735   422}\special{pa  -745   436}\special{pa  -756   450}%
\special{pa  -766   464}\special{pa  -777   479}\special{pa  -787   493}\special{pa  -797   507}%
\special{pa  -807   522}\special{pa  -817   537}\special{pa  -827   551}%
\special{fp}%
\settowidth{\Width}{$\ell$}\setlength{\Width}{-0.5\Width}%
\settoheight{\Height}{$\ell$}\settodepth{\Depth}{$\ell$}\setlength{\Height}{-0.5\Height}\setlength{\Depth}{0.5\Depth}\addtolength{\Height}{\Depth}%
\put( -0.400,  0.600){\hspace*{\Width}\raisebox{\Height}{$\ell$}}%
%
\end{picture}}%}
        水平右向きに一様な電場$E$\tanni{V/m}がかけられ,面AとBは鉛直面で,電場に垂直である。
        図vの3点P,Q,Rを考える。PQは水平で,角$\theta $をなすPRの長さを$\ell $\tanni{m},重力加速度を$g$\tanni{m/s^2}と」する。
        質量$m$\tanni{kg},電荷$q$\tanni{C}$(q>0)$をもつ荷電粒子Mについて考える。
        \begin{enumerate}
            \item PとRではどちらの電位が高いか。また,PR間の電位差を求めよ。
            \item 荷電粒子MをQ点からR点を経てP点へ移すとき,静電気力のする仕事を求めよ。
            \item MをP点で静かに放すとき,面Bに達するのに要する時間はいくらか。また,このときの運動の軌跡はどのようになるか簡潔に述べよ。
            \item MをP点で鉛直上向きに発射するとき,Q点に達するのに必要な初速度$v_0$を求めよ。
        \end{enumerate}
    \end{mawarikomi}
