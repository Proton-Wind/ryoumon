\hakosyokika
\item
    \begin{mawarikomi}(20pt,0pt){150pt}{
        %%% C:/vpn/vpn/KeTCindy/fig/fig107.tex 
%%% Generator=fig107.cdy 
{\unitlength=1cm%
\begin{picture}%
(6,5)(-3,-1.5)%
\special{pn 8}%
%
\special{pa   -28   335}\special{pa   -28    59}%
\special{fp}%
\special{pn 16}%
\special{pa    28   252}\special{pa    28   142}%
\special{fp}%
\special{pn 8}%
\special{pa   787   197}\special{pa    28   197}%
\special{fp}%
\special{pa  -787   197}\special{pa   -28   197}%
\special{fp}%
\special{pa  -787    12}\special{pa  -898  -179}%
\special{fp}%
\special{pa  -787   197}\special{pa  -787    12}%
\special{fp}%
\special{pa  -787  -394}\special{pa  -787  -209}%
\special{fp}%
{%
\color[cmyk]{0,0,0,0}%
\special{pa -772 12}\special{pa -773 10}\special{pa -773 8}\special{pa -773 6}\special{pa -774 5}%
\special{pa -775 3}\special{pa -776 2}\special{pa -778 0}\special{pa -779 -1}\special{pa -781 -2}%
\special{pa -783 -2}\special{pa -785 -3}\special{pa -786 -3}\special{pa -788 -3}\special{pa -790 -3}%
\special{pa -792 -2}\special{pa -794 -2}\special{pa -795 -1}\special{pa -797 0}\special{pa -798 2}%
\special{pa -800 3}\special{pa -801 5}\special{pa -801 6}\special{pa -802 8}\special{pa -802 10}%
\special{pa -802 12}\special{pa -802 14}\special{pa -802 16}\special{pa -801 17}\special{pa -801 19}%
\special{pa -800 21}\special{pa -798 22}\special{pa -797 23}\special{pa -795 24}\special{pa -794 25}%
\special{pa -792 26}\special{pa -790 27}\special{pa -788 27}\special{pa -786 27}\special{pa -785 27}%
\special{pa -783 26}\special{pa -781 25}\special{pa -779 24}\special{pa -778 23}\special{pa -776 22}%
\special{pa -775 21}\special{pa -774 19}\special{pa -773 17}\special{pa -773 16}\special{pa -773 14}%
\special{pa -772 12}\special{pa -772 12}\special{sh 1}\special{ip}%
}%
\special{pa  -772    12}\special{pa  -773    10}\special{pa  -773     8}\special{pa  -773     6}%
\special{pa  -774     5}\special{pa  -775     3}\special{pa  -776     2}\special{pa  -778     0}%
\special{pa  -779    -1}\special{pa  -781    -2}\special{pa  -783    -2}\special{pa  -785    -3}%
\special{pa  -786    -3}\special{pa  -788    -3}\special{pa  -790    -3}\special{pa  -792    -2}%
\special{pa  -794    -2}\special{pa  -795    -1}\special{pa  -797     0}\special{pa  -798     2}%
\special{pa  -800     3}\special{pa  -801     5}\special{pa  -801     6}\special{pa  -802     8}%
\special{pa  -802    10}\special{pa  -802    12}\special{pa  -802    14}\special{pa  -802    16}%
\special{pa  -801    17}\special{pa  -801    19}\special{pa  -800    21}\special{pa  -798    22}%
\special{pa  -797    23}\special{pa  -795    24}\special{pa  -794    25}\special{pa  -792    26}%
\special{pa  -790    27}\special{pa  -788    27}\special{pa  -786    27}\special{pa  -785    27}%
\special{pa  -783    26}\special{pa  -781    25}\special{pa  -779    24}\special{pa  -778    23}%
\special{pa  -776    22}\special{pa  -775    21}\special{pa  -774    19}\special{pa  -773    17}%
\special{pa  -773    16}\special{pa  -773    14}\special{pa  -772    12}%
\special{fp}%
{%
\color[cmyk]{0,0,0,0}%
\special{pa -772 -209}\special{pa -773 -211}\special{pa -773 -212}\special{pa -773 -214}%
\special{pa -774 -216}\special{pa -775 -217}\special{pa -776 -219}\special{pa -778 -220}%
\special{pa -779 -221}\special{pa -781 -222}\special{pa -783 -223}\special{pa -785 -223}%
\special{pa -786 -224}\special{pa -788 -224}\special{pa -790 -223}\special{pa -792 -223}%
\special{pa -794 -222}\special{pa -795 -221}\special{pa -797 -220}\special{pa -798 -219}%
\special{pa -800 -217}\special{pa -801 -216}\special{pa -801 -214}\special{pa -802 -212}%
\special{pa -802 -211}\special{pa -802 -209}\special{pa -802 -207}\special{pa -802 -205}%
\special{pa -801 -203}\special{pa -801 -201}\special{pa -800 -200}\special{pa -798 -198}%
\special{pa -797 -197}\special{pa -795 -196}\special{pa -794 -195}\special{pa -792 -194}%
\special{pa -790 -194}\special{pa -788 -194}\special{pa -786 -194}\special{pa -785 -194}%
\special{pa -783 -194}\special{pa -781 -195}\special{pa -779 -196}\special{pa -778 -197}%
\special{pa -776 -198}\special{pa -775 -200}\special{pa -774 -201}\special{pa -773 -203}%
\special{pa -773 -205}\special{pa -773 -207}\special{pa -772 -209}\special{pa -772 -209}%
\special{sh 1}\special{ip}%
}%
\special{pa  -772  -209}\special{pa  -773  -211}\special{pa  -773  -212}\special{pa  -773  -214}%
\special{pa  -774  -216}\special{pa  -775  -217}\special{pa  -776  -219}\special{pa  -778  -220}%
\special{pa  -779  -221}\special{pa  -781  -222}\special{pa  -783  -223}\special{pa  -785  -223}%
\special{pa  -786  -224}\special{pa  -788  -224}\special{pa  -790  -223}\special{pa  -792  -223}%
\special{pa  -794  -222}\special{pa  -795  -221}\special{pa  -797  -220}\special{pa  -798  -219}%
\special{pa  -800  -217}\special{pa  -801  -216}\special{pa  -801  -214}\special{pa  -802  -212}%
\special{pa  -802  -211}\special{pa  -802  -209}\special{pa  -802  -207}\special{pa  -802  -205}%
\special{pa  -801  -203}\special{pa  -801  -201}\special{pa  -800  -200}\special{pa  -798  -198}%
\special{pa  -797  -197}\special{pa  -795  -196}\special{pa  -794  -195}\special{pa  -792  -194}%
\special{pa  -790  -194}\special{pa  -788  -194}\special{pa  -786  -194}\special{pa  -785  -194}%
\special{pa  -783  -194}\special{pa  -781  -195}\special{pa  -779  -196}\special{pa  -778  -197}%
\special{pa  -776  -198}\special{pa  -775  -200}\special{pa  -774  -201}\special{pa  -773  -203}%
\special{pa  -773  -205}\special{pa  -773  -207}\special{pa  -772  -209}%
\special{fp}%
\special{pa  -787  -579}\special{pa  -898  -770}%
\special{fp}%
\special{pa  -787  -394}\special{pa  -787  -579}%
\special{fp}%
\special{pa  -787  -984}\special{pa  -787  -799}%
\special{fp}%
{%
\color[cmyk]{0,0,0,0}%
\special{pa -772 -579}\special{pa -773 -581}\special{pa -773 -582}\special{pa -773 -584}%
\special{pa -774 -586}\special{pa -775 -588}\special{pa -776 -589}\special{pa -778 -590}%
\special{pa -779 -591}\special{pa -781 -592}\special{pa -783 -593}\special{pa -785 -593}%
\special{pa -786 -594}\special{pa -788 -594}\special{pa -790 -593}\special{pa -792 -593}%
\special{pa -794 -592}\special{pa -795 -591}\special{pa -797 -590}\special{pa -798 -589}%
\special{pa -800 -588}\special{pa -801 -586}\special{pa -801 -584}\special{pa -802 -582}%
\special{pa -802 -581}\special{pa -802 -579}\special{pa -802 -577}\special{pa -802 -575}%
\special{pa -801 -573}\special{pa -801 -572}\special{pa -800 -570}\special{pa -798 -568}%
\special{pa -797 -567}\special{pa -795 -566}\special{pa -794 -565}\special{pa -792 -565}%
\special{pa -790 -564}\special{pa -788 -564}\special{pa -786 -564}\special{pa -785 -564}%
\special{pa -783 -565}\special{pa -781 -565}\special{pa -779 -566}\special{pa -778 -567}%
\special{pa -776 -568}\special{pa -775 -570}\special{pa -774 -572}\special{pa -773 -573}%
\special{pa -773 -575}\special{pa -773 -577}\special{pa -772 -579}\special{pa -772 -579}%
\special{sh 1}\special{ip}%
}%
\special{pa  -772  -579}\special{pa  -773  -581}\special{pa  -773  -582}\special{pa  -773  -584}%
\special{pa  -774  -586}\special{pa  -775  -588}\special{pa  -776  -589}\special{pa  -778  -590}%
\special{pa  -779  -591}\special{pa  -781  -592}\special{pa  -783  -593}\special{pa  -785  -593}%
\special{pa  -786  -594}\special{pa  -788  -594}\special{pa  -790  -593}\special{pa  -792  -593}%
\special{pa  -794  -592}\special{pa  -795  -591}\special{pa  -797  -590}\special{pa  -798  -589}%
\special{pa  -800  -588}\special{pa  -801  -586}\special{pa  -801  -584}\special{pa  -802  -582}%
\special{pa  -802  -581}\special{pa  -802  -579}\special{pa  -802  -577}\special{pa  -802  -575}%
\special{pa  -801  -573}\special{pa  -801  -572}\special{pa  -800  -570}\special{pa  -798  -568}%
\special{pa  -797  -567}\special{pa  -795  -566}\special{pa  -794  -565}\special{pa  -792  -565}%
\special{pa  -790  -564}\special{pa  -788  -564}\special{pa  -786  -564}\special{pa  -785  -564}%
\special{pa  -783  -565}\special{pa  -781  -565}\special{pa  -779  -566}\special{pa  -778  -567}%
\special{pa  -776  -568}\special{pa  -775  -570}\special{pa  -774  -572}\special{pa  -773  -573}%
\special{pa  -773  -575}\special{pa  -773  -577}\special{pa  -772  -579}%
\special{fp}%
{%
\color[cmyk]{0,0,0,0}%
\special{pa -772 -799}\special{pa -773 -801}\special{pa -773 -803}\special{pa -773 -805}%
\special{pa -774 -806}\special{pa -775 -808}\special{pa -776 -809}\special{pa -778 -811}%
\special{pa -779 -812}\special{pa -781 -813}\special{pa -783 -813}\special{pa -785 -814}%
\special{pa -786 -814}\special{pa -788 -814}\special{pa -790 -814}\special{pa -792 -813}%
\special{pa -794 -813}\special{pa -795 -812}\special{pa -797 -811}\special{pa -798 -809}%
\special{pa -800 -808}\special{pa -801 -806}\special{pa -801 -805}\special{pa -802 -803}%
\special{pa -802 -801}\special{pa -802 -799}\special{pa -802 -797}\special{pa -802 -795}%
\special{pa -801 -794}\special{pa -801 -792}\special{pa -800 -790}\special{pa -798 -789}%
\special{pa -797 -788}\special{pa -795 -787}\special{pa -794 -786}\special{pa -792 -785}%
\special{pa -790 -785}\special{pa -788 -784}\special{pa -786 -784}\special{pa -785 -785}%
\special{pa -783 -785}\special{pa -781 -786}\special{pa -779 -787}\special{pa -778 -788}%
\special{pa -776 -789}\special{pa -775 -790}\special{pa -774 -792}\special{pa -773 -794}%
\special{pa -773 -795}\special{pa -773 -797}\special{pa -772 -799}\special{pa -772 -799}%
\special{sh 1}\special{ip}%
}%
\special{pa  -772  -799}\special{pa  -773  -801}\special{pa  -773  -803}\special{pa  -773  -805}%
\special{pa  -774  -806}\special{pa  -775  -808}\special{pa  -776  -809}\special{pa  -778  -811}%
\special{pa  -779  -812}\special{pa  -781  -813}\special{pa  -783  -813}\special{pa  -785  -814}%
\special{pa  -786  -814}\special{pa  -788  -814}\special{pa  -790  -814}\special{pa  -792  -813}%
\special{pa  -794  -813}\special{pa  -795  -812}\special{pa  -797  -811}\special{pa  -798  -809}%
\special{pa  -800  -808}\special{pa  -801  -806}\special{pa  -801  -805}\special{pa  -802  -803}%
\special{pa  -802  -801}\special{pa  -802  -799}\special{pa  -802  -797}\special{pa  -802  -795}%
\special{pa  -801  -794}\special{pa  -801  -792}\special{pa  -800  -790}\special{pa  -798  -789}%
\special{pa  -797  -788}\special{pa  -795  -787}\special{pa  -794  -786}\special{pa  -792  -785}%
\special{pa  -790  -785}\special{pa  -788  -784}\special{pa  -786  -784}\special{pa  -785  -785}%
\special{pa  -783  -785}\special{pa  -781  -786}\special{pa  -779  -787}\special{pa  -778  -788}%
\special{pa  -776  -789}\special{pa  -775  -790}\special{pa  -774  -792}\special{pa  -773  -794}%
\special{pa  -773  -795}\special{pa  -773  -797}\special{pa  -772  -799}%
\special{fp}%
\special{pa  -352  -531}\special{pa  -352  -256}%
\special{fp}%
\special{pa  -435  -531}\special{pa  -435  -256}%
\special{fp}%
\special{pa  -787  -394}\special{pa  -435  -394}%
\special{fp}%
\special{pa     0  -394}\special{pa  -352  -394}%
\special{fp}%
\special{pa   435  -531}\special{pa   435  -256}%
\special{fp}%
\special{pa   352  -531}\special{pa   352  -256}%
\special{fp}%
\special{pa     0  -394}\special{pa   352  -394}%
\special{fp}%
\special{pa   787  -394}\special{pa   435  -394}%
\special{fp}%
\special{pa    41 -1122}\special{pa    41  -846}%
\special{fp}%
\special{pa   -41 -1122}\special{pa   -41  -846}%
\special{fp}%
\special{pa  -787  -984}\special{pa   -41  -984}%
\special{fp}%
\special{pa   787  -984}\special{pa    41  -984}%
\special{fp}%
\special{pa   719  -827}\special{pa   719  -551}\special{pa   856  -551}\special{pa   856  -827}%
\special{pa   719  -827}%
\special{fp}%
\special{pa   787  -394}\special{pa   787  -551}%
\special{fp}%
\special{pa   787  -984}\special{pa   787  -827}%
\special{fp}%
\settowidth{\Width}{$V$}\setlength{\Width}{-1\Width}%
\settoheight{\Height}{$V$}\settodepth{\Depth}{$V$}\setlength{\Height}{-\Height}%
\put( -0.050, -1.040){\hspace*{\Width}\raisebox{\Height}{$V$}}%
%
\settowidth{\Width}{$\mathrm{S_1}$}\setlength{\Width}{-1\Width}%
\settoheight{\Height}{$\mathrm{S_1}$}\settodepth{\Depth}{$\mathrm{S_1}$}\setlength{\Height}{-\Height}%
\put( -2.150,  0.100){\hspace*{\Width}\raisebox{\Height}{$\mathrm{S_1}$}}%
%
\settowidth{\Width}{$\mathrm{S_2}$}\setlength{\Width}{-1\Width}%
\settoheight{\Height}{$\mathrm{S_2}$}\settodepth{\Depth}{$\mathrm{S_2}$}\setlength{\Height}{-\Height}%
\put( -2.150,  1.600){\hspace*{\Width}\raisebox{\Height}{$\mathrm{S_2}$}}%
%
\settowidth{\Width}{$\mathrm{C_1}$}\setlength{\Width}{-0.5\Width}%
\settoheight{\Height}{$\mathrm{C_1}$}\settodepth{\Depth}{$\mathrm{C_1}$}\setlength{\Height}{\Depth}%
\put( -1.000,  1.540){\hspace*{\Width}\raisebox{\Height}{$\mathrm{C_1}$}}%
%
\settowidth{\Width}{$\mathrm{C_2}$}\setlength{\Width}{-0.5\Width}%
\settoheight{\Height}{$\mathrm{C_2}$}\settodepth{\Depth}{$\mathrm{C_2}$}\setlength{\Height}{\Depth}%
\put(  1.000,  1.540){\hspace*{\Width}\raisebox{\Height}{$\mathrm{C_2}$}}%
%
\settowidth{\Width}{$\mathrm{C_3}$}\setlength{\Width}{-0.5\Width}%
\settoheight{\Height}{$\mathrm{C_3}$}\settodepth{\Depth}{$\mathrm{C_3}$}\setlength{\Height}{\Depth}%
\put(  0.000,  3.040){\hspace*{\Width}\raisebox{\Height}{$\mathrm{C_3}$}}%
%
\settowidth{\Width}{$\mathrm{R}$}\setlength{\Width}{0\Width}%
\settoheight{\Height}{$\mathrm{R}$}\settodepth{\Depth}{$\mathrm{R}$}\setlength{\Height}{-0.5\Height}\setlength{\Depth}{0.5\Depth}\addtolength{\Height}{\Depth}%
\put(  2.300,  1.750){\hspace*{\Width}\raisebox{\Height}{$\mathrm{R}$}}%
%
\special{pa   787   197}\special{pa   787  -394}%
\special{fp}%
\special{pa -780 -394}\special{pa -780 -395}\special{pa -780 -396}\special{pa -780 -396}%
\special{pa -781 -397}\special{pa -781 -398}\special{pa -782 -399}\special{pa -783 -399}%
\special{pa -783 -400}\special{pa -784 -400}\special{pa -785 -401}\special{pa -786 -401}%
\special{pa -787 -401}\special{pa -788 -401}\special{pa -789 -401}\special{pa -790 -401}%
\special{pa -791 -400}\special{pa -791 -400}\special{pa -792 -399}\special{pa -793 -399}%
\special{pa -793 -398}\special{pa -794 -397}\special{pa -794 -396}\special{pa -795 -396}%
\special{pa -795 -395}\special{pa -795 -394}\special{pa -795 -393}\special{pa -795 -392}%
\special{pa -794 -391}\special{pa -794 -390}\special{pa -793 -389}\special{pa -793 -389}%
\special{pa -792 -388}\special{pa -791 -387}\special{pa -791 -387}\special{pa -790 -387}%
\special{pa -789 -386}\special{pa -788 -386}\special{pa -787 -386}\special{pa -786 -386}%
\special{pa -785 -387}\special{pa -784 -387}\special{pa -783 -387}\special{pa -783 -388}%
\special{pa -782 -389}\special{pa -781 -389}\special{pa -781 -390}\special{pa -780 -391}%
\special{pa -780 -392}\special{pa -780 -393}\special{pa -780 -394}\special{pa -780 -394}%
\special{sh 1}\special{ip}%
\special{pa  -780  -394}\special{pa  -780  -395}\special{pa  -780  -396}\special{pa  -780  -396}%
\special{pa  -781  -397}\special{pa  -781  -398}\special{pa  -782  -399}\special{pa  -783  -399}%
\special{pa  -783  -400}\special{pa  -784  -400}\special{pa  -785  -401}\special{pa  -786  -401}%
\special{pa  -787  -401}\special{pa  -788  -401}\special{pa  -789  -401}\special{pa  -790  -401}%
\special{pa  -791  -400}\special{pa  -791  -400}\special{pa  -792  -399}\special{pa  -793  -399}%
\special{pa  -793  -398}\special{pa  -794  -397}\special{pa  -794  -396}\special{pa  -795  -396}%
\special{pa  -795  -395}\special{pa  -795  -394}\special{pa  -795  -393}\special{pa  -795  -392}%
\special{pa  -794  -391}\special{pa  -794  -390}\special{pa  -793  -389}\special{pa  -793  -389}%
\special{pa  -792  -388}\special{pa  -791  -387}\special{pa  -791  -387}\special{pa  -790  -387}%
\special{pa  -789  -386}\special{pa  -788  -386}\special{pa  -787  -386}\special{pa  -786  -386}%
\special{pa  -785  -387}\special{pa  -784  -387}\special{pa  -783  -387}\special{pa  -783  -388}%
\special{pa  -782  -389}\special{pa  -781  -389}\special{pa  -781  -390}\special{pa  -780  -391}%
\special{pa  -780  -392}\special{pa  -780  -393}\special{pa  -780  -394}%
\special{fp}%
\special{pa 795 -394}\special{pa 795 -395}\special{pa 795 -396}\special{pa 794 -396}%
\special{pa 794 -397}\special{pa 793 -398}\special{pa 793 -399}\special{pa 792 -399}%
\special{pa 791 -400}\special{pa 791 -400}\special{pa 790 -401}\special{pa 789 -401}%
\special{pa 788 -401}\special{pa 787 -401}\special{pa 786 -401}\special{pa 785 -401}%
\special{pa 784 -400}\special{pa 783 -400}\special{pa 783 -399}\special{pa 782 -399}%
\special{pa 781 -398}\special{pa 781 -397}\special{pa 780 -396}\special{pa 780 -396}%
\special{pa 780 -395}\special{pa 780 -394}\special{pa 780 -393}\special{pa 780 -392}%
\special{pa 780 -391}\special{pa 781 -390}\special{pa 781 -389}\special{pa 782 -389}%
\special{pa 783 -388}\special{pa 783 -387}\special{pa 784 -387}\special{pa 785 -387}%
\special{pa 786 -386}\special{pa 787 -386}\special{pa 788 -386}\special{pa 789 -386}%
\special{pa 790 -387}\special{pa 791 -387}\special{pa 791 -387}\special{pa 792 -388}%
\special{pa 793 -389}\special{pa 793 -389}\special{pa 794 -390}\special{pa 794 -391}%
\special{pa 795 -392}\special{pa 795 -393}\special{pa 795 -394}\special{pa 795 -394}%
\special{sh 1}\special{ip}%
\special{pa   795  -394}\special{pa   795  -395}\special{pa   795  -396}\special{pa   794  -396}%
\special{pa   794  -397}\special{pa   793  -398}\special{pa   793  -399}\special{pa   792  -399}%
\special{pa   791  -400}\special{pa   791  -400}\special{pa   790  -401}\special{pa   789  -401}%
\special{pa   788  -401}\special{pa   787  -401}\special{pa   786  -401}\special{pa   785  -401}%
\special{pa   784  -400}\special{pa   783  -400}\special{pa   783  -399}\special{pa   782  -399}%
\special{pa   781  -398}\special{pa   781  -397}\special{pa   780  -396}\special{pa   780  -396}%
\special{pa   780  -395}\special{pa   780  -394}\special{pa   780  -393}\special{pa   780  -392}%
\special{pa   780  -391}\special{pa   781  -390}\special{pa   781  -389}\special{pa   782  -389}%
\special{pa   783  -388}\special{pa   783  -387}\special{pa   784  -387}\special{pa   785  -387}%
\special{pa   786  -386}\special{pa   787  -386}\special{pa   788  -386}\special{pa   789  -386}%
\special{pa   790  -387}\special{pa   791  -387}\special{pa   791  -387}\special{pa   792  -388}%
\special{pa   793  -389}\special{pa   793  -389}\special{pa   794  -390}\special{pa   794  -391}%
\special{pa   795  -392}\special{pa   795  -393}\special{pa   795  -394}%
\special{fp}%
\end{picture}}%
        }
        図はコンデンサー$\mathrm{C_1}$,$\mathrm{C_2}$,$\mathrm{C_3}$(電気容量はそれぞれ$C$,$2C$,$3C$)
        ,電池(起電力$V$)およびスイッチ$\mathrm{S_1}$,$\mathrm{S_2}$と抵抗$R$からなる回路である。
        最初,スイッチはどちらも開いており,いずれのコンデンサーにも電荷はない。
        \begin{enumerate}[I.]
            \item まず,スイッチ$\mathrm{S_1}$を閉じ,$\mathrm{C_1}$と$\mathrm{C_2}$を充電した。
            \begin{Enumerate}[(1)]
                \item $\mathrm{C_1}$に蓄えられる電気量はいくらか。
                \item $\mathrm{C_2}$にかかる電圧はいくらか。
            \end{Enumerate}
            \item 次に$\mathrm{S_1}$を開いてから,$\mathrm{S_2}$を閉じ,十分に時間がたった。
            \begin{Enumerate*}[(1)]
                \item $\mathrm{C_3}$にかかる電圧はいくらか。
                \item $\mathrm{C_2}$に蓄えられる電気量はいくらか。
                \item 抵抗Rで発生したジュール熱はいくらか。
            \end{Enumerate*}
        \end{enumerate}
    \end{mawarikomi}
