\hakosyokika
\item
    \begin{mawarikomi}{150pt}{\begin{zahyou*}[ul=6mm](-1,8)(-1.5,1)
    \ArrowLine{(\xmin,0)}{(\xmax,0)}
    \def\O{(0,0)}
    \def\A{(4,0)}
    \def\B{(3.5,0.5)}
    \def\C{(3.5,0)}
    \def\D{(4.5,0)}
    \def\E{(4.5,0.5)}
    \def\vvec{(1.5,0)}
    \small
    \put(5,-1.5){図1}
    \put(4.5,0.25){\Yasen\vvec}
    \Put\E{$v$\tanni{m/s}}
    \Nuritubusi{\B\C\D\E\B}
    \Drawline{\B\C\D\E\B}
    \Put{(\xmax,0)}[s]{$x$\tanni{m}}
    \xmemori{0}
    \xmemori<x>{4}
    % \zahyouMemori[g][n]<dx=1,dy=4>
\end{zahyou*}
\\
        \begin{zahyou}[ul=6mm,
                xscale=0.50,
                yscale=0.25,
                yokozikukigou=$t$\tanni{s},
                tatezikukigou=$v$\tanni{m/s}
                ](-1,19)(-4,20)
    \def\O{(0,0)}
    \def\A{(4,16)}
    \def\B{(7,16)}
    \def\C{(18,-6)}
    \small
    \put(13,9){図2}
    \Put\A[syaei=xy]{}
    \Put\B[syaei=xy]{}
    \xmemori{15}
    {\thicklines
    \Drawline{\O\A\B\C}}
    % \zahyouMemori[g][n]<dx=1,dy=4>
\end{zahyou}
}
        図1のように,$x$軸上を運動する物体があり,時刻$t$での速度$v$が図2で表される。時刻$t=0$での物体の位置を原点$x=0$とする。
        \begin{enumerate}
            \item 時刻$t=2$\sftanni{s}における物体の加速度$a$は\Hako \sftanni{m/s^2}であり,時刻$t=6$\sftanni{s}での加速度$a$は\Hako \sftanni{m/s^2}であり,$t=11$\sftanni{s}での加速度$a$は\Hako \sftanni{m/s^2}である。
            \item 時刻$t=6$\sftanni{s}における物体の位置$x$は\Hako \sftanni{m}である。
            \item 物体が原点$x=0$から右に最も離れる時刻$t$は\Hako \sftanni{s}であり,その位置$x$は\Hako \sftanni{m}である。
            \item 時刻$t=15$\sftanni{s}以後も,そのまま運動を続けた場合,物体が再び原点に戻ってくる時刻$t$は\Hako \sftanni{s}であり,そのときの速度$v$は\Hako \sftanni{m/s}である。
        \end{enumerate}
    \end{mawarikomi}