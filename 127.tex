\hakosyokika
\item
    \begin{mawarikomi}(10pt,0pt){210pt}{
        %WinTpicVersion4.32a
{\unitlength 0.1in%
\begin{picture}(30.6791,13.4350)(3.2776,-17.7165)%
% POLYGON 2 0 3 0 Black White  
% 6 500 1300 600 1300 900 1000 800 1000 800 1000 500 1300
% 
\special{pn 8}%
\special{pa 492 1280}%
\special{pa 591 1280}%
\special{pa 886 984}%
\special{pa 787 984}%
\special{pa 492 1280}%
\special{pa 591 1280}%
\special{fp}%
% LINE 2 0 3 0 Black White  
% 4 850 1000 1050 800 550 1300 350 1500
% 
\special{pn 8}%
\special{pa 837 984}%
\special{pa 1033 787}%
\special{fp}%
\special{pa 541 1280}%
\special{pa 344 1476}%
\special{fp}%
% BOX 2 0 3 0 Black White  
% 2 1050 785 3050 815
% 
\special{pn 8}%
\special{pa 1033 773}%
\special{pa 3002 773}%
\special{pa 3002 802}%
\special{pa 1033 802}%
\special{pa 1033 773}%
\special{pa 3002 773}%
\special{fp}%
% BOX 2 0 2 0 Black White  
% 2 350 1485 2350 1515
% 
\special{pn 0}%
\special{sh 0}%
\special{pa 344 1462}%
\special{pa 2313 1462}%
\special{pa 2313 1491}%
\special{pa 344 1491}%
\special{pa 344 1462}%
\special{ip}%
\special{pn 8}%
\special{pa 344 1462}%
\special{pa 2313 1462}%
\special{pa 2313 1491}%
\special{pa 344 1491}%
\special{pa 344 1462}%
\special{pa 2313 1462}%
\special{fp}%
% POLYGON 2 5 2 0 Black White  
% 5 1775 1520 2525 770 2485 730 1730 1485 1775 1520
% 
\special{pn 0}%
\special{sh 0}%
\special{pa 1747 1496}%
\special{pa 2485 758}%
\special{pa 2446 719}%
\special{pa 1703 1462}%
\special{pa 1747 1496}%
\special{ip}%
\special{pn 8}%
\special{pa 1747 1496}%
\special{pa 2485 758}%
\special{pa 2446 719}%
\special{pa 1703 1462}%
\special{pa 1747 1496}%
\special{ip}%
% CIRCLE 2 0 2 0 Black White  
% 4 1750 1505 1780 1505 1780 1505 1780 1505
% 
\special{sh 0}%
\special{ia 1722 1481 30 30 0.0000000 6.2831853}%
\special{pn 8}%
\special{ar 1722 1481 30 30 0.0000000 6.2831853}%
% LINE 2 0 3 0 Black White  
% 2 1730 1485 2460 755
% 
\special{pn 8}%
\special{pa 1703 1462}%
\special{pa 2421 743}%
\special{fp}%
% CIRCLE 2 0 3 0 Black White  
% 4 2470 785 2500 795 2500 795 2380 695
% 
\special{pn 8}%
\special{ar 2431 773 31 31 3.9269908 0.3217506}%
% LINE 2 0 3 0 Black White  
% 2 1775 1520 2500 795
% 
\special{pn 8}%
\special{pa 1747 1496}%
\special{pa 2461 782}%
\special{fp}%
% LINE 2 0 3 0 Black White  
% 2 2525 785 2495 785
% 
\special{pn 8}%
\special{pa 2485 773}%
\special{pa 2456 773}%
\special{fp}%
% VECTOR 2 0 3 0 Black White  
% 4 1500 1155 1840 815 1500 1155 1170 1485
% 
\special{pn 8}%
\special{pa 1476 1137}%
\special{pa 1811 802}%
\special{fp}%
\special{sh 1}%
\special{pa 1811 802}%
\special{pa 1751 835}%
\special{pa 1774 840}%
\special{pa 1779 862}%
\special{pa 1811 802}%
\special{fp}%
\special{pa 1476 1137}%
\special{pa 1152 1462}%
\special{fp}%
\special{sh 1}%
\special{pa 1152 1462}%
\special{pa 1212 1429}%
\special{pa 1189 1424}%
\special{pa 1184 1402}%
\special{pa 1152 1462}%
\special{fp}%
% STR 2 0 3 0 Black White  
% 4 1515 1095 1515 1145 5 0 1 0
% $\ell$
\put(14.9114,-11.2697){\makebox(0,0){{\colorbox[named]{White}{$\ell$}}}}%
% STR 2 0 3 0 Black White  
% 4 585 1030 585 1080 5 0 0 0
% $R$
\put(5.7579,-10.6299){\makebox(0,0){$R$}}%
% VECTOR 2 0 3 0 Black White  
% 6 1085 1080 1085 580 1385 1080 1385 1080 1385 1080 1385 580
% 
\special{pn 8}%
\special{pa 1068 1063}%
\special{pa 1068 571}%
\special{fp}%
\special{sh 1}%
\special{pa 1068 571}%
\special{pa 1048 637}%
\special{pa 1068 623}%
\special{pa 1088 637}%
\special{pa 1068 571}%
\special{fp}%
\special{pa 1363 1063}%
\special{pa 1363 1063}%
\special{fp}%
\special{pa 1363 1063}%
\special{pa 1363 571}%
\special{fp}%
\special{sh 1}%
\special{pa 1363 571}%
\special{pa 1344 637}%
\special{pa 1363 623}%
\special{pa 1383 637}%
\special{pa 1363 571}%
\special{fp}%
% VECTOR 2 0 3 0 Black White  
% 6 1685 1080 1685 580 1985 1080 1985 1080 1985 1080 1985 580
% 
\special{pn 8}%
\special{pa 1658 1063}%
\special{pa 1658 571}%
\special{fp}%
\special{sh 1}%
\special{pa 1658 571}%
\special{pa 1639 637}%
\special{pa 1658 623}%
\special{pa 1678 637}%
\special{pa 1658 571}%
\special{fp}%
\special{pa 1954 1063}%
\special{pa 1954 1063}%
\special{fp}%
\special{pa 1954 1063}%
\special{pa 1954 571}%
\special{fp}%
\special{sh 1}%
\special{pa 1954 571}%
\special{pa 1934 637}%
\special{pa 1954 623}%
\special{pa 1973 637}%
\special{pa 1954 571}%
\special{fp}%
% VECTOR 2 0 3 0 Black White  
% 6 2285 1080 2285 580 2585 1080 2585 1080 2585 1080 2585 580
% 
\special{pn 8}%
\special{pa 2249 1063}%
\special{pa 2249 571}%
\special{fp}%
\special{sh 1}%
\special{pa 2249 571}%
\special{pa 2229 637}%
\special{pa 2249 623}%
\special{pa 2269 637}%
\special{pa 2249 571}%
\special{fp}%
\special{pa 2544 1063}%
\special{pa 2544 1063}%
\special{fp}%
\special{pa 2544 1063}%
\special{pa 2544 571}%
\special{fp}%
\special{sh 1}%
\special{pa 2544 571}%
\special{pa 2525 637}%
\special{pa 2544 623}%
\special{pa 2564 637}%
\special{pa 2544 571}%
\special{fp}%
% STR 2 0 3 0 Black White  
% 4 1735 530 1735 580 2 0 0 0
% $B$
\put(17.0768,-5.7087){\makebox(0,0)[lb]{$B$}}%
% STR 2 0 3 0 Black White  
% 4 2465 660 2465 710 5 0 0 0
% a
\put(24.2618,-6.9882){\makebox(0,0){a}}%
% STR 2 0 3 0 Black White  
% 4 1765 1560 1765 1610 5 0 0 0
% b
\put(17.3720,-15.8465){\makebox(0,0){b}}%
% STR 2 0 3 0 Black White  
% 4 365 1560 365 1610 5 0 0 0
% e
\put(3.5925,-15.8465){\makebox(0,0){e}}%
% STR 2 0 3 0 Black White  
% 4 1020 665 1020 715 5 0 0 0
% c
\put(10.0394,-7.0374){\makebox(0,0){c}}%
% STR 2 0 3 0 Black White  
% 4 3020 665 3020 715 5 0 0 0
% d
\put(29.7244,-7.0374){\makebox(0,0){d}}%
% STR 2 0 3 0 Black White  
% 4 2320 1555 2320 1605 5 0 0 0
% f
\put(22.8346,-15.7972){\makebox(0,0){f}}%
% LINE 2 0 3 0 Black White  
% 8 2000 1300 2400 1200 2400 1200 2300 1000 2400 1200 3300 1200 3400 1300 3400 1700
% 
\special{pn 8}%
\special{pa 1969 1280}%
\special{pa 2362 1181}%
\special{fp}%
\special{pa 2362 1181}%
\special{pa 2264 984}%
\special{fp}%
\special{pa 2362 1181}%
\special{pa 3248 1181}%
\special{fp}%
\special{pa 3346 1280}%
\special{pa 3346 1673}%
\special{fp}%
% CIRCLE 2 0 1 0 Black Black  
% 4 3400 1750 3450 1750 3450 1750 3450 1750
% 
\special{sh 0.200}%
\special{ia 3346 1722 49 49 0.0000000 6.2831853}%
\special{pn 8}%
\special{ar 3346 1722 49 49 0.0000000 6.2831853}%
% CIRCLE 2 0 3 0 Black White  
% 4 3300 1300 3400 1300 3400 1300 3400 1300
% 
\special{pn 8}%
\special{ar 3248 1280 98 98 0.0000000 6.2831853}%
% CIRCLE 2 0 3 0 Black White  
% 4 3300 1300 3325 1300 3325 1300 3325 1300
% 
\special{pn 8}%
\special{ar 3248 1280 25 25 0.0000000 6.2831853}%
% CIRCLE 2 0 3 0 Black White  
% 4 1970 1285 2000 1295 2000 1295 1880 1195
% 
\special{pn 8}%
\special{ar 1939 1265 31 31 3.9269908 0.3217506}%
% CIRCLE 2 0 3 0 Black White  
% 4 2270 985 2300 995 2300 995 2180 895
% 
\special{pn 8}%
\special{ar 2234 969 31 31 3.9269908 0.3217506}%
% VECTOR 2 0 3 0 Black White  
% 2 2470 1155 2670 1155
% 
\special{pn 8}%
\special{pa 2431 1137}%
\special{pa 2628 1137}%
\special{fp}%
\special{sh 1}%
\special{pa 2628 1137}%
\special{pa 2562 1117}%
\special{pa 2576 1137}%
\special{pa 2562 1156}%
\special{pa 2628 1137}%
\special{fp}%
% STR 2 0 3 0 Black White  
% 4 2770 1090 2770 1140 5 0 0 0
% $v$
\put(27.2638,-11.2205){\makebox(0,0){$v$}}%
% STR 2 0 3 0 Black White  
% 4 3245 1690 3245 1740 5 0 0 0
% $M$
\put(31.9390,-17.1260){\makebox(0,0){$M$}}%
\end{picture}}%

    }
    抵抗のない導体棒ab,cd,efと抵抗$R$とからなる回路が,磁束密度$B$の一様な磁場(鉛直上向き)中に水平に置かれている。cd,efは距離$\ell $を隔てて平行に固定されていて,十分に長い。質量$m$,長さ$\ell $の導体棒abは,この上に直角に置かれ,左右になめらかに動けるようになっている。abに軽い糸を結んで滑車に通し,その端に質量$M$んもおもりを下げて,abを右向きに運動させた。重力加速度の大きさを$g$とする。
        \begin{enumerate}
            \item 導体棒abを速さ$v$としたとき,棒に流れる電流の大きさと向きを求めよ。また,棒が磁場から受ける力の大きさと向きを求めよ。
            \item しばらくたつと,abは一定の速さ$v_0$で運動するようになった。
                \begin{enumerate}[(ア)]
                    \item このとき,回路に流れている電流$I_0$と速さ$v_0$を求めよ。
                    \item 単位時間に,おもりが失う重力の位置エネルギーを$P$,回路で発生するジュール熱を$Q$とする。$P$と$Q$の関係を答えよ。考え方の根拠も10字以内で記せ。
                    \item (イ)の関係を,$P$と$Q$を計算して確かめよ。
                \end{enumerate}
            \item 次におもりをつないでいる糸を切った。その後に回路で発生するジュール熱の総量はいくらか。
        \end{enumerate}
    \end{mawarikomi}