\hakosyokika
\item
    \begin{mawarikomi}(10pt,0){210pt}{\input{./fig/fig077.tex}}
    ガラス管AB中にピストンPを挿入し,開口部Aの近くで音さを振動させる。
    音速を340\sftanni{m/s}とし,開口端補正は無視できるものとする。
        \begin{enumerate}
            \item PをAからBに向けてゆっくりと移動したところ,Aからの距離が20.0\sftanni{cm}のところで最初の共鳴が起こった。音さの振動数$f$\tanni{Hz}を求めよ。
            \item Pをさらに右に移動したところ,Aからある距離になったときに次の共鳴が起こった。その位置はAから何\sftanni{cm}のところか。
            \item Pをさらに右に移動したところBの位置までずっと共鳴が起こらなかったが,Pをガラス管から取り外したところ,ちょうど共鳴が起こった。このガラス管の長さ$L$\tanni{cm}を求めよ。また,このときの管内の定在波の様子を図に示せ。
            \item Pを取り外したまま,振動数のより小さい音さを用い,共鳴を起こしたい。その振動数$f'$\tanni{Hz}を求めよ。
        \end{enumerate}
    \end{mawarikomi}