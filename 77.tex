\hakosyokika
\item
    \begin{mawarikomi}(10pt,0){210pt}{%WinTpicVersion4.32a
{\unitlength 0.1in%
\begin{picture}(22.5000,10.4500)(3.4000,-13.5500)%
% LINE 2 0 3 0 Black White  
% 6 430 640 380 640 380 640 380 890 430 890 430 640
% 
\special{pn 8}%
\special{pa 430 640}%
\special{pa 380 640}%
\special{fp}%
\special{pa 380 640}%
\special{pa 380 890}%
\special{fp}%
\special{pa 430 890}%
\special{pa 430 640}%
\special{fp}%
% CIRCLE 2 0 3 1 Black White  
% 4 480 890 480 840 380 890 580 890
% 
\special{pn 8}%
\special{ar 480 890 50 50 6.2831853 3.1415927}%
% CIRCLE 2 0 3 2 Black White  
% 4 480 890 480 790 380 890 630 890
% 
\special{pn 8}%
\special{ar 480 890 100 100 6.2831853 3.1415927}%
% LINE 2 0 3 3 Black White  
% 6 530 890 530 640 530 640 580 640 580 640 580 890
% 
\special{pn 8}%
\special{pa 530 890}%
\special{pa 530 640}%
\special{fp}%
\special{pa 530 640}%
\special{pa 580 640}%
\special{fp}%
\special{pa 580 640}%
\special{pa 580 890}%
\special{fp}%
% POLYGON 2 5 2 4 Black White  
% 5 455 1338 505 1338 505 975 455 975 455 1338
% 
\special{pn 0}%
\special{sh 0}%
\special{pa 455 1338}%
\special{pa 505 1338}%
\special{pa 505 975}%
\special{pa 455 975}%
\special{pa 455 1338}%
\special{ip}%
\special{pn 8}%
\special{pa 455 1338}%
\special{pa 505 1338}%
\special{pa 505 975}%
\special{pa 455 975}%
\special{pa 455 1338}%
\special{ip}%
% LINE 2 0 3 5 Black White  
% 6 505 988 505 1338 505 1338 455 1338 455 1338 455 988
% 
\special{pn 8}%
\special{pa 505 988}%
\special{pa 505 1338}%
\special{fp}%
\special{pa 505 1338}%
\special{pa 455 1338}%
\special{fp}%
\special{pa 455 1338}%
\special{pa 455 988}%
\special{fp}%
% LINE 2 0 3 0 Black White  
% 2 780 510 780 310
% 
\special{pn 8}%
\special{pa 780 510}%
\special{pa 780 310}%
\special{fp}%
% LINE 2 0 3 0 Black White  
% 2 2590 310 2590 510
% 
\special{pn 8}%
\special{pa 2590 310}%
\special{pa 2590 510}%
\special{fp}%
% VECTOR 2 0 3 0 Black White  
% 2 1490 400 780 400
% 
\special{pn 8}%
\special{pa 1490 400}%
\special{pa 780 400}%
\special{fp}%
\special{sh 1}%
\special{pa 780 400}%
\special{pa 847 420}%
\special{pa 833 400}%
\special{pa 847 380}%
\special{pa 780 400}%
\special{fp}%
% VECTOR 2 0 3 0 Black White  
% 2 1780 400 2590 400
% 
\special{pn 8}%
\special{pa 1780 400}%
\special{pa 2590 400}%
\special{fp}%
\special{sh 1}%
\special{pa 2590 400}%
\special{pa 2523 380}%
\special{pa 2537 400}%
\special{pa 2523 420}%
\special{pa 2590 400}%
\special{fp}%
% STR 2 0 3 0 Black White  
% 4 1630 300 1630 400 5 0 0 0
% $L$
\put(16.3000,-4.0000){\makebox(0,0){$L$}}%
% BOX 2 0 3 0 Black White  
% 2 780 600 2590 580
% 
\special{pn 8}%
\special{pa 780 600}%
\special{pa 2590 600}%
\special{pa 2590 580}%
\special{pa 780 580}%
\special{pa 780 600}%
\special{pa 2590 600}%
\special{fp}%
% BOX 2 0 3 0 Black White  
% 2 780 1000 2590 980
% 
\special{pn 8}%
\special{pa 780 1000}%
\special{pa 2590 1000}%
\special{pa 2590 980}%
\special{pa 780 980}%
\special{pa 780 1000}%
\special{pa 2590 1000}%
\special{fp}%
% BOX 2 0 1 0 Black Black  
% 2 1780 980 1980 600
% 
\special{pn 0}%
\special{sh 0.300}%
\special{pa 1780 980}%
\special{pa 1980 980}%
\special{pa 1980 600}%
\special{pa 1780 600}%
\special{pa 1780 980}%
\special{ip}%
\special{pn 8}%
\special{pa 1780 980}%
\special{pa 1980 980}%
\special{pa 1980 600}%
\special{pa 1780 600}%
\special{pa 1780 980}%
\special{pa 1980 980}%
\special{fp}%
% BOX 2 0 3 0 Black White  
% 2 1980 780 2590 820
% 
\special{pn 8}%
\special{pa 1980 780}%
\special{pa 2590 780}%
\special{pa 2590 820}%
\special{pa 1980 820}%
\special{pa 1980 780}%
\special{pa 2590 780}%
\special{fp}%
% STR 2 0 3 0 Black White  
% 4 780 990 780 1090 5 0 0 0
% A
\put(7.8000,-10.9000){\makebox(0,0){A}}%
% STR 2 0 3 0 Black White  
% 4 1780 1070 1780 1170 5 0 0 0
% ピストンP
\put(17.8000,-11.7000){\makebox(0,0){ピストンP}}%
% STR 2 0 3 0 Black White  
% 4 2580 990 2580 1090 5 0 0 0
% B
\put(25.8000,-10.9000){\makebox(0,0){B}}%
% LINE 3 0 3 0 Black White  
% 2 1780 1070 1880 800
% 
\special{pn 4}%
\special{pa 1780 1070}%
\special{pa 1880 800}%
\special{fp}%
% STR 2 0 3 0 Black White  
% 4 480 1320 480 1420 5 0 0 0
% おんさ
\put(4.8000,-14.2000){\makebox(0,0){おんさ}}%
\end{picture}}%
}
    ガラス管AB中にピストンPを挿入し,開口部Aの近くで音さを振動させる。
    音速を340\sftanni{m/s}とし,開口端補正は無視できるものとする。
        \begin{enumerate}
            \item PをAからBに向けてゆっくりと移動したところ,Aからの距離が20.0\sftanni{cm}のところで最初の共鳴が起こった。音さの振動数$f$\tanni{Hz}を求めよ。
            \item Pをさらに右に移動したところ,Aからある距離になったときに次の共鳴が起こった。その位置はAから何\sftanni{cm}のところか。
            \item Pをさらに右に移動したところBの位置までずっと共鳴が起こらなかったが,Pをガラス管から取り外したところ,ちょうど共鳴が起こった。このガラス管の長さ$L$\tanni{cm}を求めよ。また,このときの管内の定在波の様子を図に示せ。
            \item Pを取り外したまま,振動数のより小さい音さを用い,共鳴を起こしたい。その振動数$f'$\tanni{Hz}を求めよ。
        \end{enumerate}
    \end{mawarikomi}