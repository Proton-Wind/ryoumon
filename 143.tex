\item
    \begin{mawarikomi}{150pt}{
        \input{./fig/fig143.tex}
    }
ナトリウム Na を陰極とする光電管を用いて図1の回路を作り,波長 $3.0 \times 10^{-7}$\sftanni{m} の紫外線を当てて光電効果の実験を行った。光速度 $c = 3.0 \times 10^8$ \sftanni{m/s},電気素量 $e = 1.6 \times 10^{-19}$ \sftanni{C},プランク定数 $h = 6.6 \times 10^{-34}$ \tanni{J\cdot s} とする。
        \begin{enumerate}
            \item AB 間に十分な電圧をかけたところ,回路に $1.6 \times 10^{-6}$ \sftanni{A} の電流が流れた。陰極 A から陽極 B に達する電子の数 $N$ は毎秒何個か。
            \item AB 間の電圧を変えながら光電流 $I$ を測定すると,図2のようなグラフが得られた。陰極から飛び出す光電子の最大運動エネルギー $K$ \tanni{J} はいくらか。
            \item 光子のエネルギー \tanni{J} と Na の仕事関数 $W$ \tanni{J} を求めよ。また,$W$ を \tanni{eV} で表せ。そして Na に対する限界振動数 $\nu_0$ \tanni{Hz} を求めよ。
            \item 光の波長を変えずに光の明るさを半分にすると,図2の曲線はどう変わるか。図に概形を書き込め。
            \item 当てる光の波長を変えながら (2) と同様の実験を行う。横軸を光の振動数 $\nu$ \tanni{Hz},縦軸を $K$ \tanni{J} とする時,得られるグラフを文字 $\nu_0$ と $W$ を用いて定性的に示せ。また,$h$ はグラフの何と対応しているか。
        \end{enumerate}
\end{mawarikomi}
