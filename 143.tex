\item
    \begin{mawarikomi}{150pt}{
        %WinTpicVersion4.32a
{\unitlength 0.1in%
\begin{picture}(16.6200,40.3500)(-1.3500,-43.6000)%
% CIRCLE 2 0 3 0 Black White  
% 4 700 625 1000 625 1000 625 1000 625
% 
\special{pn 8}%
\special{ar 700 625 300 300 0.0000000 6.2831853}%
% CIRCLE 0 0 3 0 Black White  
% 4 700 625 950 625 550 225 550 1025
% 
\special{pn 20}%
\special{ar 700 625 250 250 1.9295670 4.3536183}%
% LINE 2 0 3 0 Black White  
% 8 700 625 700 1825 150 625 150 625 450 625 200 625 200 625 200 1925
% 
\special{pn 8}%
\special{pa 700 625}%
\special{pa 700 1825}%
\special{fp}%
\special{pa 150 625}%
\special{pa 150 625}%
\special{fp}%
\special{pa 450 625}%
\special{pa 200 625}%
\special{fp}%
\special{pa 200 625}%
\special{pa 200 1925}%
\special{fp}%
% LINE 2 0 3 0 Black White  
% 2 200 1825 1100 1825
% 
\special{pn 8}%
\special{pa 200 1825}%
\special{pa 1100 1825}%
\special{fp}%
% LINE 2 0 3 0 Black White  
% 4 700 1125 1500 1125 1500 1125 1500 1725
% 
\special{pn 8}%
\special{pa 700 1125}%
\special{pa 1500 1125}%
\special{fp}%
\special{pa 1500 1125}%
\special{pa 1500 1725}%
\special{fp}%
% VECTOR 2 0 3 0 Black White  
% 2 1500 1725 1400 1725
% 
\special{pn 8}%
\special{pa 1500 1725}%
\special{pa 1400 1725}%
\special{fp}%
\special{sh 1}%
\special{pa 1400 1725}%
\special{pa 1467 1745}%
\special{pa 1453 1725}%
\special{pa 1467 1705}%
\special{pa 1400 1725}%
\special{fp}%
% BOX 2 0 3 0 Black White  
% 2 1300 1525 1400 2225
% 
\special{pn 8}%
\special{pa 1300 1525}%
\special{pa 1400 1525}%
\special{pa 1400 2225}%
\special{pa 1300 2225}%
\special{pa 1300 1525}%
\special{pa 1400 1525}%
\special{fp}%
% LINE 2 0 3 0 Black White  
% 10 1350 2225 1350 2325 1350 2325 1100 2325 1100 2325 1100 1425 1100 1425 1350 1425 1350 1425 1350 1525
% 
\special{pn 8}%
\special{pa 1350 2225}%
\special{pa 1350 2325}%
\special{fp}%
\special{pa 1350 2325}%
\special{pa 1100 2325}%
\special{fp}%
\special{pa 1100 2325}%
\special{pa 1100 1425}%
\special{fp}%
\special{pa 1100 1425}%
\special{pa 1350 1425}%
\special{fp}%
\special{pa 1350 1425}%
\special{pa 1350 1525}%
\special{fp}%
% BOX 2 5 2 0 Black White  
% 2 1045 1600 1145 1650
% 
\special{pn 0}%
\special{sh 0}%
\special{pa 1045 1600}%
\special{pa 1145 1600}%
\special{pa 1145 1650}%
\special{pa 1045 1650}%
\special{pa 1045 1600}%
\special{ip}%
\special{pn 8}%
\special{pa 1045 1600}%
\special{pa 1145 1600}%
\special{pa 1145 1650}%
\special{pa 1045 1650}%
\special{pa 1045 1600}%
\special{ip}%
% BOX 2 5 2 0 Black White  
% 2 1045 2000 1145 2050
% 
\special{pn 0}%
\special{sh 0}%
\special{pa 1045 2000}%
\special{pa 1145 2000}%
\special{pa 1145 2050}%
\special{pa 1045 2050}%
\special{pa 1045 2000}%
\special{ip}%
\special{pn 8}%
\special{pa 1045 2000}%
\special{pa 1145 2000}%
\special{pa 1145 2050}%
\special{pa 1045 2050}%
\special{pa 1045 2000}%
\special{ip}%
% LINE 2 0 3 0 Black White  
% 2 1000 2000 1200 2000
% 
\special{pn 8}%
\special{pa 1000 2000}%
\special{pa 1200 2000}%
\special{fp}%
% LINE 2 0 3 0 Black White  
% 2 1000 1600 1200 1600
% 
\special{pn 8}%
\special{pa 1000 1600}%
\special{pa 1200 1600}%
\special{fp}%
% LINE 2 0 3 0 Black White  
% 2 1050 1650 1150 1650
% 
\special{pn 8}%
\special{pa 1050 1650}%
\special{pa 1150 1650}%
\special{fp}%
% LINE 2 0 3 0 Black White  
% 2 1050 2050 1150 2050
% 
\special{pn 8}%
\special{pa 1050 2050}%
\special{pa 1150 2050}%
\special{fp}%
% CIRCLE 2 0 2 0 Black White  
% 4 700 1450 800 1450 800 1450 800 1450
% 
\special{sh 0}%
\special{ia 700 1450 100 100 0.0000000 6.2831853}%
\special{pn 8}%
\special{ar 700 1450 100 100 0.0000000 6.2831853}%
% STR 2 0 3 0 Black White  
% 4 700 1400 700 1450 5 0 0 0
% \underline{V}
\put(7.0000,-14.5000){\makebox(0,0){\underline{V}}}%
% CIRCLE 2 0 2 0 Black White  
% 4 200 1150 300 1150 300 1150 300 1150
% 
\special{sh 0}%
\special{ia 200 1150 100 100 0.0000000 6.2831853}%
\special{pn 8}%
\special{ar 200 1150 100 100 0.0000000 6.2831853}%
% STR 2 0 3 0 Black White  
% 4 200 1100 200 1150 5 0 0 0
% \underline{A}
\put(2.0000,-11.5000){\makebox(0,0){\underline{A}}}%
% LINE 2 0 3 0 Black White  
% 2 100 1920 300 1920
% 
\special{pn 8}%
\special{pa 100 1920}%
\special{pa 300 1920}%
\special{fp}%
% LINE 2 0 3 0 Black White  
% 2 150 1945 250 1945
% 
\special{pn 8}%
\special{pa 150 1945}%
\special{pa 250 1945}%
\special{fp}%
% LINE 2 0 3 0 Black White  
% 2 175 1970 225 1970
% 
\special{pn 8}%
\special{pa 175 1970}%
\special{pa 225 1970}%
\special{fp}%
% DOT 0 0 3 0 Black White  
% 3 200 1825 700 1825 700 1125
% 
\special{pn 4}%
\special{sh 1}%
\special{ar 200 1825 16 16 0 6.2831853}%
\special{sh 1}%
\special{ar 700 1825 16 16 0 6.2831853}%
\special{sh 1}%
\special{ar 700 1125 16 16 0 6.2831853}%
% CIRCLE 2 0 2 0 Black White  
% 4 700 625 725 625 725 625 725 625
% 
\special{sh 0}%
\special{ia 700 625 25 25 0.0000000 6.2831853}%
\special{pn 8}%
\special{ar 700 625 25 25 0.0000000 6.2831853}%
% STR 2 0 3 0 Black White  
% 4 785 575 785 625 5 0 0 0
% B
\put(7.8500,-6.2500){\makebox(0,0){B}}%
% STR 2 0 3 0 Black White  
% 4 685 375 685 425 5 0 0 0
% A
\put(6.8500,-4.2500){\makebox(0,0){A}}%
% VECTOR 2 0 3 0 Black White  
% 2 1385 425 1085 425
% 
\special{pn 8}%
\special{pa 1385 425}%
\special{pa 1085 425}%
\special{fp}%
\special{sh 1}%
\special{pa 1085 425}%
\special{pa 1152 445}%
\special{pa 1138 425}%
\special{pa 1152 405}%
\special{pa 1085 425}%
\special{fp}%
% VECTOR 2 0 3 0 Black White  
% 2 1385 725 1085 725
% 
\special{pn 8}%
\special{pa 1385 725}%
\special{pa 1085 725}%
\special{fp}%
\special{sh 1}%
\special{pa 1085 725}%
\special{pa 1152 745}%
\special{pa 1138 725}%
\special{pa 1152 705}%
\special{pa 1085 725}%
\special{fp}%
% STR 2 0 3 0 Black White  
% 4 1240 515 1240 565 5 0 0 0
% 光
\put(12.4000,-5.6500){\makebox(0,0){光}}%
% DOT 0 0 3 0 Black White  
% 1 1100 1825
% 
\special{pn 4}%
\special{sh 1}%
\special{ar 1100 1825 16 16 0 6.2831853}%
% STR 2 0 3 0 Black White  
% 4 700 2475 700 2525 5 0 0 0
% 図1
\put(7.0000,-25.2500){\makebox(0,0){図1}}%
% VECTOR 2 0 3 0 Black White  
% 4 200 3925 1500 3925 800 3925 800 2825
% 
\special{pn 8}%
\special{pa 200 3925}%
\special{pa 1500 3925}%
\special{fp}%
\special{sh 1}%
\special{pa 1500 3925}%
\special{pa 1433 3905}%
\special{pa 1447 3925}%
\special{pa 1433 3945}%
\special{pa 1500 3925}%
\special{fp}%
\special{pa 800 3925}%
\special{pa 800 2825}%
\special{fp}%
\special{sh 1}%
\special{pa 800 2825}%
\special{pa 780 2892}%
\special{pa 800 2878}%
\special{pa 820 2892}%
\special{pa 800 2825}%
\special{fp}%
% LINE 2 0 3 0 Black White  
% 2 830 3225 770 3225
% 
\special{pn 8}%
\special{pa 830 3225}%
\special{pa 770 3225}%
\special{fp}%
% STR 2 0 3 0 Black White  
% 4 800 3945 800 3995 5 0 0 0
% $0$
\put(8.0000,-39.9500){\makebox(0,0){$0$}}%
% LINE 2 0 3 0 Black White  
% 2 400 3900 400 3950
% 
\special{pn 8}%
\special{pa 400 3900}%
\special{pa 400 3950}%
\special{fp}%
% STR 2 0 3 0 Black White  
% 4 400 3950 400 4000 5 0 0 0
% $-1.8$
\put(4.0000,-40.0000){\makebox(0,0){$-1.8$}}%
% STR 2 0 3 0 Black White  
% 4 800 4150 800 4200 5 0 0 0
% 陽極の電位〔{\sf V}〕
\put(8.0000,-42.0000){\makebox(0,0){陽極の電位〔{\sf V}〕}}%
% STR 2 0 3 0 Black White  
% 4 880 2950 880 3000 2 0 0 0
% $I〔\times 10^{-6}\mathrm{{\sf A}}〕$
\put(8.8000,-30.0000){\makebox(0,0)[lb]{$I〔\times 10^{-6}\mathrm{{\sf A}}〕$}}%
% SPLINE 1 0 3 0 Black White  
% 5 400 3925 700 3625 800 3425 1000 3265 1300 3225
% 
\special{pn 13}%
\special{pa 400 3925}%
\special{pa 452 3885}%
\special{pa 504 3843}%
\special{pa 529 3822}%
\special{pa 553 3801}%
\special{pa 577 3779}%
\special{pa 599 3757}%
\special{pa 621 3733}%
\special{pa 642 3709}%
\special{pa 661 3685}%
\special{pa 679 3659}%
\special{pa 696 3632}%
\special{pa 711 3604}%
\special{pa 725 3576}%
\special{pa 738 3546}%
\special{pa 751 3517}%
\special{pa 779 3459}%
\special{pa 796 3432}%
\special{pa 814 3405}%
\special{pa 835 3380}%
\special{pa 857 3357}%
\special{pa 882 3335}%
\special{pa 908 3315}%
\special{pa 935 3297}%
\special{pa 963 3282}%
\special{pa 993 3268}%
\special{pa 1023 3257}%
\special{pa 1053 3248}%
\special{pa 1084 3241}%
\special{pa 1115 3235}%
\special{pa 1179 3229}%
\special{pa 1212 3227}%
\special{pa 1244 3226}%
\special{pa 1277 3225}%
\special{pa 1300 3225}%
\special{fp}%
% LINE 1 0 3 0 Black White  
% 2 1300 3225 1400 3225
% 
\special{pn 13}%
\special{pa 1300 3225}%
\special{pa 1400 3225}%
\special{fp}%
% STR 2 0 3 0 Black White  
% 4 700 4375 700 4425 5 0 0 0
% 図2
\put(7.0000,-44.2500){\makebox(0,0){図2}}%
\end{picture}}%

    }
ナトリウム Na を陰極とする光電管を用いて図1の回路を作り,波長 $3.0 \times 10^{-7}$\sftanni{m} の紫外線を当てて光電効果の実験を行った。光速度 $c = 3.0 \times 10^8$ \sftanni{m/s},電気素量 $e = 1.6 \times 10^{-19}$ \sftanni{C},プランク定数 $h = 6.6 \times 10^{-34}$ \tanni{J\cdot s} とする。
        \begin{enumerate}
            \item AB 間に十分な電圧をかけたところ,回路に $1.6 \times 10^{-6}$ \sftanni{A} の電流が流れた。陰極 A から陽極 B に達する電子の数 $N$ は毎秒何個か。
            \item AB 間の電圧を変えながら光電流 $I$ を測定すると,図2のようなグラフが得られた。陰極から飛び出す光電子の最大運動エネルギー $K$ \tanni{J} はいくらか。
            \item 光子のエネルギー \tanni{J} と Na の仕事関数 $W$ \tanni{J} を求めよ。また,$W$ を \tanni{eV} で表せ。そして Na に対する限界振動数 $\nu_0$ \tanni{Hz} を求めよ。
            \item 光の波長を変えずに光の明るさを半分にすると,図2の曲線はどう変わるか。図に概形を書き込め。
            \item 当てる光の波長を変えながら (2) と同様の実験を行う。横軸を光の振動数 $\nu$ \tanni{Hz},縦軸を $K$ \tanni{J} とする時,得られるグラフを文字 $\nu_0$ と $W$ を用いて定性的に示せ。また,$h$ はグラフの何と対応しているか。
        \end{enumerate}
\end{mawarikomi}
