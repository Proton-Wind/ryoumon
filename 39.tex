\hakosyokika
\item
    \begin{mawarikomi}{150pt}{図1{\unitlength5mm\small\Drawaxisfalse
\begin{Zahyou}[(-1,0)][(-0.1,-0.3)][(0,0.2)](-5,5)(-5,5)(-15,8)
    \def\Fx{cos(T)*3.5}
    \def\Fy{sin(T)*3.5}
    \def\Gx{cos(T)*0.3}
    \def\Gy{sin(T)*0.3}
    \def\O{(0,0,0)}
    \def\A{(5,5,0)}
    \def\B{(-5,5,0)}
    \def\C{(-5,-5,0)}
    \def\D{(5,-5,0)}
    \def\OD{(0,0,-7.5)}
    \def\ODD{(0,0,-15)}
    \calcval{150*2*$pi/360}\t
    \funcval\Fx{\t}\xi
    \funcval\Fy{\t}\yi
    \def\P{(\xi,\yi,0)}
    \calcval{(\xi-1*\yi)}\xxi
    \calcval{(\yi+\xi)}\yyi
    \def\Q{(\xxi,\yyi,0)}
    \iiiHenKo\P\O{$r$}
    \iiiKuromaru\P
    \iiiDrawline{\A\B\C\D\A}
    \iiiDrawline{\O\P}
    \iiiHenKo[0]\P\Q{$v_1$}
    \iiiDashline[80]{0.08}{\O\OD}
    \iiiDrawline{\OD\ODD}
    {\thicklines
    \iiiArrowLine\P\Q}
    \iiiBGurafu(.05)(.02)\Fx\Fy{0}{0}{$pi*2}
    \iiiBGurafu\Gx\Gy{0}{0}{$pi*2}
    {\KuromaruHankei{3pt}
    \iiiKuromaru\ODD
    }
    \iiiPut\O[nw]{O}
    \iiiPut\P[s]{A}
    \iiiPut\ODD[e]{B}
    % \iiiPut\O{\kaitenkigou< hazimekaku=20,
    %                         owarikaku=80,
    %                         tyouhankei=17mm,
    %                         tanhankei=5mm>[90]}
\end{Zahyou}}図2{\unitlength5mm\small\Drawaxisfalse
\begin{Zahyou}[(-1,0)][(-0.1,-0.3)][(0,0.2)](-5,5)(-5,5)(-15,8)
    \def\Fx{cos(T)*3.5}
    \def\Fy{sin(T)*3.5}
    \def\Gx{cos(T)*0.3}
    \def\Gy{sin(T)*0.3}
    \def\O{(0,0,0)}
    \def\A{(5,5,0)}
    \def\B{(-5,5,0)}
    \def\C{(-5,-5,0)}
    \def\D{(5,-5,0)}
    \def\E{(4,0,0)}
    \def\OD{(0,0,-7.5)}
    \def\ODD{(0,0,-15)}
    \calcval{230*2*$pi/360}\t
    \funcval\Fx{\t}\xi
    \funcval\Fy{\t}\yi
    \def\P{(\xi,\yi,0)}
    % \calcval{(\xi-1*\yi)}\xxi
    % \calcval{(\yi+\xi)}\yyi
    % \def\Q{(\xxi,\yyi,0)}
    \iiiNuritubusi[0.3]{\A\B\C\D\A}
    \Daen*[0]\O{.3}{.08}
    \iiiHenKo\O\P{$r$}
    \iiiKuromaru\P
    \iiiDrawline{\A\B\C\D\A}
    \iiiDrawline{\O\P}
    % \iiiHenKo[0]\P\Q{$v_1$}
    \iiiDashline[80]{0.08}{\O\OD}
    \iiiDrawline{\OD\ODD}
    % {\thicklines
    % \iiiArrowLine\P\Q}
    \iiiBGurafu(.05)(.02)\Fx\Fy{0}{0}{$pi*2}
    \iiiBGurafu\Gx\Gy{0}{0}{$pi*2}
    {\KuromaruHankei{3pt}
    \iiiKuromaru\ODD
    }
    \iiiPut\E[s]{$\omega$}
    \iiiPut\O[nw]{O}
    \iiiPut\P[s]{A}
    \iiiPut\ODD[e]{B}
    \iiiPut\O{\kaitenkigou< hazimekaku=20,
                            owarikaku=80,
                            tyouhankei=20mm,
                            tanhankei=7mm>[90]}
\end{Zahyou}}}
        水平な板にあけた小さな穴Oに糸を通し,その一端に質量$m$の小物体Aを結んで板の上に置き,他端に質量$M$のおもりBをつるす。糸と穴や板の間に摩擦はなく,重力加速度の大きさを$g$とする。
        \begin{enumerate}
            \item Aと板の間に摩擦はなく,図1のようにAは穴を中心とする半径$r$の等速円運動をしている。その速さ$v_1$を求めよ。
            \item Aと板の間に摩擦があり,静止摩擦係数を$\mu $とする。板を止め,Aを静かに放すとAは穴に向かって動くものとする。\\
            ~~そこで,図2のように,穴を中心として板を水平面内で角速度$\omega $で回転させ,Aを板上に置くと,板に対してAは静止した。Aと穴の距離を$r$として,Aが静止するための$\omega $の取りうる範囲を求めよ。
        \end{enumerate}
    \end{mawarikomi}