\hakosyokika
\item
    \begin{mawarikomi}{200pt}{\begin{zahyou*}[ul=5mm](-1,16)(-1,3)
	% \drawline(-1,-1)(-1,13)(13,13)(13,-1)(-1,-1)
	\def\Fx{0.2*(cos(T)+T/3.5)+2.04}
	\def\Fy{-0.3*sin(T)+0.6}
	\BGurafu\Fx\Fy{3.14}{20*3.14}
    \drawline(15,2)(15,0)
	\drawline(0,0)(15,0)
    \drawline(15,2)(15.5,2)
    \def\vvec{(2,0)}
    \def\A{(2,0)}
    \def\B{(2,1.5)}
    \def\C{(0.5,1.5)}
    \def\D{(0.5,0)}
    \def\E{(5.85,1.5)}
    \def\F{(5.85,0)}
    \def\G{(7.35,0)}
    \def\H{(7.35,1.5)}
    \def\I{(2,1.2)}
    \def\J{(5.85,1.2)}
    \small
    \Drawlines{\A\B\C\D\A}
    \Drawlines{\E\F\G\H\E}
    \Drawlines{\I\J}
    \Put\C[ne]{$2m$}
    \Put\D(8pt,6pt)[b]{A}
    \Put\E(8pt,2pt)[b]{$m$}
    \Put\F(8pt,6pt)[b]{B}
    \put(3,2){\Yasen\vvec}
    \put(3.8,2.2){$v$}
    \put(15.2,1){壁}

\end{zahyou*}
}
        質量$2m$\tanni{kg}の物体Aと質量$m$\tanni{kg}の物体Bとがあり,Aにはばね定数$k$\tanni{N/m}の軽いばねがつけられ,このばねを自然長より縮めた状態に保つため,BとAは糸で結ばれている。AとBは滑らかな水平床上を右方向へ速さ$v$\tanni{m/s}で動いている。ある点で糸が急に切れ,間もなくAは静止した。一方,Bはばねから離れて右方へ動き,壁と弾性衝突して左へ戻り,Aのばねに接触した。重力加速度の大きさを$g$\tanni{m/s^2}とする。
        \begin{enumerate}
            \item 糸が切れ,ばねから離れたときのBの速さはいくらか。
            \item はじめのばねの縮みはいくらであったか。
            \item 壁との衝突の際,Bが壁に与えた力積の大きさはいくらか。
            \item Bとばねが接触した後,ばねが最も縮んだときのBの速さはいくらか。
            \item Bとばねが接触した後,Bがばねから離れたときのAの速さはいくらか。
            \item 前問において,ばねから離れたBは図の左右どちらへ動くか。
        \end{enumerate}
    \end{mawarikomi}