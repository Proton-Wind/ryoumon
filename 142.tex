\hakosyokika
\item
    \begin{mawarikomi}{180pt}{
        % \begin{center}
        \input{./fig/fig142_1.tex}
        \\
        \\
        \hfill~%WinTpicVersion4.32a
{\unitlength 0.1in%
\begin{picture}(9.3000,9.2700)(3.4000,-12.5200)%
% BOX 2 0 3 0 Black White  
% 2 500 400 1200 435
% 
\special{pn 8}%
\special{pa 500 400}%
\special{pa 1200 400}%
\special{pa 1200 435}%
\special{pa 500 435}%
\special{pa 500 400}%
\special{pa 1200 400}%
\special{fp}%
% BOX 2 0 3 0 Black White  
% 2 500 1000 1200 1035
% 
\special{pn 8}%
\special{pa 500 1000}%
\special{pa 1200 1000}%
\special{pa 1200 1035}%
\special{pa 500 1035}%
\special{pa 500 1000}%
\special{pa 1200 1000}%
\special{fp}%
% CIRCLE 2 0 3 0 Black White  
% 4 850 700 885 700 885 700 885 700
% 
\special{pn 8}%
\special{ar 850 700 35 35 0.0000000 6.2831853}%
% STR 2 0 3 0 Black White  
% 4 705 650 705 700 5 0 0 0
% $q$
\put(7.0500,-7.0000){\makebox(0,0){$q$}}%
% VECTOR 2 0 3 0 Black White  
% 2 850 670 850 470
% 
\special{pn 8}%
\special{pa 850 670}%
\special{pa 850 470}%
\special{fp}%
\special{sh 1}%
\special{pa 850 470}%
\special{pa 830 537}%
\special{pa 850 523}%
\special{pa 870 537}%
\special{pa 850 470}%
\special{fp}%
% STR 2 0 3 0 Black White  
% 4 915 585 915 635 2 0 0 0
% $v_2$
\put(9.1500,-6.3500){\makebox(0,0)[lb]{$v_2$}}%
% STR 2 0 3 0 Black White  
% 4 375 365 375 415 5 0 0 0
% A
\put(3.7500,-4.1500){\makebox(0,0){A}}%
% STR 2 0 3 0 Black White  
% 4 375 965 375 1015 5 0 0 0
% B
\put(3.7500,-10.1500){\makebox(0,0){B}}%
% STR 2 0 3 0 Black White  
% 4 1270 1020 1270 1070 2 0 0 0
% $V${\sf〔V〕}
\put(12.7000,-10.7000){\makebox(0,0)[lb]{$V${\sf〔V〕}}}%
% STR 2 0 3 0 Black White  
% 4 1270 420 1270 470 2 0 0 0
% $0${\sf〔V〕}
\put(12.7000,-4.7000){\makebox(0,0)[lb]{$0${\sf〔V〕}}}%
% STR 2 0 3 0 Black White  
% 4 850 1275 850 1325 5 0 0 0
% 図2
\put(8.5000,-13.2500){\makebox(0,0){図2}}%
\end{picture}}%
~\hfill~
        % \end{center}
    }
\karaHako に語句または式を記し、問いに答えよ。

電気量には最小の単位があり、全ての電気量はその整数倍になっている。この最小単位を電気素量といい、これは $\boxed{\text{ (ア) }}$ のもっている電気量の大きさに等しい。ミリカンの実験は、図1のような装置に霧吹きから油滴を吹き込み、間隔 $d$\tanni{m} の平行な極板 A, B の間を上下する油滴を顕微鏡で観察し、電気素量 $e$\tanni{C} を測定した。密度 $\rho $\tanni{kg/m^3}、半径 $r$\tanni{m} の球形の油滴の運動を考える。重力加速度を $g$\tanni{m/s^2} とし、空気の浮力は無視する。

油滴は極板間に電場がないときは、重力と空気の抵抗力を受けて、鉛直下向きに一定の速さ(終端速度)$v_1$\tanni{m/s} で落下する。空気の抵抗力は $r$ と $v_1$ の積に比例するので、比例定数を $k$ とすると、この抵抗力と重力のつり合いの式は $\boxed{\text{ (イ) }}$ と書ける。

油滴は一般に帯電している。その電気量を $q$\tanni{C} とする。A に対する B の電位を $V$\tanni{V}$(V>0)$ とすると、油滴は図2に示すように、鉛直上向きに一定の速さ $v_2$\tanni{m/s} で上昇した。このときのつり合いの式は $\boxed{\text{ (ウ) }}$ となる。

(イ)と(ウ)より $q$ は $v_1, v_2, d, r, k, V$ を用いて、$q=\boxed{\text{ (エ) }}$ と表される。

    \begin{enumerate}[問1]
    \item 密度 $855$ \sftanni{kg/m^3} のパラフィン油を用いて測定したところ, ある油滴の $v_1$ は $3.0 \times 10^{-5} $\sftanni{m/s} であった。 $k$ は $3.41 \times 10^{-4} $\sftanni{kg/(m\cdot s)} なので, (イ)より $r=5.4 \times 10^{-7} $\tanni{m} であることがわかる。 この油滴は極板 A, B の間隔 $d$ が $5.0 \times 10^{-3} $\sftanni{m}, 電位 $V$ が $320 $\sftanni{V} のとき, $8.0 \times 10^{-5} $\sftanni{m/s} で上昇した。 油滴の電気量を求めよ。
    \item いろいろな油滴の電気量 $q $\tanni{C} を測定したところ, $6.4, 4.8, 11.3, 8.1$ (単位は$\times 10^{-19} $\sftanni{C}) を得た。 問1の結果も合わせて電気素量の値を求めよ。
    \end{enumerate}
\end{mawarikomi}
