\hakosyokika
\item
    \begin{mawarikomi}{180pt}{
        % \begin{center}
        %WinTpicVersion4.32a
{\unitlength 0.1in%
\begin{picture}(21.7000,21.4700)(1.1500,-22.6200)%
% POLYGON 2 5 0 0 Black White  
% 10 1235 1065 1235 1225 1211 1193 1201 1179 1195 1169 1187 1157 1187 1027 1205 1045 1219 1055 1235 1065
% 
\special{pn 0}%
\special{sh 0.400}%
\special{pa 1235 1065}%
\special{pa 1235 1225}%
\special{pa 1211 1193}%
\special{pa 1201 1179}%
\special{pa 1195 1169}%
\special{pa 1187 1157}%
\special{pa 1187 1027}%
\special{pa 1205 1045}%
\special{pa 1219 1055}%
\special{pa 1235 1065}%
\special{ip}%
\special{pn 8}%
\special{pa 1235 1065}%
\special{pa 1235 1225}%
\special{pa 1211 1193}%
\special{pa 1201 1179}%
\special{pa 1195 1169}%
\special{pa 1187 1157}%
\special{pa 1187 1027}%
\special{pa 1205 1045}%
\special{pa 1219 1055}%
\special{pa 1235 1065}%
\special{ip}%
% ELLIPSE 2 0 3 0 Black White  
% 4 600 645 690 690 690 690 690 690
% 
\special{pn 8}%
\special{ar 600 645 90 45 0.0000000 6.2831853}%
% LINE 2 0 3 0 Black White  
% 4 545 680 490 745 710 745 670 680
% 
\special{pn 8}%
\special{pa 545 680}%
\special{pa 490 745}%
\special{fp}%
\special{pa 710 745}%
\special{pa 670 680}%
\special{fp}%
% ELLIPSE 2 0 3 0 Black White  
% 4 600 710 750 760 355 790 855 790
% 
\special{pn 8}%
\special{ar 600 710 150 50 0.7551044 2.3665034}%
% LINE 2 0 3 0 Black White  
% 4 495 745 495 945 710 945 710 745
% 
\special{pn 8}%
\special{pa 495 745}%
\special{pa 495 945}%
\special{fp}%
\special{pa 710 945}%
\special{pa 710 745}%
\special{fp}%
% ELLIPSE 2 0 3 0 Black White  
% 4 600 880 755 975 455 975 765 975
% 
\special{pn 8}%
\special{ar 600 880 155 95 0.7541583 2.3228735}%
% SPLINE 2 0 3 0 Black White  
% 10 515 640 415 640 315 740 215 740 215 740 115 840 215 940 410 745 425 655 535 675
% 
\special{pn 8}%
\special{pa 515 640}%
\special{pa 481 634}%
\special{pa 448 632}%
\special{pa 418 638}%
\special{pa 394 657}%
\special{pa 372 683}%
\special{pa 351 710}%
\special{pa 329 733}%
\special{pa 302 743}%
\special{pa 272 742}%
\special{pa 238 739}%
\special{pa 204 742}%
\special{pa 171 756}%
\special{pa 143 779}%
\special{pa 123 807}%
\special{pa 115 837}%
\special{pa 121 868}%
\special{pa 140 895}%
\special{pa 166 919}%
\special{pa 198 935}%
\special{pa 231 942}%
\special{pa 263 940}%
\special{pa 294 930}%
\special{pa 322 912}%
\special{pa 348 888}%
\special{pa 370 859}%
\special{pa 388 825}%
\special{pa 401 789}%
\special{pa 409 750}%
\special{pa 412 711}%
\special{pa 415 677}%
\special{pa 425 655}%
\special{pa 447 648}%
\special{pa 479 654}%
\special{pa 516 668}%
\special{pa 535 675}%
\special{fp}%
% LINE 2 0 3 0 Black White  
% 2 685 630 1185 630
% 
\special{pn 8}%
\special{pa 685 630}%
\special{pa 1185 630}%
\special{fp}%
% LINE 2 0 3 0 Black White  
% 2 1185 660 685 660
% 
\special{pn 8}%
\special{pa 1185 660}%
\special{pa 685 660}%
\special{fp}%
% LINE 2 0 3 0 Black White  
% 2 1185 315 1185 1815
% 
\special{pn 8}%
\special{pa 1185 315}%
\special{pa 1185 1815}%
\special{fp}%
% LINE 2 0 3 0 Black White  
% 2 2285 315 2285 1815
% 
\special{pn 8}%
\special{pa 2285 315}%
\special{pa 2285 1815}%
\special{fp}%
% ELLIPSE 2 0 3 0 Black White  
% 4 1735 315 2285 515 2285 315 2285 315
% 
\special{pn 8}%
\special{ar 1735 315 550 200 0.0000000 6.2831853}%
% ELLIPSE 2 0 3 0 Black White  
% 4 1735 1815 2285 2115 1185 1815 2385 1815
% 
\special{pn 8}%
\special{ar 1735 1815 550 300 6.2831853 3.1415927}%
% ELLIPSE 2 0 3 0 Black White  
% 4 1735 300 2335 555 1135 555 2335 555
% 
\special{pn 8}%
\special{ar 1735 300 600 255 0.7853982 2.3561945}%
% POLYLINE 2 0 3 0 Black White  
% 17 1300 480 1370 780 1345 880 1445 1180 1345 1380 1445 1680 1470 1980 1565 2100 1580 2070 1485 1955 1465 1660 1370 1385 1470 1185 1370 870 1390 760 1330 485 1330 485
% 
\special{pn 8}%
\special{pa 1300 480}%
\special{pa 1370 780}%
\special{pa 1345 880}%
\special{pa 1445 1180}%
\special{pa 1345 1380}%
\special{pa 1445 1680}%
\special{pa 1470 1980}%
\special{pa 1565 2100}%
\special{pa 1580 2070}%
\special{pa 1485 1955}%
\special{pa 1465 1660}%
\special{pa 1370 1385}%
\special{pa 1470 1185}%
\special{pa 1370 870}%
\special{pa 1390 760}%
\special{pa 1330 485}%
\special{fp}%
% POLYLINE 2 0 3 0 Black White  
% 18 2155 485 2055 785 2080 905 2060 1205 2100 1410 2030 1610 2075 1900 2075 1900 1975 2075 1955 2045 2050 1895 2010 1605 2085 1405 2045 1205 2060 935 2040 775 2130 485 2130 485
% 
\special{pn 8}%
\special{pa 2155 485}%
\special{pa 2055 785}%
\special{pa 2080 905}%
\special{pa 2060 1205}%
\special{pa 2100 1410}%
\special{pa 2030 1610}%
\special{pa 2075 1900}%
\special{pa 1975 2075}%
\special{pa 1955 2045}%
\special{pa 2050 1895}%
\special{pa 2010 1605}%
\special{pa 2085 1405}%
\special{pa 2045 1205}%
\special{pa 2060 935}%
\special{pa 2040 775}%
\special{pa 2130 485}%
\special{fp}%
% ELLIPSE 2 0 3 0 Black White  
% 4 1745 1840 1255 2080 1570 2080 1990 2080
% 
\special{pn 8}%
\special{ar 1745 1840 490 240 1.1071487 1.9138203}%
% ELLIPSE 2 0 3 0 Black White  
% 4 1755 1840 2255 2080 2255 1440 1135 1440
% 
\special{pn 8}%
\special{ar 1755 1840 500 240 4.0725473 5.2529850}%
% ELLIPSE 2 0 3 0 Black White  
% 4 1735 1040 2280 1240 1350 1285 2215 1285
% 
\special{pn 8}%
\special{ar 1735 1040 545 200 0.9477204 2.0936426}%
% ELLIPSE 2 0 3 0 Black White  
% 4 1735 1015 2285 1215 1385 1240 2165 1240
% 
\special{pn 8}%
\special{ar 1735 1015 550 200 0.9636573 2.0854073}%
% ELLIPSE 2 0 3 0 Black White  
% 4 1735 1015 2285 1215 2190 790 1195 790
% 
\special{pn 8}%
\special{ar 1735 1015 550 200 3.9950478 5.3462575}%
% ELLIPSE 2 0 3 0 Black White  
% 4 1735 1015 1800 1040 1800 1040 1800 1040
% 
\special{pn 8}%
\special{ar 1735 1015 65 25 0.0000000 6.2831853}%
% ELLIPSE 2 0 3 0 Black White  
% 4 1735 1515 1185 1715 1085 1615 1382 1671
% 
\special{pn 8}%
\special{ar 1735 1515 550 200 2.2593121 2.7413520}%
% LINE 2 0 3 0 Black White  
% 2 1230 1595 1230 1795
% 
\special{pn 8}%
\special{pa 1230 1595}%
\special{pa 1230 1795}%
\special{fp}%
% LINE 2 0 3 0 Black White  
% 2 1385 1905 1385 1670
% 
\special{pn 8}%
\special{pa 1385 1905}%
\special{pa 1385 1670}%
\special{fp}%
% ELLIPSE 2 0 3 0 Black White  
% 4 1740 1570 1140 1995 835 1960 1280 2010
% 
\special{pn 8}%
\special{ar 1740 1570 600 425 2.2083451 2.5946987}%
% VECTOR 2 0 3 0 Black White  
% 2 1040 1870 1340 1735
% 
\special{pn 8}%
\special{pa 1040 1870}%
\special{pa 1340 1735}%
\special{fp}%
\special{sh 1}%
\special{pa 1340 1735}%
\special{pa 1271 1744}%
\special{pa 1291 1757}%
\special{pa 1287 1781}%
\special{pa 1340 1735}%
\special{fp}%
% STR 2 0 3 0 Black White  
% 4 990 1685 990 1735 5 0 0 0
% X
\put(9.9000,-17.3500){\makebox(0,0){X}}%
% STR 2 0 3 0 Black White  
% 4 990 1815 990 1865 5 0 0 0
% 線
\put(9.9000,-18.6500){\makebox(0,0){線}}%
% LINE 2 0 3 0 Black White  
% 6 1360 630 1715 630 1715 630 1755 660 1755 660 1370 660
% 
\special{pn 8}%
\special{pa 1360 630}%
\special{pa 1715 630}%
\special{fp}%
\special{pa 1715 630}%
\special{pa 1755 660}%
\special{fp}%
\special{pa 1755 660}%
\special{pa 1370 660}%
\special{fp}%
% LINE 2 0 3 0 Black White  
% 6 1705 660 1665 710 1665 710 1800 710 1800 710 1750 660
% 
\special{pn 8}%
\special{pa 1705 660}%
\special{pa 1665 710}%
\special{fp}%
\special{pa 1665 710}%
\special{pa 1800 710}%
\special{fp}%
\special{pa 1800 710}%
\special{pa 1750 660}%
\special{fp}%
% DOT 0 0 3 0 Black White  
% 9 1650 760 1750 860 1750 760 1865 860 1830 900 1830 1000 1675 930 1700 830 1600 930
% 
\special{pn 4}%
\special{sh 1}%
\special{ar 1650 760 16 16 0 6.2831853}%
\special{sh 1}%
\special{ar 1750 860 16 16 0 6.2831853}%
\special{sh 1}%
\special{ar 1750 760 16 16 0 6.2831853}%
\special{sh 1}%
\special{ar 1865 860 16 16 0 6.2831853}%
\special{sh 1}%
\special{ar 1830 900 16 16 0 6.2831853}%
\special{sh 1}%
\special{ar 1830 1000 16 16 0 6.2831853}%
\special{sh 1}%
\special{ar 1675 930 16 16 0 6.2831853}%
\special{sh 1}%
\special{ar 1700 830 16 16 0 6.2831853}%
\special{sh 1}%
\special{ar 1600 930 16 16 0 6.2831853}%
% DOT 0 0 3 0 Black White  
% 7 1700 1330 1800 1390 1870 1490 1760 1530 1760 1645 1860 1730 1660 1620
% 
\special{pn 4}%
\special{sh 1}%
\special{ar 1700 1330 16 16 0 6.2831853}%
\special{sh 1}%
\special{ar 1800 1390 16 16 0 6.2831853}%
\special{sh 1}%
\special{ar 1870 1490 16 16 0 6.2831853}%
\special{sh 1}%
\special{ar 1760 1530 16 16 0 6.2831853}%
\special{sh 1}%
\special{ar 1760 1645 16 16 0 6.2831853}%
\special{sh 1}%
\special{ar 1860 1730 16 16 0 6.2831853}%
\special{sh 1}%
\special{ar 1660 1620 16 16 0 6.2831853}%
% DOT 0 0 3 0 Black White  
% 4 1630 1795 1715 1710 1675 1420 1625 1535
% 
\special{pn 4}%
\special{sh 1}%
\special{ar 1630 1795 16 16 0 6.2831853}%
\special{sh 1}%
\special{ar 1715 1710 16 16 0 6.2831853}%
\special{sh 1}%
\special{ar 1675 1420 16 16 0 6.2831853}%
\special{sh 1}%
\special{ar 1625 1535 16 16 0 6.2831853}%
% STR 2 0 3 0 Black White  
% 4 1525 985 1525 1035 5 0 0 0
% A
\put(15.2500,-10.3500){\makebox(0,0){A}}%
% STR 2 0 3 0 Black White  
% 4 1585 1885 1585 1935 5 0 0 0
% B
\put(15.8500,-19.3500){\makebox(0,0){B}}%
% LINE 2 0 3 0 Black White  
% 10 1985 1735 2025 1735 2115 1735 2235 1735 2235 1785 2110 1785 2040 1785 2005 1785 2005 1785 1985 1735
% 
\special{pn 8}%
\special{pa 1985 1735}%
\special{pa 2025 1735}%
\special{fp}%
\special{pa 2115 1735}%
\special{pa 2235 1735}%
\special{fp}%
\special{pa 2235 1785}%
\special{pa 2110 1785}%
\special{fp}%
\special{pa 2040 1785}%
\special{pa 2005 1785}%
\special{fp}%
\special{pa 2005 1785}%
\special{pa 1985 1735}%
\special{fp}%
% ELLIPSE 2 0 3 0 Black White  
% 4 2110 1760 2095 1735 2110 1635 2110 1935
% 
\special{pn 8}%
\special{ar 2110 1760 15 25 1.5707963 4.7123890}%
% ELLIPSE 2 0 3 0 Black White  
% 4 2235 1760 2255 1735 2230 1655 2230 1655
% 
\special{pn 8}%
\special{ar 2235 1760 20 25 0.0000000 6.2831853}%
% STR 2 0 3 0 Black White  
% 4 2135 1650 2135 1700 2 0 1 0
% 顕微鏡
\put(21.3500,-17.0000){\makebox(0,0)[lb]{{\colorbox[named]{White}{顕微鏡}}}}%
% STR 2 0 3 0 Black White  
% 4 1805 740 1805 790 2 0 1 0
% 油滴
\put(18.0500,-7.9000){\makebox(0,0)[lb]{{\colorbox[named]{White}{油滴}}}}%
% STR 2 0 3 0 Black White  
% 4 460 1090 460 1140 2 0 0 0
% 油
\put(4.6000,-11.4000){\makebox(0,0)[lb]{油}}%
% STR 2 0 3 0 Black White  
% 4 160 560 160 610 2 0 0 0
% 霧吹き
\put(1.6000,-6.1000){\makebox(0,0)[lb]{霧吹き}}%
% ELLIPSE 2 0 3 0 Black White  
% 4 1735 1015 1185 1135 1085 1015 1060 1085
% 
\special{pn 8}%
\special{ar 1735 1015 550 120 2.6976911 3.1415927}%
% ELLIPSE 2 0 3 0 Black White  
% 4 1735 1140 1185 1340 1185 1140 1085 1240
% 
\special{pn 8}%
\special{ar 1735 1140 550 200 2.7413520 3.1415927}%
% LINE 2 0 3 0 Black White  
% 2 1235 1060 1235 1220
% 
\special{pn 8}%
\special{pa 1235 1060}%
\special{pa 1235 1220}%
\special{fp}%
% POLYGON 2 5 2 0 Black White  
% 5 1196 1067 1216 1080 1216 1100 1195 1084 1196 1067
% 
\special{pn 0}%
\special{sh 0}%
\special{pa 1196 1067}%
\special{pa 1216 1080}%
\special{pa 1216 1100}%
\special{pa 1195 1084}%
\special{pa 1196 1067}%
\special{ip}%
\special{pn 8}%
\special{pa 1196 1067}%
\special{pa 1216 1080}%
\special{pa 1216 1100}%
\special{pa 1195 1084}%
\special{pa 1196 1067}%
\special{ip}%
% LINE 2 0 3 0 Black White  
% 4 1195 1077 695 1077 695 1077 695 1377
% 
\special{pn 8}%
\special{pa 1195 1077}%
\special{pa 695 1077}%
\special{fp}%
\special{pa 695 1077}%
\special{pa 695 1377}%
\special{fp}%
% VECTOR 2 0 3 0 Black White  
% 2 695 1377 695 1477
% 
\special{pn 8}%
\special{pa 695 1377}%
\special{pa 695 1477}%
\special{fp}%
\special{sh 1}%
\special{pa 695 1477}%
\special{pa 715 1410}%
\special{pa 695 1424}%
\special{pa 675 1410}%
\special{pa 695 1477}%
\special{fp}%
% BOX 2 0 3 0 Black White  
% 2 495 1477 800 1517
% 
\special{pn 8}%
\special{pa 495 1477}%
\special{pa 800 1477}%
\special{pa 800 1517}%
\special{pa 495 1517}%
\special{pa 495 1477}%
\special{pa 800 1477}%
\special{fp}%
% LINE 2 0 3 0 Black White  
% 14 800 1497 870 1497 870 1497 870 1997 1295 1997 870 1997 270 1997 270 1997 870 1997 370 1997 370 1997 370 1497 370 1497 495 1497
% 
\special{pn 8}%
\special{pa 800 1497}%
\special{pa 870 1497}%
\special{fp}%
\special{pa 870 1497}%
\special{pa 870 1997}%
\special{fp}%
\special{pa 1295 1997}%
\special{pa 870 1997}%
\special{fp}%
\special{pa 270 1997}%
\special{pa 270 1997}%
\special{fp}%
\special{pa 870 1997}%
\special{pa 370 1997}%
\special{fp}%
\special{pa 370 1997}%
\special{pa 370 1497}%
\special{fp}%
\special{pa 370 1497}%
\special{pa 495 1497}%
\special{fp}%
% DOT 0 0 3 0 Black White  
% 1 870 1997
% 
\special{pn 4}%
\special{sh 1}%
\special{ar 870 1997 16 16 0 6.2831853}%
% BOX 2 5 2 0 Black White  
% 2 330 1702 430 1737
% 
\special{pn 0}%
\special{sh 0}%
\special{pa 330 1702}%
\special{pa 430 1702}%
\special{pa 430 1737}%
\special{pa 330 1737}%
\special{pa 330 1702}%
\special{ip}%
\special{pn 8}%
\special{pa 330 1702}%
\special{pa 430 1702}%
\special{pa 430 1737}%
\special{pa 330 1737}%
\special{pa 330 1702}%
\special{ip}%
% LINE 2 0 3 0 Black White  
% 2 340 1702 410 1702
% 
\special{pn 8}%
\special{pa 340 1702}%
\special{pa 410 1702}%
\special{fp}%
% LINE 2 0 3 0 Black White  
% 2 310 1732 440 1732
% 
\special{pn 8}%
\special{pa 310 1732}%
\special{pa 440 1732}%
\special{fp}%
% STR 2 0 3 0 Black White  
% 4 1185 2285 1185 2335 5 0 0 0
% 図1
\put(11.8500,-23.3500){\makebox(0,0){図1}}%
% DOT 0 0 3 0 Black White  
% 3 1805 800 1740 960 1630 985
% 
\special{pn 4}%
\special{sh 1}%
\special{ar 1805 800 16 16 0 6.2831853}%
\special{sh 1}%
\special{ar 1740 960 16 16 0 6.2831853}%
\special{sh 1}%
\special{ar 1630 985 16 16 0 6.2831853}%
\end{picture}}%

        \\
        \\
        \hfill~%WinTpicVersion4.32a
{\unitlength 0.1in%
\begin{picture}(9.3000,9.2700)(3.4000,-12.5200)%
% BOX 2 0 3 0 Black White  
% 2 500 400 1200 435
% 
\special{pn 8}%
\special{pa 500 400}%
\special{pa 1200 400}%
\special{pa 1200 435}%
\special{pa 500 435}%
\special{pa 500 400}%
\special{pa 1200 400}%
\special{fp}%
% BOX 2 0 3 0 Black White  
% 2 500 1000 1200 1035
% 
\special{pn 8}%
\special{pa 500 1000}%
\special{pa 1200 1000}%
\special{pa 1200 1035}%
\special{pa 500 1035}%
\special{pa 500 1000}%
\special{pa 1200 1000}%
\special{fp}%
% CIRCLE 2 0 3 0 Black White  
% 4 850 700 885 700 885 700 885 700
% 
\special{pn 8}%
\special{ar 850 700 35 35 0.0000000 6.2831853}%
% STR 2 0 3 0 Black White  
% 4 705 650 705 700 5 0 0 0
% $q$
\put(7.0500,-7.0000){\makebox(0,0){$q$}}%
% VECTOR 2 0 3 0 Black White  
% 2 850 670 850 470
% 
\special{pn 8}%
\special{pa 850 670}%
\special{pa 850 470}%
\special{fp}%
\special{sh 1}%
\special{pa 850 470}%
\special{pa 830 537}%
\special{pa 850 523}%
\special{pa 870 537}%
\special{pa 850 470}%
\special{fp}%
% STR 2 0 3 0 Black White  
% 4 915 585 915 635 2 0 0 0
% $v_2$
\put(9.1500,-6.3500){\makebox(0,0)[lb]{$v_2$}}%
% STR 2 0 3 0 Black White  
% 4 375 365 375 415 5 0 0 0
% A
\put(3.7500,-4.1500){\makebox(0,0){A}}%
% STR 2 0 3 0 Black White  
% 4 375 965 375 1015 5 0 0 0
% B
\put(3.7500,-10.1500){\makebox(0,0){B}}%
% STR 2 0 3 0 Black White  
% 4 1270 1020 1270 1070 2 0 0 0
% $V${\sf〔V〕}
\put(12.7000,-10.7000){\makebox(0,0)[lb]{$V${\sf〔V〕}}}%
% STR 2 0 3 0 Black White  
% 4 1270 420 1270 470 2 0 0 0
% $0${\sf〔V〕}
\put(12.7000,-4.7000){\makebox(0,0)[lb]{$0${\sf〔V〕}}}%
% STR 2 0 3 0 Black White  
% 4 850 1275 850 1325 5 0 0 0
% 図2
\put(8.5000,-13.2500){\makebox(0,0){図2}}%
\end{picture}}%
~\hfill~
        % \end{center}
    }
\karaHako に語句または式を記し、問いに答えよ。

電気量には最小の単位があり、全ての電気量はその整数倍になっている。この最小単位を電気素量といい、これは $\boxed{\text{ (ア) }}$ のもっている電気量の大きさに等しい。ミリカンの実験は、図1のような装置に霧吹きから油滴を吹き込み、間隔 $d$\tanni{m} の平行な極板 A, B の間を上下する油滴を顕微鏡で観察し、電気素量 $e$\tanni{C} を測定した。密度 $\rho $\tanni{kg/m^3}、半径 $r$\tanni{m} の球形の油滴の運動を考える。重力加速度を $g$\tanni{m/s^2} とし、空気の浮力は無視する。

油滴は極板間に電場がないときは、重力と空気の抵抗力を受けて、鉛直下向きに一定の速さ(終端速度)$v_1$\tanni{m/s} で落下する。空気の抵抗力は $r$ と $v_1$ の積に比例するので、比例定数を $k$ とすると、この抵抗力と重力のつり合いの式は $\boxed{\text{ (イ) }}$ と書ける。

油滴は一般に帯電している。その電気量を $q$\tanni{C} とする。A に対する B の電位を $V$\tanni{V}$(V>0)$ とすると、油滴は図2に示すように、鉛直上向きに一定の速さ $v_2$\tanni{m/s} で上昇した。このときのつり合いの式は $\boxed{\text{ (ウ) }}$ となる。

(イ)と(ウ)より $q$ は $v_1, v_2, d, r, k, V$ を用いて、$q=\boxed{\text{ (エ) }}$ と表される。

    \begin{enumerate}[問1]
    \item 密度 $855$ \sftanni{kg/m^3} のパラフィン油を用いて測定したところ, ある油滴の $v_1$ は $3.0 \times 10^{-5} $\sftanni{m/s} であった。 $k$ は $3.41 \times 10^{-4} $\sftanni{kg/(m\cdot s)} なので, (イ)より $r=5.4 \times 10^{-7} $\tanni{m} であることがわかる。 この油滴は極板 A, B の間隔 $d$ が $5.0 \times 10^{-3} $\sftanni{m}, 電位 $V$ が $320 $\sftanni{V} のとき, $8.0 \times 10^{-5} $\sftanni{m/s} で上昇した。 油滴の電気量を求めよ。
    \item いろいろな油滴の電気量 $q $\tanni{C} を測定したところ, $6.4, 4.8, 11.3, 8.1$ (単位は$\times 10^{-19} $\sftanni{C}) を得た。 問1の結果も合わせて電気素量の値を求めよ。
    \end{enumerate}
\end{mawarikomi}
