\item
    \begin{mawarikomi}{180pt}{\begin{zahyou*}[ul=4.5mm,yscale=1.5,xscale=1.5](-1,13)(-1,10)

	\def\C{(0,9.5)}
	\def\D{(0,10)}
	\def\E{(1,10)}
	\def\F{(1,9.5)}
	\def\fvec{(2,0)}
	\def\P{(0.5,9.5)}
	\def\PD{(0.5,6)}
	\def\A{(2.3,5)}
	\def\AD{(2.3,3)}
	\def\B{(8,2)}
	\def\M{(6,2.8)}
	\def\N{(9,1.5)}
	\Drawlines{\F\C}
	\Drawlines{\P\A}
		{
		\Thicklines
		\Drawlines{\B\A}
		}
	% \scriptsize
	\HenKo<henkoH=2ex>\B\A{$\ell $}
	\Put\A[ne]{A}
	\Put\B[ne]{B}
	\Put\P[sw]{P}
	{\thicklines
	\Put\B{\Yasen\fvec}
	}
	\Put\B[s]{$2m$}
	\Put\N(0pt,5pt)[t]{$F$}
	\Put\M[s]{$m$}
	\En*[1]\B{0.2}
	\Hasen{\P\PD}
	\Hasen{\A\AD}
	\Kakukigou\PD\P\A<hankei=2>[s]{$\alpha$}
	\Kakukigou\AD\A\B[se]{$\beta$}
	\Nuritubusi*<0.22>{\C\D\E\F\C}

\end{zahyou*}
}
    図のように,長さ$\ell $,質量$m$の一様な棒ABのB端に,質量$2m$の小球を取り付け,Aに軽い糸を結び点Pからつるす。小球に水平方向の力$F$を加えたところ,糸PAおよび棒ABと鉛直線のなす角度がそれぞれ$\alpha $および$\beta $となってつり合った。重力加速度の大きさを$g$とする。
        \begin{enumerate}
            \item 棒と小球全体の重心Gはどこになるか。Aからの距離を求めよ。
            \item 糸の張力を$T$として,水平方向および鉛直方向での力のつり合いの式をそれぞれ記せ。
            \item Aのまわりの力のモーメントのつり合いの式を記せ。
            \item $\tan{\alpha}$と$\tan{\beta }$および$T$を,それぞれ$m$,$g$,$F$を用いて表せ。
        \end{enumerate}
    \end{mawarikomi}