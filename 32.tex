\hakosyokika
\item
    \begin{mawarikomi}{180pt}{\begin{zahyou*}[ul=6mm](-2,9)(-1,5)
	\drawline(1.04,3.5)(0,3.5)(0,0)
	\drawline(-2,0)(9,0)
	\drawline(8,0)(8,1.5)(3.5,1.5)
    \def\A{(1.4,2.8)}
    \def\B{(3.5,1.5)}
    \def\BU{(3.5,2)}
    \def\BD{(3.5,1.2)}
    \def\C{(6,1.5)}
    \def\CU{(6,2)}
    \def\D{(3.5,4)}
    \def\E{(0,1.5)}
    \def\F{(0,2.8)}
    \def\G{(8,1.5)}
    \def\GD{(8,1.2)}
    \def\P{(3.6,1.7)}
    \def\Q{(3.4,1.7)}
    \def\R{(3.4,1.5)}
    \def\S{(3.6,1.5)}

    \Enko\D{2.5}{191}{270}
    \small
    \Put\A(-5pt,-6pt)[b]{A}
    \Put\BU[n]{B}
    \Put\CU[n]{C}
    \HenKo<henkotype=parallel,
            yazirusi=b,
            henkosideb=0,
            henkosidet=1.5>\F\E{$h$}
    \HenKo<henkotype=parallel,
            yazirusi=b,
            henkosideb=0,
            henkosidet=1.5>\C\B{$\ell $}
    \Hasen{\F\A}
    \Hasen{\E\B}
    \Kaiten\D\P{-60}\PP
    \Kaiten\D\Q{-60}\QQ
    \Kaiten\D\R{-60}\RR
    \Kaiten\D\S{-60}\SS
    \Drawline{\PP\QQ\RR\SS\PP}
    \Nuritubusi[1]{\PP\QQ\RR\SS\PP}
    \Nuritubusi*<0.15>{\B\BD\GD\G\B}
    \put(3,0.5){台}
    \put(4,0.5){$M$}
    \put(4,-0.5){床}
    \put(1.5,3){P}
\end{zahyou*}
}
        質量$M$の台が水平な床上に置かれている。この台の上面では,摩擦がない曲面と摩擦のある水平面が点Bでなめらかにつながっている。台の水平面から高さ$h$にある点Aに質量$m$の小物体Pを置き,静かに放す。重力加速度の大きさを$g$とする。
        \begin{enumerate}
            \item 台が床に固定されているとき,Pは点Bまですべり落ちたのち,点Bから距離$\ell $だけ離れた点Cで止まった。BC間の水平面とPの間の動摩擦係数$\mu$はいくらか。
            \item 次に,台が床の上で摩擦なく自由に動くことができるようにした。台が静止した状態で,点AからPを静かに放した。Pが台上の点Bに達したときの,Pの床に対する速度を$v$,台の床に対する速度を$V$とする。ただし,速度は右向きを正とする。
                \begin{enumerate}
                    \item このとき,$v$と$V$が満たすべき関係式を2つ書け。
                    \item $v$と$V$を求め,それぞれ,$h$,$m$,$M$,$g$で表せ。
                    \item Pは点Bを通り過ぎたのち,やがて台に対して静止した。このとき,台の床に対する運動はどうなるか。次のうちから選べ。
                    \begin{enumerate}[m]
                        \item Pが静止しても,台は動くがその進方向は点Pの高さ$h$によって決まる。
                        \item Pと台の間の摩擦により,Pが停止しても台は右向きに進む。
                        \item Pが曲面を下っている間は,台は小物体と反対方向に進むので,Pが停止しても,慣性の法則により台は左方向に進む。
                        \item Pと台を合わせた全体には水平方向に外力が働かないため,Pが台に対して停止すると台も停止する。
                    \end{enumerate}
                \end{enumerate}
        \end{enumerate}
    \end{mawarikomi}