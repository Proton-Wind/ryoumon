\hakosyokika
\item
    \begin{mawarikomi}(10pt,0){120pt}{%WinTpicVersion4.32a
{\unitlength 0.1in%
\begin{picture}(13.9370,28.7598)(2.6181,-28.9567)%
% LINE 2 0 3 0 Black Black  
% 6 400 600 450 600 450 600 450 490 450 490 400 490
% 
\special{pn 8}%
\special{pa 394 591}%
\special{pa 443 591}%
\special{fp}%
\special{pa 443 591}%
\special{pa 443 482}%
\special{fp}%
\special{pa 443 482}%
\special{pa 394 482}%
\special{fp}%
% LINE 2 0 3 0 Black Black  
% 2 678 516 688 506
% 
\special{pn 8}%
\special{pa 667 508}%
\special{pa 677 498}%
\special{fp}%
% LINE 2 0 3 0 Black Black  
% 2 678 580 686 588
% 
\special{pn 8}%
\special{pa 667 571}%
\special{pa 675 579}%
\special{fp}%
% LINE 2 0 3 0 Black Black  
% 6 918 2302 938 2302 938 2302 938 2338 938 2338 920 2338
% 
\special{pn 8}%
\special{pa 904 2266}%
\special{pa 923 2266}%
\special{fp}%
\special{pa 923 2266}%
\special{pa 923 2301}%
\special{fp}%
\special{pa 923 2301}%
\special{pa 906 2301}%
\special{fp}%
% LINE 2 0 3 0 Black Black  
% 6 932 528 952 528 952 528 952 562 952 562 932 562
% 
\special{pn 8}%
\special{pa 917 520}%
\special{pa 937 520}%
\special{fp}%
\special{pa 937 520}%
\special{pa 937 553}%
\special{fp}%
\special{pa 937 553}%
\special{pa 917 553}%
\special{fp}%
% LINE 2 0 3 0 Black Black  
% 14 372 282 652 282 652 282 572 202 572 202 452 202 452 202 372 282 452 162 452 162 452 202 452 162 452 162 572 162
% 
\special{pn 8}%
\special{pa 366 278}%
\special{pa 642 278}%
\special{fp}%
\special{pa 642 278}%
\special{pa 563 199}%
\special{fp}%
\special{pa 563 199}%
\special{pa 445 199}%
\special{fp}%
\special{pa 445 199}%
\special{pa 366 278}%
\special{fp}%
\special{pa 445 159}%
\special{pa 445 159}%
\special{fp}%
\special{pa 445 199}%
\special{pa 445 159}%
\special{fp}%
\special{pa 445 159}%
\special{pa 563 159}%
\special{fp}%
% LINE 2 0 3 0 Black Black  
% 2 572 202 572 162
% 
\special{pn 8}%
\special{pa 563 199}%
\special{pa 563 159}%
\special{fp}%
% STR 2 0 3 0 Black Black  
% 4 508 76 508 96 5 0 0 0
% スピーカー
\put(5.0000,-0.9449){\makebox(0,0){スピーカー}}%
% CIRCLE 2 0 3 0 Black White  
% 4 500 2200 400 2200 200 2200 800 2200
% 
\special{pn 8}%
\special{ar 492 2165 98 98 6.2831853 3.1415927}%
% LINE 2 0 3 0 Black Black  
% 16 400 2300 800 2300 800 2300 820 2280 820 2280 920 2280 920 2280 920 2360 920 2360 820 2360 820 2360 800 2340 800 2340 400 2340 400 2340 400 2300
% 
\special{pn 8}%
\special{pa 394 2264}%
\special{pa 787 2264}%
\special{fp}%
\special{pa 787 2264}%
\special{pa 807 2244}%
\special{fp}%
\special{pa 807 2244}%
\special{pa 906 2244}%
\special{fp}%
\special{pa 906 2244}%
\special{pa 906 2323}%
\special{fp}%
\special{pa 906 2323}%
\special{pa 807 2323}%
\special{fp}%
\special{pa 807 2323}%
\special{pa 787 2303}%
\special{fp}%
\special{pa 787 2303}%
\special{pa 394 2303}%
\special{fp}%
\special{pa 394 2303}%
\special{pa 394 2264}%
\special{fp}%
% LINE 2 0 3 0 Black White  
% 2 600 1000 400 1000
% 
\special{pn 8}%
\special{pa 591 984}%
\special{pa 394 984}%
\special{fp}%
% LINE 2 0 3 0 Black Black  
% 4 520 2340 520 2440 480 2440 480 2340
% 
\special{pn 8}%
\special{pa 512 2303}%
\special{pa 512 2402}%
\special{fp}%
\special{pa 472 2402}%
\special{pa 472 2303}%
\special{fp}%
% BOX 2 5 2 0 Black White  
% 2 828 502 920 590
% 
\special{pn 0}%
\special{sh 0}%
\special{pa 815 494}%
\special{pa 906 494}%
\special{pa 906 581}%
\special{pa 815 581}%
\special{pa 815 494}%
\special{ip}%
\special{pn 8}%
\special{pa 815 494}%
\special{pa 906 494}%
\special{pa 906 581}%
\special{pa 815 581}%
\special{pa 815 494}%
\special{ip}%
% LINE 2 0 3 0 Black Black  
% 6 684 590 934 590 934 590 934 500 934 500 688 500
% 
\special{pn 8}%
\special{pa 673 581}%
\special{pa 919 581}%
\special{fp}%
\special{pa 919 581}%
\special{pa 919 492}%
\special{fp}%
\special{pa 919 492}%
\special{pa 677 492}%
\special{fp}%
% LINE 2 0 3 0 Black Black  
% 2 400 390 400 2196
% 
\special{pn 8}%
\special{pa 394 384}%
\special{pa 394 2161}%
\special{fp}%
% LINE 2 0 3 0 Black Black  
% 2 600 390 600 2196
% 
\special{pn 8}%
\special{pa 591 384}%
\special{pa 591 2161}%
\special{fp}%
% BOX 2 5 2 0 Black White  
% 2 580 580 630 520
% 
\special{pn 0}%
\special{sh 0}%
\special{pa 571 571}%
\special{pa 620 571}%
\special{pa 620 512}%
\special{pa 571 512}%
\special{pa 571 571}%
\special{ip}%
\special{pn 8}%
\special{pa 571 571}%
\special{pa 620 571}%
\special{pa 620 512}%
\special{pa 571 512}%
\special{pa 571 571}%
\special{ip}%
% LINE 2 0 3 0 Black Black  
% 12 598 488 538 488 538 488 538 598 538 598 598 598 598 598 598 598 678 518 598 518 598 580 678 580
% 
\special{pn 8}%
\special{pa 589 480}%
\special{pa 530 480}%
\special{fp}%
\special{pa 530 480}%
\special{pa 530 589}%
\special{fp}%
\special{pa 530 589}%
\special{pa 589 589}%
\special{fp}%
\special{pa 589 589}%
\special{pa 589 589}%
\special{fp}%
\special{pa 667 510}%
\special{pa 589 510}%
\special{fp}%
\special{pa 589 571}%
\special{pa 667 571}%
\special{fp}%
% ELLIPSE 2 0 3 0 Black Black  
% 4 1040 2440 520 2720 440 2440 1280 2640
% 
\special{pn 8}%
\special{ar 1024 2402 512 276 0.9966033 3.1415927}%
% ELLIPSE 2 0 3 0 Black Black  
% 4 1040 2440 480 2760 440 2440 1480 2800
% 
\special{pn 8}%
\special{ar 1024 2402 551 315 0.9611365 3.1415927}%
% ELLIPSE 2 0 3 0 Black Black  
% 4 1132 2439 1492 2719 1532 2919 1772 2079
% 
\special{pn 8}%
\special{ar 1114 2401 354 276 5.6569020 0.9956184}%
% ELLIPSE 2 0 3 0 Black Black  
% 4 1132 2439 1538 2759 1458 2839 1458 2239
% 
\special{pn 8}%
\special{ar 1114 2401 400 315 5.6212932 0.9993592}%
% ELLIPSE 2 0 3 0 Black Black  
% 4 1464 1654 1304 2296 1224 1656 1434 2416
% 
\special{pn 8}%
\special{ar 1441 1628 157 632 1.7273982 3.1415927}%
% ELLIPSE 2 0 3 0 Black Black  
% 4 1488 1642 1342 2262 1222 1640 1444 2440
% 
\special{pn 8}%
\special{ar 1465 1616 144 610 1.8007008 3.1415927}%
% LINE 2 0 3 0 Black White  
% 2 1004 1156 1004 476
% 
\special{pn 8}%
\special{pa 988 1138}%
\special{pa 988 469}%
\special{fp}%
% LINE 2 0 3 0 Black White  
% 2 1644 476 1644 1156
% 
\special{pn 8}%
\special{pa 1618 469}%
\special{pa 1618 1138}%
\special{fp}%
% CIRCLE 2 0 3 0 Black White  
% 4 1724 476 1644 476 1644 396 1604 476
% 
\special{pn 8}%
\special{ar 1697 469 79 79 3.1415927 3.9269908}%
% CIRCLE 2 0 3 0 Black White  
% 4 924 476 1004 476 1044 476 1004 396
% 
\special{pn 8}%
\special{ar 909 469 79 79 5.4977871 6.2831853}%
% LINE 2 0 3 0 Black White  
% 2 1004 996 1644 996
% 
\special{pn 8}%
\special{pa 988 980}%
\special{pa 1618 980}%
\special{fp}%
% LINE 2 0 3 0 Black White  
% 2 1346 1556 1346 1476
% 
\special{pn 8}%
\special{pa 1325 1531}%
\special{pa 1325 1453}%
\special{fp}%
% LINE 2 0 3 0 Black White  
% 2 1306 1476 1306 1556
% 
\special{pn 8}%
\special{pa 1285 1453}%
\special{pa 1285 1531}%
\special{fp}%
% CIRCLE 2 0 3 0 Black White  
% 4 1324 1156 1644 1156 1004 1156 1306 1436
% 
\special{pn 8}%
\special{ar 1303 1138 315 315 1.6349937 3.1415927}%
% CIRCLE 2 0 3 0 Black White  
% 4 1324 1156 1644 1156 1354 1596 1674 1156
% 
\special{pn 8}%
\special{ar 1303 1138 315 315 6.2831853 1.5027199}%
% LINE 2 0 3 0 Black Black  
% 6 846 2700 846 2360 846 2278 846 590 846 500 846 300
% 
\special{pn 8}%
\special{pa 833 2657}%
\special{pa 833 2323}%
\special{fp}%
\special{pa 833 2242}%
\special{pa 833 581}%
\special{fp}%
\special{pa 833 492}%
\special{pa 833 295}%
\special{fp}%
% LINE 2 0 3 0 Black Black  
% 6 906 300 906 500 906 590 906 2278 906 2362 906 2712
% 
\special{pn 8}%
\special{pa 892 295}%
\special{pa 892 492}%
\special{fp}%
\special{pa 892 581}%
\special{pa 892 2242}%
\special{fp}%
\special{pa 892 2325}%
\special{pa 892 2669}%
\special{fp}%
% LINE 2 0 3 0 Black Black  
% 2 844 2742 844 2862
% 
\special{pn 8}%
\special{pa 831 2699}%
\special{pa 831 2817}%
\special{fp}%
% LINE 2 0 3 0 Black Black  
% 2 904 2862 904 2752
% 
\special{pn 8}%
\special{pa 890 2817}%
\special{pa 890 2709}%
\special{fp}%
% LINE 2 0 3 0 Black Black  
% 6 842 2862 402 2862 322 2942 1682 2942 1602 2862 902 2862
% 
\special{pn 8}%
\special{pa 829 2817}%
\special{pa 396 2817}%
\special{fp}%
\special{pa 317 2896}%
\special{pa 1656 2896}%
\special{fp}%
\special{pa 1577 2817}%
\special{pa 888 2817}%
\special{fp}%
% CIRCLE 2 0 3 0 Black Black  
% 4 402 2942 402 2862 402 2862 322 2942
% 
\special{pn 8}%
\special{ar 396 2896 79 79 3.1415927 4.7123890}%
% CIRCLE 2 0 3 0 Black Black  
% 4 1602 2942 1602 2862 1682 2942 1602 2822
% 
\special{pn 8}%
\special{ar 1577 2896 79 79 4.7123890 6.2831853}%
% BOX 2 0 2 0 Black White  
% 2 1288 1514 1360 1684
% 
\special{pn 0}%
\special{sh 0}%
\special{pa 1268 1490}%
\special{pa 1339 1490}%
\special{pa 1339 1657}%
\special{pa 1268 1657}%
\special{pa 1268 1490}%
\special{ip}%
\special{pn 8}%
\special{pa 1268 1490}%
\special{pa 1339 1490}%
\special{pa 1339 1657}%
\special{pa 1268 1657}%
\special{pa 1268 1490}%
\special{pa 1339 1490}%
\special{fp}%
% LINE 3 0 3 0 Black White  
% 78 1640 1024 1212 1452 1640 1048 1230 1458 1640 1072 1248 1464 1640 1096 1268 1468 1640 1120 1290 1470 1640 1144 1308 1476 1640 1168 1308 1500 1342 1490 1322 1510 1638 1194 1362 1470 1634 1222 1390 1466 1624 1256 1424 1456 1608 1296 1464 1440 1640 1000 1196 1444 1618 998 1180 1436 1594 998 1164 1428 1570 998 1150 1418 1546 998 1134 1410 1522 998 1122 1398 1498 998 1110 1386 1474 998 1096 1376 1450 998 1086 1362 1426 998 1074 1350 1402 998 1064 1336 1378 998 1054 1322 1354 998 1046 1306 1330 998 1038 1290 1306 998 1030 1274 1282 998 1024 1256 1258 998 1018 1238 1234 998 1014 1218 1210 998 1010 1198 1186 998 1008 1176 1162 998 1006 1154 1138 998 1006 1130 1114 998 1006 1106 1090 998 1006 1082 1066 998 1006 1058 1042 998 1006 1034 1018 998 1006 1010
% 
\special{pn 4}%
\special{pa 1614 1008}%
\special{pa 1193 1429}%
\special{fp}%
\special{pa 1614 1031}%
\special{pa 1211 1435}%
\special{fp}%
\special{pa 1614 1055}%
\special{pa 1228 1441}%
\special{fp}%
\special{pa 1614 1079}%
\special{pa 1248 1445}%
\special{fp}%
\special{pa 1614 1102}%
\special{pa 1270 1447}%
\special{fp}%
\special{pa 1614 1126}%
\special{pa 1287 1453}%
\special{fp}%
\special{pa 1614 1150}%
\special{pa 1287 1476}%
\special{fp}%
\special{pa 1321 1467}%
\special{pa 1301 1486}%
\special{fp}%
\special{pa 1612 1175}%
\special{pa 1341 1447}%
\special{fp}%
\special{pa 1608 1203}%
\special{pa 1368 1443}%
\special{fp}%
\special{pa 1598 1236}%
\special{pa 1402 1433}%
\special{fp}%
\special{pa 1583 1276}%
\special{pa 1441 1417}%
\special{fp}%
\special{pa 1614 984}%
\special{pa 1177 1421}%
\special{fp}%
\special{pa 1593 982}%
\special{pa 1161 1413}%
\special{fp}%
\special{pa 1569 982}%
\special{pa 1146 1406}%
\special{fp}%
\special{pa 1545 982}%
\special{pa 1132 1396}%
\special{fp}%
\special{pa 1522 982}%
\special{pa 1116 1388}%
\special{fp}%
\special{pa 1498 982}%
\special{pa 1104 1376}%
\special{fp}%
\special{pa 1474 982}%
\special{pa 1093 1364}%
\special{fp}%
\special{pa 1451 982}%
\special{pa 1079 1354}%
\special{fp}%
\special{pa 1427 982}%
\special{pa 1069 1341}%
\special{fp}%
\special{pa 1404 982}%
\special{pa 1057 1329}%
\special{fp}%
\special{pa 1380 982}%
\special{pa 1047 1315}%
\special{fp}%
\special{pa 1356 982}%
\special{pa 1037 1301}%
\special{fp}%
\special{pa 1333 982}%
\special{pa 1030 1285}%
\special{fp}%
\special{pa 1309 982}%
\special{pa 1022 1270}%
\special{fp}%
\special{pa 1285 982}%
\special{pa 1014 1254}%
\special{fp}%
\special{pa 1262 982}%
\special{pa 1008 1236}%
\special{fp}%
\special{pa 1238 982}%
\special{pa 1002 1219}%
\special{fp}%
\special{pa 1215 982}%
\special{pa 998 1199}%
\special{fp}%
\special{pa 1191 982}%
\special{pa 994 1179}%
\special{fp}%
\special{pa 1167 982}%
\special{pa 992 1157}%
\special{fp}%
\special{pa 1144 982}%
\special{pa 990 1136}%
\special{fp}%
\special{pa 1120 982}%
\special{pa 990 1112}%
\special{fp}%
\special{pa 1096 982}%
\special{pa 990 1089}%
\special{fp}%
\special{pa 1073 982}%
\special{pa 990 1065}%
\special{fp}%
\special{pa 1049 982}%
\special{pa 990 1041}%
\special{fp}%
\special{pa 1026 982}%
\special{pa 990 1018}%
\special{fp}%
\special{pa 1002 982}%
\special{pa 990 994}%
\special{fp}%
% LINE 3 0 3 0 Black White  
% 100 596 1876 402 2070 596 1900 402 2094 596 1924 402 2118 596 1948 402 2142 596 1972 402 2166 596 1996 402 2190 596 2020 404 2212 596 2044 410 2230 596 2068 416 2248 596 2092 426 2262 596 2116 438 2274 596 2140 452 2284 596 2164 470 2290 596 2188 488 2296 596 2212 512 2296 582 2250 550 2282 596 1852 402 2046 596 1828 402 2022 596 1804 402 1998 596 1780 402 1974 596 1756 402 1950 596 1732 402 1926 596 1708 402 1902 596 1684 402 1878 596 1660 402 1854 596 1636 402 1830 596 1612 402 1806 596 1588 402 1782 596 1564 402 1758 596 1540 402 1734 596 1516 402 1710 596 1492 402 1686 596 1468 402 1662 596 1444 402 1638 596 1420 402 1614 596 1396 402 1590 596 1372 402 1566 596 1348 402 1542 596 1324 402 1518 596 1300 402 1494 596 1276 402 1470 596 1252 402 1446 596 1228 402 1422 596 1204 402 1398 596 1180 402 1374 596 1156 402 1350 596 1132 402 1326 596 1108 402 1302 596 1084 402 1278 596 1060 402 1254
% 
\special{pn 4}%
\special{pa 587 1846}%
\special{pa 396 2037}%
\special{fp}%
\special{pa 587 1870}%
\special{pa 396 2061}%
\special{fp}%
\special{pa 587 1894}%
\special{pa 396 2085}%
\special{fp}%
\special{pa 587 1917}%
\special{pa 396 2108}%
\special{fp}%
\special{pa 587 1941}%
\special{pa 396 2132}%
\special{fp}%
\special{pa 587 1965}%
\special{pa 396 2156}%
\special{fp}%
\special{pa 587 1988}%
\special{pa 398 2177}%
\special{fp}%
\special{pa 587 2012}%
\special{pa 404 2195}%
\special{fp}%
\special{pa 587 2035}%
\special{pa 409 2213}%
\special{fp}%
\special{pa 587 2059}%
\special{pa 419 2226}%
\special{fp}%
\special{pa 587 2083}%
\special{pa 431 2238}%
\special{fp}%
\special{pa 587 2106}%
\special{pa 445 2248}%
\special{fp}%
\special{pa 587 2130}%
\special{pa 463 2254}%
\special{fp}%
\special{pa 587 2154}%
\special{pa 480 2260}%
\special{fp}%
\special{pa 587 2177}%
\special{pa 504 2260}%
\special{fp}%
\special{pa 573 2215}%
\special{pa 541 2246}%
\special{fp}%
\special{pa 587 1823}%
\special{pa 396 2014}%
\special{fp}%
\special{pa 587 1799}%
\special{pa 396 1990}%
\special{fp}%
\special{pa 587 1776}%
\special{pa 396 1967}%
\special{fp}%
\special{pa 587 1752}%
\special{pa 396 1943}%
\special{fp}%
\special{pa 587 1728}%
\special{pa 396 1919}%
\special{fp}%
\special{pa 587 1705}%
\special{pa 396 1896}%
\special{fp}%
\special{pa 587 1681}%
\special{pa 396 1872}%
\special{fp}%
\special{pa 587 1657}%
\special{pa 396 1848}%
\special{fp}%
\special{pa 587 1634}%
\special{pa 396 1825}%
\special{fp}%
\special{pa 587 1610}%
\special{pa 396 1801}%
\special{fp}%
\special{pa 587 1587}%
\special{pa 396 1778}%
\special{fp}%
\special{pa 587 1563}%
\special{pa 396 1754}%
\special{fp}%
\special{pa 587 1539}%
\special{pa 396 1730}%
\special{fp}%
\special{pa 587 1516}%
\special{pa 396 1707}%
\special{fp}%
\special{pa 587 1492}%
\special{pa 396 1683}%
\special{fp}%
\special{pa 587 1469}%
\special{pa 396 1659}%
\special{fp}%
\special{pa 587 1445}%
\special{pa 396 1636}%
\special{fp}%
\special{pa 587 1421}%
\special{pa 396 1612}%
\special{fp}%
\special{pa 587 1398}%
\special{pa 396 1589}%
\special{fp}%
\special{pa 587 1374}%
\special{pa 396 1565}%
\special{fp}%
\special{pa 587 1350}%
\special{pa 396 1541}%
\special{fp}%
\special{pa 587 1327}%
\special{pa 396 1518}%
\special{fp}%
\special{pa 587 1303}%
\special{pa 396 1494}%
\special{fp}%
\special{pa 587 1280}%
\special{pa 396 1470}%
\special{fp}%
\special{pa 587 1256}%
\special{pa 396 1447}%
\special{fp}%
\special{pa 587 1232}%
\special{pa 396 1423}%
\special{fp}%
\special{pa 587 1209}%
\special{pa 396 1400}%
\special{fp}%
\special{pa 587 1185}%
\special{pa 396 1376}%
\special{fp}%
\special{pa 587 1161}%
\special{pa 396 1352}%
\special{fp}%
\special{pa 587 1138}%
\special{pa 396 1329}%
\special{fp}%
\special{pa 587 1114}%
\special{pa 396 1305}%
\special{fp}%
\special{pa 587 1091}%
\special{pa 396 1281}%
\special{fp}%
\special{pa 587 1067}%
\special{pa 396 1258}%
\special{fp}%
\special{pa 587 1043}%
\special{pa 396 1234}%
\special{fp}%
% LINE 3 0 3 1 Black White  
% 20 596 1036 402 1230 596 1012 402 1206 582 1002 402 1182 558 1002 402 1158 534 1002 402 1134 510 1002 402 1110 486 1002 402 1086 462 1002 402 1062 438 1002 402 1038 414 1002 402 1014
% 
\special{pn 4}%
\special{pa 587 1020}%
\special{pa 396 1211}%
\special{fp}%
\special{pa 587 996}%
\special{pa 396 1187}%
\special{fp}%
\special{pa 573 986}%
\special{pa 396 1163}%
\special{fp}%
\special{pa 549 986}%
\special{pa 396 1140}%
\special{fp}%
\special{pa 526 986}%
\special{pa 396 1116}%
\special{fp}%
\special{pa 502 986}%
\special{pa 396 1093}%
\special{fp}%
\special{pa 478 986}%
\special{pa 396 1069}%
\special{fp}%
\special{pa 455 986}%
\special{pa 396 1045}%
\special{fp}%
\special{pa 431 986}%
\special{pa 396 1022}%
\special{fp}%
\special{pa 407 986}%
\special{pa 396 998}%
\special{fp}%
% LINE 3 0 3 0 Black White  
% 100 516 2388 482 2422 516 2412 486 2442 516 2436 488 2464 520 2456 492 2484 524 2476 496 2504 528 2496 504 2520 536 2512 512 2536 544 2528 522 2550 554 2542 532 2564 564 2556 542 2578 574 2570 554 2590 586 2582 566 2602 600 2592 578 2614 612 2604 592 2624 626 2614 604 2636 642 2622 618 2646 656 2632 634 2654 672 2640 648 2664 688 2648 664 2672 704 2656 680 2680 720 2664 696 2688 738 2670 712 2696 754 2678 730 2702 772 2684 748 2708 790 2690 766 2714 810 2694 784 2720 828 2700 802 2726 848 2704 822 2730 868 2708 840 2736 888 2712 860 2740 910 2714 880 2744 932 2716 902 2746 954 2718 922 2750 976 2720 944 2752 998 2722 966 2754 1022 2722 988 2756 1046 2722 1012 2756 1070 2722 1036 2756 1096 2720 1060 2756 1120 2720 1084 2756 1148 2716 1110 2754 1174 2714 1136 2752 1204 2708 1164 2748 1234 2702 1192 2744 1264 2696 1222 2738 1298 2686 1252 2732 1334 2674 1284 2724 1524 2508 1320 2712 1532 2476 1474 2534 1534 2450 1486 2498
% 
\special{pn 4}%
\special{pa 508 2350}%
\special{pa 474 2384}%
\special{fp}%
\special{pa 508 2374}%
\special{pa 478 2404}%
\special{fp}%
\special{pa 508 2398}%
\special{pa 480 2425}%
\special{fp}%
\special{pa 512 2417}%
\special{pa 484 2445}%
\special{fp}%
\special{pa 516 2437}%
\special{pa 488 2465}%
\special{fp}%
\special{pa 520 2457}%
\special{pa 496 2480}%
\special{fp}%
\special{pa 528 2472}%
\special{pa 504 2496}%
\special{fp}%
\special{pa 535 2488}%
\special{pa 514 2510}%
\special{fp}%
\special{pa 545 2502}%
\special{pa 524 2524}%
\special{fp}%
\special{pa 555 2516}%
\special{pa 533 2537}%
\special{fp}%
\special{pa 565 2530}%
\special{pa 545 2549}%
\special{fp}%
\special{pa 577 2541}%
\special{pa 557 2561}%
\special{fp}%
\special{pa 591 2551}%
\special{pa 569 2573}%
\special{fp}%
\special{pa 602 2563}%
\special{pa 583 2583}%
\special{fp}%
\special{pa 616 2573}%
\special{pa 594 2594}%
\special{fp}%
\special{pa 632 2581}%
\special{pa 608 2604}%
\special{fp}%
\special{pa 646 2591}%
\special{pa 624 2612}%
\special{fp}%
\special{pa 661 2598}%
\special{pa 638 2622}%
\special{fp}%
\special{pa 677 2606}%
\special{pa 654 2630}%
\special{fp}%
\special{pa 693 2614}%
\special{pa 669 2638}%
\special{fp}%
\special{pa 709 2622}%
\special{pa 685 2646}%
\special{fp}%
\special{pa 726 2628}%
\special{pa 701 2654}%
\special{fp}%
\special{pa 742 2636}%
\special{pa 719 2659}%
\special{fp}%
\special{pa 760 2642}%
\special{pa 736 2665}%
\special{fp}%
\special{pa 778 2648}%
\special{pa 754 2671}%
\special{fp}%
\special{pa 797 2652}%
\special{pa 772 2677}%
\special{fp}%
\special{pa 815 2657}%
\special{pa 789 2683}%
\special{fp}%
\special{pa 835 2661}%
\special{pa 809 2687}%
\special{fp}%
\special{pa 854 2665}%
\special{pa 827 2693}%
\special{fp}%
\special{pa 874 2669}%
\special{pa 846 2697}%
\special{fp}%
\special{pa 896 2671}%
\special{pa 866 2701}%
\special{fp}%
\special{pa 917 2673}%
\special{pa 888 2703}%
\special{fp}%
\special{pa 939 2675}%
\special{pa 907 2707}%
\special{fp}%
\special{pa 961 2677}%
\special{pa 929 2709}%
\special{fp}%
\special{pa 982 2679}%
\special{pa 951 2711}%
\special{fp}%
\special{pa 1006 2679}%
\special{pa 972 2713}%
\special{fp}%
\special{pa 1030 2679}%
\special{pa 996 2713}%
\special{fp}%
\special{pa 1053 2679}%
\special{pa 1020 2713}%
\special{fp}%
\special{pa 1079 2677}%
\special{pa 1043 2713}%
\special{fp}%
\special{pa 1102 2677}%
\special{pa 1067 2713}%
\special{fp}%
\special{pa 1130 2673}%
\special{pa 1093 2711}%
\special{fp}%
\special{pa 1156 2671}%
\special{pa 1118 2709}%
\special{fp}%
\special{pa 1185 2665}%
\special{pa 1146 2705}%
\special{fp}%
\special{pa 1215 2659}%
\special{pa 1173 2701}%
\special{fp}%
\special{pa 1244 2654}%
\special{pa 1203 2695}%
\special{fp}%
\special{pa 1278 2644}%
\special{pa 1232 2689}%
\special{fp}%
\special{pa 1313 2632}%
\special{pa 1264 2681}%
\special{fp}%
\special{pa 1500 2469}%
\special{pa 1299 2669}%
\special{fp}%
\special{pa 1508 2437}%
\special{pa 1451 2494}%
\special{fp}%
\special{pa 1510 2411}%
\special{pa 1463 2459}%
\special{fp}%
% LINE 3 0 3 1 Black White  
% 88 1534 2426 1492 2468 1532 2404 1494 2442 1528 2384 1494 2418 1524 2364 1492 2396 1518 2346 1486 2378 1510 2330 1480 2360 1502 2314 1474 2342 1492 2300 1466 2326 1484 2284 1458 2310 1472 2272 1448 2296 1462 2258 1438 2282 1450 2246 1426 2270 1440 2232 1414 2258 1430 2218 1406 2242 1422 2202 1398 2226 1414 2186 1390 2210 1408 2168 1382 2194 1402 2150 1376 2176 1396 2132 1370 2158 1390 2114 1364 2140 1386 2094 1360 2120 1382 2074 1354 2102 1378 2054 1350 2082 1374 2034 1346 2062 1370 2014 1342 2042 1366 1994 1338 2022 1364 1972 1334 2002 1360 1952 1332 1980 1358 1930 1328 1960 1354 1910 1326 1938 1352 1888 1324 1916 1350 1866 1320 1896 1348 1844 1318 1874 1348 1820 1316 1852 1346 1798 1316 1828 1344 1776 1314 1806 1344 1752 1312 1784 1342 1730 1312 1760 1342 1706 1310 1738 1338 1686 1310 1714 1510 2546 1356 2700 516 2364 482 2398 514 2342 482 2374 490 2342 482 2350
% 
\special{pn 4}%
\special{pa 1510 2388}%
\special{pa 1469 2429}%
\special{fp}%
\special{pa 1508 2366}%
\special{pa 1470 2404}%
\special{fp}%
\special{pa 1504 2346}%
\special{pa 1470 2380}%
\special{fp}%
\special{pa 1500 2327}%
\special{pa 1469 2358}%
\special{fp}%
\special{pa 1494 2309}%
\special{pa 1463 2341}%
\special{fp}%
\special{pa 1486 2293}%
\special{pa 1457 2323}%
\special{fp}%
\special{pa 1478 2278}%
\special{pa 1451 2305}%
\special{fp}%
\special{pa 1469 2264}%
\special{pa 1443 2289}%
\special{fp}%
\special{pa 1461 2248}%
\special{pa 1435 2274}%
\special{fp}%
\special{pa 1449 2236}%
\special{pa 1425 2260}%
\special{fp}%
\special{pa 1439 2222}%
\special{pa 1415 2246}%
\special{fp}%
\special{pa 1427 2211}%
\special{pa 1404 2234}%
\special{fp}%
\special{pa 1417 2197}%
\special{pa 1392 2222}%
\special{fp}%
\special{pa 1407 2183}%
\special{pa 1384 2207}%
\special{fp}%
\special{pa 1400 2167}%
\special{pa 1376 2191}%
\special{fp}%
\special{pa 1392 2152}%
\special{pa 1368 2175}%
\special{fp}%
\special{pa 1386 2134}%
\special{pa 1360 2159}%
\special{fp}%
\special{pa 1380 2116}%
\special{pa 1354 2142}%
\special{fp}%
\special{pa 1374 2098}%
\special{pa 1348 2124}%
\special{fp}%
\special{pa 1368 2081}%
\special{pa 1343 2106}%
\special{fp}%
\special{pa 1364 2061}%
\special{pa 1339 2087}%
\special{fp}%
\special{pa 1360 2041}%
\special{pa 1333 2069}%
\special{fp}%
\special{pa 1356 2022}%
\special{pa 1329 2049}%
\special{fp}%
\special{pa 1352 2002}%
\special{pa 1325 2030}%
\special{fp}%
\special{pa 1348 1982}%
\special{pa 1321 2010}%
\special{fp}%
\special{pa 1344 1963}%
\special{pa 1317 1990}%
\special{fp}%
\special{pa 1343 1941}%
\special{pa 1313 1970}%
\special{fp}%
\special{pa 1339 1921}%
\special{pa 1311 1949}%
\special{fp}%
\special{pa 1337 1900}%
\special{pa 1307 1929}%
\special{fp}%
\special{pa 1333 1880}%
\special{pa 1305 1907}%
\special{fp}%
\special{pa 1331 1858}%
\special{pa 1303 1886}%
\special{fp}%
\special{pa 1329 1837}%
\special{pa 1299 1866}%
\special{fp}%
\special{pa 1327 1815}%
\special{pa 1297 1844}%
\special{fp}%
\special{pa 1327 1791}%
\special{pa 1295 1823}%
\special{fp}%
\special{pa 1325 1770}%
\special{pa 1295 1799}%
\special{fp}%
\special{pa 1323 1748}%
\special{pa 1293 1778}%
\special{fp}%
\special{pa 1323 1724}%
\special{pa 1291 1756}%
\special{fp}%
\special{pa 1321 1703}%
\special{pa 1291 1732}%
\special{fp}%
\special{pa 1321 1679}%
\special{pa 1289 1711}%
\special{fp}%
\special{pa 1317 1659}%
\special{pa 1289 1687}%
\special{fp}%
\special{pa 1486 2506}%
\special{pa 1335 2657}%
\special{fp}%
\special{pa 508 2327}%
\special{pa 474 2360}%
\special{fp}%
\special{pa 506 2305}%
\special{pa 474 2337}%
\special{fp}%
\special{pa 482 2305}%
\special{pa 474 2313}%
\special{fp}%
% BOX 2 0 2 0 Black White  
% 2 460 2362 536 2442
% 
\special{pn 0}%
\special{sh 0}%
\special{pa 453 2325}%
\special{pa 528 2325}%
\special{pa 528 2404}%
\special{pa 453 2404}%
\special{pa 453 2325}%
\special{ip}%
\special{pn 8}%
\special{pa 453 2325}%
\special{pa 528 2325}%
\special{pa 528 2404}%
\special{pa 453 2404}%
\special{pa 453 2325}%
\special{pa 528 2325}%
\special{fp}%
\end{picture}}%
}
    ガラス管の管口の真上に取り付けたスピーカーから振動数$f=423$\sftanni{Hz}の音を出しておき,
    ガラス管を満たした水の面をゆっくり下げていったところ,水面が管口から$\ell _1=18.9$\sftanni{cm}のとき初めて音が大きく聞こえた。さらに水面を下げていくと,音はいったん小さくなり,管口から$\ell _2=59.1$\sftanni{cm}のときに再び大きく聞こえた。開口端補正は音の振動数などによって変わらないものとする。
        \begin{enumerate}
            \item 音波の波長$\lambda $は何\sftanni{cm}か。音速$V$は何\sftanni{m/s}か。また,開口端補正$\varDelta \ell $は何\sftanni{cm}か。
            \item $\ell _2=59.1$\sftanni{cm}のとき,管内において,次の位置を管口からの距離で答えよ。
                \begin{enumerate}
                    \item 空気の振動の振幅がとくに大きい位置
                    \item 空気の密度の変動がとくに大きい位置
                \end{enumerate}
            \item もしも,気温を上げて同じ実験をすると,$\ell _1,\ell _2$の値は増すか,減るか,それとも変化しないか。
            \item $\ell _2=59.1$\sftanni{cm}に保ったまま,スピーカーの音の振動数を$423$\sftanni{Hz}からしだいに増していくと,音はいったん小さくなり,再び大きく聞こえた。このときの振動数は何\sftanni{Hz}か。
        \end{enumerate}
    \end{mawarikomi}