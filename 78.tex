\hakosyokika
\item
    \begin{mawarikomi}(10pt,0){120pt}{\input{./fig/fig078.tex}}
    ガラス管の管口の真上に取り付けたスピーカーから振動数$f=423$\sftanni{Hz}の音を出しておき,
    ガラス管を満たした水の面をゆっくり下げていったところ,水面が管口から$\ell _1=18.9$\sftanni{cm}のとき初めて音が大きく聞こえた。さらに水面を下げていくと,音はいったん小さくなり,管口から$\ell _2=59.1$\sftanni{cm}のときに再び大きく聞こえた。開口端補正は音の振動数などによって変わらないものとする。
        \begin{enumerate}
            \item 音波の波長$\lambda $は何\sftanni{cm}か。音速$V$は何\sftanni{m/s}か。また,開口端補正$\varDelta \ell $は何\sftanni{cm}か。
            \item $\ell _2=59.1$\sftanni{cm}のとき,管内において,次の位置を管口からの距離で答えよ。
                \begin{enumerate}
                    \item 空気の振動の振幅がとくに大きい位置
                    \item 空気の密度の変動がとくに大きい位置
                \end{enumerate}
            \item もしも,気温を上げて同じ実験をすると,$\ell _1,\ell _2$の値は増すか,減るか,それとも変化しないか。
            \item $\ell _2=59.1$\sftanni{cm}に保ったまま,スピーカーの音の振動数を$423$\sftanni{Hz}からしだいに増していくと,音はいったん小さくなり,再び大きく聞こえた。このときの振動数は何\sftanni{Hz}か。
        \end{enumerate}
    \end{mawarikomi}