\hakosyokika
\item
    \begin{mawarikomi}(10pt,0){220pt}{%WinTpicVersion4.32a
{\unitlength 0.1in%
\begin{picture}(32.9232,14.7638)(4.9213,-22.6378)%
% BOX 2 0 1 0 Black Black  
% 2 1000 800 3800 1000
% 
\special{pn 0}%
\special{sh 0.200}%
\special{pa 984 787}%
\special{pa 3740 787}%
\special{pa 3740 984}%
\special{pa 984 984}%
\special{pa 984 787}%
\special{ip}%
\special{pn 8}%
\special{pa 984 787}%
\special{pa 3740 787}%
\special{pa 3740 984}%
\special{pa 984 984}%
\special{pa 984 787}%
\special{pa 3740 787}%
\special{fp}%
% BOX 2 0 1 0 Black Black  
% 2 1000 2000 3800 2200
% 
\special{pn 0}%
\special{sh 0.200}%
\special{pa 984 1969}%
\special{pa 3740 1969}%
\special{pa 3740 2165}%
\special{pa 984 2165}%
\special{pa 984 1969}%
\special{ip}%
\special{pn 8}%
\special{pa 984 1969}%
\special{pa 3740 1969}%
\special{pa 3740 2165}%
\special{pa 984 2165}%
\special{pa 984 1969}%
\special{pa 3740 1969}%
\special{fp}%
% BOX 2 0 0 0 Black Black  
% 2 1000 1000 3800 2000
% 
\special{pn 0}%
\special{sh 0.400}%
\special{pa 984 984}%
\special{pa 3740 984}%
\special{pa 3740 1969}%
\special{pa 984 1969}%
\special{pa 984 984}%
\special{ip}%
\special{pn 8}%
\special{pa 984 984}%
\special{pa 3740 984}%
\special{pa 3740 1969}%
\special{pa 984 1969}%
\special{pa 984 984}%
\special{pa 3740 984}%
\special{fp}%
% LINE 2 1 3 0 Black White  
% 2 600 1500 1600 1500
% 
\special{pn 8}%
\special{pa 591 1476}%
\special{pa 1575 1476}%
\special{da 0.030}%
% LINE 2 1 3 0 Black White  
% 2 1800 900 1800 1500
% 
\special{pn 8}%
\special{pa 1772 886}%
\special{pa 1772 1476}%
\special{da 0.030}%
% LINE 1 0 3 0 Black White  
% 2 1000 1500 800 1900
% 
\special{pn 13}%
\special{pa 984 1476}%
\special{pa 787 1870}%
\special{fp}%
% VECTOR 1 0 3 0 Black White  
% 2 600 2300 800 1900
% 
\special{pn 13}%
\special{pa 591 2264}%
\special{pa 787 1870}%
\special{fp}%
\special{sh 1}%
\special{pa 787 1870}%
\special{pa 740 1920}%
\special{pa 764 1917}%
\special{pa 776 1938}%
\special{pa 787 1870}%
\special{fp}%
% VECTOR 1 0 3 0 Black White  
% 2 1000 1500 1400 1250
% 
\special{pn 13}%
\special{pa 984 1476}%
\special{pa 1378 1230}%
\special{fp}%
\special{sh 1}%
\special{pa 1378 1230}%
\special{pa 1312 1248}%
\special{pa 1334 1258}%
\special{pa 1333 1281}%
\special{pa 1378 1230}%
\special{fp}%
% LINE 1 0 3 0 Black White  
% 2 1400 1250 1800 1000
% 
\special{pn 13}%
\special{pa 1378 1230}%
\special{pa 1772 984}%
\special{fp}%
% VECTOR 1 0 3 0 Black White  
% 2 1800 1000 2200 1250
% 
\special{pn 13}%
\special{pa 1772 984}%
\special{pa 2165 1230}%
\special{fp}%
\special{sh 1}%
\special{pa 2165 1230}%
\special{pa 2120 1179}%
\special{pa 2121 1203}%
\special{pa 2099 1213}%
\special{pa 2165 1230}%
\special{fp}%
% LINE 1 0 3 0 Black White  
% 2 2200 1250 3400 2000
% 
\special{pn 13}%
\special{pa 2165 1230}%
\special{pa 3346 1969}%
\special{fp}%
% VECTOR 1 0 3 0 Black White  
% 2 3400 2000 3800 1750
% 
\special{pn 13}%
\special{pa 3346 1969}%
\special{pa 3740 1722}%
\special{fp}%
\special{sh 1}%
\special{pa 3740 1722}%
\special{pa 3674 1740}%
\special{pa 3696 1750}%
\special{pa 3695 1774}%
\special{pa 3740 1722}%
\special{fp}%
% STR 2 0 3 0 Black White  
% 4 3880 900 3880 1000 5 0 0 0
% C
\put(38.1890,-9.8425){\makebox(0,0){C}}%
% STR 2 0 3 0 Black White  
% 4 3880 1900 3880 2000 5 0 0 0
% D
\put(38.1890,-19.6850){\makebox(0,0){D}}%
% STR 2 0 3 0 Black White  
% 4 910 1900 910 2000 5 0 0 0
% B
\put(8.9567,-19.6850){\makebox(0,0){B}}%
% STR 2 0 3 0 Black White  
% 4 910 900 910 1000 5 0 0 0
% A
\put(8.9567,-9.8425){\makebox(0,0){A}}%
% CIRCLE 2 0 3 0 Black White  
% 4 1800 1000 2000 1000 1000 1500 1800 1500
% 
\special{pn 8}%
\special{ar 1772 984 197 197 1.5707963 2.5829933}%
% STR 2 0 3 0 Black White  
% 4 1630 1180 1630 1280 5 0 0 0
% $\alpha$
\put(16.0433,-12.5984){\makebox(0,0){$\alpha$}}%
% CIRCLE 2 0 3 0 Black White  
% 4 1000 1500 800 1500 400 1500 600 2300
% 
\special{pn 8}%
\special{ar 984 1476 197 197 2.0344439 3.1415927}%
% STR 2 0 3 0 Black White  
% 4 720 1570 720 1670 5 0 0 0
% $\theta$
\put(7.0866,-16.4370){\makebox(0,0){$\theta$}}%
% STR 2 0 3 0 Black White  
% 4 2120 800 2120 900 5 0 0 0
% $n_2$
\put(20.8661,-8.8583){\makebox(0,0){$n_2$}}%
% STR 2 0 3 0 Black White  
% 4 2120 1600 2120 1700 5 0 0 0
% $n_1$
\put(20.8661,-16.7323){\makebox(0,0){$n_1$}}%
% STR 2 0 3 0 Black White  
% 4 2120 2000 2120 2100 5 0 0 0
% $n_2$
\put(20.8661,-20.6693){\makebox(0,0){$n_2$}}%
\end{picture}}%
}
    図のように,屈折率$n_1$のガラス直方体のの上面と下面に屈折率$n_2$のガラス板を密着させて,
    光線を側面ABから入射させた。このとき,ガラス直方体中で光線が全反射を繰り返しながら,側面CDまで到達するために必要な条件を調べてみよう。
    ガラスは空気中に置かれ,空気の屈折率は1としてよい。
        \begin{enumerate}
            \item 全反射が起こるための,$n_1$と$n_2$の大小関係を答えよ。
            \item AC面での臨界角を$\alpha _0$として,$\sin{\alpha _0}$を求めよ。
            \item AB面への入射角を$\theta $とし,AC面への入射角を$\alpha $とする。$\cos{\alpha}$を$\theta $と$n_1$で表せ。
            \item 図のように全反射をするための$\sin{\theta }$に対する条件を$n_1$,$n_2$を用いて表せ。
            \item $0\Deg < \theta < 90\Deg $のすべての$\theta $に対して全反射を起こさせるための条件を$n_1$,$n_2$だけを用いて表せ。
        \end{enumerate}
    \end{mawarikomi}