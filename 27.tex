\item
    \begin{mawarikomi}{120pt}{\begin{zahyou*}[ul=4mm](-5,5)(-5,5)
	\def\O{(0,0)}
	\def\R{(4.5,0)}
	\def\A{(-0.25,-4.75)}
	\def\B{(0.25,-4.75)}
	\def\M{(0,-2.5)}
	\HenKo<henkoH=2.5ex>\R\O{$R$}
	\Hasen{\O\R}
	\Kuromaru{\O}
	\En\O{4.5}
	\En\O{5.0}
	\En*[1]\A{0.25}
	\En*[1]\B{0.25}
	\Put\A(-4pt,12pt)[t]{A}
	\Put\B(4pt,12pt)[t]{B}
	\Put\M[t]{上から見た図}
\end{zahyou*}
}
        細い円形のパイプが水平に固定され,中に同じ質量$m$の小球A,Bが入って接触している。Aを速さ$2v_0$,Bを速さ$v_0$で逆向きに同時に打ち出したところ,AとBは半径$R$の等速円運動をし,パイプの内で衝突を繰り返す。衝突の際の反発係数を$e\left(0<e<\bunsuu{1}{3} \right)$とし,摩擦はなく,空気抵抗は無視する。
        \begin{Enumerate}
            \item AとBを打ち出してから1回目の衝突が起こるまでの時間$t$はいくらか。
            \item 1回目の衝突直後のAとBの速さはそれぞれいくらか。 
        \end{Enumerate}
        衝突後,AとBは同じ向きに運動し,やがてBはAに追いついて2回目の衝突が起き,以後,このような衝突を繰り返した。
        \begin{Enumerate*}
            \item 2回目の衝突直後のAとBの速さはそれぞれいくらか。
            \item 衝突を繰り返していくと,AとBの速さは同じ値に近づいていく。その値はいくらか。
        \end{Enumerate*}
    \end{mawarikomi}