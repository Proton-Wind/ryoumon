\hakosyokika
\item
    \begin{mawarikomi}(20pt,0pt){150pt}{
        %%% C:/vpn/vpn/KeTCindy/fig/fig109_1.tex 
%%% Generator=fig109_1.cdy 
{\unitlength=1cm%
\begin{picture}%
(4.5,4)(-2,-1.5)%
\special{pn 8}%
%
\special{pa   591  -591}\special{pa   585  -593}\special{pa   579  -595}\special{pa   572  -598}%
\special{pa   566  -600}\special{pa   560  -603}\special{pa   554  -605}\special{pa   548  -607}%
\special{pa   542  -609}\special{pa   536  -612}\special{pa   530  -614}\special{pa   524  -616}%
\special{pa   517  -618}\special{pa   511  -620}\special{pa   505  -622}\special{pa   499  -624}%
\special{pa   493  -626}\special{pa   486  -628}\special{pa   480  -629}\special{pa   474  -631}%
\special{pa   468  -633}\special{pa   461  -635}\special{pa   455  -636}\special{pa   449  -638}%
\special{pa   442  -639}\special{pa   436  -641}\special{pa   430  -642}\special{pa   423  -644}%
\special{pa   417  -645}\special{pa   411  -647}\special{pa   404  -648}\special{pa   398  -649}%
\special{pa   392  -651}\special{pa   385  -652}\special{pa   379  -653}\special{pa   372  -654}%
\special{pa   366  -655}\special{pa   359  -656}\special{pa   353  -657}\special{pa   347  -658}%
\special{pa   340  -659}\special{pa   334  -660}\special{pa   327  -661}\special{pa   321  -662}%
\special{pa   314  -663}\special{pa   308  -663}\special{pa   301  -664}\special{pa   295  -665}%
\special{pa   288  -665}\special{pa   282  -666}\special{pa   276  -666}%
\special{fp}%
\special{pa   118  -666}\special{pa   112  -666}\special{pa   105  -665}\special{pa    99  -665}%
\special{pa    92  -664}\special{pa    86  -663}\special{pa    79  -663}\special{pa    73  -662}%
\special{pa    66  -661}\special{pa    60  -660}\special{pa    54  -659}\special{pa    47  -658}%
\special{pa    41  -657}\special{pa    34  -656}\special{pa    28  -655}\special{pa    21  -654}%
\special{pa    15  -653}\special{pa     9  -652}\special{pa     2  -651}\special{pa    -4  -649}%
\special{pa   -11  -648}\special{pa   -17  -647}\special{pa   -23  -645}\special{pa   -30  -644}%
\special{pa   -36  -642}\special{pa   -42  -641}\special{pa   -49  -639}\special{pa   -55  -638}%
\special{pa   -61  -636}\special{pa   -68  -635}\special{pa   -74  -633}\special{pa   -80  -631}%
\special{pa   -86  -629}\special{pa   -93  -628}\special{pa   -99  -626}\special{pa  -105  -624}%
\special{pa  -111  -622}\special{pa  -118  -620}\special{pa  -124  -618}\special{pa  -130  -616}%
\special{pa  -136  -614}\special{pa  -142  -612}\special{pa  -148  -609}\special{pa  -154  -607}%
\special{pa  -161  -605}\special{pa  -167  -603}\special{pa  -173  -600}\special{pa  -179  -598}%
\special{pa  -185  -595}\special{pa  -191  -593}\special{pa  -197  -591}%
\special{fp}%
\settowidth{\Width}{$\ell$}\setlength{\Width}{-0.5\Width}%
\settoheight{\Height}{$\ell$}\settodepth{\Depth}{$\ell$}\setlength{\Height}{-0.5\Height}\setlength{\Depth}{0.5\Depth}\addtolength{\Height}{\Depth}%
\put(  0.500,  1.700){\hspace*{\Width}\raisebox{\Height}{$\ell$}}%
%
\special{pa  -197  -591}\special{pa  -200  -589}\special{pa  -203  -588}\special{pa  -206  -586}%
\special{pa  -209  -585}\special{pa  -212  -583}\special{pa  -214  -582}\special{pa  -217  -580}%
\special{pa  -220  -578}\special{pa  -223  -577}\special{pa  -226  -575}\special{pa  -229  -574}%
\special{pa  -232  -572}\special{pa  -234  -570}\special{pa  -237  -568}\special{pa  -240  -567}%
\special{pa  -243  -565}\special{pa  -245  -563}\special{pa  -248  -561}\special{pa  -251  -559}%
\special{pa  -254  -558}\special{pa  -256  -556}\special{pa  -259  -554}\special{pa  -262  -552}%
\special{pa  -264  -550}\special{pa  -267  -548}\special{pa  -270  -546}\special{pa  -272  -544}%
\special{pa  -275  -542}\special{pa  -277  -540}\special{pa  -280  -538}\special{pa  -282  -536}%
\special{pa  -285  -534}\special{pa  -287  -531}\special{pa  -290  -529}\special{pa  -292  -527}%
\special{pa  -295  -525}\special{pa  -297  -523}\special{pa  -300  -521}\special{pa  -302  -518}%
\special{pa  -305  -516}\special{pa  -307  -514}\special{pa  -309  -511}\special{pa  -312  -509}%
\special{pa  -314  -507}\special{pa  -316  -504}\special{pa  -318  -502}\special{pa  -321  -500}%
\special{pa  -323  -497}\special{pa  -325  -495}\special{pa  -327  -492}%
\special{fp}%
\special{pa  -408  -358}\special{pa  -409  -355}\special{pa  -410  -352}\special{pa  -411  -349}%
\special{pa  -412  -346}\special{pa  -413  -343}\special{pa  -414  -340}\special{pa  -415  -336}%
\special{pa  -416  -333}\special{pa  -417  -330}\special{pa  -418  -327}\special{pa  -419  -324}%
\special{pa  -420  -321}\special{pa  -421  -317}\special{pa  -421  -314}\special{pa  -422  -311}%
\special{pa  -423  -308}\special{pa  -424  -305}\special{pa  -424  -301}\special{pa  -425  -298}%
\special{pa  -426  -295}\special{pa  -426  -292}\special{pa  -427  -288}\special{pa  -427  -285}%
\special{pa  -428  -282}\special{pa  -428  -279}\special{pa  -429  -276}\special{pa  -429  -272}%
\special{pa  -430  -269}\special{pa  -430  -266}\special{pa  -431  -262}\special{pa  -431  -259}%
\special{pa  -431  -256}\special{pa  -432  -253}\special{pa  -432  -249}\special{pa  -432  -246}%
\special{pa  -432  -243}\special{pa  -433  -240}\special{pa  -433  -236}\special{pa  -433  -233}%
\special{pa  -433  -230}\special{pa  -433  -226}\special{pa  -433  -223}\special{pa  -433  -220}%
\special{pa  -433  -217}\special{pa  -433  -213}\special{pa  -433  -210}\special{pa  -433  -207}%
\special{pa  -433  -203}\special{pa  -433  -200}\special{pa  -433  -197}%
\special{fp}%
\settowidth{\Width}{$\ell$}\setlength{\Width}{-0.5\Width}%
\settoheight{\Height}{$\ell$}\settodepth{\Depth}{$\ell$}\setlength{\Height}{-0.5\Height}\setlength{\Depth}{0.5\Depth}\addtolength{\Height}{\Depth}%
\put( -0.950,  1.090){\hspace*{\Width}\raisebox{\Height}{$\ell$}}%
%
{%
\color[cmyk]{0,0,0,0}%
\special{pa 591 -354}\special{pa -197 -354}\special{pa -433 39}\special{pa 354 39}%
\special{pa 591 -354}\special{pa 591 -354}\special{sh 1}\special{ip}%
}%
\special{pa   591  -354}\special{pa  -197  -354}\special{pa  -433    39}\special{pa   354    39}%
\special{pa   591  -354}%
\special{fp}%
\special{pa 591 -354}\special{pa 591 -335}\special{pa 354 59}\special{pa 354 39}\special{pa 591 -354}%
\special{pa 591 -354}\special{sh 1}\special{ip}%
\special{pa   591  -354}\special{pa   591  -335}\special{pa   354    59}\special{pa   354    39}%
\special{pa   591  -354}%
\special{fp}%
\special{pa -433 39}\special{pa -433 59}\special{pa 354 59}\special{pa 354 39}\special{pa -433 39}%
\special{pa -433 39}\special{sh 1}\special{ip}%
\special{pa  -433    39}\special{pa  -433    59}\special{pa   354    59}\special{pa   354    39}%
\special{pa  -433    39}%
\special{fp}%
{%
\color[cmyk]{0,0,0,0}%
\special{pa 591 -591}\special{pa -197 -591}\special{pa -433 -197}\special{pa 354 -197}%
\special{pa 591 -591}\special{pa 591 -591}\special{sh 1}\special{ip}%
}%
\special{pa   591  -591}\special{pa  -197  -591}\special{pa  -433  -197}\special{pa   354  -197}%
\special{pa   591  -591}%
\special{fp}%
\special{pa 591 -591}\special{pa 591 -571}\special{pa 354 -177}\special{pa 354 -197}%
\special{pa 591 -591}\special{pa 591 -591}\special{sh 1}\special{ip}%
\special{pa   591  -591}\special{pa   591  -571}\special{pa   354  -177}\special{pa   354  -197}%
\special{pa   591  -591}%
\special{fp}%
\special{pa -433 -197}\special{pa -433 -177}\special{pa 354 -177}\special{pa 354 -197}%
\special{pa -433 -197}\special{pa -433 -197}\special{sh 1}\special{ip}%
\special{pa  -433  -197}\special{pa  -433  -177}\special{pa   354  -177}\special{pa   354  -197}%
\special{pa  -433  -197}%
\special{fp}%
\special{pa 221 -122}\special{pa 197 -197}\special{pa 173 -122}\special{pa 197 -137}%
\special{pa 221 -122}\special{pa 221 -122}\special{sh 1}\special{ip}%
\special{pn 1}%
\special{pa   221  -122}\special{pa   197  -197}\special{pa   173  -122}\special{pa   197  -137}%
\special{pa   221  -122}%
\special{fp}%
\special{pn 8}%
\special{pa   197   -79}\special{pa   197  -137}%
\special{fp}%
\special{pa 173 -36}\special{pa 197 39}\special{pa 221 -36}\special{pa 197 -21}\special{pa 173 -36}%
\special{pa 173 -36}\special{sh 1}\special{ip}%
\special{pn 1}%
\special{pa   173   -36}\special{pa   197    39}\special{pa   221   -36}\special{pa   197   -21}%
\special{pa   173   -36}%
\special{fp}%
\special{pn 8}%
\special{pa   197   -79}\special{pa   197   -21}%
\special{fp}%
\settowidth{\Width}{A}\setlength{\Width}{0\Width}%
\settoheight{\Height}{A}\settodepth{\Depth}{A}\setlength{\Height}{\Depth}%
\put(  1.650,  1.600){\hspace*{\Width}\raisebox{\Height}{A}}%
%
\settowidth{\Width}{B}\setlength{\Width}{0\Width}%
\settoheight{\Height}{B}\settodepth{\Depth}{B}\setlength{\Height}{\Depth}%
\put(  1.650,  1.000){\hspace*{\Width}\raisebox{\Height}{B}}%
%
\settowidth{\Width}{$+Q$}\setlength{\Width}{-1\Width}%
\settoheight{\Height}{$+Q$}\settodepth{\Depth}{$+Q$}\setlength{\Height}{-\Height}%
\put( -1.150,  0.450){\hspace*{\Width}\raisebox{\Height}{$+Q$}}%
%
\settowidth{\Width}{$-Q$}\setlength{\Width}{-1\Width}%
\settoheight{\Height}{$-Q$}\settodepth{\Depth}{$-Q$}\setlength{\Height}{-\Height}%
\put( -1.150, -0.150){\hspace*{\Width}\raisebox{\Height}{$-Q$}}%
%
\settowidth{\Width}{図1}\setlength{\Width}{-0.5\Width}%
\settoheight{\Height}{図1}\settodepth{\Depth}{図1}\setlength{\Height}{-0.5\Height}\setlength{\Depth}{0.5\Depth}\addtolength{\Height}{\Depth}%
\put(  0.000, -1.000){\hspace*{\Width}\raisebox{\Height}{図1}}%
%
\settowidth{\Width}{$d$}\setlength{\Width}{-1\Width}%
\settoheight{\Height}{$d$}\settodepth{\Depth}{$d$}\setlength{\Height}{-0.5\Height}\setlength{\Depth}{0.5\Depth}\addtolength{\Height}{\Depth}%
\put(  0.400,  0.200){\hspace*{\Width}\raisebox{\Height}{$d$}}%
%
\end{picture}}%
        %%% C:/vpn/vpn/KeTCindy/fig/fig109_2.tex 
%%% Generator=fig109_2.cdy 
{\unitlength=1cm%
\begin{picture}%
(5,4)(-2.5,-1.5)%
\special{pn 8}%
%
{%
\color[cmyk]{0,0,0,0}%
\special{pa 591 -354}\special{pa -197 -354}\special{pa -433 39}\special{pa 354 39}%
\special{pa 591 -354}\special{pa 591 -354}\special{sh 1}\special{ip}%
}%
\special{pa   591  -354}\special{pa  -197  -354}\special{pa  -433    39}\special{pa   354    39}%
\special{pa   591  -354}%
\special{fp}%
\special{pa 591 -354}\special{pa 591 -335}\special{pa 354 59}\special{pa 354 39}\special{pa 591 -354}%
\special{pa 591 -354}\special{sh 1}\special{ip}%
\special{pa   591  -354}\special{pa   591  -335}\special{pa   354    59}\special{pa   354    39}%
\special{pa   591  -354}%
\special{fp}%
\special{pa -433 39}\special{pa -433 59}\special{pa 354 59}\special{pa 354 39}\special{pa -433 39}%
\special{pa -433 39}\special{sh 1}\special{ip}%
\special{pa  -433    39}\special{pa  -433    59}\special{pa   354    59}\special{pa   354    39}%
\special{pa  -433    39}%
\special{fp}%
\special{pa  -197  -571}\special{pa  -591  -571}\special{pa  -827  -177}\special{pa   -39  -177}%
\special{fp}%
{%
\color[cmyk]{0,0,0,0.5}%
\special{pa -39 -177}\special{pa 118 -177}\special{pa -39 39}\special{pa -39 -177}%
\special{pa -39 -177}\special{sh 1}\special{ip}%
}%
\special{pa   -39  -177}\special{pa   118  -177}\special{pa   -39    39}\special{pa   -39  -177}%
\special{fp}%
{%
\color[cmyk]{0,0,0,0.5}%
\special{pa -827 -177}\special{pa -39 -177}\special{pa -39 39}\special{pa -827 39}%
\special{pa -827 -177}\special{pa -827 -177}\special{sh 1}\special{ip}%
}%
\special{pa  -827  -177}\special{pa   -39  -177}\special{pa   -39    39}\special{pa  -827    39}%
\special{pa  -827  -177}%
\special{fp}%
{%
\color[cmyk]{0,0,0,0}%
\special{pa 591 -591}\special{pa -197 -591}\special{pa -433 -197}\special{pa 354 -197}%
\special{pa 591 -591}\special{pa 591 -591}\special{sh 1}\special{ip}%
}%
\special{pa   591  -591}\special{pa  -197  -591}\special{pa  -433  -197}\special{pa   354  -197}%
\special{pa   591  -591}%
\special{fp}%
\special{pa 591 -591}\special{pa 591 -571}\special{pa 354 -177}\special{pa 354 -197}%
\special{pa 591 -591}\special{pa 591 -591}\special{sh 1}\special{ip}%
\special{pa   591  -591}\special{pa   591  -571}\special{pa   354  -177}\special{pa   354  -197}%
\special{pa   591  -591}%
\special{fp}%
\special{pa -433 -197}\special{pa -433 -177}\special{pa 354 -177}\special{pa 354 -197}%
\special{pa -433 -197}\special{pa -433 -197}\special{sh 1}\special{ip}%
\special{pa  -433  -197}\special{pa  -433  -177}\special{pa   354  -177}\special{pa   354  -197}%
\special{pa  -433  -197}%
\special{fp}%
\special{pa -39 -177}\special{pa -20 -209}\special{fp}\special{pa -1 -241}\special{pa 18 -273}\special{fp}%
\special{pa 37 -304}\special{pa 56 -336}\special{fp}\special{pa 75 -368}\special{pa 94 -400}\special{fp}%
\special{pa 113 -432}\special{pa 132 -463}\special{fp}\special{pa 151 -495}\special{pa 170 -527}\special{fp}%
\special{pa 190 -559}\special{pa 197 -571}\special{pa 174 -571}\special{fp}\special{pa 137 -571}\special{pa 100 -571}\special{fp}%
\special{pa 63 -571}\special{pa 26 -571}\special{fp}\special{pa -11 -571}\special{pa -49 -571}\special{fp}%
\special{pa -86 -571}\special{pa -123 -571}\special{fp}\special{pa -160 -571}\special{pa -197 -571}\special{fp}%
%
%
\special{pa 221 -122}\special{pa 197 -197}\special{pa 173 -122}\special{pa 197 -137}%
\special{pa 221 -122}\special{pa 221 -122}\special{sh 1}\special{ip}%
\special{pn 1}%
\special{pa   221  -122}\special{pa   197  -197}\special{pa   173  -122}\special{pa   197  -137}%
\special{pa   221  -122}%
\special{fp}%
\special{pn 8}%
\special{pa   197   -79}\special{pa   197  -137}%
\special{fp}%
\special{pa 173 -36}\special{pa 197 39}\special{pa 221 -36}\special{pa 197 -21}\special{pa 173 -36}%
\special{pa 173 -36}\special{sh 1}\special{ip}%
\special{pn 1}%
\special{pa   173   -36}\special{pa   197    39}\special{pa   221   -36}\special{pa   197   -21}%
\special{pa   173   -36}%
\special{fp}%
\special{pn 8}%
\special{pa   197   -79}\special{pa   197   -21}%
\special{fp}%
\special{pa 3 -352}\special{pa 79 -374}\special{pa 4 -400}\special{pa 19 -376}\special{pa 3 -352}%
\special{pa 3 -352}\special{sh 1}\special{ip}%
\special{pn 1}%
\special{pa     3  -352}\special{pa    79  -374}\special{pa     4  -400}\special{pa    19  -376}%
\special{pa     3  -352}%
\special{fp}%
\special{pn 8}%
\special{pa  -118  -379}\special{pa    19  -376}%
\special{fp}%
\special{pa -239 -406}\special{pa -315 -384}\special{pa -241 -358}\special{pa -255 -382}%
\special{pa -239 -406}\special{pa -239 -406}\special{sh 1}\special{ip}%
\special{pn 1}%
\special{pa  -239  -406}\special{pa  -315  -384}\special{pa  -241  -358}\special{pa  -255  -382}%
\special{pa  -239  -406}%
\special{fp}%
\special{pn 8}%
\special{pa  -118  -379}\special{pa  -255  -382}%
\special{fp}%
\settowidth{\Width}{A}\setlength{\Width}{0\Width}%
\settoheight{\Height}{A}\settodepth{\Depth}{A}\setlength{\Height}{\Depth}%
\put(  1.650,  1.600){\hspace*{\Width}\raisebox{\Height}{A}}%
%
\settowidth{\Width}{B}\setlength{\Width}{0\Width}%
\settoheight{\Height}{B}\settodepth{\Depth}{B}\setlength{\Height}{\Depth}%
\put(  1.650,  1.000){\hspace*{\Width}\raisebox{\Height}{B}}%
%
\settowidth{\Width}{$+Q$}\setlength{\Width}{-0.5\Width}%
\settoheight{\Height}{$+Q$}\settodepth{\Depth}{$+Q$}\setlength{\Height}{\Depth}%
\put( -0.500,  1.600){\hspace*{\Width}\raisebox{\Height}{$+Q$}}%
%
\settowidth{\Width}{$-Q$}\setlength{\Width}{0\Width}%
\settoheight{\Height}{$-Q$}\settodepth{\Depth}{$-Q$}\setlength{\Height}{-\Height}%
\put( -0.900, -0.250){\hspace*{\Width}\raisebox{\Height}{$-Q$}}%
%
\settowidth{\Width}{図2}\setlength{\Width}{-0.5\Width}%
\settoheight{\Height}{図2}\settodepth{\Depth}{図2}\setlength{\Height}{-0.5\Height}\setlength{\Depth}{0.5\Depth}\addtolength{\Height}{\Depth}%
\put(  0.000, -1.000){\hspace*{\Width}\raisebox{\Height}{図2}}%
%
\settowidth{\Width}{$d$}\setlength{\Width}{-1\Width}%
\settoheight{\Height}{$d$}\settodepth{\Depth}{$d$}\setlength{\Height}{-0.5\Height}\setlength{\Depth}{0.5\Depth}\addtolength{\Height}{\Depth}%
\put(  0.400,  0.200){\hspace*{\Width}\raisebox{\Height}{$d$}}%
%
\settowidth{\Width}{$x$}\setlength{\Width}{-0.5\Width}%
\settoheight{\Height}{$x$}\settodepth{\Depth}{$x$}\setlength{\Height}{\Depth}%
\put( -0.300,  1.060){\hspace*{\Width}\raisebox{\Height}{$x$}}%
%
\end{picture}}%
        }
        極板A,Bからなるコンデンサーがあり,電荷$Q$\tanni{C}が充電されている。
        極板は一辺が$\ell $\tanni{m}の正方形で,極板間間隔$d$\tanni{m}である。
        極板間は真空で,電場は一様とし,真空の誘電率を$\varepsilon _0$\tanni{F/m}
        とする。\\
        ~~A,B間に,図2のような誘電体を挿入する。誘電体は一辺が$\ell $\tanni{m}の
        正方形で,厚さ$d$\tanni{m},比誘電率$\varepsilon _r$である。誘電体を$x$\tanni{m}だけ
        挿入したとき,誘電体の部分の電気容量は\Hako \tanni{F}であり,真空部分の電気容量は
        \Hako \tanni{F}だから,全体での電気容量は\Hako \tanni{F}となる。
        また,静電エネルギーは\Hako \tanni{J}となり,$x$が増すと\Hako する。
        したがって,この誘電体には$x$が\Hako する方向に静電気力が働くことが分かる。
        また,極板上の電荷の面密度(単位面積あたりの電気量)は,誘電体部分と真空部分では,\Hako の比となっている。
    \end{mawarikomi}
