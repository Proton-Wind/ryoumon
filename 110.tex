\hakosyokika
\item
    \begin{mawarikomi}(20pt,0pt){150pt}{
        %%% C:/vpn/vpn/KeTCindy/fig/fig110.tex 
%%% Generator=fig110.cdy 
{\unitlength=1cm%
\begin{picture}%
(6,4)(-2.5,-1.5)%
\special{pn 8}%
%
\special{pn 32}%
\special{pa  -492  -492}\special{pa   492  -492}%
\special{fp}%
\special{pn 8}%
\special{pn 32}%
\special{pa  -492   295}\special{pa   492   295}%
\special{fp}%
\special{pn 8}%
\special{pa  -492  -492}\special{pa  -787  -492}%
\special{fp}%
\special{pa  -492   295}\special{pa  -787   295}%
\special{fp}%
\special{pa  -492   -98}\special{pa   492   -98}\special{pa   492    98}\special{pa  -492    98}%
\special{pa  -492   -98}%
\special{fp}%
\special{pa 0 -492}\special{pa 0 -455}\special{fp}\special{pa 0 -417}\special{pa 0 -380}\special{fp}%
\special{pa 0 -342}\special{pa 0 -305}\special{fp}\special{pa 0 -267}\special{pa 0 -230}\special{fp}%
\special{pa 0 -192}\special{pa 0 -155}\special{fp}\special{pa 0 -117}\special{pa 0 -80}\special{fp}%
\special{pa 0 -42}\special{pa 0 -5}\special{fp}\special{pa 0 33}\special{pa 0 70}\special{fp}%
\special{pa 0 108}\special{pa 0 145}\special{fp}\special{pa 0 183}\special{pa 0 220}\special{fp}%
\special{pa 0 258}\special{pa 0 295}\special{fp}%
%
\special{pa  -787   -20}\special{pa  -866  -156}%
\special{fp}%
\special{pa  -787   295}\special{pa  -787   -20}%
\special{fp}%
\special{pa  -787  -492}\special{pa  -787  -177}%
\special{fp}%
{%
\color[cmyk]{0,0,0,0}%
\special{pa -772 -20}\special{pa -773 -22}\special{pa -773 -23}\special{pa -773 -25}%
\special{pa -774 -27}\special{pa -775 -28}\special{pa -776 -30}\special{pa -778 -31}%
\special{pa -779 -32}\special{pa -781 -33}\special{pa -783 -34}\special{pa -785 -34}%
\special{pa -786 -35}\special{pa -788 -35}\special{pa -790 -34}\special{pa -792 -34}%
\special{pa -794 -33}\special{pa -795 -32}\special{pa -797 -31}\special{pa -798 -30}%
\special{pa -800 -28}\special{pa -801 -27}\special{pa -801 -25}\special{pa -802 -23}%
\special{pa -802 -22}\special{pa -802 -20}\special{pa -802 -18}\special{pa -802 -16}%
\special{pa -801 -14}\special{pa -801 -12}\special{pa -800 -11}\special{pa -798 -9}%
\special{pa -797 -8}\special{pa -795 -7}\special{pa -794 -6}\special{pa -792 -5}\special{pa -790 -5}%
\special{pa -788 -5}\special{pa -786 -5}\special{pa -785 -5}\special{pa -783 -5}\special{pa -781 -6}%
\special{pa -779 -7}\special{pa -778 -8}\special{pa -776 -9}\special{pa -775 -11}%
\special{pa -774 -12}\special{pa -773 -14}\special{pa -773 -16}\special{pa -773 -18}%
\special{pa -772 -20}\special{pa -772 -20}\special{sh 1}\special{ip}%
}%
\special{pa  -772   -20}\special{pa  -773   -22}\special{pa  -773   -23}\special{pa  -773   -25}%
\special{pa  -774   -27}\special{pa  -775   -28}\special{pa  -776   -30}\special{pa  -778   -31}%
\special{pa  -779   -32}\special{pa  -781   -33}\special{pa  -783   -34}\special{pa  -785   -34}%
\special{pa  -786   -35}\special{pa  -788   -35}\special{pa  -790   -34}\special{pa  -792   -34}%
\special{pa  -794   -33}\special{pa  -795   -32}\special{pa  -797   -31}\special{pa  -798   -30}%
\special{pa  -800   -28}\special{pa  -801   -27}\special{pa  -801   -25}\special{pa  -802   -23}%
\special{pa  -802   -22}\special{pa  -802   -20}\special{pa  -802   -18}\special{pa  -802   -16}%
\special{pa  -801   -14}\special{pa  -801   -12}\special{pa  -800   -11}\special{pa  -798    -9}%
\special{pa  -797    -8}\special{pa  -795    -7}\special{pa  -794    -6}\special{pa  -792    -5}%
\special{pa  -790    -5}\special{pa  -788    -5}\special{pa  -786    -5}\special{pa  -785    -5}%
\special{pa  -783    -5}\special{pa  -781    -6}\special{pa  -779    -7}\special{pa  -778    -8}%
\special{pa  -776    -9}\special{pa  -775   -11}\special{pa  -774   -12}\special{pa  -773   -14}%
\special{pa  -773   -16}\special{pa  -773   -18}\special{pa  -772   -20}%
\special{fp}%
{%
\color[cmyk]{0,0,0,0}%
\special{pa -772 -177}\special{pa -773 -179}\special{pa -773 -181}\special{pa -773 -183}%
\special{pa -774 -184}\special{pa -775 -186}\special{pa -776 -187}\special{pa -778 -189}%
\special{pa -779 -190}\special{pa -781 -191}\special{pa -783 -191}\special{pa -785 -192}%
\special{pa -786 -192}\special{pa -788 -192}\special{pa -790 -192}\special{pa -792 -191}%
\special{pa -794 -191}\special{pa -795 -190}\special{pa -797 -189}\special{pa -798 -187}%
\special{pa -800 -186}\special{pa -801 -184}\special{pa -801 -183}\special{pa -802 -181}%
\special{pa -802 -179}\special{pa -802 -177}\special{pa -802 -175}\special{pa -802 -173}%
\special{pa -801 -172}\special{pa -801 -170}\special{pa -800 -168}\special{pa -798 -167}%
\special{pa -797 -166}\special{pa -795 -165}\special{pa -794 -164}\special{pa -792 -163}%
\special{pa -790 -162}\special{pa -788 -162}\special{pa -786 -162}\special{pa -785 -162}%
\special{pa -783 -163}\special{pa -781 -164}\special{pa -779 -165}\special{pa -778 -166}%
\special{pa -776 -167}\special{pa -775 -168}\special{pa -774 -170}\special{pa -773 -172}%
\special{pa -773 -173}\special{pa -773 -175}\special{pa -772 -177}\special{pa -772 -177}%
\special{sh 1}\special{ip}%
}%
\special{pa  -772  -177}\special{pa  -773  -179}\special{pa  -773  -181}\special{pa  -773  -183}%
\special{pa  -774  -184}\special{pa  -775  -186}\special{pa  -776  -187}\special{pa  -778  -189}%
\special{pa  -779  -190}\special{pa  -781  -191}\special{pa  -783  -191}\special{pa  -785  -192}%
\special{pa  -786  -192}\special{pa  -788  -192}\special{pa  -790  -192}\special{pa  -792  -191}%
\special{pa  -794  -191}\special{pa  -795  -190}\special{pa  -797  -189}\special{pa  -798  -187}%
\special{pa  -800  -186}\special{pa  -801  -184}\special{pa  -801  -183}\special{pa  -802  -181}%
\special{pa  -802  -179}\special{pa  -802  -177}\special{pa  -802  -175}\special{pa  -802  -173}%
\special{pa  -801  -172}\special{pa  -801  -170}\special{pa  -800  -168}\special{pa  -798  -167}%
\special{pa  -797  -166}\special{pa  -795  -165}\special{pa  -794  -164}\special{pa  -792  -163}%
\special{pa  -790  -162}\special{pa  -788  -162}\special{pa  -786  -162}\special{pa  -785  -162}%
\special{pa  -783  -163}\special{pa  -781  -164}\special{pa  -779  -165}\special{pa  -778  -166}%
\special{pa  -776  -167}\special{pa  -775  -168}\special{pa  -774  -170}\special{pa  -773  -172}%
\special{pa  -773  -173}\special{pa  -773  -175}\special{pa  -772  -177}%
\special{fp}%
\special{pa   709  -492}\special{pa   845  -571}%
\special{fp}%
\special{pa   492  -492}\special{pa   709  -492}%
\special{fp}%
\special{pa  1083  -492}\special{pa   866  -492}%
\special{fp}%
{%
\color[cmyk]{0,0,0,0}%
\special{pa 724 -492}\special{pa 724 -494}\special{pa 723 -496}\special{pa 723 -498}%
\special{pa 722 -499}\special{pa 721 -501}\special{pa 720 -502}\special{pa 718 -504}%
\special{pa 717 -505}\special{pa 715 -506}\special{pa 713 -506}\special{pa 711 -507}%
\special{pa 710 -507}\special{pa 708 -507}\special{pa 706 -507}\special{pa 704 -506}%
\special{pa 702 -506}\special{pa 701 -505}\special{pa 699 -504}\special{pa 698 -502}%
\special{pa 697 -501}\special{pa 696 -499}\special{pa 695 -498}\special{pa 694 -496}%
\special{pa 694 -494}\special{pa 694 -492}\special{pa 694 -490}\special{pa 694 -488}%
\special{pa 695 -487}\special{pa 696 -485}\special{pa 697 -483}\special{pa 698 -482}%
\special{pa 699 -481}\special{pa 701 -479}\special{pa 702 -479}\special{pa 704 -478}%
\special{pa 706 -477}\special{pa 708 -477}\special{pa 710 -477}\special{pa 711 -477}%
\special{pa 713 -478}\special{pa 715 -479}\special{pa 717 -479}\special{pa 718 -481}%
\special{pa 720 -482}\special{pa 721 -483}\special{pa 722 -485}\special{pa 723 -487}%
\special{pa 723 -488}\special{pa 724 -490}\special{pa 724 -492}\special{pa 724 -492}%
\special{sh 1}\special{ip}%
}%
\special{pa   724  -492}\special{pa   724  -494}\special{pa   723  -496}\special{pa   723  -498}%
\special{pa   722  -499}\special{pa   721  -501}\special{pa   720  -502}\special{pa   718  -504}%
\special{pa   717  -505}\special{pa   715  -506}\special{pa   713  -506}\special{pa   711  -507}%
\special{pa   710  -507}\special{pa   708  -507}\special{pa   706  -507}\special{pa   704  -506}%
\special{pa   702  -506}\special{pa   701  -505}\special{pa   699  -504}\special{pa   698  -502}%
\special{pa   697  -501}\special{pa   696  -499}\special{pa   695  -498}\special{pa   694  -496}%
\special{pa   694  -494}\special{pa   694  -492}\special{pa   694  -490}\special{pa   694  -488}%
\special{pa   695  -487}\special{pa   696  -485}\special{pa   697  -483}\special{pa   698  -482}%
\special{pa   699  -481}\special{pa   701  -479}\special{pa   702  -479}\special{pa   704  -478}%
\special{pa   706  -477}\special{pa   708  -477}\special{pa   710  -477}\special{pa   711  -477}%
\special{pa   713  -478}\special{pa   715  -479}\special{pa   717  -479}\special{pa   718  -481}%
\special{pa   720  -482}\special{pa   721  -483}\special{pa   722  -485}\special{pa   723  -487}%
\special{pa   723  -488}\special{pa   724  -490}\special{pa   724  -492}%
\special{fp}%
{%
\color[cmyk]{0,0,0,0}%
\special{pa 881 -492}\special{pa 881 -494}\special{pa 881 -496}\special{pa 880 -498}%
\special{pa 879 -499}\special{pa 878 -501}\special{pa 877 -502}\special{pa 876 -504}%
\special{pa 874 -505}\special{pa 873 -506}\special{pa 871 -506}\special{pa 869 -507}%
\special{pa 867 -507}\special{pa 865 -507}\special{pa 863 -507}\special{pa 862 -506}%
\special{pa 860 -506}\special{pa 858 -505}\special{pa 857 -504}\special{pa 855 -502}%
\special{pa 854 -501}\special{pa 853 -499}\special{pa 852 -498}\special{pa 852 -496}%
\special{pa 851 -494}\special{pa 851 -492}\special{pa 851 -490}\special{pa 852 -488}%
\special{pa 852 -487}\special{pa 853 -485}\special{pa 854 -483}\special{pa 855 -482}%
\special{pa 857 -481}\special{pa 858 -479}\special{pa 860 -479}\special{pa 862 -478}%
\special{pa 863 -477}\special{pa 865 -477}\special{pa 867 -477}\special{pa 869 -477}%
\special{pa 871 -478}\special{pa 873 -479}\special{pa 874 -479}\special{pa 876 -481}%
\special{pa 877 -482}\special{pa 878 -483}\special{pa 879 -485}\special{pa 880 -487}%
\special{pa 881 -488}\special{pa 881 -490}\special{pa 881 -492}\special{pa 881 -492}%
\special{sh 1}\special{ip}%
}%
\special{pa   881  -492}\special{pa   881  -494}\special{pa   881  -496}\special{pa   880  -498}%
\special{pa   879  -499}\special{pa   878  -501}\special{pa   877  -502}\special{pa   876  -504}%
\special{pa   874  -505}\special{pa   873  -506}\special{pa   871  -506}\special{pa   869  -507}%
\special{pa   867  -507}\special{pa   865  -507}\special{pa   863  -507}\special{pa   862  -506}%
\special{pa   860  -506}\special{pa   858  -505}\special{pa   857  -504}\special{pa   855  -502}%
\special{pa   854  -501}\special{pa   853  -499}\special{pa   852  -498}\special{pa   852  -496}%
\special{pa   851  -494}\special{pa   851  -492}\special{pa   851  -490}\special{pa   852  -488}%
\special{pa   852  -487}\special{pa   853  -485}\special{pa   854  -483}\special{pa   855  -482}%
\special{pa   857  -481}\special{pa   858  -479}\special{pa   860  -479}\special{pa   862  -478}%
\special{pa   863  -477}\special{pa   865  -477}\special{pa   867  -477}\special{pa   869  -477}%
\special{pa   871  -478}\special{pa   873  -479}\special{pa   874  -479}\special{pa   876  -481}%
\special{pa   877  -482}\special{pa   878  -483}\special{pa   879  -485}\special{pa   880  -487}%
\special{pa   881  -488}\special{pa   881  -490}\special{pa   881  -492}%
\special{fp}%
\special{pa   984  -266}\special{pa  1181  -266}%
\special{fp}%
\special{pn 16}%
\special{pa  1043  -226}\special{pa  1122  -226}%
\special{fp}%
\special{pn 8}%
\special{pa  1083    -0}\special{pa  1083  -226}%
\special{fp}%
\special{pa  1083  -492}\special{pa  1083  -266}%
\special{fp}%
\special{pa 1098 0}\special{pa 1098 -2}\special{pa 1097 -4}\special{pa 1097 -6}\special{pa 1096 -7}%
\special{pa 1095 -9}\special{pa 1094 -10}\special{pa 1092 -12}\special{pa 1091 -13}%
\special{pa 1089 -14}\special{pa 1087 -14}\special{pa 1085 -15}\special{pa 1084 -15}%
\special{pa 1082 -15}\special{pa 1080 -15}\special{pa 1078 -14}\special{pa 1076 -14}%
\special{pa 1075 -13}\special{pa 1073 -12}\special{pa 1072 -10}\special{pa 1071 -9}%
\special{pa 1070 -7}\special{pa 1069 -6}\special{pa 1068 -4}\special{pa 1068 -2}\special{pa 1068 0}%
\special{pa 1068 2}\special{pa 1068 4}\special{pa 1069 6}\special{pa 1070 7}\special{pa 1071 9}%
\special{pa 1072 10}\special{pa 1073 12}\special{pa 1075 13}\special{pa 1076 14}\special{pa 1078 14}%
\special{pa 1080 15}\special{pa 1082 15}\special{pa 1084 15}\special{pa 1085 15}\special{pa 1087 14}%
\special{pa 1089 14}\special{pa 1091 13}\special{pa 1092 12}\special{pa 1094 10}\special{pa 1095 9}%
\special{pa 1096 7}\special{pa 1097 6}\special{pa 1097 4}\special{pa 1098 2}\special{pa 1098 0}%
\special{pa 1098 0}\special{sh 1}\special{ip}%
\special{pa  1098    -0}\special{pa  1098    -2}\special{pa  1097    -4}\special{pa  1097    -6}%
\special{pa  1096    -7}\special{pa  1095    -9}\special{pa  1094   -10}\special{pa  1092   -12}%
\special{pa  1091   -13}\special{pa  1089   -14}\special{pa  1087   -14}\special{pa  1085   -15}%
\special{pa  1084   -15}\special{pa  1082   -15}\special{pa  1080   -15}\special{pa  1078   -14}%
\special{pa  1076   -14}\special{pa  1075   -13}\special{pa  1073   -12}\special{pa  1072   -10}%
\special{pa  1071    -9}\special{pa  1070    -7}\special{pa  1069    -6}\special{pa  1068    -4}%
\special{pa  1068    -2}\special{pa  1068     0}\special{pa  1068     2}\special{pa  1068     4}%
\special{pa  1069     6}\special{pa  1070     7}\special{pa  1071     9}\special{pa  1072    10}%
\special{pa  1073    12}\special{pa  1075    13}\special{pa  1076    14}\special{pa  1078    14}%
\special{pa  1080    15}\special{pa  1082    15}\special{pa  1084    15}\special{pa  1085    15}%
\special{pa  1087    14}\special{pa  1089    14}\special{pa  1091    13}\special{pa  1092    12}%
\special{pa  1094    10}\special{pa  1095     9}\special{pa  1096     7}\special{pa  1097     6}%
\special{pa  1097     4}\special{pa  1098     2}\special{pa  1098     0}%
\special{fp}%
\special{pa  1083    -0}\special{pa  1083   197}%
\special{fp}%
\special{pa   984   197}\special{pa  1181   197}%
\special{fp}%
\special{pa  1024   236}\special{pa  1142   236}%
\special{fp}%
\special{pa  1063   276}\special{pa  1102   276}%
\special{fp}%
\settowidth{\Width}{$\mathrm{K_2}$}\setlength{\Width}{-1\Width}%
\settoheight{\Height}{$\mathrm{K_2}$}\settodepth{\Depth}{$\mathrm{K_2}$}\setlength{\Height}{\Depth}%
\put( -2.150,  0.500){\hspace*{\Width}\raisebox{\Height}{$\mathrm{K_2}$}}%
%
\settowidth{\Width}{$\mathrm{K_1}$}\setlength{\Width}{0\Width}%
\settoheight{\Height}{$\mathrm{K_1}$}\settodepth{\Depth}{$\mathrm{K_1}$}\setlength{\Height}{\Depth}%
\put(  2.250,  1.400){\hspace*{\Width}\raisebox{\Height}{$\mathrm{K_1}$}}%
%
\settowidth{\Width}{E}\setlength{\Width}{0\Width}%
\settoheight{\Height}{E}\settodepth{\Depth}{E}\setlength{\Height}{\Depth}%
\put(  3.150,  0.670){\hspace*{\Width}\raisebox{\Height}{E}}%
%
\settowidth{\Width}{G}\setlength{\Width}{0\Width}%
\settoheight{\Height}{G}\settodepth{\Depth}{G}\setlength{\Height}{\Depth}%
\put(  2.900, -0.350){\hspace*{\Width}\raisebox{\Height}{G}}%
%
\settowidth{\Width}{L}\setlength{\Width}{0\Width}%
\settoheight{\Height}{L}\settodepth{\Depth}{L}\setlength{\Height}{-\Height}%
\put(  0.150,  1.100){\hspace*{\Width}\raisebox{\Height}{L}}%
%
\settowidth{\Width}{M}\setlength{\Width}{0\Width}%
\settoheight{\Height}{M}\settodepth{\Depth}{M}\setlength{\Height}{\Depth}%
\put(  0.150, -0.600){\hspace*{\Width}\raisebox{\Height}{M}}%
%
\settowidth{\Width}{A}\setlength{\Width}{-0.5\Width}%
\settoheight{\Height}{A}\settodepth{\Depth}{A}\setlength{\Height}{\Depth}%
\put( -1.250,  1.400){\hspace*{\Width}\raisebox{\Height}{A}}%
%
\settowidth{\Width}{B}\setlength{\Width}{-0.5\Width}%
\settoheight{\Height}{B}\settodepth{\Depth}{B}\setlength{\Height}{-\Height}%
\put( -1.250, -0.900){\hspace*{\Width}\raisebox{\Height}{B}}%
%
\settowidth{\Width}{D}\setlength{\Width}{-1\Width}%
\settoheight{\Height}{D}\settodepth{\Depth}{D}\setlength{\Height}{-0.5\Height}\setlength{\Depth}{0.5\Depth}\addtolength{\Height}{\Depth}%
\put( -1.400,  0.000){\hspace*{\Width}\raisebox{\Height}{D}}%
%
\settowidth{\Width}{$\mathrm{D_1}$}\setlength{\Width}{-0.5\Width}%
\settoheight{\Height}{$\mathrm{D_1}$}\settodepth{\Depth}{$\mathrm{D_1}$}\setlength{\Height}{\Depth}%
\put( -0.500,  0.350){\hspace*{\Width}\raisebox{\Height}{$\mathrm{D_1}$}}%
%
\settowidth{\Width}{$\mathrm{D_2}$}\setlength{\Width}{-0.5\Width}%
\settoheight{\Height}{$\mathrm{D_2}$}\settodepth{\Depth}{$\mathrm{D_2}$}\setlength{\Height}{-\Height}%
\put( -0.500, -0.350){\hspace*{\Width}\raisebox{\Height}{$\mathrm{D_2}$}}%
%
\special{pa   492    -0}\special{pa   515     1}\special{pa   536     1}\special{pa   557     0}%
\special{pa   577    -2}\special{pa   597    -5}\special{pa   615    -9}\special{pa   633   -15}%
\special{pa   650   -22}\special{pa   666   -30}\special{pa   681   -39}\special{pa   696   -47}%
\special{pa   710   -52}\special{pa   725   -55}\special{pa   739   -55}\special{pa   753   -52}%
\special{pa   767   -46}\special{pa   780   -37}\special{pa   794   -26}\special{pa   807   -11}%
\special{pa   820     6}\special{pa   834    23}\special{pa   847    37}\special{pa   860    48}%
\special{pa   874    56}\special{pa   887    61}\special{pa   901    63}\special{pa   915    62}%
\special{pa   929    57}\special{pa   943    50}\special{pa   958    40}\special{pa   972    29}%
\special{pa   986    20}\special{pa   999    12}\special{pa  1012     6}\special{pa  1025     1}%
\special{pa  1037    -2}\special{pa  1049    -4}\special{pa  1061    -4}\special{pa  1072    -3}%
\special{pa  1083    -0}%
\special{fp}%
\special{pa 507 0}\special{pa 507 -2}\special{pa 507 -4}\special{pa 506 -6}\special{pa 505 -7}%
\special{pa 504 -9}\special{pa 503 -10}\special{pa 502 -12}\special{pa 500 -13}\special{pa 498 -14}%
\special{pa 497 -14}\special{pa 495 -15}\special{pa 493 -15}\special{pa 491 -15}\special{pa 489 -15}%
\special{pa 488 -14}\special{pa 486 -14}\special{pa 484 -13}\special{pa 483 -12}\special{pa 481 -10}%
\special{pa 480 -9}\special{pa 479 -7}\special{pa 478 -6}\special{pa 478 -4}\special{pa 477 -2}%
\special{pa 477 0}\special{pa 477 2}\special{pa 478 4}\special{pa 478 6}\special{pa 479 7}%
\special{pa 480 9}\special{pa 481 10}\special{pa 483 12}\special{pa 484 13}\special{pa 486 14}%
\special{pa 488 14}\special{pa 489 15}\special{pa 491 15}\special{pa 493 15}\special{pa 495 15}%
\special{pa 497 14}\special{pa 498 14}\special{pa 500 13}\special{pa 502 12}\special{pa 503 10}%
\special{pa 504 9}\special{pa 505 7}\special{pa 506 6}\special{pa 507 4}\special{pa 507 2}%
\special{pa 507 0}\special{pa 507 0}\special{sh 1}\special{ip}%
\special{pa   507    -0}\special{pa   507    -2}\special{pa   507    -4}\special{pa   506    -6}%
\special{pa   505    -7}\special{pa   504    -9}\special{pa   503   -10}\special{pa   502   -12}%
\special{pa   500   -13}\special{pa   498   -14}\special{pa   497   -14}\special{pa   495   -15}%
\special{pa   493   -15}\special{pa   491   -15}\special{pa   489   -15}\special{pa   488   -14}%
\special{pa   486   -14}\special{pa   484   -13}\special{pa   483   -12}\special{pa   481   -10}%
\special{pa   480    -9}\special{pa   479    -7}\special{pa   478    -6}\special{pa   478    -4}%
\special{pa   477    -2}\special{pa   477     0}\special{pa   477     2}\special{pa   478     4}%
\special{pa   478     6}\special{pa   479     7}\special{pa   480     9}\special{pa   481    10}%
\special{pa   483    12}\special{pa   484    13}\special{pa   486    14}\special{pa   488    14}%
\special{pa   489    15}\special{pa   491    15}\special{pa   493    15}\special{pa   495    15}%
\special{pa   497    14}\special{pa   498    14}\special{pa   500    13}\special{pa   502    12}%
\special{pa   503    10}\special{pa   504     9}\special{pa   505     7}\special{pa   506     6}%
\special{pa   507     4}\special{pa   507     2}\special{pa   507     0}%
\special{fp}%
\end{picture}}%
        }
        図でA,B,Dは同じ大きさの金属板で,A,Bは薄く,Dの厚さは$d$である。
        それらは平行で,AD,DBの間隔がそれぞれ$2d$と$d$になるように配置されている。
        また,Eは起電力$V$の電池,$\mathrm{K_1}$,$\mathrm{K_2}$はスイッチ,Gは接地点である。
        なお,金属板A,Bだけで間隔$d$のコンデンサーをつくったときの電気容量は$C$である。
        まず,$\mathrm{K_1}$,$\mathrm{K_2}$を閉じる。
        \begin{Enumerate}
            \item DのA,Bに対向した面$\mathrm{D_1}$,$\mathrm{D_2}$に現れる電荷はそれぞれいくらか。
            \item 図の直線LM上の各点の電位を,Lからの距離を横軸にとってグラフに描け。接地点の電位を0とする。
            \item 同様に,直線LMの上の各点の電場をグラフに描け。ただし,LからMに向かう向きを電場の正とする。
        \end{Enumerate}
        次に,$\mathrm{K_1}$と$\mathrm{K_2}$を共に開き,Dを平行に保ったまま,距離$d$だけ
        Aの方に動かす。
        \begin{Enumerate*}
            \item 動かした後の直線LM上の各点の電位をグラフに描け。
            \item Dを移動させるのに必要な仕事はいくらか。
        \end{Enumerate*}
    \end{mawarikomi}
