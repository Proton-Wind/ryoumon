\hakosyokika
\item
    \begin{mawarikomi}{170pt}{%WinTpicVersion4.32a
{\unitlength 0.1in%
\begin{picture}(21.6800,14.0000)(0.3200,-16.0000)%
% POLYGON 2 5 0 0 Black White  
% 5 400 1400 400 1600 2200 1600 2200 1400 400 1400
% 
\special{pn 0}%
\special{sh 0.400}%
\special{pa 400 1400}%
\special{pa 400 1600}%
\special{pa 2200 1600}%
\special{pa 2200 1400}%
\special{pa 400 1400}%
\special{ip}%
\special{pn 8}%
\special{pa 400 1400}%
\special{pa 400 1600}%
\special{pa 2200 1600}%
\special{pa 2200 1400}%
\special{pa 400 1400}%
\special{ip}%
% POLYGON 2 5 1 0 Black White  
% 6 400 1400 400 800 2200 800 2200 1400 2200 1400 400 1400
% 
\special{pn 0}%
\special{sh 0.200}%
\special{pa 400 1400}%
\special{pa 400 800}%
\special{pa 2200 800}%
\special{pa 2200 1400}%
\special{pa 2200 1400}%
\special{pa 400 1400}%
\special{ip}%
\special{pn 8}%
\special{pa 400 1400}%
\special{pa 400 800}%
\special{pa 2200 800}%
\special{pa 2200 1400}%
\special{pa 400 1400}%
\special{ip}%
% LINE 2 0 3 0 Black White  
% 4 400 800 2200 800 2200 1400 400 1400
% 
\special{pn 8}%
\special{pa 400 800}%
\special{pa 2200 800}%
\special{fp}%
\special{pa 2200 1400}%
\special{pa 400 1400}%
\special{fp}%
% LINE 2 0 3 0 Black White  
% 2 400 200 1000 800
% 
\special{pn 8}%
\special{pa 400 200}%
\special{pa 1000 800}%
\special{fp}%
% LINE 2 0 3 0 Black White  
% 2 1000 200 1600 800
% 
\special{pn 8}%
\special{pa 1000 200}%
\special{pa 1600 800}%
\special{fp}%
% VECTOR 2 0 3 0 Black White  
% 2 1600 800 2000 400
% 
\special{pn 8}%
\special{pa 1600 800}%
\special{pa 2000 400}%
\special{fp}%
\special{sh 1}%
\special{pa 2000 400}%
\special{pa 1939 433}%
\special{pa 1962 438}%
\special{pa 1967 461}%
\special{pa 2000 400}%
\special{fp}%
% LINE 2 0 3 0 Black White  
% 4 1000 800 1300 1400 1300 1400 1600 800
% 
\special{pn 8}%
\special{pa 1000 800}%
\special{pa 1300 1400}%
\special{fp}%
\special{pa 1300 1400}%
\special{pa 1600 800}%
\special{fp}%
% VECTOR 2 0 3 0 Black White  
% 2 600 400 640 440
% 
\special{pn 8}%
\special{pa 600 400}%
\special{pa 640 440}%
\special{fp}%
\special{sh 1}%
\special{pa 640 440}%
\special{pa 607 379}%
\special{pa 602 402}%
\special{pa 579 407}%
\special{pa 640 440}%
\special{fp}%
% VECTOR 2 0 3 0 Black White  
% 2 1120 320 1160 360
% 
\special{pn 8}%
\special{pa 1120 320}%
\special{pa 1160 360}%
\special{fp}%
\special{sh 1}%
\special{pa 1160 360}%
\special{pa 1127 299}%
\special{pa 1122 322}%
\special{pa 1099 327}%
\special{pa 1160 360}%
\special{fp}%
% LINE 2 1 3 0 Black White  
% 2 1000 800 1280 480
% 
\special{pn 8}%
\special{pa 1000 800}%
\special{pa 1280 480}%
\special{da 0.015}%
% LINE 2 1 3 0 Black White  
% 2 1600 800 1120 1040
% 
\special{pn 8}%
\special{pa 1600 800}%
\special{pa 1120 1040}%
\special{da 0.015}%
% LINE 2 0 3 0 Black White  
% 4 1192 1004 1156 933 1156 933 1084 968
% 
\special{pn 8}%
\special{pa 1192 1004}%
\special{pa 1156 933}%
\special{fp}%
\special{pa 1156 933}%
\special{pa 1084 968}%
\special{fp}%
% LINE 2 0 3 0 Black White  
% 4 1221 551 1283 601 1283 601 1335 540
% 
\special{pn 8}%
\special{pa 1221 551}%
\special{pa 1283 601}%
\special{fp}%
\special{pa 1283 601}%
\special{pa 1335 540}%
\special{fp}%
% LINE 2 1 3 0 Black White  
% 2 1000 396 1000 1236
% 
\special{pn 8}%
\special{pa 1000 396}%
\special{pa 1000 1236}%
\special{da 0.015}%
% CIRCLE 2 0 3 0 Black White  
% 4 1000 796 1000 1036 1000 1396 1300 1396
% 
\special{pn 8}%
\special{ar 1000 796 240 240 1.1071487 1.5707963}%
% STR 2 0 3 0 Black White  
% 4 1062 1082 1062 1102 5 0 0 0
% $\phi$
\put(10.6200,-11.0200){\makebox(0,0){$\phi$}}%
% STR 2 0 3 0 Black White  
% 4 1280 460 1280 480 2 0 0 0
% $\mathrm{A_2}$
\put(12.8000,-4.8000){\makebox(0,0)[lb]{$\mathrm{A_2}$}}%
% STR 2 0 3 0 Black White  
% 4 1000 806 1000 826 4 0 0 0
% $\mathrm{A_1}$
\put(10.0000,-8.2600){\makebox(0,0)[rt]{$\mathrm{A_1}$}}%
% STR 2 0 3 0 Black White  
% 4 1300 1460 1300 1480 5 0 0 0
% $\mathrm{C}$
\put(13.0000,-14.8000){\makebox(0,0){$\mathrm{C}$}}%
% STR 2 0 3 0 Black White  
% 4 1626 808 1626 828 1 0 0 0
% $\mathrm{B_2}$
\put(16.2600,-8.2800){\makebox(0,0)[lt]{$\mathrm{B_2}$}}%
% STR 2 0 3 0 Black White  
% 4 1188 1032 1188 1052 1 0 0 0
% $\mathrm{B_2}$
\put(11.8800,-10.5200){\makebox(0,0)[lt]{$\mathrm{B_2}$}}%
% STR 2 0 3 0 Black White  
% 4 2014 372 2014 392 1 0 0 0
% D
\put(20.1400,-3.9200){\makebox(0,0)[lt]{D}}%
% STR 2 0 3 0 Black White  
% 4 1894 746 1894 766 2 0 0 0
% 空気
\put(18.9400,-7.6600){\makebox(0,0)[lb]{空気}}%
% STR 2 0 3 0 Black White  
% 4 1894 1186 1894 1206 2 0 0 0
% 薄膜
\put(18.9400,-12.0600){\makebox(0,0)[lb]{薄膜}}%
% STR 2 0 3 0 Black White  
% 4 1734 1546 1734 1566 2 0 0 0
% 媒質G
\put(17.3400,-15.6600){\makebox(0,0)[lb]{媒質G}}%
% VECTOR 2 0 3 0 Black White  
% 4 1122 1040 1162 1120 1442 1120 1482 1040
% 
\special{pn 8}%
\special{pa 1122 1040}%
\special{pa 1162 1120}%
\special{fp}%
\special{sh 1}%
\special{pa 1162 1120}%
\special{pa 1150 1051}%
\special{pa 1138 1072}%
\special{pa 1114 1069}%
\special{pa 1162 1120}%
\special{fp}%
\special{pa 1442 1120}%
\special{pa 1482 1040}%
\special{fp}%
\special{sh 1}%
\special{pa 1482 1040}%
\special{pa 1434 1091}%
\special{pa 1458 1088}%
\special{pa 1470 1109}%
\special{pa 1482 1040}%
\special{fp}%
\end{picture}}%
}
        媒質Gの上に厚さ$d$の薄膜があり,空気中から単色光が斜めに入射する場合の干渉を考える。空気中での光の波長を$\lambda $,屈折角を$\phi $とする。また,空気,薄膜,Gの屈折率はそれぞれ1,$n$,
        $n\mathrm{_G}$,とする。$\mathrm{A_1A_2}$は入射波の波面で,$\mathrm{B_1B_2}$は屈折波の波面である。同じ波面上では同位相である。したがって,$\mathrm{A_1}\rightarrow\mathrm{C}\rightarrow \mathrm{B_2}\rightarrow \mathrm{D}$の経路をとる光と,$\mathrm{A_2}\rightarrow \mathrm{B_2}\rightarrow \mathrm{D}$の経路をとる光との間に位相差をもたらす経路の差は,$\mathrm{B_1C+CB_2}$になる。この長さは$d$,$\phi$を用いて表すと\Hako となる。これらの2つの光が点$\mathrm{B_2}$で同位相であれば干渉により強め合い,Dの方向から観測すると反射光は明るく見える。薄膜の中では光の波長は\Hako である。$\mathrm{n<n_G}$の場合,各反射面での反射光の位相のずれの有無を考慮すると,干渉して反射光が明るくなる条件は,正の整数$m$を用いて\Hako と書ける。$\lambda = 6.0\times 10^{-7}$\tanni{m},$\phi = 60\Deg $,$n=1.5$,$n\mathrm{_G}=1.6$のとき,反射光が明るくなる薄膜の最小の厚さは\Hako \tanni{m}である。またGのみを替え,$n\mathrm{_G}=1.4$とすると,明るくなる最小の厚さは\Hako \tanni{m}となる。
    \end{mawarikomi}