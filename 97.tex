\hakosyokika
\item
    \begin{mawarikomi}{170pt}{\input{./fig/fig097.tex}}
        媒質Gの上に厚さ$d$の薄膜があり,空気中から単色光が斜めに入射する場合の干渉を考える。空気中での光の波長を$\lambda $,屈折角を$\phi $とする。また,空気,薄膜,Gの屈折率はそれぞれ1,$n$,
        $n\mathrm{_G}$,とする。$\mathrm{A_1A_2}$は入射波の波面で,$\mathrm{B_1B_2}$は屈折波の波面である。同じ波面上では同位相である。したがって,$\mathrm{A_1}\rightarrow\mathrm{C}\rightarrow \mathrm{B_2}\rightarrow \mathrm{D}$の経路をとる光と,$\mathrm{A_2}\rightarrow \mathrm{B_2}\rightarrow \mathrm{D}$の経路をとる光との間に位相差をもたらす経路の差は,$\mathrm{B_1C+CB_2}$になる。この長さは$d$,$\phi$を用いて表すと\Hako となる。これらの2つの光が点$\mathrm{B_2}$で同位相であれば干渉により強め合い,Dの方向から観測すると反射光は明るく見える。薄膜の中では光の波長は\Hako である。$\mathrm{n<n_G}$の場合,各反射面での反射光の位相のずれの有無を考慮すると,干渉して反射光が明るくなる条件は,正の整数$m$を用いて\Hako と書ける。$\lambda = 6.0\times 10^{-7}$\tanni{m},$\phi = 60\Deg $,$n=1.5$,$n\mathrm{_G}=1.6$のとき,反射光が明るくなる薄膜の最小の厚さは\Hako \tanni{m}である。またGのみを替え,$n\mathrm{_G}=1.4$とすると,明るくなる最小の厚さは\Hako \tanni{m}となる。
    \end{mawarikomi}