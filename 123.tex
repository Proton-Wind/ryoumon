\hakosyokika
\item
    \begin{mawarikomi}(10pt,0pt){80pt}{
        %%% C:/vpn/KeTCindy/fig/fig123.tex 
%%% Generator=fig123.cdy 
{\unitlength=1cm%
\begin{picture}%
(5,5)(-2.5,-2.5)%
\special{pn 8}%
%
\special{pa 243 -574}\special{pa 295 -591}\special{pa 243 -608}\special{pa 253 -591}%
\special{pa 243 -574}\special{pa 243 -574}\special{sh 1}\special{ip}%
\special{pn 1}%
\special{pa   243  -574}\special{pa   295  -591}\special{pa   243  -608}\special{pa   253  -591}%
\special{pa   243  -574}%
\special{fp}%
\special{pn 8}%
\special{pa  -295  -591}\special{pa   253  -591}%
\special{fp}%
\special{pa -214 46}\special{pa -197 98}\special{pa -180 46}\special{pa -197 56}\special{pa -214 46}%
\special{pa -214 46}\special{sh 1}\special{ip}%
\special{pn 1}%
\special{pa  -214    46}\special{pa  -197    98}\special{pa  -180    46}\special{pa  -197    56}%
\special{pa  -214    46}%
\special{fp}%
\special{pn 8}%
\special{pa  -197  -295}\special{pa  -197    56}%
\special{fp}%
{%
\color[cmyk]{0,0,0,0.3}%
\special{pa -98 -492}\special{pa 98 -492}\special{pa 98 295}\special{pa -98 295}\special{pa -98 -492}%
\special{pa -98 -492}\special{sh 1}\special{ip}%
}%
\special{pa   -98  -492}\special{pa    98  -492}\special{pa    98   295}\special{pa   -98   295}%
\special{pa   -98  -492}%
\special{fp}%
\settowidth{\Width}{$I$}\setlength{\Width}{-1\Width}%
\settoheight{\Height}{$I$}\settodepth{\Depth}{$I$}\setlength{\Height}{-0.5\Height}\setlength{\Depth}{0.5\Depth}\addtolength{\Height}{\Depth}%
\put( -0.600, -0.250){\hspace*{\Width}\raisebox{\Height}{$I$}}%
%
\settowidth{\Width}{$B$}\setlength{\Width}{0\Width}%
\settoheight{\Height}{$B$}\settodepth{\Depth}{$B$}\setlength{\Height}{-0.5\Height}\setlength{\Depth}{0.5\Depth}\addtolength{\Height}{\Depth}%
\put(  0.850,  1.500){\hspace*{\Width}\raisebox{\Height}{$B$}}%
%
\special{pa -295 -492}\special{pa -256 -492}\special{fp}\special{pa -217 -492}\special{pa -177 -492}\special{fp}%
\special{pa -138 -492}\special{pa -98 -492}\special{fp}%
%
\special{pa -295 295}\special{pa -256 295}\special{fp}\special{pa -217 295}\special{pa -177 295}\special{fp}%
\special{pa -138 295}\special{pa -98 295}\special{fp}%
%
\special{pa 155 -440}\special{pa 138 -492}\special{pa 121 -440}\special{pa 138 -450}%
\special{pa 155 -440}\special{pa 155 -440}\special{sh 1}\special{ip}%
\special{pn 1}%
\special{pa   155  -440}\special{pa   138  -492}\special{pa   121  -440}\special{pa   138  -450}%
\special{pa   155  -440}%
\special{fp}%
\special{pn 8}%
\special{pa   138   -98}\special{pa   138  -450}%
\special{fp}%
\special{pa 121 243}\special{pa 138 295}\special{pa 155 243}\special{pa 138 253}\special{pa 121 243}%
\special{pa 121 243}\special{sh 1}\special{ip}%
\special{pn 1}%
\special{pa   121   243}\special{pa   138   295}\special{pa   155   243}\special{pa   138   253}%
\special{pa   121   243}%
\special{fp}%
\special{pn 8}%
\special{pa   138   -98}\special{pa   138   253}%
\special{fp}%
\settowidth{\Width}{$\ell $}\setlength{\Width}{0\Width}%
\settoheight{\Height}{$\ell $}\settodepth{\Depth}{$\ell $}\setlength{\Height}{-0.5\Height}\setlength{\Depth}{0.5\Depth}\addtolength{\Height}{\Depth}%
\put(  0.450,  0.250){\hspace*{\Width}\raisebox{\Height}{$\ell $}}%
%
\settowidth{\Width}{$S$}\setlength{\Width}{-0.5\Width}%
\settoheight{\Height}{$S$}\settodepth{\Depth}{$S$}\setlength{\Height}{-\Height}%
\put(  0.000, -0.850){\hspace*{\Width}\raisebox{\Height}{$S$}}%
%
\end{picture}}%
    }
    図のように,磁束密度$B$\tanni{T}で右向きの磁場中に置かれた導線内を電流$I$\tanni{A}が下向きに流れている。導線の断面積を$S$\tanni{m^2},長さを$\ell$\tanni{m}とし,導線内の自由電子の個数密度を$n$\tanni{個/m^3},自由電子の速さ$v$\tanni{m/s},電荷を$-e$\tanni{C}とする。まず,電流$I$は$I=$\Hako \tanni{A}と表される。一方,1つの自由電子は磁場から大きさ\Hako \tanni{N}の\Hako 力と呼ばれる力を,図で\Hako 向きに受ける。導線内の電子の総数は\Hako 個であるから,導線を流れる電流が磁場から受ける力はの大きさ$F$は\Hako \tanni{N}となり,$I$を用いると,\Hako \tanni{N}と表され,力の向きは\Hako 向きとなる。
    \end{mawarikomi}