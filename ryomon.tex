\documentclass[a4paper,9pt]{jsarticle}
% \documentclass[b5j,9.5pt]{jsbook}
\usepackage[noalphabet]{pxchfon}
\setminchofont{UDDIGIKYOKASHON-R.TTC} 
\setgothicfont{UDDIGIKYOKASHON-B.TTC} 
% \setminchofont{BIZ-UDMINCHOM.TTC} 
% \setgothicfont{BIZ-UDGOTHICR.TTC} 
\usepackage{okumacro}
\usepackage{amsmath,amsthm,amssymb,fancybox}
\usepackage{enumerate,multicol}
\usepackage{ascmac,itembbox,emath,hako,scrpage,ulinej,emathP,emathPp,emathMw,emathEy}
\usepackage[version=4]{mhchem}
%\usepackage[draft]{graphicx}
\usepackage{graphicx}
\usepackage{picins}
% \usepackage{amsmath}
% \usepackage{amsthm,amssymb,color}
% \usepackage{enumerate,multicol,mystyle}
% \usepackage{picins,emath,emathP,emathPp,itembbox,hako}
% \usepackage{emathMw}
% \usepackage{graphicx}
%\usepackage{array}
\pagestyle{empty}
\setlength{\textheight}{265mm}
\setlength{\textwidth}{195mm}
\setlength{\oddsidemargin}{-15.4mm}
\setlength{\evensidemargin}{-15.4mm}
\setlength{\topmargin}{-25.4mm}
\renewcommand\labelenumi{\fbox{\bfseries{\sffamily{\theenumi}}}}
\renewcommand{\labelenumii}{(\arabic{enumii})}
\renewcommand{\labelenumiii}{\bfseries{\カタカナ{enumiii}.}}
\def\genshi#1#2#3{$^{#1}_{#2}${{\large \sffamily{#3}}}}
\def\gen#1{\large \sf{#1}}
\def\hic{cal/g$\cdot $K}
\def\hij{J/g$\cdot $K}
\def\tanni#1{$〔\mathrm{\sf #1}〕\kern -2pt$}%
\def\sftanni#1{$\kern 2pt{\mathrm{\sf #1}}$}
\def\gen#1#2#3{{\sf{\ce{_{#2}^{#1}#3}}}}

\begin{document}
\hakosyokika
\begin{center}
{\Large 物理補習}
\end{center}
\hfill ~\underline{~~~~~~番 氏名~~~~~~~~~~~~~~~~~~~~~~~~~~~~~~~~~~~}
\hakosyokika

\begin{enumerate}
% \hakosyokika
\item
    \begin{mawarikomi}{150pt}{\begin{zahyou*}[ul=6mm](-1,8)(-1.5,1)
    \ArrowLine{(\xmin,0)}{(\xmax,0)}
    \def\O{(0,0)}
    \def\A{(4,0)}
    \def\B{(3.5,0.5)}
    \def\C{(3.5,0)}
    \def\D{(4.5,0)}
    \def\E{(4.5,0.5)}
    \def\vvec{(1.5,0)}
    \small
    \put(5,-1.5){図1}
    \put(4.5,0.25){\Yasen\vvec}
    \Put\E{$v$\tanni{m/s}}
    \Nuritubusi{\B\C\D\E\B}
    \Drawline{\B\C\D\E\B}
    \Put{(\xmax,0)}[s]{$x$\tanni{m}}
    \xmemori{0}
    \xmemori<x>{4}
    % \zahyouMemori[g][n]<dx=1,dy=4>
\end{zahyou*}
\\
        \begin{zahyou}[ul=6mm,
                xscale=0.50,
                yscale=0.25,
                yokozikukigou=$t$\tanni{s},
                tatezikukigou=$v$\tanni{m/s}
                ](-1,19)(-4,20)
    \def\O{(0,0)}
    \def\A{(4,16)}
    \def\B{(7,16)}
    \def\C{(18,-6)}
    \small
    \put(13,9){図2}
    \Put\A[syaei=xy]{}
    \Put\B[syaei=xy]{}
    \xmemori{15}
    {\thicklines
    \Drawline{\O\A\B\C}}
    % \zahyouMemori[g][n]<dx=1,dy=4>
\end{zahyou}
}
        図1のように,$x$軸上を運動する物体があり,時刻$t$での速度$v$が図2で表される。時刻$t=0$での物体の位置を原点$x=0$とする。
        \begin{enumerate}
            \item 時刻$t=2$\sftanni{s}における物体の加速度$a$は\Hako \sftanni{m/s^2}であり,時刻$t=6$\sftanni{s}での加速度$a$は\Hako \sftanni{m/s^2}であり,$t=11$\sftanni{s}での加速度$a$は\Hako \sftanni{m/s^2}である。
            \item 時刻$t=6$\sftanni{s}における物体の位置$x$は\Hako \sftanni{m}である。
            \item 物体が原点$x=0$から右に最も離れる時刻$t$は\Hako \sftanni{s}であり,その位置$x$は\Hako \sftanni{m}である。
            \item 時刻$t=15$\sftanni{s}以後も,そのまま運動を続けた場合,物体が再び原点に戻ってくる時刻$t$は\Hako \sftanni{s}であり,そのときの速度$v$は\Hako \sftanni{m/s}である。
        \end{enumerate}
    \end{mawarikomi}
% \newpage
% \hakosyokika
\item 高さ144\sftanni{m}の高層ビルの屋上までエレベーターで昇った。はじめ地上に静止していたエレベーターは,最初の6秒間は一定の加速度$a$で,次の8秒間は一定の速さで上昇して高さ99\sftanni{m}まで達し,あとは一定の加速度で減速しながら上昇して屋上に着いた。
        \begin{enumerate}
            \item 最初の6秒までのエレベーターの高さ$y$と速さ$v$を,$a$と出発からの時間$t$を用いてそれぞれ文字式で表せ。
            \item 加速度$a$はいくらか。
            \item 一定の速さで上昇した距離はいくらか。
            \item 減速のときの加速度はいくらか。上向き正として答えよ。
            \item エレベーターは地上から屋上まで昇るのに全部でどれだけ時間を要したか。
        \end{enumerate}

% \newpage
% \hakosyokika
\item 静水なら速さ$v$で進む船がある。この船が流速$\bunsuu{1}{2}v$の川を上り下りして$\ell $の距離を往復するのに要する時間$t_1$を求めよ。また,川の流れに垂直に横断して$\ell $の距離を往復するのに要する時間$t_2$を求めよ。\\
~~次に,川に沿い,上流に向かって速さ$v$で走る自動車がある。下流に向かって進む船との距離を$L$とすると,出会うまでの時間$t_3$を,相対速度を考えることにより求めよ。

% \newpage
% \hakosyokika
\item
    \begin{mawarikomi}{150pt}{\begin{zahyou*}[ul=6mm](-0.3,7)(-1,7.5)
    \small
    \def\Fx{-1*(X+2.5)*(X-2.5)}
    \def\Gx{-1*(X-2.5)*(X-6.5)}
    \def\A{(0,0)}
    \def\AL{(\xmin,0)}
    \def\B{(0,6.25)}
    \def\C{(2.5,0)}
    \def\D{(6.5,0)}
    \def\DR{(\xmax,0)}
    \def\E{(0,-0.5)}
    \def\EL{(\xmin,-0.5)}
    \def\F{(6.5,-0.5)}
    \def\FR{(\xmax,-0.5)}
    \def\vvec{(2,0)}
    \Nuritubusi{\AL\EL\FR\DR\AL}
    \Drawline{\AL\DR}
    \Put\A(0pt,-10pt)[t]{A}
    \Put\C(0pt,-10pt)[t]{C}
    \Put\D(0pt,-10pt)[t]{D}
    \YGurafu\Fx{0}{2.5}
    \YGurafu\Gx{2.5}{6.5}
    \HenKo<henkotype=parallel,
    henkoH=0ex,
    yazirusi=b
    % henkosideb=0,
    % henkosidet=1.2
    >\A\B{$h$}
    {\thicklines
    \Kuromaru{\B}
    \Put\B{\Yasen\vvec}}
    \Put\B[w]{B}
    \Put\B(35pt,0pt)[l]{$v_0$}

\end{zahyou*}
}
        なめらかな水平面の点Aの真上,高さ$h$の点Bから,小球を初速$v_0$で水平方向に投げ出した。小球は水平面の点Cではね返り,次に落下した点をDとする。ここで小球と水平面との反発係数(はね返り係数)を$e$とする。重力加速度の大きさを$g$とし,問(2),(3)では,水平成分は右向きを正,鉛直成分は上向きを正とする。
        \begin{enumerate}
            \item 点Bから点Cに落下するまでの時間$t_1$と,AC間の距離を求めよ。
            \item 点Cに落下する直前の,速度の水平成分と鉛直成分をそれぞれ求めよ。
            \item 点Cではね返った直後の,速度の水平成分と鉛直成分をそれぞれ求めよ。
            \item CD間での最高点の高さ$H$を求めよ。
            \item 点Cから点Dに達するまでの時間$t_2$と,CD間の距離を求めよ。
        \end{enumerate}
    \end{mawarikomi}
% \newpage
% \hakosyokika
\item
    \begin{mawarikomi}{150pt}{\begin{zahyou*}[ul=6mm](0,10)(-1,4)
    \small
    \def\A{(0.5,1.5)}
    \def\B{(0.5,0.2)}
    \def\C{(3,0.2)}
    \def\D{(3.0,1.5)}
    \def\L{(1,0.2)}
    \def\R{(2.5,0.2)}
    \def\P{(1.5,1.7)}
    \def\xvec{(1.5,0)}
    \def\yvec{(0,1.5)}
    \drawline(0,0)(10,0)
    \Drawline{\A\B\C\D\A}
    \En*[0]\L{0.2}
    \En\L{0.2}
    \En*[0]\R{0.2}
    \En\L{0.2}
    \En*\P{0.2}
    \En\P{0.2}
    \Kuromaru{\L;\R}
    {\thicklines
    \put(1.5,1.9){\Yasen\yvec}
    \put(3,0.8){\Yasen\xvec}
    }
    \put(1.5,-0.5){A}
    \put(8,-0.5){B}
    \put(4.7,0.8){$v$}
    \put(1.5,3.5){$u$}
\end{zahyou*}
}
        台車が一定の速度$v$で水平に運動している。台車がA点を通過する瞬間に,台車から台車に対して初速$u$で鉛直上向きにボールを打ち上げたら,ボールはB点に落下した。次に台車を$\bunsuu{1}{2}v$の速度で運動させたとき,台車がA点を通過する瞬間に台車に対して鉛直上向きにボールを打ち上げたら,ボールはやはりB点に落下した。重力加速度の大きさは$9.8$\sftanni{m/s^2}とする。
        \begin{Enumerate}
            \item 2度目にボールを打ち上げた鉛直方向の初速は最初の初速$u$の\Hako 倍である。
            \item このとき,ボールが到達した最高点の高さは最初の場合の\Hako 倍である。
        \end{Enumerate}
        ~~ところで,台車を$5.6$\sftanni{m/s}の速度で運動させて,台車がA点を通過する瞬間に台車から鉛直上向きにボールを打ち上げたら,ボールは$10$\sftanni{m}の高さまで上がって,やはりB点で台車に落下した。
        \begin{Enumerate*}
            \item このとき,ボールを打ち上げた鉛直方向の初速は\Hako \sftanni{m/s}である。
            \item そして,AB間の距離は\Hako \sftanni{m}である。
        \end{Enumerate*}
    \end{mawarikomi}
% \newpage
% \hakosyokika
\item
    \begin{mawarikomi}{120pt}{\begin{zahyou*}[ul=6mm,xscale=1.2,yscale=1.5](-6.8,-1)(-1,5)
    \small
    \def\Fx{-0.1*(X+6)*(X-6)}
    \def\Gx{-0.1*(X+4)**2+3.6}
    \def\vvec{(1.5,1.8)}
    \def\A{(-6,0)}
    \YTen\Fx{-2}\B
    \YTen\Gx{-5}\C
    \def\O{(-2,0)}
    \YTen\Gx{-4}\H
    \Candl\A{1}\A\vvec\P\Q
    \Kakukigou\O\A\Q<hankei=0.5>(2pt,2pt)[l]{$\theta$}
    \drawline(-6.8,0)(-2,0)(-2,4)
    \YGurafu(*)(0.03)\Fx{-6}{-2}
    \YGurafu(*)(0.03)\Gx{-5}{-2}
    {\thicklines
    \Put\A{\Yasen\vvec}}
    \Kuromaru{\H}
    \En*[1]\C{0.1}
    \En*[1]{(-6,0.05)}{0.1}
    \put(-5.2,1.8){$v_0$}
    \Put\A(-10pt,-7pt)[r]{床}
    \put(-2.6,3.9){壁}
    \Put\H[n]{H}
    \Put\B[e]{B}
    \Put\A[s]{A}
    \HenKo[40]\A\O{$\ell $}
    \Tyokkakukigou\A\O\B
    % \def\B{(0.5,0.2)}
    % \def\C{(3,0.2)}
    % \def\D{(3.0,1.5)}
    % \def\L{(1,0.2)}
    % \def\R{(2.5,0.2)}
    % \def\P{(1.5,1.7)}
    % \def\xvec{(1.5,0)}
    % \def\yvec{(0,1.5)}
    % \drawline(0,0)(10,0)
    % \Drawline{\A\B\C\D\A}
    % \En*[0]\L{0.2}
    % \En\L{0.2}
    % \En*[0]\R{0.2}
    % \En\L{0.2}
    % \En*\P{0.2}
    % \En\P{0.2}
    % \Kuromaru{\L;\R}
    % {\thicklines
    % \put(1.5,1.9){\Yasen\yvec}
    % \put(3,0.8){\Yasen\xvec}
    % }
    % \put(1.5,-0.5){A}
    % \put(8,-0.5){B}
    % \put(4.7,0.8){$v$}
    % \put(1.5,3.5){$u$}
\end{zahyou*}
}
        水平な床面上で鉛直な壁より$\ell $だけ離れた点Aから,壁に向かって初速$v_0$,角度$\theta $で投げた小球が,なめらかな壁に衝突してはね返り,最高点Hに達した後,再び床に落ちた。衝突の際の反発係数を$e$とし,重力加速度の大きさを$g$とする。
        \begin{enumerate}
            \item 小球が投げられてから壁に衝突するまでの時間$t_1$はいくらか。衝突した点Bの高さ$h$は,床からどれだけか。
            \item 小球が投げられてから最高点Hに達するまでの時間$t_2$はいくらか。また,点Hの高さ$H$は,床からどれだけか。
            \item 最高点に達する前に壁に衝突するために$v_0$が満たすべき条件は何か。
            \item はね返った小球が床上に落ちた点は,壁からどれだけ離れた距離にあるか。
        \end{enumerate}
    \end{mawarikomi}
% \newpage
% \item
    \begin{mawarikomi}{150pt}{\begin{zahyou*}[ul=3mm,yscale=2,xscale=2](0,12)(-3,3)
	\drawline(0,0)(0,0.2)(10,0.2)(10,0)(0,0)
	\def\O{(0,0)}
	\def\A{(4,0.2)}
	\def\AA{(4,2)}
	\def\B{(6,0.2)}
	\def\BB{(6,2)}
	\def\C{(10,0)}
	\def\CU{(10,0.2)}
	\def\BU{(10,2)}
	\def\CC{(10,-2)}
    \def\D{(10.5,-2)}
    \def\E{(10.5,-3)}
    \def\F{(9.5,-3)}
    \def\G{(9.5,-2)}
    \def\H{(10,-2.5)}
    \def\I{(5,-3)}
    \Drawlines{\D\E\F\G\D}
    \Drawlines{\C\CC}
    \Drawlines{\B\BB}
	% \scriptsize
	\HenKo<henkotype=parallel
			,henkoH=2ex
			,yazirusi=b
			,henkosideb=0
			,henkosidet=1.5>\CU\B{0.4\sftanni{m}}
	\Put\H(-.15pt,-2pt)[b]{A}
	\Put\I(-.15pt,-2pt)[b]{{\bf 図1}}
\end{zahyou*}
\\
\begin{zahyou*}[ul=3mm,yscale=2,xscale=2](0,12)(-3,3)
	\drawline(0,0)(0,0.2)(10,0.2)(10,0)(0,0)
	\def\O{(0,0)}
    \def\OD{(0,-2)}
	\def\OU{(0,0.2)}
	\def\A{(4,0.2)}
	\def\AA{(4,2)}
	\def\C{(10,0)}
	\def\CU{(10,0.2)}
	\def\CC{(10,-2)}
    \def\D{(10.5,-2)}
    \def\E{(10.5,-3)}
    \def\F{(9.5,-3)}
    \def\G{(9.5,-2)}
    \def\H{(10,-2.5)}
    \def\II{(5,-3)}
    \def\I{(-.5,-3)}
    \def\J{(-.5,-2)}
    \def\K{(.5,-2)}
    \def\L{(.5,-3)}
    \def\M{(-.5,-3)}
    \def\N{(0,-2.5)}
    \Drawlines{\D\E\F\G\D}
    \Drawlines{\C\CC}
    \Drawlines{\O\OD}
    \Drawlines{\A\AA}
    \Drawlines{\I\J\K\L\M\I}

	% \scriptsize
	\HenKo<henkotype=parallel
			,henkoH=2ex
			,yazirusi=b
			,henkosideb=0
			,henkosidet=1.5>\A\OU{0.4\sftanni{m}}
	\Put\H(-.15pt,-2pt)[b]{A}
	\Put\N(-.15pt,-2pt)[b]{B}
	\Put\AA[se]{糸S}
	\Put\II(-.15pt,-2pt)[b]{{\bf 図2}}
\end{zahyou*}
}
    密度と太さが一様な長さ1\sftanni{m}の棒の一端に質量2\sftanni{kg}のおもりAをつるしたところ,0.4\sftanni{m}の位置でつりあった({\bf 図1})。もう一端に別のおもりBをつるしたところ,この端から0.4\sftanni{m}のところでつり合った({\bf 図2})。おもりBの質量はいくらか。重力加速度の大きさを9.8\sftanni{m/s}とする。
    \end{mawarikomi}
% \newpage
% \item
    \begin{mawarikomi}{180pt}{\begin{zahyou*}[ul=4.5mm,yscale=1.5,xscale=1.5](-1,13)(-1,10)

	\def\C{(0,9.5)}
	\def\D{(0,10)}
	\def\E{(1,10)}
	\def\F{(1,9.5)}
	\def\fvec{(2,0)}
	\def\P{(0.5,9.5)}
	\def\PD{(0.5,6)}
	\def\A{(2.3,5)}
	\def\AD{(2.3,3)}
	\def\B{(8,2)}
	\def\M{(6,2.8)}
	\def\N{(9,1.5)}
	\Drawlines{\F\C}
	\Drawlines{\P\A}
		{
		\Thicklines
		\Drawlines{\B\A}
		}
	% \scriptsize
	\HenKo<henkoH=2ex>\B\A{$\ell $}
	\Put\A[ne]{A}
	\Put\B[ne]{B}
	\Put\P[sw]{P}
	{\thicklines
	\Put\B{\Yasen\fvec}
	}
	\Put\B[s]{$2m$}
	\Put\N(0pt,5pt)[t]{$F$}
	\Put\M[s]{$m$}
	\En*[1]\B{0.2}
	\Hasen{\P\PD}
	\Hasen{\A\AD}
	\Kakukigou\PD\P\A<hankei=2>[s]{$\alpha$}
	\Kakukigou\AD\A\B[se]{$\beta$}
	\Nuritubusi*<0.22>{\C\D\E\F\C}

\end{zahyou*}
}
    図のように,長さ$\ell $,質量$m$の一様な棒ABのB端に,質量$2m$の小球を取り付け,Aに軽い糸を結び点Pからつるす。小球に水平方向の力$F$を加えたところ,糸PAおよび棒ABと鉛直線のなす角度がそれぞれ$\alpha $および$\beta $となってつり合った。重力加速度の大きさを$g$とする。
        \begin{enumerate}
            \item 棒と小球全体の重心Gはどこになるか。Aからの距離を求めよ。
            \item 糸の張力を$T$として,水平方向および鉛直方向での力のつり合いの式をそれぞれ記せ。
            \item Aのまわりの力のモーメントのつり合いの式を記せ。
            \item $\tan{\alpha}$と$\tan{\beta }$および$T$を,それぞれ$m$,$g$,$F$を用いて表せ。
        \end{enumerate}
    \end{mawarikomi}
% \newpage
% \hakosyokika
\item
    \begin{mawarikomi}{150pt}{\begin{zahyou*}[ul=4.5mm](-1,13)(-1,10)

	\def\T{(2,10)}
	\def\TD{(2,0)}
	\def\TL{(1.8,10)}
	\def\TDL{(1.8,0)}
	\def\A{(2,2)}
	\def\AD{(2,1.8)}
	\def\B{(12,2)}
	\def\BD{(12,1.8)}
	\kandk\A{90}\B{150}\C
	\def\M{(12,0)}
	\Drawline{\B\A}
	\Drawline{\T\TD}
	\Drawline{\B\M}
	\Drawline{\AD\A\B\BD\AD}
	\Drawline{\A\B\C\A}
	% \scriptsize
	\HenKo[0]<henkoH=3ex>\B\C{糸}
	\Put\A[w]{A}
	\Put\B[e]{B}
	\Put\C[w]{C}
	\Put\M(0pt,-8pt)[t]{M}
	\En*[1]\M{0.3}
	\En\M{0.3}
	\Kakukigou\C\B\A<hankei=2>[w]{$30\Deg$}
	\Nuritubusi*<0.22>{\TD\T\TL\TDL\TD}

\end{zahyou*}
}
    長さが$\ell $で質量が$M$の一様な棒ABをA端を鉛直な粗い壁面に押し当て,B端を糸で結び,糸の他端をC点に固定する。
    B端に質量$M$のおもりMをつり下げた状態で,棒はA点で壁に垂直になっている。糸BCと棒ABのなす角度は$30\Deg$であり,重力加速度の大きさを$g$とする。\\
    ~~Aまわりの力のモーメントのつり合いより,糸の張力は\Hako である。また,A点での垂直抗力は\Hako であり,静止摩擦力は\Hako である。\\
    ~~Mをつり下げる位置をB点からAの方にゆっくりと移動していくと,MがB点から$x$離れたPの位置に来たとき棒のA端がすべり始めた。壁面と棒の間の静止摩擦係数を$\mu $壁面の垂直抗力を$N$とすると,棒がすべり出す直前では,棒のBまわりでの力のモーメントのつり合いから,糸の張力は$N$,$M$,$\ell $,$x$,$g$,$\mu $を用いて表すと,$\bunsuu{1}{2}Mg\ell + \Hako =0$となる。この式と水平方向での力のつり合いから,糸の張力は$M$,$\ell $,$x$,$g$,$\mu $を用いて\Hako と表される。そして,PB間の距離$x$は$\ell $,$\mu $を用いて表すと,\Hako である。
    \end{mawarikomi}
% \newpage
% \hakosyokika
\item
    \begin{mawarikomi}{120pt}{\begin{zahyou*}[ul=4.5mm](-1,13)(-1,13)
	\def\A{(11,0.5)}
	\def\AD{(11,0)}
	\def\BD{(-3,0)}
	\def\B{(-3,0.5)}
	\def\T{(0,12)}
	\def\TL{(-0.5,12)}
	\def\BSW{(-0.5,-0.5)}
	\def\RD{(12,-0.5)}
	\def\R{(12,0)}
	\def\O{(0,0)}
	\def\C{(4,1)}
	\def\D{(5,1)}
	\def\E{(5,0.5)}
	\def\F{(4,0.5)}
	\Kaiten\AD\A{-38.3}\AA
	\Kaiten\AD\BD{-38.3}\BDB
	\Kaiten\AD\B{-38.3}\BB
	\Kaiten\AD\C{-38.3}\CC
	\Kaiten\AD\D{-38.3}\DD
	\Kaiten\AD\E{-38.3}\EE
	\Kaiten\AD\F{-38.3}\FF
	% \Drawline{\BD\B\A\AD\BD}
	\Drawline{\AD\BDB\BB\AA\AD}
	\Drawline{\CC\DD\EE\FF\CC}
	\Drawline{\T\O\R}
	% \Drawline{\C\D\E\F\C}
	\Nuritubusi*<0.22>{\TL\T\O\R\RD\BSW\TL}
	% \Nuritubusi[1]{\C\D\E\F\C}
	\Nuritubusi[1]{\CC\DD\EE\FF\CC}
	\Put\BB[ne]{B}
	\Put\CC[ne]{P}
	\Put\AA[ne]{A}
	\Kakukigou\BDB\AD\O<hankei=2>[w]{$\theta $}

\end{zahyou*}
}
    粗い水平な床となめらかで鉛直な壁に,質量$M$,長さ$\ell $の一様な棒ABを,床から角$\theta $だけ傾けて立てかけた。そして棒の中点に質量$m$の小物体Pを置いたところ,棒の表面が粗いため,Pは棒の上で静止し,棒も静止したままであった。A点で棒が床から受ける摩擦力の大きさは\Hako である。ただし,重力加速度の大きさを$g$とする。\\
    ~~また,棒と床との静止摩擦係数を$\mu $とすると,棒が静止していることから$\mu \geqq$\Hako の条件が成り立っている。Pの位置を少しずつ変えていくと,A点からの距離が$x$の位置に置いたとき棒がすべらずに静止する限界となった。$x=$\Hako である。
    \end{mawarikomi}
% \newpage
% \item
\begin{mawarikomi}{150pt}{%WinTpicVersion4.32a
{\unitlength 0.1in%
\begin{picture}(24.0000,19.9000)(2.5000,-26.0000)%
% LINE 2 0 3 0 Black White  
% 6 1400 1400 2510 760 2510 760 2510 1400 2510 1400 1400 1400
% 
\special{pn 8}%
\special{pa 1400 1400}%
\special{pa 2510 760}%
\special{fp}%
\special{pa 2510 760}%
\special{pa 2510 1400}%
\special{fp}%
\special{pa 2510 1400}%
\special{pa 1400 1400}%
\special{fp}%
% CIRCLE 2 0 3 0 Black White  
% 4 1400 1400 1800 1400 2510 1400 2510 760
% 
\special{pn 8}%
\special{ar 1400 1400 400 400 5.7601670 6.2831853}%
% STR 2 0 3 0 Black White  
% 4 1830 1240 1830 1340 2 0 0 0
% 30\Deg
\put(18.3000,-13.4000){\makebox(0,0)[lb]{30\Deg}}%
% CIRCLE 3 0 3 0 Black White  
% 4 2230 1080 2650 1080 2510 1400 2510 770
% 
\special{pn 4}%
\special{ar 2230 1080 420 420 5.4469834 0.8519663}%
% STR 2 0 3 0 Black White  
% 4 2640 960 2640 1060 5 0 1 0
% $h$
\put(26.4000,-10.6000){\makebox(0,0){{\colorbox[named]{White}{$h$}}}}%
% STR 2 0 3 0 Black White  
% 4 2530 640 2530 740 2 0 0 0
% A
\put(25.3000,-7.4000){\makebox(0,0)[lb]{A}}%
% STR 2 0 3 0 Black White  
% 4 1380 1260 1380 1360 3 0 0 0
% B
\put(13.8000,-13.6000){\makebox(0,0)[rb]{B}}%
% VECTOR 2 0 3 0 Black White  
% 2 1400 1400 1020 1620
% 
\special{pn 8}%
\special{pa 1400 1400}%
\special{pa 1020 1620}%
\special{fp}%
\special{sh 1}%
\special{pa 1020 1620}%
\special{pa 1088 1604}%
\special{pa 1066 1593}%
\special{pa 1068 1569}%
\special{pa 1020 1620}%
\special{fp}%
% SPLINE 2 1 3 0 Black White  
% 3 1400 1400 1000 1750 430 2600
% 
\special{pn 8}%
\special{pn 8}%
\special{pa 1400 1400}%
\special{pa 1390 1408}%
\special{fp}%
\special{pa 1379 1417}%
\special{pa 1368 1425}%
\special{fp}%
\special{pa 1358 1434}%
\special{pa 1347 1442}%
\special{fp}%
\special{pa 1337 1451}%
\special{pa 1326 1459}%
\special{fp}%
\special{pa 1316 1468}%
\special{pa 1305 1477}%
\special{fp}%
\special{pa 1295 1485}%
\special{pa 1284 1494}%
\special{fp}%
\special{pa 1274 1502}%
\special{pa 1263 1511}%
\special{fp}%
\special{pa 1253 1520}%
\special{pa 1250 1522}%
\special{pa 1242 1528}%
\special{fp}%
\special{pa 1232 1537}%
\special{pa 1225 1542}%
\special{pa 1221 1545}%
\special{fp}%
\special{pa 1211 1554}%
\special{pa 1201 1563}%
\special{pa 1201 1563}%
\special{fp}%
\special{pa 1190 1572}%
\special{pa 1180 1580}%
\special{fp}%
\special{pa 1170 1589}%
\special{pa 1160 1598}%
\special{fp}%
\special{pa 1149 1607}%
\special{pa 1139 1616}%
\special{fp}%
\special{pa 1129 1625}%
\special{pa 1128 1626}%
\special{pa 1119 1634}%
\special{fp}%
\special{pa 1109 1643}%
\special{pa 1105 1647}%
\special{pa 1099 1653}%
\special{fp}%
\special{pa 1089 1662}%
\special{pa 1080 1671}%
\special{fp}%
\special{pa 1070 1681}%
\special{pa 1060 1690}%
\special{fp}%
\special{pa 1050 1700}%
\special{pa 1040 1709}%
\special{fp}%
\special{pa 1031 1718}%
\special{pa 1021 1728}%
\special{fp}%
\special{pa 1012 1738}%
\special{pa 1002 1747}%
\special{fp}%
\special{pa 993 1757}%
\special{pa 991 1759}%
\special{pa 984 1767}%
\special{fp}%
\special{pa 974 1777}%
\special{pa 970 1782}%
\special{pa 966 1787}%
\special{fp}%
\special{pa 956 1797}%
\special{pa 948 1808}%
\special{fp}%
\special{pa 939 1818}%
\special{pa 930 1828}%
\special{fp}%
\special{pa 921 1838}%
\special{pa 912 1848}%
\special{fp}%
\special{pa 903 1859}%
\special{pa 894 1869}%
\special{fp}%
\special{pa 886 1880}%
\special{pa 877 1890}%
\special{fp}%
\special{pa 869 1901}%
\special{pa 867 1903}%
\special{pa 861 1911}%
\special{fp}%
\special{pa 852 1922}%
\special{pa 848 1928}%
\special{pa 844 1933}%
\special{fp}%
\special{pa 836 1944}%
\special{pa 828 1954}%
\special{pa 828 1954}%
\special{fp}%
\special{pa 819 1965}%
\special{pa 811 1976}%
\special{fp}%
\special{pa 803 1987}%
\special{pa 796 1998}%
\special{fp}%
\special{pa 788 2009}%
\special{pa 780 2020}%
\special{fp}%
\special{pa 772 2031}%
\special{pa 764 2042}%
\special{fp}%
\special{pa 756 2054}%
\special{pa 754 2057}%
\special{pa 749 2065}%
\special{fp}%
\special{pa 741 2076}%
\special{pa 736 2084}%
\special{pa 734 2087}%
\special{fp}%
\special{pa 726 2098}%
\special{pa 718 2109}%
\special{fp}%
\special{pa 711 2121}%
\special{pa 704 2132}%
\special{fp}%
\special{pa 696 2144}%
\special{pa 689 2155}%
\special{fp}%
\special{pa 681 2166}%
\special{pa 674 2178}%
\special{fp}%
\special{pa 667 2189}%
\special{pa 666 2191}%
\special{pa 660 2201}%
\special{fp}%
\special{pa 653 2213}%
\special{pa 649 2219}%
\special{pa 646 2224}%
\special{fp}%
\special{pa 639 2236}%
\special{pa 632 2246}%
\special{pa 631 2247}%
\special{fp}%
\special{pa 625 2259}%
\special{pa 618 2270}%
\special{fp}%
\special{pa 611 2282}%
\special{pa 604 2293}%
\special{fp}%
\special{pa 597 2305}%
\special{pa 590 2317}%
\special{fp}%
\special{pa 583 2329}%
\special{pa 577 2340}%
\special{fp}%
\special{pa 570 2352}%
\special{pa 563 2364}%
\special{fp}%
\special{pa 556 2376}%
\special{pa 551 2385}%
\special{pa 550 2387}%
\special{fp}%
\special{pa 542 2399}%
\special{pa 535 2411}%
\special{fp}%
\special{pa 529 2422}%
\special{pa 522 2434}%
\special{fp}%
\special{pa 516 2446}%
\special{pa 509 2458}%
\special{fp}%
\special{pa 503 2470}%
\special{pa 496 2482}%
\special{fp}%
\special{pa 489 2494}%
\special{pa 483 2505}%
\special{fp}%
\special{pa 476 2517}%
\special{pa 471 2526}%
\special{pa 469 2529}%
\special{fp}%
\special{pa 463 2541}%
\special{pa 456 2553}%
\special{fp}%
\special{pa 450 2565}%
\special{pa 443 2576}%
\special{fp}%
\special{pa 436 2588}%
\special{pa 430 2600}%
\special{fp}%
% CIRCLE 2 0 3 0 Black White  
% 4 430 2600 630 2600 1030 2600 830 1910
% 
\special{pn 8}%
\special{ar 430 2600 200 200 5.2377559 6.2831853}%
% STR 2 0 3 0 Black White  
% 4 650 2390 650 2490 2 0 0 0
% 60\Deg
\put(6.5000,-24.9000){\makebox(0,0)[lb]{60\Deg}}%
% LINE 2 0 3 0 Black White  
% 2 250 2600 2250 2600
% 
\special{pn 8}%
\special{pa 250 2600}%
\special{pa 2250 2600}%
\special{fp}%
% STR 2 0 3 0 Black White  
% 4 2250 2460 2250 2560 2 0 0 0
% 水面
\put(22.5000,-25.6000){\makebox(0,0)[lb]{水面}}%
% VECTOR 2 0 3 0 Black White  
% 4 1600 2000 1600 1400 1600 2000 1600 2600
% 
\special{pn 8}%
\special{pa 1600 2000}%
\special{pa 1600 1400}%
\special{fp}%
\special{sh 1}%
\special{pa 1600 1400}%
\special{pa 1580 1467}%
\special{pa 1600 1453}%
\special{pa 1620 1467}%
\special{pa 1600 1400}%
\special{fp}%
\special{pa 1600 2000}%
\special{pa 1600 2600}%
\special{fp}%
\special{sh 1}%
\special{pa 1600 2600}%
\special{pa 1620 2533}%
\special{pa 1600 2547}%
\special{pa 1580 2533}%
\special{pa 1600 2600}%
\special{fp}%
% STR 2 0 3 0 Black White  
% 4 1600 1900 1600 2000 5 0 1 0
% $H$
\put(16.0000,-20.0000){\makebox(0,0){{\colorbox[named]{White}{$H$}}}}%
% LINE 2 0 3 0 Black White  
% 2 2460 1350 2510 1350
% 
\special{pn 8}%
\special{pa 2460 1350}%
\special{pa 2510 1350}%
\special{fp}%
% LINE 2 0 3 0 Black White  
% 2 2460 1350 2460 1400
% 
\special{pn 8}%
\special{pa 2460 1350}%
\special{pa 2460 1400}%
\special{fp}%
\end{picture}}%
}
    水平面に対して,30\Deg だけ傾いている高さ$h$の滑らかな斜面がある。
    その頂点Aから質量$m$の小物体を手放したところ,物体は斜面を滑り落ちてB点に達し,
    さらにその下の水面に60\Deg の角度で飛び込んだ。重力加速度を$g$とする。
    \begin{enumerate}
        \item 物体が斜面を滑り落ち,B点に達するまでの時間$t_1$と斜面から受ける垂直抗力$N$を求めよ。
        \item B点での物体の速さ$v$を求めよ。
        \item B点から水面に飛び込むまでの時間$t_2$を求め,$h$,$g$を用いて表せ。
        \item 水面からB点までの高さ$H$を$h$を用いて表せ。
    \end{enumerate}
\end{mawarikomi}

% \newpage
% \item 平板上に置かれた質量$m$の物体がある。平板と物体との間の動摩擦係数を$\mu $,重力加速度の大きさを$g$とする。
\begin{enumerate}
    \item 平板を水平にして,物体を初速$v_0$で滑らせた。止まるまでに滑る距離$\ell $を求めよ。また,止まるまでの時間$t$を求めよ。
    \item 平板を水平から45\Deg 傾け,物体を斜面に沿って上方に,(1)と同じ初速$v_0$で滑らせたら,$\bunsuu{1}{2}\ell $の距離を滑って点Aで止まった。動摩擦係数$\mu $の値を求めよ。
    \item (2)で物体は点Aで完全に静止した。平板と物体との間の静止摩擦整数$\mu _0$の値はいくら以上か。
\end{enumerate}

% \newpage
% \item
\begin{mawarikomi}{150pt}{%WinTpicVersion4.32a
{\unitlength 0.1in%
\begin{picture}(26.3000,20.4500)(2.0000,-27.3500)%
% LINE 2 0 3 0 Black White  
% 4 600 2600 2600 2600 2600 2600 2600 1600
% 
\special{pn 8}%
\special{pa 600 2600}%
\special{pa 2600 2600}%
\special{fp}%
\special{pa 2600 2600}%
\special{pa 2600 1600}%
\special{fp}%
% LINE 2 0 3 0 Black White  
% 2 2600 1600 600 2600
% 
\special{pn 8}%
\special{pa 2600 1600}%
\special{pa 600 2600}%
\special{fp}%
% POLYGON 2 0 1 0 Black Black  
% 5 1942 1786 2000 1903 2179 1813 2121 1697 1942 1786
% 
\special{pn 0}%
\special{sh 0.300}%
\special{pa 1942 1786}%
\special{pa 2000 1903}%
\special{pa 2179 1813}%
\special{pa 2121 1697}%
\special{pa 1942 1786}%
\special{ip}%
\special{pn 8}%
\special{pa 1942 1786}%
\special{pa 2000 1903}%
\special{pa 2179 1813}%
\special{pa 2121 1697}%
\special{pa 1942 1786}%
\special{pa 2000 1903}%
\special{fp}%
% CIRCLE 2 0 3 0 Black Black  
% 4 605 2600 1005 2600 2605 2600 2605 1600
% 
\special{pn 8}%
\special{ar 605 2600 400 400 5.8195377 6.2831853}%
% STR 2 0 3 0 Black Black  
% 4 1055 2430 1055 2530 2 0 0 0
% $\theta $
\put(10.5500,-25.3000){\makebox(0,0)[lb]{$\theta $}}%
% LINE 2 0 3 0 Black Black  
% 2 2125 1690 2625 690
% 
\special{pn 8}%
\special{pa 2125 1690}%
\special{pa 2625 690}%
\special{fp}%
% LINE 2 1 3 0 Black Black  
% 2 2625 690 2625 1410
% 
\special{pn 8}%
\special{pa 2625 690}%
\special{pa 2625 1410}%
\special{da 0.015}%
% CIRCLE 2 0 3 0 Black Black  
% 4 2625 700 2895 700 2115 1710 2625 1710
% 
\special{pn 8}%
\special{ar 2625 700 270 270 1.5707963 2.0383965}%
% STR 2 0 3 0 Black Black  
% 4 2525 1000 2525 1100 5 0 0 0
% $\theta $
\put(25.2500,-11.0000){\makebox(0,0){$\theta $}}%
% STR 2 0 3 0 Black Black  
% 4 2020 1600 2020 1700 3 0 0 0
% P
\put(20.2000,-17.0000){\makebox(0,0)[rb]{P}}%
% STR 2 0 3 0 Black Black  
% 4 2310 2300 2310 2400 5 0 0 0
% $M$
\put(23.1000,-24.0000){\makebox(0,0){$M$}}%
% LINE 2 1 3 0 Black Black  
% 2 2000 1900 310 1900
% 
\special{pn 8}%
\special{pa 2000 1900}%
\special{pa 310 1900}%
\special{da 0.015}%
% LINE 2 0 3 0 Black Black  
% 2 200 2600 2800 2600
% 
\special{pn 8}%
\special{pa 200 2600}%
\special{pa 2800 2600}%
\special{fp}%
% LINE 3 0 3 0 Black Black  
% 88 340 2600 240 2700 280 2600 200 2680 400 2600 300 2700 460 2600 360 2700 520 2600 420 2700 580 2600 480 2700 640 2600 540 2700 700 2600 600 2700 760 2600 660 2700 820 2600 720 2700 880 2600 780 2700 940 2600 840 2700 1000 2600 900 2700 1060 2600 960 2700 1120 2600 1020 2700 1180 2600 1080 2700 1240 2600 1140 2700 1300 2600 1200 2700 1360 2600 1260 2700 1420 2600 1320 2700 1480 2600 1380 2700 1540 2600 1440 2700 1600 2600 1500 2700 1660 2600 1560 2700 1720 2600 1620 2700 1780 2600 1680 2700 1840 2600 1740 2700 1900 2600 1800 2700 1960 2600 1860 2700 2020 2600 1920 2700 2080 2600 1980 2700 2140 2600 2040 2700 2200 2600 2100 2700 2260 2600 2160 2700 2320 2600 2220 2700 2380 2600 2280 2700 2440 2600 2340 2700 2500 2600 2400 2700 2560 2600 2460 2700 2620 2600 2520 2700 2680 2600 2580 2700 2740 2600 2640 2700 2790 2610 2700 2700 2800 2660 2760 2700
% 
\special{pn 4}%
\special{pa 340 2600}%
\special{pa 240 2700}%
\special{fp}%
\special{pa 280 2600}%
\special{pa 200 2680}%
\special{fp}%
\special{pa 400 2600}%
\special{pa 300 2700}%
\special{fp}%
\special{pa 460 2600}%
\special{pa 360 2700}%
\special{fp}%
\special{pa 520 2600}%
\special{pa 420 2700}%
\special{fp}%
\special{pa 580 2600}%
\special{pa 480 2700}%
\special{fp}%
\special{pa 640 2600}%
\special{pa 540 2700}%
\special{fp}%
\special{pa 700 2600}%
\special{pa 600 2700}%
\special{fp}%
\special{pa 760 2600}%
\special{pa 660 2700}%
\special{fp}%
\special{pa 820 2600}%
\special{pa 720 2700}%
\special{fp}%
\special{pa 880 2600}%
\special{pa 780 2700}%
\special{fp}%
\special{pa 940 2600}%
\special{pa 840 2700}%
\special{fp}%
\special{pa 1000 2600}%
\special{pa 900 2700}%
\special{fp}%
\special{pa 1060 2600}%
\special{pa 960 2700}%
\special{fp}%
\special{pa 1120 2600}%
\special{pa 1020 2700}%
\special{fp}%
\special{pa 1180 2600}%
\special{pa 1080 2700}%
\special{fp}%
\special{pa 1240 2600}%
\special{pa 1140 2700}%
\special{fp}%
\special{pa 1300 2600}%
\special{pa 1200 2700}%
\special{fp}%
\special{pa 1360 2600}%
\special{pa 1260 2700}%
\special{fp}%
\special{pa 1420 2600}%
\special{pa 1320 2700}%
\special{fp}%
\special{pa 1480 2600}%
\special{pa 1380 2700}%
\special{fp}%
\special{pa 1540 2600}%
\special{pa 1440 2700}%
\special{fp}%
\special{pa 1600 2600}%
\special{pa 1500 2700}%
\special{fp}%
\special{pa 1660 2600}%
\special{pa 1560 2700}%
\special{fp}%
\special{pa 1720 2600}%
\special{pa 1620 2700}%
\special{fp}%
\special{pa 1780 2600}%
\special{pa 1680 2700}%
\special{fp}%
\special{pa 1840 2600}%
\special{pa 1740 2700}%
\special{fp}%
\special{pa 1900 2600}%
\special{pa 1800 2700}%
\special{fp}%
\special{pa 1960 2600}%
\special{pa 1860 2700}%
\special{fp}%
\special{pa 2020 2600}%
\special{pa 1920 2700}%
\special{fp}%
\special{pa 2080 2600}%
\special{pa 1980 2700}%
\special{fp}%
\special{pa 2140 2600}%
\special{pa 2040 2700}%
\special{fp}%
\special{pa 2200 2600}%
\special{pa 2100 2700}%
\special{fp}%
\special{pa 2260 2600}%
\special{pa 2160 2700}%
\special{fp}%
\special{pa 2320 2600}%
\special{pa 2220 2700}%
\special{fp}%
\special{pa 2380 2600}%
\special{pa 2280 2700}%
\special{fp}%
\special{pa 2440 2600}%
\special{pa 2340 2700}%
\special{fp}%
\special{pa 2500 2600}%
\special{pa 2400 2700}%
\special{fp}%
\special{pa 2560 2600}%
\special{pa 2460 2700}%
\special{fp}%
\special{pa 2620 2600}%
\special{pa 2520 2700}%
\special{fp}%
\special{pa 2680 2600}%
\special{pa 2580 2700}%
\special{fp}%
\special{pa 2740 2600}%
\special{pa 2640 2700}%
\special{fp}%
\special{pa 2790 2610}%
\special{pa 2700 2700}%
\special{fp}%
\special{pa 2800 2660}%
\special{pa 2760 2700}%
\special{fp}%
% POLYGON 2 1 3 0 Black Black  
% 5 530 2490 588 2607 767 2517 709 2401 530 2490
% 
\special{pn 8}%
\special{pn 8}%
\special{pa 530 2490}%
\special{pa 536 2502}%
\special{fp}%
\special{pa 542 2514}%
\special{pa 548 2526}%
\special{fp}%
\special{pa 554 2538}%
\special{pa 560 2550}%
\special{fp}%
\special{pa 566 2562}%
\special{pa 572 2574}%
\special{fp}%
\special{pa 578 2587}%
\special{pa 584 2599}%
\special{fp}%
\special{pa 592 2605}%
\special{pa 604 2599}%
\special{fp}%
\special{pa 616 2593}%
\special{pa 628 2587}%
\special{fp}%
\special{pa 640 2581}%
\special{pa 652 2575}%
\special{fp}%
\special{pa 664 2569}%
\special{pa 676 2563}%
\special{fp}%
\special{pa 688 2557}%
\special{pa 700 2551}%
\special{fp}%
\special{pa 712 2545}%
\special{pa 724 2539}%
\special{fp}%
\special{pa 736 2532}%
\special{pa 748 2526}%
\special{fp}%
\special{pa 760 2520}%
\special{pa 767 2517}%
\special{pa 764 2512}%
\special{fp}%
\special{pa 758 2499}%
\special{pa 752 2488}%
\special{fp}%
\special{pa 746 2475}%
\special{pa 740 2463}%
\special{fp}%
\special{pa 734 2451}%
\special{pa 728 2439}%
\special{fp}%
\special{pa 722 2427}%
\special{pa 716 2415}%
\special{fp}%
\special{pa 710 2403}%
\special{pa 709 2401}%
\special{pa 699 2406}%
\special{fp}%
\special{pa 687 2412}%
\special{pa 675 2418}%
\special{fp}%
\special{pa 663 2424}%
\special{pa 651 2430}%
\special{fp}%
\special{pa 639 2436}%
\special{pa 627 2442}%
\special{fp}%
\special{pa 614 2448}%
\special{pa 602 2454}%
\special{fp}%
\special{pa 590 2460}%
\special{pa 578 2466}%
\special{fp}%
\special{pa 566 2472}%
\special{pa 554 2478}%
\special{fp}%
\special{pa 542 2484}%
\special{pa 530 2490}%
\special{fp}%
% VECTOR 2 0 3 0 Black Black  
% 2 410 2270 410 1900
% 
\special{pn 8}%
\special{pa 410 2270}%
\special{pa 410 1900}%
\special{fp}%
\special{sh 1}%
\special{pa 410 1900}%
\special{pa 390 1967}%
\special{pa 410 1953}%
\special{pa 430 1967}%
\special{pa 410 1900}%
\special{fp}%
% VECTOR 2 0 3 0 Black Black  
% 2 410 2300 410 2600
% 
\special{pn 8}%
\special{pa 410 2300}%
\special{pa 410 2600}%
\special{fp}%
\special{sh 1}%
\special{pa 410 2600}%
\special{pa 430 2533}%
\special{pa 410 2547}%
\special{pa 390 2533}%
\special{pa 410 2600}%
\special{fp}%
% STR 2 0 3 0 Black Black  
% 4 410 2100 410 2200 5 0 1 0
% $h$
\put(4.1000,-22.0000){\makebox(0,0){{\colorbox[named]{White}{$h$}}}}%
% STR 2 0 3 0 Black Black  
% 4 1830 1720 1830 1820 5 0 0 0
% $m$
\put(18.3000,-18.2000){\makebox(0,0){$m$}}%
% STR 2 0 3 0 Black Black  
% 4 600 2700 600 2800 5 0 0 0
% B
\put(6.0000,-28.0000){\makebox(0,0){B}}%
% STR 2 0 3 0 Black Black  
% 4 2030 1900 2030 2000 5 0 0 0
% A
\put(20.3000,-20.0000){\makebox(0,0){A}}%
% LINE 2 0 3 0 Black Black  
% 2 2230 690 2830 690
% 
\special{pn 8}%
\special{pa 2230 690}%
\special{pa 2830 690}%
\special{fp}%
\end{picture}}%
}
    水平な粗い床の上に,なめらかな斜面をもつ質量$M$の台が置かれている。斜面の角度は$\theta $である。
    質量$m$の小物体Pが,天井に固定された糸で斜め上方に引っ張られ,斜面上の点Aで静止していて,糸が鉛直方向となす角度も$\theta $である。
    Pの床からの高さを$h$とし,重力加速度を$g$とする。
    \begin{Enumerate}
        \item 糸の張力$T$,およびPが斜面から受ける垂直抗力$N_1$をそれぞれ求めよ。ただし,$\sin{2\theta }=2\sin{\theta }\cos{\theta }$,$\cos{2\theta}=\cos^2{\theta}-\sin^2{\theta}$とする。
        \item 台が床から受ける静止摩擦力$F$と垂直抗力$R$をそれぞれ求めよ。
        \item 台と床の静止摩擦係数$\mu $はいくら以上か。
    \end{Enumerate}
    糸を切るとPは斜面に沿って滑り出した。一方,台は静止していた。
    \begin{Enumerate*}
        \item Pの加速度$a$,およびPが斜面から受ける垂直抗力$N_2$を求めよ。
        \item 糸を切ってから,Pが斜面を滑り,下端Bに達するまでに要する時間$t$を求めよ。また,Bに達したときの速さ$v$を求めよ。
        \item このようにPが斜面上を滑っている間,台が静止しているためには,台と床との間の静止摩擦係数$\mu $はいくら以上であればよいか。
    \end{Enumerate*}
\end{mawarikomi}

% \newpage
% \item
        \begin{mawarikomi}{80pt}{%WinTpicVersion4.32a
{\unitlength 0.1in%
\begin{picture}(15.3500,25.9000)(2.7500,-29.0000)%
% LINE 2 0 3 0 Black White  
% 2 610 400 1410 400
% 
\special{pn 8}%
\special{pa 610 400}%
\special{pa 1410 400}%
\special{fp}%
% LINE 3 0 3 0 Black White  
% 28 1310 310 1220 400 1250 310 1160 400 1190 310 1100 400 1130 310 1040 400 1070 310 980 400 1010 310 920 400 950 310 860 400 890 310 800 400 830 310 740 400 770 310 680 400 710 310 620 400 650 310 610 350 1370 310 1280 400 1410 330 1340 400
% 
\special{pn 4}%
\special{pa 1310 310}%
\special{pa 1220 400}%
\special{fp}%
\special{pa 1250 310}%
\special{pa 1160 400}%
\special{fp}%
\special{pa 1190 310}%
\special{pa 1100 400}%
\special{fp}%
\special{pa 1130 310}%
\special{pa 1040 400}%
\special{fp}%
\special{pa 1070 310}%
\special{pa 980 400}%
\special{fp}%
\special{pa 1010 310}%
\special{pa 920 400}%
\special{fp}%
\special{pa 950 310}%
\special{pa 860 400}%
\special{fp}%
\special{pa 890 310}%
\special{pa 800 400}%
\special{fp}%
\special{pa 830 310}%
\special{pa 740 400}%
\special{fp}%
\special{pa 770 310}%
\special{pa 680 400}%
\special{fp}%
\special{pa 710 310}%
\special{pa 620 400}%
\special{fp}%
\special{pa 650 310}%
\special{pa 610 350}%
\special{fp}%
\special{pa 1370 310}%
\special{pa 1280 400}%
\special{fp}%
\special{pa 1410 330}%
\special{pa 1340 400}%
\special{fp}%
% LINE 2 0 3 0 Black White  
% 2 1010 400 1010 1000
% 
\special{pn 8}%
\special{pa 1010 400}%
\special{pa 1010 1000}%
\special{fp}%
% DOT 0 0 3 0 Black White  
% 1 1010 1000
% 
\special{pn 4}%
\special{sh 1}%
\special{ar 1010 1000 16 16 0 6.2831853}%
% CIRCLE 2 0 3 0 Black White  
% 4 1010 1000 1410 1000 1410 1000 1410 1000
% 
\special{pn 8}%
\special{ar 1010 1000 400 400 0.0000000 6.2831853}%
% LINE 2 0 3 0 Black White  
% 4 610 1000 610 2000 710 2100 710 2100
% 
\special{pn 8}%
\special{pa 610 1000}%
\special{pa 610 2000}%
\special{fp}%
\special{pa 710 2100}%
\special{pa 710 2100}%
\special{fp}%
% CIRCLE 2 0 1 0 Black Black  
% 4 610 2100 710 2100 710 2100 710 2100
% 
\special{sh 0.300}%
\special{ia 610 2100 100 100 0.0000000 6.2831853}%
\special{pn 8}%
\special{ar 610 2100 100 100 0.0000000 6.2831853}%
% LINE 2 0 3 0 Black White  
% 2 610 2200 610 2800
% 
\special{pn 8}%
\special{pa 610 2200}%
\special{pa 610 2800}%
\special{fp}%
% LINE 2 0 3 0 Black White  
% 2 410 2800 810 2800
% 
\special{pn 8}%
\special{pa 410 2800}%
\special{pa 810 2800}%
\special{fp}%
% LINE 3 0 3 0 Black White  
% 16 500 2800 410 2890 560 2800 460 2900 620 2800 520 2900 680 2800 580 2900 740 2800 640 2900 800 2800 700 2900 810 2850 760 2900 440 2800 410 2830
% 
\special{pn 4}%
\special{pa 500 2800}%
\special{pa 410 2890}%
\special{fp}%
\special{pa 560 2800}%
\special{pa 460 2900}%
\special{fp}%
\special{pa 620 2800}%
\special{pa 520 2900}%
\special{fp}%
\special{pa 680 2800}%
\special{pa 580 2900}%
\special{fp}%
\special{pa 740 2800}%
\special{pa 640 2900}%
\special{fp}%
\special{pa 800 2800}%
\special{pa 700 2900}%
\special{fp}%
\special{pa 810 2850}%
\special{pa 760 2900}%
\special{fp}%
\special{pa 440 2800}%
\special{pa 410 2830}%
\special{fp}%
% LINE 2 0 3 0 Black White  
% 14 1410 1000 1410 1800 1410 1800 1010 2400 1010 2400 1810 2400 1810 2400 1410 1800 1010 2400 1010 2500 1010 2500 1810 2500 1810 2500 1810 2400
% 
\special{pn 8}%
\special{pa 1410 1000}%
\special{pa 1410 1800}%
\special{fp}%
\special{pa 1410 1800}%
\special{pa 1010 2400}%
\special{fp}%
\special{pa 1010 2400}%
\special{pa 1810 2400}%
\special{fp}%
\special{pa 1810 2400}%
\special{pa 1410 1800}%
\special{fp}%
\special{pa 1010 2400}%
\special{pa 1010 2500}%
\special{fp}%
\special{pa 1010 2500}%
\special{pa 1810 2500}%
\special{fp}%
\special{pa 1810 2500}%
\special{pa 1810 2400}%
\special{fp}%
% BOX 2 0 3 0 Black White  
% 2 1310 2400 1510 2100
% 
\special{pn 8}%
\special{pa 1310 2400}%
\special{pa 1510 2400}%
\special{pa 1510 2100}%
\special{pa 1310 2100}%
\special{pa 1310 2400}%
\special{pa 1510 2400}%
\special{fp}%
% LINE 3 0 3 0 Black White  
% 14 1510 2270 1380 2400 1510 2210 1320 2400 1510 2150 1310 2350 1500 2100 1310 2290 1440 2100 1310 2230 1380 2100 1310 2170 1510 2330 1440 2400
% 
\special{pn 4}%
\special{pa 1510 2270}%
\special{pa 1380 2400}%
\special{fp}%
\special{pa 1510 2210}%
\special{pa 1320 2400}%
\special{fp}%
\special{pa 1510 2150}%
\special{pa 1310 2350}%
\special{fp}%
\special{pa 1500 2100}%
\special{pa 1310 2290}%
\special{fp}%
\special{pa 1440 2100}%
\special{pa 1310 2230}%
\special{fp}%
\special{pa 1380 2100}%
\special{pa 1310 2170}%
\special{fp}%
\special{pa 1510 2330}%
\special{pa 1440 2400}%
\special{fp}%
% STR 2 0 3 0 Black White  
% 4 1410 1900 1410 2000 5 0 0 0
% B
\put(14.1000,-20.0000){\makebox(0,0){B}}%
% STR 2 0 3 0 Black White  
% 4 1610 2190 1610 2290 5 0 0 0
% $M$
\put(16.1000,-22.9000){\makebox(0,0){$M$}}%
% STR 2 0 3 0 Black White  
% 4 1710 2460 1710 2560 5 0 0 0
% $m$
\put(17.1000,-25.6000){\makebox(0,0){$m$}}%
% STR 2 0 3 0 Black White  
% 4 1620 1860 1620 1960 5 0 0 0
% 糸
\put(16.2000,-19.6000){\makebox(0,0){糸}}%
% STR 2 0 3 0 Black White  
% 4 1230 2490 1230 2590 5 0 0 0
% 板
\put(12.3000,-25.9000){\makebox(0,0){板}}%
% STR 2 0 3 0 Black White  
% 4 1520 1330 1520 1430 5 0 0 0
% $\alpha $
\put(15.2000,-14.3000){\makebox(0,0){$\alpha $}}%
% STR 2 0 3 0 Black White  
% 4 510 1330 510 1430 5 0 0 0
% $\alpha $
\put(5.1000,-14.3000){\makebox(0,0){$\alpha $}}%
% STR 2 0 3 0 Black White  
% 4 510 2330 510 2430 5 0 0 0
% $\beta $
\put(5.1000,-24.3000){\makebox(0,0){$\beta $}}%
% STR 2 0 3 0 Black White  
% 4 420 1840 420 1940 5 0 0 0
% A
\put(4.2000,-19.4000){\makebox(0,0){A}}%
% STR 2 0 3 0 Black White  
% 4 420 2130 420 2230 5 0 0 0
% $m$
\put(4.2000,-22.3000){\makebox(0,0){$m$}}%
% STR 2 0 3 0 Black White  
% 4 910 2770 910 2870 5 0 0 0
% 床
\put(9.1000,-28.7000){\makebox(0,0){床}}%
% STR 2 0 3 0 Black White  
% 4 1090 420 1090 520 5 0 0 0
% $\gamma $
\put(10.9000,-5.2000){\makebox(0,0){$\gamma $}}%
\end{picture}}%
}
            天井から糸$\gamma $でつるされた定滑車に糸$\alpha$をかけ,
            左には質量$m$の物体Aを,右には質量$m$の板をつるす。Aと床の間を糸$\beta $で結び,板上には質量$m$の物体Bを置く。滑車は滑らかで,質量は無視でき,重力加速度の大きさを$g$とする。
            \begin{enumerate}
                \item 糸$\alpha$,$\beta $,$\gamma $の張力はそれぞれいくらか。
                \item 糸$\beta $を切ると,全体が動き出した。
                    \begin{enumerate}[(ア)]
                        \item Aの加速度はいくらか。また,Aが距離$h$だけ上がるのにかかる時間はいくらか。
                        \item 糸$\gamma$の張力はいくらか。
                        \item Bが板を押している力はいくらか。
                    \end{enumerate}
            \end{enumerate}     
        \end{mawarikomi}

% \newpage
% \item
\begin{mawarikomi}{110pt}{%WinTpicVersion4.32a
{\unitlength 0.1in%
\begin{picture}(19.7000,9.2000)(4.0000,-12.0000)%
% LINE 2 0 3 0 Black White  
% 6 400 1200 1770 410 1770 410 1770 1200 1770 1200 400 1200
% 
\special{pn 8}%
\special{pa 400 1200}%
\special{pa 1770 410}%
\special{fp}%
\special{pa 1770 410}%
\special{pa 1770 1200}%
\special{fp}%
\special{pa 1770 1200}%
\special{pa 400 1200}%
\special{fp}%
% POLYGON 2 0 3 0 Black White  
% 5 525 950 600 1080 773 980 698 850 525 950
% 
\special{pn 8}%
\special{pa 525 950}%
\special{pa 600 1080}%
\special{pa 773 980}%
\special{pa 698 850}%
\special{pa 525 950}%
\special{pa 600 1080}%
\special{fp}%
% CIRCLE 2 0 3 0 Black White  
% 4 405 1200 705 1200 1775 1200 1775 410
% 
\special{pn 8}%
\special{ar 405 1200 300 300 5.7601176 6.2831853}%
% STR 2 0 3 0 Black White  
% 4 765 1050 765 1150 2 0 0 0
% 30\Deg
\put(7.6500,-11.5000){\makebox(0,0)[lb]{30\Deg}}%
% LINE 2 0 3 0 Black White  
% 4 735 920 1825 290 1775 410 1865 340
% 
\special{pn 8}%
\special{pa 735 920}%
\special{pa 1825 290}%
\special{fp}%
\special{pa 1775 410}%
\special{pa 1865 340}%
\special{fp}%
% DOT 2 0 3 0 Black White  
% 1 1865 340
% 
\special{pn 4}%
\special{sh 1}%
\special{ar 1865 340 8 8 0 6.2831853}%
% CIRCLE 2 0 3 0 Black White  
% 4 1865 340 1925 340 1925 340 1925 340
% 
\special{pn 8}%
\special{ar 1865 340 60 60 0.0000000 6.2831853}%
% LINE 2 0 3 0 Black White  
% 2 1925 340 1925 740
% 
\special{pn 8}%
\special{pa 1925 340}%
\special{pa 1925 740}%
\special{fp}%
% CIRCLE 2 0 3 0 Black White  
% 4 1925 800 1985 800 1985 800 1985 800
% 
\special{pn 8}%
\special{ar 1925 800 60 60 0.0000000 6.2831853}%
% STR 2 0 3 0 Black White  
% 4 565 700 565 800 5 0 0 0
% $m$
\put(5.6500,-8.0000){\makebox(0,0){$m$}}%
% STR 2 0 3 0 Black White  
% 4 400 900 400 1000 2 0 0 0
% P
\put(4.0000,-10.0000){\makebox(0,0)[lb]{P}}%
% STR 2 0 3 0 Black White  
% 4 2005 620 2005 720 2 0 0 0
% Q
\put(20.0500,-7.2000){\makebox(0,0)[lb]{Q}}%
% STR 2 0 3 0 Black White  
% 4 1925 830 1925 930 5 0 0 0
% $M$
\put(19.2500,-9.3000){\makebox(0,0){$M$}}%
% VECTOR 2 0 3 0 Black White  
% 6 2125 930 2125 800 2125 1000 2125 1000 2125 1000 2125 1200
% 
\special{pn 8}%
\special{pa 2125 930}%
\special{pa 2125 800}%
\special{fp}%
\special{sh 1}%
\special{pa 2125 800}%
\special{pa 2105 867}%
\special{pa 2125 853}%
\special{pa 2145 867}%
\special{pa 2125 800}%
\special{fp}%
\special{pa 2125 1000}%
\special{pa 2125 1000}%
\special{fp}%
\special{pa 2125 1000}%
\special{pa 2125 1200}%
\special{fp}%
\special{sh 1}%
\special{pa 2125 1200}%
\special{pa 2145 1133}%
\special{pa 2125 1147}%
\special{pa 2105 1133}%
\special{pa 2125 1200}%
\special{fp}%
% STR 2 0 3 0 Black White  
% 4 2120 890 2120 990 5 0 1 0
% $h$
\put(21.2000,-9.9000){\makebox(0,0){{\colorbox[named]{White}{$h$}}}}%
% LINE 2 0 3 0 Black White  
% 2 1770 1200 2370 1200
% 
\special{pn 8}%
\special{pa 1770 1200}%
\special{pa 2370 1200}%
\special{fp}%
\end{picture}}%
}
    水平な床から30\Deg 傾いた斜面上に,質量$m$の物体Pがあり,質量$M$の小物体Qと滑らかな
    滑車を介して糸で結ばれている。Pと斜面の間の静止摩擦係数を$\bunsuu{1}{3}$,動摩擦係数を$\bunsuu{1}{2\sqrt{3}}$とし,重力加速度を$g$とする。
    \begin{enumerate}
        \item PとQが静止しているための$M$の範囲を$m$を用いて表せ。
        \item 床からの高さを$h$とし,$M=\bunsuu{3}{2}m$として静かに放すと,Qが下がり始めた。
        Pが滑車に衝突することはないものとする。
            \begin{enumerate}[(ア)]
                \item Qの加速度の大きさ$a$と,Qが床に達するときの速さ$v$を求めよ。
                \item Qが床に達した後,Pがやがて斜面上で最高点に達して止まった。Pが動き始めてから止まるまでに移動した距離$\ell $とかかった時間$t$を求めよ。
            \end{enumerate}
    \end{enumerate}
\end{mawarikomi}

% \newpage
% \item
\begin{mawarikomi}{50pt}{%WinTpicVersion4.32a
{\unitlength 0.1in%
\begin{picture}(10.5700,14.0000)(2.7000,-18.0000)%
% CIRCLE 2 0 3 0 Black White  
% 4 800 800 1200 800 1200 800 1200 800
% 
\special{pn 8}%
\special{ar 800 800 400 400 0.0000000 6.2831853}%
% LINE 2 0 3 0 Black White  
% 2 800 1200 800 1600
% 
\special{pn 8}%
\special{pa 800 1200}%
\special{pa 800 1600}%
\special{fp}%
% CIRCLE 2 0 1 0 Black Black  
% 4 800 1700 900 1700 900 1700 900 1700
% 
\special{sh 0.300}%
\special{ia 800 1700 100 100 0.0000000 6.2831853}%
\special{pn 8}%
\special{ar 800 1700 100 100 0.0000000 6.2831853}%
% STR 2 0 3 0 Black White  
% 4 600 1600 600 1700 5 0 0 0
% A
\put(6.0000,-17.0000){\makebox(0,0){A}}%
% STR 2 0 3 0 Black White  
% 4 300 700 300 800 5 0 0 0
% B
\put(3.0000,-8.0000){\makebox(0,0){B}}%
% VECTOR 2 0 3 0 Black White  
% 2 1300 1000 1300 600
% 
\special{pn 8}%
\special{pa 1300 1000}%
\special{pa 1300 600}%
\special{fp}%
\special{sh 1}%
\special{pa 1300 600}%
\special{pa 1280 667}%
\special{pa 1300 653}%
\special{pa 1320 667}%
\special{pa 1300 600}%
\special{fp}%
% STR 2 0 3 0 Black White  
% 4 1300 430 1300 530 5 0 0 0
% $v$
\put(13.0000,-5.3000){\makebox(0,0){$v$}}%
\end{picture}}%
}
質量$M$の気球B(内部の気体も含む)が,質量$m$の小物体Aを質量の無視できる糸でつるして,一定の速さ$v$で上昇している。重力加速度を$g$とし,空気の抵抗および物体Aにはたらく浮力は無視できるものとする。
    \begin{Enumerate}
        \item 糸の張力$T$はいくらか。
        \item 気球Bにはたらく浮力$F$はいくらか。また,外部の空気の密度を$\rho$とすると,気体の体積$V$はいくらか。
    \end{Enumerate}
    物体Aが地面から$h$の高さになったとき,糸を切断した。
    \begin{Enumerate*}
        \item Aが地面に到達するまでに要する時間$t_0$はいくらか。
        \item 糸が切断された後,気球がさらに$h$だけ上がったときの気球の速さ$v_1$はいくらか。
    \end{Enumerate*}
\end{mawarikomi}

% \newpage
\item
\begin{mawarikomi}{150pt}{%WinTpicVersion4.32a
{\unitlength 0.1in%
\begin{picture}(23.3800,35.7200)(2.7000,-41.0000)%
% LINE 2 0 3 0 Black White  
% 2 400 1400 2200 600
% 
\special{pn 8}%
\special{pa 400 1400}%
\special{pa 2200 600}%
\special{fp}%
% POLYGON 2 0 3 0 Black White  
% 5 1940 710 1858 528 1310 772 1392 955 1940 710
% 
\special{pn 8}%
\special{pa 1940 710}%
\special{pa 1858 528}%
\special{pa 1310 772}%
\special{pa 1392 955}%
\special{pa 1940 710}%
\special{pa 1858 528}%
\special{fp}%
% LINE 0 0 3 1 Black White  
% 2 1585 650 1422 285
% 
\special{pn 20}%
\special{pa 1585 650}%
\special{pa 1422 285}%
\special{fp}%
% VECTOR 2 0 3 0 Black White  
% 2 1148 694 748 884
% 
\special{pn 8}%
\special{pa 1148 694}%
\special{pa 748 884}%
\special{fp}%
\special{sh 1}%
\special{pa 748 884}%
\special{pa 817 873}%
\special{pa 796 861}%
\special{pa 800 837}%
\special{pa 748 884}%
\special{fp}%
% LINE 2 1 3 0 Black White  
% 2 408 1404 1808 1404
% 
\special{pn 8}%
\special{pa 408 1404}%
\special{pa 1808 1404}%
\special{da 0.015}%
% CIRCLE 2 0 3 0 Black White  
% 4 408 1404 808 1404 2208 1404 2208 604
% 
\special{pn 8}%
\special{ar 408 1404 400 400 5.8649610 6.2831853}%
% STR 2 0 3 0 Black White  
% 4 858 1254 858 1354 2 0 0 0
% $\theta $
\put(8.5800,-13.5400){\makebox(0,0)[lb]{$\theta $}}%
% STR 2 0 3 0 Black White  
% 4 1608 1465 1608 1565 5 0 0 0
% {\bf 図1}
\put(16.0800,-15.6500){\makebox(0,0){{\bf 図1}}}%
% VECTOR 2 0 3 0 Black White  
% 2 408 3795 408 1995
% 
\special{pn 8}%
\special{pa 408 3795}%
\special{pa 408 1995}%
\special{fp}%
\special{sh 1}%
\special{pa 408 1995}%
\special{pa 388 2062}%
\special{pa 408 2048}%
\special{pa 428 2062}%
\special{pa 408 1995}%
\special{fp}%
% VECTOR 2 0 3 0 Black White  
% 2 408 3795 2608 3795
% 
\special{pn 8}%
\special{pa 408 3795}%
\special{pa 2608 3795}%
\special{fp}%
\special{sh 1}%
\special{pa 2608 3795}%
\special{pa 2541 3775}%
\special{pa 2555 3795}%
\special{pa 2541 3815}%
\special{pa 2608 3795}%
\special{fp}%
% LINE 2 0 3 0 Black White  
% 2 808 3795 808 3845
% 
\special{pn 8}%
\special{pa 808 3795}%
\special{pa 808 3845}%
\special{fp}%
% LINE 2 0 3 0 Black White  
% 2 1200 3791 1200 3841
% 
\special{pn 8}%
\special{pa 1200 3791}%
\special{pa 1200 3841}%
\special{fp}%
% LINE 2 0 3 0 Black White  
% 2 1600 3791 1600 3841
% 
\special{pn 8}%
\special{pa 1600 3791}%
\special{pa 1600 3841}%
\special{fp}%
% LINE 2 0 3 0 Black White  
% 2 2000 3791 2000 3841
% 
\special{pn 8}%
\special{pa 2000 3791}%
\special{pa 2000 3841}%
\special{fp}%
% LINE 2 0 3 0 Black White  
% 2 2400 3791 2400 3841
% 
\special{pn 8}%
\special{pa 2400 3791}%
\special{pa 2400 3841}%
\special{fp}%
% LINE 2 0 3 0 Black White  
% 2 401 3395 351 3395
% 
\special{pn 8}%
\special{pa 401 3395}%
\special{pa 351 3395}%
\special{fp}%
% LINE 2 0 3 0 Black White  
% 2 400 2991 350 2991
% 
\special{pn 8}%
\special{pa 400 2991}%
\special{pa 350 2991}%
\special{fp}%
% LINE 2 0 3 0 Black White  
% 2 400 2591 350 2591
% 
\special{pn 8}%
\special{pa 400 2591}%
\special{pa 350 2591}%
\special{fp}%
% LINE 2 0 3 0 Black White  
% 2 400 2191 350 2191
% 
\special{pn 8}%
\special{pa 400 2191}%
\special{pa 350 2191}%
\special{fp}%
% LINE 2 1 3 0 Black White  
% 4 400 2591 400 2591 800 2591 800 2591
% 
\special{pn 8}%
\special{pa 400 2591}%
\special{pa 400 2591}%
\special{da 0.015}%
\special{pa 800 2591}%
\special{pa 800 2591}%
\special{da 0.015}%
% LINE 2 1 3 0 Black White  
% 4 400 2591 800 2591 800 2591 800 3791
% 
\special{pn 8}%
\special{pa 400 2591}%
\special{pa 800 2591}%
\special{da 0.015}%
\special{pa 800 2591}%
\special{pa 800 3791}%
\special{da 0.015}%
% LINE 2 1 3 0 Black White  
% 2 400 2191 2600 2191
% 
\special{pn 8}%
\special{pa 400 2191}%
\special{pa 2600 2191}%
\special{da 0.015}%
% SPLINE 1 0 3 0 Black White  
% 4 400 3791 1200 2391 2200 2191 2600 2191
% 
\special{pn 13}%
\special{pa 400 3791}%
\special{pa 444 3655}%
\special{pa 454 3621}%
\special{pa 466 3587}%
\special{pa 488 3519}%
\special{pa 499 3486}%
\special{pa 510 3452}%
\special{pa 534 3386}%
\special{pa 545 3353}%
\special{pa 557 3320}%
\special{pa 569 3288}%
\special{pa 582 3255}%
\special{pa 594 3223}%
\special{pa 607 3191}%
\special{pa 646 3098}%
\special{pa 660 3067}%
\special{pa 702 2977}%
\special{pa 747 2890}%
\special{pa 763 2862}%
\special{pa 795 2808}%
\special{pa 812 2781}%
\special{pa 829 2755}%
\special{pa 847 2729}%
\special{pa 883 2679}%
\special{pa 921 2631}%
\special{pa 941 2608}%
\special{pa 981 2564}%
\special{pa 1002 2543}%
\special{pa 1024 2522}%
\special{pa 1046 2502}%
\special{pa 1092 2464}%
\special{pa 1115 2446}%
\special{pa 1140 2429}%
\special{pa 1190 2397}%
\special{pa 1216 2382}%
\special{pa 1270 2354}%
\special{pa 1297 2341}%
\special{pa 1326 2329}%
\special{pa 1354 2318}%
\special{pa 1383 2307}%
\special{pa 1413 2296}%
\special{pa 1473 2278}%
\special{pa 1504 2269}%
\special{pa 1535 2262}%
\special{pa 1567 2254}%
\special{pa 1598 2247}%
\special{pa 1662 2235}%
\special{pa 1728 2225}%
\special{pa 1760 2220}%
\special{pa 1826 2212}%
\special{pa 1860 2209}%
\special{pa 1926 2203}%
\special{pa 1959 2201}%
\special{pa 1993 2199}%
\special{pa 2059 2195}%
\special{pa 2093 2194}%
\special{pa 2192 2191}%
\special{pa 2224 2191}%
\special{pa 2257 2190}%
\special{pa 2321 2190}%
\special{pa 2353 2189}%
\special{pa 2385 2189}%
\special{pa 2417 2190}%
\special{pa 2544 2190}%
\special{pa 2576 2191}%
\special{pa 2600 2191}%
\special{fp}%
% LINE 2 1 3 0 Black White  
% 4 400 3791 800 2591 800 2591 1000 1991
% 
\special{pn 8}%
\special{pa 400 3791}%
\special{pa 800 2591}%
\special{da 0.015}%
\special{pa 800 2591}%
\special{pa 1000 1991}%
\special{da 0.015}%
% STR 2 0 3 0 Black White  
% 4 300 3291 300 3391 5 0 0 0
% 1
\put(3.0000,-33.9100){\makebox(0,0){1}}%
% STR 2 0 3 0 Black White  
% 4 300 2891 300 2991 5 0 0 0
% 2
\put(3.0000,-29.9100){\makebox(0,0){2}}%
% STR 2 0 3 0 Black White  
% 4 300 2491 300 2591 5 0 0 0
% 3
\put(3.0000,-25.9100){\makebox(0,0){3}}%
% STR 2 0 3 0 Black White  
% 4 300 2091 300 2191 5 0 0 0
% 4
\put(3.0000,-21.9100){\makebox(0,0){4}}%
% STR 2 0 3 0 Black White  
% 4 300 1841 300 1941 2 0 0 0
% $v$\kern-4pt{\sf 〔m/s〕}
\put(3.0000,-19.4100){\makebox(0,0)[lb]{$v$\kern-4pt{\sf 〔m/s〕}}}%
% STR 2 0 3 0 Black White  
% 4 2480 3931 2480 4031 2 0 0 0
% $t$\kern-4pt{\sf 〔s〕}
\put(24.8000,-40.3100){\makebox(0,0)[lb]{$t$\kern-4pt{\sf 〔s〕}}}%
% STR 2 0 3 0 Black White  
% 4 408 3835 408 3935 5 0 0 0
% 0
\put(4.0800,-39.3500){\makebox(0,0){0}}%
% STR 2 0 3 0 Black White  
% 4 800 3831 800 3931 5 0 0 0
% 1
\put(8.0000,-39.3100){\makebox(0,0){1}}%
% STR 2 0 3 0 Black White  
% 4 1200 3831 1200 3931 5 0 0 0
% 2
\put(12.0000,-39.3100){\makebox(0,0){2}}%
% STR 2 0 3 0 Black White  
% 4 1600 3831 1600 3931 5 0 0 0
% 3
\put(16.0000,-39.3100){\makebox(0,0){3}}%
% STR 2 0 3 0 Black White  
% 4 2000 3831 2000 3931 5 0 0 0
% 4
\put(20.0000,-39.3100){\makebox(0,0){4}}%
% STR 2 0 3 0 Black White  
% 4 2400 3831 2400 3931 5 0 0 0
% 5
\put(24.0000,-39.3100){\makebox(0,0){5}}%
% STR 2 0 3 0 Black White  
% 4 1608 4065 1608 4165 5 0 0 0
% {\bf 図2}
\put(16.0800,-41.6500){\makebox(0,0){{\bf 図2}}}%
\end{picture}}%
}
    傾角$\theta $の斜面上を{\bf 図1}のようなT型の物体がすべる運動を考える。物体の質量を $M$,動摩擦係数を$\mu $,重力加速度を$g$とする。速さが$v$のとき,空気の抵抗力$kv$がはたらくものとする。
    \begin{Enumerate}
        \item 運動中の物体に作用する力の名称とその向きを,矢印で図の上に示せ。
        \item 物体が速さ$v$,加速度$a$で運動しているときの運動方程式を記せ。
        \item しばらくして,等速度運動になった場合の速さ$v$を求めよ。
    \end{Enumerate}
    $M=2.0$\tanni{kg},$\theta =30\Deg$のとき,{\bf 図2}の曲線のような結果が得られた。\\ なお,{\bf 図2}の斜めの破線は,時刻$t=0$のときの接線とし,$g=10$\tanni{m/s^2}とする。
    \begin{Enumerate*}
        \item 動摩擦係数$\mu $を求めよ。
        \item 空気の抵抗力の係数$k$を求めよ。
    \end{Enumerate*}
\end{mawarikomi}

\newpage
\item
        \begin{mawarikomi}{150pt}{%WinTpicVersion4.32a
{\unitlength 0.1in%
\begin{picture}(24.0000,12.2000)(4.0000,-18.0000)%
% LINE 2 0 3 0 Black White  
% 12 400 1000 1000 1200 1000 1200 1600 1000 1600 1000 2200 800 2200 800 1600 600 1000 1200 1000 1400 1000 1400 1300 1300
% 
\special{pn 8}%
\special{pa 400 1000}%
\special{pa 1000 1200}%
\special{fp}%
\special{pa 1000 1200}%
\special{pa 1600 1000}%
\special{fp}%
\special{pa 1600 1000}%
\special{pa 2200 800}%
\special{fp}%
\special{pa 2200 800}%
\special{pa 1600 600}%
\special{fp}%
\special{pa 1000 1200}%
\special{pa 1000 1400}%
\special{fp}%
\special{pa 1000 1400}%
\special{pa 1300 1300}%
\special{fp}%
% LINE 2 0 3 0 Black White  
% 8 1300 1300 1300 1100 1300 1100 1900 1300 1900 1300 2500 1100 2500 1100 1900 900
% 
\special{pn 8}%
\special{pa 1300 1300}%
\special{pa 1300 1100}%
\special{fp}%
\special{pa 1300 1100}%
\special{pa 1900 1300}%
\special{fp}%
\special{pa 1900 1300}%
\special{pa 2500 1100}%
\special{fp}%
\special{pa 2500 1100}%
\special{pa 1900 900}%
\special{fp}%
% LINE 2 0 3 0 Black White  
% 2 2200 1000 2200 800
% 
\special{pn 8}%
\special{pa 2200 1000}%
\special{pa 2200 800}%
\special{fp}%
% LINE 2 0 3 0 Black White  
% 8 1300 1300 1900 1500 1900 1500 2500 1300 2500 1300 2500 1100 1900 1300 1900 1500
% 
\special{pn 8}%
\special{pa 1300 1300}%
\special{pa 1900 1500}%
\special{fp}%
\special{pa 1900 1500}%
\special{pa 2500 1300}%
\special{fp}%
\special{pa 2500 1300}%
\special{pa 2500 1100}%
\special{fp}%
\special{pa 1900 1300}%
\special{pa 1900 1500}%
\special{fp}%
% LINE 2 0 3 0 Black White  
% 4 1000 1400 2200 1800 2200 1000 2800 1200
% 
\special{pn 8}%
\special{pa 1000 1400}%
\special{pa 2200 1800}%
\special{fp}%
\special{pa 2200 1000}%
\special{pa 2800 1200}%
\special{fp}%
% STR 2 0 3 0 Black White  
% 4 1600 1210 1600 1310 5 0 0 0
% A
\put(16.0000,-13.1000){\makebox(0,0){A}}%
% STR 2 0 3 0 Black White  
% 4 1160 1160 1160 1260 5 0 0 0
% S$_3$
\put(11.6000,-12.6000){\makebox(0,0){S$_3$}}%
% STR 2 0 3 0 Black White  
% 4 2180 1500 2180 1600 5 0 0 0
% S$_2$
\put(21.8000,-16.0000){\makebox(0,0){S$_2$}}%
% STR 2 0 3 0 Black White  
% 4 780 900 780 1000 5 0 0 0
% S$_1$
\put(7.8000,-10.0000){\makebox(0,0){S$_1$}}%
% LINE 2 2 3 0 Black White  
% 2 2230 1190 1030 790
% 
\special{pn 8}%
\special{pa 2230 1190}%
\special{pa 1030 790}%
\special{dt 0.025}%
% CIRCLE 2 0 0 0 Black Black  
% 4 980 780 1020 780 1020 780 1020 780
% 
\special{sh 1.000}%
\special{ia 980 780 40 40 0.0000000 6.2831853}%
\special{pn 8}%
\special{ar 980 780 40 40 0.0000000 6.2831853}%
% VECTOR 2 0 3 0 Black White  
% 2 990 780 1290 880
% 
\special{pn 8}%
\special{pa 990 780}%
\special{pa 1290 880}%
\special{fp}%
\special{sh 1}%
\special{pa 1290 880}%
\special{pa 1233 840}%
\special{pa 1239 863}%
\special{pa 1220 878}%
\special{pa 1290 880}%
\special{fp}%
% STR 2 0 3 0 Black White  
% 4 900 610 900 710 2 0 0 0
% B
\put(9.0000,-7.1000){\makebox(0,0)[lb]{B}}%
% STR 2 0 3 0 Black White  
% 4 1170 680 1170 780 2 0 0 0
% $v_0$
\put(11.7000,-7.8000){\makebox(0,0)[lb]{$v_0$}}%
\end{picture}}%
}
            なめらかな水平面S$_1$,S$_2$と鉛直面S$_3$からなる段差のある固定台がある。面S$_2$上に,質量$M$の直方体Aを面S$_3$に接するように置く。Aの上面はあらく,その高さはS$_1$の高さに等しい。質量$m$の小物体BとAの間の動摩擦係数を$\mu $とし,重力加速度を$g$とする。いま,Bを初速$v_0$で水平面S$_1$上から,Aの上面中央を直進させたところ,Aは運動をはじめ,ある時刻$t_0$以後,両物体の速さは等しくなった。\\
            \hakosyokika
            ~~BがA上に達した時刻を$t=0$とする。時刻$t_0$より以前の時刻$t$におけるBの速さは\Hako で,Aの速さは\Hako である。$t_0$は\Hako で,そのときの速さは\Hako である。また,BがA上を進んだ距離$\ell $は\Hako である。
        \end{mawarikomi}

\newpage
\item 
    \begin{mawarikomi}{120pt}{%WinTpicVersion4.32a
{\unitlength 0.1in%
\begin{picture}(18.5000,6.9200)(4.0000,-8.2200)%
% LINE 2 0 3 0 Black White  
% 2 400 600 2200 600
% 
\special{pn 8}%
\special{pa 400 600}%
\special{pa 2200 600}%
\special{fp}%
% BOX 2 0 3 0 Black White  
% 2 600 400 1800 600
% 
\special{pn 8}%
\special{pa 600 400}%
\special{pa 1800 400}%
\special{pa 1800 600}%
\special{pa 600 600}%
\special{pa 600 400}%
\special{pa 1800 400}%
\special{fp}%
% BOX 2 0 3 0 Black White  
% 2 1800 400 1650 300
% 
\special{pn 8}%
\special{pa 1800 400}%
\special{pa 1650 400}%
\special{pa 1650 300}%
\special{pa 1800 300}%
\special{pa 1800 400}%
\special{pa 1650 400}%
\special{fp}%
% LINE 3 0 3 0 Black White  
% 8 1790 310 1700 400 1740 300 1650 390 1680 300 1650 330 1800 360 1760 400
% 
\special{pn 4}%
\special{pa 1790 310}%
\special{pa 1700 400}%
\special{fp}%
\special{pa 1740 300}%
\special{pa 1650 390}%
\special{fp}%
\special{pa 1680 300}%
\special{pa 1650 330}%
\special{fp}%
\special{pa 1800 360}%
\special{pa 1760 400}%
\special{fp}%
% STR 2 0 3 0 Black White  
% 4 1830 170 1830 270 2 0 0 0
% $m$
\put(18.3000,-2.7000){\makebox(0,0)[lb]{$m$}}%
% VECTOR 1 0 3 0 Black White  
% 2 1800 500 2200 500
% 
\special{pn 13}%
\special{pa 1800 500}%
\special{pa 2200 500}%
\special{fp}%
\special{sh 1}%
\special{pa 2200 500}%
\special{pa 2133 480}%
\special{pa 2147 500}%
\special{pa 2133 520}%
\special{pa 2200 500}%
\special{fp}%
% STR 2 0 3 0 Black White  
% 4 2310 400 2310 500 5 0 0 0
% 力
\put(23.1000,-5.0000){\makebox(0,0){力}}%
% STR 2 0 3 0 Black White  
% 4 910 400 910 500 5 0 0 0
% $M$
\put(9.1000,-5.0000){\makebox(0,0){$M$}}%
% STR 2 0 3 0 Black White  
% 4 490 400 490 500 5 0 0 0
% 板
\put(4.9000,-5.0000){\makebox(0,0){板}}%
% VECTOR 2 0 3 0 Black White  
% 4 1197 795 597 795 1397 795 1797 795
% 
\special{pn 8}%
\special{pa 1197 795}%
\special{pa 597 795}%
\special{fp}%
\special{sh 1}%
\special{pa 597 795}%
\special{pa 664 815}%
\special{pa 650 795}%
\special{pa 664 775}%
\special{pa 597 795}%
\special{fp}%
\special{pa 1397 795}%
\special{pa 1797 795}%
\special{fp}%
\special{sh 1}%
\special{pa 1797 795}%
\special{pa 1730 775}%
\special{pa 1744 795}%
\special{pa 1730 815}%
\special{pa 1797 795}%
\special{fp}%
% STR 2 0 3 0 Black White  
% 4 1297 695 1297 795 5 0 0 0
% $L$
\put(12.9700,-7.9500){\makebox(0,0){$L$}}%
% LINE 3 0 3 0 Black White  
% 60 2100 600 2020 680 2040 600 1960 680 1980 600 1900 680 1920 600 1840 680 1860 600 1780 680 1800 600 1720 680 1740 600 1660 680 1680 600 1600 680 1620 600 1540 680 1560 600 1480 680 1500 600 1420 680 1440 600 1360 680 1380 600 1300 680 1320 600 1240 680 1260 600 1180 680 1200 600 1120 680 1140 600 1060 680 1080 600 1000 680 1020 600 940 680 960 600 880 680 900 600 820 680 840 600 760 680 780 600 700 680 720 600 640 680 660 600 580 680 600 600 520 680 540 600 460 680 480 600 410 670 2160 600 2080 680 2200 620 2140 680
% 
\special{pn 4}%
\special{pa 2100 600}%
\special{pa 2020 680}%
\special{fp}%
\special{pa 2040 600}%
\special{pa 1960 680}%
\special{fp}%
\special{pa 1980 600}%
\special{pa 1900 680}%
\special{fp}%
\special{pa 1920 600}%
\special{pa 1840 680}%
\special{fp}%
\special{pa 1860 600}%
\special{pa 1780 680}%
\special{fp}%
\special{pa 1800 600}%
\special{pa 1720 680}%
\special{fp}%
\special{pa 1740 600}%
\special{pa 1660 680}%
\special{fp}%
\special{pa 1680 600}%
\special{pa 1600 680}%
\special{fp}%
\special{pa 1620 600}%
\special{pa 1540 680}%
\special{fp}%
\special{pa 1560 600}%
\special{pa 1480 680}%
\special{fp}%
\special{pa 1500 600}%
\special{pa 1420 680}%
\special{fp}%
\special{pa 1440 600}%
\special{pa 1360 680}%
\special{fp}%
\special{pa 1380 600}%
\special{pa 1300 680}%
\special{fp}%
\special{pa 1320 600}%
\special{pa 1240 680}%
\special{fp}%
\special{pa 1260 600}%
\special{pa 1180 680}%
\special{fp}%
\special{pa 1200 600}%
\special{pa 1120 680}%
\special{fp}%
\special{pa 1140 600}%
\special{pa 1060 680}%
\special{fp}%
\special{pa 1080 600}%
\special{pa 1000 680}%
\special{fp}%
\special{pa 1020 600}%
\special{pa 940 680}%
\special{fp}%
\special{pa 960 600}%
\special{pa 880 680}%
\special{fp}%
\special{pa 900 600}%
\special{pa 820 680}%
\special{fp}%
\special{pa 840 600}%
\special{pa 760 680}%
\special{fp}%
\special{pa 780 600}%
\special{pa 700 680}%
\special{fp}%
\special{pa 720 600}%
\special{pa 640 680}%
\special{fp}%
\special{pa 660 600}%
\special{pa 580 680}%
\special{fp}%
\special{pa 600 600}%
\special{pa 520 680}%
\special{fp}%
\special{pa 540 600}%
\special{pa 460 680}%
\special{fp}%
\special{pa 480 600}%
\special{pa 410 670}%
\special{fp}%
\special{pa 2160 600}%
\special{pa 2080 680}%
\special{fp}%
\special{pa 2200 620}%
\special{pa 2140 680}%
\special{fp}%
% STR 2 0 3 0 Black White  
% 4 1610 160 1610 260 3 0 0 0
% P
\put(16.1000,-2.6000){\makebox(0,0)[rb]{P}}%
\end{picture}}%
}
        滑らかな水平面上に質量$M$,長さ$L$の板を置く。板上の上面はあらい水平面で,右端に質量$m$の小物体Pが置かれている。
        重力加速度を$g$とする。
        \begin{enumerate}
            \item 板に一定の大きさの力$F_1$を水平右向きに加え続けたところ,Pと板は一体となって運動した。
            \begin{enumerate}[(ア)]
                \item 板の加速度$\alpha $を求めよ。
                \item Pが板から受けている摩擦力の大きさ$f$を求めよ。
            \end{enumerate}
            \item 板とPを静止させ,板に$F_1$よりも大きい一定の力$F_2$を水平右向きに加え続けたところ,板は運動し,Pは板の上を滑り続けた。Pと板の間の静止摩擦係数を$\mu $,動摩擦係数を$\mu '$とする。
                \begin{enumerate}[(ア)]
                    \item Pが板上ですべるためには,$F_2$はある値$F_0$より大きくなければならない。$F_0$を求めよ。
                    \item $F_2$の力を加えているときの板の加速度$A$を求めよ。
                    \item Pが板の左端に達するまでの時間$t$を求めよ。
                \end{enumerate}
        \end{enumerate}
    \end{mawarikomi}

\newpage
\item
    \begin{mawarikomi}{150pt}{\begin{zahyou*}[ul=4mm](-1,13)(-1,6)
\def\O{(4,4)}
\def\A{(0,4)}
\def\AR{(0.2,4)}
\def\AD{(0,0)}
\def\B{(4,0)}
\def\CD{(7.464,0)}
\def\D{(9.196,3.5)}
\def\DD{(9.196,0)}
\def\E{(11.8,0)}
\def\F{(-1,0)}
\def\G{(13,0)}
\def\H{(13,-0.5)}
\def\I{(-1,-0.5)}
\def\vvec{(1,1.732)}
\def\Gx{(4*cos(T)+4)}
\def\Gy{(4*sin(T)+4)}
\BGurafu\Gx\Gy{3.1415}{5.759}
\Drawline{\F\G}

\Candk\O{4}\O{330}\CC\C
\Nuritubusi*{\A\AD\CD\C\A}
\BNuri[0]\Gx\Gy{3.1415}{5.8}
\Enko\O{4}{180}{330}

\Drawline{\C\CD}
\Drawline{\A\AD}
\Put\C[se]{C}
\Put\C{\Yasen\vvec}
\Put\A[w]{A}
\Put\O[n]{O}
\Put\D[n]{D}
\HenKo<henkoH=0.001ex,henkotype=parallel,yazirusi=b>\DD\D{$h$}
\Hasen{\O\A}
\Hasen{\O\C}
\Hasen{\O\B}
\Tyokkakukigou\A\O\B
\Kakukigou\B\O\C(0pt,-10pt)[l]{60\Deg}
\Put\B(0pt,-10pt){B}
\Put\E[ne]{E}
\def\Fx{6.261*0.5*T+7.464}
\def\Fy{6.261*1.732*0.5*T-4.9*T*T+2}
\BGurafu(0.05)(0.02)\Fx\Fy{0}{1.4}

\En*[0]\AR{0.2}
\En\AR{0.2}
\Put\AR(3pt,5pt)[l]{$m$}
\Nuritubusi*{\F\G\H\I\F}
\HenKo<henkoH=0.001ex,henkotype=parallel,yazirusi=b>\O\C{$r$}

\end{zahyou*}
}
        半径$r$の円弧の形をした滑らかなすべり台ABCが,水平な床にB点で接して固定されている。
        中心をOとする円弧ABCは鉛直な平面内にあり,$\angle $AOB=90\Deg ,$\angle $BOC=60\Deg
        である。A点に静止していた質量$m$の小球が,すべり台をすべり落ちてB点を通り,C点ですべり台から飛び出す。
        そののち,最高点Dに達し,再び落下してE点において床と衝突する。重力加速度を$g$とする。
        \begin{enumerate}
            \item 小球のB点での速さ$v_\mathrm{B}$を求めよ。また,C点での速さ$v_\mathrm{C}$を求めよ。
            \item AC間で,小球にはたらく重力のした仕事と垂直抗力のした仕事をそれぞれ求めよ。
            \item D点での小球の速さ$v_\mathrm{D}$とD点の高さ$h$を求め,それぞれ$r$,$g$を用いて表せ。
            \item E点で床に衝突するときの速さ$v_\mathrm{E}$を求め,$r$,$g$を用いて表せ。
        \end{enumerate}
    \end{mawarikomi}
\newpage
\item
    \begin{mawarikomi}{180pt}{\begin{zahyou*}[ul=5mm](-1,13)(-1,5)

	% \drawline(-1,-1)(-1,13)(13,13)(13,-1)(-1,-1)
	\def\Fx{0.2*(cos(T)+T/3.5)+0.25}
	\def\Fy{-0.3*sin(T)+0.4}
	\BGurafu\Fx\Fy{3.14}{20*3.14}
	\drawline(0,1)(0,0)
	\drawline(0,0)(13,0)
	\drawline(4,0.4)(3.8,0.4)
	\def\A{(8,0)}
	\def\AD{(8,-0.5)}
	\def\B{(11.5,0)}
	\def\C{(13,0)}
	\Kaiten\A\C{30}\CC
	\def\CD{(13,-0.5)}
	\def\P{(4,0.8)}
	\def\PD{(4,0.1)}
	\def\PR{(6,0.8)}
	\def\OU{(0,0.8)}
	\Drawlines{\P\PD}
	\Hasen{\A\CC}
	\En*[0]{(4.3,0.3)}{0.3}
	% \scriptsize
	\hasen(6,1)(6,0)
	\HenKo<henkotype=parallel
			,henkoH=4ex
			,yazirusi=b
			,henkosideb=0.3
			,henkosidet=1.5>\PR\OU{自然長}
	\HenKo<henkotype=parallel
			,henkoH=4.6ex
			,yazirusi=b
			,henkosideb=0.8
			,henkosidet=1.2>\P\PR{$a$}
	\Put\B(0pt,-10pt)[t]{B}
	\Put\A(0pt,-10pt)[t]{A}
	\Put\P(0pt,5pt)[l]{P}
	\Nuritubusi*{\A\C\CD\AD\A}
	\Kakukigou\C\A\CC<1.8>(2pt,1pt)[l]{30\Deg}
\end{zahyou*}
}
        水平に置かれたばね定数$k$\tanni{N/m}の軽いばねに質量$m$\tanni{kg}の小球Pを押し当て,ばねを自然長から$a$\tanni{m}だけ縮ませ,静かにPを放した。水平面は図の点Aより左側はなめらかであるが,右側は粗く,Pとの動摩擦係数は$\mu $である。重力加速度の大きさを$g$\tanni{m/s^2}とする。
        \begin{enumerate}
            \item ばねから離れたPが点Aに達するときの速さ$v$\tanni{m/s}を求めよ。
            \item ばねの縮みが$\bunsuu{1}{2}a$\tanni{m}であったときの,Pの速さ$v$を求めよ。
            \item はじめにばねを自然長から$a$\tanni{m}だけ縮ませるのに必要であった外力の仕事$W$を求めよ。
            \item 点Aを通り過ぎたPはやがて点Bで静止した。距離ABを$v$を用いて求めよ。
            \item 粗い面が水平から30\Deg 傾いた斜面(図の点線)であった場合に,Pが達する最高点をCとし,距離ACを$v$を用いて求めよ。斜面と水平面はなだらかにつながるものとする。
        \end{enumerate}
    \end{mawarikomi}
\newpage
\item
    \begin{mawarikomi}{80pt}{\begin{zahyou*}[ul=4.5mm](-1,7)(-1,15)
	% \drawline(-1,-1)(-1,13)(13,13)(13,-1)(-1,-1)
	\def\LU{(-1,15)}
	\def\RU{(6,15)}
	\def\R{(6,14.5)}
	\def\L{(-1,14.5)}
	\def\KL{(2,7)}
	\def\KR{(4,12)}
	\def\TL{(1,14.5)}
	\def\TLD{(1,7)}
	\def\TMD{(3,7)}
	\def\TM{(3,12)}
	\def\TR{(5,12)}
	\def\TRD{(5,6)}
	\def\TDD{(2,4)}
	\def\TLM{(1,10)}
	\def\H{(4.2,11.8)}
	\def\I{(4.2,14.5)}
	\def\J{(3.8,14.5)}
	\def\K{(3.8,11.8)}
	\def\A{(2.5,3.5)}
	\def\B{(2.5,4)}
	\def\C{(1.5,4)}
	\def\D{(1.5,3.5)}
	\def\E{(5.5,5.5)}
	\def\F{(5.5,6)}
	\def\G{(4.5,6)}
	\def\HD{(4.5,5.5)}
	\En*[0.2]\KL{1}
	\En*[0.2]\KR{1}
	\En\KL{1}
	\En\KR{1}
	\Nuritubusi*{\RU\LU\L\R\RU}
	\Drawline{\TL\TLD}
	\Drawline{\TMD\TM}
	\Drawline{\TR\TRD}
	\Drawline{\KL\TDD}
	\Nuritubusi[0]{\I\H\K\J\I}
	\Drawline{\I\H\K\J}
	\Drawline{\L\R}
	\Nuritubusi*{\A\B\C\D\A}
	\Drawline{\A\B\C\D\A}
	\Nuritubusi*{\E\F\G\HD\E}
	\Drawline{\E\F\G\HD\E}
	\Kuromaru{\KR}
	\Kuromaru{\KL}
	\Put\TLM[e]{$\alpha $}
	\Put\C[sw]{A}
	\Put\D[se]{$M$}
	\Put\G[sw]{B}
	\Put\HD[se]{$m$}
\end{zahyou*}
}
    質量$M$のおもりAと,おもりBを糸で結び,なめらかな定滑車と動滑車に図のように図のように糸$\alpha $をかけてつるす。定滑車の質量は無視でき,重力加速度の大きさを$g$とする。
        \begin{enumerate}
            \item Bの質量が$m_0$のとき,全体は静止した。糸$\alpha $の張力$T$と$m_0$を$M$,$g$を用いて表せ。
            \item 次に,Bの質量を$m(>m_0)$とし,全体が静止している状態からA,Bを静かに放す。
            \begin{enumerate}
                \item Aが高さ$h$だけ上がったときの速さを$v$とする。このときのBの下がった距離と速さを求めよ。
                \item 前門の間に,Bが失った重力の位置エネルギーはいくらか。
                \item Aの速さ$v$を$M$,$m$,$h$,$g$を用いて表せ。
            \end{enumerate}
        \end{enumerate}
    \end{mawarikomi}
\newpage
\item
    \begin{mawarikomi}{150pt}{\begin{zahyou*}[ul=5mm](-1,10)(-1,5)
	% \drawline(-1,-1)(-1,13)(13,13)(13,-1)(-1,-1)
	\def\kakudo{20}
	\calcval{\kakudo *2*3.14159265/360}\TH
	\def\O{(0,0)}
	\def\A{(4.1,0.7)}
	\def\B{(4.1,0)}
	\def\C{(3.5,0)}
	\def\D{(3.5,0.7)}
	\def\E{(9,1)}
	\def\F{(9.5,1)}
	\def\G{(9.5,0)}
	\def\H{(9,0)}
	\def\I{(1,0)}
	\def\Dy{1.353}
	\calcval{\Dy*(1/tan(\TH))}\Dx
	\Kaiten\O\A{\kakudo}\AA
	\Kaiten\O\B{\kakudo}\BB
	\Kaiten\O\C{\kakudo}\CC
	\Kaiten\O\D{\kakudo}\DD
	\Kaiten\O\E{\kakudo}\EE
	\Kaiten\O\F{\kakudo}\FF
	\Kaiten\O\G{\kakudo}\GG
	\Kaiten\O\H{\kakudo}\HH
	\def\Fx{(0.2*(cos(T)+T/3.5))*cos(\TH)-(-0.3*sin(T))*sin(\TH)+\Dx}
	\def\Fy{(0.2*(cos(T)+T/3.5))*sin(\TH)+(-0.3*sin(T))*cos(\TH)+\Dy+0.5}
	\Drawline{\AA\BB\CC\DD\AA}
	\BGurafu\Fx\Fy{3.14}{26*3.14}
	\Suisen\GG\O\G\GGD
	\kandk\I{\kakudo}\GGD{90}\GGDU
	\Nuritubusi*[75]{\EE\FF\GG\HH}
	\Nuritubusi*[75]{\AA\BB\CC\DD}
	\Nuritubusi[0.2]{\I\GGDU\GG\O\I}
	\Drawline{\EE\HH}
	\Drawline{\O\GG}
	\Drawline{\O\GGD}
	\Drawline{\GG\GGD}
	\Put\DD[nw]{P}
	\Kakukigou\G\O\GG<hankei=1.5>(2pt,0.8pt)[l]{$\theta$}
\end{zahyou*}
}
    質量$m$のおもりPを鉛直につるすと$\ell $だけ伸びる軽いばねがある。
    重力加速度の大きさを$g$とする。
    図のように,傾角$\theta $の斜面上で,Pをつけたばねの上端を固定する。
    斜面とPの間の静止摩擦係数を$\mu $,動摩擦係数を$\mu '$とし,ばねが自然の長さに保たれるようにPを手で支えておく。
        \begin{enumerate}
            \item 手を放したとき,Pが動き始めるためには,斜面の傾角$\theta $は,$\alpha $より大きくなければならない。$\tan{\alpha }$を求めよ。
            \item 傾角$\theta (>\alpha )$の斜面上で手を放すとPが動き始めた。ばねの伸びが最大値$x$になったとき,Pの最初の位置から重力の位置エネルギーはいくら減少するか。
            \item (2)において,ばねの弾性エネルギーはいくらか。
            \item ばねの最大の伸び$x$を求めよ。
            \item ばねの伸びが最大になったのち,Pが再び動き始めるためには,$\tan{\theta }$はある値より大きくなければならない。その値を$\mu $,$\mu '$で表せ。
        \end{enumerate}
    \end{mawarikomi}
\newpage
\item
    \begin{mawarikomi}{190pt}{\begin{zahyou*}[ul=5mm](-1,15)(-1,5)
	% \drawline(-1,-1)(-1,13)(13,13)(13,-1)(-1,-1)
	\def\kakudo{35}
	\calcval{\kakudo *2*3.14159265/360}\TH
	\def\A{(0,0)}
	\def\B{(7,0)}
	\def\C{(15,0)}
	\def\BD{(7,-0.3)}
	\def\CD{(15,-0.3)}
	\def\E{(10,0)}
	\def\F{(1,0)}
	\def\G{(1.5,0)}
	\def\H{(1.5,0.3)}
	\def\I{(1,0.3)}
	\def\J{(5,0)}
	\def\K{(5.5,0)}
	\def\L{(5.5,0.3)}
	\def\M{(5,0.3)}
	\def\N{(1.25,0.6)}
	\def\vvec{(2,0)}
	\Kaiten\B\C{\kakudo}\CC
	\Kaiten\B\CD{\kakudo}\CCD
	\Kaiten\B\BD{\kakudo}\BBD
	\Nuritubusi*[75]{\B\CC\CCD\BBD}
	\Hasen{\B\E}
	\Drawline{\A\B}
	\Drawline{\B\CC}
	\Drawline{\F\G\H\I\F}
	\Drawline{\J\K\L\M\J}
	\Nuritubusi*{\J\K\L\M}
	\Put\CC[ne]{C}
	\Put\B[s]{B}
	\Put\A[sw]{A}
	\Put\I[nw]{P}
	\Put\L[ne]{Q}
	\Put\N{\Yasen\vvec}
	\Put\N[ne]{$v_0$}
	\Kakukigou\E\B\CC<hankei=1.5>(2pt,0.8pt)[l]{$\theta$}
\end{zahyou*}
}
    水平面ABと斜面BCがなだらかにつながっていて,AB間は摩擦がなく,
    傾角$\theta $の斜面には摩擦がある。AB上で,質量$m$の小物体Pが速さ$v_0$で,静止している質量$M$の小物体Qに正面衝突する。P,Qの反発係数(はね返り係数)を$e$,Qと斜面の間の動摩擦係数を$\mu $,重力加速度の大きさを$g$とする。
        \begin{enumerate}
            \item 衝突直後のPの速度$v$と,Qの速度$V$を,右向きを正としてそれぞれ求めよ。
            \item 衝突の際,Pが受けた力積を,右向きを正として求めよ。
            \item 衝突後,Pが左へ動くための条件を求めよ。
            \item 衝突後,Qは斜面上の点Dに達した後,下降した。$V$を用いてBD間の距離$\ell $を求めよ。また,Qが点Bに戻ったときの速さ$V_1$を$V$を用いて求めよ。
        \end{enumerate}
    \end{mawarikomi}
% \newpage
% \hakosyokika
\item
    \begin{mawarikomi}{150pt}{\begin{zahyou*}[ul=4mm](-10,2)(-10,1)
	% \drawline(-1,-1)(-1,13)(13,13)(13,-1)(-1,-1)
	\def\kakudo{-60}
%	\calcval{\kakudo *2*3.14159265/360}\TH
	\def\O{(0,0)}
	\def\B{(0,-9)}
	\def\F{(2,0)}
	\def\G{(2,0.5)}
	\def\H{(-2,0.5)}
	\def\I{(-2,0)}
	\Kaiten\O\B{\kakudo}\A
	\HenKo<henkoH=2.5ex>\B\O{$\ell $}
	\HenKo<henkoH=2.5ex>\O\A{$\ell $}
	\Nuritubusi*{\F\G\H\I\F}
	\Drawline{\O\B}
	\Drawline{\O\A}
	\Drawline{\I\F}
	\En*\B{0.4}
	\En\B{0.4}
	\En*[0]\A{0.4}
	\En\A{0.4}
	\Kakukigou\A\O\B<hankei=1.5>(-7pt,-8pt)[l]{60\Deg}
	\Put\B(5pt,0pt)[l]{B}
	\Put\O(8pt,-3pt)[t]{O}
	\Put\A(0pt,8pt)[b]{A}
\end{zahyou*}
}
        質量$m$の小球Aと$2m$の小球Bがあり,それぞれ長さ$\ell $の糸で天井の点Oからつるされている。Bを鉛直線に沿って静止させ,Aを糸が鉛直線から60\Deg 傾いた位置に持ち上げて,静かに放したところ,最下点でBに衝突した\\
        ~~AとBの衝突が完全弾性衝突のとき,衝突直後のBの速さは,重力加速度
        重力加速度の大きさを$g$とすると\Hako である。\\
        ~~AとBの衝突の直後にAが最下点でそのまま静止して,Bのみが運動する場合がある。
        このとき,AとBのはね返り係数は\Hako であり,Bは最下点より\Hako の高さまで上昇する。\\
        ~~次に,小球Aを取り去り,鉛直線に沿って静止させた小球Bに弾丸Cを水平に打ち込んだところ,BはCと一体になって運動を始めた。衝突直後の速さが衝突直前のCの速さの$\bunsuu{1}{5}$になったとすると,Cの質量は\Hako である。また,この衝突で失われた力学的エネルギーは,衝突直前のCの運動エネルギーの\Hako 倍である。
    \end{mawarikomi}
% \newpage
% \item
    \begin{mawarikomi}{120pt}{\begin{zahyou*}[ul=4mm](-5,5)(-5,5)
	\def\O{(0,0)}
	\def\R{(4.5,0)}
	\def\A{(-0.25,-4.75)}
	\def\B{(0.25,-4.75)}
	\def\M{(0,-2.5)}
	\HenKo<henkoH=2.5ex>\R\O{$R$}
	\Hasen{\O\R}
	\Kuromaru{\O}
	\En\O{4.5}
	\En\O{5.0}
	\En*[1]\A{0.25}
	\En*[1]\B{0.25}
	\Put\A(-4pt,12pt)[t]{A}
	\Put\B(4pt,12pt)[t]{B}
	\Put\M[t]{上から見た図}
\end{zahyou*}
}
        細い円形のパイプが水平に固定され,中に同じ質量$m$の小球A,Bが入って接触している。Aを速さ$2v_0$,Bを速さ$v_0$で逆向きに同時に打ち出したところ,AとBは半径$R$の等速円運動をし,パイプの内で衝突を繰り返す。衝突の際の反発係数を$e\left(0<e<\bunsuu{1}{3} \right)$とし,摩擦はなく,空気抵抗は無視する。
        \begin{Enumerate}
            \item AとBを打ち出してから1回目の衝突が起こるまでの時間$t$はいくらか。
            \item 1回目の衝突直後のAとBの速さはそれぞれいくらか。 
        \end{Enumerate}
        衝突後,AとBは同じ向きに運動し,やがてBはAに追いついて2回目の衝突が起き,以後,このような衝突を繰り返した。
        \begin{Enumerate*}
            \item 2回目の衝突直後のAとBの速さはそれぞれいくらか。
            \item 衝突を繰り返していくと,AとBの速さは同じ値に近づいていく。その値はいくらか。
        \end{Enumerate*}
    \end{mawarikomi}
% \newpage
% \item
    \begin{mawarikomi}{120pt}{\begin{zahyou*}[ul=4mm](-1,10)(-2,5)
	\def\O{(0,0)}
	\def\X{(10,0)}
	\def\OD{(0,-0.5)}
	\def\XD{(10,-0.5)}
	\def\B{(1,0)}
	\def\BL{(0.8,0)}
	\def\A{(8,0)}
	\def\P{(8,0.25)}
	\def\Q{(1,0.25)}
	\def\Pvec{(0,3)}
	\def\PL{(2.8,2.5)}
	\def\QR{(8,2.5)}
	\def\Qvec{(2.5,2.8)}
	\def\M{(5,-1)}
	\Candl\Q{1}\BL{\Qvec}\R\S
	\Put\Q{\Yasen\Qvec}
	\Put\P{\Yasen\Pvec}
	\En*[0]\Q{0.25}
	\En*[1]\P{0.25}
	\En\Q{0.25}
	\En\P{0.25}
	\Put\Q(-4pt,12pt)[t]{B}
	\Put\P(4pt,12pt)[t]{A}
	\Put\PL(0,0)[rb]{$V$}
	\Put\QR(3pt,0)[lb]{$v$}
	\Put\M[t]{地面}
	\Nuritubusi[0.25]{\O\OD\XD\X}
	\Drawline{\O\X}
	\Kakukigou\X\BL\S<hankei=1>(2pt,3pt)[l]{$\alpha $}
\end{zahyou*}
}
        水平な地面上のP点から質量$m$の小物体Aを鉛直に打ち上げ,同時にQ点から質量$M$の小球Bを打ち出す。Bの打ち上げ角度$\alpha $は変化させることができる。Aの打ち上げの初速を$v$,Bの初速を$V(>v)$とし,重力加速度の大きさを$g$とする。
        \begin{enumerate}
            \item AとBが衝突しない場合,Aの打ち上げから着地までの時間を求めよ。
            \item BをAに衝突させるには,角度$\alpha $をいくらにすべきか。$\sin{\alpha }$を求めよ。
            \item Aが最高点に達したときに衝突が起こるようにしたい。そのためにはPQ間の距離$\ell $をいくらにすればよいか。$\alpha $を用いずに表せ(以下,同様)。
            \item AとBが最も高い位置で衝突し,両者は合体した。合体直後の速度の水平成分と鉛直成分の大きさはそれぞれいくらか。
            \item AとBは合体した後,地面に落下した。P点から落下点までの距離$x$を求めよ。
        \end{enumerate}
    \end{mawarikomi}
% \newpage
% \hakosyokika
\item
    \begin{mawarikomi}{120pt}{\begin{zahyou*}[ul=4mm](-1,10)(-3,3)
	\def\RA{(2,1)}
	\def\RB{(0,1)}
	\def\RC{(0,-1)}
	\def\RD{(2,-1)}
	\def\QA{(5,1)}
	\def\QB{(3,1)}
	\def\QC{(3,-1)}
	\def\QD{(5,-1)}
	\def\PA{(9,1.5)}
	\def\PB{(6,1.5)}
	\def\PC{(6,-1.5)}
	\def\PD{(9,-1.5)}
	\def\R{(2,0)}
	\def\Q{(5,0)}
	\def\P{(9,0)}
	\def\RM{(1.5,-2.5)}
	\def\QM{(4,-2.5)}
	\def\PM{(7.5,-2.5)}
	\Enko\R{1}{-90}{90}
	\Enko\Q{1}{-90}{90}
	\Enko\P{1.5}{-90}{90}
	\Drawline{\RA\RB\RC\RD}
	\Drawline{\QA\QB\QC\QD}
	\Drawline{\PA\PB\PC\PD}
	\Put\RM(0,0)[t]{$m$}
	\Put\QM(0,0)[t]{$m$}
	\Put\PM(5pt,0)[t]{$2m$}
	\Put\R(-3pt,0)[r]{R}
	\Put\Q(-3pt,0)[r]{Q}
	\Put\P(-6pt,0)[r]{P}
\end{zahyou*}
}
        質量がそれぞれ$2m$,$m$,$m$の3つの部分P,Q,Rから成るロケットが宇宙空間で静止している。はじめ,Rを左向きに打ち出した。放出後のP・Qから見たRの速さは$u$であったので,P・Qの速さは\Hako である。また,この際に要したエネルギーは\Hako である。\\
        ~~続いて,Qを左向きに打ち出した。放出後のPから見たQの速さはやはり$u$であったことから,Pの速さは\Hako となっている。
    \end{mawarikomi}
% \newpage
% \hakosyokika
\item
    \begin{mawarikomi}{200pt}{\begin{zahyou*}[ul=5mm](-1,16)(-1,3)
	% \drawline(-1,-1)(-1,13)(13,13)(13,-1)(-1,-1)
	\def\Fx{0.2*(cos(T)+T/3.5)+2.04}
	\def\Fy{-0.3*sin(T)+0.6}
	\BGurafu\Fx\Fy{3.14}{20*3.14}
    \drawline(15,2)(15,0)
	\drawline(0,0)(15,0)
    \drawline(15,2)(15.5,2)
    \def\vvec{(2,0)}
    \def\A{(2,0)}
    \def\B{(2,1.5)}
    \def\C{(0.5,1.5)}
    \def\D{(0.5,0)}
    \def\E{(5.85,1.5)}
    \def\F{(5.85,0)}
    \def\G{(7.35,0)}
    \def\H{(7.35,1.5)}
    \def\I{(2,1.2)}
    \def\J{(5.85,1.2)}
    \small
    \Drawlines{\A\B\C\D\A}
    \Drawlines{\E\F\G\H\E}
    \Drawlines{\I\J}
    \Put\C[ne]{$2m$}
    \Put\D(8pt,6pt)[b]{A}
    \Put\E(8pt,2pt)[b]{$m$}
    \Put\F(8pt,6pt)[b]{B}
    \put(3,2){\Yasen\vvec}
    \put(3.8,2.2){$v$}
    \put(15.2,1){壁}

\end{zahyou*}
}
        質量$2m$\tanni{kg}の物体Aと質量$m$\tanni{kg}の物体Bとがあり,Aにはばね定数$k$\tanni{N/m}の軽いばねがつけられ,このばねを自然長より縮めた状態に保つため,BとAは糸で結ばれている。AとBは滑らかな水平床上を右方向へ速さ$v$\tanni{m/s}で動いている。ある点で糸が急に切れ,間もなくAは静止した。一方,Bはばねから離れて右方へ動き,壁と弾性衝突して左へ戻り,Aのばねに接触した。重力加速度の大きさを$g$\tanni{m/s^2}とする。
        \begin{enumerate}
            \item 糸が切れ,ばねから離れたときのBの速さはいくらか。
            \item はじめのばねの縮みはいくらであったか。
            \item 壁との衝突の際,Bが壁に与えた力積の大きさはいくらか。
            \item Bとばねが接触した後,ばねが最も縮んだときのBの速さはいくらか。
            \item Bとばねが接触した後,Bがばねから離れたときのAの速さはいくらか。
            \item 前問において,ばねから離れたBは図の左右どちらへ動くか。
        \end{enumerate}
    \end{mawarikomi}
% \newpage
% \hakosyokika
\item
    \begin{mawarikomi}{180pt}{\begin{zahyou*}[ul=6mm](-2,9)(-1,5)
	\drawline(1.04,3.5)(0,3.5)(0,0)
	\drawline(-2,0)(9,0)
	\drawline(8,0)(8,1.5)(3.5,1.5)
    \def\A{(1.4,2.8)}
    \def\B{(3.5,1.5)}
    \def\BU{(3.5,2)}
    \def\BD{(3.5,1.2)}
    \def\C{(6,1.5)}
    \def\CU{(6,2)}
    \def\D{(3.5,4)}
    \def\E{(0,1.5)}
    \def\F{(0,2.8)}
    \def\G{(8,1.5)}
    \def\GD{(8,1.2)}
    \def\P{(3.6,1.7)}
    \def\Q{(3.4,1.7)}
    \def\R{(3.4,1.5)}
    \def\S{(3.6,1.5)}

    \Enko\D{2.5}{191}{270}
    \small
    \Put\A(-5pt,-6pt)[b]{A}
    \Put\BU[n]{B}
    \Put\CU[n]{C}
    \HenKo<henkotype=parallel,
            yazirusi=b,
            henkosideb=0,
            henkosidet=1.5>\F\E{$h$}
    \HenKo<henkotype=parallel,
            yazirusi=b,
            henkosideb=0,
            henkosidet=1.5>\C\B{$\ell $}
    \Hasen{\F\A}
    \Hasen{\E\B}
    \Kaiten\D\P{-60}\PP
    \Kaiten\D\Q{-60}\QQ
    \Kaiten\D\R{-60}\RR
    \Kaiten\D\S{-60}\SS
    \Drawline{\PP\QQ\RR\SS\PP}
    \Nuritubusi[1]{\PP\QQ\RR\SS\PP}
    \Nuritubusi*<0.15>{\B\BD\GD\G\B}
    \put(3,0.5){台}
    \put(4,0.5){$M$}
    \put(4,-0.5){床}
    \put(1.5,3){P}
\end{zahyou*}
}
        質量$M$の台が水平な床上に置かれている。この台の上面では,摩擦がない曲面と摩擦のある水平面が点Bでなめらかにつながっている。台の水平面から高さ$h$にある点Aに質量$m$の小物体Pを置き,静かに放す。重力加速度の大きさを$g$とする。
        \begin{enumerate}
            \item 台が床に固定されているとき,Pは点Bまですべり落ちたのち,点Bから距離$\ell $だけ離れた点Cで止まった。BC間の水平面とPの間の動摩擦係数$\mu$はいくらか。
            \item 次に,台が床の上で摩擦なく自由に動くことができるようにした。台が静止した状態で,点AからPを静かに放した。Pが台上の点Bに達したときの,Pの床に対する速度を$v$,台の床に対する速度を$V$とする。ただし,速度は右向きを正とする。
                \begin{enumerate}
                    \item このとき,$v$と$V$が満たすべき関係式を2つ書け。
                    \item $v$と$V$を求め,それぞれ,$h$,$m$,$M$,$g$で表せ。
                    \item Pは点Bを通り過ぎたのち,やがて台に対して静止した。このとき,台の床に対する運動はどうなるか。次のうちから選べ。
                    \begin{enumerate}[m]
                        \item Pが静止しても,台は動くがその進方向は点Pの高さ$h$によって決まる。
                        \item Pと台の間の摩擦により,Pが停止しても台は右向きに進む。
                        \item Pが曲面を下っている間は,台は小物体と反対方向に進むので,Pが停止しても,慣性の法則により台は左方向に進む。
                        \item Pと台を合わせた全体には水平方向に外力が働かないため,Pが台に対して停止すると台も停止する。
                    \end{enumerate}
                \end{enumerate}
        \end{enumerate}
    \end{mawarikomi}
% \newpage

\vfill
\end{enumerate}
\end{document}