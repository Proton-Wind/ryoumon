\item
    \begin{mawarikomi}{150pt}{
        \input{./fig/fig144.tex}
    }
    X 線の粒子性は,コンプトン効果の実験からわかる。静止している質量 $m$ の電子に波長 $\lambda$ の X 線光子をあて,電子を角度 $\phi$ の方向に速さ $v$ ではね飛ばす。散乱 X 線は波長が $\lambda'$ となり,角度 $\theta$ の方向に進む。光速を $c$,プランク定数を $h$ とする。
    \begin{enumerate}
        \item 衝突前後のエネルギー保存則を表す式を示せ。
        \item 入射方向 ($x$ 方向) およびそれと垂直な方向 ($y$ 方向) の運動量保存則を表す式を示せ。
        \item 以上の結果から次の関係式を導け。ただし,$\lambda' \fallingdotseq \lambda$ であり,$\dfrac{\lambda'}{\lambda} + \bunsuu{\lambda}{\lambda'} \fallingdotseq 2$ と近似できる。
$$
\lambda' - \lambda = \bunsuu{h}{mc}(1 - \cos\theta)
$$

        \item $\theta = 90^\circ$ の場合の $\tan\phi$ を $\lambda, \lambda'$ を用いて表せ。
    \end{enumerate}
\end{mawarikomi}
