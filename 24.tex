\item
    \begin{mawarikomi}{150pt}{\begin{zahyou*}[ul=5mm](-1,10)(-1,5)
	% \drawline(-1,-1)(-1,13)(13,13)(13,-1)(-1,-1)
	\def\kakudo{20}
	\calcval{\kakudo *2*3.14159265/360}\TH
	\def\O{(0,0)}
	\def\A{(4.1,0.7)}
	\def\B{(4.1,0)}
	\def\C{(3.5,0)}
	\def\D{(3.5,0.7)}
	\def\E{(9,1)}
	\def\F{(9.5,1)}
	\def\G{(9.5,0)}
	\def\H{(9,0)}
	\def\I{(1,0)}
	\def\Dy{1.353}
	\calcval{\Dy*(1/tan(\TH))}\Dx
	\Kaiten\O\A{\kakudo}\AA
	\Kaiten\O\B{\kakudo}\BB
	\Kaiten\O\C{\kakudo}\CC
	\Kaiten\O\D{\kakudo}\DD
	\Kaiten\O\E{\kakudo}\EE
	\Kaiten\O\F{\kakudo}\FF
	\Kaiten\O\G{\kakudo}\GG
	\Kaiten\O\H{\kakudo}\HH
	\def\Fx{(0.2*(cos(T)+T/3.5))*cos(\TH)-(-0.3*sin(T))*sin(\TH)+\Dx}
	\def\Fy{(0.2*(cos(T)+T/3.5))*sin(\TH)+(-0.3*sin(T))*cos(\TH)+\Dy+0.5}
	\Drawline{\AA\BB\CC\DD\AA}
	\BGurafu\Fx\Fy{3.14}{26*3.14}
	\Suisen\GG\O\G\GGD
	\kandk\I{\kakudo}\GGD{90}\GGDU
	\Nuritubusi*[75]{\EE\FF\GG\HH}
	\Nuritubusi*[75]{\AA\BB\CC\DD}
	\Nuritubusi[0.2]{\I\GGDU\GG\O\I}
	\Drawline{\EE\HH}
	\Drawline{\O\GG}
	\Drawline{\O\GGD}
	\Drawline{\GG\GGD}
	\Put\DD[nw]{P}
	\Kakukigou\G\O\GG<hankei=1.5>(2pt,0.8pt)[l]{$\theta$}
\end{zahyou*}
}
    質量$m$のおもりPを鉛直につるすと$\ell $だけ伸びる軽いばねがある。
    重力加速度の大きさを$g$とする。
    図のように,傾角$\theta $の斜面上で,Pをつけたばねの上端を固定する。
    斜面とPの間の静止摩擦係数を$\mu $,動摩擦係数を$\mu '$とし,ばねが自然の長さに保たれるようにPを手で支えておく。
        \begin{enumerate}
            \item 手を放したとき,Pが動き始めるためには,斜面の傾角$\theta $は,$\alpha $より大きくなければならない。$\tan{\alpha }$を求めよ。
            \item 傾角$\theta (>\alpha )$の斜面上で手を放すとPが動き始めた。ばねの伸びが最大値$x$になったとき,Pの最初の位置から重力の位置エネルギーはいくら減少するか。
            \item (2)において,ばねの弾性エネルギーはいくらか。
            \item ばねの最大の伸び$x$を求めよ。
            \item ばねの伸びが最大になったのち,Pが再び動き始めるためには,$\tan{\theta }$はある値より大きくなければならない。その値を$\mu $,$\mu '$で表せ。
        \end{enumerate}
    \end{mawarikomi}