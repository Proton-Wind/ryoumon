\hakosyokika
\item
    \begin{mawarikomi}(10pt,0pt){150pt}{
        %WinTpicVersion4.32a
{\unitlength 0.1in%
\begin{picture}(24.5000,10.3000)(3.5500,-11.0000)%
% CIRCLE 2 0 3 0 Black White  
% 4 805 550 805 500 805 500 805 600
% 
\special{pn 8}%
\special{ar 805 550 50 50 1.5707963 4.7123890}%
% LINE 2 0 3 0 Black White  
% 2 805 500 2805 500
% 
\special{pn 8}%
\special{pa 805 500}%
\special{pa 2805 500}%
\special{fp}%
% LINE 2 0 3 0 Black White  
% 2 805 600 2805 600
% 
\special{pn 8}%
\special{pa 805 600}%
\special{pa 2805 600}%
\special{fp}%
% CIRCLE 2 0 3 0 Black White  
% 4 405 950 405 900 405 900 405 1000
% 
\special{pn 8}%
\special{ar 405 950 50 50 1.5707963 4.7123890}%
% LINE 2 0 3 0 Black White  
% 2 405 900 2405 900
% 
\special{pn 8}%
\special{pa 405 900}%
\special{pa 2405 900}%
\special{fp}%
% LINE 2 0 3 0 Black White  
% 2 405 1000 2405 1000
% 
\special{pn 8}%
\special{pa 405 1000}%
\special{pa 2405 1000}%
\special{fp}%
% BOX 2 0 3 0 Black White  
% 2 705 1000 805 1100
% 
\special{pn 8}%
\special{pa 705 1000}%
\special{pa 805 1000}%
\special{pa 805 1100}%
\special{pa 705 1100}%
\special{pa 705 1000}%
\special{pa 805 1000}%
\special{fp}%
% POLYGON 2 0 2 0 Black White  
% 5 805 1000 1405 300 1405 400 805 1100 805 1000
% 
\special{pn 0}%
\special{sh 0}%
\special{pa 805 1000}%
\special{pa 1405 300}%
\special{pa 1405 400}%
\special{pa 805 1100}%
\special{pa 805 1000}%
\special{ip}%
\special{pn 8}%
\special{pa 805 1000}%
\special{pa 1405 300}%
\special{pa 1405 400}%
\special{pa 805 1100}%
\special{pa 805 1000}%
\special{pa 1405 300}%
\special{fp}%
% POLYGON 2 0 2 0 Black White  
% 5 705 1000 1305 300 1405 300 805 1000 705 1000
% 
\special{pn 0}%
\special{sh 0}%
\special{pa 705 1000}%
\special{pa 1305 300}%
\special{pa 1405 300}%
\special{pa 805 1000}%
\special{pa 705 1000}%
\special{ip}%
\special{pn 8}%
\special{pa 705 1000}%
\special{pa 1305 300}%
\special{pa 1405 300}%
\special{pa 805 1000}%
\special{pa 705 1000}%
\special{pa 1305 300}%
\special{fp}%
% BOX 2 0 3 0 Black White  
% 2 1705 1000 1805 1100
% 
\special{pn 8}%
\special{pa 1705 1000}%
\special{pa 1805 1000}%
\special{pa 1805 1100}%
\special{pa 1705 1100}%
\special{pa 1705 1000}%
\special{pa 1805 1000}%
\special{fp}%
% POLYGON 2 0 2 0 Black White  
% 5 1805 1000 2405 300 2405 400 1805 1100 1805 1000
% 
\special{pn 0}%
\special{sh 0}%
\special{pa 1805 1000}%
\special{pa 2405 300}%
\special{pa 2405 400}%
\special{pa 1805 1100}%
\special{pa 1805 1000}%
\special{ip}%
\special{pn 8}%
\special{pa 1805 1000}%
\special{pa 2405 300}%
\special{pa 2405 400}%
\special{pa 1805 1100}%
\special{pa 1805 1000}%
\special{pa 2405 300}%
\special{fp}%
% POLYGON 2 0 2 0 Black White  
% 5 1705 1000 2305 300 2405 300 1805 1000 1705 1000
% 
\special{pn 0}%
\special{sh 0}%
\special{pa 1705 1000}%
\special{pa 2305 300}%
\special{pa 2405 300}%
\special{pa 1805 1000}%
\special{pa 1705 1000}%
\special{ip}%
\special{pn 8}%
\special{pa 1705 1000}%
\special{pa 2305 300}%
\special{pa 2405 300}%
\special{pa 1805 1000}%
\special{pa 1705 1000}%
\special{pa 2305 300}%
\special{fp}%
% VECTOR 2 0 3 0 Black White  
% 2 1505 700 1505 100
% 
\special{pn 8}%
\special{pa 1505 700}%
\special{pa 1505 100}%
\special{fp}%
\special{sh 1}%
\special{pa 1505 100}%
\special{pa 1485 167}%
\special{pa 1505 153}%
\special{pa 1525 167}%
\special{pa 1505 100}%
\special{fp}%
% VECTOR 2 0 3 0 Black White  
% 2 2505 700 2505 100
% 
\special{pn 8}%
\special{pa 2505 700}%
\special{pa 2505 100}%
\special{fp}%
\special{sh 1}%
\special{pa 2505 100}%
\special{pa 2485 167}%
\special{pa 2505 153}%
\special{pa 2525 167}%
\special{pa 2505 100}%
\special{fp}%
% STR 2 0 3 0 Black White  
% 4 1540 150 1540 200 2 0 0 0
% $B$
\put(15.4000,-2.0000){\makebox(0,0)[lb]{$B$}}%
% STR 2 0 3 0 Black White  
% 4 2540 150 2540 200 2 0 0 0
% $B$
\put(25.4000,-2.0000){\makebox(0,0)[lb]{$B$}}%
% VECTOR 2 0 3 0 Black White  
% 4 540 900 840 600 840 600 540 900
% 
\special{pn 8}%
\special{pa 540 900}%
\special{pa 840 600}%
\special{fp}%
\special{sh 1}%
\special{pa 840 600}%
\special{pa 779 633}%
\special{pa 802 638}%
\special{pa 807 661}%
\special{pa 840 600}%
\special{fp}%
\special{pa 840 600}%
\special{pa 540 900}%
\special{fp}%
\special{sh 1}%
\special{pa 540 900}%
\special{pa 601 867}%
\special{pa 578 862}%
\special{pa 573 839}%
\special{pa 540 900}%
\special{fp}%
% STR 2 0 3 0 Black White  
% 4 500 690 500 740 5 0 0 0
% $d$
\put(5.0000,-7.4000){\makebox(0,0){$d$}}%
% VECTOR 2 0 3 0 Black White  
% 2 2135 780 2435 780
% 
\special{pn 8}%
\special{pa 2135 780}%
\special{pa 2435 780}%
\special{fp}%
\special{sh 1}%
\special{pa 2435 780}%
\special{pa 2368 760}%
\special{pa 2382 780}%
\special{pa 2368 800}%
\special{pa 2435 780}%
\special{fp}%
% STR 2 0 3 0 Black White  
% 4 2400 795 2400 845 5 0 0 0
% $v_0$
\put(24.0000,-8.4500){\makebox(0,0){$v_0$}}%
% VECTOR 2 0 3 0 Black White  
% 2 2140 745 2265 610
% 
\special{pn 8}%
\special{pa 2140 745}%
\special{pa 2265 610}%
\special{fp}%
\special{sh 1}%
\special{pa 2265 610}%
\special{pa 2205 645}%
\special{pa 2229 649}%
\special{pa 2234 673}%
\special{pa 2265 610}%
\special{fp}%
% STR 2 0 3 0 Black White  
% 4 2290 630 2290 680 5 0 0 0
% b
\put(22.9000,-6.8000){\makebox(0,0){b}}%
% VECTOR 2 0 3 0 Black White  
% 2 1990 625 1880 755
% 
\special{pn 8}%
\special{pa 1990 625}%
\special{pa 1880 755}%
\special{fp}%
\special{sh 1}%
\special{pa 1880 755}%
\special{pa 1938 717}%
\special{pa 1914 714}%
\special{pa 1908 691}%
\special{pa 1880 755}%
\special{fp}%
% STR 2 0 3 0 Black White  
% 4 1870 605 1870 655 5 0 0 0
% a
\put(18.7000,-6.5500){\makebox(0,0){a}}%
% STR 2 0 3 0 Black White  
% 4 1195 245 1195 295 5 0 0 0
% Q
\put(11.9500,-2.9500){\makebox(0,0){Q}}%
% STR 2 0 3 0 Black White  
% 4 2190 240 2190 290 5 0 0 0
% P
\put(21.9000,-2.9000){\makebox(0,0){P}}%
\end{picture}}%

    }
    鉛直上向きで磁束密度の大きさ$B$の一様な磁場中に,十分長い2本の金属レールが水平面内に間隔$d$で平行に固定されている。その上に導体棒P,Qをのせ,静止させた。PとQは質量はともに$m$,単位長さあたりの抵抗値は$r$である。導体棒はレールと垂直を保ったままレール上を摩擦なく動くものとする。また,自己誘導の影響と,レールの電気抵抗は無視できる。時刻$t=0$にPにのみ,右向きの初速度$v_0$を与えた。
        \begin{enumerate}
            \item Pが動き出した直後に,Pを流れる電流の向きと大きさ$I_0$を求めよ。向きは図のaかbで答えよ。
            \item Pが動き始めると,Qも動き始めた。PとQが磁場から受ける力の大きさは等しいか,異なるか。また力の向きは同じか,反対か。
            \item Pが動き始めた後の,PとQの速度(右向きを正)の時間変化のグラフを描け。概略でよいので,Pは実線,Qは点線で一つのグラフに描け。また,Pの終端速度$v_\mathrm{f}$を求めよ。さらに,Pの実線のグラフに対して,$t=0$での接線の傾きを求めよ。
        \end{enumerate}
    \end{mawarikomi}