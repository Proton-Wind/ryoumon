\hakosyokika
\item
    \begin{mawarikomi}(10pt,0pt){150pt}{
        \input{./fig/fig132.tex}
    }
    鉛直上向きで磁束密度の大きさ$B$の一様な磁場中に,十分長い2本の金属レールが水平面内に間隔$d$で平行に固定されている。その上に導体棒P,Qをのせ,静止させた。PとQは質量はともに$m$,単位長さあたりの抵抗値は$r$である。導体棒はレールと垂直を保ったままレール上を摩擦なく動くものとする。また,自己誘導の影響と,レールの電気抵抗は無視できる。時刻$t=0$にPにのみ,右向きの初速度$v_0$を与えた。
        \begin{enumerate}
            \item Pが動き出した直後に,Pを流れる電流の向きと大きさ$I_0$を求めよ。向きは図のaかbで答えよ。
            \item Pが動き始めると,Qも動き始めた。PとQが磁場から受ける力の大きさは等しいか,異なるか。また力の向きは同じか,反対か。
            \item Pが動き始めた後の,PとQの速度(右向きを正)の時間変化のグラフを描け。概略でよいので,Pは実線,Qは点線で一つのグラフに描け。また,Pの終端速度$v_\mathrm{f}$を求めよ。さらに,Pの実線のグラフに対して,$t=0$での接線の傾きを求めよ。
        \end{enumerate}
    \end{mawarikomi}