\item
\begin{mawarikomi}{50pt}{%WinTpicVersion4.32a
{\unitlength 0.1in%
\begin{picture}(10.5700,14.0000)(2.7000,-18.0000)%
% CIRCLE 2 0 3 0 Black White  
% 4 800 800 1200 800 1200 800 1200 800
% 
\special{pn 8}%
\special{ar 800 800 400 400 0.0000000 6.2831853}%
% LINE 2 0 3 0 Black White  
% 2 800 1200 800 1600
% 
\special{pn 8}%
\special{pa 800 1200}%
\special{pa 800 1600}%
\special{fp}%
% CIRCLE 2 0 1 0 Black Black  
% 4 800 1700 900 1700 900 1700 900 1700
% 
\special{sh 0.300}%
\special{ia 800 1700 100 100 0.0000000 6.2831853}%
\special{pn 8}%
\special{ar 800 1700 100 100 0.0000000 6.2831853}%
% STR 2 0 3 0 Black White  
% 4 600 1600 600 1700 5 0 0 0
% A
\put(6.0000,-17.0000){\makebox(0,0){A}}%
% STR 2 0 3 0 Black White  
% 4 300 700 300 800 5 0 0 0
% B
\put(3.0000,-8.0000){\makebox(0,0){B}}%
% VECTOR 2 0 3 0 Black White  
% 2 1300 1000 1300 600
% 
\special{pn 8}%
\special{pa 1300 1000}%
\special{pa 1300 600}%
\special{fp}%
\special{sh 1}%
\special{pa 1300 600}%
\special{pa 1280 667}%
\special{pa 1300 653}%
\special{pa 1320 667}%
\special{pa 1300 600}%
\special{fp}%
% STR 2 0 3 0 Black White  
% 4 1300 430 1300 530 5 0 0 0
% $v$
\put(13.0000,-5.3000){\makebox(0,0){$v$}}%
\end{picture}}%
}
質量$M$の気球B(内部の気体も含む)が,質量$m$の小物体Aを質量の無視できる糸でつるして,一定の速さ$v$で上昇している。重力加速度を$g$とし,空気の抵抗および物体Aにはたらく浮力は無視できるものとする。
    \begin{Enumerate}
        \item 糸の張力$T$はいくらか。
        \item 気球Bにはたらく浮力$F$はいくらか。また,外部の空気の密度を$\rho$とすると,気体の体積$V$はいくらか。
    \end{Enumerate}
    物体Aが地面から$h$の高さになったとき,糸を切断した。
    \begin{Enumerate*}
        \item Aが地面に到達するまでに要する時間$t_0$はいくらか。
        \item 糸が切断された後,気球がさらに$h$だけ上がったときの気球の速さ$v_1$はいくらか。
    \end{Enumerate*}
\end{mawarikomi}
