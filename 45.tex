\hakosyokika
\item
    \begin{mawarikomi}{150pt}{\begin{zahyou*}[ul=5mm](-2,11)(-7.5,2)
    \small
    \def\Fx{0.2*(cos(T)+T/3.5)}
	\def\Fy{-0.3*sin(T)+0.6}
	\BGurafu\Fx\Fy{$pi}{20*$pi}
    \drawline(0,2)(0,0)(11,0)
    \def\O{(0,0)}
    \def\A{(3.8,0)}
    \def\B{(3.8,1.5)}
    \def\C{(4.8,1.5)}
    \def\D{(4.8,0)}
    \def\E{(6.5,1.5)}
    \def\F{(6.5,0)}
    \def\G{(5.5,0)}
    \Nuritubusi[0.3]{\A\B\C\D\A}
    \Nuritubusi[0.5]{\D\C\E\F\D}
    \Drawline{\A\B\E\F\A}
    \Drawline{\C\D}
    \Put\B[ne]{A}
    \Put\C[ne]{B}
    \HenKo<henkotype=parallel,
    henkoH=4ex,
    yazirusi=b,
    henkosideb=0,
    henkosidet=1.5>\O\G{$\ell _0$}
    \HenKo<henkotype=parallel,
    % henkoH=11ex,
    yazirusi=b,
    henkosideb=0,
    henkosidet=1.5>\A\G{$x_0$}
    {\def\sensyu{\dashline[40]{0.1}}
    \put(1,-7){\kousi{8}{4}}}
    \put(1,-5){\yasen(9,0)}
    \put(1,-7){\yasen(0,5)}
    \put(0.3,-5.15){O}
    \put(0,-4.15){$\bunsuu{x_0}{2}$}
    \put(0.2,-3.15){$x_0$}
    \put(-0.5,-6.15){$-\bunsuu{x_0}{2}$}
    \put(-0.3,-7.15){$-x_0$}
    \put(3,-5.8){$\bunsuu{T}{2}$}
    \put(5,-5.8){T}
    \put(7,-5.8){$\bunsuu{3}{2}T$}
    \put(9,-5.8){$2T$}
    \put(9.5,-4.8){$t$}
    \put(1.2,-2.5){$x$}
    \drawline(3,-4.8)(3,-5.2)
    \drawline(5,-4.8)(5,-5.2)
    \drawline(7,-4.8)(7,-5.2)
    \drawline(9,-4.8)(9,-5.2)

    % \def\Fx{1.2+2*X-X*X*0.48}
    % \Kaiten\O\AU{-30}\P
    % \Kaiten\O\AU{-50}\PP
    % \Kaiten\O\A{-50}\PPP
    % \Nuritubusi*{\L\LD\RD\R\L}
    % \Drawline{\L\R}
    % \Hasen{\O\A}
    % \Hasen{\O\P}
    % \Hasen{\O\PPP}
    % \YGurafu(*)(0.05)\Fx{3.2}{4.7}
    % \Kakukigou\P\O\A<Hankei=10mm>(0pt,4pt)[l]{$\theta $}
    % \Kakukigou\PP\O\A<Hankei=5mm>(-1.8pt,4pt)[l]{$\theta _0$}
    % \HenKo\A\O{$a$}
    % \En*[1]\P{0.2}
    % \En<hasen=[3][2.7]>\PP{0.2}
    % \Enko\O{4}{0}{180}
    % \Put\A[n]{A}
    % \Put\P[ne]{P}
    % \Put\O[s]{O}
    % \Put\Q[s]{Q}
\end{zahyou*}
}
        物体AとBがあり,質量はそれぞれ$m$と$3m$である。なめらかで水平な床の上で,一端を壁にとめた軽いばねの他端にAをつなぎ,離れないようにする。次に,BをAに接触させて,ばねを自然長$\ell _0$より$x_0$だけ押し縮め,静かに手を離した。ばね定数を$k$とする。
        \begin{enumerate}
            \item 手を離したあと,はじめAとBとはいっしょに運動する。
            \begin{enumerate}
                \item ばねの長さが$\ell $のときの運動方程式をA,Bそれぞれについて記せ。ただし,加速度を$a$とし,AがBを押す力を$N$とする。
                \item $N$を$\ell $,$\ell _0$,$k$を用いて表し,BがAから離れるときの$\ell $を求めよ。
            \end{enumerate}
            \item Aから離れたあとのBの速さ$v$はいくらか。
            \item Bが離れたあと,ばねの最大の長さ$\ell _\mathrm{m}$はいくらか。
            \item 自然長からのばねの伸びを$x$とし,$x$の変化を時間$t$についてグラフに描け。(だだし,図中の$T$は$T=2\pi \sqrt{\bunsuu{m}{k}}$であり,$t=0$のとき,$x=-x_0$である)。
        \end{enumerate}
    \end{mawarikomi}