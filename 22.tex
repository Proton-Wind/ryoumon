\item
    \begin{mawarikomi}{180pt}{\begin{zahyou*}[ul=5mm](-1,13)(-1,5)

	% \drawline(-1,-1)(-1,13)(13,13)(13,-1)(-1,-1)
	\def\Fx{0.2*(cos(T)+T/3.5)+0.25}
	\def\Fy{-0.3*sin(T)+0.4}
	\BGurafu\Fx\Fy{3.14}{20*3.14}
	\drawline(0,1)(0,0)
	\drawline(0,0)(13,0)
	\drawline(4,0.4)(3.8,0.4)
	\def\A{(8,0)}
	\def\AD{(8,-0.5)}
	\def\B{(11.5,0)}
	\def\C{(13,0)}
	\Kaiten\A\C{30}\CC
	\def\CD{(13,-0.5)}
	\def\P{(4,0.8)}
	\def\PD{(4,0.1)}
	\def\PR{(6,0.8)}
	\def\OU{(0,0.8)}
	\Drawlines{\P\PD}
	\Hasen{\A\CC}
	\En*[0]{(4.3,0.3)}{0.3}
	% \scriptsize
	\hasen(6,1)(6,0)
	\HenKo<henkotype=parallel
			,henkoH=4ex
			,yazirusi=b
			,henkosideb=0.3
			,henkosidet=1.5>\PR\OU{自然長}
	\HenKo<henkotype=parallel
			,henkoH=4.6ex
			,yazirusi=b
			,henkosideb=0.8
			,henkosidet=1.2>\P\PR{$a$}
	\Put\B(0pt,-10pt)[t]{B}
	\Put\A(0pt,-10pt)[t]{A}
	\Put\P(0pt,5pt)[l]{P}
	\Nuritubusi*{\A\C\CD\AD\A}
	\Kakukigou\C\A\CC<1.8>(2pt,1pt)[l]{30\Deg}
\end{zahyou*}
}
        水平に置かれたばね定数$k$\tanni{N/m}の軽いばねに質量$m$\tanni{kg}の小球Pを押し当て,ばねを自然長から$a$\tanni{m}だけ縮ませ,静かにPを放した。水平面は図の点Aより左側はなめらかであるが,右側は粗く,Pとの動摩擦係数は$\mu $である。重力加速度の大きさを$g$\tanni{m/s^2}とする。
        \begin{enumerate}
            \item ばねから離れたPが点Aに達するときの速さ$v$\tanni{m/s}を求めよ。
            \item ばねの縮みが$\bunsuu{1}{2}a$\tanni{m}であったときの,Pの速さ$v$を求めよ。
            \item はじめにばねを自然長から$a$\tanni{m}だけ縮ませるのに必要であった外力の仕事$W$を求めよ。
            \item 点Aを通り過ぎたPはやがて点Bで静止した。距離ABを$v$を用いて求めよ。
            \item 粗い面が水平から30\Deg 傾いた斜面(図の点線)であった場合に,Pが達する最高点をCとし,距離ACを$v$を用いて求めよ。斜面と水平面はなだらかにつながるものとする。
        \end{enumerate}
    \end{mawarikomi}