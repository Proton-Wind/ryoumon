\hakosyokika
\item
    \begin{mawarikomi}{150pt}{\begin{zahyou*}[ul=6mm](-10,4)(0,7)
    \small
	\def\kakudo{-30}
	\calcval{\kakudo *2*$pi/360}\TH
	\def\O{(0,0)}
    \def\A{(-4.35,0.4)}
    \def\B{(-8,0.4)}
    \def\AL{(-4.75,0)}
    \def\BR{(-7.6,0)}
    \def\C{(-4.75,1.4)}
    \def\CU{(-4.75,1.8)}
    \def\D{(-7,1.6)}
    \def\E{(-0.8,1.6)}
	\def\G{(-9.5,0)}
    \def\J{(0,1)}
	\def\Dy{-0.1}
	\calcval{\Dy*(1/tan(\TH))}\Dx
	\Kaiten\O\C{\kakudo}\CC
	\Kaiten\O\CU{\kakudo}\CCU
	\Kaiten\O\D{\kakudo}\DD
	\Kaiten\O\E{\kakudo}\EE
	\Kaiten\O\A{\kakudo}\AA
	\Kaiten\O\B{\kakudo}\BB
	\Kaiten\O\AL{\kakudo}\AAL
	\Kaiten\O\BR{\kakudo}\BBR
	\Kaiten\O\G{\kakudo}\GG
	\Kaiten\O\J{\kakudo}\JJ
	\def\Fx{(-0.16*(cos(T)+T/3.5))*cos(\TH)-(-0.3*sin(T))*sin(\TH)+\Dx}
	\def\Fy{(-0.16*(cos(T)+T/3.5))*sin(\TH)+(-0.3*sin(T))*cos(\TH)+\Dy+0.5}
    \hasen(-9,0)(1,0)
    \Drawline{\CC\CCU}
    \Arrowline{\DD\EE}
    \Put\CCU{O}
    \Put\EE{$x$}
    \Put\AA(3pt,10pt)[l]{$M$}
    \Put\BB(-1pt,10pt)[r]{$m$}
    \Put\AAL[sw]{A}
    \Put\BBR[sw]{B}
	\BGurafu\Fx\Fy{3.14159}{26*3.14159}
    \En**\AA{0.4}
    \En\AA{0.4}
    \En**\BB{0.4}
    \En\BB{0.4}
	\Drawline{\JJ\O\GG}
	\Kakukigou\GG\O\G<hankei=1.5>(-2pt,0.8pt)[r]{30\Deg}
    \HenKo<henkotype=parallel,
    henkoH=4ex,
    yazirusi=b,
    henkosideb=0,
    henkosidet=1.1>\AAL\BBR{$d$}
\end{zahyou*}
}
        傾角30\Deg のなめらかな斜面上に,質量$M$の小球Aが壁と軽いばね(ばね定数$k$)で結ばれ,静止している。質量$m$($<M$)の小球BをAから距離$d$だけ離して静かに置いたところ,斜面上を滑り降り,Aと弾性衝突した。斜面に平行に$x$軸をとり,初めのAの位置を原点($x=0$)とし,重力加速度の大きさを$g$とする。
        \begin{enumerate}
            \item 衝突直前のBの速度$u$を求めよ。
            \item 衝突直後のA,Bの速度$v_\mathrm{A}$,$v_\mathrm{B}$を求めよ。
            \item Aが達する最下点の座標$x_0$を求め,$v_\mathrm{A}$,$M$,$k$を用いて表せ。ただし,Aが再び原点に戻るまでの間にBとの衝突は起こらないものとする。
            \item Aが$x=\bunsuu{1}{2}x_0$を通るときの速さ$w$を求め,$v_\mathrm{A}$で表せ。
            \item Aが初めて原点に戻ったとき,Bと2回目の衝突をするためには$d$をいくらにすればよいか。$M$,$m$,$k$,$g$で表せ。
        \end{enumerate}
    \end{mawarikomi}