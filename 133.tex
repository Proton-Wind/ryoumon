\hakosyokika
\item
    \begin{mawarikomi}(10pt,0pt){150pt}{
        %WinTpicVersion4.32a
{\unitlength 0.1in%
\begin{picture}(19.7800,14.5100)(3.4700,-15.7800)%
% CIRCLE 2 0 0 0 Black Black  
% 4 400 998 425 998 425 998 425 998
% 
\special{sh 0.400}%
\special{ia 400 998 25 25 0.0000000 6.2831853}%
\special{pn 8}%
\special{ar 400 998 25 25 0.0000000 6.2831853}%
% CIRCLE 2 0 0 0 Black Black  
% 4 500 998 525 998 525 998 525 998
% 
\special{sh 0.400}%
\special{ia 500 998 25 25 0.0000000 6.2831853}%
\special{pn 8}%
\special{ar 500 998 25 25 0.0000000 6.2831853}%
% CIRCLE 2 0 0 0 Black Black  
% 4 600 998 625 998 625 998 625 998
% 
\special{sh 0.400}%
\special{ia 600 998 25 25 0.0000000 6.2831853}%
\special{pn 8}%
\special{ar 600 998 25 25 0.0000000 6.2831853}%
% CIRCLE 2 0 0 0 Black Black  
% 4 700 998 725 998 725 998 725 998
% 
\special{sh 0.400}%
\special{ia 700 998 25 25 0.0000000 6.2831853}%
\special{pn 8}%
\special{ar 700 998 25 25 0.0000000 6.2831853}%
% CIRCLE 2 0 0 0 Black Black  
% 4 800 998 825 998 825 998 825 998
% 
\special{sh 0.400}%
\special{ia 800 998 25 25 0.0000000 6.2831853}%
\special{pn 8}%
\special{ar 800 998 25 25 0.0000000 6.2831853}%
% CIRCLE 2 0 0 0 Black Black  
% 4 900 998 925 998 925 998 925 998
% 
\special{sh 0.400}%
\special{ia 900 998 25 25 0.0000000 6.2831853}%
\special{pn 8}%
\special{ar 900 998 25 25 0.0000000 6.2831853}%
% CIRCLE 2 0 0 0 Black Black  
% 4 900 998 925 998 925 998 925 998
% 
\special{sh 0.400}%
\special{ia 900 998 25 25 0.0000000 6.2831853}%
\special{pn 8}%
\special{ar 900 998 25 25 0.0000000 6.2831853}%
% CIRCLE 2 0 0 0 Black Black  
% 4 1000 998 1025 998 1025 998 1025 998
% 
\special{sh 0.400}%
\special{ia 1000 998 25 25 0.0000000 6.2831853}%
\special{pn 8}%
\special{ar 1000 998 25 25 0.0000000 6.2831853}%
% CIRCLE 2 0 0 0 Black Black  
% 4 1100 998 1125 998 1125 998 1125 998
% 
\special{sh 0.400}%
\special{ia 1100 998 25 25 0.0000000 6.2831853}%
\special{pn 8}%
\special{ar 1100 998 25 25 0.0000000 6.2831853}%
% CIRCLE 2 0 0 0 Black Black  
% 4 1200 998 1225 998 1225 998 1225 998
% 
\special{sh 0.400}%
\special{ia 1200 998 25 25 0.0000000 6.2831853}%
\special{pn 8}%
\special{ar 1200 998 25 25 0.0000000 6.2831853}%
% CIRCLE 2 0 0 0 Black Black  
% 4 1300 998 1325 998 1325 998 1325 998
% 
\special{sh 0.400}%
\special{ia 1300 998 25 25 0.0000000 6.2831853}%
\special{pn 8}%
\special{ar 1300 998 25 25 0.0000000 6.2831853}%
% CIRCLE 2 0 0 0 Black Black  
% 4 1400 998 1425 998 1425 998 1425 998
% 
\special{sh 0.400}%
\special{ia 1400 998 25 25 0.0000000 6.2831853}%
\special{pn 8}%
\special{ar 1400 998 25 25 0.0000000 6.2831853}%
% CIRCLE 2 0 0 0 Black Black  
% 4 1500 998 1525 998 1525 998 1525 998
% 
\special{sh 0.400}%
\special{ia 1500 998 25 25 0.0000000 6.2831853}%
\special{pn 8}%
\special{ar 1500 998 25 25 0.0000000 6.2831853}%
% CIRCLE 2 0 0 0 Black Black  
% 4 1600 998 1625 998 1625 998 1625 998
% 
\special{sh 0.400}%
\special{ia 1600 998 25 25 0.0000000 6.2831853}%
\special{pn 8}%
\special{ar 1600 998 25 25 0.0000000 6.2831853}%
% CIRCLE 2 0 0 0 Black Black  
% 4 1700 998 1725 998 1725 998 1725 998
% 
\special{sh 0.400}%
\special{ia 1700 998 25 25 0.0000000 6.2831853}%
\special{pn 8}%
\special{ar 1700 998 25 25 0.0000000 6.2831853}%
% CIRCLE 2 0 0 0 Black Black  
% 4 1800 998 1825 998 1825 998 1825 998
% 
\special{sh 0.400}%
\special{ia 1800 998 25 25 0.0000000 6.2831853}%
\special{pn 8}%
\special{ar 1800 998 25 25 0.0000000 6.2831853}%
% CIRCLE 2 0 0 0 Black Black  
% 4 1900 998 1925 998 1925 998 1925 998
% 
\special{sh 0.400}%
\special{ia 1900 998 25 25 0.0000000 6.2831853}%
\special{pn 8}%
\special{ar 1900 998 25 25 0.0000000 6.2831853}%
% CIRCLE 2 0 0 0 Black Black  
% 4 2000 998 2025 998 2025 998 2025 998
% 
\special{sh 0.400}%
\special{ia 2000 998 25 25 0.0000000 6.2831853}%
\special{pn 8}%
\special{ar 2000 998 25 25 0.0000000 6.2831853}%
% CIRCLE 2 0 0 0 Black Black  
% 4 2100 998 2125 998 2125 998 2125 998
% 
\special{sh 0.400}%
\special{ia 2100 998 25 25 0.0000000 6.2831853}%
\special{pn 8}%
\special{ar 2100 998 25 25 0.0000000 6.2831853}%
% CIRCLE 2 0 0 0 Black Black  
% 4 2200 998 2225 998 2225 998 2225 998
% 
\special{sh 0.400}%
\special{ia 2200 998 25 25 0.0000000 6.2831853}%
\special{pn 8}%
\special{ar 2200 998 25 25 0.0000000 6.2831853}%
% CIRCLE 2 0 0 0 Black Black  
% 4 2300 998 2325 998 2325 998 2325 998
% 
\special{sh 0.400}%
\special{ia 2300 998 25 25 0.0000000 6.2831853}%
\special{pn 8}%
\special{ar 2300 998 25 25 0.0000000 6.2831853}%
% CIRCLE 2 0 0 0 Black Black  
% 4 900 853 925 853 925 853 925 853
% 
\special{sh 0.400}%
\special{ia 900 853 25 25 0.0000000 6.2831853}%
\special{pn 8}%
\special{ar 900 853 25 25 0.0000000 6.2831853}%
% CIRCLE 2 0 0 0 Black Black  
% 4 1000 853 1025 853 1025 853 1025 853
% 
\special{sh 0.400}%
\special{ia 1000 853 25 25 0.0000000 6.2831853}%
\special{pn 8}%
\special{ar 1000 853 25 25 0.0000000 6.2831853}%
% CIRCLE 2 0 0 0 Black Black  
% 4 1100 853 1125 853 1125 853 1125 853
% 
\special{sh 0.400}%
\special{ia 1100 853 25 25 0.0000000 6.2831853}%
\special{pn 8}%
\special{ar 1100 853 25 25 0.0000000 6.2831853}%
% CIRCLE 2 0 0 0 Black Black  
% 4 1200 853 1225 853 1225 853 1225 853
% 
\special{sh 0.400}%
\special{ia 1200 853 25 25 0.0000000 6.2831853}%
\special{pn 8}%
\special{ar 1200 853 25 25 0.0000000 6.2831853}%
% CIRCLE 2 0 0 0 Black Black  
% 4 1300 853 1325 853 1325 853 1325 853
% 
\special{sh 0.400}%
\special{ia 1300 853 25 25 0.0000000 6.2831853}%
\special{pn 8}%
\special{ar 1300 853 25 25 0.0000000 6.2831853}%
% CIRCLE 2 0 0 0 Black Black  
% 4 1400 853 1425 853 1425 853 1425 853
% 
\special{sh 0.400}%
\special{ia 1400 853 25 25 0.0000000 6.2831853}%
\special{pn 8}%
\special{ar 1400 853 25 25 0.0000000 6.2831853}%
% CIRCLE 2 0 0 0 Black Black  
% 4 1500 853 1525 853 1525 853 1525 853
% 
\special{sh 0.400}%
\special{ia 1500 853 25 25 0.0000000 6.2831853}%
\special{pn 8}%
\special{ar 1500 853 25 25 0.0000000 6.2831853}%
% CIRCLE 2 0 0 0 Black Black  
% 4 1600 853 1625 853 1625 853 1625 853
% 
\special{sh 0.400}%
\special{ia 1600 853 25 25 0.0000000 6.2831853}%
\special{pn 8}%
\special{ar 1600 853 25 25 0.0000000 6.2831853}%
% CIRCLE 2 0 0 0 Black Black  
% 4 1700 853 1725 853 1725 853 1725 853
% 
\special{sh 0.400}%
\special{ia 1700 853 25 25 0.0000000 6.2831853}%
\special{pn 8}%
\special{ar 1700 853 25 25 0.0000000 6.2831853}%
% CIRCLE 2 0 0 0 Black Black  
% 4 1800 853 1825 853 1825 853 1825 853
% 
\special{sh 0.400}%
\special{ia 1800 853 25 25 0.0000000 6.2831853}%
\special{pn 8}%
\special{ar 1800 853 25 25 0.0000000 6.2831853}%
% CIRCLE 2 0 0 0 Black Black  
% 4 400 1398 425 1398 425 1398 425 1398
% 
\special{sh 0.400}%
\special{ia 400 1398 25 25 0.0000000 6.2831853}%
\special{pn 8}%
\special{ar 400 1398 25 25 0.0000000 6.2831853}%
% CIRCLE 2 0 0 0 Black Black  
% 4 500 1398 525 1398 525 1398 525 1398
% 
\special{sh 0.400}%
\special{ia 500 1398 25 25 0.0000000 6.2831853}%
\special{pn 8}%
\special{ar 500 1398 25 25 0.0000000 6.2831853}%
% CIRCLE 2 0 0 0 Black Black  
% 4 600 1398 625 1398 625 1398 625 1398
% 
\special{sh 0.400}%
\special{ia 600 1398 25 25 0.0000000 6.2831853}%
\special{pn 8}%
\special{ar 600 1398 25 25 0.0000000 6.2831853}%
% CIRCLE 2 0 0 0 Black Black  
% 4 700 1398 725 1398 725 1398 725 1398
% 
\special{sh 0.400}%
\special{ia 700 1398 25 25 0.0000000 6.2831853}%
\special{pn 8}%
\special{ar 700 1398 25 25 0.0000000 6.2831853}%
% CIRCLE 2 0 0 0 Black Black  
% 4 800 1398 825 1398 825 1398 825 1398
% 
\special{sh 0.400}%
\special{ia 800 1398 25 25 0.0000000 6.2831853}%
\special{pn 8}%
\special{ar 800 1398 25 25 0.0000000 6.2831853}%
% CIRCLE 2 0 0 0 Black Black  
% 4 900 1398 925 1398 925 1398 925 1398
% 
\special{sh 0.400}%
\special{ia 900 1398 25 25 0.0000000 6.2831853}%
\special{pn 8}%
\special{ar 900 1398 25 25 0.0000000 6.2831853}%
% CIRCLE 2 0 0 0 Black Black  
% 4 900 1398 925 1398 925 1398 925 1398
% 
\special{sh 0.400}%
\special{ia 900 1398 25 25 0.0000000 6.2831853}%
\special{pn 8}%
\special{ar 900 1398 25 25 0.0000000 6.2831853}%
% CIRCLE 2 0 0 0 Black Black  
% 4 1000 1398 1025 1398 1025 1398 1025 1398
% 
\special{sh 0.400}%
\special{ia 1000 1398 25 25 0.0000000 6.2831853}%
\special{pn 8}%
\special{ar 1000 1398 25 25 0.0000000 6.2831853}%
% CIRCLE 2 0 0 0 Black Black  
% 4 1100 1398 1125 1398 1125 1398 1125 1398
% 
\special{sh 0.400}%
\special{ia 1100 1398 25 25 0.0000000 6.2831853}%
\special{pn 8}%
\special{ar 1100 1398 25 25 0.0000000 6.2831853}%
% CIRCLE 2 0 0 0 Black Black  
% 4 1200 1398 1225 1398 1225 1398 1225 1398
% 
\special{sh 0.400}%
\special{ia 1200 1398 25 25 0.0000000 6.2831853}%
\special{pn 8}%
\special{ar 1200 1398 25 25 0.0000000 6.2831853}%
% CIRCLE 2 0 0 0 Black Black  
% 4 1300 1398 1325 1398 1325 1398 1325 1398
% 
\special{sh 0.400}%
\special{ia 1300 1398 25 25 0.0000000 6.2831853}%
\special{pn 8}%
\special{ar 1300 1398 25 25 0.0000000 6.2831853}%
% CIRCLE 2 0 0 0 Black Black  
% 4 1400 1398 1425 1398 1425 1398 1425 1398
% 
\special{sh 0.400}%
\special{ia 1400 1398 25 25 0.0000000 6.2831853}%
\special{pn 8}%
\special{ar 1400 1398 25 25 0.0000000 6.2831853}%
% CIRCLE 2 0 0 0 Black Black  
% 4 1500 1398 1525 1398 1525 1398 1525 1398
% 
\special{sh 0.400}%
\special{ia 1500 1398 25 25 0.0000000 6.2831853}%
\special{pn 8}%
\special{ar 1500 1398 25 25 0.0000000 6.2831853}%
% CIRCLE 2 0 0 0 Black Black  
% 4 1600 1398 1625 1398 1625 1398 1625 1398
% 
\special{sh 0.400}%
\special{ia 1600 1398 25 25 0.0000000 6.2831853}%
\special{pn 8}%
\special{ar 1600 1398 25 25 0.0000000 6.2831853}%
% CIRCLE 2 0 0 0 Black Black  
% 4 1700 1398 1725 1398 1725 1398 1725 1398
% 
\special{sh 0.400}%
\special{ia 1700 1398 25 25 0.0000000 6.2831853}%
\special{pn 8}%
\special{ar 1700 1398 25 25 0.0000000 6.2831853}%
% CIRCLE 2 0 0 0 Black Black  
% 4 1800 1398 1825 1398 1825 1398 1825 1398
% 
\special{sh 0.400}%
\special{ia 1800 1398 25 25 0.0000000 6.2831853}%
\special{pn 8}%
\special{ar 1800 1398 25 25 0.0000000 6.2831853}%
% CIRCLE 2 0 0 0 Black Black  
% 4 1900 1398 1925 1398 1925 1398 1925 1398
% 
\special{sh 0.400}%
\special{ia 1900 1398 25 25 0.0000000 6.2831853}%
\special{pn 8}%
\special{ar 1900 1398 25 25 0.0000000 6.2831853}%
% CIRCLE 2 0 0 0 Black Black  
% 4 2000 1398 2025 1398 2025 1398 2025 1398
% 
\special{sh 0.400}%
\special{ia 2000 1398 25 25 0.0000000 6.2831853}%
\special{pn 8}%
\special{ar 2000 1398 25 25 0.0000000 6.2831853}%
% CIRCLE 2 0 0 0 Black Black  
% 4 2100 1398 2125 1398 2125 1398 2125 1398
% 
\special{sh 0.400}%
\special{ia 2100 1398 25 25 0.0000000 6.2831853}%
\special{pn 8}%
\special{ar 2100 1398 25 25 0.0000000 6.2831853}%
% CIRCLE 2 0 0 0 Black Black  
% 4 2200 1398 2225 1398 2225 1398 2225 1398
% 
\special{sh 0.400}%
\special{ia 2200 1398 25 25 0.0000000 6.2831853}%
\special{pn 8}%
\special{ar 2200 1398 25 25 0.0000000 6.2831853}%
% CIRCLE 2 0 0 0 Black Black  
% 4 2300 1398 2325 1398 2325 1398 2325 1398
% 
\special{sh 0.400}%
\special{ia 2300 1398 25 25 0.0000000 6.2831853}%
\special{pn 8}%
\special{ar 2300 1398 25 25 0.0000000 6.2831853}%
% CIRCLE 2 0 0 0 Black Black  
% 4 900 1553 925 1553 925 1553 925 1553
% 
\special{sh 0.400}%
\special{ia 900 1553 25 25 0.0000000 6.2831853}%
\special{pn 8}%
\special{ar 900 1553 25 25 0.0000000 6.2831853}%
% CIRCLE 2 0 0 0 Black Black  
% 4 1000 1553 1025 1553 1025 1553 1025 1553
% 
\special{sh 0.400}%
\special{ia 1000 1553 25 25 0.0000000 6.2831853}%
\special{pn 8}%
\special{ar 1000 1553 25 25 0.0000000 6.2831853}%
% CIRCLE 2 0 0 0 Black Black  
% 4 1100 1553 1125 1553 1125 1553 1125 1553
% 
\special{sh 0.400}%
\special{ia 1100 1553 25 25 0.0000000 6.2831853}%
\special{pn 8}%
\special{ar 1100 1553 25 25 0.0000000 6.2831853}%
% CIRCLE 2 0 0 0 Black Black  
% 4 1200 1553 1225 1553 1225 1553 1225 1553
% 
\special{sh 0.400}%
\special{ia 1200 1553 25 25 0.0000000 6.2831853}%
\special{pn 8}%
\special{ar 1200 1553 25 25 0.0000000 6.2831853}%
% CIRCLE 2 0 0 0 Black Black  
% 4 1300 1553 1325 1553 1325 1553 1325 1553
% 
\special{sh 0.400}%
\special{ia 1300 1553 25 25 0.0000000 6.2831853}%
\special{pn 8}%
\special{ar 1300 1553 25 25 0.0000000 6.2831853}%
% CIRCLE 2 0 0 0 Black Black  
% 4 1400 1553 1425 1553 1425 1553 1425 1553
% 
\special{sh 0.400}%
\special{ia 1400 1553 25 25 0.0000000 6.2831853}%
\special{pn 8}%
\special{ar 1400 1553 25 25 0.0000000 6.2831853}%
% CIRCLE 2 0 0 0 Black Black  
% 4 1500 1553 1525 1553 1525 1553 1525 1553
% 
\special{sh 0.400}%
\special{ia 1500 1553 25 25 0.0000000 6.2831853}%
\special{pn 8}%
\special{ar 1500 1553 25 25 0.0000000 6.2831853}%
% CIRCLE 2 0 0 0 Black Black  
% 4 1600 1553 1625 1553 1625 1553 1625 1553
% 
\special{sh 0.400}%
\special{ia 1600 1553 25 25 0.0000000 6.2831853}%
\special{pn 8}%
\special{ar 1600 1553 25 25 0.0000000 6.2831853}%
% CIRCLE 2 0 0 0 Black Black  
% 4 1700 1553 1725 1553 1725 1553 1725 1553
% 
\special{sh 0.400}%
\special{ia 1700 1553 25 25 0.0000000 6.2831853}%
\special{pn 8}%
\special{ar 1700 1553 25 25 0.0000000 6.2831853}%
% CIRCLE 2 0 0 0 Black Black  
% 4 1800 1553 1825 1553 1825 1553 1825 1553
% 
\special{sh 0.400}%
\special{ia 1800 1553 25 25 0.0000000 6.2831853}%
\special{pn 8}%
\special{ar 1800 1553 25 25 0.0000000 6.2831853}%
% VECTOR 2 0 3 0 Black Black  
% 4 1300 550 900 550 1300 550 1800 550
% 
\special{pn 8}%
\special{pa 1300 550}%
\special{pa 900 550}%
\special{fp}%
\special{sh 1}%
\special{pa 900 550}%
\special{pa 967 570}%
\special{pa 953 550}%
\special{pa 967 530}%
\special{pa 900 550}%
\special{fp}%
\special{pa 1300 550}%
\special{pa 1800 550}%
\special{fp}%
\special{sh 1}%
\special{pa 1800 550}%
\special{pa 1733 530}%
\special{pa 1747 550}%
\special{pa 1733 570}%
\special{pa 1800 550}%
\special{fp}%
% VECTOR 2 0 3 0 Black Black  
% 4 1300 200 400 200 1300 200 2300 200
% 
\special{pn 8}%
\special{pa 1300 200}%
\special{pa 400 200}%
\special{fp}%
\special{sh 1}%
\special{pa 400 200}%
\special{pa 467 220}%
\special{pa 453 200}%
\special{pa 467 180}%
\special{pa 400 200}%
\special{fp}%
\special{pa 1300 200}%
\special{pa 2300 200}%
\special{fp}%
\special{sh 1}%
\special{pa 2300 200}%
\special{pa 2233 180}%
\special{pa 2247 200}%
\special{pa 2233 220}%
\special{pa 2300 200}%
\special{fp}%
% STR 2 0 3 0 Black Black  
% 4 1300 150 1300 200 5 0 1 0
% $\ell$
\put(13.0000,-2.0000){\makebox(0,0){{\colorbox[named]{White}{$\ell$}}}}%
% STR 2 0 3 0 Black Black  
% 4 1300 498 1300 548 5 0 1 0
% $\bunsuu{\ell}{2}$
\put(13.0000,-5.4800){\makebox(0,0){{\colorbox[named]{White}{$\bunsuu{\ell}{2}$}}}}%
% LINE 2 1 3 0 Black Black  
% 2 900 540 900 800
% 
\special{pn 8}%
\special{pa 900 540}%
\special{pa 900 800}%
\special{da 0.015}%
% LINE 2 1 3 0 Black Black  
% 2 1800 540 1800 800
% 
\special{pn 8}%
\special{pa 1800 540}%
\special{pa 1800 800}%
\special{da 0.015}%
% LINE 2 1 3 0 Black Black  
% 2 400 160 400 960
% 
\special{pn 8}%
\special{pa 400 160}%
\special{pa 400 960}%
\special{da 0.015}%
% LINE 2 1 3 0 Black Black  
% 2 2300 160 2300 960
% 
\special{pn 8}%
\special{pa 2300 160}%
\special{pa 2300 960}%
\special{da 0.015}%
% STR 2 0 3 0 Black Black  
% 4 800 800 800 850 5 0 0 0
% B
\put(8.0000,-8.5000){\makebox(0,0){B}}%
% STR 2 0 3 0 Black Black  
% 4 800 1500 800 1550 5 0 0 0
% B
\put(8.0000,-15.5000){\makebox(0,0){B}}%
% STR 2 0 3 0 Black Black  
% 4 400 1145 400 1195 5 0 0 0
% A
\put(4.0000,-11.9500){\makebox(0,0){A}}%
\end{picture}}%

    }
    単位長さ当たりの巻き数$n$,長さ$\ell $,断面積$S$のソレノイドAがあり,その外側に単位長さあたりの巻き数$n$,長さ$\bunsuu{\ell}{2}$のソレノイドBが巻きつけられている。図はソレノイドの中心軸を含む断面である。Aの両端には電源が接続されており,Bには電流は流れない。両ソレノイドは真空中に置かれていて,真空の透磁率を$\mu _0$とする。
        \begin{enumerate}
            \item ソレノイドAに電流$I$を流したとき,Aの内部に生じる磁場の強さ$H$,およびAを貫く磁束$\varPhi$はそれぞれいくらか。
            \item 微小時間$\varDelta t$の間に,Aを流れる電流が$\varDelta I$だけ増加した。
            \begin{enumerate}[(ア)]
                \item Aを貫く磁束の変化$\varDelta \varPhi $はいくらか。
                \item Aに生じる誘導起電力の大きさ$V_1$はいくらか。
                \item Aの自己インダクタンス$L$はいくらか。
                \item Bに生じる誘導起電力の大きさ$V_2$はいくらか。
                \item AとBの間の相互インダクタンス$M$はいくらか。
            \end{enumerate}
        \end{enumerate}
    \end{mawarikomi}