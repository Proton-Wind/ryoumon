\hakosyokika
\item
    \begin{mawarikomi}(20pt,0pt){150pt}{%%% C:/Users/yuich/ketcindy2025Apr19/fig/fig102.tex 
%%% Generator=fig102.cdy 
{\unitlength=0.7cm%
\begin{picture}%
(6.13,9.4)(-1.13,-4.4)%
\special{pn 8}%
%
\special{pa 10 -827}\special{pa 10 -828}\special{pa 10 -829}\special{pa 10 -831}\special{pa 9 -832}%
\special{pa 8 -833}\special{pa 8 -834}\special{pa 7 -835}\special{pa 6 -836}\special{pa 4 -836}%
\special{pa 3 -837}\special{pa 2 -837}\special{pa 1 -837}\special{pa -1 -837}\special{pa -2 -837}%
\special{pa -3 -837}\special{pa -4 -836}\special{pa -6 -836}\special{pa -7 -835}\special{pa -8 -834}%
\special{pa -8 -833}\special{pa -9 -832}\special{pa -10 -831}\special{pa -10 -829}%
\special{pa -10 -828}\special{pa -10 -827}\special{pa -10 -825}\special{pa -10 -824}%
\special{pa -10 -823}\special{pa -9 -822}\special{pa -8 -821}\special{pa -8 -820}%
\special{pa -7 -819}\special{pa -6 -818}\special{pa -4 -817}\special{pa -3 -817}\special{pa -2 -816}%
\special{pa -1 -816}\special{pa 1 -816}\special{pa 2 -816}\special{pa 3 -817}\special{pa 4 -817}%
\special{pa 6 -818}\special{pa 7 -819}\special{pa 8 -820}\special{pa 8 -821}\special{pa 9 -822}%
\special{pa 10 -823}\special{pa 10 -824}\special{pa 10 -825}\special{pa 10 -827}\special{pa 10 -827}%
\special{sh 1}\special{ip}%
\special{pa    10  -827}\special{pa    10  -828}\special{pa    10  -829}\special{pa    10  -831}%
\special{pa     9  -832}\special{pa     8  -833}\special{pa     8  -834}\special{pa     7  -835}%
\special{pa     6  -836}\special{pa     4  -836}\special{pa     3  -837}\special{pa     2  -837}%
\special{pa     1  -837}\special{pa    -1  -837}\special{pa    -2  -837}\special{pa    -3  -837}%
\special{pa    -4  -836}\special{pa    -6  -836}\special{pa    -7  -835}\special{pa    -8  -834}%
\special{pa    -8  -833}\special{pa    -9  -832}\special{pa   -10  -831}\special{pa   -10  -829}%
\special{pa   -10  -828}\special{pa   -10  -827}\special{pa   -10  -825}\special{pa   -10  -824}%
\special{pa   -10  -823}\special{pa    -9  -822}\special{pa    -8  -821}\special{pa    -8  -820}%
\special{pa    -7  -819}\special{pa    -6  -818}\special{pa    -4  -817}\special{pa    -3  -817}%
\special{pa    -2  -816}\special{pa    -1  -816}\special{pa     1  -816}\special{pa     2  -816}%
\special{pa     3  -817}\special{pa     4  -817}\special{pa     6  -818}\special{pa     7  -819}%
\special{pa     8  -820}\special{pa     8  -821}\special{pa     9  -822}\special{pa    10  -823}%
\special{pa    10  -824}\special{pa    10  -825}\special{pa    10  -827}%
\special{fp}%
\special{pa 10 827}\special{pa 10 825}\special{pa 10 824}\special{pa 10 823}\special{pa 9 822}%
\special{pa 8 821}\special{pa 8 820}\special{pa 7 819}\special{pa 6 818}\special{pa 4 817}%
\special{pa 3 817}\special{pa 2 816}\special{pa 1 816}\special{pa -1 816}\special{pa -2 816}%
\special{pa -3 817}\special{pa -4 817}\special{pa -6 818}\special{pa -7 819}\special{pa -8 820}%
\special{pa -8 821}\special{pa -9 822}\special{pa -10 823}\special{pa -10 824}\special{pa -10 825}%
\special{pa -10 827}\special{pa -10 828}\special{pa -10 829}\special{pa -10 831}\special{pa -9 832}%
\special{pa -8 833}\special{pa -8 834}\special{pa -7 835}\special{pa -6 836}\special{pa -4 836}%
\special{pa -3 837}\special{pa -2 837}\special{pa -1 837}\special{pa 1 837}\special{pa 2 837}%
\special{pa 3 837}\special{pa 4 836}\special{pa 6 836}\special{pa 7 835}\special{pa 8 834}%
\special{pa 8 833}\special{pa 9 832}\special{pa 10 831}\special{pa 10 829}\special{pa 10 828}%
\special{pa 10 827}\special{pa 10 827}\special{sh 1}\special{ip}%
\special{pa    10   827}\special{pa    10   825}\special{pa    10   824}\special{pa    10   823}%
\special{pa     9   822}\special{pa     8   821}\special{pa     8   820}\special{pa     7   819}%
\special{pa     6   818}\special{pa     4   817}\special{pa     3   817}\special{pa     2   816}%
\special{pa     1   816}\special{pa    -1   816}\special{pa    -2   816}\special{pa    -3   817}%
\special{pa    -4   817}\special{pa    -6   818}\special{pa    -7   819}\special{pa    -8   820}%
\special{pa    -8   821}\special{pa    -9   822}\special{pa   -10   823}\special{pa   -10   824}%
\special{pa   -10   825}\special{pa   -10   827}\special{pa   -10   828}\special{pa   -10   829}%
\special{pa   -10   831}\special{pa    -9   832}\special{pa    -8   833}\special{pa    -8   834}%
\special{pa    -7   835}\special{pa    -6   836}\special{pa    -4   836}\special{pa    -3   837}%
\special{pa    -2   837}\special{pa    -1   837}\special{pa     1   837}\special{pa     2   837}%
\special{pa     3   837}\special{pa     4   836}\special{pa     6   836}\special{pa     7   835}%
\special{pa     8   834}\special{pa     8   833}\special{pa     9   832}\special{pa    10   831}%
\special{pa    10   829}\special{pa    10   828}\special{pa    10   827}%
\special{fp}%
\special{pa 837 0}\special{pa 837 -1}\special{pa 837 -3}\special{pa 837 -4}\special{pa 836 -5}%
\special{pa 835 -6}\special{pa 834 -7}\special{pa 833 -8}\special{pa 832 -9}\special{pa 831 -9}%
\special{pa 830 -10}\special{pa 829 -10}\special{pa 827 -10}\special{pa 826 -10}\special{pa 825 -10}%
\special{pa 824 -10}\special{pa 822 -9}\special{pa 821 -9}\special{pa 820 -8}\special{pa 819 -7}%
\special{pa 818 -6}\special{pa 818 -5}\special{pa 817 -4}\special{pa 817 -3}\special{pa 816 -1}%
\special{pa 816 0}\special{pa 816 1}\special{pa 817 3}\special{pa 817 4}\special{pa 818 5}%
\special{pa 818 6}\special{pa 819 7}\special{pa 820 8}\special{pa 821 9}\special{pa 822 9}%
\special{pa 824 10}\special{pa 825 10}\special{pa 826 10}\special{pa 827 10}\special{pa 829 10}%
\special{pa 830 10}\special{pa 831 9}\special{pa 832 9}\special{pa 833 8}\special{pa 834 7}%
\special{pa 835 6}\special{pa 836 5}\special{pa 837 4}\special{pa 837 3}\special{pa 837 1}%
\special{pa 837 0}\special{pa 837 0}\special{sh 1}\special{ip}%
\special{pa   837    -0}\special{pa   837    -1}\special{pa   837    -3}\special{pa   837    -4}%
\special{pa   836    -5}\special{pa   835    -6}\special{pa   834    -7}\special{pa   833    -8}%
\special{pa   832    -9}\special{pa   831    -9}\special{pa   830   -10}\special{pa   829   -10}%
\special{pa   827   -10}\special{pa   826   -10}\special{pa   825   -10}\special{pa   824   -10}%
\special{pa   822    -9}\special{pa   821    -9}\special{pa   820    -8}\special{pa   819    -7}%
\special{pa   818    -6}\special{pa   818    -5}\special{pa   817    -4}\special{pa   817    -3}%
\special{pa   816    -1}\special{pa   816     0}\special{pa   816     1}\special{pa   817     3}%
\special{pa   817     4}\special{pa   818     5}\special{pa   818     6}\special{pa   819     7}%
\special{pa   820     8}\special{pa   821     9}\special{pa   822     9}\special{pa   824    10}%
\special{pa   825    10}\special{pa   826    10}\special{pa   827    10}\special{pa   829    10}%
\special{pa   830    10}\special{pa   831     9}\special{pa   832     9}\special{pa   833     8}%
\special{pa   834     7}\special{pa   835     6}\special{pa   836     5}\special{pa   837     4}%
\special{pa   837     3}\special{pa   837     1}\special{pa   837     0}%
\special{fp}%
\settowidth{\Width}{$Q$}\setlength{\Width}{-1\Width}%
\settoheight{\Height}{$Q$}\settodepth{\Depth}{$Q$}\setlength{\Height}{-0.5\Height}\setlength{\Depth}{0.5\Depth}\addtolength{\Height}{\Depth}%
\put( -0.214,  3.000){\hspace*{\Width}\raisebox{\Height}{$Q$}}%
%
\settowidth{\Width}{$Q$}\setlength{\Width}{-1\Width}%
\settoheight{\Height}{$Q$}\settodepth{\Depth}{$Q$}\setlength{\Height}{-0.5\Height}\setlength{\Depth}{0.5\Depth}\addtolength{\Height}{\Depth}%
\put( -0.214, -3.000){\hspace*{\Width}\raisebox{\Height}{$Q$}}%
%
\settowidth{\Width}{A(0,$d$)}\setlength{\Width}{0\Width}%
\settoheight{\Height}{A(0,$d$)}\settodepth{\Depth}{A(0,$d$)}\setlength{\Height}{-0.5\Height}\setlength{\Depth}{0.5\Depth}\addtolength{\Height}{\Depth}%
\put(  0.214,  3.000){\hspace*{\Width}\raisebox{\Height}{A(0,$d$)}}%
%
\settowidth{\Width}{B(0,$-d$)}\setlength{\Width}{0\Width}%
\settoheight{\Height}{B(0,$-d$)}\settodepth{\Depth}{B(0,$-d$)}\setlength{\Height}{-0.5\Height}\setlength{\Depth}{0.5\Depth}\addtolength{\Height}{\Depth}%
\put(  0.214, -3.000){\hspace*{\Width}\raisebox{\Height}{B(0,$-d$)}}%
%
\settowidth{\Width}{C($d$,0)}\setlength{\Width}{-0.5\Width}%
\settoheight{\Height}{C($d$,0)}\settodepth{\Depth}{C($d$,0)}\setlength{\Height}{-\Height}%
\put(  3.000, -0.214){\hspace*{\Width}\raisebox{\Height}{C($d$,0)}}%
%
\special{pa  -311    -0}\special{pa  1378    -0}%
\special{fp}%
\special{pa     0  1212}\special{pa     0 -1378}%
\special{fp}%
\special{pa 1303 24}\special{pa 1378 0}\special{pa 1303 -24}\special{pa 1318 0}\special{pa 1303 24}%
\special{pa 1303 24}\special{sh 1}\special{ip}%
\special{pn 1}%
\special{pa  1303    24}\special{pa  1378    -0}\special{pa  1303   -24}\special{pa  1318    -0}%
\special{pa  1303    24}%
\special{fp}%
\special{pn 8}%
\special{pa 24 -1303}\special{pa 0 -1378}\special{pa -24 -1303}\special{pa 0 -1318}%
\special{pa 24 -1303}\special{pa 24 -1303}\special{sh 1}\special{ip}%
\special{pn 1}%
\special{pa    24 -1303}\special{pa     0 -1378}\special{pa   -24 -1303}\special{pa     0 -1318}%
\special{pa    24 -1303}%
\special{fp}%
\special{pn 8}%
\settowidth{\Width}{$x〔$\sf{m}$〕$}\setlength{\Width}{-0.5\Width}%
\settoheight{\Height}{$x〔$\sf{m}$〕$}\settodepth{\Depth}{$x〔$\sf{m}$〕$}\setlength{\Height}{-\Height}%
\put(  5.000, -0.214){\hspace*{\Width}\raisebox{\Height}{$x〔$\sf{m}$〕$}}%
%
\settowidth{\Width}{$y〔$\sf{m}$〕$}\setlength{\Width}{0\Width}%
\settoheight{\Height}{$y〔$\sf{m}$〕$}\settodepth{\Depth}{$y〔$\sf{m}$〕$}\setlength{\Height}{-0.5\Height}\setlength{\Depth}{0.5\Depth}\addtolength{\Height}{\Depth}%
\put(  0.214,  5.000){\hspace*{\Width}\raisebox{\Height}{$y〔$\sf{m}$〕$}}%
%
\settowidth{\Width}{O}\setlength{\Width}{-1\Width}%
\settoheight{\Height}{O}\settodepth{\Depth}{O}\setlength{\Height}{-\Height}%
\put( -0.071, -0.071){\hspace*{\Width}\raisebox{\Height}{O}}%
%
\end{picture}}%}
           図のように,2つの正電荷$Q$\tanni{C}をそれぞれ点A$(0,d)$,B$(0,-d)$に置いて固定した。
           クーロンの法則の比例定数を$k$\tanni{N \cdot m^2 /C^2}とする。
        \begin{enumerate}
            \item 原点Oおよび点Cでの電場の強さはそれぞれいくらか。
            \item 原点Oおよび点Cの電位$V_\mathrm{O}$,$V_\mathrm{C}$はそれぞれいくらか。ただし,電位の基準は無限遠点とする。
            \item 点C$(d,0)$に正電荷$q$\tanni{C},質量$m$\tanni{kg}の点電荷Pを置くとき,Pが受ける静電気力の大きさはいくらか。また,その向きを答えよ。
            \item Pを点C$(d,0)$から原点Oまで静かに運ぶのに要する仕事$W_1$はいくらか。また,その際,静電気力のする仕事$W_2$はいくらか。
            \item 点C$(d,0)$に点電荷Pを置いて静かに放す。十分に時間が経過した後のPの速さ$v$はいくらか。
        \end{enumerate}
    \end{mawarikomi}
