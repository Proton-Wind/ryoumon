\hakosyokika
\item
    \begin{mawarikomi}{150pt}{\begin{zahyou*}[ul=6mm](0,10)(-1,4)
    \small
    \def\A{(0.5,1.5)}
    \def\B{(0.5,0.2)}
    \def\C{(3,0.2)}
    \def\D{(3.0,1.5)}
    \def\L{(1,0.2)}
    \def\R{(2.5,0.2)}
    \def\P{(1.5,1.7)}
    \def\xvec{(1.5,0)}
    \def\yvec{(0,1.5)}
    \drawline(0,0)(10,0)
    \Drawline{\A\B\C\D\A}
    \En*[0]\L{0.2}
    \En\L{0.2}
    \En*[0]\R{0.2}
    \En\L{0.2}
    \En*\P{0.2}
    \En\P{0.2}
    \Kuromaru{\L;\R}
    {\thicklines
    \put(1.5,1.9){\Yasen\yvec}
    \put(3,0.8){\Yasen\xvec}
    }
    \put(1.5,-0.5){A}
    \put(8,-0.5){B}
    \put(4.7,0.8){$v$}
    \put(1.5,3.5){$u$}
\end{zahyou*}
}
        台車が一定の速度$v$で水平に運動している。台車がA点を通過する瞬間に,台車から台車に対して初速$u$で鉛直上向きにボールを打ち上げたら,ボールはB点に落下した。次に台車を$\bunsuu{1}{2}v$の速度で運動させたとき,台車がA点を通過する瞬間に台車に対して鉛直上向きにボールを打ち上げたら,ボールはやはりB点に落下した。重力加速度の大きさは$9.8$\sftanni{m/s^2}とする。
        \begin{Enumerate}
            \item 2度目にボールを打ち上げた鉛直方向の初速は最初の初速$u$の\Hako 倍である。
            \item このとき,ボールが到達した最高点の高さは最初の場合の\Hako 倍である。
        \end{Enumerate}
        ~~ところで,台車を$5.6$\sftanni{m/s}の速度で運動させて,台車がA点を通過する瞬間に台車から鉛直上向きにボールを打ち上げたら,ボールは$10$\sftanni{m}の高さまで上がって,やはりB点で台車に落下した。
        \begin{Enumerate*}
            \item このとき,ボールを打ち上げた鉛直方向の初速は\Hako \sftanni{m/s}である。
            \item そして,AB間の距離は\Hako \sftanni{m}である。
        \end{Enumerate*}
    \end{mawarikomi}