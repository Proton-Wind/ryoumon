\item
    % \begin{mawarikomi}{150pt}{
    % }
    図1のように,抵抗値$R$の抵抗,電気容量$C$のコンデンサーおよび自己インダクタンス$L$のコイルを直列に接続し,交流電源につないだ回路がある。オシロスコープで抵抗の両端の電圧を観測したところ,図2のような周期T,最大値$V_0$の正弦曲線であった。
    \begin{center}
        %%% C:/vpn/vpn/KeTCindy/fig/fig136_1.tex 
%%% Generator=fig136_1.cdy 
{\unitlength=1cm%
\begin{picture}%
(7.5,5)(-4,-2.5)%
\special{pn 8}%
%
\special{pa   -72   394}\special{pa   -71   392}\special{pa   -71   390}\special{pa   -70   389}%
\special{pa   -69   387}\special{pa   -69   385}\special{pa   -68   383}\special{pa   -67   382}%
\special{pa   -66   380}\special{pa   -66   378}\special{pa   -65   377}\special{pa   -64   375}%
\special{pa   -63   373}\special{pa   -63   372}\special{pa   -62   370}\special{pa   -61   369}%
\special{pa   -61   367}\special{pa   -60   366}\special{pa   -59   364}\special{pa   -58   363}%
\special{pa   -58   361}\special{pa   -57   360}\special{pa   -56   359}\special{pa   -56   357}%
\special{pa   -55   356}\special{pa   -54   355}\special{pa   -53   354}\special{pa   -53   352}%
\special{pa   -52   351}\special{pa   -51   350}\special{pa   -51   349}\special{pa   -50   348}%
\special{pa   -49   347}\special{pa   -48   346}\special{pa   -48   345}\special{pa   -47   345}%
\special{pa   -46   344}\special{pa   -45   343}\special{pa   -45   342}\special{pa   -44   342}%
\special{pa   -43   341}\special{pa   -43   341}\special{pa   -42   340}\special{pa   -41   340}%
\special{pa   -40   340}\special{pa   -40   339}\special{pa   -39   339}\special{pa   -38   339}%
\special{pa   -38   339}\special{pa   -37   339}\special{pa   -36   339}\special{pa   -35   339}%
\special{pa   -35   339}\special{pa   -34   339}\special{pa   -33   339}\special{pa   -32   339}%
\special{pa   -32   340}\special{pa   -31   340}\special{pa   -30   340}\special{pa   -30   341}%
\special{pa   -29   341}\special{pa   -28   342}\special{pa   -27   342}\special{pa   -27   343}%
\special{pa   -26   344}\special{pa   -25   345}\special{pa   -25   345}\special{pa   -24   346}%
\special{pa   -23   347}\special{pa   -22   348}\special{pa   -22   349}\special{pa   -21   350}%
\special{pa   -20   351}\special{pa   -19   352}\special{pa   -19   354}\special{pa   -18   355}%
\special{pa   -17   356}\special{pa   -17   357}\special{pa   -16   359}\special{pa   -15   360}%
\special{pa   -14   361}\special{pa   -14   363}\special{pa   -13   364}\special{pa   -12   366}%
\special{pa   -12   367}\special{pa   -11   369}\special{pa   -10   370}\special{pa    -9   372}%
\special{pa    -9   373}\special{pa    -8   375}\special{pa    -7   377}\special{pa    -6   378}%
\special{pa    -6   380}\special{pa    -5   382}\special{pa    -4   383}\special{pa    -4   385}%
\special{pa    -3   387}\special{pa    -2   389}\special{pa    -1   390}\special{pa    -1   392}%
\special{pa     0   394}\special{pa     1   395}\special{pa     1   397}\special{pa     2   399}%
\special{pa     3   401}\special{pa     4   402}\special{pa     4   404}\special{pa     5   406}%
\special{pa     6   407}\special{pa     6   409}\special{pa     7   411}\special{pa     8   412}%
\special{pa     9   414}\special{pa     9   416}\special{pa    10   417}\special{pa    11   419}%
\special{pa    12   420}\special{pa    12   422}\special{pa    13   423}\special{pa    14   425}%
\special{pa    14   426}\special{pa    15   427}\special{pa    16   429}\special{pa    17   430}%
\special{pa    17   431}\special{pa    18   433}\special{pa    19   434}\special{pa    19   435}%
\special{pa    20   436}\special{pa    21   437}\special{pa    22   438}\special{pa    22   439}%
\special{pa    23   440}\special{pa    24   441}\special{pa    25   442}\special{pa    25   443}%
\special{pa    26   444}\special{pa    27   444}\special{pa    27   445}\special{pa    28   446}%
\special{pa    29   446}\special{pa    30   447}\special{pa    30   447}\special{pa    31   447}%
\special{pa    32   448}\special{pa    32   448}\special{pa    33   448}\special{pa    34   449}%
\special{pa    35   449}\special{pa    35   449}\special{pa    36   449}\special{pa    37   449}%
\special{pa    38   449}\special{pa    38   449}\special{pa    39   448}\special{pa    40   448}%
\special{pa    40   448}\special{pa    41   447}\special{pa    42   447}\special{pa    43   447}%
\special{pa    43   446}\special{pa    44   446}\special{pa    45   445}\special{pa    45   444}%
\special{pa    46   444}\special{pa    47   443}\special{pa    48   442}\special{pa    48   441}%
\special{pa    49   440}\special{pa    50   439}\special{pa    51   438}\special{pa    51   437}%
\special{pa    52   436}\special{pa    53   435}\special{pa    53   434}\special{pa    54   433}%
\special{pa    55   431}\special{pa    56   430}\special{pa    56   429}\special{pa    57   427}%
\special{pa    58   426}\special{pa    58   425}\special{pa    59   423}\special{pa    60   422}%
\special{pa    61   420}\special{pa    61   419}\special{pa    62   417}\special{pa    63   416}%
\special{pa    63   414}\special{pa    64   412}\special{pa    65   411}\special{pa    66   409}%
\special{pa    66   407}\special{pa    67   406}\special{pa    68   404}\special{pa    69   402}%
\special{pa    69   401}\special{pa    70   399}\special{pa    71   397}\special{pa    71   395}%
\special{pa    72   394}%
\special{fp}%
\special{pa   138   394}\special{pa   137   376}\special{pa   133   359}\special{pa   128   343}%
\special{pa   121   327}\special{pa   111   313}\special{pa   100   299}\special{pa    88   288}%
\special{pa    74   277}\special{pa    59   269}\special{pa    43   263}\special{pa    26   258}%
\special{pa     9   256}\special{pa    -9   256}\special{pa   -26   258}\special{pa   -43   263}%
\special{pa   -59   269}\special{pa   -74   277}\special{pa   -88   288}\special{pa  -100   299}%
\special{pa  -111   313}\special{pa  -121   327}\special{pa  -128   343}\special{pa  -133   359}%
\special{pa  -137   376}\special{pa  -138   394}\special{pa  -137   411}\special{pa  -133   428}%
\special{pa  -128   444}\special{pa  -121   460}\special{pa  -111   475}\special{pa  -100   488}%
\special{pa   -88   500}\special{pa   -74   510}\special{pa   -59   518}\special{pa   -43   525}%
\special{pa   -26   529}\special{pa    -9   531}\special{pa     9   531}\special{pa    26   529}%
\special{pa    43   525}\special{pa    59   518}\special{pa    74   510}\special{pa    88   500}%
\special{pa   100   488}\special{pa   111   475}\special{pa   121   460}\special{pa   128   444}%
\special{pa   133   428}\special{pa   137   411}\special{pa   138   394}%
\special{fp}%
\special{pa -1181   394}\special{pa  -138   394}%
\special{fp}%
\special{pa  1181   394}\special{pa   138   394}%
\special{fp}%
\special{pa  -650  -266}\special{pa  -925  -266}\special{pa  -925  -128}\special{pa  -650  -128}%
\special{pa  -650  -266}%
\special{fp}%
\special{pa -1181  -197}\special{pa  -925  -197}%
\special{fp}%
\special{pa  -394  -197}\special{pa  -650  -197}%
\special{fp}%
\special{pa    41  -335}\special{pa    41   -59}%
\special{fp}%
\special{pa   -41  -335}\special{pa   -41   -59}%
\special{fp}%
\special{pa  -394  -197}\special{pa   -41  -197}%
\special{fp}%
\special{pa   394  -197}\special{pa    41  -197}%
\special{fp}%
\special{pa   677  -197}\special{pa   677  -200}\special{pa   677  -204}\special{pa   678  -207}%
\special{pa   678  -211}\special{pa   679  -214}\special{pa   679  -217}\special{pa   680  -220}%
\special{pa   681  -223}\special{pa   681  -226}\special{pa   682  -229}\special{pa   683  -232}%
\special{pa   685  -235}\special{pa   686  -237}\special{pa   687  -239}\special{pa   689  -241}%
\special{pa   690  -243}\special{pa   691  -245}\special{pa   693  -247}\special{pa   695  -248}%
\special{pa   696  -249}\special{pa   698  -250}\special{pa   700  -251}\special{pa   701  -252}%
\special{pa   703  -252}\special{pa   705  -252}\special{pa   706  -252}\special{pa   708  -252}%
\special{pa   710  -251}\special{pa   712  -250}\special{pa   713  -249}\special{pa   715  -248}%
\special{pa   716  -247}\special{pa   718  -245}\special{pa   719  -243}\special{pa   721  -241}%
\special{pa   722  -239}\special{pa   724  -237}\special{pa   725  -235}\special{pa   726  -232}%
\special{pa   727  -229}\special{pa   728  -226}\special{pa   729  -223}\special{pa   730  -220}%
\special{pa   730  -217}\special{pa   731  -214}\special{pa   731  -211}\special{pa   732  -207}%
\special{pa   732  -204}\special{pa   732  -200}\special{pa   732  -197}\special{pa   732  -200}%
\special{pa   733  -204}\special{pa   733  -207}\special{pa   733  -211}\special{pa   734  -214}%
\special{pa   734  -217}\special{pa   735  -220}\special{pa   736  -223}\special{pa   737  -226}%
\special{pa   738  -229}\special{pa   739  -232}\special{pa   740  -235}\special{pa   741  -237}%
\special{pa   742  -239}\special{pa   744  -241}\special{pa   745  -243}\special{pa   747  -245}%
\special{pa   748  -247}\special{pa   750  -248}\special{pa   751  -249}\special{pa   753  -250}%
\special{pa   755  -251}\special{pa   756  -252}\special{pa   758  -252}\special{pa   760  -252}%
\special{pa   762  -252}\special{pa   763  -252}\special{pa   765  -251}\special{pa   767  -250}%
\special{pa   768  -249}\special{pa   770  -248}\special{pa   772  -247}\special{pa   773  -245}%
\special{pa   775  -243}\special{pa   776  -241}\special{pa   777  -239}\special{pa   779  -237}%
\special{pa   780  -235}\special{pa   781  -232}\special{pa   782  -229}\special{pa   783  -226}%
\special{pa   784  -223}\special{pa   785  -220}\special{pa   785  -217}\special{pa   786  -214}%
\special{pa   787  -211}\special{pa   787  -207}\special{pa   787  -204}\special{pa   787  -200}%
\special{pa   787  -197}\special{pa   787  -200}\special{pa   788  -204}\special{pa   788  -207}%
\special{pa   788  -211}\special{pa   789  -214}\special{pa   789  -217}\special{pa   790  -220}%
\special{pa   791  -223}\special{pa   792  -226}\special{pa   793  -229}\special{pa   794  -232}%
\special{pa   795  -235}\special{pa   796  -237}\special{pa   797  -239}\special{pa   799  -241}%
\special{pa   800  -243}\special{pa   802  -245}\special{pa   803  -247}\special{pa   805  -248}%
\special{pa   806  -249}\special{pa   808  -250}\special{pa   810  -251}\special{pa   812  -252}%
\special{pa   813  -252}\special{pa   815  -252}\special{pa   817  -252}\special{pa   818  -252}%
\special{pa   820  -251}\special{pa   822  -250}\special{pa   823  -249}\special{pa   825  -248}%
\special{pa   827  -247}\special{pa   828  -245}\special{pa   830  -243}\special{pa   831  -241}%
\special{pa   833  -239}\special{pa   834  -237}\special{pa   835  -235}\special{pa   836  -232}%
\special{pa   837  -229}\special{pa   838  -226}\special{pa   839  -223}\special{pa   840  -220}%
\special{pa   841  -217}\special{pa   841  -214}\special{pa   842  -211}\special{pa   842  -207}%
\special{pa   842  -204}\special{pa   842  -200}\special{pa   843  -197}\special{pa   843  -200}%
\special{pa   843  -204}\special{pa   843  -207}\special{pa   843  -211}\special{pa   844  -214}%
\special{pa   844  -217}\special{pa   845  -220}\special{pa   846  -223}\special{pa   847  -226}%
\special{pa   848  -229}\special{pa   849  -232}\special{pa   850  -235}\special{pa   851  -237}%
\special{pa   853  -239}\special{pa   854  -241}\special{pa   855  -243}\special{pa   857  -245}%
\special{pa   858  -247}\special{pa   860  -248}\special{pa   862  -249}\special{pa   863  -250}%
\special{pa   865  -251}\special{pa   867  -252}\special{pa   868  -252}\special{pa   870  -252}%
\special{pa   872  -252}\special{pa   874  -252}\special{pa   875  -251}\special{pa   877  -250}%
\special{pa   879  -249}\special{pa   880  -248}\special{pa   882  -247}\special{pa   883  -245}%
\special{pa   885  -243}\special{pa   886  -241}\special{pa   888  -239}\special{pa   889  -237}%
\special{pa   890  -235}\special{pa   891  -232}\special{pa   892  -229}\special{pa   893  -226}%
\special{pa   894  -223}\special{pa   895  -220}\special{pa   896  -217}\special{pa   896  -214}%
\special{pa   897  -211}\special{pa   897  -207}\special{pa   897  -204}\special{pa   898  -200}%
\special{pa   898  -197}%
\special{fp}%
\special{pa   394  -197}\special{pa   677  -197}%
\special{fp}%
\special{pa  1181  -197}\special{pa   898  -197}%
\special{fp}%
\special{pa -1181   394}\special{pa -1181  -197}%
\special{fp}%
\special{pa  1181  -197}\special{pa  1181   394}%
\special{fp}%
\special{pa -1166 -197}\special{pa -1166 -199}\special{pa -1167 -201}\special{pa -1167 -202}%
\special{pa -1168 -204}\special{pa -1169 -206}\special{pa -1170 -207}\special{pa -1172 -208}%
\special{pa -1173 -209}\special{pa -1175 -210}\special{pa -1176 -211}\special{pa -1178 -212}%
\special{pa -1180 -212}\special{pa -1182 -212}\special{pa -1184 -212}\special{pa -1186 -211}%
\special{pa -1187 -210}\special{pa -1189 -209}\special{pa -1191 -208}\special{pa -1192 -207}%
\special{pa -1193 -206}\special{pa -1194 -204}\special{pa -1195 -202}\special{pa -1196 -201}%
\special{pa -1196 -199}\special{pa -1196 -197}\special{pa -1196 -195}\special{pa -1196 -193}%
\special{pa -1195 -191}\special{pa -1194 -190}\special{pa -1193 -188}\special{pa -1192 -187}%
\special{pa -1191 -185}\special{pa -1189 -184}\special{pa -1187 -183}\special{pa -1186 -183}%
\special{pa -1184 -182}\special{pa -1182 -182}\special{pa -1180 -182}\special{pa -1178 -182}%
\special{pa -1176 -183}\special{pa -1175 -183}\special{pa -1173 -184}\special{pa -1172 -185}%
\special{pa -1170 -187}\special{pa -1169 -188}\special{pa -1168 -190}\special{pa -1167 -191}%
\special{pa -1167 -193}\special{pa -1166 -195}\special{pa -1166 -197}\special{pa -1166 -197}%
\special{sh 1}\special{ip}%
\special{pa -1166  -197}\special{pa -1166  -199}\special{pa -1167  -201}\special{pa -1167  -202}%
\special{pa -1168  -204}\special{pa -1169  -206}\special{pa -1170  -207}\special{pa -1172  -208}%
\special{pa -1173  -209}\special{pa -1175  -210}\special{pa -1176  -211}\special{pa -1178  -212}%
\special{pa -1180  -212}\special{pa -1182  -212}\special{pa -1184  -212}\special{pa -1186  -211}%
\special{pa -1187  -210}\special{pa -1189  -209}\special{pa -1191  -208}\special{pa -1192  -207}%
\special{pa -1193  -206}\special{pa -1194  -204}\special{pa -1195  -202}\special{pa -1196  -201}%
\special{pa -1196  -199}\special{pa -1196  -197}\special{pa -1196  -195}\special{pa -1196  -193}%
\special{pa -1195  -191}\special{pa -1194  -190}\special{pa -1193  -188}\special{pa -1192  -187}%
\special{pa -1191  -185}\special{pa -1189  -184}\special{pa -1187  -183}\special{pa -1186  -183}%
\special{pa -1184  -182}\special{pa -1182  -182}\special{pa -1180  -182}\special{pa -1178  -182}%
\special{pa -1176  -183}\special{pa -1175  -183}\special{pa -1173  -184}\special{pa -1172  -185}%
\special{pa -1170  -187}\special{pa -1169  -188}\special{pa -1168  -190}\special{pa -1167  -191}%
\special{pa -1167  -193}\special{pa -1166  -195}\special{pa -1166  -197}%
\special{fp}%
\special{pa -379 -197}\special{pa -379 -199}\special{pa -379 -201}\special{pa -380 -202}%
\special{pa -381 -204}\special{pa -382 -206}\special{pa -383 -207}\special{pa -384 -208}%
\special{pa -386 -209}\special{pa -387 -210}\special{pa -389 -211}\special{pa -391 -212}%
\special{pa -393 -212}\special{pa -395 -212}\special{pa -397 -212}\special{pa -398 -211}%
\special{pa -400 -210}\special{pa -402 -209}\special{pa -403 -208}\special{pa -405 -207}%
\special{pa -406 -206}\special{pa -407 -204}\special{pa -408 -202}\special{pa -408 -201}%
\special{pa -409 -199}\special{pa -409 -197}\special{pa -409 -195}\special{pa -408 -193}%
\special{pa -408 -191}\special{pa -407 -190}\special{pa -406 -188}\special{pa -405 -187}%
\special{pa -403 -185}\special{pa -402 -184}\special{pa -400 -183}\special{pa -398 -183}%
\special{pa -397 -182}\special{pa -395 -182}\special{pa -393 -182}\special{pa -391 -182}%
\special{pa -389 -183}\special{pa -387 -183}\special{pa -386 -184}\special{pa -384 -185}%
\special{pa -383 -187}\special{pa -382 -188}\special{pa -381 -190}\special{pa -380 -191}%
\special{pa -379 -193}\special{pa -379 -195}\special{pa -379 -197}\special{pa -379 -197}%
\special{sh 1}\special{ip}%
\special{pa  -379  -197}\special{pa  -379  -199}\special{pa  -379  -201}\special{pa  -380  -202}%
\special{pa  -381  -204}\special{pa  -382  -206}\special{pa  -383  -207}\special{pa  -384  -208}%
\special{pa  -386  -209}\special{pa  -387  -210}\special{pa  -389  -211}\special{pa  -391  -212}%
\special{pa  -393  -212}\special{pa  -395  -212}\special{pa  -397  -212}\special{pa  -398  -211}%
\special{pa  -400  -210}\special{pa  -402  -209}\special{pa  -403  -208}\special{pa  -405  -207}%
\special{pa  -406  -206}\special{pa  -407  -204}\special{pa  -408  -202}\special{pa  -408  -201}%
\special{pa  -409  -199}\special{pa  -409  -197}\special{pa  -409  -195}\special{pa  -408  -193}%
\special{pa  -408  -191}\special{pa  -407  -190}\special{pa  -406  -188}\special{pa  -405  -187}%
\special{pa  -403  -185}\special{pa  -402  -184}\special{pa  -400  -183}\special{pa  -398  -183}%
\special{pa  -397  -182}\special{pa  -395  -182}\special{pa  -393  -182}\special{pa  -391  -182}%
\special{pa  -389  -183}\special{pa  -387  -183}\special{pa  -386  -184}\special{pa  -384  -185}%
\special{pa  -383  -187}\special{pa  -382  -188}\special{pa  -381  -190}\special{pa  -380  -191}%
\special{pa  -379  -193}\special{pa  -379  -195}\special{pa  -379  -197}%
\special{fp}%
\special{pa 409 -197}\special{pa 409 -199}\special{pa 408 -201}\special{pa 408 -202}%
\special{pa 407 -204}\special{pa 406 -206}\special{pa 405 -207}\special{pa 403 -208}%
\special{pa 402 -209}\special{pa 400 -210}\special{pa 398 -211}\special{pa 397 -212}%
\special{pa 395 -212}\special{pa 393 -212}\special{pa 391 -212}\special{pa 389 -211}%
\special{pa 387 -210}\special{pa 386 -209}\special{pa 384 -208}\special{pa 383 -207}%
\special{pa 382 -206}\special{pa 381 -204}\special{pa 380 -202}\special{pa 379 -201}%
\special{pa 379 -199}\special{pa 379 -197}\special{pa 379 -195}\special{pa 379 -193}%
\special{pa 380 -191}\special{pa 381 -190}\special{pa 382 -188}\special{pa 383 -187}%
\special{pa 384 -185}\special{pa 386 -184}\special{pa 387 -183}\special{pa 389 -183}%
\special{pa 391 -182}\special{pa 393 -182}\special{pa 395 -182}\special{pa 397 -182}%
\special{pa 398 -183}\special{pa 400 -183}\special{pa 402 -184}\special{pa 403 -185}%
\special{pa 405 -187}\special{pa 406 -188}\special{pa 407 -190}\special{pa 408 -191}%
\special{pa 408 -193}\special{pa 409 -195}\special{pa 409 -197}\special{pa 409 -197}%
\special{sh 1}\special{ip}%
\special{pa   409  -197}\special{pa   409  -199}\special{pa   408  -201}\special{pa   408  -202}%
\special{pa   407  -204}\special{pa   406  -206}\special{pa   405  -207}\special{pa   403  -208}%
\special{pa   402  -209}\special{pa   400  -210}\special{pa   398  -211}\special{pa   397  -212}%
\special{pa   395  -212}\special{pa   393  -212}\special{pa   391  -212}\special{pa   389  -211}%
\special{pa   387  -210}\special{pa   386  -209}\special{pa   384  -208}\special{pa   383  -207}%
\special{pa   382  -206}\special{pa   381  -204}\special{pa   380  -202}\special{pa   379  -201}%
\special{pa   379  -199}\special{pa   379  -197}\special{pa   379  -195}\special{pa   379  -193}%
\special{pa   380  -191}\special{pa   381  -190}\special{pa   382  -188}\special{pa   383  -187}%
\special{pa   384  -185}\special{pa   386  -184}\special{pa   387  -183}\special{pa   389  -183}%
\special{pa   391  -182}\special{pa   393  -182}\special{pa   395  -182}\special{pa   397  -182}%
\special{pa   398  -183}\special{pa   400  -183}\special{pa   402  -184}\special{pa   403  -185}%
\special{pa   405  -187}\special{pa   406  -188}\special{pa   407  -190}\special{pa   408  -191}%
\special{pa   408  -193}\special{pa   409  -195}\special{pa   409  -197}%
\special{fp}%
\special{pa 1196 -197}\special{pa 1196 -199}\special{pa 1196 -201}\special{pa 1195 -202}%
\special{pa 1194 -204}\special{pa 1193 -206}\special{pa 1192 -207}\special{pa 1191 -208}%
\special{pa 1189 -209}\special{pa 1187 -210}\special{pa 1186 -211}\special{pa 1184 -212}%
\special{pa 1182 -212}\special{pa 1180 -212}\special{pa 1178 -212}\special{pa 1176 -211}%
\special{pa 1175 -210}\special{pa 1173 -209}\special{pa 1172 -208}\special{pa 1170 -207}%
\special{pa 1169 -206}\special{pa 1168 -204}\special{pa 1167 -202}\special{pa 1167 -201}%
\special{pa 1166 -199}\special{pa 1166 -197}\special{pa 1166 -195}\special{pa 1167 -193}%
\special{pa 1167 -191}\special{pa 1168 -190}\special{pa 1169 -188}\special{pa 1170 -187}%
\special{pa 1172 -185}\special{pa 1173 -184}\special{pa 1175 -183}\special{pa 1176 -183}%
\special{pa 1178 -182}\special{pa 1180 -182}\special{pa 1182 -182}\special{pa 1184 -182}%
\special{pa 1186 -183}\special{pa 1187 -183}\special{pa 1189 -184}\special{pa 1191 -185}%
\special{pa 1192 -187}\special{pa 1193 -188}\special{pa 1194 -190}\special{pa 1195 -191}%
\special{pa 1196 -193}\special{pa 1196 -195}\special{pa 1196 -197}\special{pa 1196 -197}%
\special{sh 1}\special{ip}%
\special{pa  1196  -197}\special{pa  1196  -199}\special{pa  1196  -201}\special{pa  1195  -202}%
\special{pa  1194  -204}\special{pa  1193  -206}\special{pa  1192  -207}\special{pa  1191  -208}%
\special{pa  1189  -209}\special{pa  1187  -210}\special{pa  1186  -211}\special{pa  1184  -212}%
\special{pa  1182  -212}\special{pa  1180  -212}\special{pa  1178  -212}\special{pa  1176  -211}%
\special{pa  1175  -210}\special{pa  1173  -209}\special{pa  1172  -208}\special{pa  1170  -207}%
\special{pa  1169  -206}\special{pa  1168  -204}\special{pa  1167  -202}\special{pa  1167  -201}%
\special{pa  1166  -199}\special{pa  1166  -197}\special{pa  1166  -195}\special{pa  1167  -193}%
\special{pa  1167  -191}\special{pa  1168  -190}\special{pa  1169  -188}\special{pa  1170  -187}%
\special{pa  1172  -185}\special{pa  1173  -184}\special{pa  1175  -183}\special{pa  1176  -183}%
\special{pa  1178  -182}\special{pa  1180  -182}\special{pa  1182  -182}\special{pa  1184  -182}%
\special{pa  1186  -183}\special{pa  1187  -183}\special{pa  1189  -184}\special{pa  1191  -185}%
\special{pa  1192  -187}\special{pa  1193  -188}\special{pa  1194  -190}\special{pa  1195  -191}%
\special{pa  1196  -193}\special{pa  1196  -195}\special{pa  1196  -197}%
\special{fp}%
\special{pa -1088 -483}\special{pa -1088 -485}\special{pa -1089 -487}\special{pa -1089 -488}%
\special{pa -1090 -490}\special{pa -1091 -492}\special{pa -1092 -493}\special{pa -1094 -494}%
\special{pa -1095 -496}\special{pa -1097 -497}\special{pa -1099 -497}\special{pa -1100 -498}%
\special{pa -1102 -498}\special{pa -1104 -498}\special{pa -1106 -498}\special{pa -1108 -497}%
\special{pa -1110 -497}\special{pa -1111 -496}\special{pa -1113 -494}\special{pa -1114 -493}%
\special{pa -1115 -492}\special{pa -1116 -490}\special{pa -1117 -488}\special{pa -1118 -487}%
\special{pa -1118 -485}\special{pa -1118 -483}\special{pa -1118 -481}\special{pa -1118 -479}%
\special{pa -1117 -477}\special{pa -1116 -476}\special{pa -1115 -474}\special{pa -1114 -473}%
\special{pa -1113 -471}\special{pa -1111 -470}\special{pa -1110 -469}\special{pa -1108 -469}%
\special{pa -1106 -468}\special{pa -1104 -468}\special{pa -1102 -468}\special{pa -1100 -468}%
\special{pa -1099 -469}\special{pa -1097 -469}\special{pa -1095 -470}\special{pa -1094 -471}%
\special{pa -1092 -473}\special{pa -1091 -474}\special{pa -1090 -476}\special{pa -1089 -477}%
\special{pa -1089 -479}\special{pa -1088 -481}\special{pa -1088 -483}\special{pa -1088 -483}%
\special{sh 1}\special{ip}%
\special{pa -1088  -483}\special{pa -1088  -485}\special{pa -1089  -487}\special{pa -1089  -488}%
\special{pa -1090  -490}\special{pa -1091  -492}\special{pa -1092  -493}\special{pa -1094  -494}%
\special{pa -1095  -496}\special{pa -1097  -497}\special{pa -1099  -497}\special{pa -1100  -498}%
\special{pa -1102  -498}\special{pa -1104  -498}\special{pa -1106  -498}\special{pa -1108  -497}%
\special{pa -1110  -497}\special{pa -1111  -496}\special{pa -1113  -494}\special{pa -1114  -493}%
\special{pa -1115  -492}\special{pa -1116  -490}\special{pa -1117  -488}\special{pa -1118  -487}%
\special{pa -1118  -485}\special{pa -1118  -483}\special{pa -1118  -481}\special{pa -1118  -479}%
\special{pa -1117  -477}\special{pa -1116  -476}\special{pa -1115  -474}\special{pa -1114  -473}%
\special{pa -1113  -471}\special{pa -1111  -470}\special{pa -1110  -469}\special{pa -1108  -469}%
\special{pa -1106  -468}\special{pa -1104  -468}\special{pa -1102  -468}\special{pa -1100  -468}%
\special{pa -1099  -469}\special{pa -1097  -469}\special{pa -1095  -470}\special{pa -1094  -471}%
\special{pa -1092  -473}\special{pa -1091  -474}\special{pa -1090  -476}\special{pa -1089  -477}%
\special{pa -1089  -479}\special{pa -1088  -481}\special{pa -1088  -483}%
\special{fp}%
\special{pa -457 -484}\special{pa -457 -485}\special{pa -458 -487}\special{pa -458 -489}%
\special{pa -459 -491}\special{pa -460 -492}\special{pa -461 -494}\special{pa -463 -495}%
\special{pa -464 -496}\special{pa -466 -497}\special{pa -467 -498}\special{pa -469 -498}%
\special{pa -471 -498}\special{pa -473 -498}\special{pa -475 -498}\special{pa -477 -498}%
\special{pa -478 -497}\special{pa -480 -496}\special{pa -482 -495}\special{pa -483 -494}%
\special{pa -484 -492}\special{pa -485 -491}\special{pa -486 -489}\special{pa -487 -487}%
\special{pa -487 -485}\special{pa -487 -484}\special{pa -487 -482}\special{pa -487 -480}%
\special{pa -486 -478}\special{pa -485 -476}\special{pa -484 -475}\special{pa -483 -473}%
\special{pa -482 -472}\special{pa -480 -471}\special{pa -478 -470}\special{pa -477 -469}%
\special{pa -475 -469}\special{pa -473 -469}\special{pa -471 -469}\special{pa -469 -469}%
\special{pa -467 -469}\special{pa -466 -470}\special{pa -464 -471}\special{pa -463 -472}%
\special{pa -461 -473}\special{pa -460 -475}\special{pa -459 -476}\special{pa -458 -478}%
\special{pa -458 -480}\special{pa -457 -482}\special{pa -457 -484}\special{pa -457 -484}%
\special{sh 1}\special{ip}%
\special{pa  -457  -484}\special{pa  -457  -485}\special{pa  -458  -487}\special{pa  -458  -489}%
\special{pa  -459  -491}\special{pa  -460  -492}\special{pa  -461  -494}\special{pa  -463  -495}%
\special{pa  -464  -496}\special{pa  -466  -497}\special{pa  -467  -498}\special{pa  -469  -498}%
\special{pa  -471  -498}\special{pa  -473  -498}\special{pa  -475  -498}\special{pa  -477  -498}%
\special{pa  -478  -497}\special{pa  -480  -496}\special{pa  -482  -495}\special{pa  -483  -494}%
\special{pa  -484  -492}\special{pa  -485  -491}\special{pa  -486  -489}\special{pa  -487  -487}%
\special{pa  -487  -485}\special{pa  -487  -484}\special{pa  -487  -482}\special{pa  -487  -480}%
\special{pa  -486  -478}\special{pa  -485  -476}\special{pa  -484  -475}\special{pa  -483  -473}%
\special{pa  -482  -472}\special{pa  -480  -471}\special{pa  -478  -470}\special{pa  -477  -469}%
\special{pa  -475  -469}\special{pa  -473  -469}\special{pa  -471  -469}\special{pa  -469  -469}%
\special{pa  -467  -469}\special{pa  -466  -470}\special{pa  -464  -471}\special{pa  -463  -472}%
\special{pa  -461  -473}\special{pa  -460  -475}\special{pa  -459  -476}\special{pa  -458  -478}%
\special{pa  -458  -480}\special{pa  -457  -482}\special{pa  -457  -484}%
\special{fp}%
\settowidth{\Width}{a}\setlength{\Width}{-0.5\Width}%
\settoheight{\Height}{a}\settodepth{\Depth}{a}\setlength{\Height}{\Depth}%
\put( -3.000,  0.700){\hspace*{\Width}\raisebox{\Height}{a}}%
%
\settowidth{\Width}{b}\setlength{\Width}{-0.5\Width}%
\settoheight{\Height}{b}\settodepth{\Depth}{b}\setlength{\Height}{\Depth}%
\put(  1.000,  0.700){\hspace*{\Width}\raisebox{\Height}{b}}%
%
\settowidth{\Width}{c}\setlength{\Width}{-0.5\Width}%
\settoheight{\Height}{c}\settodepth{\Depth}{c}\setlength{\Height}{\Depth}%
\put(  3.000,  0.700){\hspace*{\Width}\raisebox{\Height}{c}}%
%
\special{pa -1181  -197}\special{pa -1103  -304}\special{pa -1103  -483}%
\special{fp}%
\special{pa  -394  -197}\special{pa  -472  -305}\special{pa  -472  -484}%
\special{fp}%
\special{pa -1181  -787}\special{pa -1181  -394}\special{pa  -394  -394}\special{pa  -394  -787}%
\special{pa -1181  -787}%
\special{fp}%
\settowidth{\Width}{抵抗}\setlength{\Width}{-0.5\Width}%
\settoheight{\Height}{抵抗}\settodepth{\Depth}{抵抗}\setlength{\Height}{-\Height}%
\put( -2.000,  0.150){\hspace*{\Width}\raisebox{\Height}{抵抗}}%
%
\settowidth{\Width}{コンデンサー}\setlength{\Width}{-0.5\Width}%
\settoheight{\Height}{コンデンサー}\settodepth{\Depth}{コンデンサー}\setlength{\Height}{-\Height}%
\put(  0.000,  0.040){\hspace*{\Width}\raisebox{\Height}{コンデンサー}}%
%
\settowidth{\Width}{コイル}\setlength{\Width}{-0.5\Width}%
\settoheight{\Height}{コイル}\settodepth{\Depth}{コイル}\setlength{\Height}{-\Height}%
\put(  2.000,  0.200){\hspace*{\Width}\raisebox{\Height}{コイル}}%
%
\settowidth{\Width}{オシロ}\setlength{\Width}{0\Width}%
\settoheight{\Height}{オシロ}\settodepth{\Depth}{オシロ}\setlength{\Height}{-\Height}%
\put( -0.800,  1.950){\hspace*{\Width}\raisebox{\Height}{オシロ}}%
%
\settowidth{\Width}{スコープ}\setlength{\Width}{0\Width}%
\settoheight{\Height}{スコープ}\settodepth{\Depth}{スコープ}\setlength{\Height}{-\Height}%
\put( -0.800,  1.650){\hspace*{\Width}\raisebox{\Height}{スコープ}}%
%
\settowidth{\Width}{図1}\setlength{\Width}{-0.5\Width}%
\settoheight{\Height}{図1}\settodepth{\Depth}{図1}\setlength{\Height}{-0.5\Height}\setlength{\Depth}{0.5\Depth}\addtolength{\Height}{\Depth}%
\put(  0.000, -2.000){\hspace*{\Width}\raisebox{\Height}{図1}}%
%
\special{pa  -531  -591}\special{pa  -531  -650}\special{pa  -532  -662}\special{pa  -535  -674}%
\special{pa  -540  -685}\special{pa  -547  -696}\special{pa  -555  -705}\special{pa  -564  -713}%
\special{pa  -574  -720}\special{pa  -586  -724}\special{pa  -598  -727}\special{pa  -610  -728}%
\special{pa  -787  -728}\special{pa  -965  -728}\special{pa  -977  -727}\special{pa  -989  -724}%
\special{pa -1000  -720}\special{pa -1011  -713}\special{pa -1020  -705}\special{pa -1028  -696}%
\special{pa -1035  -685}\special{pa -1039  -674}\special{pa -1042  -662}\special{pa -1043  -650}%
\special{pa -1043  -591}\special{pa -1043  -531}\special{pa -1042  -519}\special{pa -1039  -507}%
\special{pa -1035  -496}\special{pa -1028  -485}\special{pa -1020  -476}\special{pa -1011  -468}%
\special{pa -1000  -461}\special{pa  -989  -457}\special{pa  -977  -454}\special{pa  -965  -453}%
\special{pa  -787  -453}\special{pa  -610  -453}\special{pa  -598  -454}\special{pa  -586  -457}%
\special{pa  -574  -461}\special{pa  -564  -468}\special{pa  -555  -476}\special{pa  -547  -485}%
\special{pa  -540  -496}\special{pa  -535  -507}\special{pa  -532  -519}\special{pa  -531  -531}%
\special{pa  -531  -591}%
\special{fp}%
\end{picture}}%
        %%% C:/vpn/vpn/KeTCindy/fig/fig136_2.tex 
%%% Generator=fig136_2.cdy 
{\unitlength=1cm%
\begin{picture}%
(6,5)(-1,-2.5)%
\special{pn 8}%
%
\special{pa 1657 24}\special{pa 1732 0}\special{pa 1657 -24}\special{pa 1672 0}\special{pa 1657 24}%
\special{pa 1657 24}\special{sh 1}\special{ip}%
\special{pn 1}%
\special{pa  1657    24}\special{pa  1732    -0}\special{pa  1657   -24}\special{pa  1672    -0}%
\special{pa  1657    24}%
\special{fp}%
\special{pn 8}%
\special{pa     0    -0}\special{pa  1672    -0}%
\special{fp}%
\special{pa 24 -516}\special{pa 0 -591}\special{pa -24 -516}\special{pa 0 -531}\special{pa 24 -516}%
\special{pa 24 -516}\special{sh 1}\special{ip}%
\special{pn 1}%
\special{pa    24  -516}\special{pa     0  -591}\special{pa   -24  -516}\special{pa     0  -531}%
\special{pa    24  -516}%
\special{fp}%
\special{pn 8}%
\special{pa     0   591}\special{pa     0  -531}%
\special{fp}%
\special{pa     0    -0}\special{pa    32   -64}\special{pa    64  -126}\special{pa    96  -185}%
\special{pa   129  -239}\special{pa   161  -287}\special{pa   193  -327}\special{pa   225  -358}%
\special{pa   257  -380}\special{pa   289  -392}\special{pa   322  -393}\special{pa   354  -384}%
\special{pa   386  -364}\special{pa   418  -335}\special{pa   450  -297}\special{pa   482  -251}%
\special{pa   515  -198}\special{pa   547  -140}\special{pa   579   -79}\special{pa   611   -15}%
\special{pa   643    49}\special{pa   675   112}\special{pa   707   172}\special{pa   740   227}%
\special{pa   772   277}\special{pa   804   319}\special{pa   836   352}\special{pa   868   376}%
\special{pa   900   390}\special{pa   933   394}\special{pa   965   387}\special{pa   997   370}%
\special{pa  1029   343}\special{pa  1061   306}\special{pa  1093   262}\special{pa  1126   211}%
\special{pa  1158   154}\special{pa  1190    93}\special{pa  1222    30}\special{pa  1254   -35}%
\special{pa  1286   -98}\special{pa  1318  -159}\special{pa  1351  -215}\special{pa  1383  -266}%
\special{pa  1415  -310}\special{pa  1447  -345}\special{pa  1479  -371}\special{pa  1511  -388}%
\special{pa  1544  -394}\special{pa  1576  -389}\special{pa  1608  -374}%
\special{fp}%
\special{pa 1546 -394}\special{pa 1508 -394}\special{fp}\special{pa 1471 -394}\special{pa 1433 -394}\special{fp}%
\special{pa 1395 -394}\special{pa 1358 -394}\special{fp}\special{pa 1320 -394}\special{pa 1282 -394}\special{fp}%
\special{pa 1244 -394}\special{pa 1207 -394}\special{fp}\special{pa 1169 -394}\special{pa 1131 -394}\special{fp}%
\special{pa 1094 -394}\special{pa 1056 -394}\special{fp}\special{pa 1018 -394}\special{pa 980 -394}\special{fp}%
\special{pa 943 -394}\special{pa 905 -394}\special{fp}\special{pa 867 -394}\special{pa 830 -394}\special{fp}%
\special{pa 792 -394}\special{pa 754 -394}\special{fp}\special{pa 716 -394}\special{pa 679 -394}\special{fp}%
\special{pa 641 -394}\special{pa 603 -394}\special{fp}\special{pa 566 -394}\special{pa 528 -394}\special{fp}%
\special{pa 490 -394}\special{pa 453 -394}\special{fp}\special{pa 415 -394}\special{pa 377 -394}\special{fp}%
\special{pa 339 -394}\special{pa 302 -394}\special{fp}\special{pa 264 -394}\special{pa 226 -394}\special{fp}%
\special{pa 189 -394}\special{pa 151 -394}\special{fp}\special{pa 113 -394}\special{pa 75 -394}\special{fp}%
\special{pa 38 -394}\special{pa 0 -394}\special{fp}%
%
\special{pa 928 394}\special{pa 891 394}\special{fp}\special{pa 853 394}\special{pa 816 394}\special{fp}%
\special{pa 779 394}\special{pa 742 394}\special{fp}\special{pa 705 394}\special{pa 668 394}\special{fp}%
\special{pa 631 394}\special{pa 594 394}\special{fp}\special{pa 557 394}\special{pa 519 394}\special{fp}%
\special{pa 482 394}\special{pa 445 394}\special{fp}\special{pa 408 394}\special{pa 371 394}\special{fp}%
\special{pa 334 394}\special{pa 297 394}\special{fp}\special{pa 260 394}\special{pa 223 394}\special{fp}%
\special{pa 186 394}\special{pa 148 394}\special{fp}\special{pa 111 394}\special{pa 74 394}\special{fp}%
\special{pa 37 394}\special{pa 0 394}\special{fp}%
%
\settowidth{\Width}{$-V_0$}\setlength{\Width}{-1\Width}%
\settoheight{\Height}{$-V_0$}\settodepth{\Depth}{$-V_0$}\setlength{\Height}{-0.5\Height}\setlength{\Depth}{0.5\Depth}\addtolength{\Height}{\Depth}%
\put( -0.150, -1.000){\hspace*{\Width}\raisebox{\Height}{$-V_0$}}%
%
\settowidth{\Width}{$V_0$}\setlength{\Width}{-1\Width}%
\settoheight{\Height}{$V_0$}\settodepth{\Depth}{$V_0$}\setlength{\Height}{-0.5\Height}\setlength{\Depth}{0.5\Depth}\addtolength{\Height}{\Depth}%
\put( -0.150,  1.000){\hspace*{\Width}\raisebox{\Height}{$V_0$}}%
%
\settowidth{\Width}{$t$}\setlength{\Width}{0\Width}%
\settoheight{\Height}{$t$}\settodepth{\Depth}{$t$}\setlength{\Height}{-0.5\Height}\setlength{\Depth}{0.5\Depth}\addtolength{\Height}{\Depth}%
\put(  4.550,  0.000){\hspace*{\Width}\raisebox{\Height}{$t$}}%
%
\settowidth{\Width}{$\frac{T}{2}$}\setlength{\Width}{0\Width}%
\settoheight{\Height}{$\frac{T}{2}$}\settodepth{\Depth}{$\frac{T}{2}$}\setlength{\Height}{\Depth}%
\put(  1.670,  0.150){\hspace*{\Width}\raisebox{\Height}{$\frac{T}{2}$}}%
%
\settowidth{\Width}{$T$}\setlength{\Width}{-1\Width}%
\settoheight{\Height}{$T$}\settodepth{\Depth}{$T$}\setlength{\Height}{\Depth}%
\put(  3.040,  0.100){\hspace*{\Width}\raisebox{\Height}{$T$}}%
%
\settowidth{\Width}{O}\setlength{\Width}{-1\Width}%
\settoheight{\Height}{O}\settodepth{\Depth}{O}\setlength{\Height}{-0.5\Height}\setlength{\Depth}{0.5\Depth}\addtolength{\Height}{\Depth}%
\put( -0.150,  0.000){\hspace*{\Width}\raisebox{\Height}{O}}%
%
\settowidth{\Width}{図2}\setlength{\Width}{-0.5\Width}%
\settoheight{\Height}{図2}\settodepth{\Depth}{図2}\setlength{\Height}{-0.5\Height}\setlength{\Depth}{0.5\Depth}\addtolength{\Height}{\Depth}%
\put(  2.250, -1.990){\hspace*{\Width}\raisebox{\Height}{図2}}%
%
\settowidth{\Width}{電圧}\setlength{\Width}{0\Width}%
\settoheight{\Height}{電圧}\settodepth{\Depth}{電圧}\setlength{\Height}{-0.5\Height}\setlength{\Depth}{0.5\Depth}\addtolength{\Height}{\Depth}%
\put(  0.150,  1.500){\hspace*{\Width}\raisebox{\Height}{電圧}}%
%
\settowidth{\Width}{時刻}\setlength{\Width}{-0.5\Width}%
\settoheight{\Height}{時刻}\settodepth{\Depth}{時刻}\setlength{\Height}{-0.5\Height}\setlength{\Depth}{0.5\Depth}\addtolength{\Height}{\Depth}%
\put(  4.080, -0.300){\hspace*{\Width}\raisebox{\Height}{時刻}}%
%
\end{picture}}%
    \end{center}
        \begin{Enumerate}
            \item 交流の角周波数を求めよ。
        \end{Enumerate}
    以下,(5)以外は$T$の代わりに$\omega$を用いて答えよ。
        \begin{Enumerate*}
            \item 抵抗に流れる電流を時刻$t$の関数として表せ。また実効値を求めよ。
            \item この直列回路での消費電力(平均電力)を求めよ。
            \item コンデンサーにかかる電圧の実効値を求めよ。また,電圧$v_\mathrm{C}$を時刻$t$の関数として表せ。
            \item 図2で,コンデンサーにかかる電圧が0になる時刻$t$を$0 \leqq t \leqq T$の範囲で求めよ。
            \item コイルにかかる電圧の実効値を求めよ。また,電圧$v_\mathrm{L}$を時刻$t$の関数として表せ。
            \item 電源電圧の最大値$V_1$を求めよ。また,ab間の電圧の最大値$V_2$を求めよ。
        \end{Enumerate*}
    % \end{mawarikomi}