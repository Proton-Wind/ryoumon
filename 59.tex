\hakosyokika
\item
    \begin{mawarikomi}(40pt,0){150pt}{{\unitlength6mm\small
% \Drawaxisfalse
\begin{zahyou}[yscale=0.20
    ,yokozikukigou={体積}
    ,tatezikukigou={圧力}
    ,gentenkigou={}](0,8)(0,32)
\small
\def\XB{1.11}
\def\XC{4}
\def\XD{6}
\def\XA{3.75}

\def\Fx{16*X**{-1*5/3}}
\def\Gx{90*X**{-1}}
\funcval\Fx{\XB}\YB
\funcval\Fx{\XC}\YC
\funcval\Gx{\XA}\YA
\funcval\Gx{\XD}\YE
\def\A{(\XA,\YA)}
\def\B{(\XB,\YB)}
\def\C{(\XC,\YC)}
\def\D{(\XD,\YC)}
\def\E{(\XD,\YE)}
% \kuromaru{\A;\B;\C;\D;\E}
\Put\A{A}
\Put\B[nw]{B}
\Put\C[s]{C}
\Put\D[se]{D}
\Put\E[ne]{E}
\funcval\Fx{2}\BCY
\funcval\Gx{4.9}\AEY
{
\put(2.5,\YB){\yasen(-0.2,0)}
\put(2,\BCY){\yasen(0.2,-1.3)}
\put(5,\YC){\yasen(0.2,0)}
\calcval{(\YC+\YE)*0.5}\CDY
\put(\XD,\CDY){\yasen(0,0.2)}
\put(4.9,\AEY){\yasen(-0.2,0.8)}
}
{\thicklines
\YGurafu(*)\Fx{\XB}{\XC}
\YGurafu(*)\Gx{\XA}{\XD}
\Drawlines{\A\B;\C\D\E}
% \changeArrowHeadSize<0.333>{2}
% \put(2.5,\YB){\yasen(-0.2,0)}
% \put(2,\BCY){\yasen(0.2,-1.3)}
% \put(5,\YC){\yasen(0.2,0)}
}
\Put\B[syaei=y,ylabel=]{}
\Put\C[syaei=y,ylabel=]{}
\Put\D[syaei=x,xlabel=]{}
\end{zahyou}}
}
容器に閉じ込めた理想気体の状態変化を図のABCDEAの順に行った。B$\rightarrow$Cは断熱過程,E$\rightarrow$Aは等温過程である。次の問の答えを{\sf ア}~{\sf ウ}から選べ。
        \begin{enumerate}
            \item 過程A$\rightarrow$Bにおいて気体は
                \begin{edaenumerate}<3>
                    \item 仕事をされた。
                    \item 仕事をした。
                    \item 仕事をしない。
                \end{edaenumerate}
            \item B$\rightarrow$Cにおいて気体の内部エネルギーは
                \begin{edaenumerate}<3>
                    \item 増加した。
                    \item 減少した。
                    \item 変わらない。
                \end{edaenumerate}
            \item C$\rightarrow$Dにおいて気体の温度は
                \begin{edaenumerate}<3>
                    \item 上昇した。
                    \item 下降した。
                    \item 変わらない。
                \end{edaenumerate}
            \item D$\rightarrow$Eにおいて気体は
                \begin{edaenumerate}<3>
                    \item 熱を放出した。
                    \item 熱を吸収した。
                    \item 熱を授受しない。
                \end{edaenumerate}
            \item E$\rightarrow$A気体分子1個当たりの運動エネルギーは
                \begin{edaenumerate}<3>
                    \item 増加した。
                    \item 減少した。
                    \item 変わらない。
                \end{edaenumerate}
            \item 1サイクルABCDEAにおいて気体が外部にした仕事は
                \begin{edaenumerate}<3>
                    \item ゼロである。
                    \item 正である。
                    \item 負である。
                \end{edaenumerate}
        \end{enumerate}
    \end{mawarikomi}