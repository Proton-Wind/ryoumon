\hakosyokika
\item
    \begin{mawarikomi}(20pt,0){150pt}{{\small
\begin{zahyou*}[ul=4mm](0,10)(0,6)
    \def\A{(0,0)}
    \def\B{(10,0)}
    \def\C{(10,1)}
    \def\D{(1,1)}
    \def\E{(1,5)}
    \def\F{(10,5)}
    \def\G{(10,6)}
    \def\H{(0,6)}
    \def\I{(5,1)}
    \def\J{(6,1)}
    \def\K{(6,5)}
    \def\L{(5,5)}
    \def\M{(3,3)}
    \def\N{(8,3)}
    \Nuritubusi*{\I\J\K\L\I}
    \Drawlines{\A\B\C\D\E\F\G\H\A;\L\I;\K\J}
    \Put\M(0pt,0pt)[b]{A}
    \Put\N(0pt,0pt)[b]{B}
    {\thicklines
    \Drawlines{\G\B}
    }
    % \Put\B(0,-15pt)[b]{$3p$}
    % \Put\A(0,10pt)[b]{1モル}
    % \Put\B(0,10pt)[b]{2モル}
    % \Put\K(-3pt,0pt)[b]{K}
\end{zahyou*}}
}
        図のように両端を密閉したシリンダーが,なめらかに動くピストンで2つの部分A,Bに分けられており,それぞれに単原子分子理想気体が1モルずつ入れられている。シリンダーの右端は熱を通しやすい材料で作られている。
        初めの状態では,A,B内の気体の体積は等しく,温度はともに$T_0$\tanni{K}であった。次に,右端からB内の気体をゆっくりと熱したところ,ピストンは左方向に移動し,最終的にA内の気体の体積はもとの半分になり,温度は$T_1$\tanni{K}になった。気体定数を$R$\tanni{J/(mol\cdot K)}とする。
        \begin{enumerate}
            \item この変化の過程で,A内の気体が受けた仕事はいくらか。
            \item 変化後のA内の気体の圧力は最初の状態の何倍になったか。
            \item 変化後のB内の気体の温度はいくらになったか。
            \item この変化の過程で,B内の気体の内部エネルギーはどれだけ増加したか。
            \item この変化の過程で,B内の気体が外部から吸収した熱量はいくらか。
        \end{enumerate}
    \end{mawarikomi}