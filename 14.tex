\item
\begin{mawarikomi}{150pt}{%WinTpicVersion4.32a
{\unitlength 0.1in%
\begin{picture}(26.3000,20.4500)(2.0000,-27.3500)%
% LINE 2 0 3 0 Black White  
% 4 600 2600 2600 2600 2600 2600 2600 1600
% 
\special{pn 8}%
\special{pa 600 2600}%
\special{pa 2600 2600}%
\special{fp}%
\special{pa 2600 2600}%
\special{pa 2600 1600}%
\special{fp}%
% LINE 2 0 3 0 Black White  
% 2 2600 1600 600 2600
% 
\special{pn 8}%
\special{pa 2600 1600}%
\special{pa 600 2600}%
\special{fp}%
% POLYGON 2 0 1 0 Black Black  
% 5 1942 1786 2000 1903 2179 1813 2121 1697 1942 1786
% 
\special{pn 0}%
\special{sh 0.300}%
\special{pa 1942 1786}%
\special{pa 2000 1903}%
\special{pa 2179 1813}%
\special{pa 2121 1697}%
\special{pa 1942 1786}%
\special{ip}%
\special{pn 8}%
\special{pa 1942 1786}%
\special{pa 2000 1903}%
\special{pa 2179 1813}%
\special{pa 2121 1697}%
\special{pa 1942 1786}%
\special{pa 2000 1903}%
\special{fp}%
% CIRCLE 2 0 3 0 Black Black  
% 4 605 2600 1005 2600 2605 2600 2605 1600
% 
\special{pn 8}%
\special{ar 605 2600 400 400 5.8195377 6.2831853}%
% STR 2 0 3 0 Black Black  
% 4 1055 2430 1055 2530 2 0 0 0
% $\theta $
\put(10.5500,-25.3000){\makebox(0,0)[lb]{$\theta $}}%
% LINE 2 0 3 0 Black Black  
% 2 2125 1690 2625 690
% 
\special{pn 8}%
\special{pa 2125 1690}%
\special{pa 2625 690}%
\special{fp}%
% LINE 2 1 3 0 Black Black  
% 2 2625 690 2625 1410
% 
\special{pn 8}%
\special{pa 2625 690}%
\special{pa 2625 1410}%
\special{da 0.015}%
% CIRCLE 2 0 3 0 Black Black  
% 4 2625 700 2895 700 2115 1710 2625 1710
% 
\special{pn 8}%
\special{ar 2625 700 270 270 1.5707963 2.0383965}%
% STR 2 0 3 0 Black Black  
% 4 2525 1000 2525 1100 5 0 0 0
% $\theta $
\put(25.2500,-11.0000){\makebox(0,0){$\theta $}}%
% STR 2 0 3 0 Black Black  
% 4 2020 1600 2020 1700 3 0 0 0
% P
\put(20.2000,-17.0000){\makebox(0,0)[rb]{P}}%
% STR 2 0 3 0 Black Black  
% 4 2310 2300 2310 2400 5 0 0 0
% $M$
\put(23.1000,-24.0000){\makebox(0,0){$M$}}%
% LINE 2 1 3 0 Black Black  
% 2 2000 1900 310 1900
% 
\special{pn 8}%
\special{pa 2000 1900}%
\special{pa 310 1900}%
\special{da 0.015}%
% LINE 2 0 3 0 Black Black  
% 2 200 2600 2800 2600
% 
\special{pn 8}%
\special{pa 200 2600}%
\special{pa 2800 2600}%
\special{fp}%
% LINE 3 0 3 0 Black Black  
% 88 340 2600 240 2700 280 2600 200 2680 400 2600 300 2700 460 2600 360 2700 520 2600 420 2700 580 2600 480 2700 640 2600 540 2700 700 2600 600 2700 760 2600 660 2700 820 2600 720 2700 880 2600 780 2700 940 2600 840 2700 1000 2600 900 2700 1060 2600 960 2700 1120 2600 1020 2700 1180 2600 1080 2700 1240 2600 1140 2700 1300 2600 1200 2700 1360 2600 1260 2700 1420 2600 1320 2700 1480 2600 1380 2700 1540 2600 1440 2700 1600 2600 1500 2700 1660 2600 1560 2700 1720 2600 1620 2700 1780 2600 1680 2700 1840 2600 1740 2700 1900 2600 1800 2700 1960 2600 1860 2700 2020 2600 1920 2700 2080 2600 1980 2700 2140 2600 2040 2700 2200 2600 2100 2700 2260 2600 2160 2700 2320 2600 2220 2700 2380 2600 2280 2700 2440 2600 2340 2700 2500 2600 2400 2700 2560 2600 2460 2700 2620 2600 2520 2700 2680 2600 2580 2700 2740 2600 2640 2700 2790 2610 2700 2700 2800 2660 2760 2700
% 
\special{pn 4}%
\special{pa 340 2600}%
\special{pa 240 2700}%
\special{fp}%
\special{pa 280 2600}%
\special{pa 200 2680}%
\special{fp}%
\special{pa 400 2600}%
\special{pa 300 2700}%
\special{fp}%
\special{pa 460 2600}%
\special{pa 360 2700}%
\special{fp}%
\special{pa 520 2600}%
\special{pa 420 2700}%
\special{fp}%
\special{pa 580 2600}%
\special{pa 480 2700}%
\special{fp}%
\special{pa 640 2600}%
\special{pa 540 2700}%
\special{fp}%
\special{pa 700 2600}%
\special{pa 600 2700}%
\special{fp}%
\special{pa 760 2600}%
\special{pa 660 2700}%
\special{fp}%
\special{pa 820 2600}%
\special{pa 720 2700}%
\special{fp}%
\special{pa 880 2600}%
\special{pa 780 2700}%
\special{fp}%
\special{pa 940 2600}%
\special{pa 840 2700}%
\special{fp}%
\special{pa 1000 2600}%
\special{pa 900 2700}%
\special{fp}%
\special{pa 1060 2600}%
\special{pa 960 2700}%
\special{fp}%
\special{pa 1120 2600}%
\special{pa 1020 2700}%
\special{fp}%
\special{pa 1180 2600}%
\special{pa 1080 2700}%
\special{fp}%
\special{pa 1240 2600}%
\special{pa 1140 2700}%
\special{fp}%
\special{pa 1300 2600}%
\special{pa 1200 2700}%
\special{fp}%
\special{pa 1360 2600}%
\special{pa 1260 2700}%
\special{fp}%
\special{pa 1420 2600}%
\special{pa 1320 2700}%
\special{fp}%
\special{pa 1480 2600}%
\special{pa 1380 2700}%
\special{fp}%
\special{pa 1540 2600}%
\special{pa 1440 2700}%
\special{fp}%
\special{pa 1600 2600}%
\special{pa 1500 2700}%
\special{fp}%
\special{pa 1660 2600}%
\special{pa 1560 2700}%
\special{fp}%
\special{pa 1720 2600}%
\special{pa 1620 2700}%
\special{fp}%
\special{pa 1780 2600}%
\special{pa 1680 2700}%
\special{fp}%
\special{pa 1840 2600}%
\special{pa 1740 2700}%
\special{fp}%
\special{pa 1900 2600}%
\special{pa 1800 2700}%
\special{fp}%
\special{pa 1960 2600}%
\special{pa 1860 2700}%
\special{fp}%
\special{pa 2020 2600}%
\special{pa 1920 2700}%
\special{fp}%
\special{pa 2080 2600}%
\special{pa 1980 2700}%
\special{fp}%
\special{pa 2140 2600}%
\special{pa 2040 2700}%
\special{fp}%
\special{pa 2200 2600}%
\special{pa 2100 2700}%
\special{fp}%
\special{pa 2260 2600}%
\special{pa 2160 2700}%
\special{fp}%
\special{pa 2320 2600}%
\special{pa 2220 2700}%
\special{fp}%
\special{pa 2380 2600}%
\special{pa 2280 2700}%
\special{fp}%
\special{pa 2440 2600}%
\special{pa 2340 2700}%
\special{fp}%
\special{pa 2500 2600}%
\special{pa 2400 2700}%
\special{fp}%
\special{pa 2560 2600}%
\special{pa 2460 2700}%
\special{fp}%
\special{pa 2620 2600}%
\special{pa 2520 2700}%
\special{fp}%
\special{pa 2680 2600}%
\special{pa 2580 2700}%
\special{fp}%
\special{pa 2740 2600}%
\special{pa 2640 2700}%
\special{fp}%
\special{pa 2790 2610}%
\special{pa 2700 2700}%
\special{fp}%
\special{pa 2800 2660}%
\special{pa 2760 2700}%
\special{fp}%
% POLYGON 2 1 3 0 Black Black  
% 5 530 2490 588 2607 767 2517 709 2401 530 2490
% 
\special{pn 8}%
\special{pn 8}%
\special{pa 530 2490}%
\special{pa 536 2502}%
\special{fp}%
\special{pa 542 2514}%
\special{pa 548 2526}%
\special{fp}%
\special{pa 554 2538}%
\special{pa 560 2550}%
\special{fp}%
\special{pa 566 2562}%
\special{pa 572 2574}%
\special{fp}%
\special{pa 578 2587}%
\special{pa 584 2599}%
\special{fp}%
\special{pa 592 2605}%
\special{pa 604 2599}%
\special{fp}%
\special{pa 616 2593}%
\special{pa 628 2587}%
\special{fp}%
\special{pa 640 2581}%
\special{pa 652 2575}%
\special{fp}%
\special{pa 664 2569}%
\special{pa 676 2563}%
\special{fp}%
\special{pa 688 2557}%
\special{pa 700 2551}%
\special{fp}%
\special{pa 712 2545}%
\special{pa 724 2539}%
\special{fp}%
\special{pa 736 2532}%
\special{pa 748 2526}%
\special{fp}%
\special{pa 760 2520}%
\special{pa 767 2517}%
\special{pa 764 2512}%
\special{fp}%
\special{pa 758 2499}%
\special{pa 752 2488}%
\special{fp}%
\special{pa 746 2475}%
\special{pa 740 2463}%
\special{fp}%
\special{pa 734 2451}%
\special{pa 728 2439}%
\special{fp}%
\special{pa 722 2427}%
\special{pa 716 2415}%
\special{fp}%
\special{pa 710 2403}%
\special{pa 709 2401}%
\special{pa 699 2406}%
\special{fp}%
\special{pa 687 2412}%
\special{pa 675 2418}%
\special{fp}%
\special{pa 663 2424}%
\special{pa 651 2430}%
\special{fp}%
\special{pa 639 2436}%
\special{pa 627 2442}%
\special{fp}%
\special{pa 614 2448}%
\special{pa 602 2454}%
\special{fp}%
\special{pa 590 2460}%
\special{pa 578 2466}%
\special{fp}%
\special{pa 566 2472}%
\special{pa 554 2478}%
\special{fp}%
\special{pa 542 2484}%
\special{pa 530 2490}%
\special{fp}%
% VECTOR 2 0 3 0 Black Black  
% 2 410 2270 410 1900
% 
\special{pn 8}%
\special{pa 410 2270}%
\special{pa 410 1900}%
\special{fp}%
\special{sh 1}%
\special{pa 410 1900}%
\special{pa 390 1967}%
\special{pa 410 1953}%
\special{pa 430 1967}%
\special{pa 410 1900}%
\special{fp}%
% VECTOR 2 0 3 0 Black Black  
% 2 410 2300 410 2600
% 
\special{pn 8}%
\special{pa 410 2300}%
\special{pa 410 2600}%
\special{fp}%
\special{sh 1}%
\special{pa 410 2600}%
\special{pa 430 2533}%
\special{pa 410 2547}%
\special{pa 390 2533}%
\special{pa 410 2600}%
\special{fp}%
% STR 2 0 3 0 Black Black  
% 4 410 2100 410 2200 5 0 1 0
% $h$
\put(4.1000,-22.0000){\makebox(0,0){{\colorbox[named]{White}{$h$}}}}%
% STR 2 0 3 0 Black Black  
% 4 1830 1720 1830 1820 5 0 0 0
% $m$
\put(18.3000,-18.2000){\makebox(0,0){$m$}}%
% STR 2 0 3 0 Black Black  
% 4 600 2700 600 2800 5 0 0 0
% B
\put(6.0000,-28.0000){\makebox(0,0){B}}%
% STR 2 0 3 0 Black Black  
% 4 2030 1900 2030 2000 5 0 0 0
% A
\put(20.3000,-20.0000){\makebox(0,0){A}}%
% LINE 2 0 3 0 Black Black  
% 2 2230 690 2830 690
% 
\special{pn 8}%
\special{pa 2230 690}%
\special{pa 2830 690}%
\special{fp}%
\end{picture}}%
}
    水平な粗い床の上に,なめらかな斜面をもつ質量$M$の台が置かれている。斜面の角度は$\theta $である。
    質量$m$の小物体Pが,天井に固定された糸で斜め上方に引っ張られ,斜面上の点Aで静止していて,糸が鉛直方向となす角度も$\theta $である。
    Pの床からの高さを$h$とし,重力加速度を$g$とする。
    \begin{Enumerate}
        \item 糸の張力$T$,およびPが斜面から受ける垂直抗力$N_1$をそれぞれ求めよ。ただし,$\sin{2\theta }=2\sin{\theta }\cos{\theta }$,$\cos{2\theta}=\cos^2{\theta}-\sin^2{\theta}$とする。
        \item 台が床から受ける静止摩擦力$F$と垂直抗力$R$をそれぞれ求めよ。
        \item 台と床の静止摩擦係数$\mu $はいくら以上か。
    \end{Enumerate}
    糸を切るとPは斜面に沿って滑り出した。一方,台は静止していた。
    \begin{Enumerate*}
        \item Pの加速度$a$,およびPが斜面から受ける垂直抗力$N_2$を求めよ。
        \item 糸を切ってから,Pが斜面を滑り,下端Bに達するまでに要する時間$t$を求めよ。また,Bに達したときの速さ$v$を求めよ。
        \item このようにPが斜面上を滑っている間,台が静止しているためには,台と床との間の静止摩擦係数$\mu $はいくら以上であればよいか。
    \end{Enumerate*}
\end{mawarikomi}
