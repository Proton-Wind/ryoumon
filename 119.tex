\hakosyokika
\item 図1は電球Lに加えた電圧と,それを流れる電流を測定した結果を示したものである。
    この電球Lを含む図2の回路を考える。ただし,100\sftanni{V}の直流電源の内部抵抗は無視できるものとする。
    \begin{center}
        %%% C:/vpn/vpn/KeTCindy/fig/fig119_1.tex 
%%% Generator=fig119_1.cdy 
{\unitlength=1.5cm%
\begin{picture}%
(3.58,3.6)(-0.83,-0.85)%
\special{pn 8}%
%
\settowidth{\Width}{$50$}\setlength{\Width}{-0.5\Width}%
\settoheight{\Height}{$50$}\settodepth{\Depth}{$50$}\setlength{\Height}{-\Height}%
\put(  1.250, -0.100){\hspace*{\Width}\raisebox{\Height}{$50$}}%
%
\special{pa   738    -0}\special{pa   738    59}%
\special{fp}%
\settowidth{\Width}{$100$}\setlength{\Width}{-0.5\Width}%
\settoheight{\Height}{$100$}\settodepth{\Depth}{$100$}\setlength{\Height}{-\Height}%
\put(  2.500, -0.100){\hspace*{\Width}\raisebox{\Height}{$100$}}%
%
\special{pa  1476    -0}\special{pa  1476    59}%
\special{fp}%
\settowidth{\Width}{$0$}\setlength{\Width}{-1\Width}%
\settoheight{\Height}{$0$}\settodepth{\Depth}{$0$}\setlength{\Height}{-0.5\Height}\setlength{\Depth}{0.5\Depth}\addtolength{\Height}{\Depth}%
\put( -0.100,  0.000){\hspace*{\Width}\raisebox{\Height}{$0$}}%
%
\special{pa     0    -0}\special{pa   -59    -0}%
\special{fp}%
\settowidth{\Width}{$0.5$}\setlength{\Width}{-1\Width}%
\settoheight{\Height}{$0.5$}\settodepth{\Depth}{$0.5$}\setlength{\Height}{-0.5\Height}\setlength{\Depth}{0.5\Depth}\addtolength{\Height}{\Depth}%
\put( -0.100,  1.250){\hspace*{\Width}\raisebox{\Height}{$0.5$}}%
%
\special{pa     0  -738}\special{pa   -59  -738}%
\special{fp}%
\settowidth{\Width}{$1.0$}\setlength{\Width}{-1\Width}%
\settoheight{\Height}{$1.0$}\settodepth{\Depth}{$1.0$}\setlength{\Height}{-0.5\Height}\setlength{\Depth}{0.5\Depth}\addtolength{\Height}{\Depth}%
\put( -0.100,  2.500){\hspace*{\Width}\raisebox{\Height}{$1.0$}}%
%
\special{pa     0 -1476}\special{pa   -59 -1476}%
\special{fp}%
\special{pa 148 0}\special{pa 148 -38}\special{fp}\special{pa 148 -76}\special{pa 148 -114}\special{fp}%
\special{pa 148 -151}\special{pa 148 -189}\special{fp}\special{pa 148 -227}\special{pa 148 -265}\special{fp}%
\special{pa 148 -303}\special{pa 148 -341}\special{fp}\special{pa 148 -379}\special{pa 148 -416}\special{fp}%
\special{pa 148 -454}\special{pa 148 -492}\special{fp}\special{pa 148 -530}\special{pa 148 -568}\special{fp}%
\special{pa 148 -606}\special{pa 148 -644}\special{fp}\special{pa 148 -681}\special{pa 148 -719}\special{fp}%
\special{pa 148 -757}\special{pa 148 -795}\special{fp}\special{pa 148 -833}\special{pa 148 -871}\special{fp}%
\special{pa 148 -909}\special{pa 148 -946}\special{fp}\special{pa 148 -984}\special{pa 148 -1022}\special{fp}%
\special{pa 148 -1060}\special{pa 148 -1098}\special{fp}\special{pa 148 -1136}\special{pa 148 -1174}\special{fp}%
\special{pa 148 -1211}\special{pa 148 -1249}\special{fp}\special{pa 148 -1287}\special{pa 148 -1325}\special{fp}%
\special{pa 148 -1363}\special{pa 148 -1401}\special{fp}\special{pa 148 -1439}\special{pa 148 -1476}\special{fp}%
%
%
\special{pa 295 0}\special{pa 295 -38}\special{fp}\special{pa 295 -76}\special{pa 295 -114}\special{fp}%
\special{pa 295 -151}\special{pa 295 -189}\special{fp}\special{pa 295 -227}\special{pa 295 -265}\special{fp}%
\special{pa 295 -303}\special{pa 295 -341}\special{fp}\special{pa 295 -379}\special{pa 295 -416}\special{fp}%
\special{pa 295 -454}\special{pa 295 -492}\special{fp}\special{pa 295 -530}\special{pa 295 -568}\special{fp}%
\special{pa 295 -606}\special{pa 295 -644}\special{fp}\special{pa 295 -681}\special{pa 295 -719}\special{fp}%
\special{pa 295 -757}\special{pa 295 -795}\special{fp}\special{pa 295 -833}\special{pa 295 -871}\special{fp}%
\special{pa 295 -909}\special{pa 295 -946}\special{fp}\special{pa 295 -984}\special{pa 295 -1022}\special{fp}%
\special{pa 295 -1060}\special{pa 295 -1098}\special{fp}\special{pa 295 -1136}\special{pa 295 -1174}\special{fp}%
\special{pa 295 -1211}\special{pa 295 -1249}\special{fp}\special{pa 295 -1287}\special{pa 295 -1325}\special{fp}%
\special{pa 295 -1363}\special{pa 295 -1401}\special{fp}\special{pa 295 -1439}\special{pa 295 -1476}\special{fp}%
%
%
\special{pa 443 0}\special{pa 443 -38}\special{fp}\special{pa 443 -76}\special{pa 443 -114}\special{fp}%
\special{pa 443 -151}\special{pa 443 -189}\special{fp}\special{pa 443 -227}\special{pa 443 -265}\special{fp}%
\special{pa 443 -303}\special{pa 443 -341}\special{fp}\special{pa 443 -379}\special{pa 443 -416}\special{fp}%
\special{pa 443 -454}\special{pa 443 -492}\special{fp}\special{pa 443 -530}\special{pa 443 -568}\special{fp}%
\special{pa 443 -606}\special{pa 443 -644}\special{fp}\special{pa 443 -681}\special{pa 443 -719}\special{fp}%
\special{pa 443 -757}\special{pa 443 -795}\special{fp}\special{pa 443 -833}\special{pa 443 -871}\special{fp}%
\special{pa 443 -909}\special{pa 443 -946}\special{fp}\special{pa 443 -984}\special{pa 443 -1022}\special{fp}%
\special{pa 443 -1060}\special{pa 443 -1098}\special{fp}\special{pa 443 -1136}\special{pa 443 -1174}\special{fp}%
\special{pa 443 -1211}\special{pa 443 -1249}\special{fp}\special{pa 443 -1287}\special{pa 443 -1325}\special{fp}%
\special{pa 443 -1363}\special{pa 443 -1401}\special{fp}\special{pa 443 -1439}\special{pa 443 -1476}\special{fp}%
%
%
\special{pa 591 0}\special{pa 591 -38}\special{fp}\special{pa 591 -76}\special{pa 591 -114}\special{fp}%
\special{pa 591 -151}\special{pa 591 -189}\special{fp}\special{pa 591 -227}\special{pa 591 -265}\special{fp}%
\special{pa 591 -303}\special{pa 591 -341}\special{fp}\special{pa 591 -379}\special{pa 591 -416}\special{fp}%
\special{pa 591 -454}\special{pa 591 -492}\special{fp}\special{pa 591 -530}\special{pa 591 -568}\special{fp}%
\special{pa 591 -606}\special{pa 591 -644}\special{fp}\special{pa 591 -681}\special{pa 591 -719}\special{fp}%
\special{pa 591 -757}\special{pa 591 -795}\special{fp}\special{pa 591 -833}\special{pa 591 -871}\special{fp}%
\special{pa 591 -909}\special{pa 591 -946}\special{fp}\special{pa 591 -984}\special{pa 591 -1022}\special{fp}%
\special{pa 591 -1060}\special{pa 591 -1098}\special{fp}\special{pa 591 -1136}\special{pa 591 -1174}\special{fp}%
\special{pa 591 -1211}\special{pa 591 -1249}\special{fp}\special{pa 591 -1287}\special{pa 591 -1325}\special{fp}%
\special{pa 591 -1363}\special{pa 591 -1401}\special{fp}\special{pa 591 -1439}\special{pa 591 -1476}\special{fp}%
%
%
\special{pa 738 0}\special{pa 738 -38}\special{fp}\special{pa 738 -76}\special{pa 738 -114}\special{fp}%
\special{pa 738 -151}\special{pa 738 -189}\special{fp}\special{pa 738 -227}\special{pa 738 -265}\special{fp}%
\special{pa 738 -303}\special{pa 738 -341}\special{fp}\special{pa 738 -379}\special{pa 738 -416}\special{fp}%
\special{pa 738 -454}\special{pa 738 -492}\special{fp}\special{pa 738 -530}\special{pa 738 -568}\special{fp}%
\special{pa 738 -606}\special{pa 738 -644}\special{fp}\special{pa 738 -681}\special{pa 738 -719}\special{fp}%
\special{pa 738 -757}\special{pa 738 -795}\special{fp}\special{pa 738 -833}\special{pa 738 -871}\special{fp}%
\special{pa 738 -909}\special{pa 738 -946}\special{fp}\special{pa 738 -984}\special{pa 738 -1022}\special{fp}%
\special{pa 738 -1060}\special{pa 738 -1098}\special{fp}\special{pa 738 -1136}\special{pa 738 -1174}\special{fp}%
\special{pa 738 -1211}\special{pa 738 -1249}\special{fp}\special{pa 738 -1287}\special{pa 738 -1325}\special{fp}%
\special{pa 738 -1363}\special{pa 738 -1401}\special{fp}\special{pa 738 -1439}\special{pa 738 -1476}\special{fp}%
%
%
\special{pa 886 0}\special{pa 886 -38}\special{fp}\special{pa 886 -76}\special{pa 886 -114}\special{fp}%
\special{pa 886 -151}\special{pa 886 -189}\special{fp}\special{pa 886 -227}\special{pa 886 -265}\special{fp}%
\special{pa 886 -303}\special{pa 886 -341}\special{fp}\special{pa 886 -379}\special{pa 886 -416}\special{fp}%
\special{pa 886 -454}\special{pa 886 -492}\special{fp}\special{pa 886 -530}\special{pa 886 -568}\special{fp}%
\special{pa 886 -606}\special{pa 886 -644}\special{fp}\special{pa 886 -681}\special{pa 886 -719}\special{fp}%
\special{pa 886 -757}\special{pa 886 -795}\special{fp}\special{pa 886 -833}\special{pa 886 -871}\special{fp}%
\special{pa 886 -909}\special{pa 886 -946}\special{fp}\special{pa 886 -984}\special{pa 886 -1022}\special{fp}%
\special{pa 886 -1060}\special{pa 886 -1098}\special{fp}\special{pa 886 -1136}\special{pa 886 -1174}\special{fp}%
\special{pa 886 -1211}\special{pa 886 -1249}\special{fp}\special{pa 886 -1287}\special{pa 886 -1325}\special{fp}%
\special{pa 886 -1363}\special{pa 886 -1401}\special{fp}\special{pa 886 -1439}\special{pa 886 -1476}\special{fp}%
%
%
\special{pa 1033 0}\special{pa 1033 -38}\special{fp}\special{pa 1033 -76}\special{pa 1033 -114}\special{fp}%
\special{pa 1033 -151}\special{pa 1033 -189}\special{fp}\special{pa 1033 -227}\special{pa 1033 -265}\special{fp}%
\special{pa 1033 -303}\special{pa 1033 -341}\special{fp}\special{pa 1033 -379}\special{pa 1033 -416}\special{fp}%
\special{pa 1033 -454}\special{pa 1033 -492}\special{fp}\special{pa 1033 -530}\special{pa 1033 -568}\special{fp}%
\special{pa 1033 -606}\special{pa 1033 -644}\special{fp}\special{pa 1033 -681}\special{pa 1033 -719}\special{fp}%
\special{pa 1033 -757}\special{pa 1033 -795}\special{fp}\special{pa 1033 -833}\special{pa 1033 -871}\special{fp}%
\special{pa 1033 -909}\special{pa 1033 -946}\special{fp}\special{pa 1033 -984}\special{pa 1033 -1022}\special{fp}%
\special{pa 1033 -1060}\special{pa 1033 -1098}\special{fp}\special{pa 1033 -1136}\special{pa 1033 -1174}\special{fp}%
\special{pa 1033 -1211}\special{pa 1033 -1249}\special{fp}\special{pa 1033 -1287}\special{pa 1033 -1325}\special{fp}%
\special{pa 1033 -1363}\special{pa 1033 -1401}\special{fp}\special{pa 1033 -1439}\special{pa 1033 -1476}\special{fp}%
%
%
\special{pa 1181 0}\special{pa 1181 -38}\special{fp}\special{pa 1181 -76}\special{pa 1181 -114}\special{fp}%
\special{pa 1181 -151}\special{pa 1181 -189}\special{fp}\special{pa 1181 -227}\special{pa 1181 -265}\special{fp}%
\special{pa 1181 -303}\special{pa 1181 -341}\special{fp}\special{pa 1181 -379}\special{pa 1181 -416}\special{fp}%
\special{pa 1181 -454}\special{pa 1181 -492}\special{fp}\special{pa 1181 -530}\special{pa 1181 -568}\special{fp}%
\special{pa 1181 -606}\special{pa 1181 -644}\special{fp}\special{pa 1181 -681}\special{pa 1181 -719}\special{fp}%
\special{pa 1181 -757}\special{pa 1181 -795}\special{fp}\special{pa 1181 -833}\special{pa 1181 -871}\special{fp}%
\special{pa 1181 -909}\special{pa 1181 -946}\special{fp}\special{pa 1181 -984}\special{pa 1181 -1022}\special{fp}%
\special{pa 1181 -1060}\special{pa 1181 -1098}\special{fp}\special{pa 1181 -1136}\special{pa 1181 -1174}\special{fp}%
\special{pa 1181 -1211}\special{pa 1181 -1249}\special{fp}\special{pa 1181 -1287}\special{pa 1181 -1325}\special{fp}%
\special{pa 1181 -1363}\special{pa 1181 -1401}\special{fp}\special{pa 1181 -1439}\special{pa 1181 -1476}\special{fp}%
%
%
\special{pa 1329 0}\special{pa 1329 -38}\special{fp}\special{pa 1329 -76}\special{pa 1329 -114}\special{fp}%
\special{pa 1329 -151}\special{pa 1329 -189}\special{fp}\special{pa 1329 -227}\special{pa 1329 -265}\special{fp}%
\special{pa 1329 -303}\special{pa 1329 -341}\special{fp}\special{pa 1329 -379}\special{pa 1329 -416}\special{fp}%
\special{pa 1329 -454}\special{pa 1329 -492}\special{fp}\special{pa 1329 -530}\special{pa 1329 -568}\special{fp}%
\special{pa 1329 -606}\special{pa 1329 -644}\special{fp}\special{pa 1329 -681}\special{pa 1329 -719}\special{fp}%
\special{pa 1329 -757}\special{pa 1329 -795}\special{fp}\special{pa 1329 -833}\special{pa 1329 -871}\special{fp}%
\special{pa 1329 -909}\special{pa 1329 -946}\special{fp}\special{pa 1329 -984}\special{pa 1329 -1022}\special{fp}%
\special{pa 1329 -1060}\special{pa 1329 -1098}\special{fp}\special{pa 1329 -1136}\special{pa 1329 -1174}\special{fp}%
\special{pa 1329 -1211}\special{pa 1329 -1249}\special{fp}\special{pa 1329 -1287}\special{pa 1329 -1325}\special{fp}%
\special{pa 1329 -1363}\special{pa 1329 -1401}\special{fp}\special{pa 1329 -1439}\special{pa 1329 -1476}\special{fp}%
%
%
\special{pa 1476 0}\special{pa 1476 -38}\special{fp}\special{pa 1476 -76}\special{pa 1476 -114}\special{fp}%
\special{pa 1476 -151}\special{pa 1476 -189}\special{fp}\special{pa 1476 -227}\special{pa 1476 -265}\special{fp}%
\special{pa 1476 -303}\special{pa 1476 -341}\special{fp}\special{pa 1476 -379}\special{pa 1476 -416}\special{fp}%
\special{pa 1476 -454}\special{pa 1476 -492}\special{fp}\special{pa 1476 -530}\special{pa 1476 -568}\special{fp}%
\special{pa 1476 -606}\special{pa 1476 -644}\special{fp}\special{pa 1476 -681}\special{pa 1476 -719}\special{fp}%
\special{pa 1476 -757}\special{pa 1476 -795}\special{fp}\special{pa 1476 -833}\special{pa 1476 -871}\special{fp}%
\special{pa 1476 -909}\special{pa 1476 -946}\special{fp}\special{pa 1476 -984}\special{pa 1476 -1022}\special{fp}%
\special{pa 1476 -1060}\special{pa 1476 -1098}\special{fp}\special{pa 1476 -1136}\special{pa 1476 -1174}\special{fp}%
\special{pa 1476 -1211}\special{pa 1476 -1249}\special{fp}\special{pa 1476 -1287}\special{pa 1476 -1325}\special{fp}%
\special{pa 1476 -1363}\special{pa 1476 -1401}\special{fp}\special{pa 1476 -1439}\special{pa 1476 -1476}\special{fp}%
%
%
\special{pa 0 -148}\special{pa 38 -148}\special{fp}\special{pa 76 -148}\special{pa 114 -148}\special{fp}%
\special{pa 151 -148}\special{pa 189 -148}\special{fp}\special{pa 227 -148}\special{pa 265 -148}\special{fp}%
\special{pa 303 -148}\special{pa 341 -148}\special{fp}\special{pa 379 -148}\special{pa 416 -148}\special{fp}%
\special{pa 454 -148}\special{pa 492 -148}\special{fp}\special{pa 530 -148}\special{pa 568 -148}\special{fp}%
\special{pa 606 -148}\special{pa 644 -148}\special{fp}\special{pa 681 -148}\special{pa 719 -148}\special{fp}%
\special{pa 757 -148}\special{pa 795 -148}\special{fp}\special{pa 833 -148}\special{pa 871 -148}\special{fp}%
\special{pa 909 -148}\special{pa 946 -148}\special{fp}\special{pa 984 -148}\special{pa 1022 -148}\special{fp}%
\special{pa 1060 -148}\special{pa 1098 -148}\special{fp}\special{pa 1136 -148}\special{pa 1174 -148}\special{fp}%
\special{pa 1211 -148}\special{pa 1249 -148}\special{fp}\special{pa 1287 -148}\special{pa 1325 -148}\special{fp}%
\special{pa 1363 -148}\special{pa 1401 -148}\special{fp}\special{pa 1439 -148}\special{pa 1476 -148}\special{fp}%
%
%
\special{pa 0 -295}\special{pa 38 -295}\special{fp}\special{pa 76 -295}\special{pa 114 -295}\special{fp}%
\special{pa 151 -295}\special{pa 189 -295}\special{fp}\special{pa 227 -295}\special{pa 265 -295}\special{fp}%
\special{pa 303 -295}\special{pa 341 -295}\special{fp}\special{pa 379 -295}\special{pa 416 -295}\special{fp}%
\special{pa 454 -295}\special{pa 492 -295}\special{fp}\special{pa 530 -295}\special{pa 568 -295}\special{fp}%
\special{pa 606 -295}\special{pa 644 -295}\special{fp}\special{pa 681 -295}\special{pa 719 -295}\special{fp}%
\special{pa 757 -295}\special{pa 795 -295}\special{fp}\special{pa 833 -295}\special{pa 871 -295}\special{fp}%
\special{pa 909 -295}\special{pa 946 -295}\special{fp}\special{pa 984 -295}\special{pa 1022 -295}\special{fp}%
\special{pa 1060 -295}\special{pa 1098 -295}\special{fp}\special{pa 1136 -295}\special{pa 1174 -295}\special{fp}%
\special{pa 1211 -295}\special{pa 1249 -295}\special{fp}\special{pa 1287 -295}\special{pa 1325 -295}\special{fp}%
\special{pa 1363 -295}\special{pa 1401 -295}\special{fp}\special{pa 1439 -295}\special{pa 1476 -295}\special{fp}%
%
%
\special{pa 0 -443}\special{pa 38 -443}\special{fp}\special{pa 76 -443}\special{pa 114 -443}\special{fp}%
\special{pa 151 -443}\special{pa 189 -443}\special{fp}\special{pa 227 -443}\special{pa 265 -443}\special{fp}%
\special{pa 303 -443}\special{pa 341 -443}\special{fp}\special{pa 379 -443}\special{pa 416 -443}\special{fp}%
\special{pa 454 -443}\special{pa 492 -443}\special{fp}\special{pa 530 -443}\special{pa 568 -443}\special{fp}%
\special{pa 606 -443}\special{pa 644 -443}\special{fp}\special{pa 681 -443}\special{pa 719 -443}\special{fp}%
\special{pa 757 -443}\special{pa 795 -443}\special{fp}\special{pa 833 -443}\special{pa 871 -443}\special{fp}%
\special{pa 909 -443}\special{pa 946 -443}\special{fp}\special{pa 984 -443}\special{pa 1022 -443}\special{fp}%
\special{pa 1060 -443}\special{pa 1098 -443}\special{fp}\special{pa 1136 -443}\special{pa 1174 -443}\special{fp}%
\special{pa 1211 -443}\special{pa 1249 -443}\special{fp}\special{pa 1287 -443}\special{pa 1325 -443}\special{fp}%
\special{pa 1363 -443}\special{pa 1401 -443}\special{fp}\special{pa 1439 -443}\special{pa 1476 -443}\special{fp}%
%
%
\special{pa 0 -591}\special{pa 38 -591}\special{fp}\special{pa 76 -591}\special{pa 114 -591}\special{fp}%
\special{pa 151 -591}\special{pa 189 -591}\special{fp}\special{pa 227 -591}\special{pa 265 -591}\special{fp}%
\special{pa 303 -591}\special{pa 341 -591}\special{fp}\special{pa 379 -591}\special{pa 416 -591}\special{fp}%
\special{pa 454 -591}\special{pa 492 -591}\special{fp}\special{pa 530 -591}\special{pa 568 -591}\special{fp}%
\special{pa 606 -591}\special{pa 644 -591}\special{fp}\special{pa 681 -591}\special{pa 719 -591}\special{fp}%
\special{pa 757 -591}\special{pa 795 -591}\special{fp}\special{pa 833 -591}\special{pa 871 -591}\special{fp}%
\special{pa 909 -591}\special{pa 946 -591}\special{fp}\special{pa 984 -591}\special{pa 1022 -591}\special{fp}%
\special{pa 1060 -591}\special{pa 1098 -591}\special{fp}\special{pa 1136 -591}\special{pa 1174 -591}\special{fp}%
\special{pa 1211 -591}\special{pa 1249 -591}\special{fp}\special{pa 1287 -591}\special{pa 1325 -591}\special{fp}%
\special{pa 1363 -591}\special{pa 1401 -591}\special{fp}\special{pa 1439 -591}\special{pa 1476 -591}\special{fp}%
%
%
\special{pa 0 -738}\special{pa 38 -738}\special{fp}\special{pa 76 -738}\special{pa 114 -738}\special{fp}%
\special{pa 151 -738}\special{pa 189 -738}\special{fp}\special{pa 227 -738}\special{pa 265 -738}\special{fp}%
\special{pa 303 -738}\special{pa 341 -738}\special{fp}\special{pa 379 -738}\special{pa 416 -738}\special{fp}%
\special{pa 454 -738}\special{pa 492 -738}\special{fp}\special{pa 530 -738}\special{pa 568 -738}\special{fp}%
\special{pa 606 -738}\special{pa 644 -738}\special{fp}\special{pa 681 -738}\special{pa 719 -738}\special{fp}%
\special{pa 757 -738}\special{pa 795 -738}\special{fp}\special{pa 833 -738}\special{pa 871 -738}\special{fp}%
\special{pa 909 -738}\special{pa 946 -738}\special{fp}\special{pa 984 -738}\special{pa 1022 -738}\special{fp}%
\special{pa 1060 -738}\special{pa 1098 -738}\special{fp}\special{pa 1136 -738}\special{pa 1174 -738}\special{fp}%
\special{pa 1211 -738}\special{pa 1249 -738}\special{fp}\special{pa 1287 -738}\special{pa 1325 -738}\special{fp}%
\special{pa 1363 -738}\special{pa 1401 -738}\special{fp}\special{pa 1439 -738}\special{pa 1476 -738}\special{fp}%
%
%
\special{pa 0 -886}\special{pa 38 -886}\special{fp}\special{pa 76 -886}\special{pa 114 -886}\special{fp}%
\special{pa 151 -886}\special{pa 189 -886}\special{fp}\special{pa 227 -886}\special{pa 265 -886}\special{fp}%
\special{pa 303 -886}\special{pa 341 -886}\special{fp}\special{pa 379 -886}\special{pa 416 -886}\special{fp}%
\special{pa 454 -886}\special{pa 492 -886}\special{fp}\special{pa 530 -886}\special{pa 568 -886}\special{fp}%
\special{pa 606 -886}\special{pa 644 -886}\special{fp}\special{pa 681 -886}\special{pa 719 -886}\special{fp}%
\special{pa 757 -886}\special{pa 795 -886}\special{fp}\special{pa 833 -886}\special{pa 871 -886}\special{fp}%
\special{pa 909 -886}\special{pa 946 -886}\special{fp}\special{pa 984 -886}\special{pa 1022 -886}\special{fp}%
\special{pa 1060 -886}\special{pa 1098 -886}\special{fp}\special{pa 1136 -886}\special{pa 1174 -886}\special{fp}%
\special{pa 1211 -886}\special{pa 1249 -886}\special{fp}\special{pa 1287 -886}\special{pa 1325 -886}\special{fp}%
\special{pa 1363 -886}\special{pa 1401 -886}\special{fp}\special{pa 1439 -886}\special{pa 1476 -886}\special{fp}%
%
%
\special{pa 0 -1033}\special{pa 38 -1033}\special{fp}\special{pa 76 -1033}\special{pa 114 -1033}\special{fp}%
\special{pa 151 -1033}\special{pa 189 -1033}\special{fp}\special{pa 227 -1033}\special{pa 265 -1033}\special{fp}%
\special{pa 303 -1033}\special{pa 341 -1033}\special{fp}\special{pa 379 -1033}\special{pa 416 -1033}\special{fp}%
\special{pa 454 -1033}\special{pa 492 -1033}\special{fp}\special{pa 530 -1033}\special{pa 568 -1033}\special{fp}%
\special{pa 606 -1033}\special{pa 644 -1033}\special{fp}\special{pa 681 -1033}\special{pa 719 -1033}\special{fp}%
\special{pa 757 -1033}\special{pa 795 -1033}\special{fp}\special{pa 833 -1033}\special{pa 871 -1033}\special{fp}%
\special{pa 909 -1033}\special{pa 946 -1033}\special{fp}\special{pa 984 -1033}\special{pa 1022 -1033}\special{fp}%
\special{pa 1060 -1033}\special{pa 1098 -1033}\special{fp}\special{pa 1136 -1033}\special{pa 1174 -1033}\special{fp}%
\special{pa 1211 -1033}\special{pa 1249 -1033}\special{fp}\special{pa 1287 -1033}\special{pa 1325 -1033}\special{fp}%
\special{pa 1363 -1033}\special{pa 1401 -1033}\special{fp}\special{pa 1439 -1033}\special{pa 1476 -1033}\special{fp}%
%
%
\special{pa 0 -1181}\special{pa 38 -1181}\special{fp}\special{pa 76 -1181}\special{pa 114 -1181}\special{fp}%
\special{pa 151 -1181}\special{pa 189 -1181}\special{fp}\special{pa 227 -1181}\special{pa 265 -1181}\special{fp}%
\special{pa 303 -1181}\special{pa 341 -1181}\special{fp}\special{pa 379 -1181}\special{pa 416 -1181}\special{fp}%
\special{pa 454 -1181}\special{pa 492 -1181}\special{fp}\special{pa 530 -1181}\special{pa 568 -1181}\special{fp}%
\special{pa 606 -1181}\special{pa 644 -1181}\special{fp}\special{pa 681 -1181}\special{pa 719 -1181}\special{fp}%
\special{pa 757 -1181}\special{pa 795 -1181}\special{fp}\special{pa 833 -1181}\special{pa 871 -1181}\special{fp}%
\special{pa 909 -1181}\special{pa 946 -1181}\special{fp}\special{pa 984 -1181}\special{pa 1022 -1181}\special{fp}%
\special{pa 1060 -1181}\special{pa 1098 -1181}\special{fp}\special{pa 1136 -1181}\special{pa 1174 -1181}\special{fp}%
\special{pa 1211 -1181}\special{pa 1249 -1181}\special{fp}\special{pa 1287 -1181}\special{pa 1325 -1181}\special{fp}%
\special{pa 1363 -1181}\special{pa 1401 -1181}\special{fp}\special{pa 1439 -1181}\special{pa 1476 -1181}\special{fp}%
%
%
\special{pa 0 -1329}\special{pa 38 -1329}\special{fp}\special{pa 76 -1329}\special{pa 114 -1329}\special{fp}%
\special{pa 151 -1329}\special{pa 189 -1329}\special{fp}\special{pa 227 -1329}\special{pa 265 -1329}\special{fp}%
\special{pa 303 -1329}\special{pa 341 -1329}\special{fp}\special{pa 379 -1329}\special{pa 416 -1329}\special{fp}%
\special{pa 454 -1329}\special{pa 492 -1329}\special{fp}\special{pa 530 -1329}\special{pa 568 -1329}\special{fp}%
\special{pa 606 -1329}\special{pa 644 -1329}\special{fp}\special{pa 681 -1329}\special{pa 719 -1329}\special{fp}%
\special{pa 757 -1329}\special{pa 795 -1329}\special{fp}\special{pa 833 -1329}\special{pa 871 -1329}\special{fp}%
\special{pa 909 -1329}\special{pa 946 -1329}\special{fp}\special{pa 984 -1329}\special{pa 1022 -1329}\special{fp}%
\special{pa 1060 -1329}\special{pa 1098 -1329}\special{fp}\special{pa 1136 -1329}\special{pa 1174 -1329}\special{fp}%
\special{pa 1211 -1329}\special{pa 1249 -1329}\special{fp}\special{pa 1287 -1329}\special{pa 1325 -1329}\special{fp}%
\special{pa 1363 -1329}\special{pa 1401 -1329}\special{fp}\special{pa 1439 -1329}\special{pa 1476 -1329}\special{fp}%
%
%
\special{pa 0 -1476}\special{pa 38 -1476}\special{fp}\special{pa 76 -1476}\special{pa 114 -1476}\special{fp}%
\special{pa 151 -1476}\special{pa 189 -1476}\special{fp}\special{pa 227 -1476}\special{pa 265 -1476}\special{fp}%
\special{pa 303 -1476}\special{pa 341 -1476}\special{fp}\special{pa 379 -1476}\special{pa 416 -1476}\special{fp}%
\special{pa 454 -1476}\special{pa 492 -1476}\special{fp}\special{pa 530 -1476}\special{pa 568 -1476}\special{fp}%
\special{pa 606 -1476}\special{pa 644 -1476}\special{fp}\special{pa 681 -1476}\special{pa 719 -1476}\special{fp}%
\special{pa 757 -1476}\special{pa 795 -1476}\special{fp}\special{pa 833 -1476}\special{pa 871 -1476}\special{fp}%
\special{pa 909 -1476}\special{pa 946 -1476}\special{fp}\special{pa 984 -1476}\special{pa 1022 -1476}\special{fp}%
\special{pa 1060 -1476}\special{pa 1098 -1476}\special{fp}\special{pa 1136 -1476}\special{pa 1174 -1476}\special{fp}%
\special{pa 1211 -1476}\special{pa 1249 -1476}\special{fp}\special{pa 1287 -1476}\special{pa 1325 -1476}\special{fp}%
\special{pa 1363 -1476}\special{pa 1401 -1476}\special{fp}\special{pa 1439 -1476}\special{pa 1476 -1476}\special{fp}%
%
%
\special{pa 1401 24}\special{pa 1476 0}\special{pa 1401 -24}\special{pa 1416 0}\special{pa 1401 24}%
\special{pa 1401 24}\special{sh 1}\special{ip}%
\special{pn 1}%
\special{pa  1401    24}\special{pa  1476    -0}\special{pa  1401   -24}\special{pa  1416    -0}%
\special{pa  1401    24}%
\special{fp}%
\special{pn 8}%
\special{pa     0    -0}\special{pa  1416    -0}%
\special{fp}%
\special{pa 24 -1401}\special{pa 0 -1476}\special{pa -24 -1401}\special{pa 0 -1416}%
\special{pa 24 -1401}\special{pa 24 -1401}\special{sh 1}\special{ip}%
\special{pn 1}%
\special{pa    24 -1401}\special{pa     0 -1476}\special{pa   -24 -1401}\special{pa     0 -1416}%
\special{pa    24 -1401}%
\special{fp}%
\special{pn 8}%
\special{pa     0    -0}\special{pa     0 -1416}%
\special{fp}%
\special{pn 16}%
\special{pa     0    -0}\special{pa    12   -39}\special{pa    24   -77}\special{pa    36  -112}%
\special{pa    47  -145}\special{pa    58  -176}\special{pa    69  -204}\special{pa    80  -231}%
\special{pa    90  -256}\special{pa   100  -278}\special{pa   110  -298}\special{pa   125  -327}%
\special{pa   140  -357}\special{pa   156  -386}\special{pa   173  -415}\special{pa   191  -445}%
\special{pa   210  -474}\special{pa   230  -503}\special{pa   251  -533}\special{pa   272  -562}%
\special{pa   295  -591}\special{pa   322  -622}\special{pa   349  -653}\special{pa   377  -683}%
\special{pa   406  -713}\special{pa   435  -742}\special{pa   465  -772}\special{pa   496  -800}%
\special{pa   527  -829}\special{pa   559  -857}\special{pa   591  -886}\special{pa   620  -911}%
\special{pa   650  -936}\special{pa   679  -961}\special{pa   709  -984}\special{pa   739 -1007}%
\special{pa   769 -1030}\special{pa   798 -1052}\special{pa   828 -1073}\special{pa   858 -1095}%
\special{pa   887 -1115}\special{pa   917 -1135}\special{pa   947 -1155}\special{pa   977 -1174}%
\special{pa  1007 -1193}\special{pa  1037 -1211}\special{pa  1067 -1229}\special{pa  1097 -1246}%
\special{pa  1127 -1264}\special{pa  1157 -1282}\special{pa  1186 -1300}\special{pa  1206 -1312}%
\special{pa  1229 -1326}\special{pa  1253 -1341}\special{pa  1279 -1357}\special{pa  1307 -1374}%
\special{pa  1337 -1392}\special{pa  1369 -1411}\special{pa  1403 -1432}\special{pa  1439 -1454}%
\special{pa  1476 -1476}%
\special{fp}%
\special{pn 8}%
\settowidth{\Width}{$電圧\mathrm{〔V〕}$}\setlength{\Width}{-0.5\Width}%
\settoheight{\Height}{$電圧\mathrm{〔V〕}$}\settodepth{\Depth}{$電圧\mathrm{〔V〕}$}\setlength{\Height}{-0.5\Height}\setlength{\Depth}{0.5\Depth}\addtolength{\Height}{\Depth}%
\put(  1.260, -0.420){\hspace*{\Width}\raisebox{\Height}{$電圧\mathrm{〔V〕}$}}%
%
\settowidth{\Width}{電}\setlength{\Width}{-0.5\Width}%
\settoheight{\Height}{電}\settodepth{\Depth}{電}\setlength{\Height}{-0.5\Height}\setlength{\Depth}{0.5\Depth}\addtolength{\Height}{\Depth}%
\put( -0.630,  1.750){\hspace*{\Width}\raisebox{\Height}{電}}%
%
\settowidth{\Width}{流}\setlength{\Width}{-0.5\Width}%
\settoheight{\Height}{流}\settodepth{\Depth}{流}\setlength{\Height}{-0.5\Height}\setlength{\Depth}{0.5\Depth}\addtolength{\Height}{\Depth}%
\put( -0.630,  1.500){\hspace*{\Width}\raisebox{\Height}{流}}%
%
\settowidth{\Width}{$\mathrm{〔A〕}$}\setlength{\Width}{-0.5\Width}%
\settoheight{\Height}{$\mathrm{〔A〕}$}\settodepth{\Depth}{$\mathrm{〔A〕}$}\setlength{\Height}{-0.5\Height}\setlength{\Depth}{0.5\Depth}\addtolength{\Height}{\Depth}%
\put( -0.630,  1.250){\hspace*{\Width}\raisebox{\Height}{$\mathrm{〔A〕}$}}%
%
\settowidth{\Width}{$図1$}\setlength{\Width}{-0.5\Width}%
\settoheight{\Height}{$図1$}\settodepth{\Depth}{$図1$}\setlength{\Height}{-0.5\Height}\setlength{\Depth}{0.5\Depth}\addtolength{\Height}{\Depth}%
\put(  1.250, -0.750){\hspace*{\Width}\raisebox{\Height}{$図1$}}%
%
\end{picture}}%~~~~~~~~~~
        %%% C:/vpn/vpn/KeTCindy/fig/fig119_2.tex 
%%% Generator=fig119_2.cdy 
{\unitlength=1cm%
\begin{picture}%
(5.5,6)(-3,-3.5)%
\special{pn 8}%
%
\special{pa   -65   295}\special{pa   -66   285}\special{pa   -68   275}\special{pa   -71   265}%
\special{pa   -75   255}\special{pa   -81   247}\special{pa   -87   239}\special{pa   -95   232}%
\special{pa  -103   225}\special{pa  -112   220}\special{pa  -122   217}\special{pa  -132   214}%
\special{pa  -142   213}\special{pa  -153   213}\special{pa  -163   214}\special{pa  -173   217}%
\special{pa  -183   220}\special{pa  -192   225}\special{pa  -200   232}\special{pa  -208   239}%
\special{pa  -215   247}\special{pa  -220   255}\special{pa  -225   265}\special{pa  -228   275}%
\special{pa  -230   285}\special{pa  -230   295}\special{pa  -230   306}\special{pa  -228   316}%
\special{pa  -225   326}\special{pa  -220   335}\special{pa  -215   344}\special{pa  -208   352}%
\special{pa  -200   359}\special{pa  -192   365}\special{pa  -183   370}\special{pa  -173   374}%
\special{pa  -163   376}\special{pa  -153   378}\special{pa  -142   378}\special{pa  -132   376}%
\special{pa  -122   374}\special{pa  -112   370}\special{pa  -103   365}\special{pa   -95   359}%
\special{pa   -87   352}\special{pa   -81   344}\special{pa   -75   335}\special{pa   -71   326}%
\special{pa   -68   316}\special{pa   -66   306}\special{pa   -65   295}%
\special{fp}%
\special{pa   -89   237}\special{pa  -206   354}%
\special{fp}%
\special{pa  -206   237}\special{pa   -89   354}%
\special{fp}%
\special{pa  -591   295}\special{pa  -230   295}%
\special{fp}%
\special{pa   295   295}\special{pa   -65   295}%
\special{fp}%
\special{pa  -504  -394}\special{pa  -313  -504}%
\special{fp}%
\special{pa  -591  -394}\special{pa  -504  -394}%
\special{fp}%
\special{pa  -197  -394}\special{pa  -283  -394}%
\special{fp}%
{%
\color[cmyk]{0,0,0,0}%
\special{pa -489 -394}\special{pa -489 -396}\special{pa -489 -397}\special{pa -490 -399}%
\special{pa -491 -401}\special{pa -492 -402}\special{pa -493 -404}\special{pa -494 -405}%
\special{pa -496 -406}\special{pa -498 -407}\special{pa -499 -408}\special{pa -501 -408}%
\special{pa -503 -409}\special{pa -505 -409}\special{pa -507 -408}\special{pa -509 -408}%
\special{pa -510 -407}\special{pa -512 -406}\special{pa -513 -405}\special{pa -515 -404}%
\special{pa -516 -402}\special{pa -517 -401}\special{pa -518 -399}\special{pa -518 -397}%
\special{pa -519 -396}\special{pa -519 -394}\special{pa -519 -392}\special{pa -518 -390}%
\special{pa -518 -388}\special{pa -517 -386}\special{pa -516 -385}\special{pa -515 -383}%
\special{pa -513 -382}\special{pa -512 -381}\special{pa -510 -380}\special{pa -509 -379}%
\special{pa -507 -379}\special{pa -505 -379}\special{pa -503 -379}\special{pa -501 -379}%
\special{pa -499 -379}\special{pa -498 -380}\special{pa -496 -381}\special{pa -494 -382}%
\special{pa -493 -383}\special{pa -492 -385}\special{pa -491 -386}\special{pa -490 -388}%
\special{pa -489 -390}\special{pa -489 -392}\special{pa -489 -394}\special{pa -489 -394}%
\special{sh 1}\special{ip}%
}%
\special{pa  -489  -394}\special{pa  -489  -396}\special{pa  -489  -397}\special{pa  -490  -399}%
\special{pa  -491  -401}\special{pa  -492  -402}\special{pa  -493  -404}\special{pa  -494  -405}%
\special{pa  -496  -406}\special{pa  -498  -407}\special{pa  -499  -408}\special{pa  -501  -408}%
\special{pa  -503  -409}\special{pa  -505  -409}\special{pa  -507  -408}\special{pa  -509  -408}%
\special{pa  -510  -407}\special{pa  -512  -406}\special{pa  -513  -405}\special{pa  -515  -404}%
\special{pa  -516  -402}\special{pa  -517  -401}\special{pa  -518  -399}\special{pa  -518  -397}%
\special{pa  -519  -396}\special{pa  -519  -394}\special{pa  -519  -392}\special{pa  -518  -390}%
\special{pa  -518  -388}\special{pa  -517  -386}\special{pa  -516  -385}\special{pa  -515  -383}%
\special{pa  -513  -382}\special{pa  -512  -381}\special{pa  -510  -380}\special{pa  -509  -379}%
\special{pa  -507  -379}\special{pa  -505  -379}\special{pa  -503  -379}\special{pa  -501  -379}%
\special{pa  -499  -379}\special{pa  -498  -380}\special{pa  -496  -381}\special{pa  -494  -382}%
\special{pa  -493  -383}\special{pa  -492  -385}\special{pa  -491  -386}\special{pa  -490  -388}%
\special{pa  -489  -390}\special{pa  -489  -392}\special{pa  -489  -394}%
\special{fp}%
{%
\color[cmyk]{0,0,0,0}%
\special{pa -269 -394}\special{pa -269 -396}\special{pa -269 -397}\special{pa -270 -399}%
\special{pa -270 -401}\special{pa -271 -402}\special{pa -273 -404}\special{pa -274 -405}%
\special{pa -275 -406}\special{pa -277 -407}\special{pa -279 -408}\special{pa -281 -408}%
\special{pa -283 -409}\special{pa -284 -409}\special{pa -286 -408}\special{pa -288 -408}%
\special{pa -290 -407}\special{pa -291 -406}\special{pa -293 -405}\special{pa -294 -404}%
\special{pa -296 -402}\special{pa -297 -401}\special{pa -297 -399}\special{pa -298 -397}%
\special{pa -298 -396}\special{pa -298 -394}\special{pa -298 -392}\special{pa -298 -390}%
\special{pa -297 -388}\special{pa -297 -386}\special{pa -296 -385}\special{pa -294 -383}%
\special{pa -293 -382}\special{pa -291 -381}\special{pa -290 -380}\special{pa -288 -379}%
\special{pa -286 -379}\special{pa -284 -379}\special{pa -283 -379}\special{pa -281 -379}%
\special{pa -279 -379}\special{pa -277 -380}\special{pa -275 -381}\special{pa -274 -382}%
\special{pa -273 -383}\special{pa -271 -385}\special{pa -270 -386}\special{pa -270 -388}%
\special{pa -269 -390}\special{pa -269 -392}\special{pa -269 -394}\special{pa -269 -394}%
\special{sh 1}\special{ip}%
}%
\special{pa  -269  -394}\special{pa  -269  -396}\special{pa  -269  -397}\special{pa  -270  -399}%
\special{pa  -270  -401}\special{pa  -271  -402}\special{pa  -273  -404}\special{pa  -274  -405}%
\special{pa  -275  -406}\special{pa  -277  -407}\special{pa  -279  -408}\special{pa  -281  -408}%
\special{pa  -283  -409}\special{pa  -284  -409}\special{pa  -286  -408}\special{pa  -288  -408}%
\special{pa  -290  -407}\special{pa  -291  -406}\special{pa  -293  -405}\special{pa  -294  -404}%
\special{pa  -296  -402}\special{pa  -297  -401}\special{pa  -297  -399}\special{pa  -298  -397}%
\special{pa  -298  -396}\special{pa  -298  -394}\special{pa  -298  -392}\special{pa  -298  -390}%
\special{pa  -297  -388}\special{pa  -297  -386}\special{pa  -296  -385}\special{pa  -294  -383}%
\special{pa  -293  -382}\special{pa  -291  -381}\special{pa  -290  -380}\special{pa  -288  -379}%
\special{pa  -286  -379}\special{pa  -284  -379}\special{pa  -283  -379}\special{pa  -281  -379}%
\special{pa  -279  -379}\special{pa  -277  -380}\special{pa  -275  -381}\special{pa  -274  -382}%
\special{pa  -273  -383}\special{pa  -271  -385}\special{pa  -270  -386}\special{pa  -270  -388}%
\special{pa  -269  -390}\special{pa  -269  -392}\special{pa  -269  -394}%
\special{fp}%
\special{pa   167  -453}\special{pa   -69  -453}\special{pa   -69  -335}\special{pa   167  -335}%
\special{pa   167  -453}%
\special{fp}%
\special{pa  -197  -394}\special{pa   -69  -394}%
\special{fp}%
\special{pa   295  -394}\special{pa   167  -394}%
\special{fp}%
\special{pa   659   -59}\special{pa   423   -59}\special{pa   423    59}\special{pa   659    59}%
\special{pa   659   -59}%
\special{fp}%
\special{pa   295    -0}\special{pa   423    -0}%
\special{fp}%
\special{pa   787    -0}\special{pa   659    -0}%
\special{fp}%
\special{pa   -20   787}\special{pa   -20   591}%
\special{fp}%
\special{pn 16}%
\special{pa    20   728}\special{pa    20   650}%
\special{fp}%
\special{pn 8}%
\special{pa   787   689}\special{pa    20   689}%
\special{fp}%
\special{pa  -787   689}\special{pa   -20   689}%
\special{fp}%
\special{pa  -787   689}\special{pa  -787    -0}\special{pa  -591    -0}%
\special{fp}%
\special{pa  -591  -394}\special{pa  -591   295}%
\special{fp}%
\special{pa   295  -394}\special{pa   295   295}%
\special{fp}%
\special{pa   787    -0}\special{pa   787   689}%
\special{fp}%
\special{pa -576 0}\special{pa -576 -2}\special{pa -576 -4}\special{pa -577 -6}\special{pa -577 -7}%
\special{pa -578 -9}\special{pa -580 -10}\special{pa -581 -12}\special{pa -583 -13}%
\special{pa -584 -14}\special{pa -586 -14}\special{pa -588 -15}\special{pa -590 -15}%
\special{pa -591 -15}\special{pa -593 -15}\special{pa -595 -14}\special{pa -597 -14}%
\special{pa -599 -13}\special{pa -600 -12}\special{pa -601 -10}\special{pa -603 -9}%
\special{pa -604 -7}\special{pa -604 -6}\special{pa -605 -4}\special{pa -605 -2}\special{pa -606 0}%
\special{pa -605 2}\special{pa -605 4}\special{pa -604 6}\special{pa -604 7}\special{pa -603 9}%
\special{pa -601 10}\special{pa -600 12}\special{pa -599 13}\special{pa -597 14}\special{pa -595 14}%
\special{pa -593 15}\special{pa -591 15}\special{pa -590 15}\special{pa -588 15}\special{pa -586 14}%
\special{pa -584 14}\special{pa -583 13}\special{pa -581 12}\special{pa -580 10}\special{pa -578 9}%
\special{pa -577 7}\special{pa -577 6}\special{pa -576 4}\special{pa -576 2}\special{pa -576 0}%
\special{pa -576 0}\special{sh 1}\special{ip}%
\special{pa  -576    -0}\special{pa  -576    -2}\special{pa  -576    -4}\special{pa  -577    -6}%
\special{pa  -577    -7}\special{pa  -578    -9}\special{pa  -580   -10}\special{pa  -581   -12}%
\special{pa  -583   -13}\special{pa  -584   -14}\special{pa  -586   -14}\special{pa  -588   -15}%
\special{pa  -590   -15}\special{pa  -591   -15}\special{pa  -593   -15}\special{pa  -595   -14}%
\special{pa  -597   -14}\special{pa  -599   -13}\special{pa  -600   -12}\special{pa  -601   -10}%
\special{pa  -603    -9}\special{pa  -604    -7}\special{pa  -604    -6}\special{pa  -605    -4}%
\special{pa  -605    -2}\special{pa  -606     0}\special{pa  -605     2}\special{pa  -605     4}%
\special{pa  -604     6}\special{pa  -604     7}\special{pa  -603     9}\special{pa  -601    10}%
\special{pa  -600    12}\special{pa  -599    13}\special{pa  -597    14}\special{pa  -595    14}%
\special{pa  -593    15}\special{pa  -591    15}\special{pa  -590    15}\special{pa  -588    15}%
\special{pa  -586    14}\special{pa  -584    14}\special{pa  -583    13}\special{pa  -581    12}%
\special{pa  -580    10}\special{pa  -578     9}\special{pa  -577     7}\special{pa  -577     6}%
\special{pa  -576     4}\special{pa  -576     2}\special{pa  -576     0}%
\special{fp}%
\special{pa 310 0}\special{pa 310 -2}\special{pa 310 -4}\special{pa 309 -6}\special{pa 308 -7}%
\special{pa 307 -9}\special{pa 306 -10}\special{pa 305 -12}\special{pa 303 -13}\special{pa 302 -14}%
\special{pa 300 -14}\special{pa 298 -15}\special{pa 296 -15}\special{pa 294 -15}\special{pa 292 -15}%
\special{pa 291 -14}\special{pa 289 -14}\special{pa 287 -13}\special{pa 286 -12}\special{pa 284 -10}%
\special{pa 283 -9}\special{pa 282 -7}\special{pa 281 -6}\special{pa 281 -4}\special{pa 280 -2}%
\special{pa 280 0}\special{pa 280 2}\special{pa 281 4}\special{pa 281 6}\special{pa 282 7}%
\special{pa 283 9}\special{pa 284 10}\special{pa 286 12}\special{pa 287 13}\special{pa 289 14}%
\special{pa 291 14}\special{pa 292 15}\special{pa 294 15}\special{pa 296 15}\special{pa 298 15}%
\special{pa 300 14}\special{pa 302 14}\special{pa 303 13}\special{pa 305 12}\special{pa 306 10}%
\special{pa 307 9}\special{pa 308 7}\special{pa 309 6}\special{pa 310 4}\special{pa 310 2}%
\special{pa 310 0}\special{pa 310 0}\special{sh 1}\special{ip}%
\special{pa   310    -0}\special{pa   310    -2}\special{pa   310    -4}\special{pa   309    -6}%
\special{pa   308    -7}\special{pa   307    -9}\special{pa   306   -10}\special{pa   305   -12}%
\special{pa   303   -13}\special{pa   302   -14}\special{pa   300   -14}\special{pa   298   -15}%
\special{pa   296   -15}\special{pa   294   -15}\special{pa   292   -15}\special{pa   291   -14}%
\special{pa   289   -14}\special{pa   287   -13}\special{pa   286   -12}\special{pa   284   -10}%
\special{pa   283    -9}\special{pa   282    -7}\special{pa   281    -6}\special{pa   281    -4}%
\special{pa   280    -2}\special{pa   280     0}\special{pa   280     2}\special{pa   281     4}%
\special{pa   281     6}\special{pa   282     7}\special{pa   283     9}\special{pa   284    10}%
\special{pa   286    12}\special{pa   287    13}\special{pa   289    14}\special{pa   291    14}%
\special{pa   292    15}\special{pa   294    15}\special{pa   296    15}\special{pa   298    15}%
\special{pa   300    14}\special{pa   302    14}\special{pa   303    13}\special{pa   305    12}%
\special{pa   306    10}\special{pa   307     9}\special{pa   308     7}\special{pa   309     6}%
\special{pa   310     4}\special{pa   310     2}\special{pa   310     0}%
\special{fp}%
\settowidth{\Width}{L}\setlength{\Width}{-0.5\Width}%
\settoheight{\Height}{L}\settodepth{\Depth}{L}\setlength{\Height}{\Depth}%
\put( -0.380, -0.400){\hspace*{\Width}\raisebox{\Height}{L}}%
%
\settowidth{\Width}{S}\setlength{\Width}{-0.5\Width}%
\settoheight{\Height}{S}\settodepth{\Depth}{S}\setlength{\Height}{\Depth}%
\put( -1.000,  1.350){\hspace*{\Width}\raisebox{\Height}{S}}%
%
\settowidth{\Width}{図2}\setlength{\Width}{-0.5\Width}%
\settoheight{\Height}{図2}\settodepth{\Depth}{図2}\setlength{\Height}{-0.5\Height}\setlength{\Depth}{0.5\Depth}\addtolength{\Height}{\Depth}%
\put(  0.250, -2.750){\hspace*{\Width}\raisebox{\Height}{図2}}%
%
\settowidth{\Width}{$50{\sf \Omega}$}\setlength{\Width}{-0.5\Width}%
\settoheight{\Height}{$50{\sf \Omega}$}\settodepth{\Depth}{$50{\sf \Omega}$}\setlength{\Height}{\Depth}%
\put(  0.120,  1.350){\hspace*{\Width}\raisebox{\Height}{$50{\sf \Omega}$}}%
%
\settowidth{\Width}{$100{\sf \Omega}$}\setlength{\Width}{-0.5\Width}%
\settoheight{\Height}{$100{\sf \Omega}$}\settodepth{\Depth}{$100{\sf \Omega}$}\setlength{\Height}{\Depth}%
\put(  1.380,  0.350){\hspace*{\Width}\raisebox{\Height}{$100{\sf \Omega}$}}%
%
\settowidth{\Width}{$100{\sf V}$}\setlength{\Width}{-0.5\Width}%
\settoheight{\Height}{$100{\sf V}$}\settodepth{\Depth}{$100{\sf V}$}\setlength{\Height}{-\Height}%
\put(  0.000, -2.150){\hspace*{\Width}\raisebox{\Height}{$100{\sf V}$}}%
%
\end{picture}}%
    \end{center}
    ~~まず,スイッチSが開いている状態を考える。Lにかかる電圧は\Hako \tanni{V}で,流れる電流は\Hako \tanni{A}である。このときのLの消費電力は\Hako \tanni{W}である。\\
    ~~次に,Sが閉じた状態を考える。このときのLの抵抗値は\Hako \tanni{\Omega }である。また,100\sftanni{\Omega }の抵抗には\Hako \tanni{A}の電流が流れる。\\
    ~~最後に,50\sftanni{\Omega }の抵抗を電球Lに取り替えて,2つのLを並列にし,Sを閉じる。このとき回路全体での消費電力は\Hako \tanni{W}となる。
    % \end{mawarikomi}