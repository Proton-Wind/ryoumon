\item
    \begin{mawarikomi}{180pt}{
        %WinTpicVersion4.32a
{\unitlength 0.1in%
\begin{picture}(21.5000,11.0000)(2.0000,-12.0000)%
% BOX 2 5 0 0 Black Black  
% 2 200 350 2300 400
% 
\special{pn 0}%
\special{sh 0.400}%
\special{pa 200 350}%
\special{pa 2300 350}%
\special{pa 2300 400}%
\special{pa 200 400}%
\special{pa 200 350}%
\special{ip}%
\special{pn 8}%
\special{pa 200 350}%
\special{pa 2300 350}%
\special{pa 2300 400}%
\special{pa 200 400}%
\special{pa 200 350}%
\special{ip}%
% DOT 0 5 3 0 Black White  
% 5 300 370 700 370 1200 370 1700 370 2200 370
% 
\special{pn 4}%
\special{sh 1}%
\special{ar 300 370 16 16 0 6.2831853}%
\special{sh 1}%
\special{ar 700 370 16 16 0 6.2831853}%
\special{sh 1}%
\special{ar 1200 370 16 16 0 6.2831853}%
\special{sh 1}%
\special{ar 1700 370 16 16 0 6.2831853}%
\special{sh 1}%
\special{ar 2200 370 16 16 0 6.2831853}%
% BOX 2 5 0 0 Black Black  
% 2 200 750 2300 800
% 
\special{pn 0}%
\special{sh 0.400}%
\special{pa 200 750}%
\special{pa 2300 750}%
\special{pa 2300 800}%
\special{pa 200 800}%
\special{pa 200 750}%
\special{ip}%
\special{pn 8}%
\special{pa 200 750}%
\special{pa 2300 750}%
\special{pa 2300 800}%
\special{pa 200 800}%
\special{pa 200 750}%
\special{ip}%
% DOT 0 5 3 0 Black White  
% 5 300 770 700 770 1200 770 1700 770 2200 770
% 
\special{pn 4}%
\special{sh 1}%
\special{ar 300 770 16 16 0 6.2831853}%
\special{sh 1}%
\special{ar 700 770 16 16 0 6.2831853}%
\special{sh 1}%
\special{ar 1200 770 16 16 0 6.2831853}%
\special{sh 1}%
\special{ar 1700 770 16 16 0 6.2831853}%
\special{sh 1}%
\special{ar 2200 770 16 16 0 6.2831853}%
% BOX 2 5 0 0 Black Black  
% 2 200 1150 2300 1200
% 
\special{pn 0}%
\special{sh 0.400}%
\special{pa 200 1150}%
\special{pa 2300 1150}%
\special{pa 2300 1200}%
\special{pa 200 1200}%
\special{pa 200 1150}%
\special{ip}%
\special{pn 8}%
\special{pa 200 1150}%
\special{pa 2300 1150}%
\special{pa 2300 1200}%
\special{pa 200 1200}%
\special{pa 200 1150}%
\special{ip}%
% DOT 0 5 3 0 Black White  
% 5 300 1170 700 1170 1200 1170 1700 1170 2200 1170
% 
\special{pn 4}%
\special{sh 1}%
\special{ar 300 1170 16 16 0 6.2831853}%
\special{sh 1}%
\special{ar 700 1170 16 16 0 6.2831853}%
\special{sh 1}%
\special{ar 1200 1170 16 16 0 6.2831853}%
\special{sh 1}%
\special{ar 1700 1170 16 16 0 6.2831853}%
\special{sh 1}%
\special{ar 2200 1170 16 16 0 6.2831853}%
% LINE 2 0 3 0 Black White  
% 2 700 100 1200 375
% 
\special{pn 8}%
\special{pa 700 100}%
\special{pa 1200 375}%
\special{fp}%
% LINE 2 0 3 1 Black White  
% 2 700 500 1200 775
% 
\special{pn 8}%
\special{pa 700 500}%
\special{pa 1200 775}%
\special{fp}%
% LINE 2 0 3 2 Black White  
% 2 700 900 1200 1175
% 
\special{pn 8}%
\special{pa 700 900}%
\special{pa 1200 1175}%
\special{fp}%
% LINE 2 0 3 3 Black White  
% 2 200 625 700 900
% 
\special{pn 8}%
\special{pa 200 625}%
\special{pa 700 900}%
\special{fp}%
% LINE 2 0 3 4 Black White  
% 2 200 225 700 500
% 
\special{pn 8}%
\special{pa 200 225}%
\special{pa 700 500}%
\special{fp}%
% LINE 2 0 3 0 Black White  
% 2 1700 100 1200 375
% 
\special{pn 8}%
\special{pa 1700 100}%
\special{pa 1200 375}%
\special{fp}%
% LINE 2 0 3 1 Black White  
% 2 1700 500 1200 775
% 
\special{pn 8}%
\special{pa 1700 500}%
\special{pa 1200 775}%
\special{fp}%
% LINE 2 0 3 2 Black White  
% 2 1700 900 1200 1175
% 
\special{pn 8}%
\special{pa 1700 900}%
\special{pa 1200 1175}%
\special{fp}%
% LINE 2 0 3 3 Black White  
% 2 2200 625 1700 900
% 
\special{pn 8}%
\special{pa 2200 625}%
\special{pa 1700 900}%
\special{fp}%
% LINE 2 0 3 4 Black White  
% 2 2200 225 1700 500
% 
\special{pn 8}%
\special{pa 2200 225}%
\special{pa 1700 500}%
\special{fp}%
% VECTOR 2 0 3 0 Black White  
% 4 875 197 1005 267 969 256 969 256
% 
\special{pn 8}%
\special{pa 875 197}%
\special{pa 1005 267}%
\special{fp}%
\special{sh 1}%
\special{pa 1005 267}%
\special{pa 956 218}%
\special{pa 958 242}%
\special{pa 937 253}%
\special{pa 1005 267}%
\special{fp}%
\special{pa 969 256}%
\special{pa 969 256}%
\special{fp}%
% VECTOR 2 0 3 0 Black White  
% 4 695 497 825 567 789 556 789 556
% 
\special{pn 8}%
\special{pa 695 497}%
\special{pa 825 567}%
\special{fp}%
\special{sh 1}%
\special{pa 825 567}%
\special{pa 776 518}%
\special{pa 778 542}%
\special{pa 757 553}%
\special{pa 825 567}%
\special{fp}%
\special{pa 789 556}%
\special{pa 789 556}%
\special{fp}%
% VECTOR 2 0 3 0 Black White  
% 4 535 809 665 879 629 868 629 868
% 
\special{pn 8}%
\special{pa 535 809}%
\special{pa 665 879}%
\special{fp}%
\special{sh 1}%
\special{pa 665 879}%
\special{pa 616 830}%
\special{pa 618 854}%
\special{pa 597 865}%
\special{pa 665 879}%
\special{fp}%
\special{pa 629 868}%
\special{pa 629 868}%
\special{fp}%
% VECTOR 2 0 3 0 Black White  
% 4 1395 268 1525 198 1489 209 1489 209
% 
\special{pn 8}%
\special{pa 1395 268}%
\special{pa 1525 198}%
\special{fp}%
\special{sh 1}%
\special{pa 1525 198}%
\special{pa 1457 212}%
\special{pa 1478 223}%
\special{pa 1476 247}%
\special{pa 1525 198}%
\special{fp}%
\special{pa 1489 209}%
\special{pa 1489 209}%
\special{fp}%
% VECTOR 2 0 3 0 Black White  
% 4 1539 588 1669 518 1633 529 1633 529
% 
\special{pn 8}%
\special{pa 1539 588}%
\special{pa 1669 518}%
\special{fp}%
\special{sh 1}%
\special{pa 1669 518}%
\special{pa 1601 532}%
\special{pa 1622 543}%
\special{pa 1620 567}%
\special{pa 1669 518}%
\special{fp}%
\special{pa 1633 529}%
\special{pa 1633 529}%
\special{fp}%
% VECTOR 2 0 3 0 Black White  
% 4 1679 911 1809 841 1773 852 1773 852
% 
\special{pn 8}%
\special{pa 1679 911}%
\special{pa 1809 841}%
\special{fp}%
\special{sh 1}%
\special{pa 1809 841}%
\special{pa 1741 855}%
\special{pa 1762 866}%
\special{pa 1760 890}%
\special{pa 1809 841}%
\special{fp}%
\special{pa 1773 852}%
\special{pa 1773 852}%
\special{fp}%
% CIRCLE 2 0 3 0 Black White  
% 4 1204 351 1004 351 604 96 604 351
% 
\special{pn 8}%
\special{ar 1204 351 200 200 3.1415927 3.5434633}%
% STR 2 0 3 0 Black White  
% 4 880 275 880 325 3 0 0 0
% $\theta$
\put(8.8000,-3.2500){\makebox(0,0)[rb]{$\theta$}}%
% STR 2 0 3 0 Black White  
% 4 880 650 880 700 3 0 0 0
% $\theta$
\put(8.8000,-7.0000){\makebox(0,0)[rb]{$\theta$}}%
% CIRCLE 2 0 3 0 Black White  
% 4 1204 751 1004 751 604 496 604 751
% 
\special{pn 8}%
\special{ar 1204 751 200 200 3.1415927 3.5434633}%
% VECTOR 2 0 3 0 Black White  
% 8 2329 571 2329 771 2329 571 2329 371 2329 971 2329 771 2329 971 2329 1171
% 
\special{pn 8}%
\special{pa 2329 571}%
\special{pa 2329 771}%
\special{fp}%
\special{sh 1}%
\special{pa 2329 771}%
\special{pa 2349 704}%
\special{pa 2329 718}%
\special{pa 2309 704}%
\special{pa 2329 771}%
\special{fp}%
\special{pa 2329 571}%
\special{pa 2329 371}%
\special{fp}%
\special{sh 1}%
\special{pa 2329 371}%
\special{pa 2309 438}%
\special{pa 2329 424}%
\special{pa 2349 438}%
\special{pa 2329 371}%
\special{fp}%
\special{pa 2329 971}%
\special{pa 2329 771}%
\special{fp}%
\special{sh 1}%
\special{pa 2329 771}%
\special{pa 2309 838}%
\special{pa 2329 824}%
\special{pa 2349 838}%
\special{pa 2329 771}%
\special{fp}%
\special{pa 2329 971}%
\special{pa 2329 1171}%
\special{fp}%
\special{sh 1}%
\special{pa 2329 1171}%
\special{pa 2349 1104}%
\special{pa 2329 1118}%
\special{pa 2309 1104}%
\special{pa 2329 1171}%
\special{fp}%
% STR 2 0 3 0 Black White  
% 4 2429 521 2429 571 5 0 0 0
% $d$
\put(24.2900,-5.7100){\makebox(0,0){$d$}}%
% STR 2 0 3 0 Black White  
% 4 2425 920 2425 970 5 0 0 0
% $d$
\put(24.2500,-9.7000){\makebox(0,0){$d$}}%
\end{picture}}%

    }
結晶に入射した電子線(電子波)は,規則正しく並んだ原子によって散乱され,互いに \fbox{(1)} して特定の方向に強く反射することがある。結晶中の原子は格子面(原子面)上に並んでおり,入射した電子線は各格子面で鏡面のように反射すると考えられる。格子面に対して角度 $\theta$ で電子線が入射するとき,隣り合う2つの電子線の道のりの差(経路差)は,格子面間隔 $d$ と角度 $\theta$ で表すと \fbox{(2)} である。そして反射電子線が互いに強め合う条件は,電子線の波長を $\lambda$ とし,自然数 $n$ を用いると \fbox{(3)} と表される。
電気素量 $e = 1.6 \times 10^{-19}$ \sftanni{C},電子の質量 $m = 9.1 \times 10^{-31}$ \sftanni{kg},プランク定数 $h = 6.6 \times 10^{-34}$ \sftanni{J\cdot s} とする。静止している電子を $2.9 \times 10^2$ \sftanni{V} の電圧で加速したとき,電子の速さ $v = \fbox{(4)}$ \sftanni{m/s} となり,その波長は $\lambda = \fbox{(5)}$ \sftanni{m} となる。この電子線を角度 $\theta = 50^\circ$ で入射させ,そのあと $\theta$ を増加させると,強い反射が起こる角度がいくつかある。その最初の角度を $\theta_1$ とすると,$\sin\theta_1 = \fbox{(6)}$ である。ただし,$d = 3.5 \times 10^{-10}$ \sftanni{m} であり,$\sin 50^\circ = 0.77$ とする。$\theta$ を $50^\circ \leqq \theta < 90^\circ$ の範囲で変化させると \fbox{(7)} 回強い反射が起こる。
\end{mawarikomi}
