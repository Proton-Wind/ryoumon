\item
    \begin{mawarikomi}{150pt}{\begin{zahyou*}[ul=4.5mm,yscale=1.2,xscale=1.2](-1,13)(-1,13)

	% \drawline(-1,-1)(-1,13)(13,13)(13,-1)(-1,-1)
	\def\Fx{-0.5*sin(T)+10.2}
	\def\Fy{-0.2*(cos(T)+T/3)+10}
	\BGurafu\Fx\Fy{3.14}{20*3.14}
	\drawline(10,10)(10,11)
	\drawline(10,3)(10,5.628)
	\def\A{(1,3)}
	\def\B{(10,3)}
	\def\AD{(1,2.5)}
	\def\BD{(10,2.5)}
	\def\O{(0,2.5)}
	\def\P{(7,2.5)}
	\def\PU{(7,3)}
	\def\Q{(7,0)}
	\def\QR{(6.5,0)}
	\def\PSW{(6.5,2)}
	\def\OD{(0,2)}
	\def\R{(10,11)}
	\def\fvec{(0,2)}
	\def\base{(3,1)}
	\Drawlines{\A\B\BD\AD\A}
	\Drawlines{\O\P\Q}
	% \scriptsize
	\HenKo<henkotype=parallel
			,henkoH=5ex
			,yazirusi=b
			,henkosideb=0.3
			,henkosidet=1.5>\B\A{$L$}
	\HenKo<henkotype=parallel
			,henkoH=1.6ex
			,yazirusi=b
			,henkosideb=0.3
			,henkosidet=1.5>\B\PU{$\ell$}
	\Put\B[e]{A}
	\Put\A[nw]{B}
	\Put\P[se]{P}
	\Put\R{\Yasen\fvec}
	\Put\base{台}
	\Kuromaru\R
	\Nuritubusi*<.22>{\A\B\BD\AD\A}
	\Nuritubusi[0.3]{\OD\O\P\Q\QR\PSW\OD}


\end{zahyou*}
}
    長さ$L$の一様でまっすぐな棒ABが,台の上にその一部がはみだして置かれている。
    このとき,A端から長さ$\ell $だけ離れた点Pが台の端に当たっている。棒のA端にばね定数$k$のばねをつけて鉛直上方に引っ張ると,ばねが$a$だけ伸びたとき点Pが台の端を離れた。ただし,台の上の面は十分に粗くて棒は台に対してすべらないものとする。また重力加速度の大きさを$g$とし,$\ell < \bunsuu{1}{2}L$とする。
        \begin{enumerate}
            \item 棒の質量$m$を求めよ。また,点Pが台の端を離れるとき,棒が台から受ける垂直抗力$N$を求めよ。また,点Pが台の端を離れるとき,棒が台から受ける垂直抗力を求めよ。
            \item 次にばねをA端からはずし,B端につけかえて鉛直上方に引っ張ると,ばねが$b$だけ伸びたときにB端が台から離れた。$b$は$a$の何倍か。
        \end{enumerate}
    \end{mawarikomi}