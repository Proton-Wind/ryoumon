\item 電車がまっすぐなレールの上を一定の加速度$\alpha$\tanni{m/s^2}で走り出した。このとき電車の床の上で静止していた質量$M$\tanni{kg}の物体が,電車が走り出すと同時に床上を滑り始めた。物体と床との間の動摩擦係数を$\mu $,重力加速度の大きさを$g$\tanni{m/s^2}とする。
    \begin{enumerate}
        \item 車内の人から見て,物体に作用している慣性力の大きさと摩擦力の大きさはそれぞれいくらか。
        \item 車内の人が見た物体の加速度の大きさ$\beta $を$\alpha $,$g$,$\mu $を用いて表せ。
        \item 車内の人が見て,物体が床を$\ell $\tanni{m}滑るのに要した時間$t$\tanni{s}と,その時の速さ$v$\tanni{m/s}(車内の人が見た速さ)を$\alpha $,$g$,$\mu $,$\ell $を用いてそれぞれ表せ。
        \item 物体が滑り出したことから,静止摩擦係数$\mu _0$はいくらより小さいことが分かるか。$\alpha $,$g$を用いて表せ。
    \end{enumerate}