\hakosyokika
\item
    \begin{mawarikomi}[10](120pt,0pt){150pt}{%WinTpicVersion4.32a
{\unitlength 0.1in%
\begin{picture}(5.0197,2.4016)(13.3858,-8.8386)%
% SPLINE 2 0 3 0 Black White  
% 12 1870 656 1742 663 1636 706 1572 727 1466 790 1403 833 1360 876 1403 897 1509 854 1679 796 1730 790 1819 834
% 
\special{pn 8}%
\special{pa 1841 646}%
\special{pa 1808 645}%
\special{pa 1777 645}%
\special{pa 1745 647}%
\special{pa 1715 653}%
\special{pa 1685 662}%
\special{pa 1627 688}%
\special{pa 1597 699}%
\special{pa 1568 709}%
\special{pa 1538 719}%
\special{pa 1509 734}%
\special{pa 1482 751}%
\special{pa 1456 769}%
\special{pa 1432 785}%
\special{pa 1407 802}%
\special{pa 1378 822}%
\special{pa 1347 844}%
\special{pa 1341 866}%
\special{pa 1369 881}%
\special{pa 1405 883}%
\special{pa 1435 873}%
\special{pa 1462 856}%
\special{pa 1488 839}%
\special{pa 1516 824}%
\special{pa 1544 812}%
\special{pa 1575 802}%
\special{pa 1605 794}%
\special{pa 1637 786}%
\special{pa 1668 781}%
\special{pa 1700 778}%
\special{pa 1729 783}%
\special{pa 1757 798}%
\special{pa 1784 817}%
\special{pa 1790 821}%
\special{fp}%
% SPLINE 2 0 3 1 Black White  
% 17 1810 794 1818 824 1774 886 1710 970 1668 1034 1682 1056 1725 1065 1768 1032 1854 968 1938 926 2002 926 2108 884 2172 798 2172 798 2172 798 2196 795 2196 795
% 
\special{pn 8}%
\special{pa 1781 781}%
\special{pa 1789 812}%
\special{pa 1778 839}%
\special{pa 1754 863}%
\special{pa 1732 888}%
\special{pa 1715 913}%
\special{pa 1697 938}%
\special{pa 1674 964}%
\special{pa 1651 989}%
\special{pa 1642 1017}%
\special{pa 1659 1042}%
\special{pa 1691 1049}%
\special{pa 1719 1036}%
\special{pa 1742 1014}%
\special{pa 1766 993}%
\special{pa 1791 975}%
\special{pa 1844 940}%
\special{pa 1872 923}%
\special{pa 1902 912}%
\special{pa 1932 910}%
\special{pa 1964 911}%
\special{pa 1997 909}%
\special{pa 2029 902}%
\special{pa 2057 887}%
\special{pa 2080 864}%
\special{pa 2096 834}%
\special{pa 2112 805}%
\special{pa 2135 786}%
\special{pa 2161 782}%
\special{fp}%
% SPLINE 2 0 3 2 Black White  
% 4 1664 806 1645 848 1645 899 1722 956
% 
\special{pn 8}%
\special{pa 1638 793}%
\special{pa 1624 821}%
\special{pa 1616 852}%
\special{pa 1619 884}%
\special{pa 1635 908}%
\special{pa 1662 926}%
\special{pa 1694 941}%
\special{pa 1695 941}%
\special{fp}%
% SPLINE 2 0 3 3 Black White  
% 4 1756 811 1722 874 1697 874 1686 874
% 
\special{pn 8}%
\special{pa 1728 798}%
\special{pa 1719 835}%
\special{pa 1701 858}%
\special{pa 1671 860}%
\special{pa 1659 860}%
\special{fp}%
% SPLINE 2 0 3 4 Black White  
% 3 1730 872 1758 881 1769 888
% 
\special{pn 8}%
\special{pa 1703 858}%
\special{pa 1732 868}%
\special{pa 1741 874}%
\special{fp}%
% SPLINE 2 0 3 5 Black White  
% 4 1824 987 1867 999 1896 978 1883 948
% 
\special{pn 8}%
\special{pa 1795 971}%
\special{pa 1825 983}%
\special{pa 1857 975}%
\special{pa 1864 949}%
\special{pa 1853 933}%
\special{fp}%
% SPLINE 2 0 3 6 Black White  
% 5 1896 969 1924 981 1948 960 1952 937 1948 924
% 
\special{pn 8}%
\special{pa 1866 954}%
\special{pa 1895 966}%
\special{pa 1918 943}%
\special{pa 1918 912}%
\special{pa 1917 909}%
\special{fp}%
% SPLINE 2 0 3 7 Black White  
% 4 1952 936 1971 948 1994 935 1994 926
% 
\special{pn 8}%
\special{pa 1921 921}%
\special{pa 1950 932}%
\special{pa 1963 911}%
\special{fp}%
% SPLINE 2 0 3 8 Black White  
% 5 1366 884 1412 858 1428 836 1428 826 1424 820
% 
\special{pn 8}%
\special{pa 1344 870}%
\special{pa 1373 856}%
\special{pa 1398 837}%
\special{pa 1403 808}%
\special{pa 1402 807}%
\special{fp}%
% SPLINE 2 0 3 9 Black White  
% 8 1679 1033 1700 1054 1722 1055 1744 1034 1724 1016 1702 994 1680 1015 1679 1033
% 
\special{pn 8}%
\special{pa 1653 1017}%
\special{pa 1675 1038}%
\special{pa 1708 1031}%
\special{pa 1709 1009}%
\special{pa 1683 983}%
\special{pa 1659 988}%
\special{pa 1653 1017}%
\special{fp}%
% LINE 2 0 3 10 Black White  
% 4 800 900 1500 900 1500 950 800 950
% 
\special{pn 8}%
\special{pa 787 886}%
\special{pa 1476 886}%
\special{fp}%
\special{pa 1476 935}%
\special{pa 787 935}%
\special{fp}%
% CIRCLE 2 0 3 11 Black White  
% 4 800 925 800 900 800 900 800 950
% 
\special{pn 8}%
\special{ar 787 910 25 25 1.5707963 4.7123890}%
% CIRCLE 2 0 3 12 Black White  
% 4 1500 925 1500 950 1500 950 1500 900
% 
\special{pn 8}%
\special{ar 1476 910 25 25 4.7123890 1.5707963}%
% BOX 2 0 3 13 Black White  
% 2 1125 950 1175 1900
% 
\special{pn 8}%
\special{pa 1107 935}%
\special{pa 1156 935}%
\special{pa 1156 1870}%
\special{pa 1107 1870}%
\special{pa 1107 935}%
\special{pa 1156 935}%
\special{fp}%
% BOX 2 0 2 14 Black White  
% 2 925 1200 1375 1250
% 
\special{pn 0}%
\special{sh 0}%
\special{pa 910 1181}%
\special{pa 1353 1181}%
\special{pa 1353 1230}%
\special{pa 910 1230}%
\special{pa 910 1181}%
\special{ip}%
\special{pn 8}%
\special{pa 910 1181}%
\special{pa 1353 1181}%
\special{pa 1353 1230}%
\special{pa 910 1230}%
\special{pa 910 1181}%
\special{pa 1353 1181}%
\special{fp}%
% LINE 2 0 3 15 Black White  
% 4 925 1250 775 1250 1375 1250 1525 1250
% 
\special{pn 8}%
\special{pa 910 1230}%
\special{pa 763 1230}%
\special{fp}%
\special{pa 1353 1230}%
\special{pa 1501 1230}%
\special{fp}%
% CIRCLE 2 0 3 16 Black White  
% 4 1525 1350 1625 1350 1675 1350 1525 1250
% 
\special{pn 8}%
\special{ar 1501 1329 98 98 4.7123890 6.2831853}%
% LINE 2 0 3 17 Black White  
% 2 1625 1350 1625 2300
% 
\special{pn 8}%
\special{pa 1599 1329}%
\special{pa 1599 2264}%
\special{fp}%
% CIRCLE 2 0 3 18 Black White  
% 4 1525 2300 1625 2300 1525 2400 1625 2300
% 
\special{pn 8}%
\special{ar 1501 2264 98 98 6.2831853 1.5707963}%
% LINE 2 0 3 19 Black White  
% 2 1525 2400 775 2400
% 
\special{pn 8}%
\special{pa 1501 2362}%
\special{pa 763 2362}%
\special{fp}%
% CIRCLE 2 0 3 20 Black White  
% 4 775 1350 675 1350 775 1250 625 1350
% 
\special{pn 8}%
\special{ar 763 1329 98 98 3.1415927 4.7123890}%
% LINE 2 0 3 21 Black White  
% 2 675 1350 675 2300
% 
\special{pn 8}%
\special{pa 664 1329}%
\special{pa 664 2264}%
\special{fp}%
% CIRCLE 2 0 3 22 Black White  
% 4 775 2300 675 2300 675 2300 775 2400
% 
\special{pn 8}%
\special{ar 763 2264 98 98 1.5707963 3.1415927}%
% LINE 1 1 3 23 Black White  
% 2 1150 1900 1150 2300
% 
\special{pn 13}%
\special{pa 1132 1870}%
\special{pa 1132 2264}%
\special{da 0.030}%
% LINE 1 1 3 24 Black White  
% 4 1150 1900 1450 2150 1150 1900 850 2150
% 
\special{pn 13}%
\special{pa 1132 1870}%
\special{pa 1427 2116}%
\special{da 0.030}%
\special{pa 1132 1870}%
\special{pa 837 2116}%
\special{da 0.030}%
% STR 2 0 3 25 Black White  
% 4 1000 2175 1000 2200 5 0 0 0
% ?
\put(9.8425,-21.6535){\makebox(0,0){?}}%
% STR 2 0 3 26 Black White  
% 4 1300 2175 1300 2200 5 0 0 0
% ?
\put(12.7953,-21.6535){\makebox(0,0){?}}%
% LINE 2 0 3 27 Black White  
% 4 400 600 1100 600 1100 700 400 700
% 
\special{pn 8}%
\special{pa 394 591}%
\special{pa 1083 591}%
\special{fp}%
\special{pa 1083 689}%
\special{pa 394 689}%
\special{fp}%
% CIRCLE 2 0 3 28 Black White  
% 4 1100 650 1100 600 1100 800 1100 500
% 
\special{pn 8}%
\special{ar 1083 640 49 49 4.7123890 1.5707963}%
% LINE 2 0 3 29 Black White  
% 2 532 648 492 648
% 
\special{pn 8}%
\special{pa 524 638}%
\special{pa 484 638}%
\special{fp}%
% LINE 2 0 3 30 Black White  
% 2 512 628 512 668
% 
\special{pn 8}%
\special{pa 504 618}%
\special{pa 504 657}%
\special{fp}%
% LINE 2 0 3 31 Black White  
% 2 612 648 572 648
% 
\special{pn 8}%
\special{pa 602 638}%
\special{pa 563 638}%
\special{fp}%
% LINE 2 0 3 32 Black White  
% 2 592 628 592 668
% 
\special{pn 8}%
\special{pa 583 618}%
\special{pa 583 657}%
\special{fp}%
% LINE 2 0 3 33 Black White  
% 2 692 648 652 648
% 
\special{pn 8}%
\special{pa 681 638}%
\special{pa 642 638}%
\special{fp}%
% LINE 2 0 3 34 Black White  
% 2 672 628 672 668
% 
\special{pn 8}%
\special{pa 661 618}%
\special{pa 661 657}%
\special{fp}%
% LINE 2 0 3 35 Black White  
% 2 772 648 732 648
% 
\special{pn 8}%
\special{pa 760 638}%
\special{pa 720 638}%
\special{fp}%
% LINE 2 0 3 36 Black White  
% 2 752 628 752 668
% 
\special{pn 8}%
\special{pa 740 618}%
\special{pa 740 657}%
\special{fp}%
% LINE 2 0 3 37 Black White  
% 2 852 648 812 648
% 
\special{pn 8}%
\special{pa 839 638}%
\special{pa 799 638}%
\special{fp}%
% LINE 2 0 3 38 Black White  
% 2 832 628 832 668
% 
\special{pn 8}%
\special{pa 819 618}%
\special{pa 819 657}%
\special{fp}%
% LINE 2 0 3 39 Black White  
% 2 932 648 892 648
% 
\special{pn 8}%
\special{pa 917 638}%
\special{pa 878 638}%
\special{fp}%
% LINE 2 0 3 40 Black White  
% 2 912 628 912 668
% 
\special{pn 8}%
\special{pa 898 618}%
\special{pa 898 657}%
\special{fp}%
% LINE 2 0 3 41 Black White  
% 2 1012 648 972 648
% 
\special{pn 8}%
\special{pa 996 638}%
\special{pa 957 638}%
\special{fp}%
% LINE 2 0 3 42 Black White  
% 2 992 628 992 668
% 
\special{pn 8}%
\special{pa 976 618}%
\special{pa 976 657}%
\special{fp}%
% LINE 2 0 3 43 Black White  
% 2 1092 648 1052 648
% 
\special{pn 8}%
\special{pa 1075 638}%
\special{pa 1035 638}%
\special{fp}%
% LINE 2 0 3 44 Black White  
% 2 1072 628 1072 668
% 
\special{pn 8}%
\special{pa 1055 618}%
\special{pa 1055 657}%
\special{fp}%
\end{picture}}%
}
        手で触れているはく検電器の電極に,正に帯電したガラス棒を近づけた後,手を離してから次にガラス棒を遠ざけた場合の記述として最も適当なものを選べ。
        \begin{enumerate}[m]
            \item 閉じていたはくは,手を離しても変化せず,ガラス棒を遠ざけると開く。
            \item 閉じていたはくは,手を離すと開き,ガラス棒を遠ざけると再び閉じる。
            \item 開いていたはくは,手を離しても変化せず,ガラス棒を遠ざけると閉じる。
            \item 開いていたはくは,手を離すと閉じ,ガラス棒を遠ざけると再び開く。
        \end{enumerate}
        \begin{enumerate}
            \item 帯電していないはく検電器の電極に,正に帯電したガラス棒を近づけた後,手を電極に触れ,そして離す。次にガラス棒を遠ざけた場合にのはくのふるまいを上に習って記述せよ。
        \end{enumerate}
    \end{mawarikomi}
