\documentclass[a4paper,9pt]{jsarticle}
% \documentclass[b5j,9.5pt]{jsbook}
\usepackage[noalphabet]{pxchfon}
\setminchofont{UDDIGIKYOKASHON-R.TTC} 
\setgothicfont{UDDIGIKYOKASHON-B.TTC} 
% \setminchofont{BIZ-UDMINCHOM.TTC} 
% \setgothicfont{BIZ-UDGOTHICR.TTC} 
\usepackage{okumacro}
\usepackage{amsmath,amsthm,amssymb,fancybox}
\usepackage{enumerate,multicol}
\usepackage{ascmac,itembbox,emath,hako,scrpage,ulinej,emathP,emathPp,emathMw,emathEy}
\usepackage[version=4]{mhchem}
%\usepackage[draft]{graphicx}
\usepackage{graphicx}
\usepackage{picins}
% \usepackage{amsmath}
% \usepackage{amsthm,amssymb,color}
% \usepackage{enumerate,multicol,mystyle}
% \usepackage{picins,emath,emathP,emathPp,itembbox,hako}
% \usepackage{emathMw}
% \usepackage{graphicx}
%\usepackage{array}
\pagestyle{empty}
\setlength{\textheight}{265mm}
\setlength{\textwidth}{195mm}
\setlength{\oddsidemargin}{-15.4mm}
\setlength{\evensidemargin}{-15.4mm}
\setlength{\topmargin}{-25.4mm}
\renewcommand\labelenumi{\fbox{\bfseries{\sffamily{\theenumi}}}}
\renewcommand{\labelenumii}{(\arabic{enumii})}
\renewcommand{\labelenumiii}{\bfseries{\カタカナ{enumiii}.}}
\def\genshi#1#2#3{$^{#1}_{#2}${{\large \sffamily{#3}}}}
\def\gen#1{\large \sf{#1}}
\def\hic{cal/g$\cdot $K}
\def\hij{J/g$\cdot $K}
\def\tanni#1{$〔\mathrm{\sf #1}〕\kern -1pt$}%
\def\sftanni#1{$\kern 2pt{\mathrm{\sf #1}}$}
\def\gen#1#2#3{{\sf{\ce{_{#2}^{#1}#3}}}}

\begin{document}
\hakosyokika
\begin{center}
{\Large 物理補習(干渉2)}
\end{center}
\hfill ~\underline{~~~~~~番 氏名~~~~~~~~~~~~~~~~~~~~~~~~~~~~~~~~~~~}
\hakosyokika

\begin{enumerate}
% \hakosyokika
\item
    \begin{mawarikomi}(10pt,0){160pt}{%WinTpicVersion4.32a
{\unitlength 0.1in%
\begin{picture}(20.2756,15.1083)(3.9370,-17.0768)%
% LINE 1 0 3 0 Black White  
% 2 400 800 2400 800
% 
\special{pn 13}%
\special{pa 394 787}%
\special{pa 2362 787}%
\special{fp}%
% LINE 1 0 3 0 Black White  
% 2 2400 1600 400 1600
% 
\special{pn 13}%
\special{pa 2362 1575}%
\special{pa 394 1575}%
\special{fp}%
% LINE 2 0 3 0 Black White  
% 2 400 1600 1780 800
% 
\special{pn 8}%
\special{pa 394 1575}%
\special{pa 1752 787}%
\special{fp}%
% LINE 2 0 3 0 Black White  
% 2 580 800 1180 200
% 
\special{pn 8}%
\special{pa 571 787}%
\special{pa 1161 197}%
\special{fp}%
% LINE 2 0 3 0 Black White  
% 2 980 800 1580 200
% 
\special{pn 8}%
\special{pa 965 787}%
\special{pa 1555 197}%
\special{fp}%
% LINE 2 0 3 0 Black White  
% 2 1380 800 1980 200
% 
\special{pn 8}%
\special{pa 1358 787}%
\special{pa 1949 197}%
\special{fp}%
% LINE 2 0 3 0 Black White  
% 2 2180 800 2380 600
% 
\special{pn 8}%
\special{pa 2146 787}%
\special{pa 2343 591}%
\special{fp}%
% LINE 2 0 3 0 Black White  
% 2 1780 800 2380 200
% 
\special{pn 8}%
\special{pa 1752 787}%
\special{pa 2343 197}%
\special{fp}%
% LINE 2 0 3 0 Black White  
% 4 1380 800 400 1370 1000 800 400 1150
% 
\special{pn 8}%
\special{pa 1358 787}%
\special{pa 394 1348}%
\special{fp}%
\special{pa 984 787}%
\special{pa 394 1132}%
\special{fp}%
% LINE 2 0 3 0 Black White  
% 2 600 800 400 920
% 
\special{pn 8}%
\special{pa 591 787}%
\special{pa 394 906}%
\special{fp}%
% LINE 2 0 3 0 Black White  
% 2 800 1600 2180 800
% 
\special{pn 8}%
\special{pa 787 1575}%
\special{pa 2146 787}%
\special{fp}%
% LINE 2 0 3 0 Black White  
% 4 1200 1600 2400 910 1850 1290 1850 1290
% 
\special{pn 8}%
\special{pa 1181 1575}%
\special{pa 2362 896}%
\special{fp}%
\special{pa 1821 1270}%
\special{pa 1821 1270}%
\special{fp}%
% LINE 2 0 3 0 Black White  
% 2 1600 1600 2400 1140
% 
\special{pn 8}%
\special{pa 1575 1575}%
\special{pa 2362 1122}%
\special{fp}%
% LINE 2 0 3 0 Black White  
% 2 2000 1600 2400 1370
% 
\special{pn 8}%
\special{pa 1969 1575}%
\special{pa 2362 1348}%
\special{fp}%
% CIRCLE 2 0 3 0 Black White  
% 4 1380 800 1580 800 2380 800 1980 200
% 
\special{pn 8}%
\special{ar 1358 787 197 197 5.4977871 6.2831853}%
% STR 2 0 3 0 Black White  
% 4 1610 600 1610 700 2 0 0 0
% 45\Deg
\put(15.8465,-6.8898){\makebox(0,0)[lb]{45\Deg}}%
% CIRCLE 2 0 3 0 Black White  
% 4 1380 800 1580 800 780 800 370 1380
% 
\special{pn 8}%
\special{ar 1358 787 197 197 2.6203166 3.1415927}%
% STR 2 0 3 0 Black White  
% 4 970 860 970 960 2 0 0 0
% 30\Deg
\put(9.5472,-9.4488){\makebox(0,0)[lb]{30\Deg}}%
% STR 2 0 3 0 Black White  
% 4 2490 700 2490 800 5 0 0 0
% A
\put(24.5079,-7.8740){\makebox(0,0){A}}%
% STR 2 0 3 0 Black White  
% 4 2490 1500 2490 1600 5 0 0 0
% B
\put(24.5079,-15.7480){\makebox(0,0){B}}%
% STR 2 0 3 0 Black White  
% 4 2180 300 2180 400 5 0 1 0
% 媒質1
\put(21.4567,-3.9370){\makebox(0,0){{\colorbox[named]{White}{媒質1}}}}%
% STR 2 0 3 0 Black White  
% 4 2180 1100 2180 1200 5 0 1 0
% 媒質2
\put(21.4567,-11.8110){\makebox(0,0){{\colorbox[named]{White}{媒質2}}}}%
% STR 2 0 3 0 Black White  
% 4 2180 1700 2180 1800 5 0 1 0
% 媒質3
\put(21.4567,-17.7165){\makebox(0,0){{\colorbox[named]{White}{媒質3}}}}%
\end{picture}}%
}
    図のように,平行な境界面A,Bで接した3種の媒質1,2,3がある。媒質1から入射した平面波の一部が屈折して媒質2へ入っていく。
    図中の平行線は入射波と屈折波の波面を表している。
     媒質1における波の波長は$2.0$\sftanni{cm},振動数は$25$\sftanni{Hz}である。
        \begin{enumerate}
            \item 媒質1に対する媒質2の屈折率はいくらか。
            \item 媒質1の中での波の速さは何\sftanni{cm/s}か。
            \item 媒質2の中での,波の波長は\sftanni{cm}か。振動数は何\sftanni{Hz}か。速さは何\sftanni{cm/s}か。
            \item 媒質1に対する媒質3の屈折率は0.80であった。媒質2に対する媒質3の屈折率はいくらか。
            \item 境界面Bで反射された波は,媒質2を通って,その一部が媒質1へもどる。そのときの屈折角は何度か。
        \end{enumerate}
    \end{mawarikomi}
%  \vfill
%  \newpage
%  \hakosyokika
\item
    \begin{mawarikomi}(10pt,0){160pt}{%WinTpicVersion4.32a
{\unitlength 0.1in%
\begin{picture}(25.0000,15.6496)(3.2480,-17.7165)%
% POLYGON 2 0 3 0 Black White  
% 4 400 400 2820 1800 2820 400 400 400
% 
\special{pn 8}%
\special{pa 394 394}%
\special{pa 2776 1772}%
\special{pa 2776 394}%
\special{pa 394 394}%
\special{pa 2776 1772}%
\special{fp}%
% CIRCLE 2 0 3 0 Black White  
% 4 420 400 820 400 2820 1800 2820 400
% 
\special{pn 8}%
\special{ar 413 394 394 394 6.2831853 0.5280744}%
% STR 2 0 3 0 Black White  
% 4 870 460 870 560 2 0 0 0
% 30\Deg
\put(8.5630,-5.5118){\makebox(0,0)[lb]{30\Deg}}%
% LINE 2 0 3 0 Black White  
% 2 2670 400 2670 550
% 
\special{pn 8}%
\special{pa 2628 394}%
\special{pa 2628 541}%
\special{fp}%
% LINE 2 0 3 0 Black White  
% 2 2820 550 2670 550
% 
\special{pn 8}%
\special{pa 2776 541}%
\special{pa 2628 541}%
\special{fp}%
% STR 2 0 3 0 Black White  
% 4 2870 250 2870 350 2 0 0 0
% B
\put(28.2480,-3.4449){\makebox(0,0)[lb]{B}}%
% STR 2 0 3 0 Black White  
% 4 2870 1780 2870 1880 2 0 0 0
% C
\put(28.2480,-18.5039){\makebox(0,0)[lb]{C}}%
% STR 2 0 3 0 Black White  
% 4 330 240 330 340 2 0 0 0
% A
\put(3.2480,-3.3465){\makebox(0,0)[lb]{A}}%
% LINE 2 0 3 0 Black White  
% 2 1530 1740 1830 1220
% 
\special{pn 8}%
\special{pa 1506 1713}%
\special{pa 1801 1201}%
\special{fp}%
% DOT 0 0 3 0 Black White  
% 1 1830 1220
% 
\special{pn 4}%
\special{sh 1}%
\special{ar 1801 1201 16 16 0 6.2831853}%
% VECTOR 2 0 3 0 Black White  
% 2 1530 1740 1680 1480
% 
\special{pn 8}%
\special{pa 1506 1713}%
\special{pa 1654 1457}%
\special{fp}%
\special{sh 1}%
\special{pa 1654 1457}%
\special{pa 1603 1504}%
\special{pa 1627 1502}%
\special{pa 1638 1524}%
\special{pa 1654 1457}%
\special{fp}%
% STR 2 0 3 0 Black White  
% 4 1710 1210 1710 1310 2 0 0 0
% P
\put(16.8307,-12.8937){\makebox(0,0)[lb]{P}}%
\end{picture}}%
}
    図のようなガラスがある。光が空気からAC面上の点Pへガラス面に垂直に入射した。
    その後,はじめてガラス面上に達した点をQとする。空気とガラスの屈折率はそれぞれ1,$\sqrt{3}$
    とし,点Pは図の位置で考えるものとする。
        \begin{enumerate}
            \item 点Qから空気中へ出ていく光の屈折角を求めよ。
            \item 点Pで入射し,点Qで反射した光が空気中へでるまでの光の進路を図示せよ。
        \end{enumerate}
    \end{mawarikomi}
%  \vfill
%  \newpage
%  \hakosyokika
\item
    \begin{mawarikomi}(10pt,0){220pt}{%WinTpicVersion4.32a
{\unitlength 0.1in%
\begin{picture}(32.9232,14.7638)(4.9213,-22.6378)%
% BOX 2 0 1 0 Black Black  
% 2 1000 800 3800 1000
% 
\special{pn 0}%
\special{sh 0.200}%
\special{pa 984 787}%
\special{pa 3740 787}%
\special{pa 3740 984}%
\special{pa 984 984}%
\special{pa 984 787}%
\special{ip}%
\special{pn 8}%
\special{pa 984 787}%
\special{pa 3740 787}%
\special{pa 3740 984}%
\special{pa 984 984}%
\special{pa 984 787}%
\special{pa 3740 787}%
\special{fp}%
% BOX 2 0 1 0 Black Black  
% 2 1000 2000 3800 2200
% 
\special{pn 0}%
\special{sh 0.200}%
\special{pa 984 1969}%
\special{pa 3740 1969}%
\special{pa 3740 2165}%
\special{pa 984 2165}%
\special{pa 984 1969}%
\special{ip}%
\special{pn 8}%
\special{pa 984 1969}%
\special{pa 3740 1969}%
\special{pa 3740 2165}%
\special{pa 984 2165}%
\special{pa 984 1969}%
\special{pa 3740 1969}%
\special{fp}%
% BOX 2 0 0 0 Black Black  
% 2 1000 1000 3800 2000
% 
\special{pn 0}%
\special{sh 0.400}%
\special{pa 984 984}%
\special{pa 3740 984}%
\special{pa 3740 1969}%
\special{pa 984 1969}%
\special{pa 984 984}%
\special{ip}%
\special{pn 8}%
\special{pa 984 984}%
\special{pa 3740 984}%
\special{pa 3740 1969}%
\special{pa 984 1969}%
\special{pa 984 984}%
\special{pa 3740 984}%
\special{fp}%
% LINE 2 1 3 0 Black White  
% 2 600 1500 1600 1500
% 
\special{pn 8}%
\special{pa 591 1476}%
\special{pa 1575 1476}%
\special{da 0.030}%
% LINE 2 1 3 0 Black White  
% 2 1800 900 1800 1500
% 
\special{pn 8}%
\special{pa 1772 886}%
\special{pa 1772 1476}%
\special{da 0.030}%
% LINE 1 0 3 0 Black White  
% 2 1000 1500 800 1900
% 
\special{pn 13}%
\special{pa 984 1476}%
\special{pa 787 1870}%
\special{fp}%
% VECTOR 1 0 3 0 Black White  
% 2 600 2300 800 1900
% 
\special{pn 13}%
\special{pa 591 2264}%
\special{pa 787 1870}%
\special{fp}%
\special{sh 1}%
\special{pa 787 1870}%
\special{pa 740 1920}%
\special{pa 764 1917}%
\special{pa 776 1938}%
\special{pa 787 1870}%
\special{fp}%
% VECTOR 1 0 3 0 Black White  
% 2 1000 1500 1400 1250
% 
\special{pn 13}%
\special{pa 984 1476}%
\special{pa 1378 1230}%
\special{fp}%
\special{sh 1}%
\special{pa 1378 1230}%
\special{pa 1312 1248}%
\special{pa 1334 1258}%
\special{pa 1333 1281}%
\special{pa 1378 1230}%
\special{fp}%
% LINE 1 0 3 0 Black White  
% 2 1400 1250 1800 1000
% 
\special{pn 13}%
\special{pa 1378 1230}%
\special{pa 1772 984}%
\special{fp}%
% VECTOR 1 0 3 0 Black White  
% 2 1800 1000 2200 1250
% 
\special{pn 13}%
\special{pa 1772 984}%
\special{pa 2165 1230}%
\special{fp}%
\special{sh 1}%
\special{pa 2165 1230}%
\special{pa 2120 1179}%
\special{pa 2121 1203}%
\special{pa 2099 1213}%
\special{pa 2165 1230}%
\special{fp}%
% LINE 1 0 3 0 Black White  
% 2 2200 1250 3400 2000
% 
\special{pn 13}%
\special{pa 2165 1230}%
\special{pa 3346 1969}%
\special{fp}%
% VECTOR 1 0 3 0 Black White  
% 2 3400 2000 3800 1750
% 
\special{pn 13}%
\special{pa 3346 1969}%
\special{pa 3740 1722}%
\special{fp}%
\special{sh 1}%
\special{pa 3740 1722}%
\special{pa 3674 1740}%
\special{pa 3696 1750}%
\special{pa 3695 1774}%
\special{pa 3740 1722}%
\special{fp}%
% STR 2 0 3 0 Black White  
% 4 3880 900 3880 1000 5 0 0 0
% C
\put(38.1890,-9.8425){\makebox(0,0){C}}%
% STR 2 0 3 0 Black White  
% 4 3880 1900 3880 2000 5 0 0 0
% D
\put(38.1890,-19.6850){\makebox(0,0){D}}%
% STR 2 0 3 0 Black White  
% 4 910 1900 910 2000 5 0 0 0
% B
\put(8.9567,-19.6850){\makebox(0,0){B}}%
% STR 2 0 3 0 Black White  
% 4 910 900 910 1000 5 0 0 0
% A
\put(8.9567,-9.8425){\makebox(0,0){A}}%
% CIRCLE 2 0 3 0 Black White  
% 4 1800 1000 2000 1000 1000 1500 1800 1500
% 
\special{pn 8}%
\special{ar 1772 984 197 197 1.5707963 2.5829933}%
% STR 2 0 3 0 Black White  
% 4 1630 1180 1630 1280 5 0 0 0
% $\alpha$
\put(16.0433,-12.5984){\makebox(0,0){$\alpha$}}%
% CIRCLE 2 0 3 0 Black White  
% 4 1000 1500 800 1500 400 1500 600 2300
% 
\special{pn 8}%
\special{ar 984 1476 197 197 2.0344439 3.1415927}%
% STR 2 0 3 0 Black White  
% 4 720 1570 720 1670 5 0 0 0
% $\theta$
\put(7.0866,-16.4370){\makebox(0,0){$\theta$}}%
% STR 2 0 3 0 Black White  
% 4 2120 800 2120 900 5 0 0 0
% $n_2$
\put(20.8661,-8.8583){\makebox(0,0){$n_2$}}%
% STR 2 0 3 0 Black White  
% 4 2120 1600 2120 1700 5 0 0 0
% $n_1$
\put(20.8661,-16.7323){\makebox(0,0){$n_1$}}%
% STR 2 0 3 0 Black White  
% 4 2120 2000 2120 2100 5 0 0 0
% $n_2$
\put(20.8661,-20.6693){\makebox(0,0){$n_2$}}%
\end{picture}}%
}
    図のように,屈折率$n_1$のガラス直方体のの上面と下面に屈折率$n_2$のガラス板を密着させて,
    光線を側面ABから入射させた。このとき,ガラス直方体中で光線が全反射を繰り返しながら,側面CDまで到達するために必要な条件を調べてみよう。
    ガラスは空気中に置かれ,空気の屈折率は1としてよい。
        \begin{enumerate}
            \item 全反射が起こるための,$n_1$と$n_2$の大小関係を答えよ。
            \item AC面での臨界角を$\alpha _0$として,$\sin{\alpha _0}$を求めよ。
            \item AB面への入射角を$\theta $とし,AC面への入射角を$\alpha $とする。$\cos{\alpha}$を$\theta $と$n_1$で表せ。
            \item 図のように全反射をするための$\sin{\theta }$に対する条件を$n_1$,$n_2$を用いて表せ。
            \item $0\Deg < \theta < 90\Deg $のすべての$\theta $に対して全反射を起こさせるための条件を$n_1$,$n_2$だけを用いて表せ。
        \end{enumerate}
    \end{mawarikomi}
%  \vfill
%  \newpage
%  \hakosyokika
\item
    \begin{mawarikomi}(10pt,0){260pt}{%WinTpicVersion4.32a
{\unitlength 0.1in%
\begin{picture}(41.0039,13.9469)(2.6969,-16.9390)%
% LINE 2 0 3 0 Black White  
% 4 1000 600 920 800 1000 600 1080 800
% 
\special{pn 8}%
\special{pa 984 591}%
\special{pa 906 787}%
\special{fp}%
\special{pa 984 591}%
\special{pa 1063 787}%
\special{fp}%
% ELLIPSE 2 0 3 1 Black White  
% 4 1000 800 1080 820 880 800 1120 800
% 
\special{pn 8}%
\special{ar 984 787 79 20 6.2831853 3.1415927}%
% LINE 2 0 3 2 Black White  
% 2 980 860 900 900
% 
\special{pn 8}%
\special{pa 965 846}%
\special{pa 886 886}%
\special{fp}%
% LINE 2 0 3 3 Black White  
% 2 838 962 870 885
% 
\special{pn 8}%
\special{pa 825 947}%
\special{pa 856 871}%
\special{fp}%
% SPLINE 2 0 3 4 Black White  
% 41 870 885 871 885 873 884 875 884 877 884 878 884 881 885 882 885 885 886 887 888 889 889 891 891 893 892 895 894 897 896 899 899 900 901 902 904 904 906 905 909 907 912 908 915 909 917 910 921 911 923 912 926 912 929 913 932 913 935 913 937 913 940 913 943 912 945 912 947 911 949 910 951 909 953 908 954 907 956 906 956 904 957
% 
\special{pn 8}%
\special{pa 856 871}%
\special{pa 883 882}%
\special{pa 898 910}%
\special{pa 894 940}%
\special{pa 890 942}%
\special{fp}%
% LINE 2 0 3 5 Black White  
% 2 841 961 905 955
% 
\special{pn 8}%
\special{pa 828 946}%
\special{pa 891 940}%
\special{fp}%
% SPLINE 2 0 3 6 Black White  
% 41 982 859 983 859 983 859 984 859 984 859 985 859 985 860 986 860 987 861 987 861 988 862 989 863 989 864 990 865 991 866 991 867 992 868 992 870 993 871 993 872 994 873 994 875 994 876 995 877 995 879 995 880 995 881 995 882 995 883 995 885 995 886 995 887 995 888 995 889 995 889 994 890 994 891 994 891 993 891 993 892 992 892
% 
\special{pn 8}%
\special{pa 967 845}%
\special{pa 979 869}%
\special{pa 976 878}%
\special{fp}%
% LINE 2 0 3 7 Black White  
% 2 992 892 910 932
% 
\special{pn 8}%
\special{pa 976 878}%
\special{pa 896 917}%
\special{fp}%
% LINE 2 0 3 8 Black White  
% 2 962 818 962 868
% 
\special{pn 8}%
\special{pa 947 805}%
\special{pa 947 854}%
\special{fp}%
% LINE 2 0 3 9 Black White  
% 4 962 908 962 1188 1042 1188 1042 818
% 
\special{pn 8}%
\special{pa 947 894}%
\special{pa 947 1169}%
\special{fp}%
\special{pa 1026 1169}%
\special{pa 1026 805}%
\special{fp}%
% ELLIPSE 2 0 3 10 Black White  
% 4 1002 1184 1042 1208 922 1184 1082 1184
% 
\special{pn 8}%
\special{ar 986 1165 39 24 6.2831853 3.1415927}%
% LINE 2 0 3 0 Black White  
% 4 400 1010 960 1010 1040 1010 4440 1010
% 
\special{pn 8}%
\special{pa 394 994}%
\special{pa 945 994}%
\special{fp}%
\special{pa 1024 994}%
\special{pa 4370 994}%
\special{fp}%
% VECTOR 2 0 3 0 Black White  
% 2 3490 960 4100 1060
% 
\special{pn 8}%
\special{pa 3435 945}%
\special{pa 4035 1043}%
\special{fp}%
\special{sh 1}%
\special{pa 4035 1043}%
\special{pa 3973 1013}%
\special{pa 3983 1034}%
\special{pa 3968 1052}%
\special{pa 4035 1043}%
\special{fp}%
% STR 2 0 3 0 Black White  
% 4 4150 1020 4150 1120 5 0 0 0
% $x$
\put(40.8465,-11.0236){\makebox(0,0){$x$}}%
% VECTOR 2 0 3 0 Black White  
% 2 3780 1520 3780 320
% 
\special{pn 8}%
\special{pa 3720 1496}%
\special{pa 3720 315}%
\special{fp}%
\special{sh 1}%
\special{pa 3720 315}%
\special{pa 3701 381}%
\special{pa 3720 367}%
\special{pa 3740 381}%
\special{pa 3720 315}%
\special{fp}%
% STR 2 0 3 0 Black White  
% 4 3880 280 3880 380 5 0 0 0
% $y$
\put(38.1890,-3.7402){\makebox(0,0){$y$}}%
% LINE 2 0 3 0 Black White  
% 2 3650 600 3650 1400
% 
\special{pn 8}%
\special{pa 3593 591}%
\special{pa 3593 1378}%
\special{fp}%
% LINE 2 0 3 0 Black White  
% 6 3650 1400 3940 1600 3940 1600 3940 490 3940 490 3650 600
% 
\special{pn 8}%
\special{pa 3593 1378}%
\special{pa 3878 1575}%
\special{fp}%
\special{pa 3878 1575}%
\special{pa 3878 482}%
\special{fp}%
\special{pa 3878 482}%
\special{pa 3593 591}%
\special{fp}%
% ELLIPSE 2 0 3 0 Black White  
% 4 2740 1007 3340 2207 1940 207 1940 1807
% 
\special{pn 8}%
\special{ar 2697 991 591 1181 2.6779450 3.6052403}%
% ELLIPSE 2 0 3 0 Black White  
% 4 1666 1007 2266 2207 2466 1807 2466 207
% 
\special{pn 8}%
\special{ar 1640 991 591 1181 5.8195377 0.4636476}%
% STR 2 0 3 0 Black White  
% 4 2360 423 2360 473 2 0 0 0
% 凸レンズ
\put(23.2283,-4.6555){\makebox(0,0)[lb]{凸レンズ}}%
% DOT 0 0 3 0 Black White  
% 1 3000 1008
% 
\special{pn 4}%
\special{sh 1}%
\special{ar 2953 992 16 16 0 6.2831853}%
% DOT 0 0 3 0 Black White  
% 1 1400 1008
% 
\special{pn 4}%
\special{sh 1}%
\special{ar 1378 992 16 16 0 6.2831853}%
% STR 2 0 3 0 Black White  
% 4 3000 1074 3000 1094 5 0 0 0
% F'
\put(29.5276,-10.7677){\makebox(0,0){F'}}%
% STR 2 0 3 0 Black White  
% 4 1400 1074 1400 1094 5 0 0 0
% F
\put(13.7795,-10.7677){\makebox(0,0){F}}%
% STR 2 0 3 0 Black White  
% 4 400 1074 400 1094 5 0 0 0
% 光軸
\put(3.9370,-10.7677){\makebox(0,0){光軸}}%
% STR 2 0 3 0 Black White  
% 4 846 1200 846 1220 1 0 0 0
% 2本の矢印形の
\put(8.3268,-12.0079){\makebox(0,0)[lt]{2本の矢印形の}}%
% STR 2 0 3 0 Black White  
% 4 846 1316 846 1336 1 0 0 0
% 光源
\put(8.3268,-13.1496){\makebox(0,0)[lt]{光源}}%
% STR 2 0 3 0 Black White  
% 4 3560 1706 3560 1756 2 0 0 0
% スクリーン
\put(35.0394,-17.2835){\makebox(0,0)[lb]{スクリーン}}%
% LINE 2 0 3 0 Black White  
% 2 2337 1721 2439 1567
% 
\special{pn 8}%
\special{pa 2300 1694}%
\special{pa 2401 1542}%
\special{fp}%
% SPLINE 2 0 3 1 Black White  
% 41 2413 1605 2415 1607 2418 1610 2420 1612 2422 1613 2424 1615 2425 1618 2428 1620 2429 1622 2432 1624 2433 1627 2435 1629 2437 1631 2439 1634 2441 1636 2442 1639 2444 1641 2445 1643 2447 1645 2448 1648 2450 1650 2452 1653 2453 1656 2454 1658 2456 1661 2457 1664 2458 1666 2460 1669 2461 1672 2462 1675 2463 1678 2464 1680 2466 1683 2466 1686 2467 1689 2468 1692 2469 1694 2469 1697 2470 1699 2471 1703 2471 1706
% 
\special{pn 8}%
\special{pa 2375 1580}%
\special{pa 2396 1602}%
\special{pa 2413 1627}%
\special{pa 2427 1656}%
\special{pa 2432 1679}%
\special{fp}%
% LINE 2 0 3 2 Black White  
% 2 2337 1721 2484 1705
% 
\special{pn 8}%
\special{pa 2300 1694}%
\special{pa 2445 1678}%
\special{fp}%
% SPLINE 2 0 3 3 Black White  
% 41 2454 1709 2452 1708 2450 1705 2449 1703 2447 1701 2445 1700 2444 1697 2442 1696 2440 1695 2439 1692 2437 1691 2436 1688 2434 1687 2432 1684 2430 1682 2429 1680 2428 1677 2426 1676 2425 1673 2424 1670 2423 1669 2421 1666 2419 1665 2418 1662 2417 1659 2416 1657 2415 1655 2413 1652 2412 1650 2411 1648 2410 1646 2409 1643 2407 1640 2407 1638 2406 1636 2405 1633 2404 1631 2403 1628 2402 1626 2401 1623 2401 1621
% 
\special{pn 8}%
\special{pa 2415 1682}%
\special{pa 2396 1660}%
\special{pa 2380 1634}%
\special{pa 2366 1606}%
\special{pa 2363 1595}%
\special{fp}%
% SPLINE 2 0 0 4 Black Black  
% 41 2448 1649 2450 1654 2452 1658 2454 1663 2456 1668 2457 1670 2459 1674 2459 1676 2459 1678 2459 1681 2458 1681 2457 1682 2456 1681 2454 1680 2452 1678 2449 1675 2447 1672 2444 1669 2441 1665 2439 1661 2436 1657 2433 1652 2431 1648 2429 1643 2427 1640 2426 1636 2425 1633 2424 1630 2424 1627 2424 1625 2425 1625 2426 1624 2428 1626 2430 1627 2432 1628 2434 1631 2436 1634 2440 1638 2442 1641 2445 1645 2448 1649
% 
\special{sh 0.400}%
\special{pn 0}%
\special{pa 2409 1623}%
\special{pa 2420 1653}%
\special{pa 2401 1636}%
\special{pa 2387 1607}%
\special{pa 2403 1613}%
\special{pa 2409 1623}%
\special{fp}%
\special{pn 8}%
\special{pa 2409 1623}%
\special{pa 2420 1653}%
\special{pa 2401 1636}%
\special{pa 2387 1607}%
\special{pa 2403 1613}%
\special{pa 2409 1623}%
\special{fp}%
% CIRCLE 2 0 2 5 Black White  
% 4 2422 1636 2429 1634 2429 1634 2429 1634
% 
\special{sh 0}%
\special{ia 2384 1610 7 7 0.0000000 6.2831853}%
\special{pn 8}%
\special{ar 2384 1610 7 7 0.0000000 6.2831853}%
% STR 2 0 3 0 Black White  
% 4 2506 1786 2506 1806 2 0 0 0
% 観測者
\put(24.6654,-17.7756){\makebox(0,0)[lb]{観測者}}%
\end{picture}}%
}
    矢印を組み合わせた形の光源を凸レンズの光軸上に配置したところ,
    スクリーン上に実像ができた。スクリーンは光軸に対して垂直であり,
    F,F'はレンズの焦点である。スクリーンと光軸の交点を原点にして,
    水平方向に$x$軸をとり,レンズ側から見て右向きを正とし,鉛直方向に
    $y$軸をとり上向きを正とする。光源の太い矢印は$y$軸の正の向き,
    細い矢印は$x$軸正の向きを向いている。このとき,観測者がレンズ側から見ると,
    スクリーン上の像は次の\Hako である。
    \end{mawarikomi}
    \begin{center}
        %WinTpicVersion4.32a
{\unitlength 0.1in%
\begin{picture}(49.0945,11.8406)(2.4606,-15.1870)%
% BOX 2 0 1 0 Black Black  
% 2 700 732 1066 1092
% 
\special{pn 0}%
\special{sh 0.200}%
\special{pa 689 720}%
\special{pa 1049 720}%
\special{pa 1049 1075}%
\special{pa 689 1075}%
\special{pa 689 720}%
\special{ip}%
\special{pn 8}%
\special{pa 689 720}%
\special{pa 1049 720}%
\special{pa 1049 1075}%
\special{pa 689 1075}%
\special{pa 689 720}%
\special{pa 1049 720}%
\special{fp}%
% BOX 2 0 1 0 Black Black  
% 2 700 1092 1066 1452
% 
\special{pn 0}%
\special{sh 0.200}%
\special{pa 689 1075}%
\special{pa 1049 1075}%
\special{pa 1049 1429}%
\special{pa 689 1429}%
\special{pa 689 1075}%
\special{ip}%
\special{pn 8}%
\special{pa 689 1075}%
\special{pa 1049 1075}%
\special{pa 1049 1429}%
\special{pa 689 1429}%
\special{pa 689 1075}%
\special{pa 1049 1075}%
\special{fp}%
% BOX 2 0 1 0 Black Black  
% 2 334 1092 700 1452
% 
\special{pn 0}%
\special{sh 0.200}%
\special{pa 329 1075}%
\special{pa 689 1075}%
\special{pa 689 1429}%
\special{pa 329 1429}%
\special{pa 329 1075}%
\special{ip}%
\special{pn 8}%
\special{pa 329 1075}%
\special{pa 689 1075}%
\special{pa 689 1429}%
\special{pa 329 1429}%
\special{pa 329 1075}%
\special{pa 689 1075}%
\special{fp}%
% BOX 2 0 1 0 Black Black  
% 2 334 733 700 1093
% 
\special{pn 0}%
\special{sh 0.200}%
\special{pa 329 721}%
\special{pa 689 721}%
\special{pa 689 1076}%
\special{pa 329 1076}%
\special{pa 329 721}%
\special{ip}%
\special{pn 8}%
\special{pa 329 721}%
\special{pa 689 721}%
\special{pa 689 1076}%
\special{pa 329 1076}%
\special{pa 329 721}%
\special{pa 689 721}%
\special{fp}%
% STR 2 0 3 0 Black White  
% 4 1231 1074 1231 1092 5 0 0 0
% $x$
\put(12.1161,-10.7480){\makebox(0,0){$x$}}%
% STR 2 0 3 0 Black White  
% 4 700 588 700 606 5 0 0 0
% $y$
\put(6.8898,-5.9646){\makebox(0,0){$y$}}%
% VECTOR 2 0 3 0 Black Black  
% 2 288 1093 1188 1093
% 
\special{pn 8}%
\special{pa 283 1076}%
\special{pa 1169 1076}%
\special{fp}%
\special{sh 1}%
\special{pa 1169 1076}%
\special{pa 1103 1056}%
\special{pa 1117 1076}%
\special{pa 1103 1095}%
\special{pa 1169 1076}%
\special{fp}%
% VECTOR 2 0 3 0 Black Black  
% 2 700 1543 700 643
% 
\special{pn 8}%
\special{pa 689 1519}%
\special{pa 689 633}%
\special{fp}%
\special{sh 1}%
\special{pa 689 633}%
\special{pa 669 699}%
\special{pa 689 685}%
\special{pa 709 699}%
\special{pa 689 633}%
\special{fp}%
% POLYGON 2 0 2 0 Black White  
% 16 700 792 588 942 644 942 644 1017 494 1017 494 998 456 1036 494 1074 494 1055 644 1055 644 1430 756 1430 756 942 794 942 794 942 700 792
% 
\special{pn 0}%
\special{sh 0}%
\special{pa 689 780}%
\special{pa 579 927}%
\special{pa 634 927}%
\special{pa 634 1001}%
\special{pa 486 1001}%
\special{pa 486 982}%
\special{pa 449 1020}%
\special{pa 486 1057}%
\special{pa 486 1038}%
\special{pa 634 1038}%
\special{pa 634 1407}%
\special{pa 744 1407}%
\special{pa 744 927}%
\special{pa 781 927}%
\special{pa 781 927}%
\special{pa 689 780}%
\special{ip}%
\special{pn 8}%
\special{pa 689 780}%
\special{pa 579 927}%
\special{pa 634 927}%
\special{pa 634 1001}%
\special{pa 486 1001}%
\special{pa 486 982}%
\special{pa 449 1020}%
\special{pa 486 1057}%
\special{pa 486 1038}%
\special{pa 634 1038}%
\special{pa 634 1407}%
\special{pa 744 1407}%
\special{pa 744 927}%
\special{pa 781 927}%
\special{pa 689 780}%
\special{pa 579 927}%
\special{fp}%
% BOX 2 0 1 0 Black Black  
% 2 2050 732 2416 1092
% 
\special{pn 0}%
\special{sh 0.200}%
\special{pa 2018 720}%
\special{pa 2378 720}%
\special{pa 2378 1075}%
\special{pa 2018 1075}%
\special{pa 2018 720}%
\special{ip}%
\special{pn 8}%
\special{pa 2018 720}%
\special{pa 2378 720}%
\special{pa 2378 1075}%
\special{pa 2018 1075}%
\special{pa 2018 720}%
\special{pa 2378 720}%
\special{fp}%
% BOX 2 0 1 0 Black Black  
% 2 2050 1092 2416 1452
% 
\special{pn 0}%
\special{sh 0.200}%
\special{pa 2018 1075}%
\special{pa 2378 1075}%
\special{pa 2378 1429}%
\special{pa 2018 1429}%
\special{pa 2018 1075}%
\special{ip}%
\special{pn 8}%
\special{pa 2018 1075}%
\special{pa 2378 1075}%
\special{pa 2378 1429}%
\special{pa 2018 1429}%
\special{pa 2018 1075}%
\special{pa 2378 1075}%
\special{fp}%
% BOX 2 0 1 0 Black Black  
% 2 1684 1092 2050 1452
% 
\special{pn 0}%
\special{sh 0.200}%
\special{pa 1657 1075}%
\special{pa 2018 1075}%
\special{pa 2018 1429}%
\special{pa 1657 1429}%
\special{pa 1657 1075}%
\special{ip}%
\special{pn 8}%
\special{pa 1657 1075}%
\special{pa 2018 1075}%
\special{pa 2018 1429}%
\special{pa 1657 1429}%
\special{pa 1657 1075}%
\special{pa 2018 1075}%
\special{fp}%
% BOX 2 0 1 0 Black Black  
% 2 1684 733 2050 1093
% 
\special{pn 0}%
\special{sh 0.200}%
\special{pa 1657 721}%
\special{pa 2018 721}%
\special{pa 2018 1076}%
\special{pa 1657 1076}%
\special{pa 1657 721}%
\special{ip}%
\special{pn 8}%
\special{pa 1657 721}%
\special{pa 2018 721}%
\special{pa 2018 1076}%
\special{pa 1657 1076}%
\special{pa 1657 721}%
\special{pa 2018 721}%
\special{fp}%
% STR 2 0 3 0 Black White  
% 4 2581 1074 2581 1092 5 0 0 0
% $x$
\put(25.4035,-10.7480){\makebox(0,0){$x$}}%
% STR 2 0 3 0 Black White  
% 4 2050 588 2050 606 5 0 0 0
% $y$
\put(20.1772,-5.9646){\makebox(0,0){$y$}}%
% VECTOR 2 0 3 0 Black Black  
% 2 1638 1093 2538 1093
% 
\special{pn 8}%
\special{pa 1612 1076}%
\special{pa 2498 1076}%
\special{fp}%
\special{sh 1}%
\special{pa 2498 1076}%
\special{pa 2432 1056}%
\special{pa 2446 1076}%
\special{pa 2432 1095}%
\special{pa 2498 1076}%
\special{fp}%
% VECTOR 2 0 3 0 Black Black  
% 2 2050 1543 2050 643
% 
\special{pn 8}%
\special{pa 2018 1519}%
\special{pa 2018 633}%
\special{fp}%
\special{sh 1}%
\special{pa 2018 633}%
\special{pa 1998 699}%
\special{pa 2018 685}%
\special{pa 2037 699}%
\special{pa 2018 633}%
\special{fp}%
% POLYGON 2 0 2 0 Black White  
% 16 2050 792 2162 942 2106 942 2106 1017 2256 1017 2256 998 2294 1036 2256 1074 2256 1055 2106 1055 2106 1430 1994 1430 1994 942 1956 942 1956 942 2050 792
% 
\special{pn 0}%
\special{sh 0}%
\special{pa 2018 780}%
\special{pa 2128 927}%
\special{pa 2073 927}%
\special{pa 2073 1001}%
\special{pa 2220 1001}%
\special{pa 2220 982}%
\special{pa 2258 1020}%
\special{pa 2220 1057}%
\special{pa 2220 1038}%
\special{pa 2073 1038}%
\special{pa 2073 1407}%
\special{pa 1963 1407}%
\special{pa 1963 927}%
\special{pa 1925 927}%
\special{pa 1925 927}%
\special{pa 2018 780}%
\special{ip}%
\special{pn 8}%
\special{pa 2018 780}%
\special{pa 2128 927}%
\special{pa 2073 927}%
\special{pa 2073 1001}%
\special{pa 2220 1001}%
\special{pa 2220 982}%
\special{pa 2258 1020}%
\special{pa 2220 1057}%
\special{pa 2220 1038}%
\special{pa 2073 1038}%
\special{pa 2073 1407}%
\special{pa 1963 1407}%
\special{pa 1963 927}%
\special{pa 1925 927}%
\special{pa 2018 780}%
\special{pa 2128 927}%
\special{fp}%
% BOX 2 0 1 0 Black Black  
% 2 3400 732 3766 1092
% 
\special{pn 0}%
\special{sh 0.200}%
\special{pa 3346 720}%
\special{pa 3707 720}%
\special{pa 3707 1075}%
\special{pa 3346 1075}%
\special{pa 3346 720}%
\special{ip}%
\special{pn 8}%
\special{pa 3346 720}%
\special{pa 3707 720}%
\special{pa 3707 1075}%
\special{pa 3346 1075}%
\special{pa 3346 720}%
\special{pa 3707 720}%
\special{fp}%
% BOX 2 0 1 0 Black Black  
% 2 3400 1092 3766 1452
% 
\special{pn 0}%
\special{sh 0.200}%
\special{pa 3346 1075}%
\special{pa 3707 1075}%
\special{pa 3707 1429}%
\special{pa 3346 1429}%
\special{pa 3346 1075}%
\special{ip}%
\special{pn 8}%
\special{pa 3346 1075}%
\special{pa 3707 1075}%
\special{pa 3707 1429}%
\special{pa 3346 1429}%
\special{pa 3346 1075}%
\special{pa 3707 1075}%
\special{fp}%
% BOX 2 0 1 0 Black Black  
% 2 3034 1092 3400 1452
% 
\special{pn 0}%
\special{sh 0.200}%
\special{pa 2986 1075}%
\special{pa 3346 1075}%
\special{pa 3346 1429}%
\special{pa 2986 1429}%
\special{pa 2986 1075}%
\special{ip}%
\special{pn 8}%
\special{pa 2986 1075}%
\special{pa 3346 1075}%
\special{pa 3346 1429}%
\special{pa 2986 1429}%
\special{pa 2986 1075}%
\special{pa 3346 1075}%
\special{fp}%
% BOX 2 0 1 0 Black Black  
% 2 3034 733 3400 1093
% 
\special{pn 0}%
\special{sh 0.200}%
\special{pa 2986 721}%
\special{pa 3346 721}%
\special{pa 3346 1076}%
\special{pa 2986 1076}%
\special{pa 2986 721}%
\special{ip}%
\special{pn 8}%
\special{pa 2986 721}%
\special{pa 3346 721}%
\special{pa 3346 1076}%
\special{pa 2986 1076}%
\special{pa 2986 721}%
\special{pa 3346 721}%
\special{fp}%
% STR 2 0 3 0 Black White  
% 4 3931 1074 3931 1092 5 0 0 0
% $x$
\put(38.6909,-10.7480){\makebox(0,0){$x$}}%
% STR 2 0 3 0 Black White  
% 4 3400 588 3400 606 5 0 0 0
% $y$
\put(33.4646,-5.9646){\makebox(0,0){$y$}}%
% VECTOR 2 0 3 0 Black Black  
% 2 2988 1093 3888 1093
% 
\special{pn 8}%
\special{pa 2941 1076}%
\special{pa 3827 1076}%
\special{fp}%
\special{sh 1}%
\special{pa 3827 1076}%
\special{pa 3761 1056}%
\special{pa 3775 1076}%
\special{pa 3761 1095}%
\special{pa 3827 1076}%
\special{fp}%
% VECTOR 2 0 3 0 Black Black  
% 2 3400 1543 3400 643
% 
\special{pn 8}%
\special{pa 3346 1519}%
\special{pa 3346 633}%
\special{fp}%
\special{sh 1}%
\special{pa 3346 633}%
\special{pa 3327 699}%
\special{pa 3346 685}%
\special{pa 3366 699}%
\special{pa 3346 633}%
\special{fp}%
% POLYGON 2 0 2 0 Black White  
% 16 3400 1430 3288 1280 3344 1280 3344 1205 3194 1205 3194 1224 3156 1186 3194 1148 3194 1167 3344 1167 3344 792 3456 792 3456 1280 3494 1280 3494 1280 3400 1430
% 
\special{pn 0}%
\special{sh 0}%
\special{pa 3346 1407}%
\special{pa 3236 1260}%
\special{pa 3291 1260}%
\special{pa 3291 1186}%
\special{pa 3144 1186}%
\special{pa 3144 1205}%
\special{pa 3106 1167}%
\special{pa 3144 1130}%
\special{pa 3144 1149}%
\special{pa 3291 1149}%
\special{pa 3291 780}%
\special{pa 3402 780}%
\special{pa 3402 1260}%
\special{pa 3439 1260}%
\special{pa 3439 1260}%
\special{pa 3346 1407}%
\special{ip}%
\special{pn 8}%
\special{pa 3346 1407}%
\special{pa 3236 1260}%
\special{pa 3291 1260}%
\special{pa 3291 1186}%
\special{pa 3144 1186}%
\special{pa 3144 1205}%
\special{pa 3106 1167}%
\special{pa 3144 1130}%
\special{pa 3144 1149}%
\special{pa 3291 1149}%
\special{pa 3291 780}%
\special{pa 3402 780}%
\special{pa 3402 1260}%
\special{pa 3439 1260}%
\special{pa 3346 1407}%
\special{pa 3236 1260}%
\special{fp}%
% BOX 2 0 1 0 Black Black  
% 2 4750 732 5116 1092
% 
\special{pn 0}%
\special{sh 0.200}%
\special{pa 4675 720}%
\special{pa 5035 720}%
\special{pa 5035 1075}%
\special{pa 4675 1075}%
\special{pa 4675 720}%
\special{ip}%
\special{pn 8}%
\special{pa 4675 720}%
\special{pa 5035 720}%
\special{pa 5035 1075}%
\special{pa 4675 1075}%
\special{pa 4675 720}%
\special{pa 5035 720}%
\special{fp}%
% BOX 2 0 1 0 Black Black  
% 2 4750 1092 5116 1452
% 
\special{pn 0}%
\special{sh 0.200}%
\special{pa 4675 1075}%
\special{pa 5035 1075}%
\special{pa 5035 1429}%
\special{pa 4675 1429}%
\special{pa 4675 1075}%
\special{ip}%
\special{pn 8}%
\special{pa 4675 1075}%
\special{pa 5035 1075}%
\special{pa 5035 1429}%
\special{pa 4675 1429}%
\special{pa 4675 1075}%
\special{pa 5035 1075}%
\special{fp}%
% BOX 2 0 1 0 Black Black  
% 2 4384 1092 4750 1452
% 
\special{pn 0}%
\special{sh 0.200}%
\special{pa 4315 1075}%
\special{pa 4675 1075}%
\special{pa 4675 1429}%
\special{pa 4315 1429}%
\special{pa 4315 1075}%
\special{ip}%
\special{pn 8}%
\special{pa 4315 1075}%
\special{pa 4675 1075}%
\special{pa 4675 1429}%
\special{pa 4315 1429}%
\special{pa 4315 1075}%
\special{pa 4675 1075}%
\special{fp}%
% BOX 2 0 1 0 Black Black  
% 2 4384 733 4750 1093
% 
\special{pn 0}%
\special{sh 0.200}%
\special{pa 4315 721}%
\special{pa 4675 721}%
\special{pa 4675 1076}%
\special{pa 4315 1076}%
\special{pa 4315 721}%
\special{ip}%
\special{pn 8}%
\special{pa 4315 721}%
\special{pa 4675 721}%
\special{pa 4675 1076}%
\special{pa 4315 1076}%
\special{pa 4315 721}%
\special{pa 4675 721}%
\special{fp}%
% STR 2 0 3 0 Black White  
% 4 5281 1074 5281 1092 5 0 0 0
% $x$
\put(51.9783,-10.7480){\makebox(0,0){$x$}}%
% STR 2 0 3 0 Black White  
% 4 4750 588 4750 606 5 0 0 0
% $y$
\put(46.7520,-5.9646){\makebox(0,0){$y$}}%
% VECTOR 2 0 3 0 Black Black  
% 2 4338 1093 5238 1093
% 
\special{pn 8}%
\special{pa 4270 1076}%
\special{pa 5156 1076}%
\special{fp}%
\special{sh 1}%
\special{pa 5156 1076}%
\special{pa 5090 1056}%
\special{pa 5103 1076}%
\special{pa 5090 1095}%
\special{pa 5156 1076}%
\special{fp}%
% VECTOR 2 0 3 0 Black Black  
% 2 4750 1543 4750 643
% 
\special{pn 8}%
\special{pa 4675 1519}%
\special{pa 4675 633}%
\special{fp}%
\special{sh 1}%
\special{pa 4675 633}%
\special{pa 4656 699}%
\special{pa 4675 685}%
\special{pa 4695 699}%
\special{pa 4675 633}%
\special{fp}%
% POLYGON 2 0 2 0 Black White  
% 16 4750 1430 4862 1280 4806 1280 4806 1205 4956 1205 4956 1224 4994 1186 4956 1148 4956 1167 4806 1167 4806 792 4694 792 4694 1280 4656 1280 4656 1280 4750 1430
% 
\special{pn 0}%
\special{sh 0}%
\special{pa 4675 1407}%
\special{pa 4785 1260}%
\special{pa 4730 1260}%
\special{pa 4730 1186}%
\special{pa 4878 1186}%
\special{pa 4878 1205}%
\special{pa 4915 1167}%
\special{pa 4878 1130}%
\special{pa 4878 1149}%
\special{pa 4730 1149}%
\special{pa 4730 780}%
\special{pa 4620 780}%
\special{pa 4620 1260}%
\special{pa 4583 1260}%
\special{pa 4583 1260}%
\special{pa 4675 1407}%
\special{ip}%
\special{pn 8}%
\special{pa 4675 1407}%
\special{pa 4785 1260}%
\special{pa 4730 1260}%
\special{pa 4730 1186}%
\special{pa 4878 1186}%
\special{pa 4878 1205}%
\special{pa 4915 1167}%
\special{pa 4878 1130}%
\special{pa 4878 1149}%
\special{pa 4730 1149}%
\special{pa 4730 780}%
\special{pa 4620 780}%
\special{pa 4620 1260}%
\special{pa 4583 1260}%
\special{pa 4675 1407}%
\special{pa 4785 1260}%
\special{fp}%
% STR 2 0 3 0 Black White  
% 4 250 432 250 470 2 0 0 0
% \maru{1}
\put(2.4606,-4.6260){\makebox(0,0)[lb]{\maru{1}}}%
% STR 2 0 3 0 Black White  
% 4 1590 440 1590 478 2 0 0 0
% \maru{2}
\put(15.6496,-4.7047){\makebox(0,0)[lb]{\maru{2}}}%
% STR 2 0 3 0 Black White  
% 4 2940 432 2940 470 2 0 0 0
% \maru{3}
\put(28.9370,-4.6260){\makebox(0,0)[lb]{\maru{3}}}%
% STR 2 0 3 0 Black White  
% 4 4290 432 4290 470 2 0 0 0
% \maru{4}
\put(42.2244,-4.6260){\makebox(0,0)[lb]{\maru{4}}}%
\end{picture}}%

    \end{center}
     光源とスクリーンの距離が100\sftanni{cm}で,実像の倍率が1だったので,
    レンズの焦点距離は\Hako \sftanni{cm}である。\\
     次にレンズの中心より上半分に黒い紙を貼った。スクリーン上の像はどのようになるか。
    次のうちから選べ。\Hako 
        \begin{edaenumerate}[m]
            \item $y>0$の部分が見えなくなった。
            \item 全体が暗くなった。
            \item $y<0$の部分が見えなくなった。
            \item 何も見えなくなった。
        \end{edaenumerate}

%  \vfill
% %  \newpage
%  \hakosyokika
\item 高さ$8$\sftanni{cm}のろうそくと,焦点距離$10$\sftanni{cm}の凸レンズ$\mathrm{L_1}$,焦点距離$10$\sftanni{cm}の凹レンズ$\mathrm{L_2}$がある。レンズの厚さは考えなくてよい。
        \begin{enumerate}
            \item 凸レンズ$\mathrm{L_1}$の前方$30$\sftanni{cm}の位置に光軸に垂直にろうそくを立てる。$\mathrm{L_1}$によって作られるろうそくの像の位置および像の大きさを求めよ。さらに,像が実像か虚像か,また正立か倒立かを答えよ。
            \item $\mathrm{L_1}$により倍率1の実像ができるとき,ろうそくから$\mathrm{L_1}$までの距離はいくらか。
            \item 凹レンズ$\mathrm{L_2}$の前方$30$\sftanni{cm}の位置にろうそくを立てる。像の位置および像の大きさを答えよ。さらに実像か,虚像か,また正立か倒立かを答えよ。
            \item $\mathrm{L_1}$と$\mathrm{L_2}$を$30$\sftanni{cm}離し,光軸を合わせる。$\mathrm{L_1}$の前方($\mathrm{L_2}$とは反対方向)$5$\sftanni{cm}の位置にろうそくを立てる。まず,$\mathrm{L_1}$だけによる像の位置を求めよ。次に$\mathrm{L_1}$,$\mathrm{L_2}$全体による像の位置と大きさを求めよ。
        \end{enumerate}
%  \hakosyokika
\item 異なる位置にある2つの音源SおよびTを結ぶ直線状に観測者がいる。音源SおよびTから出る音の振動数200\sftanni{Hz},音速340\sftanni{m/s}であり,音源SおよびTは疎蜜で同位相の音を左右に送り出し,音は減衰しないものとする。
    \begin{enumerate}
        \item 音源SおよびTの右側にいた観測者が,音が強めあっていると観測できたとすれば,2つの音源の距離は最短で\Hako \tanni{m}である。
        \item 音源Sと音源Tとの間隔を$5.6$\sftanni{m}とし,SとTの間にいた観測者が音が強めあっていると観測できたとする。そのような位置は間隔が\Hako \tanni{m}をなしていくつかあるが,観測者がTにもっとも近い位置にいたとすれば,Tまでの距離は\Hako \tanni{m}である。そして,観測者がSに向かって$1.7$\tanni{m/s}の速さで歩くと,音の大きさが繰り返し変化して聞こえる。音が強めあっていると観測する回数は1秒当たり\Hako \tanni{回/s}である。
    \end{enumerate}
%  \vfill
% %  \newpage
%  \hakosyokika
\item
    \begin{mawarikomi}(10pt,0){160pt}{%WinTpicVersion4.32a
{\unitlength 0.1in%
\begin{picture}(17.7165,19.1043)(3.9370,-30.2362)%
% CIRCLE 2 0 3 0 Black White  
% 4 700 2135 1000 2135 1000 2135 1000 2135
% 
\special{pn 8}%
\special{ar 689 2101 295 295 0.0000000 6.2831853}%
% DOT 0 0 3 0 Black White  
% 1 700 2135
% 
\special{pn 4}%
\special{sh 1}%
\special{ar 689 2101 16 16 0 6.2831853}%
% CIRCLE 2 0 3 0 Black White  
% 4 700 2135 1300 2135 400 3035 400 1535
% 
\special{pn 8}%
\special{ar 689 2101 591 591 4.2487414 1.8925469}%
% CIRCLE 2 0 3 0 Black White  
% 4 700 2135 1600 2135 400 3335 400 1235
% 
\special{pn 8}%
\special{ar 689 2101 886 886 4.3906384 1.8157750}%
% CIRCLE 2 0 3 0 Black White  
% 4 700 2135 1900 2135 1900 3635 1300 1235
% 
\special{pn 8}%
\special{ar 689 2101 1181 1181 5.3003916 0.8960554}%
% DOT 0 0 3 0 Black White  
% 1 1900 2135
% 
\special{pn 4}%
\special{sh 1}%
\special{ar 1870 2101 16 16 0 6.2831853}%
% CIRCLE 2 0 3 0 Black White  
% 4 1900 2129 1600 2129 1600 2129 1600 2129
% 
\special{pn 8}%
\special{ar 1870 2095 295 295 0.0000000 6.2831853}%
% CIRCLE 2 0 3 0 Black White  
% 4 1900 2129 1300 2129 2200 1529 2200 3029
% 
\special{pn 8}%
\special{ar 1870 2095 591 591 1.2490458 5.1760366}%
% CIRCLE 2 0 3 0 Black White  
% 4 1900 2129 1000 2129 2200 1229 2200 3329
% 
\special{pn 8}%
\special{ar 1870 2095 886 886 1.3258177 5.0341395}%
% CIRCLE 2 0 3 0 Black White  
% 4 1900 2129 700 2129 1300 1229 700 3629
% 
\special{pn 8}%
\special{ar 1870 2095 1181 1181 2.2455373 4.1243864}%
% DOT 0 1 3 0 Black White  
% 1 1201 2693
% 
\special{pn 4}%
\special{sh 1}%
\special{ar 1182 2651 16 16 0 6.2831853}%
% DOT 0 2 3 0 Black White  
% 1 1075 1490
% 
\special{pn 4}%
\special{sh 1}%
\special{ar 1058 1467 16 16 0 6.2831853}%
% STR 2 0 3 0 Black White  
% 4 1075 1538 1075 1568 5 0 0 0
% P$_3$
\put(10.5807,-15.4331){\makebox(0,0){P$_3$}}%
% STR 2 0 3 0 Black White  
% 4 580 2105 580 2135 5 0 0 0
% $A$
\put(5.7087,-21.0138){\makebox(0,0){$A$}}%
% STR 2 0 3 0 Black White  
% 4 2020 2105 2020 2135 5 0 0 0
% $B$
\put(19.8819,-21.0138){\makebox(0,0){$B$}}%
% STR 2 0 3 0 Black White  
% 4 1255 2618 1255 2648 5 0 0 0
% P$_2$
\put(12.3524,-26.0630){\makebox(0,0){P$_2$}}%
% DOT 0 0 3 0 Black White  
% 1 1747 1550
% 
\special{pn 4}%
\special{sh 1}%
\special{ar 1719 1526 16 16 0 6.2831853}%
% STR 2 0 3 0 Black White  
% 4 1801 1385 1801 1415 5 0 0 0
% P$_1$
\put(17.7264,-13.9272){\makebox(0,0){P$_1$}}%
\end{picture}}%
}
    水面上で$d$\tanni{m}離れた2点A,Bを周期$T$\tanni{s}で振動させ,2つの波をつくった。図は,波源A,Bから出る波の,ある時刻での山の位置を描いたものである。
        \begin{enumerate}
            \item この波の波長$\lambda $,および波の速さ$v$はいくらか。
            \item 図中の点$\mathrm{P_1}$,$\mathrm{P_2}$,$\mathrm{P_3}$はそれぞれ強めあいの位置か,弱めあいの位置か。
            \item 点A,Bから各点までの距離の差$\mathrm{AP_1-BP_1}$,$\mathrm{AP_2-BP_2}$,$\mathrm{AP_3-BP_3}$を波長$\lambda $で表せ。
            \item 線分ABを横切り,波源Aの最も近くを通る強めあいの線を図に描き入れよ。
            \item AとBを逆位相で振動させると,AB間には何本の弱めあいの線が現れるか(波源は除く)。
        \end{enumerate}
    \end{mawarikomi}
%  \vfill
%  \newpage
%  \hakosyokika
\item
    \begin{mawarikomi}(10pt,0){200pt}{%WinTpicVersion4.32a
{\unitlength 0.1in%
\begin{picture}(29.0256,12.5492)(3.4449,-23.3760)%
% BOX 2 0 3 0 Black White  
% 2 916 1100 990 1325
% 
\special{pn 8}%
\special{pa 902 1083}%
\special{pa 974 1083}%
\special{pa 974 1304}%
\special{pa 902 1304}%
\special{pa 902 1083}%
\special{pa 974 1083}%
\special{fp}%
% BOX 2 0 3 0 Black White  
% 2 916 1400 990 2075
% 
\special{pn 8}%
\special{pa 902 1378}%
\special{pa 974 1378}%
\special{pa 974 2042}%
\special{pa 902 2042}%
\special{pa 902 1378}%
\special{pa 974 1378}%
\special{fp}%
% BOX 2 0 3 0 Black White  
% 2 916 2150 990 2375
% 
\special{pn 8}%
\special{pa 902 2116}%
\special{pa 974 2116}%
\special{pa 974 2338}%
\special{pa 902 2338}%
\special{pa 902 2116}%
\special{pa 974 2116}%
\special{fp}%
% LINE 3 0 3 0 Black White  
% 24 988 2255 912 2330 988 2278 912 2352 988 2300 916 2371 988 2322 935 2375 988 2345 958 2375 988 2232 912 2308 988 2210 912 2285 988 2188 912 2262 988 2165 912 2240 980 2150 912 2218 958 2150 912 2195 935 2150 912 2172
% 
\special{pn 4}%
\special{pa 972 2219}%
\special{pa 898 2293}%
\special{fp}%
\special{pa 972 2242}%
\special{pa 898 2315}%
\special{fp}%
\special{pa 972 2264}%
\special{pa 902 2334}%
\special{fp}%
\special{pa 972 2285}%
\special{pa 920 2338}%
\special{fp}%
\special{pa 972 2308}%
\special{pa 943 2338}%
\special{fp}%
\special{pa 972 2197}%
\special{pa 898 2272}%
\special{fp}%
\special{pa 972 2175}%
\special{pa 898 2249}%
\special{fp}%
\special{pa 972 2154}%
\special{pa 898 2226}%
\special{fp}%
\special{pa 972 2131}%
\special{pa 898 2205}%
\special{fp}%
\special{pa 965 2116}%
\special{pa 898 2183}%
\special{fp}%
\special{pa 943 2116}%
\special{pa 898 2160}%
\special{fp}%
\special{pa 920 2116}%
\special{pa 898 2138}%
\special{fp}%
% LINE 3 0 3 0 Black White  
% 64 988 1962 912 2038 988 1985 912 2060 988 2008 920 2075 988 2030 942 2075 988 2052 965 2075 988 1940 912 2015 988 1918 912 1992 988 1895 912 1970 988 1872 912 1948 988 1850 912 1925 988 1828 912 1902 988 1805 912 1880 988 1782 912 1858 988 1760 912 1835 988 1738 912 1812 988 1715 912 1790 988 1692 912 1768 988 1670 912 1745 988 1648 912 1722 988 1625 912 1700 988 1602 912 1678 988 1580 912 1655 988 1558 912 1632 988 1535 912 1610 988 1512 912 1588 988 1490 912 1565 988 1468 912 1542 988 1445 912 1520 988 1422 912 1498 984 1404 912 1475 965 1400 912 1452 942 1400 912 1430
% 
\special{pn 4}%
\special{pa 972 1931}%
\special{pa 898 2006}%
\special{fp}%
\special{pa 972 1954}%
\special{pa 898 2028}%
\special{fp}%
\special{pa 972 1976}%
\special{pa 906 2042}%
\special{fp}%
\special{pa 972 1998}%
\special{pa 927 2042}%
\special{fp}%
\special{pa 972 2020}%
\special{pa 950 2042}%
\special{fp}%
\special{pa 972 1909}%
\special{pa 898 1983}%
\special{fp}%
\special{pa 972 1888}%
\special{pa 898 1961}%
\special{fp}%
\special{pa 972 1865}%
\special{pa 898 1939}%
\special{fp}%
\special{pa 972 1843}%
\special{pa 898 1917}%
\special{fp}%
\special{pa 972 1821}%
\special{pa 898 1895}%
\special{fp}%
\special{pa 972 1799}%
\special{pa 898 1872}%
\special{fp}%
\special{pa 972 1777}%
\special{pa 898 1850}%
\special{fp}%
\special{pa 972 1754}%
\special{pa 898 1829}%
\special{fp}%
\special{pa 972 1732}%
\special{pa 898 1806}%
\special{fp}%
\special{pa 972 1711}%
\special{pa 898 1783}%
\special{fp}%
\special{pa 972 1688}%
\special{pa 898 1762}%
\special{fp}%
\special{pa 972 1665}%
\special{pa 898 1740}%
\special{fp}%
\special{pa 972 1644}%
\special{pa 898 1718}%
\special{fp}%
\special{pa 972 1622}%
\special{pa 898 1695}%
\special{fp}%
\special{pa 972 1599}%
\special{pa 898 1673}%
\special{fp}%
\special{pa 972 1577}%
\special{pa 898 1652}%
\special{fp}%
\special{pa 972 1555}%
\special{pa 898 1629}%
\special{fp}%
\special{pa 972 1533}%
\special{pa 898 1606}%
\special{fp}%
\special{pa 972 1511}%
\special{pa 898 1585}%
\special{fp}%
\special{pa 972 1488}%
\special{pa 898 1563}%
\special{fp}%
\special{pa 972 1467}%
\special{pa 898 1540}%
\special{fp}%
\special{pa 972 1445}%
\special{pa 898 1518}%
\special{fp}%
\special{pa 972 1422}%
\special{pa 898 1496}%
\special{fp}%
\special{pa 972 1400}%
\special{pa 898 1474}%
\special{fp}%
\special{pa 969 1382}%
\special{pa 898 1452}%
\special{fp}%
\special{pa 950 1378}%
\special{pa 898 1429}%
\special{fp}%
\special{pa 927 1378}%
\special{pa 898 1407}%
\special{fp}%
% LINE 3 0 3 0 Black White  
% 26 988 1220 912 1295 988 1242 912 1318 988 1265 928 1325 988 1288 950 1325 988 1310 972 1325 988 1198 912 1272 988 1175 912 1250 988 1152 912 1228 988 1130 912 1205 988 1108 912 1182 972 1100 912 1160 950 1100 912 1138 928 1100 912 1115
% 
\special{pn 4}%
\special{pa 972 1201}%
\special{pa 898 1275}%
\special{fp}%
\special{pa 972 1222}%
\special{pa 898 1297}%
\special{fp}%
\special{pa 972 1245}%
\special{pa 913 1304}%
\special{fp}%
\special{pa 972 1268}%
\special{pa 935 1304}%
\special{fp}%
\special{pa 972 1289}%
\special{pa 957 1304}%
\special{fp}%
\special{pa 972 1179}%
\special{pa 898 1252}%
\special{fp}%
\special{pa 972 1156}%
\special{pa 898 1230}%
\special{fp}%
\special{pa 972 1134}%
\special{pa 898 1209}%
\special{fp}%
\special{pa 972 1112}%
\special{pa 898 1186}%
\special{fp}%
\special{pa 972 1091}%
\special{pa 898 1163}%
\special{fp}%
\special{pa 957 1083}%
\special{pa 898 1142}%
\special{fp}%
\special{pa 935 1083}%
\special{pa 898 1120}%
\special{fp}%
\special{pa 913 1083}%
\special{pa 898 1097}%
\special{fp}%
% LINE 2 1 3 0 Black White  
% 2 957 1366 3282 1366
% 
\special{pn 8}%
\special{pa 942 1344}%
\special{pa 3230 1344}%
\special{da 0.030}%
% STR 2 0 3 0 Black White  
% 4 3334 1329 3334 1366 5 0 0 0
% T
\put(32.8150,-13.4449){\makebox(0,0){T}}%
% DOT 0 1 3 0 Black White  
% 1 2753 1366
% 
\special{pn 4}%
\special{sh 1}%
\special{ar 2710 1344 16 16 0 6.2831853}%
% VECTOR 2 0 3 0 Black White  
% 4 1062 1666 1062 1366 1062 1741 1062 2112
% 
\special{pn 8}%
\special{pa 1045 1640}%
\special{pa 1045 1344}%
\special{fp}%
\special{sh 1}%
\special{pa 1045 1344}%
\special{pa 1026 1410}%
\special{pa 1045 1397}%
\special{pa 1065 1410}%
\special{pa 1045 1344}%
\special{fp}%
\special{pa 1045 1714}%
\special{pa 1045 2079}%
\special{fp}%
\special{sh 1}%
\special{pa 1045 2079}%
\special{pa 1065 2013}%
\special{pa 1045 2027}%
\special{pa 1026 2013}%
\special{pa 1045 2079}%
\special{fp}%
% STR 2 0 3 0 Black White  
% 4 1180 1662 1180 1700 5 0 1 0
% 5{\sf cm}
\put(11.6142,-16.7323){\makebox(0,0){{\colorbox[named]{White}{5{\sf cm}}}}}%
% VECTOR 2 0 3 0 Black White  
% 4 1732 1174 982 1174 1732 1174 2748 1174
% 
\special{pn 8}%
\special{pa 1705 1156}%
\special{pa 967 1156}%
\special{fp}%
\special{sh 1}%
\special{pa 967 1156}%
\special{pa 1032 1175}%
\special{pa 1019 1156}%
\special{pa 1032 1136}%
\special{pa 967 1156}%
\special{fp}%
\special{pa 1705 1156}%
\special{pa 2705 1156}%
\special{fp}%
\special{sh 1}%
\special{pa 2705 1156}%
\special{pa 2639 1136}%
\special{pa 2653 1156}%
\special{pa 2639 1175}%
\special{pa 2705 1156}%
\special{fp}%
% STR 2 0 3 0 Black White  
% 4 1774 1136 1774 1174 5 0 1 0
% 12{\sf cm}
\put(17.4606,-11.5551){\makebox(0,0){{\colorbox[named]{White}{12{\sf cm}}}}}%
% STR 2 0 3 0 Black White  
% 4 2750 1231 2750 1269 5 0 0 0
% A$_1$
\put(27.0669,-12.4902){\makebox(0,0){A$_1$}}%
% STR 2 0 3 0 Black White  
% 4 1475 1231 1475 1269 5 0 0 0
% A$_2$
\put(14.5177,-12.4902){\makebox(0,0){A$_2$}}%
% DOT 0 1 3 0 Black White  
% 1 1478 1366
% 
\special{pn 4}%
\special{sh 1}%
\special{ar 1455 1344 16 16 0 6.2831853}%
% LINE 2 0 3 0 Black White  
% 2 800 1100 800 2375
% 
\special{pn 8}%
\special{pa 787 1083}%
\special{pa 787 2338}%
\special{fp}%
% LINE 2 0 3 0 Black White  
% 2 500 2375 500 1100
% 
\special{pn 8}%
\special{pa 492 2338}%
\special{pa 492 1083}%
\special{fp}%
% LINE 2 0 3 0 Black White  
% 2 650 1100 650 2375
% 
\special{pn 8}%
\special{pa 640 1083}%
\special{pa 640 2338}%
\special{fp}%
% LINE 2 0 3 0 Black White  
% 2 350 1100 350 2375
% 
\special{pn 8}%
\special{pa 344 1083}%
\special{pa 344 2338}%
\special{fp}%
% STR 2 0 3 0 Black White  
% 4 800 1310 800 1348 5 0 1 0
% S$_1$
\put(7.8740,-13.2677){\makebox(0,0){{\colorbox[named]{White}{S$_1$}}}}%
% STR 2 0 3 0 Black White  
% 4 800 2075 800 2112 5 0 1 0
% S$_2$
\put(7.8740,-20.7874){\makebox(0,0){{\colorbox[named]{White}{S$_2$}}}}%
% STR 2 0 3 0 Black White  
% 4 500 1666 500 1704 5 0 1 0
% 平
\put(4.9213,-16.7717){\makebox(0,0){{\colorbox[named]{White}{平}}}}%
% STR 2 0 3 0 Black White  
% 4 500 1814 500 1851 5 0 1 0
% 面
\put(4.9213,-18.2185){\makebox(0,0){{\colorbox[named]{White}{面}}}}%
% STR 2 0 3 0 Black White  
% 4 500 1971 500 2009 5 0 1 0
% 波
\put(4.9213,-19.7736){\makebox(0,0){{\colorbox[named]{White}{波}}}}%
\end{picture}}%
}
        図のように,一定波長の平面波の水面波を,波面と平行に並んだ間隔5\sftanni{cm}の2つのスリット$\mathrm{S_1}$および$\mathrm{S_2}$を通して干渉させた。$\mathrm{S_1}$を通りm$\mathrm{S_1}$と$\mathrm{S_2}$を結ぶ直線$\mathrm{S_1T}$に沿って水面の動きを調べたところ,波が弱めあって,水位がほとんど変化しない場所が2つだけ見つかった。そのうち,$\mathrm{S_1}$から遠い方を$\mathrm{A_1}$,$\mathrm{S_1}$から遠い方を$\mathrm{A_2}$とすると,$\mathrm{S_1}$から$\mathrm{A_1}$までの距離は$12$\sftanni{cm}であった。
        \begin{enumerate}
            \item 距離$\mathrm{S_1A_1}$と$\mathrm{S_2A_1}$の差は,波長の何倍か。また$\mathrm{S_1A_2}$と$\mathrm{S_2A_2}$との差は,波長の何倍か。
            \item この水面波の波長は何\sftanni{cm}か。
            \item 水面には強めあいの線(双曲線や直線)が何本生じているか。
            \item 次に,スリット$\mathrm{S_1}$は固定したまま,$\mathrm{S_2}$を動かし,$\mathrm{S_1S_2}$の間隔を広げていった。このとき,直線$\mathrm{S_1}$での,水位がほとんど変化しない点の個数は増すか,減るか。また,$\mathrm{A_1}$は$\mathrm{S_1}$に近づくか遠ざかるか。
        \end{enumerate}
    \end{mawarikomi}
%  \vfill
% %  \newpage
%  \hakosyokika
\item
    \begin{mawarikomi}(10pt,0){250pt}{%WinTpicVersion4.32a
{\unitlength 0.1in%
\begin{picture}(35.9252,15.2461)(0.4921,-19.7047)%
% STR 2 0 3 0 Black White  
% 4 915 997 915 1035 2 0 0 0
% S$_1$
\put(9.0059,-10.1870){\makebox(0,0)[lb]{S$_1$}}%
% STR 2 0 3 0 Black White  
% 4 925 1413 925 1450 1 0 0 0
% S$_2$
\put(9.1043,-14.2717){\makebox(0,0)[lt]{S$_2$}}%
% LINE 0 0 3 0 Black White  
% 4 500 902 500 1202 500 1302 500 1602
% 
\special{pn 20}%
\special{pa 492 888}%
\special{pa 492 1183}%
\special{fp}%
\special{pa 492 1281}%
\special{pa 492 1577}%
\special{fp}%
% LINE 0 0 3 0 Black White  
% 6 900 602 900 1002 900 1102 900 1402 900 1502 900 2002
% 
\special{pn 20}%
\special{pa 886 593}%
\special{pa 886 986}%
\special{fp}%
\special{pa 886 1085}%
\special{pa 886 1380}%
\special{fp}%
\special{pa 886 1478}%
\special{pa 886 1970}%
\special{fp}%
% LINE 2 1 3 0 Black White  
% 2 300 1252 3700 1252
% 
\special{pn 8}%
\special{pa 295 1232}%
\special{pa 3642 1232}%
\special{da 0.030}%
% LINE 0 0 3 0 Black White  
% 2 3400 602 3400 2002
% 
\special{pn 20}%
\special{pa 3346 593}%
\special{pa 3346 1970}%
\special{fp}%
% LINE 2 1 3 0 Black White  
% 2 900 1052 700 1052
% 
\special{pn 8}%
\special{pa 886 1035}%
\special{pa 689 1035}%
\special{da 0.030}%
% LINE 2 1 3 0 Black White  
% 2 700 1450 900 1450
% 
\special{pn 8}%
\special{pa 689 1427}%
\special{pa 886 1427}%
\special{da 0.030}%
% VECTOR 2 0 3 0 Black White  
% 8 800 1150 800 1050 800 1150 800 1250 800 1350 800 1250 800 1350 800 1450
% 
\special{pn 8}%
\special{pa 787 1132}%
\special{pa 787 1033}%
\special{fp}%
\special{sh 1}%
\special{pa 787 1033}%
\special{pa 768 1099}%
\special{pa 787 1086}%
\special{pa 807 1099}%
\special{pa 787 1033}%
\special{fp}%
\special{pa 787 1132}%
\special{pa 787 1230}%
\special{fp}%
\special{sh 1}%
\special{pa 787 1230}%
\special{pa 807 1164}%
\special{pa 787 1178}%
\special{pa 768 1164}%
\special{pa 787 1230}%
\special{fp}%
\special{pa 787 1329}%
\special{pa 787 1230}%
\special{fp}%
\special{sh 1}%
\special{pa 787 1230}%
\special{pa 768 1296}%
\special{pa 787 1282}%
\special{pa 807 1296}%
\special{pa 787 1230}%
\special{fp}%
\special{pa 787 1329}%
\special{pa 787 1427}%
\special{fp}%
\special{sh 1}%
\special{pa 787 1427}%
\special{pa 807 1361}%
\special{pa 787 1375}%
\special{pa 768 1361}%
\special{pa 787 1427}%
\special{fp}%
% STR 2 0 3 0 Black White  
% 4 705 1100 705 1150 5 0 0 0
% $a$
\put(6.9390,-11.3189){\makebox(0,0){$a$}}%
% STR 2 0 3 0 Black White  
% 4 705 1300 705 1350 5 0 0 0
% $a$
\put(6.9390,-13.2874){\makebox(0,0){$a$}}%
% LINE 2 0 3 0 Black White  
% 4 900 1050 3400 850 900 1450 3400 850
% 
\special{pn 8}%
\special{pa 886 1033}%
\special{pa 3346 837}%
\special{fp}%
\special{pa 886 1427}%
\special{pa 3346 837}%
\special{fp}%
% LINE 2 1 3 0 Black White  
% 2 3400 850 3600 850
% 
\special{pn 8}%
\special{pa 3346 837}%
\special{pa 3543 837}%
\special{da 0.030}%
% VECTOR 2 0 3 0 Black White  
% 4 3500 1050 3500 850 3500 1050 3500 1250
% 
\special{pn 8}%
\special{pa 3445 1033}%
\special{pa 3445 837}%
\special{fp}%
\special{sh 1}%
\special{pa 3445 837}%
\special{pa 3425 903}%
\special{pa 3445 889}%
\special{pa 3465 903}%
\special{pa 3445 837}%
\special{fp}%
\special{pa 3445 1033}%
\special{pa 3445 1230}%
\special{fp}%
\special{sh 1}%
\special{pa 3445 1230}%
\special{pa 3465 1164}%
\special{pa 3445 1178}%
\special{pa 3425 1164}%
\special{pa 3445 1230}%
\special{fp}%
% STR 2 0 3 0 Black White  
% 4 3580 1000 3580 1050 5 0 0 0
% $x$
\put(35.2362,-10.3346){\makebox(0,0){$x$}}%
% STR 2 0 3 0 Black White  
% 4 3415 797 3415 835 2 0 1 0
% P
\put(33.6122,-8.2185){\makebox(0,0)[lb]{{\colorbox[named]{White}{P}}}}%
% STR 2 0 3 0 Black White  
% 4 3415 1238 3415 1275 1 0 1 0
% O
\put(33.6122,-12.5492){\makebox(0,0)[lt]{{\colorbox[named]{White}{O}}}}%
% VECTOR 2 0 3 0 Black White  
% 4 2000 1875 900 1875 2000 1875 3400 1875
% 
\special{pn 8}%
\special{pa 1969 1845}%
\special{pa 886 1845}%
\special{fp}%
\special{sh 1}%
\special{pa 886 1845}%
\special{pa 952 1865}%
\special{pa 938 1845}%
\special{pa 952 1826}%
\special{pa 886 1845}%
\special{fp}%
\special{pa 1969 1845}%
\special{pa 3346 1845}%
\special{fp}%
\special{sh 1}%
\special{pa 3346 1845}%
\special{pa 3281 1826}%
\special{pa 3294 1845}%
\special{pa 3281 1865}%
\special{pa 3346 1845}%
\special{fp}%
% STR 2 0 3 0 Black White  
% 4 2100 1825 2100 1875 5 0 1 0
% $\ell $
\put(20.6693,-18.4547){\makebox(0,0){{\colorbox[named]{White}{$\ell $}}}}%
% CIRCLE 2 0 3 0 Black White  
% 4 170 1250 225 1250 225 1250 225 1250
% 
\special{pn 8}%
\special{ar 167 1230 54 54 0.0000000 6.2831853}%
% LINE 2 0 3 0 Black White  
% 2 244 1250 269 1250
% 
\special{pn 8}%
\special{pa 240 1230}%
\special{pa 265 1230}%
\special{fp}%
% LINE 2 0 3 1 Black White  
% 2 72 1250 97 1250
% 
\special{pn 8}%
\special{pa 71 1230}%
\special{pa 95 1230}%
\special{fp}%
% LINE 2 0 3 0 Black White  
% 2 172 1178 172 1153
% 
\special{pn 8}%
\special{pa 169 1159}%
\special{pa 169 1135}%
\special{fp}%
% LINE 2 0 3 1 Black White  
% 2 172 1350 172 1325
% 
\special{pn 8}%
\special{pa 169 1329}%
\special{pa 169 1304}%
\special{fp}%
% LINE 2 0 3 0 Black White  
% 2 118 1198 101 1181
% 
\special{pn 8}%
\special{pa 116 1179}%
\special{pa 99 1162}%
\special{fp}%
% LINE 2 0 3 1 Black White  
% 2 240 1320 222 1302
% 
\special{pn 8}%
\special{pa 236 1299}%
\special{pa 219 1281}%
\special{fp}%
% LINE 2 0 3 0 Black White  
% 2 222 1198 239 1181
% 
\special{pn 8}%
\special{pa 219 1179}%
\special{pa 235 1162}%
\special{fp}%
% LINE 2 0 3 1 Black White  
% 2 100 1320 118 1302
% 
\special{pn 8}%
\special{pa 98 1299}%
\special{pa 116 1281}%
\special{fp}%
% STR 2 0 3 0 Black White  
% 4 170 1018 170 1038 5 0 0 0
% 光源
\put(1.6732,-10.2165){\makebox(0,0){光源}}%
% STR 2 0 3 0 Black White  
% 4 458 778 458 798 5 0 0 0
% スリット面I
\put(4.5079,-7.8543){\makebox(0,0){スリット面I}}%
% STR 2 0 3 0 Black White  
% 4 898 498 898 518 5 0 0 0
% スリット面II
\put(8.8386,-5.0984){\makebox(0,0){スリット面II}}%
% STR 2 0 3 0 Black White  
% 4 3404 498 3404 518 5 0 0 0
% スクリーン
\put(33.5039,-5.0984){\makebox(0,0){スクリーン}}%
% STR 2 0 3 0 Black White  
% 4 492 1218 492 1256 4 0 0 0
% S$_0$
\put(4.8425,-12.3622){\makebox(0,0)[rt]{S$_0$}}%
\end{picture}}%
}
        図で$\mathrm{S_0}$,$\mathrm{S_1}$,$\mathrm{S_2}$は互いに平行なスリットである。
        $\mathrm{S_1}$,$\mathrm{S_2}$は間隔が$2a$で,$\mathrm{S_0}$から等距離にある。
        スクリーンはスリット面IおよびIIに平行で,面IIから$\ell $だけ離してある。
        $\mathrm{S_1}$と$\mathrm{S_2}$の中点からスクリーンに下した垂線の足をOとし,Oから距離$x$だけ離れたスクリーン上の点をPとする。ここで,$a$および$x$は$\ell $に比べて十分小さい。光源から出た波長$\lambda $の単色光を$S_0$にあてると,スクリーン上に明暗のしまが現れる。
        \begin{description}
            \item[A] まず,空気(屈折率1)中に置かれた装置で実験する。
            \begin{Enumerate}
                \item P点が暗くなるとき,$x$が満たしている条件を整数$m$を用いて表せ。
                \item $a$を$0.47$\sftanni{mm},$\ell $を$6.1$\sftanni{m}にとって実験をした。このとき,スクリーン上に現れた暗線の間隔は$4.1$\sftanni{mm}であった。単色光の波長は何\sftanni{m}か。
            \end{Enumerate}
        \end{description}
        \begin{description}
            \item[B] 次に,装置の一部を屈折率$n$の媒質で満たす。
            \begin{Enumerate*}
                \item スリット面Iとスリット面IIの間だけを,この媒質で満たしたとき,暗線の間隔は,Aの場合の何倍になるか。
                \item スリットIIとスクリーンの間だけを,この媒質で満たしたとき,暗線の間隔は,Aの場合の何倍になるか。
            \end{Enumerate*} 
        \end{description}
\end{mawarikomi}
%  \vfill
%  \newpage
%  \hakosyokika
\item
    \begin{mawarikomi}(10pt,0){160pt}{%WinTpicVersion4.32a
{\unitlength 0.1in%
\begin{picture}(15.7185,37.0866)(4.1339,-37.3031)%
% VECTOR 2 0 3 0 Black White  
% 2 1045 1986 1045 195
% 
\special{pn 8}%
\special{pa 1029 1955}%
\special{pa 1029 192}%
\special{fp}%
\special{sh 1}%
\special{pa 1029 192}%
\special{pa 1009 258}%
\special{pa 1029 244}%
\special{pa 1048 258}%
\special{pa 1029 192}%
\special{fp}%
% LINE 2 1 3 0 Black White  
% 2 1045 1091 1941 1091
% 
\special{pn 8}%
\special{pa 1029 1074}%
\special{pa 1910 1074}%
\special{da 0.030}%
% CIRCLE 2 1 3 0 Black White  
% 4 1045 1091 1792 1091 1045 1837 1045 344
% 
\special{pn 8}%
\special{pn 8}%
\special{pa 1029 339}%
\special{pa 1049 339}%
\special{pa 1055 340}%
\special{pa 1056 340}%
\special{fp}%
\special{pa 1083 341}%
\special{pa 1090 342}%
\special{pa 1095 342}%
\special{pa 1109 344}%
\special{pa 1110 344}%
\special{fp}%
\special{pa 1137 346}%
\special{pa 1150 348}%
\special{pa 1156 349}%
\special{pa 1162 351}%
\special{pa 1164 351}%
\special{fp}%
\special{pa 1191 356}%
\special{pa 1196 358}%
\special{pa 1202 359}%
\special{pa 1209 361}%
\special{pa 1216 362}%
\special{pa 1217 362}%
\special{fp}%
\special{pa 1243 371}%
\special{pa 1248 372}%
\special{pa 1254 374}%
\special{pa 1261 376}%
\special{pa 1267 378}%
\special{pa 1269 379}%
\special{fp}%
\special{pa 1294 389}%
\special{pa 1298 390}%
\special{pa 1305 393}%
\special{pa 1311 395}%
\special{pa 1319 399}%
\special{fp}%
\special{pa 1345 410}%
\special{pa 1347 411}%
\special{pa 1353 414}%
\special{pa 1360 417}%
\special{pa 1369 421}%
\special{fp}%
\special{pa 1393 435}%
\special{pa 1396 437}%
\special{pa 1402 440}%
\special{pa 1407 444}%
\special{pa 1412 447}%
\special{pa 1415 449}%
\special{fp}%
\special{pa 1438 464}%
\special{pa 1441 466}%
\special{pa 1447 469}%
\special{pa 1452 472}%
\special{pa 1461 478}%
\special{fp}%
\special{pa 1483 496}%
\special{pa 1484 497}%
\special{pa 1490 501}%
\special{pa 1495 506}%
\special{pa 1500 510}%
\special{pa 1504 512}%
\special{fp}%
\special{pa 1524 531}%
\special{pa 1531 536}%
\special{pa 1535 541}%
\special{pa 1540 546}%
\special{pa 1544 549}%
\special{fp}%
\special{pa 1563 569}%
\special{pa 1569 575}%
\special{pa 1573 580}%
\special{pa 1578 585}%
\special{pa 1581 589}%
\special{fp}%
\special{pa 1599 610}%
\special{pa 1599 610}%
\special{pa 1603 616}%
\special{pa 1608 621}%
\special{pa 1612 626}%
\special{pa 1615 631}%
\special{fp}%
\special{pa 1631 653}%
\special{pa 1632 654}%
\special{pa 1640 664}%
\special{pa 1643 670}%
\special{pa 1647 675}%
\special{fp}%
\special{pa 1661 699}%
\special{pa 1661 699}%
\special{pa 1667 711}%
\special{pa 1671 717}%
\special{pa 1674 722}%
\special{pa 1674 722}%
\special{fp}%
\special{pa 1687 746}%
\special{pa 1690 752}%
\special{pa 1693 759}%
\special{pa 1699 770}%
\special{fp}%
\special{pa 1709 796}%
\special{pa 1712 801}%
\special{pa 1714 808}%
\special{pa 1717 814}%
\special{pa 1719 821}%
\special{fp}%
\special{pa 1727 847}%
\special{pa 1729 852}%
\special{pa 1731 859}%
\special{pa 1733 865}%
\special{pa 1735 872}%
\special{pa 1735 873}%
\special{fp}%
\special{pa 1742 899}%
\special{pa 1744 905}%
\special{pa 1745 910}%
\special{pa 1747 917}%
\special{pa 1748 924}%
\special{pa 1748 926}%
\special{fp}%
\special{pa 1753 953}%
\special{pa 1755 958}%
\special{pa 1757 970}%
\special{pa 1758 977}%
\special{pa 1758 979}%
\special{fp}%
\special{pa 1761 1006}%
\special{pa 1761 1011}%
\special{pa 1762 1018}%
\special{pa 1762 1025}%
\special{pa 1763 1031}%
\special{pa 1763 1033}%
\special{fp}%
\special{pa 1764 1061}%
\special{pa 1764 1088}%
\special{fp}%
\special{pa 1763 1115}%
\special{pa 1763 1118}%
\special{pa 1762 1125}%
\special{pa 1762 1132}%
\special{pa 1761 1142}%
\special{fp}%
\special{pa 1757 1169}%
\special{pa 1756 1179}%
\special{pa 1755 1185}%
\special{pa 1753 1196}%
\special{fp}%
\special{pa 1748 1223}%
\special{pa 1748 1225}%
\special{pa 1747 1232}%
\special{pa 1745 1238}%
\special{pa 1743 1245}%
\special{pa 1742 1249}%
\special{fp}%
\special{pa 1736 1276}%
\special{pa 1735 1278}%
\special{pa 1731 1290}%
\special{pa 1729 1297}%
\special{pa 1728 1301}%
\special{fp}%
\special{pa 1719 1327}%
\special{pa 1719 1329}%
\special{pa 1716 1335}%
\special{pa 1714 1342}%
\special{pa 1711 1347}%
\special{pa 1709 1352}%
\special{fp}%
\special{pa 1699 1378}%
\special{pa 1698 1379}%
\special{pa 1687 1402}%
\special{fp}%
\special{pa 1674 1426}%
\special{pa 1673 1427}%
\special{pa 1667 1439}%
\special{pa 1663 1444}%
\special{pa 1660 1449}%
\special{fp}%
\special{pa 1646 1473}%
\special{pa 1646 1473}%
\special{pa 1643 1478}%
\special{pa 1635 1490}%
\special{pa 1631 1495}%
\special{fp}%
\special{pa 1615 1517}%
\special{pa 1611 1523}%
\special{pa 1607 1528}%
\special{pa 1602 1533}%
\special{pa 1598 1538}%
\special{fp}%
\special{pa 1581 1559}%
\special{pa 1577 1564}%
\special{pa 1567 1574}%
\special{pa 1563 1579}%
\special{pa 1563 1579}%
\special{fp}%
\special{pa 1543 1599}%
\special{pa 1539 1603}%
\special{pa 1534 1607}%
\special{pa 1530 1612}%
\special{pa 1525 1617}%
\special{pa 1524 1618}%
\special{fp}%
\special{pa 1503 1635}%
\special{pa 1494 1643}%
\special{pa 1488 1648}%
\special{pa 1483 1652}%
\special{fp}%
\special{pa 1461 1669}%
\special{pa 1457 1672}%
\special{pa 1451 1676}%
\special{pa 1445 1679}%
\special{pa 1440 1683}%
\special{pa 1439 1684}%
\special{fp}%
\special{pa 1416 1699}%
\special{pa 1411 1702}%
\special{pa 1406 1705}%
\special{pa 1400 1709}%
\special{pa 1394 1712}%
\special{pa 1393 1713}%
\special{fp}%
\special{pa 1368 1725}%
\special{pa 1364 1728}%
\special{pa 1346 1737}%
\special{pa 1344 1738}%
\special{fp}%
\special{pa 1319 1749}%
\special{pa 1316 1751}%
\special{pa 1309 1753}%
\special{pa 1297 1759}%
\special{pa 1295 1760}%
\special{fp}%
\special{pa 1269 1769}%
\special{pa 1265 1770}%
\special{pa 1259 1772}%
\special{pa 1252 1775}%
\special{pa 1246 1777}%
\special{pa 1243 1778}%
\special{fp}%
\special{pa 1217 1784}%
\special{pa 1214 1785}%
\special{pa 1207 1787}%
\special{pa 1201 1788}%
\special{pa 1194 1790}%
\special{pa 1191 1790}%
\special{fp}%
\special{pa 1164 1796}%
\special{pa 1160 1797}%
\special{pa 1155 1798}%
\special{pa 1137 1801}%
\special{fp}%
\special{pa 1110 1805}%
\special{pa 1107 1805}%
\special{pa 1100 1805}%
\special{pa 1088 1807}%
\special{pa 1083 1807}%
\special{fp}%
\special{pa 1056 1809}%
\special{pa 1054 1809}%
\special{pa 1029 1809}%
\special{fp}%
% CIRCLE 1 0 3 0 Black White  
% 4 1045 1091 1792 1091 1493 1837 1493 344
% 
\special{pn 13}%
\special{ar 1029 1074 735 735 5.2526116 1.0299827}%
% LINE 2 1 3 0 Black White  
% 2 1045 1091 1418 449
% 
\special{pn 8}%
\special{pa 1029 1074}%
\special{pa 1396 442}%
\special{da 0.030}%
% LINE 2 1 3 0 Black White  
% 2 1045 1091 1418 1732
% 
\special{pn 8}%
\special{pa 1029 1074}%
\special{pa 1396 1705}%
\special{da 0.030}%
% VECTOR 2 0 3 0 Black White  
% 2 1045 1091 1717 792
% 
\special{pn 8}%
\special{pa 1029 1074}%
\special{pa 1690 780}%
\special{fp}%
\special{sh 1}%
\special{pa 1690 780}%
\special{pa 1622 788}%
\special{pa 1642 801}%
\special{pa 1638 824}%
\special{pa 1690 780}%
\special{fp}%
% CIRCLE 2 0 3 0 Black White  
% 4 1045 1091 1344 1091 1792 1091 1717 792
% 
\special{pn 8}%
\special{ar 1029 1074 294 294 5.8645468 6.2831853}%
% STR 2 0 3 0 Black White  
% 4 1366 991 1366 1066 2 0 0 0
% $\theta $
\put(13.4449,-10.4921){\makebox(0,0)[lb]{$\theta $}}%
% LINE 2 0 3 0 Black White  
% 2 1008 344 1083 344
% 
\special{pn 8}%
\special{pa 992 339}%
\special{pa 1066 339}%
\special{fp}%
% LINE 2 0 3 0 Black White  
% 2 1008 1837 1083 1837
% 
\special{pn 8}%
\special{pa 992 1808}%
\special{pa 1066 1808}%
\special{fp}%
% STR 2 0 3 0 Black White  
% 4 909 270 909 344 5 0 0 0
% 1.0
\put(8.9469,-3.3858){\makebox(0,0){1.0}}%
% STR 2 0 3 0 Black White  
% 4 859 1762 859 1837 5 0 0 0
% $-1.0$
\put(8.4547,-18.0807){\makebox(0,0){$-1.0$}}%
% BOX 2 0 3 0 Black White  
% 2 1049 994 1009 1194
% 
\special{pn 8}%
\special{pa 1032 978}%
\special{pa 993 978}%
\special{pa 993 1175}%
\special{pa 1032 1175}%
\special{pa 1032 978}%
\special{pa 993 978}%
\special{fp}%
% LINE 3 0 3 0 Black White  
% 16 1045 1145 1005 1185 1045 1175 1030 1190 1045 1115 1005 1155 1045 1085 1005 1125 1045 1055 1005 1095 1045 1025 1005 1065 1045 995 1005 1035 1020 990 1005 1005
% 
\special{pn 4}%
\special{pa 1029 1127}%
\special{pa 989 1166}%
\special{fp}%
\special{pa 1029 1156}%
\special{pa 1014 1171}%
\special{fp}%
\special{pa 1029 1097}%
\special{pa 989 1137}%
\special{fp}%
\special{pa 1029 1068}%
\special{pa 989 1107}%
\special{fp}%
\special{pa 1029 1038}%
\special{pa 989 1078}%
\special{fp}%
\special{pa 1029 1009}%
\special{pa 989 1048}%
\special{fp}%
\special{pa 1029 979}%
\special{pa 989 1019}%
\special{fp}%
\special{pa 1004 974}%
\special{pa 989 989}%
\special{fp}%
% LINE 3 0 3 0 Black White  
% 2 1009 994 809 794
% 
\special{pn 4}%
\special{pa 993 978}%
\special{pa 796 781}%
\special{fp}%
% STR 2 0 3 0 Black White  
% 4 420 740 420 790 2 0 0 0
% 回折格子
\put(4.1339,-7.7756){\makebox(0,0)[lb]{回折格子}}%
% VECTOR 1 0 3 0 Black White  
% 2 550 1090 985 1090
% 
\special{pn 13}%
\special{pa 541 1073}%
\special{pa 969 1073}%
\special{fp}%
\special{sh 1}%
\special{pa 969 1073}%
\special{pa 904 1053}%
\special{pa 917 1073}%
\special{pa 904 1093}%
\special{pa 969 1073}%
\special{fp}%
% STR 2 0 3 0 Black White  
% 4 505 1025 505 1075 2 0 0 0
% 単色光
\put(4.9705,-10.5807){\makebox(0,0)[lb]{単色光}}%
% STR 2 0 3 0 Black White  
% 4 1317 1985 1317 2035 2 0 0 0
% 図1
\put(12.9626,-20.0295){\makebox(0,0)[lb]{図1}}%
% STR 2 0 3 0 Black White  
% 4 1045 20 1045 95 5 0 0 0
% $x$\kern-4pt〔{\sf m}〕
\put(10.2854,-0.9350){\makebox(0,0){$x$\kern-4pt〔{\sf m}〕}}%
% VECTOR 2 0 3 0 Black White  
% 2 1042 3695 1042 2395
% 
\special{pn 8}%
\special{pa 1026 3637}%
\special{pa 1026 2357}%
\special{fp}%
\special{sh 1}%
\special{pa 1026 2357}%
\special{pa 1006 2423}%
\special{pa 1026 2409}%
\special{pa 1045 2423}%
\special{pa 1026 2357}%
\special{fp}%
% BOX 2 0 3 0 Black White  
% 2 1046 3394 1006 3594
% 
\special{pn 8}%
\special{pa 1030 3341}%
\special{pa 990 3341}%
\special{pa 990 3537}%
\special{pa 1030 3537}%
\special{pa 1030 3341}%
\special{pa 990 3341}%
\special{fp}%
% LINE 3 0 3 0 Black White  
% 16 1042 3545 1002 3585 1042 3575 1027 3590 1042 3515 1002 3555 1042 3485 1002 3525 1042 3455 1002 3495 1042 3425 1002 3465 1042 3395 1002 3435 1017 3390 1002 3405
% 
\special{pn 4}%
\special{pa 1026 3489}%
\special{pa 986 3529}%
\special{fp}%
\special{pa 1026 3519}%
\special{pa 1011 3533}%
\special{fp}%
\special{pa 1026 3460}%
\special{pa 986 3499}%
\special{fp}%
\special{pa 1026 3430}%
\special{pa 986 3469}%
\special{fp}%
\special{pa 1026 3401}%
\special{pa 986 3440}%
\special{fp}%
\special{pa 1026 3371}%
\special{pa 986 3410}%
\special{fp}%
\special{pa 1026 3342}%
\special{pa 986 3381}%
\special{fp}%
\special{pa 1001 3337}%
\special{pa 986 3351}%
\special{fp}%
% LINE 2 1 3 0 Black White  
% 2 1042 3486 1342 3486
% 
\special{pn 8}%
\special{pa 1026 3431}%
\special{pa 1321 3431}%
\special{da 0.030}%
% STR 2 0 3 0 Black White  
% 4 982 3725 982 3775 2 0 0 0
% ~
\put(9.6654,-37.1555){\makebox(0,0)[lb]{~}}%
% SPLINE 2 0 3 0 Black White  
% 4 1371 3430 1357 3459 1371 3501 1357 3530
% 
\special{pn 8}%
\special{pa 1349 3376}%
\special{pa 1336 3405}%
\special{pa 1346 3434}%
\special{pa 1344 3464}%
\special{pa 1336 3474}%
\special{fp}%
% CIRCLE 2 1 3 0 Black White  
% 4 1042 3485 2042 3485 2242 3185 1042 3185
% 
\special{pn 8}%
\special{pn 8}%
\special{pa 1026 2446}%
\special{pa 1053 2446}%
\special{fp}%
\special{pa 1080 2447}%
\special{pa 1082 2448}%
\special{pa 1094 2448}%
\special{pa 1096 2449}%
\special{pa 1106 2449}%
\special{fp}%
\special{pa 1132 2452}%
\special{pa 1136 2452}%
\special{pa 1139 2453}%
\special{pa 1144 2453}%
\special{pa 1147 2454}%
\special{pa 1153 2454}%
\special{pa 1156 2455}%
\special{pa 1159 2455}%
\special{fp}%
\special{pa 1185 2459}%
\special{pa 1186 2459}%
\special{pa 1189 2460}%
\special{pa 1191 2460}%
\special{pa 1194 2461}%
\special{pa 1197 2461}%
\special{pa 1200 2462}%
\special{pa 1203 2462}%
\special{pa 1206 2463}%
\special{pa 1208 2463}%
\special{pa 1211 2464}%
\special{fp}%
\special{pa 1237 2469}%
\special{pa 1241 2469}%
\special{pa 1247 2471}%
\special{pa 1250 2471}%
\special{pa 1252 2472}%
\special{pa 1255 2473}%
\special{pa 1258 2473}%
\special{pa 1261 2474}%
\special{pa 1263 2475}%
\special{fp}%
\special{pa 1289 2482}%
\special{pa 1293 2483}%
\special{pa 1296 2483}%
\special{pa 1299 2484}%
\special{pa 1301 2485}%
\special{pa 1310 2488}%
\special{pa 1312 2488}%
\special{pa 1314 2489}%
\special{fp}%
\special{pa 1340 2497}%
\special{pa 1340 2497}%
\special{pa 1344 2499}%
\special{pa 1347 2500}%
\special{pa 1349 2501}%
\special{pa 1358 2504}%
\special{pa 1360 2505}%
\special{pa 1365 2507}%
\special{fp}%
\special{pa 1390 2516}%
\special{pa 1390 2516}%
\special{pa 1392 2517}%
\special{pa 1398 2519}%
\special{pa 1400 2520}%
\special{pa 1406 2522}%
\special{pa 1407 2523}%
\special{pa 1410 2525}%
\special{pa 1413 2526}%
\special{pa 1414 2527}%
\special{fp}%
\special{pa 1439 2536}%
\special{pa 1441 2538}%
\special{pa 1447 2540}%
\special{pa 1449 2541}%
\special{pa 1452 2543}%
\special{pa 1454 2544}%
\special{pa 1460 2546}%
\special{pa 1462 2548}%
\special{pa 1463 2548}%
\special{fp}%
\special{pa 1486 2561}%
\special{pa 1487 2561}%
\special{pa 1489 2562}%
\special{pa 1492 2563}%
\special{pa 1494 2565}%
\special{pa 1497 2566}%
\special{pa 1499 2567}%
\special{pa 1502 2569}%
\special{pa 1504 2570}%
\special{pa 1507 2572}%
\special{pa 1509 2573}%
\special{pa 1510 2573}%
\special{fp}%
\special{pa 1533 2587}%
\special{pa 1533 2587}%
\special{pa 1536 2589}%
\special{pa 1538 2590}%
\special{pa 1540 2592}%
\special{pa 1543 2593}%
\special{pa 1545 2594}%
\special{pa 1548 2596}%
\special{pa 1550 2597}%
\special{pa 1553 2599}%
\special{pa 1555 2600}%
\special{pa 1555 2600}%
\special{fp}%
\special{pa 1578 2615}%
\special{pa 1581 2618}%
\special{pa 1584 2619}%
\special{pa 1586 2621}%
\special{pa 1588 2622}%
\special{pa 1591 2624}%
\special{pa 1593 2626}%
\special{pa 1595 2627}%
\special{pa 1598 2630}%
\special{fp}%
\special{pa 1620 2646}%
\special{pa 1622 2648}%
\special{pa 1625 2649}%
\special{pa 1631 2655}%
\special{pa 1634 2656}%
\special{pa 1636 2657}%
\special{pa 1640 2661}%
\special{pa 1641 2662}%
\special{fp}%
\special{pa 1662 2679}%
\special{pa 1662 2679}%
\special{pa 1670 2687}%
\special{pa 1672 2688}%
\special{pa 1675 2690}%
\special{pa 1682 2697}%
\special{fp}%
\special{pa 1702 2715}%
\special{pa 1706 2719}%
\special{pa 1708 2720}%
\special{pa 1712 2724}%
\special{pa 1715 2726}%
\special{pa 1719 2730}%
\special{pa 1720 2732}%
\special{pa 1721 2733}%
\special{fp}%
\special{pa 1739 2752}%
\special{pa 1748 2761}%
\special{pa 1749 2763}%
\special{pa 1751 2766}%
\special{pa 1757 2772}%
\special{fp}%
\special{pa 1775 2792}%
\special{pa 1776 2793}%
\special{pa 1780 2797}%
\special{pa 1782 2801}%
\special{pa 1784 2804}%
\special{pa 1792 2812}%
\special{pa 1792 2812}%
\special{fp}%
\special{pa 1808 2834}%
\special{pa 1809 2835}%
\special{pa 1811 2837}%
\special{pa 1813 2840}%
\special{pa 1814 2842}%
\special{pa 1820 2848}%
\special{pa 1821 2850}%
\special{pa 1823 2852}%
\special{pa 1825 2854}%
\special{fp}%
\special{pa 1840 2877}%
\special{pa 1841 2878}%
\special{pa 1843 2881}%
\special{pa 1845 2885}%
\special{pa 1846 2888}%
\special{pa 1848 2890}%
\special{pa 1850 2893}%
\special{pa 1851 2895}%
\special{pa 1853 2897}%
\special{pa 1854 2899}%
\special{fp}%
\special{pa 1868 2921}%
\special{pa 1869 2923}%
\special{pa 1871 2926}%
\special{pa 1872 2928}%
\special{pa 1874 2931}%
\special{pa 1875 2933}%
\special{pa 1877 2936}%
\special{pa 1878 2938}%
\special{pa 1880 2941}%
\special{pa 1881 2943}%
\special{pa 1881 2944}%
\special{fp}%
\special{pa 1894 2967}%
\special{pa 1895 2968}%
\special{pa 1897 2972}%
\special{pa 1899 2975}%
\special{pa 1900 2977}%
\special{pa 1901 2980}%
\special{pa 1903 2983}%
\special{pa 1904 2985}%
\special{pa 1905 2988}%
\special{pa 1906 2990}%
\special{pa 1906 2991}%
\special{fp}%
\special{pa 1918 3015}%
\special{pa 1919 3019}%
\special{pa 1920 3021}%
\special{pa 1922 3024}%
\special{pa 1923 3026}%
\special{pa 1924 3029}%
\special{pa 1926 3033}%
\special{pa 1928 3036}%
\special{pa 1929 3039}%
\special{pa 1929 3039}%
\special{fp}%
\special{pa 1940 3064}%
\special{pa 1941 3068}%
\special{pa 1942 3070}%
\special{pa 1944 3076}%
\special{pa 1945 3078}%
\special{pa 1948 3087}%
\special{pa 1949 3089}%
\special{pa 1949 3089}%
\special{fp}%
\special{pa 1959 3115}%
\special{pa 1959 3115}%
\special{pa 1959 3118}%
\special{pa 1960 3121}%
\special{pa 1961 3123}%
\special{pa 1964 3132}%
\special{pa 1965 3134}%
\special{pa 1966 3137}%
\special{pa 1966 3140}%
\special{fp}%
\special{pa 1974 3166}%
\special{pa 1974 3166}%
\special{pa 1974 3169}%
\special{pa 1976 3175}%
\special{pa 1977 3177}%
\special{pa 1977 3180}%
\special{pa 1979 3186}%
\special{pa 1979 3189}%
\special{pa 1980 3191}%
\special{fp}%
% CIRCLE 1 0 3 0 Black White  
% 4 1042 3485 2042 3485 2142 3185 1542 2185
% 
\special{pn 13}%
\special{ar 1026 3430 984 984 5.0795628 6.0169333}%
% LINE 2 1 3 0 Black White  
% 2 1042 2845 1807 2845
% 
\special{pn 8}%
\special{pa 1026 2800}%
\special{pa 1779 2800}%
\special{da 0.030}%
% LINE 2 1 3 0 Black White  
% 2 1042 3165 1987 3165
% 
\special{pn 8}%
\special{pa 1026 3115}%
\special{pa 1956 3115}%
\special{da 0.030}%
% LINE 2 1 3 0 Black White  
% 2 1042 3005 1917 3005
% 
\special{pn 8}%
\special{pa 1026 2958}%
\special{pa 1887 2958}%
\special{da 0.030}%
% STR 2 0 3 0 Black White  
% 4 882 3115 882 3165 5 0 0 0
% 0.32
\put(8.6811,-31.1516){\makebox(0,0){0.32}}%
% STR 2 0 3 0 Black White  
% 4 882 2955 882 3005 5 0 0 0
% 0.48
\put(8.6811,-29.5768){\makebox(0,0){0.48}}%
% STR 2 0 3 0 Black White  
% 4 882 2795 882 2845 5 0 0 0
% 0.64
\put(8.6811,-28.0020){\makebox(0,0){0.64}}%
% STR 2 0 3 0 Black White  
% 4 1952 3265 1952 3315 2 0 0 0
% ~
\put(19.2126,-32.6280){\makebox(0,0)[lb]{~}}%
% STR 2 0 3 0 Black White  
% 4 882 2435 882 2485 5 0 0 0
% 1.0
\put(8.6811,-24.4587){\makebox(0,0){1.0}}%
% STR 2 0 3 0 Black White  
% 4 1042 2220 1042 2295 5 0 0 0
% $x$\kern-4pt〔{\sf m}〕
\put(10.2559,-22.5886){\makebox(0,0){$x$\kern-4pt〔{\sf m}〕}}%
% STR 2 0 3 0 Black White  
% 4 1832 2780 1832 2830 2 0 0 0
% R
\put(18.0315,-27.8543){\makebox(0,0)[lb]{R}}%
% DOT 0 0 3 0 Black White  
% 3 1812 2845 1917 3005 1987 3165
% 
\special{pn 4}%
\special{sh 1}%
\special{ar 1783 2800 16 16 0 6.2831853}%
\special{sh 1}%
\special{ar 1887 2958 16 16 0 6.2831853}%
\special{sh 1}%
\special{ar 1956 3115 16 16 0 6.2831853}%
% STR 2 0 3 0 Black White  
% 4 1942 2945 1942 2995 2 0 0 0
% Q
\put(19.1142,-29.4783){\makebox(0,0)[lb]{Q}}%
% STR 2 0 3 0 Black White  
% 4 2017 3115 2017 3165 2 0 0 0
% P
\put(19.8524,-31.1516){\makebox(0,0)[lb]{P}}%
% LINE 3 0 3 0 Black White  
% 2 1900 425 1635 640
% 
\special{pn 4}%
\special{pa 1870 418}%
\special{pa 1609 630}%
\special{fp}%
% STR 2 0 3 0 Black White  
% 4 1620 370 1620 420 2 0 0 0
% スクリーン
\put(15.9449,-4.1339){\makebox(0,0)[lb]{スクリーン}}%
% STR 2 0 3 0 Black White  
% 4 1717 3370 1717 3420 2 0 0 0
% スクリーン
\put(16.8996,-33.6614){\makebox(0,0)[lb]{スクリーン}}%
% STR 2 0 3 0 Black White  
% 4 965 1090 965 1140 4 0 0 0
% O
\put(9.4980,-11.2205){\makebox(0,0)[rt]{O}}%
% STR 2 0 3 0 Black White  
% 4 1057 3450 1057 3500 1 0 0 0
% O
\put(10.4035,-34.4488){\makebox(0,0)[lt]{O}}%
% STR 2 0 3 0 Black White  
% 4 1317 3885 1317 3935 2 0 0 0
% 図2
\put(12.9626,-38.7303){\makebox(0,0)[lb]{図2}}%
\end{picture}}%
}
    格子定数$d$の回折格子に垂直に単色光を入射させ,入射光の進行方向と回折光の進行方向のなす角度を$\theta $
    として,円筒状のスクリーン上に現れる明線を$-60\Deg <\theta <60\Deg $の範囲で観測する。
    回折格子の位置を原点Oとして,入射光および円筒の中心軸に垂直な方向に$x$軸を定める。
        \begin{enumerate}
            \item $d=1.2\times 10^{-6}$\sftanni{m}の回折格子に,波長$\lambda =6.0\times 10^{-7}$\sftanni{m}光を入射させたとき,スクリーン上($-60\Deg <\theta <60\Deg $)に現れる明線の本数は何本か。
            \item 格子定数の分かっていない回折格子に取り替えた。この回折格子に,赤色の単色光と青色の単色光を同時に入射させたところ,スクリーンの$0\Deg < \theta <60\Deg $の範囲には,図2のP,Q,Rの位置にのみ明線が観測された。3本の明線のうち,青色の明線はどれか。次のうちから選べ。
            \begin{edaenumerate}<3>[m]
                \item Pのみ
                \item Qのみ
                \item Rのみ
                \item PとQ
                \item QとR
                \item PとR
            \end{edaenumerate}
            \item (2)において,赤色の波長を$\lambda_\mathrm{R}=6.8\times 10^{-7}$\sftanni{m}とする。格子定数を求めよ。
        \end{enumerate}
    \end{mawarikomi}
%  \vfill
% %  \newpage
%  \hakosyokika
\item
    \begin{mawarikomi}[10](140pt,0){260pt}{%WinTpicVersion4.32a
{\unitlength 0.1in%
\begin{picture}(14.6161,12.9921)(40.3543,-16.5354)%
% LINE 2 0 3 0 Black White  
% 8 4100 360 4100 690 4100 690 5585 1680 5585 1680 5585 1350 5585 1350 4100 360
% 
\special{pn 8}%
\special{pa 4035 354}%
\special{pa 4035 679}%
\special{fp}%
\special{pa 4035 679}%
\special{pa 5497 1654}%
\special{fp}%
\special{pa 5497 1654}%
\special{pa 5497 1329}%
\special{fp}%
\special{pa 5497 1329}%
\special{pa 4035 354}%
\special{fp}%
% DOT 0 0 3 1 Black White  
% 1 4842 1016
% 
\special{pn 4}%
\special{sh 1}%
\special{ar 4766 1000 16 16 0 6.2831853}%
% DOT 0 0 3 2 Black White  
% 1 5007 1127
% 
\special{pn 4}%
\special{sh 1}%
\special{ar 4928 1109 16 16 0 6.2831853}%
% DOT 0 0 3 3 Black White  
% 1 5172 1234
% 
\special{pn 4}%
\special{sh 1}%
\special{ar 5091 1215 16 16 0 6.2831853}%
% DOT 0 0 3 4 Black White  
% 1 5337 1346
% 
\special{pn 4}%
\special{sh 1}%
\special{ar 5253 1325 16 16 0 6.2831853}%
% DOT 0 0 3 5 Black White  
% 1 5502 1453
% 
\special{pn 4}%
\special{sh 1}%
\special{ar 5415 1430 16 16 0 6.2831853}%
% DOT 0 0 3 6 Black White  
% 1 4182 583
% 
\special{pn 4}%
\special{sh 1}%
\special{ar 4116 574 16 16 0 6.2831853}%
% DOT 0 0 3 7 Black White  
% 1 4347 690
% 
\special{pn 4}%
\special{sh 1}%
\special{ar 4279 679 16 16 0 6.2831853}%
% DOT 0 0 3 8 Black White  
% 1 4512 801
% 
\special{pn 4}%
\special{sh 1}%
\special{ar 4441 788 16 16 0 6.2831853}%
% DOT 0 0 3 9 Black White  
% 1 4677 909
% 
\special{pn 4}%
\special{sh 1}%
\special{ar 4603 895 16 16 0 6.2831853}%
% LINE 1 0 3 10 Black White  
% 2 4842 1020 2801 1762
% 
\special{pn 13}%
\special{pa 4766 1004}%
\special{pa 2757 1734}%
\special{fp}%
% VECTOR 1 0 3 11 Black White  
% 2 2801 1762 3956 1342
% 
\special{pn 13}%
\special{pa 2757 1734}%
\special{pa 3894 1321}%
\special{fp}%
\special{sh 1}%
\special{pa 3894 1321}%
\special{pa 3826 1325}%
\special{pa 3844 1339}%
\special{pa 3839 1362}%
\special{pa 3894 1321}%
\special{fp}%
% POLYGON 2 0 3 12 Black White  
% 6 3296 1457 3296 1585 3378 1668 3378 1540 3378 1540 3296 1457
% 
\special{pn 8}%
\special{pa 3244 1434}%
\special{pa 3244 1560}%
\special{pa 3325 1642}%
\special{pa 3325 1516}%
\special{pa 3244 1434}%
\special{pa 3244 1560}%
\special{fp}%
% LINE 3 0 3 13 Black White  
% 10 3308 1597 3308 1470 3325 1614 3325 1486 3341 1630 3341 1503 3358 1647 3358 1519 3374 1663 3374 1536
% 
\special{pn 4}%
\special{pa 3256 1572}%
\special{pa 3256 1447}%
\special{fp}%
\special{pa 3273 1589}%
\special{pa 3273 1463}%
\special{fp}%
\special{pa 3288 1604}%
\special{pa 3288 1479}%
\special{fp}%
\special{pa 3305 1621}%
\special{pa 3305 1495}%
\special{fp}%
\special{pa 3321 1637}%
\special{pa 3321 1512}%
\special{fp}%
% LINE 2 1 3 14 Black White  
% 2 3337 1647 3337 1894
% 
\special{pn 8}%
\special{pa 3284 1621}%
\special{pa 3284 1864}%
\special{da 0.030}%
% LINE 2 1 3 15 Black White  
% 2 4842 739 4842 1317
% 
\special{pn 8}%
\special{pa 4766 727}%
\special{pa 4766 1296}%
\special{da 0.030}%
% LINE 2 1 3 16 Black White  
% 2 5337 1069 5337 1346
% 
\special{pn 8}%
\special{pa 5253 1052}%
\special{pa 5253 1325}%
\special{da 0.030}%
% VECTOR 2 0 3 17 Black White  
% 2 3935 1548 3337 1767
% 
\special{pn 8}%
\special{pa 3873 1524}%
\special{pa 3284 1739}%
\special{fp}%
\special{sh 1}%
\special{pa 3284 1739}%
\special{pa 3352 1735}%
\special{pa 3334 1721}%
\special{pa 3340 1698}%
\special{pa 3284 1739}%
\special{fp}%
% VECTOR 2 0 3 18 Black White  
% 2 3935 1552 4842 1222
% 
\special{pn 8}%
\special{pa 3873 1528}%
\special{pa 4766 1203}%
\special{fp}%
\special{sh 1}%
\special{pa 4766 1203}%
\special{pa 4698 1207}%
\special{pa 4717 1220}%
\special{pa 4711 1244}%
\special{pa 4766 1203}%
\special{fp}%
% STR 2 0 3 19 Black White  
% 4 3939 1507 3939 1548 5 0 1 0
% $L$
\put(38.7697,-15.2362){\makebox(0,0){{\colorbox[named]{White}{$L$}}}}%
% STR 2 0 3 20 Black White  
% 4 3168 1408 3168 1449 2 0 0 0
% 回折格子
\put(31.1811,-14.2618){\makebox(0,0)[lb]{回折格子}}%
% ELLIPSE 2 0 3 21 Black White  
% 4 2450 1894 2487 1812 2487 1812 2487 1812
% 
\special{pn 8}%
\special{ar 2411 1864 36 81 0.0000000 6.2831853}%
% LINE 2 0 3 22 Black White  
% 2 2458 1975 2762 1865
% 
\special{pn 8}%
\special{pa 2419 1944}%
\special{pa 2719 1836}%
\special{fp}%
% LINE 2 0 3 23 Black White  
% 2 2453 1812 2757 1701
% 
\special{pn 8}%
\special{pa 2414 1783}%
\special{pa 2714 1674}%
\special{fp}%
% ELLIPSE 2 0 3 24 Black White  
% 4 2760 1782 2797 1700 2764 1898 2764 1667
% 
\special{pn 8}%
\special{ar 2717 1754 36 81 4.7891609 1.4940244}%
% STR 2 0 3 25 Black White  
% 4 2112 1441 2112 1482 2 0 0 0
% レーザー光源
\put(20.7874,-14.5866){\makebox(0,0)[lb]{レーザー光源}}%
% STR 2 0 3 26 Black White  
% 4 2112 1555 2112 1596 2 0 0 0
% または白色光源
\put(20.7874,-15.7087){\makebox(0,0)[lb]{または白色光源}}%
% VECTOR 2 0 3 27 Black White  
% 4 5082 933 4842 777 5090 942 5337 1107
% 
\special{pn 8}%
\special{pa 5002 918}%
\special{pa 4766 765}%
\special{fp}%
\special{sh 1}%
\special{pa 4766 765}%
\special{pa 4810 817}%
\special{pa 4810 793}%
\special{pa 4832 784}%
\special{pa 4766 765}%
\special{fp}%
\special{pa 5010 927}%
\special{pa 5253 1090}%
\special{fp}%
\special{sh 1}%
\special{pa 5253 1090}%
\special{pa 5210 1036}%
\special{pa 5210 1060}%
\special{pa 5187 1070}%
\special{pa 5253 1090}%
\special{fp}%
% STR 2 0 3 28 Black White  
% 4 5123 843 5123 884 5 0 1 0
% $x_m$
\put(50.4232,-8.7008){\makebox(0,0){{\colorbox[named]{White}{$x_m$}}}}%
\end{picture}}%
}
    図のように,格子定数$d$\tanni{m}の回折格子に光源から光を垂直に入射し,回折格子から距離$L$だけ離れているところにスクリーンを配置して以下の実験を行った。
        \begin{enumerate}
            \item 光源として波長$\lambda $\tanni{m}のレーザー光を用いたところ,スクリーン上に明るい点の列が観測された。中心の明るい点から測って,$m$番目の明るい点までの距離を$x_m$\tanni{m}とし,$x_m$が$L$\tanni{m}に比べて十分小さいとした場合,$x_m$を$\lambda $,$d$,$L$,$m$を用いて表せ。ただし,中心の明るい点を$m=0$とし,微小角$\theta $に対して$\sin{\theta }\kinzi \tan{\theta }$の近似を用いてよい。また,このとき明るい点の間隔$\varDelta x$\tanni{m}を求めよ。
            \item 前問において,回折格子のすじが$1$\sftanni{mm}あたり100本あり,$L$が1.00\sftanni{m}のとき,$m=3$の明るい点までの距離$x_3$が19.0\sftanni{cm}と測定された。レーザー光の波長$\lambda $\tanni{nm}を求めよ。
            \item 光源としてレーザー光のかわりに可視光領域の白色光を用いると,スクリーン上ではどのような像が見られるか。$m=0$と$m=1$の明るい点について簡単に説明せよ。
            \item 可視光の波長範囲は$380$\sftanni{nm}~$770$\sftanni{nm}である。$m=1$のときの$x_1$の広がる範囲\tanni{cm}を求めよ。
        \end{enumerate}
    \end{mawarikomi}
%  \vfill
%  \newpage
%  \hakosyokika
\item 油膜が水面に広がっていて,空気中での波長が$6.0\times 10^{-7}$\sftanni{m}の光がこの油膜へ垂直に入射している。空気,水,および油膜の屈折率はそれぞれ1.0,1.3,1.5とし,空気中の光速を$3.0\times 10^8$\sftanni{m/s}とする。
    \begin{enumerate}
            \item 油膜中での光の速さと波長はいくらか。
            \item 油膜の表面と裏面で反射した光が干渉によって強めあう膜の最小の厚さはいくらか。
            \item 油膜の厚さを前問から厚くしていった場合,次に強めあう膜の厚さはいくらか。
            \item 波長$6.0\times 10^{-7}$\sftanni{m}の光では強めあい,波長$4.5\times 10^{-7}$\sftanni{m}の光では弱めあう膜の最小の厚さはいくらか。
    \end{enumerate}

%  \vfill
%  \hakosyokika
\item
    \begin{mawarikomi}{170pt}{%WinTpicVersion4.32a
{\unitlength 0.1in%
\begin{picture}(21.6800,14.0000)(0.3200,-16.0000)%
% POLYGON 2 5 0 0 Black White  
% 5 400 1400 400 1600 2200 1600 2200 1400 400 1400
% 
\special{pn 0}%
\special{sh 0.400}%
\special{pa 400 1400}%
\special{pa 400 1600}%
\special{pa 2200 1600}%
\special{pa 2200 1400}%
\special{pa 400 1400}%
\special{ip}%
\special{pn 8}%
\special{pa 400 1400}%
\special{pa 400 1600}%
\special{pa 2200 1600}%
\special{pa 2200 1400}%
\special{pa 400 1400}%
\special{ip}%
% POLYGON 2 5 1 0 Black White  
% 6 400 1400 400 800 2200 800 2200 1400 2200 1400 400 1400
% 
\special{pn 0}%
\special{sh 0.200}%
\special{pa 400 1400}%
\special{pa 400 800}%
\special{pa 2200 800}%
\special{pa 2200 1400}%
\special{pa 2200 1400}%
\special{pa 400 1400}%
\special{ip}%
\special{pn 8}%
\special{pa 400 1400}%
\special{pa 400 800}%
\special{pa 2200 800}%
\special{pa 2200 1400}%
\special{pa 400 1400}%
\special{ip}%
% LINE 2 0 3 0 Black White  
% 4 400 800 2200 800 2200 1400 400 1400
% 
\special{pn 8}%
\special{pa 400 800}%
\special{pa 2200 800}%
\special{fp}%
\special{pa 2200 1400}%
\special{pa 400 1400}%
\special{fp}%
% LINE 2 0 3 0 Black White  
% 2 400 200 1000 800
% 
\special{pn 8}%
\special{pa 400 200}%
\special{pa 1000 800}%
\special{fp}%
% LINE 2 0 3 0 Black White  
% 2 1000 200 1600 800
% 
\special{pn 8}%
\special{pa 1000 200}%
\special{pa 1600 800}%
\special{fp}%
% VECTOR 2 0 3 0 Black White  
% 2 1600 800 2000 400
% 
\special{pn 8}%
\special{pa 1600 800}%
\special{pa 2000 400}%
\special{fp}%
\special{sh 1}%
\special{pa 2000 400}%
\special{pa 1939 433}%
\special{pa 1962 438}%
\special{pa 1967 461}%
\special{pa 2000 400}%
\special{fp}%
% LINE 2 0 3 0 Black White  
% 4 1000 800 1300 1400 1300 1400 1600 800
% 
\special{pn 8}%
\special{pa 1000 800}%
\special{pa 1300 1400}%
\special{fp}%
\special{pa 1300 1400}%
\special{pa 1600 800}%
\special{fp}%
% VECTOR 2 0 3 0 Black White  
% 2 600 400 640 440
% 
\special{pn 8}%
\special{pa 600 400}%
\special{pa 640 440}%
\special{fp}%
\special{sh 1}%
\special{pa 640 440}%
\special{pa 607 379}%
\special{pa 602 402}%
\special{pa 579 407}%
\special{pa 640 440}%
\special{fp}%
% VECTOR 2 0 3 0 Black White  
% 2 1120 320 1160 360
% 
\special{pn 8}%
\special{pa 1120 320}%
\special{pa 1160 360}%
\special{fp}%
\special{sh 1}%
\special{pa 1160 360}%
\special{pa 1127 299}%
\special{pa 1122 322}%
\special{pa 1099 327}%
\special{pa 1160 360}%
\special{fp}%
% LINE 2 1 3 0 Black White  
% 2 1000 800 1280 480
% 
\special{pn 8}%
\special{pa 1000 800}%
\special{pa 1280 480}%
\special{da 0.015}%
% LINE 2 1 3 0 Black White  
% 2 1600 800 1120 1040
% 
\special{pn 8}%
\special{pa 1600 800}%
\special{pa 1120 1040}%
\special{da 0.015}%
% LINE 2 0 3 0 Black White  
% 4 1192 1004 1156 933 1156 933 1084 968
% 
\special{pn 8}%
\special{pa 1192 1004}%
\special{pa 1156 933}%
\special{fp}%
\special{pa 1156 933}%
\special{pa 1084 968}%
\special{fp}%
% LINE 2 0 3 0 Black White  
% 4 1221 551 1283 601 1283 601 1335 540
% 
\special{pn 8}%
\special{pa 1221 551}%
\special{pa 1283 601}%
\special{fp}%
\special{pa 1283 601}%
\special{pa 1335 540}%
\special{fp}%
% LINE 2 1 3 0 Black White  
% 2 1000 396 1000 1236
% 
\special{pn 8}%
\special{pa 1000 396}%
\special{pa 1000 1236}%
\special{da 0.015}%
% CIRCLE 2 0 3 0 Black White  
% 4 1000 796 1000 1036 1000 1396 1300 1396
% 
\special{pn 8}%
\special{ar 1000 796 240 240 1.1071487 1.5707963}%
% STR 2 0 3 0 Black White  
% 4 1062 1082 1062 1102 5 0 0 0
% $\phi$
\put(10.6200,-11.0200){\makebox(0,0){$\phi$}}%
% STR 2 0 3 0 Black White  
% 4 1280 460 1280 480 2 0 0 0
% $\mathrm{A_2}$
\put(12.8000,-4.8000){\makebox(0,0)[lb]{$\mathrm{A_2}$}}%
% STR 2 0 3 0 Black White  
% 4 1000 806 1000 826 4 0 0 0
% $\mathrm{A_1}$
\put(10.0000,-8.2600){\makebox(0,0)[rt]{$\mathrm{A_1}$}}%
% STR 2 0 3 0 Black White  
% 4 1300 1460 1300 1480 5 0 0 0
% $\mathrm{C}$
\put(13.0000,-14.8000){\makebox(0,0){$\mathrm{C}$}}%
% STR 2 0 3 0 Black White  
% 4 1626 808 1626 828 1 0 0 0
% $\mathrm{B_2}$
\put(16.2600,-8.2800){\makebox(0,0)[lt]{$\mathrm{B_2}$}}%
% STR 2 0 3 0 Black White  
% 4 1188 1032 1188 1052 1 0 0 0
% $\mathrm{B_2}$
\put(11.8800,-10.5200){\makebox(0,0)[lt]{$\mathrm{B_2}$}}%
% STR 2 0 3 0 Black White  
% 4 2014 372 2014 392 1 0 0 0
% D
\put(20.1400,-3.9200){\makebox(0,0)[lt]{D}}%
% STR 2 0 3 0 Black White  
% 4 1894 746 1894 766 2 0 0 0
% 空気
\put(18.9400,-7.6600){\makebox(0,0)[lb]{空気}}%
% STR 2 0 3 0 Black White  
% 4 1894 1186 1894 1206 2 0 0 0
% 薄膜
\put(18.9400,-12.0600){\makebox(0,0)[lb]{薄膜}}%
% STR 2 0 3 0 Black White  
% 4 1734 1546 1734 1566 2 0 0 0
% 媒質G
\put(17.3400,-15.6600){\makebox(0,0)[lb]{媒質G}}%
% VECTOR 2 0 3 0 Black White  
% 4 1122 1040 1162 1120 1442 1120 1482 1040
% 
\special{pn 8}%
\special{pa 1122 1040}%
\special{pa 1162 1120}%
\special{fp}%
\special{sh 1}%
\special{pa 1162 1120}%
\special{pa 1150 1051}%
\special{pa 1138 1072}%
\special{pa 1114 1069}%
\special{pa 1162 1120}%
\special{fp}%
\special{pa 1442 1120}%
\special{pa 1482 1040}%
\special{fp}%
\special{sh 1}%
\special{pa 1482 1040}%
\special{pa 1434 1091}%
\special{pa 1458 1088}%
\special{pa 1470 1109}%
\special{pa 1482 1040}%
\special{fp}%
\end{picture}}%
}
        媒質Gの上に厚さ$d$の薄膜があり,空気中から単色光が斜めに入射する場合の干渉を考える。空気中での光の波長を$\lambda $,屈折角を$\phi $とする。また,空気,薄膜,Gの屈折率はそれぞれ1,$n$,
        $n\mathrm{_G}$,とする。$\mathrm{A_1A_2}$は入射波の波面で,$\mathrm{B_1B_2}$は屈折波の波面である。同じ波面上では同位相である。したがって,$\mathrm{A_1}\rightarrow\mathrm{C}\rightarrow \mathrm{B_2}\rightarrow \mathrm{D}$の経路をとる光と,$\mathrm{A_2}\rightarrow \mathrm{B_2}\rightarrow \mathrm{D}$の経路をとる光との間に位相差をもたらす経路の差は,$\mathrm{B_1C+CB_2}$になる。この長さは$d$,$\phi$を用いて表すと\Hako となる。これらの2つの光が点$\mathrm{B_2}$で同位相であれば干渉により強め合い,Dの方向から観測すると反射光は明るく見える。薄膜の中では光の波長は\Hako である。$\mathrm{n<n_G}$の場合,各反射面での反射光の位相のずれの有無を考慮すると,干渉して反射光が明るくなる条件は,正の整数$m$を用いて\Hako と書ける。$\lambda = 6.0\times 10^{-7}$\tanni{m},$\phi = 60\Deg $,$n=1.5$,$n\mathrm{_G}=1.6$のとき,反射光が明るくなる薄膜の最小の厚さは\Hako \tanni{m}である。またGのみを替え,$n\mathrm{_G}=1.4$とすると,明るくなる最小の厚さは\Hako \tanni{m}となる。
    \end{mawarikomi}
%  \vfill
 \hakosyokika
\item
    \begin{mawarikomi}(20pt,0pt){160pt}{%WinTpicVersion4.32a
{\unitlength 0.1in%
\begin{picture}(21.8500,12.0000)(3.2000,-14.0000)%
% BOX 2 0 3 0 Black White  
% 2 400 1000 2000 1200
% 
\special{pn 8}%
\special{pa 400 1000}%
\special{pa 2000 1000}%
\special{pa 2000 1200}%
\special{pa 400 1200}%
\special{pa 400 1000}%
\special{pa 2000 1000}%
\special{fp}%
% BOX 2 0 0 0 Black Black  
% 2 2000 1000 2400 800
% 
\special{pn 0}%
\special{sh 0.400}%
\special{pa 2000 1000}%
\special{pa 2400 1000}%
\special{pa 2400 800}%
\special{pa 2000 800}%
\special{pa 2000 1000}%
\special{ip}%
\special{pn 8}%
\special{pa 2000 1000}%
\special{pa 2400 1000}%
\special{pa 2400 800}%
\special{pa 2000 800}%
\special{pa 2000 1000}%
\special{pa 2400 1000}%
\special{fp}%
% POLYGON 2 0 3 0 Black White  
% 5 376 801 400 1000 2018 801 1994 603 376 801
% 
\special{pn 8}%
\special{pa 376 801}%
\special{pa 400 1000}%
\special{pa 2018 801}%
\special{pa 1994 603}%
\special{pa 376 801}%
\special{pa 400 1000}%
\special{fp}%
% LINE 2 1 3 0 Black White  
% 4 400 1200 400 1400 2000 1400 2000 1200
% 
\special{pn 8}%
\special{pa 400 1200}%
\special{pa 400 1400}%
\special{da 0.015}%
\special{pa 2000 1400}%
\special{pa 2000 1200}%
\special{da 0.015}%
% VECTOR 2 0 3 0 Black White  
% 4 1200 1300 400 1300 1200 1300 2000 1300
% 
\special{pn 8}%
\special{pa 1200 1300}%
\special{pa 400 1300}%
\special{fp}%
\special{sh 1}%
\special{pa 400 1300}%
\special{pa 467 1320}%
\special{pa 453 1300}%
\special{pa 467 1280}%
\special{pa 400 1300}%
\special{fp}%
\special{pa 1200 1300}%
\special{pa 2000 1300}%
\special{fp}%
\special{sh 1}%
\special{pa 2000 1300}%
\special{pa 1933 1280}%
\special{pa 1947 1300}%
\special{pa 1933 1320}%
\special{pa 2000 1300}%
\special{fp}%
% STR 2 0 3 0 Black White  
% 4 1200 1200 1200 1300 5 0 1 0
% $L$
\put(12.0000,-13.0000){\makebox(0,0){{\colorbox[named]{White}{$L$}}}}%
% STR 2 0 3 0 Black White  
% 4 1200 700 1200 800 5 0 0 0
% A
\put(12.0000,-8.0000){\makebox(0,0){A}}%
% STR 2 0 3 0 Black White  
% 4 1200 1000 1200 1100 5 0 0 0
% B
\put(12.0000,-11.0000){\makebox(0,0){B}}%
% VECTOR 2 0 3 0 Black White  
% 2 2470 900 2470 800
% 
\special{pn 8}%
\special{pa 2470 900}%
\special{pa 2470 800}%
\special{fp}%
\special{sh 1}%
\special{pa 2470 800}%
\special{pa 2450 867}%
\special{pa 2470 853}%
\special{pa 2490 867}%
\special{pa 2470 800}%
\special{fp}%
% VECTOR 2 0 3 0 Black White  
% 2 2470 900 2470 1000
% 
\special{pn 8}%
\special{pa 2470 900}%
\special{pa 2470 1000}%
\special{fp}%
\special{sh 1}%
\special{pa 2470 1000}%
\special{pa 2490 933}%
\special{pa 2470 947}%
\special{pa 2450 933}%
\special{pa 2470 1000}%
\special{fp}%
% STR 2 0 3 0 Black White  
% 4 2600 800 2600 900 5 0 0 0
% $D$
\put(26.0000,-9.0000){\makebox(0,0){$D$}}%
% STR 2 0 3 0 Black White  
% 4 2080 660 2080 760 2 0 0 0
% アルミ箔
\put(20.8000,-7.6000){\makebox(0,0)[lb]{アルミ箔}}%
% VECTOR 2 0 3 0 Black White  
% 6 610 200 610 600 1210 200 1210 600 1810 200 1810 600
% 
\special{pn 8}%
\special{pa 610 200}%
\special{pa 610 600}%
\special{fp}%
\special{sh 1}%
\special{pa 610 600}%
\special{pa 630 533}%
\special{pa 610 547}%
\special{pa 590 533}%
\special{pa 610 600}%
\special{fp}%
\special{pa 1210 200}%
\special{pa 1210 600}%
\special{fp}%
\special{sh 1}%
\special{pa 1210 600}%
\special{pa 1230 533}%
\special{pa 1210 547}%
\special{pa 1190 533}%
\special{pa 1210 600}%
\special{fp}%
\special{pa 1810 200}%
\special{pa 1810 600}%
\special{fp}%
\special{sh 1}%
\special{pa 1810 600}%
\special{pa 1830 533}%
\special{pa 1810 547}%
\special{pa 1790 533}%
\special{pa 1810 600}%
\special{fp}%
% STR 2 0 3 0 Black White  
% 4 410 300 410 400 5 0 0 0
% 光
\put(4.1000,-4.0000){\makebox(0,0){光}}%
% STR 2 0 3 0 Black White  
% 4 1060 300 1060 400 5 0 0 0
% $\lambda $
\put(10.6000,-4.0000){\makebox(0,0){$\lambda $}}%
% STR 2 0 3 0 Black White  
% 4 400 900 400 1000 4 0 0 0
% O
\put(4.0000,-10.0000){\makebox(0,0)[rt]{O}}%
\end{picture}}%
}
    2枚の平板ガラスA,Bの一端Oから$L=0.10$\sftanni{m}離れたところにアルミ箔をはさむ。
    真上から波長$\lambda = 5.9\times 10^{-7}$\sftanni{m}の光を当てて,上から見ると干渉縞が見えた。空気の屈折率を1とする。
        \begin{enumerate}
            \item O点の縞は明線になるか,暗線になるか。それともそのいずれでもないかを答えよ。
            \item 隣り合う明線の間隔$\varDelta x$が$\varDelta x=2.0$\sftanni{mm}のとき,はさんだアルミ箔の厚さ$D$\tanni{m}を求めよ。
            \item 光の方向と反対側(ガラス板B側)から干渉縞を観察する。上から見る場合と比べて,干渉縞はどう変わるか,簡潔に述べよ。
            \item 2枚のガラス板の間を屈折率$n$の水で満たす。空気中と比べて明線の間隔は何倍になるか。
        \end{enumerate}
    \end{mawarikomi}
 \vfill
 \hakosyokika
\item
    \begin{mawarikomi}(20pt,0pt){160pt}{%WinTpicVersion4.32a
{\unitlength 0.1in%
\begin{picture}(22.0000,19.5500)(6.0000,-26.0000)%
% CIRCLE 2 0 3 0 Black White  
% 4 1600 800 1600 2200 600 1800 2600 1800
% 
\special{pn 8}%
\special{ar 1600 800 1400 1400 0.7853982 2.3561945}%
% LINE 2 1 3 0 Black White  
% 4 1600 800 1600 2200 1600 800 2310 2000
% 
\special{pn 8}%
\special{pa 1600 800}%
\special{pa 1600 2200}%
\special{da 0.015}%
\special{pa 1600 800}%
\special{pa 2310 2000}%
\special{da 0.015}%
% STR 2 0 3 0 Black White  
% 4 1600 610 1600 710 5 0 0 0
% O
\put(16.0000,-7.1000){\makebox(0,0){O}}%
% STR 2 0 3 0 Black White  
% 4 2000 1120 2000 1220 5 0 0 0
% $R$
\put(20.0000,-12.2000){\makebox(0,0){$R$}}%
% STR 2 0 3 0 Black White  
% 4 1600 2210 1600 2310 5 0 0 0
% C
\put(16.0000,-23.1000){\makebox(0,0){C}}%
% BOX 2 0 3 0 Black White  
% 2 600 2200 2600 2400
% 
\special{pn 8}%
\special{pa 600 2200}%
\special{pa 2600 2200}%
\special{pa 2600 2400}%
\special{pa 600 2400}%
\special{pa 600 2200}%
\special{pa 2600 2200}%
\special{fp}%
% LINE 2 0 3 0 Black White  
% 6 600 1800 600 1700 600 1700 2600 1700 2600 1700 2600 1800
% 
\special{pn 8}%
\special{pa 600 1800}%
\special{pa 600 1700}%
\special{fp}%
\special{pa 600 1700}%
\special{pa 2600 1700}%
\special{fp}%
\special{pa 2600 1700}%
\special{pa 2600 1800}%
\special{fp}%
% LINE 2 2 3 0 Black White  
% 4 1600 2000 2800 2000 2800 2200 1600 2200
% 
\special{pn 8}%
\special{pa 1600 2000}%
\special{pa 2800 2000}%
\special{dt 0.025}%
\special{pa 2800 2200}%
\special{pa 1600 2200}%
\special{dt 0.025}%
% STR 2 0 3 0 Black White  
% 4 2700 2000 2700 2100 5 0 0 0
% $d$
\put(27.0000,-21.0000){\makebox(0,0){$d$}}%
% VECTOR 2 0 3 0 Black White  
% 4 2700 1800 2700 2000 2700 2400 2700 2200
% 
\special{pn 8}%
\special{pa 2700 1800}%
\special{pa 2700 2000}%
\special{fp}%
\special{sh 1}%
\special{pa 2700 2000}%
\special{pa 2720 1933}%
\special{pa 2700 1947}%
\special{pa 2680 1933}%
\special{pa 2700 2000}%
\special{fp}%
\special{pa 2700 2400}%
\special{pa 2700 2200}%
\special{fp}%
\special{sh 1}%
\special{pa 2700 2200}%
\special{pa 2680 2267}%
\special{pa 2700 2253}%
\special{pa 2720 2267}%
\special{pa 2700 2200}%
\special{fp}%
% LINE 2 2 3 0 Black White  
% 2 2300 2000 2300 2600
% 
\special{pn 8}%
\special{pa 2300 2000}%
\special{pa 2300 2600}%
\special{dt 0.025}%
% LINE 2 2 3 0 Black White  
% 2 1600 2400 1600 2600
% 
\special{pn 8}%
\special{pa 1600 2400}%
\special{pa 1600 2600}%
\special{dt 0.025}%
% VECTOR 2 0 3 0 Black White  
% 4 2000 2500 1600 2500 2000 2500 2300 2500
% 
\special{pn 8}%
\special{pa 2000 2500}%
\special{pa 1600 2500}%
\special{fp}%
\special{sh 1}%
\special{pa 1600 2500}%
\special{pa 1667 2520}%
\special{pa 1653 2500}%
\special{pa 1667 2480}%
\special{pa 1600 2500}%
\special{fp}%
\special{pa 2000 2500}%
\special{pa 2300 2500}%
\special{fp}%
\special{sh 1}%
\special{pa 2300 2500}%
\special{pa 2233 2480}%
\special{pa 2247 2500}%
\special{pa 2233 2520}%
\special{pa 2300 2500}%
\special{fp}%
% STR 2 0 3 0 Black White  
% 4 1970 2400 1970 2500 5 0 1 0
% $r$
\put(19.7000,-25.0000){\makebox(0,0){{\colorbox[named]{White}{$r$}}}}%
\end{picture}}%
}
    平面ガラスの板の上に,大きい曲率半径$R$をもつ平凸レンズをのせ,上から波長$\lambda $の単色光をあてて上から見ると,レンズとガラス板の接点Cを中心とする明暗の輪が同心円状に並んでいるのが見える(ニュートンリング)。
        \begin{Enumerate}
            \item 輪の半径を$r$とする。その位置での空気層の厚さ$d$を$R$,$r$を用いて表せ。ただし,$d$は$R$に比べて十分小さいとする。
            \item 平凸レンズの中心部は明るく見えるか,暗く見えるか。また,青色の光と赤色の光では,輪の半径はどちらが大きいか。
        \end{Enumerate}
    $\lambda = 540$\sftanni{nm}の光を用いたところ,中心から3番目の明輪が$r=3.0$\sftanni{mm}の位置に見えた。
        \begin{Enumerate*}
            \item 平凸レンズの曲率半径$R$\sftanni{m}を求めよ。
            \item 平凸レンズと平面ガラスの間に,ある液体を満たして,今度はガラスの下から単色光をあててレンズの上から見るとする。この場合,ニュートンリングはどのように見えるか簡潔に述べよ。
        \end{Enumerate*}
    \end{mawarikomi}

 \vfill
\end{enumerate}
\end{document}