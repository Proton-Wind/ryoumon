\hakosyokika
\item
    \begin{mawarikomi}(20pt,0){120pt}{{\unitlength5mm\small
% \Drawaxisfalse
\begin{zahyou}[yokozikukigou={$V$}
    ,tatezikukigou={$P$}
    ,gentenkigou={0}](0,10)(0,10)
\small
\def\A{(4,4)}
\def\B{(4,8)}
\def\C{(8,4)}
% \kuromaru{\A;\B;\C;\D;\E}
\Put\A[sw]{A}
\Put\B[n]{B}
\Put\C[se]{C}
\Put\A[syaei=xy,xlabel=V_1,ylabel=p_1]{}
\Put\B[syaei=y,ylabel=2p_1]{}
\Put\C[syaei=x,xlabel=2V_1]{}
% \put(2.5,\YB){\yasen(-0.2,0)}
% \put(2,\BCY){\yasen(0.2,-1.3)}
% \put(5,\YC){\yasen(0.2,0)}
% \calcval{(\YC+\YE)*0.5}\CDY
% \put(\XD,\CDY){\yasen(0,0.5)}
% \put(4.9,\AEY){\yasen(-0.2,0.8)}
{\thicklines
\Drawlines{\A\B\C\A}
\changeArrowHeadSize<0.333>{2}
\put(4,6){\yasen(0,0.1)}
\put(6,6){\yasen(0.1,-0.1)}
\put(6,4){\yasen(-0.1,0)}
}
\end{zahyou}}
}
        単原子分子理想気体をなめらかに動くピストンのついたシリンダー内に閉じ込め,外部との熱のやりとりをすることにより,気体の圧力$p$と体積$V$を図のサイクルA→B→C→Aのように変化させる。気体定数を$R$とする。
        \begin{enumerate}
            \item Aにおける絶対温度を$T_1$とするとき,BおよびCにおける絶対温度をそれぞれ求めよ。
            \item A→BおよびC→Aの過程において,気体が吸収する熱量をそれぞれ求め,$p_1$,$V_1$を用いて表せ。
            \item B→Cの過程で気体がする仕事と,吸収する熱量を求め,$p_1$,$V_1$を用いて表せ。
            \item このサイクルを一巡する間に,気体がする仕事を求め,$p_1$,$V_1$を用いて表せ。
            \item このサイクルにおいて,絶対温度$T$(縦軸),と体積$V$(横軸)の関係を表すグラフの概形を描け。
        \end{enumerate}
    \end{mawarikomi}