\hakosyokika
\item
    \begin{mawarikomi}{50pt}{\begin{zahyou*}[ul=5mm](-2,2)(-5,0.5)
    \small
    \def\Fy{-0.2*(cos(T)+T/3.5)}
	\def\Fx{-0.3*sin(T)}
	\BGurafu\Fx\Fy{$pi}{20*$pi}
    \drawline(-1,0)(1,0)
    \def\A{(-1,0.5)}
    \def\B{(-1,0)}
    \def\C{(1,0)}
    \def\D{(1,0.5)}
    \def\P{(0,-4.15)}
    \def\Q{(0,-4.7)}
    \En*[0.5]\P{0.35}
    \En\P{0.35}
    \En*[0.5]\Q{0.2}
    \En\Q{0.2}
    \Put\P(-12pt,0)[l]{P}
    \Put\Q(-12pt,-2pt)[l]{Q}
    \Nuritubusi*{\A\B\C\D\A}
\end{zahyou*}
}
        軽いばねの下端に,質量$2m$の物体Pと質量$m$の物体Qを接合したものをつるすと,ばねは自然長から$a$だけ伸びてつり合った。重力加速度の大きさを$g$とする。
        \begin{enumerate}
            \item ばねのばね定数は\Hako であり,物体を上下させたときの周期は\Hako である。\\
            ~~つり合いの状態にあるとき,Qを静かに切り離すと,Pはもとのつり合いの位置から\Hako だけ上の位置を中心にして,振幅\Hako ,周期\Hako で振動する。また,振動の中心を通過するときの速さは\Hako である。
            \item 次にPだけつるしたばねをエレベーターに付けた場合を考える。\\
            ~~エレベーターが,上向きに大きさ$\alpha $の加速度で運動しているとき,エレベーター内で見ると,ばねが自然長から$a$だけ伸びてPは静止した。$\alpha $は\Hako である。また,Pを上下に振動させたときの周期は\hako{オ}の\Hako 倍である。
        \end{enumerate}
    \end{mawarikomi}