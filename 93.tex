\hakosyokika
\item
    \begin{mawarikomi}(10pt,0){250pt}{%WinTpicVersion4.32a
{\unitlength 0.1in%
\begin{picture}(35.9252,15.2461)(0.4921,-19.7047)%
% STR 2 0 3 0 Black White  
% 4 915 997 915 1035 2 0 0 0
% S$_1$
\put(9.0059,-10.1870){\makebox(0,0)[lb]{S$_1$}}%
% STR 2 0 3 0 Black White  
% 4 925 1413 925 1450 1 0 0 0
% S$_2$
\put(9.1043,-14.2717){\makebox(0,0)[lt]{S$_2$}}%
% LINE 0 0 3 0 Black White  
% 4 500 902 500 1202 500 1302 500 1602
% 
\special{pn 20}%
\special{pa 492 888}%
\special{pa 492 1183}%
\special{fp}%
\special{pa 492 1281}%
\special{pa 492 1577}%
\special{fp}%
% LINE 0 0 3 0 Black White  
% 6 900 602 900 1002 900 1102 900 1402 900 1502 900 2002
% 
\special{pn 20}%
\special{pa 886 593}%
\special{pa 886 986}%
\special{fp}%
\special{pa 886 1085}%
\special{pa 886 1380}%
\special{fp}%
\special{pa 886 1478}%
\special{pa 886 1970}%
\special{fp}%
% LINE 2 1 3 0 Black White  
% 2 300 1252 3700 1252
% 
\special{pn 8}%
\special{pa 295 1232}%
\special{pa 3642 1232}%
\special{da 0.030}%
% LINE 0 0 3 0 Black White  
% 2 3400 602 3400 2002
% 
\special{pn 20}%
\special{pa 3346 593}%
\special{pa 3346 1970}%
\special{fp}%
% LINE 2 1 3 0 Black White  
% 2 900 1052 700 1052
% 
\special{pn 8}%
\special{pa 886 1035}%
\special{pa 689 1035}%
\special{da 0.030}%
% LINE 2 1 3 0 Black White  
% 2 700 1450 900 1450
% 
\special{pn 8}%
\special{pa 689 1427}%
\special{pa 886 1427}%
\special{da 0.030}%
% VECTOR 2 0 3 0 Black White  
% 8 800 1150 800 1050 800 1150 800 1250 800 1350 800 1250 800 1350 800 1450
% 
\special{pn 8}%
\special{pa 787 1132}%
\special{pa 787 1033}%
\special{fp}%
\special{sh 1}%
\special{pa 787 1033}%
\special{pa 768 1099}%
\special{pa 787 1086}%
\special{pa 807 1099}%
\special{pa 787 1033}%
\special{fp}%
\special{pa 787 1132}%
\special{pa 787 1230}%
\special{fp}%
\special{sh 1}%
\special{pa 787 1230}%
\special{pa 807 1164}%
\special{pa 787 1178}%
\special{pa 768 1164}%
\special{pa 787 1230}%
\special{fp}%
\special{pa 787 1329}%
\special{pa 787 1230}%
\special{fp}%
\special{sh 1}%
\special{pa 787 1230}%
\special{pa 768 1296}%
\special{pa 787 1282}%
\special{pa 807 1296}%
\special{pa 787 1230}%
\special{fp}%
\special{pa 787 1329}%
\special{pa 787 1427}%
\special{fp}%
\special{sh 1}%
\special{pa 787 1427}%
\special{pa 807 1361}%
\special{pa 787 1375}%
\special{pa 768 1361}%
\special{pa 787 1427}%
\special{fp}%
% STR 2 0 3 0 Black White  
% 4 705 1100 705 1150 5 0 0 0
% $a$
\put(6.9390,-11.3189){\makebox(0,0){$a$}}%
% STR 2 0 3 0 Black White  
% 4 705 1300 705 1350 5 0 0 0
% $a$
\put(6.9390,-13.2874){\makebox(0,0){$a$}}%
% LINE 2 0 3 0 Black White  
% 4 900 1050 3400 850 900 1450 3400 850
% 
\special{pn 8}%
\special{pa 886 1033}%
\special{pa 3346 837}%
\special{fp}%
\special{pa 886 1427}%
\special{pa 3346 837}%
\special{fp}%
% LINE 2 1 3 0 Black White  
% 2 3400 850 3600 850
% 
\special{pn 8}%
\special{pa 3346 837}%
\special{pa 3543 837}%
\special{da 0.030}%
% VECTOR 2 0 3 0 Black White  
% 4 3500 1050 3500 850 3500 1050 3500 1250
% 
\special{pn 8}%
\special{pa 3445 1033}%
\special{pa 3445 837}%
\special{fp}%
\special{sh 1}%
\special{pa 3445 837}%
\special{pa 3425 903}%
\special{pa 3445 889}%
\special{pa 3465 903}%
\special{pa 3445 837}%
\special{fp}%
\special{pa 3445 1033}%
\special{pa 3445 1230}%
\special{fp}%
\special{sh 1}%
\special{pa 3445 1230}%
\special{pa 3465 1164}%
\special{pa 3445 1178}%
\special{pa 3425 1164}%
\special{pa 3445 1230}%
\special{fp}%
% STR 2 0 3 0 Black White  
% 4 3580 1000 3580 1050 5 0 0 0
% $x$
\put(35.2362,-10.3346){\makebox(0,0){$x$}}%
% STR 2 0 3 0 Black White  
% 4 3415 797 3415 835 2 0 1 0
% P
\put(33.6122,-8.2185){\makebox(0,0)[lb]{{\colorbox[named]{White}{P}}}}%
% STR 2 0 3 0 Black White  
% 4 3415 1238 3415 1275 1 0 1 0
% O
\put(33.6122,-12.5492){\makebox(0,0)[lt]{{\colorbox[named]{White}{O}}}}%
% VECTOR 2 0 3 0 Black White  
% 4 2000 1875 900 1875 2000 1875 3400 1875
% 
\special{pn 8}%
\special{pa 1969 1845}%
\special{pa 886 1845}%
\special{fp}%
\special{sh 1}%
\special{pa 886 1845}%
\special{pa 952 1865}%
\special{pa 938 1845}%
\special{pa 952 1826}%
\special{pa 886 1845}%
\special{fp}%
\special{pa 1969 1845}%
\special{pa 3346 1845}%
\special{fp}%
\special{sh 1}%
\special{pa 3346 1845}%
\special{pa 3281 1826}%
\special{pa 3294 1845}%
\special{pa 3281 1865}%
\special{pa 3346 1845}%
\special{fp}%
% STR 2 0 3 0 Black White  
% 4 2100 1825 2100 1875 5 0 1 0
% $\ell $
\put(20.6693,-18.4547){\makebox(0,0){{\colorbox[named]{White}{$\ell $}}}}%
% CIRCLE 2 0 3 0 Black White  
% 4 170 1250 225 1250 225 1250 225 1250
% 
\special{pn 8}%
\special{ar 167 1230 54 54 0.0000000 6.2831853}%
% LINE 2 0 3 0 Black White  
% 2 244 1250 269 1250
% 
\special{pn 8}%
\special{pa 240 1230}%
\special{pa 265 1230}%
\special{fp}%
% LINE 2 0 3 1 Black White  
% 2 72 1250 97 1250
% 
\special{pn 8}%
\special{pa 71 1230}%
\special{pa 95 1230}%
\special{fp}%
% LINE 2 0 3 0 Black White  
% 2 172 1178 172 1153
% 
\special{pn 8}%
\special{pa 169 1159}%
\special{pa 169 1135}%
\special{fp}%
% LINE 2 0 3 1 Black White  
% 2 172 1350 172 1325
% 
\special{pn 8}%
\special{pa 169 1329}%
\special{pa 169 1304}%
\special{fp}%
% LINE 2 0 3 0 Black White  
% 2 118 1198 101 1181
% 
\special{pn 8}%
\special{pa 116 1179}%
\special{pa 99 1162}%
\special{fp}%
% LINE 2 0 3 1 Black White  
% 2 240 1320 222 1302
% 
\special{pn 8}%
\special{pa 236 1299}%
\special{pa 219 1281}%
\special{fp}%
% LINE 2 0 3 0 Black White  
% 2 222 1198 239 1181
% 
\special{pn 8}%
\special{pa 219 1179}%
\special{pa 235 1162}%
\special{fp}%
% LINE 2 0 3 1 Black White  
% 2 100 1320 118 1302
% 
\special{pn 8}%
\special{pa 98 1299}%
\special{pa 116 1281}%
\special{fp}%
% STR 2 0 3 0 Black White  
% 4 170 1018 170 1038 5 0 0 0
% 光源
\put(1.6732,-10.2165){\makebox(0,0){光源}}%
% STR 2 0 3 0 Black White  
% 4 458 778 458 798 5 0 0 0
% スリット面I
\put(4.5079,-7.8543){\makebox(0,0){スリット面I}}%
% STR 2 0 3 0 Black White  
% 4 898 498 898 518 5 0 0 0
% スリット面II
\put(8.8386,-5.0984){\makebox(0,0){スリット面II}}%
% STR 2 0 3 0 Black White  
% 4 3404 498 3404 518 5 0 0 0
% スクリーン
\put(33.5039,-5.0984){\makebox(0,0){スクリーン}}%
% STR 2 0 3 0 Black White  
% 4 492 1218 492 1256 4 0 0 0
% S$_0$
\put(4.8425,-12.3622){\makebox(0,0)[rt]{S$_0$}}%
\end{picture}}%
}
        図で$\mathrm{S_0}$,$\mathrm{S_1}$,$\mathrm{S_2}$は互いに平行なスリットである。
        $\mathrm{S_1}$,$\mathrm{S_2}$は間隔が$2a$で,$\mathrm{S_0}$から等距離にある。
        スクリーンはスリット面IおよびIIに平行で,面IIから$\ell $だけ離してある。
        $\mathrm{S_1}$と$\mathrm{S_2}$の中点からスクリーンに下した垂線の足をOとし,Oから距離$x$だけ離れたスクリーン上の点をPとする。ここで,$a$および$x$は$\ell $に比べて十分小さい。光源から出た波長$\lambda $の単色光を$S_0$にあてると,スクリーン上に明暗のしまが現れる。
        \begin{description}
            \item[A] まず,空気(屈折率1)中に置かれた装置で実験する。
            \begin{Enumerate}
                \item P点が暗くなるとき,$x$が満たしている条件を整数$m$を用いて表せ。
                \item $a$を$0.47$\sftanni{mm},$\ell $を$6.1$\sftanni{m}にとって実験をした。このとき,スクリーン上に現れた暗線の間隔は$4.1$\sftanni{mm}であった。単色光の波長は何\sftanni{m}か。
            \end{Enumerate}
        \end{description}
        \begin{description}
            \item[B] 次に,装置の一部を屈折率$n$の媒質で満たす。
            \begin{Enumerate*}
                \item スリット面Iとスリット面IIの間だけを,この媒質で満たしたとき,暗線の間隔は,Aの場合の何倍になるか。
                \item スリットIIとスクリーンの間だけを,この媒質で満たしたとき,暗線の間隔は,Aの場合の何倍になるか。
            \end{Enumerate*} 
        \end{description}
\end{mawarikomi}