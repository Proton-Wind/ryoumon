\hakosyokika
\item
    \begin{mawarikomi}(10pt,0){250pt}{\input{./fig/fig093.tex}}
        図で$\mathrm{S_0}$,$\mathrm{S_1}$,$\mathrm{S_2}$は互いに平行なスリットである。
        $\mathrm{S_1}$,$\mathrm{S_2}$は間隔が$2a$で,$\mathrm{S_0}$から等距離にある。
        スクリーンはスリット面IおよびIIに平行で,面IIから$\ell $だけ離してある。
        $\mathrm{S_1}$と$\mathrm{S_2}$の中点からスクリーンに下した垂線の足をOとし,Oから距離$x$だけ離れたスクリーン上の点をPとする。ここで,$a$および$x$は$\ell $に比べて十分小さい。光源から出た波長$\lambda $の単色光を$S_0$にあてると,スクリーン上に明暗のしまが現れる。
        \begin{description}
            \item[A] まず,空気(屈折率1)中に置かれた装置で実験する。
            \begin{Enumerate}
                \item P点が暗くなるとき,$x$が満たしている条件を整数$m$を用いて表せ。
                \item $a$を$0.47$\sftanni{mm},$\ell $を$6.1$\sftanni{m}にとって実験をした。このとき,スクリーン上に現れた暗線の間隔は$4.1$\sftanni{mm}であった。単色光の波長は何\sftanni{m}か。
            \end{Enumerate}
        \end{description}
        \begin{description}
            \item[B] 次に,装置の一部を屈折率$n$の媒質で満たす。
            \begin{Enumerate*}
                \item スリット面Iとスリット面IIの間だけを,この媒質で満たしたとき,暗線の間隔は,Aの場合の何倍になるか。
                \item スリットIIとスクリーンの間だけを,この媒質で満たしたとき,暗線の間隔は,Aの場合の何倍になるか。
            \end{Enumerate*} 
        \end{description}
\end{mawarikomi}