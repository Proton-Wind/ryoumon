\hakosyokika
\item
    \begin{mawarikomi}(20pt,0){150pt}{{\unitlength6mm\small
% \Drawaxisfalse
\begin{zahyou}[%
    % yokozikuhaiti{[s]}
    % ,tatezikuhaiti{[e]}
    yokozikukigou={温度\tanni{K}}
    ,tatezikukigou={体積\tanni{m^3}}
    ,gentenkigou={0}
    ,hidariyohaku=0.5
    ,tatezikuhaiti={[e]}
    ,yokozikuhaiti={[s]}
    ](0,8)(-1,5.5)
\small
\def\O{(0,0)}
\def\A{(1,2)}
\def\B{(3,2)}
\def\C{(6,4)}
\def\D{(2,4)}
% \kuromaru{\A;\B;\C}
\Put\A[se]{A}
\Put\B[se]{B}
\Put\C[ne]{C}
\Put\D[n]{D}
\Put\A[syaei=xy,xlabel=T_0,ylabel=V_0]{}
\Put\B[syaei=x,xlabel=T_1]{}
\Put\C[syaei=x,xlabel=T_2]{}
% \put(2.5,\YB){\yasen(-0.2,0)}
% \put(2,\BCY){\yasen(0.2,-1.3)}
% \put(5,\YC){\yasen(0.2,0)}
% \calcval{(\YC+\YE)*0.5}\CDY
% \put(\XD,\CDY){\yasen(0,0.5)}
% \put(4.9,\AEY){\yasen(-0.2,0.8)}
\Dashline{0.1}{\O\A}
\Dashline{0.1}{\O\B}
{\thicklines
\Drawlines{\A\B\C\D\A}
\changeArrowHeadSize<0.333>{2}
\put(2,2){\yasen(0.1,0)}
\put(4.5,3){\yasen(0.3,0.2)}
\put(4,4){\yasen(-0.1,0)}
\put(4,4){\yasen(-0.1,0)}
\put(1.5,3){\yasen(-0.1,-0.2)}
}
\end{zahyou}}
}
        なめらかに動くピストンをもったシリンダーの中に1\sftanni{mol}の単原子分子理想気体が入れられている。図のように,気体の状態を温度$T_0$\tanni{K},体積$V_0$\tanni{m^3}の状態Aからゆっくり変化させて,A$\rightarrow$B,B$\rightarrow$C,C$\rightarrow$D,D$\rightarrow$Aの過程を経て状態Aにもどした。\\
        ~~A$\rightarrow$BおよびC$\rightarrow$Dの過程では体積が一定に保たれ,B$\rightarrow$CおよびD$\rightarrow$Aの過程では体積は温度に対して直線的に変化している。気体定数を$R$\tanni{J/(mol \cdot K)}とする。
        \begin{enumerate}
            \item A$\rightarrow$Bの過程で気体が吸収した熱量は何\sftanni{J}か。
            \item 状態Cでの気体の圧力は何\sftanni{N/m^2}か。
            \item D$\rightarrow$Aの過程で気体が外部へ放出した熱量は何\sftanni{J}か。
            \item A$\rightarrow$B$\rightarrow$C$\rightarrow$D$\rightarrow$Aの1サイクルしたとき,気体が外部へした仕事は何\sftanni{J}か。
            \item この1サイクルの熱効率$e$はいくらか。
        \end{enumerate}
    \end{mawarikomi}