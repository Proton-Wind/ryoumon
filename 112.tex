\hakosyokika
\item
    \begin{mawarikomi}(20pt,0pt){150pt}{
        %%% C:/vpn/vpn/KeTCindy/fig/fig112.tex 
%%% Generator=fig112.cdy 
{\unitlength=1cm%
\begin{picture}%
(5.09,4.64)(-2.09,-0.47)%
\special{pn 8}%
%
\special{pa  -591 -1378}\special{pa  -599 -1376}\special{pa  -607 -1369}\special{pa  -615 -1357}%
\special{pa  -623 -1341}\special{pa  -630 -1322}\special{pa  -637 -1298}\special{pa  -643 -1271}%
\special{pa  -648 -1241}\special{pa  -652 -1208}\special{pa  -655 -1174}\special{pa  -657 -1138}%
\special{pa  -658 -1101}\special{pa  -658 -1064}\special{pa  -657 -1027}\special{pa  -655  -991}%
\special{pa  -652  -957}\special{pa  -648  -924}\special{pa  -643  -894}\special{pa  -637  -867}%
\special{pa  -630  -844}\special{pa  -623  -824}\special{pa  -615  -808}\special{pa  -607  -797}%
\special{pa  -599  -790}\special{pa  -591  -787}\special{pa  -582  -790}\special{pa  -574  -797}%
\special{pa  -566  -808}\special{pa  -558  -824}\special{pa  -551  -844}\special{pa  -544  -867}%
\special{pa  -538  -894}\special{pa  -533  -924}\special{pa  -529  -957}\special{pa  -526  -991}%
\special{pa  -524 -1027}\special{pa  -523 -1064}\special{pa  -523 -1101}\special{pa  -524 -1138}%
\special{pa  -526 -1174}\special{pa  -529 -1208}\special{pa  -533 -1241}\special{pa  -538 -1271}%
\special{pa  -544 -1298}\special{pa  -551 -1322}\special{pa  -558 -1341}\special{pa  -566 -1357}%
\special{pa  -574 -1369}\special{pa  -582 -1376}\special{pa  -591 -1378}%
\special{fp}%
\special{pa   984  -787}\special{pa   989  -788}\special{pa   993  -790}\special{pa   997  -793}%
\special{pa  1001  -797}\special{pa  1005  -802}\special{pa  1009  -808}\special{pa  1013  -816}%
\special{pa  1017  -824}\special{pa  1021  -833}\special{pa  1024  -844}\special{pa  1027  -855}%
\special{pa  1031  -867}\special{pa  1034  -881}\special{pa  1036  -894}\special{pa  1039  -909}%
\special{pa  1041  -924}\special{pa  1044  -940}\special{pa  1046  -957}\special{pa  1047  -974}%
\special{pa  1049  -991}\special{pa  1050 -1009}\special{pa  1051 -1027}\special{pa  1051 -1046}%
\special{pa  1052 -1064}\special{pa  1052 -1083}\special{pa  1052 -1101}\special{pa  1051 -1120}%
\special{pa  1051 -1138}\special{pa  1050 -1156}\special{pa  1049 -1174}\special{pa  1047 -1191}%
\special{pa  1046 -1208}\special{pa  1044 -1225}\special{pa  1041 -1241}\special{pa  1039 -1256}%
\special{pa  1036 -1271}\special{pa  1034 -1285}\special{pa  1031 -1298}\special{pa  1027 -1310}%
\special{pa  1024 -1322}\special{pa  1021 -1332}\special{pa  1017 -1341}\special{pa  1013 -1350}%
\special{pa  1009 -1357}\special{pa  1005 -1364}\special{pa  1001 -1369}\special{pa   997 -1373}%
\special{pa   993 -1376}\special{pa   989 -1377}\special{pa   984 -1378}%
\special{fp}%
\special{pa  -591 -1378}\special{pa   984 -1378}%
\special{fp}%
\special{pa  -591  -787}\special{pa   984  -787}%
\special{fp}%
\special{pa  -591    -0}\special{pa  -591  -787}%
\special{fp}%
\special{pa   984  -787}\special{pa   984    -0}%
\special{fp}%
\special{pa   217   -98}\special{pa   217    98}%
\special{fp}%
\special{pn 16}%
\special{pa   177   -39}\special{pa   177    39}%
\special{fp}%
\special{pn 8}%
\special{pa  -591    -0}\special{pa   177    -0}%
\special{fp}%
\special{pa   984    -0}\special{pa   217    -0}%
\special{fp}%
\special{pa 319 -1157}\special{pa 394 -1181}\special{pa 319 -1205}\special{pa 334 -1181}%
\special{pa 319 -1157}\special{pa 319 -1157}\special{sh 1}\special{ip}%
\special{pn 1}%
\special{pa   319 -1157}\special{pa   394 -1181}\special{pa   319 -1205}\special{pa   334 -1181}%
\special{pa   319 -1157}%
\special{fp}%
\special{pn 8}%
\special{pn 16}%
\special{pa     0 -1181}\special{pa   334 -1181}%
\special{fp}%
\special{pn 8}%
\settowidth{\Width}{$V$}\setlength{\Width}{-1\Width}%
\settoheight{\Height}{$V$}\settodepth{\Depth}{$V$}\setlength{\Height}{\Depth}%
\put(  0.450,  0.400){\hspace*{\Width}\raisebox{\Height}{$V$}}%
%
\settowidth{\Width}{$v$}\setlength{\Width}{0\Width}%
\settoheight{\Height}{$v$}\settodepth{\Depth}{$v$}\setlength{\Height}{-0.5\Height}\setlength{\Depth}{0.5\Depth}\addtolength{\Height}{\Depth}%
\put(  1.100,  3.000){\hspace*{\Width}\raisebox{\Height}{$v$}}%
%
\settowidth{\Width}{$-e$}\setlength{\Width}{-0.5\Width}%
\settoheight{\Height}{$-e$}\settodepth{\Depth}{$-e$}\setlength{\Height}{-\Height}%
\put(  0.510,  2.660){\hspace*{\Width}\raisebox{\Height}{$-e$}}%
%
\settowidth{\Width}{S}\setlength{\Width}{-0.5\Width}%
\settoheight{\Height}{S}\settodepth{\Depth}{S}\setlength{\Height}{-0.5\Height}\setlength{\Depth}{0.5\Depth}\addtolength{\Height}{\Depth}%
\put( -1.900,  2.870){\hspace*{\Width}\raisebox{\Height}{S}}%
%
\special{pa 209 -1087}\special{pa 209 -1088}\special{pa 208 -1089}\special{pa 208 -1090}%
\special{pa 208 -1091}\special{pa 207 -1092}\special{pa 207 -1092}\special{pa 206 -1093}%
\special{pa 205 -1094}\special{pa 204 -1094}\special{pa 203 -1094}\special{pa 202 -1095}%
\special{pa 202 -1095}\special{pa 201 -1095}\special{pa 200 -1095}\special{pa 199 -1094}%
\special{pa 198 -1094}\special{pa 197 -1094}\special{pa 196 -1093}\special{pa 196 -1092}%
\special{pa 195 -1092}\special{pa 195 -1091}\special{pa 194 -1090}\special{pa 194 -1089}%
\special{pa 194 -1088}\special{pa 194 -1087}\special{pa 194 -1086}\special{pa 194 -1085}%
\special{pa 194 -1085}\special{pa 195 -1084}\special{pa 195 -1083}\special{pa 196 -1082}%
\special{pa 196 -1082}\special{pa 197 -1081}\special{pa 198 -1081}\special{pa 199 -1080}%
\special{pa 200 -1080}\special{pa 201 -1080}\special{pa 202 -1080}\special{pa 202 -1080}%
\special{pa 203 -1080}\special{pa 204 -1081}\special{pa 205 -1081}\special{pa 206 -1082}%
\special{pa 207 -1082}\special{pa 207 -1083}\special{pa 208 -1084}\special{pa 208 -1085}%
\special{pa 208 -1085}\special{pa 209 -1086}\special{pa 209 -1087}\special{pa 209 -1087}%
\special{sh 1}\special{ip}%
\special{pa   209 -1087}\special{pa   209 -1088}\special{pa   208 -1089}\special{pa   208 -1090}%
\special{pa   208 -1091}\special{pa   207 -1092}\special{pa   207 -1092}\special{pa   206 -1093}%
\special{pa   205 -1094}\special{pa   204 -1094}\special{pa   203 -1094}\special{pa   202 -1095}%
\special{pa   202 -1095}\special{pa   201 -1095}\special{pa   200 -1095}\special{pa   199 -1094}%
\special{pa   198 -1094}\special{pa   197 -1094}\special{pa   196 -1093}\special{pa   196 -1092}%
\special{pa   195 -1092}\special{pa   195 -1091}\special{pa   194 -1090}\special{pa   194 -1089}%
\special{pa   194 -1088}\special{pa   194 -1087}\special{pa   194 -1086}\special{pa   194 -1085}%
\special{pa   194 -1085}\special{pa   195 -1084}\special{pa   195 -1083}\special{pa   196 -1082}%
\special{pa   196 -1082}\special{pa   197 -1081}\special{pa   198 -1081}\special{pa   199 -1080}%
\special{pa   200 -1080}\special{pa   201 -1080}\special{pa   202 -1080}\special{pa   202 -1080}%
\special{pa   203 -1080}\special{pa   204 -1081}\special{pa   205 -1081}\special{pa   206 -1082}%
\special{pa   207 -1082}\special{pa   207 -1083}\special{pa   208 -1084}\special{pa   208 -1085}%
\special{pa   208 -1085}\special{pa   209 -1086}\special{pa   209 -1087}%
\special{fp}%
\special{pa   984 -1378}\special{pa   970 -1384}\special{pa   957 -1389}\special{pa   943 -1395}%
\special{pa   929 -1400}\special{pa   915 -1405}\special{pa   901 -1411}\special{pa   886 -1416}%
\special{pa   872 -1421}\special{pa   858 -1426}\special{pa   844 -1430}\special{pa   830 -1435}%
\special{pa   815 -1440}\special{pa   801 -1444}\special{pa   787 -1449}\special{pa   772 -1453}%
\special{pa   758 -1457}\special{pa   744 -1461}\special{pa   729 -1465}\special{pa   715 -1469}%
\special{pa   700 -1473}\special{pa   686 -1476}\special{pa   671 -1480}\special{pa   657 -1483}%
\special{pa   642 -1486}\special{pa   627 -1490}\special{pa   613 -1493}\special{pa   598 -1496}%
\special{pa   583 -1499}\special{pa   568 -1501}\special{pa   554 -1504}\special{pa   539 -1507}%
\special{pa   524 -1509}\special{pa   509 -1511}\special{pa   495 -1514}\special{pa   480 -1516}%
\special{pa   465 -1518}\special{pa   450 -1520}\special{pa   435 -1522}\special{pa   420 -1523}%
\special{pa   405 -1525}\special{pa   390 -1526}\special{pa   376 -1528}\special{pa   361 -1529}%
\special{pa   346 -1530}\special{pa   331 -1531}\special{pa   316 -1532}\special{pa   301 -1533}%
\special{pa   286 -1533}\special{pa   271 -1534}\special{pa   256 -1535}%
\special{fp}%
\special{pa   138 -1535}\special{pa   123 -1534}\special{pa   108 -1533}\special{pa    93 -1533}%
\special{pa    78 -1532}\special{pa    63 -1531}\special{pa    48 -1530}\special{pa    33 -1529}%
\special{pa    18 -1528}\special{pa     3 -1526}\special{pa   -12 -1525}\special{pa   -27 -1523}%
\special{pa   -41 -1522}\special{pa   -56 -1520}\special{pa   -71 -1518}\special{pa   -86 -1516}%
\special{pa  -101 -1514}\special{pa  -116 -1511}\special{pa  -130 -1509}\special{pa  -145 -1507}%
\special{pa  -160 -1504}\special{pa  -175 -1501}\special{pa  -190 -1499}\special{pa  -204 -1496}%
\special{pa  -219 -1493}\special{pa  -234 -1490}\special{pa  -248 -1486}\special{pa  -263 -1483}%
\special{pa  -277 -1480}\special{pa  -292 -1476}\special{pa  -306 -1473}\special{pa  -321 -1469}%
\special{pa  -335 -1465}\special{pa  -350 -1461}\special{pa  -364 -1457}\special{pa  -379 -1453}%
\special{pa  -393 -1449}\special{pa  -407 -1444}\special{pa  -422 -1440}\special{pa  -436 -1435}%
\special{pa  -450 -1430}\special{pa  -464 -1426}\special{pa  -479 -1421}\special{pa  -493 -1416}%
\special{pa  -507 -1411}\special{pa  -521 -1405}\special{pa  -535 -1400}\special{pa  -549 -1395}%
\special{pa  -563 -1389}\special{pa  -577 -1384}\special{pa  -591 -1378}%
\special{fp}%
\settowidth{\Width}{$\ell$}\setlength{\Width}{-0.5\Width}%
\settoheight{\Height}{$\ell$}\settodepth{\Depth}{$\ell$}\setlength{\Height}{-0.5\Height}\setlength{\Depth}{0.5\Depth}\addtolength{\Height}{\Depth}%
\put(  0.500,  3.900){\hspace*{\Width}\raisebox{\Height}{$\ell$}}%
%
\end{picture}}%
        }
        断面積$S$\tanni{m^2},長さ$\ell $\tanni{m}の導体中に,
        電荷$-e$\tanni{C}の自由電子が1\sftanni{m^3}あたり$n$個
        ある。この導体の両端に電圧$V$\tanni{V}をかけると,
        導体内部には強さ\Hako \tanni{V/m}の一様な電場が生じる。
        1個の自由電子は大きさ\Hako \tanni{N}の力を受ける。
        また,自由電子は導体中のイオンなどとの衝突によって,
        その速さ$v$\tanni{m/s}に比例する抵抗力$kv$\tanni{N}
        ($k$\tanni{N\cdot s/m}は比例定数)を受け,一定の速さ$v$\tanni{m/s}
        で移動すると考えることができる。この速さ$v$を用いると導体を
        流れる電流$I$は$I=$\Hako \tanni{A}と表される。\\
        ~~したがって,電圧と電流は比例することが分かり,導体の
        電気抵抗$R$は$R=$\Hako \tanni{\Omega }と表される。
        また,導体の抵抗率$\rho $が$\rho =$\Hako \tanni{\Omega \cdot m}と
        表されることも分かる。
    \end{mawarikomi}
