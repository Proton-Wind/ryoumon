\hakosyokika
\item
    \begin{mawarikomi}{150pt}{
        %WinTpicVersion4.32a
{\unitlength 0.1in%
\begin{picture}(25.0000,20.0800)(3.3000,-37.5800)%
% VECTOR 2 0 3 0 Black White  
% 2 1300 2298 1300 2098
% 
\special{pn 8}%
\special{pa 1300 2298}%
\special{pa 1300 2098}%
\special{fp}%
\special{sh 1}%
\special{pa 1300 2098}%
\special{pa 1280 2165}%
\special{pa 1300 2151}%
\special{pa 1320 2165}%
\special{pa 1300 2098}%
\special{fp}%
% VECTOR 2 0 3 0 Black White  
% 2 1500 2298 1500 2098
% 
\special{pn 8}%
\special{pa 1500 2298}%
\special{pa 1500 2098}%
\special{fp}%
\special{sh 1}%
\special{pa 1500 2098}%
\special{pa 1480 2165}%
\special{pa 1500 2151}%
\special{pa 1520 2165}%
\special{pa 1500 2098}%
\special{fp}%
% VECTOR 2 0 3 0 Black White  
% 2 1700 2298 1700 2098
% 
\special{pn 8}%
\special{pa 1700 2298}%
\special{pa 1700 2098}%
\special{fp}%
\special{sh 1}%
\special{pa 1700 2098}%
\special{pa 1680 2165}%
\special{pa 1700 2151}%
\special{pa 1720 2165}%
\special{pa 1700 2098}%
\special{fp}%
% VECTOR 2 0 3 0 Black White  
% 2 1900 2298 1900 2098
% 
\special{pn 8}%
\special{pa 1900 2298}%
\special{pa 1900 2098}%
\special{fp}%
\special{sh 1}%
\special{pa 1900 2098}%
\special{pa 1880 2165}%
\special{pa 1900 2151}%
\special{pa 1920 2165}%
\special{pa 1900 2098}%
\special{fp}%
% STR 2 0 3 0 Black White  
% 4 1765 2148 1765 2198 5 0 0 0
% $B$
\put(17.6500,-21.9800){\makebox(0,0){$B$}}%
% STR 2 0 3 0 Black White  
% 4 2670 2748 2670 2798 5 0 0 0
% $y$
\put(26.7000,-27.9800){\makebox(0,0){$y$}}%
% STR 2 0 3 0 Black White  
% 4 970 3473 970 3523 5 0 0 0
% $x$
\put(9.7000,-35.2300){\makebox(0,0){$x$}}%
% STR 2 0 3 0 Black White  
% 4 1400 1773 1400 1823 5 0 0 0
% $z$
\put(14.0000,-18.2300){\makebox(0,0){$z$}}%
% VECTOR 2 0 3 0 Black White  
% 2 1400 2798 1400 1898
% 
\special{pn 8}%
\special{pa 1400 2798}%
\special{pa 1400 1898}%
\special{fp}%
\special{sh 1}%
\special{pa 1400 1898}%
\special{pa 1380 1965}%
\special{pa 1400 1951}%
\special{pa 1420 1965}%
\special{pa 1400 1898}%
\special{fp}%
% VECTOR 2 0 3 0 Black White  
% 2 1400 2798 2575 2798
% 
\special{pn 8}%
\special{pa 1400 2798}%
\special{pa 2575 2798}%
\special{fp}%
\special{sh 1}%
\special{pa 2575 2798}%
\special{pa 2508 2778}%
\special{pa 2522 2798}%
\special{pa 2508 2818}%
\special{pa 2575 2798}%
\special{fp}%
% VECTOR 2 0 3 0 Black White  
% 2 1400 2798 900 3498
% 
\special{pn 8}%
\special{pa 1400 2798}%
\special{pa 900 3498}%
\special{fp}%
\special{sh 1}%
\special{pa 900 3498}%
\special{pa 955 3455}%
\special{pa 931 3455}%
\special{pa 922 3432}%
\special{pa 900 3498}%
\special{fp}%
% LINE 2 0 3 0 Black White  
% 4 1400 2498 1100 2898 1100 2898 1100 3218
% 
\special{pn 8}%
\special{pa 1400 2498}%
\special{pa 1100 2898}%
\special{fp}%
\special{pa 1100 2898}%
\special{pa 1100 3218}%
\special{fp}%
% LINE 2 0 3 0 Black White  
% 6 1100 3218 2100 3218 2100 3218 2100 2898 2100 2898 1100 2898
% 
\special{pn 8}%
\special{pa 1100 3218}%
\special{pa 2100 3218}%
\special{fp}%
\special{pa 2100 3218}%
\special{pa 2100 2898}%
\special{fp}%
\special{pa 2100 2898}%
\special{pa 1100 2898}%
\special{fp}%
% LINE 2 0 3 0 Black White  
% 8 1400 2498 2400 2498 2400 2498 2100 2898 2100 3218 2400 2798 2400 2798 2400 2498
% 
\special{pn 8}%
\special{pa 1400 2498}%
\special{pa 2400 2498}%
\special{fp}%
\special{pa 2400 2498}%
\special{pa 2100 2898}%
\special{fp}%
\special{pa 2100 3218}%
\special{pa 2400 2798}%
\special{fp}%
\special{pa 2400 2798}%
\special{pa 2400 2498}%
\special{fp}%
% DOT 0 0 3 0 Black White  
% 2 1600 3058 1900 2658
% 
\special{pn 4}%
\special{sh 1}%
\special{ar 1600 3058 16 16 0 6.2831853}%
\special{sh 1}%
\special{ar 1900 2658 16 16 0 6.2831853}%
% DOT 0 0 3 0 Black White  
% 1 2250 2858
% 
\special{pn 4}%
\special{sh 1}%
\special{ar 2250 2858 16 16 0 6.2831853}%
% DOT 0 0 3 0 Black White  
% 1 1250 2858
% 
\special{pn 4}%
\special{sh 1}%
\special{ar 1250 2858 16 16 0 6.2831853}%
% LINE 2 1 3 0 Black White  
% 2 1250 2858 1135 2858
% 
\special{pn 8}%
\special{pa 1250 2858}%
\special{pa 1135 2858}%
\special{da 0.015}%
% LINE 2 0 3 0 Black White  
% 2 1135 2858 935 2858
% 
\special{pn 8}%
\special{pa 1135 2858}%
\special{pa 935 2858}%
\special{fp}%
% LINE 2 0 3 0 Black White  
% 8 930 2855 330 3655 330 3655 2230 3655 2230 3655 2830 2855 2830 2855 2250 2855
% 
\special{pn 8}%
\special{pa 930 2855}%
\special{pa 330 3655}%
\special{fp}%
\special{pa 330 3655}%
\special{pa 2230 3655}%
\special{fp}%
\special{pa 2230 3655}%
\special{pa 2830 2855}%
\special{fp}%
\special{pa 2830 2855}%
\special{pa 2250 2855}%
\special{fp}%
% CIRCLE 3 0 3 0 Black White  
% 4 2100 2648 2400 2498 2400 2798 2400 2498
% 
\special{pn 4}%
\special{ar 2100 2648 335 335 5.8195377 0.4636476}%
% STR 2 0 3 0 Black White  
% 4 2485 2598 2485 2648 5 0 1 0
% $c$
\put(24.8500,-26.4800){\makebox(0,0){{\colorbox[named]{White}{$c$}}}}%
% CIRCLE 3 0 3 0 Black White  
% 4 1900 3398 2400 2498 2400 2498 1400 2498
% 
\special{pn 4}%
\special{ar 1900 3398 1030 1030 4.2052905 5.2194875}%
% STR 2 0 3 0 Black White  
% 4 1900 2318 1900 2368 5 0 1 0
% $b$
\put(19.0000,-23.6800){\makebox(0,0){{\colorbox[named]{White}{$b$}}}}%
% CIRCLE 3 0 3 0 Black White  
% 4 1415 2823 1415 2503 1400 2503 1100 2898
% 
\special{pn 4}%
\special{ar 1415 2823 320 320 2.9078495 4.6655483}%
% STR 2 0 3 0 Black White  
% 4 1140 2568 1140 2618 5 0 1 0
% $a$
\put(11.4000,-26.1800){\makebox(0,0){{\colorbox[named]{White}{$a$}}}}%
% BOX 3 5 2 0 Black White  
% 2 1640 3558 1690 3758
% 
\special{pn 0}%
\special{sh 0}%
\special{pa 1640 3558}%
\special{pa 1690 3558}%
\special{pa 1690 3758}%
\special{pa 1640 3758}%
\special{pa 1640 3558}%
\special{ip}%
\special{pn 4}%
\special{pa 1640 3558}%
\special{pa 1690 3558}%
\special{pa 1690 3758}%
\special{pa 1640 3758}%
\special{pa 1640 3558}%
\special{ip}%
% LINE 2 0 3 0 Black White  
% 2 1690 3633 1690 3683
% 
\special{pn 8}%
\special{pa 1690 3633}%
\special{pa 1690 3683}%
\special{fp}%
% LINE 2 0 3 0 Black White  
% 2 1640 3583 1640 3733
% 
\special{pn 8}%
\special{pa 1640 3583}%
\special{pa 1640 3733}%
\special{fp}%
% STR 2 0 3 0 Black White  
% 4 1640 3033 1640 3083 1 0 0 0
% N
\put(16.4000,-30.8300){\makebox(0,0)[lt]{N}}%
% STR 2 0 3 0 Black White  
% 4 1940 2633 1940 2683 1 0 0 0
% M
\put(19.4000,-26.8300){\makebox(0,0)[lt]{M}}%
% VECTOR 2 0 3 0 Black White  
% 2 940 2783 1170 2783
% 
\special{pn 8}%
\special{pa 940 2783}%
\special{pa 1170 2783}%
\special{fp}%
\special{sh 1}%
\special{pa 1170 2783}%
\special{pa 1103 2763}%
\special{pa 1117 2783}%
\special{pa 1103 2803}%
\special{pa 1170 2783}%
\special{fp}%
% STR 2 0 3 0 Black White  
% 4 840 2733 840 2783 5 0 0 0
% $I$
\put(8.4000,-27.8300){\makebox(0,0){$I$}}%
\end{picture}}%

    }
次の(1)〜(5)には式を、(a)〜(e)には適当な語句を入れよ。

直方体の $n$ 型半導体があり、$x, y, z$ 方向の長さをそれぞれ $a, b, c$ とする。
また、半導体は単位体積あたり $n$ 個の電子をもつ。図のように $y$ 軸の正の向きに強さ $I$ の一様な電流が流れている。
電子の電荷の大きさを $e$、平均の速さを $v$ とすると、電流 $I$ は $\boxed{\text{ (1) }}$ と表される。

いま、$z$ 軸の正の向きに磁束密度 $B$ の一様な磁場を加えた。電子はやはり平均の速さ $v$ で運動しているとすると、大きさ $\boxed{\text{ (2) }}$ の力を $x$ 軸の $\boxed{\text{ (a) }}$ の向きに受ける。この力は $\boxed{\text{ (b) }}$ とよばれる。その結果、電子が $x$ 軸方向に移動するため、$\mathrm{M}$ に対して $\mathrm{N}$ の電位は $\boxed{\text{ (c) }}$ なり、$\mathrm{MN}$ 間には電場が発生する。やがて半導体内の電子に対して磁場による力と電場による力がつりあうことになる。この状態での電場の強さは $\boxed{\text{ (3) }}$ と表される。したがって、$\mathrm{MN}$ 間の電位差 $V$ は $\boxed{\text{ (4) }}$ と表され、$I$ を用いると $V=\boxed{\text{ (5) }}$ と表される。

次に、$n$ 型半導体の代わりに $p$ 型半導体で同様な実験を行った。$p$ 型では $\boxed{\text{ (d) }}$ が電流の担い手となるので、$\mathrm{M}$ に対して $\mathrm{N}$ の電位は $\boxed{\text{ (e) }}$ なる。

\end{mawarikomi}
