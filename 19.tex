\item
        \begin{mawarikomi}{150pt}{%WinTpicVersion4.32a
{\unitlength 0.1in%
\begin{picture}(24.0000,12.2000)(4.0000,-18.0000)%
% LINE 2 0 3 0 Black White  
% 12 400 1000 1000 1200 1000 1200 1600 1000 1600 1000 2200 800 2200 800 1600 600 1000 1200 1000 1400 1000 1400 1300 1300
% 
\special{pn 8}%
\special{pa 400 1000}%
\special{pa 1000 1200}%
\special{fp}%
\special{pa 1000 1200}%
\special{pa 1600 1000}%
\special{fp}%
\special{pa 1600 1000}%
\special{pa 2200 800}%
\special{fp}%
\special{pa 2200 800}%
\special{pa 1600 600}%
\special{fp}%
\special{pa 1000 1200}%
\special{pa 1000 1400}%
\special{fp}%
\special{pa 1000 1400}%
\special{pa 1300 1300}%
\special{fp}%
% LINE 2 0 3 0 Black White  
% 8 1300 1300 1300 1100 1300 1100 1900 1300 1900 1300 2500 1100 2500 1100 1900 900
% 
\special{pn 8}%
\special{pa 1300 1300}%
\special{pa 1300 1100}%
\special{fp}%
\special{pa 1300 1100}%
\special{pa 1900 1300}%
\special{fp}%
\special{pa 1900 1300}%
\special{pa 2500 1100}%
\special{fp}%
\special{pa 2500 1100}%
\special{pa 1900 900}%
\special{fp}%
% LINE 2 0 3 0 Black White  
% 2 2200 1000 2200 800
% 
\special{pn 8}%
\special{pa 2200 1000}%
\special{pa 2200 800}%
\special{fp}%
% LINE 2 0 3 0 Black White  
% 8 1300 1300 1900 1500 1900 1500 2500 1300 2500 1300 2500 1100 1900 1300 1900 1500
% 
\special{pn 8}%
\special{pa 1300 1300}%
\special{pa 1900 1500}%
\special{fp}%
\special{pa 1900 1500}%
\special{pa 2500 1300}%
\special{fp}%
\special{pa 2500 1300}%
\special{pa 2500 1100}%
\special{fp}%
\special{pa 1900 1300}%
\special{pa 1900 1500}%
\special{fp}%
% LINE 2 0 3 0 Black White  
% 4 1000 1400 2200 1800 2200 1000 2800 1200
% 
\special{pn 8}%
\special{pa 1000 1400}%
\special{pa 2200 1800}%
\special{fp}%
\special{pa 2200 1000}%
\special{pa 2800 1200}%
\special{fp}%
% STR 2 0 3 0 Black White  
% 4 1600 1210 1600 1310 5 0 0 0
% A
\put(16.0000,-13.1000){\makebox(0,0){A}}%
% STR 2 0 3 0 Black White  
% 4 1160 1160 1160 1260 5 0 0 0
% S$_3$
\put(11.6000,-12.6000){\makebox(0,0){S$_3$}}%
% STR 2 0 3 0 Black White  
% 4 2180 1500 2180 1600 5 0 0 0
% S$_2$
\put(21.8000,-16.0000){\makebox(0,0){S$_2$}}%
% STR 2 0 3 0 Black White  
% 4 780 900 780 1000 5 0 0 0
% S$_1$
\put(7.8000,-10.0000){\makebox(0,0){S$_1$}}%
% LINE 2 2 3 0 Black White  
% 2 2230 1190 1030 790
% 
\special{pn 8}%
\special{pa 2230 1190}%
\special{pa 1030 790}%
\special{dt 0.025}%
% CIRCLE 2 0 0 0 Black Black  
% 4 980 780 1020 780 1020 780 1020 780
% 
\special{sh 1.000}%
\special{ia 980 780 40 40 0.0000000 6.2831853}%
\special{pn 8}%
\special{ar 980 780 40 40 0.0000000 6.2831853}%
% VECTOR 2 0 3 0 Black White  
% 2 990 780 1290 880
% 
\special{pn 8}%
\special{pa 990 780}%
\special{pa 1290 880}%
\special{fp}%
\special{sh 1}%
\special{pa 1290 880}%
\special{pa 1233 840}%
\special{pa 1239 863}%
\special{pa 1220 878}%
\special{pa 1290 880}%
\special{fp}%
% STR 2 0 3 0 Black White  
% 4 900 610 900 710 2 0 0 0
% B
\put(9.0000,-7.1000){\makebox(0,0)[lb]{B}}%
% STR 2 0 3 0 Black White  
% 4 1170 680 1170 780 2 0 0 0
% $v_0$
\put(11.7000,-7.8000){\makebox(0,0)[lb]{$v_0$}}%
\end{picture}}%
}
            なめらかな水平面S$_1$,S$_2$と鉛直面S$_3$からなる段差のある固定台がある。面S$_2$上に,質量$M$の直方体Aを面S$_3$に接するように置く。Aの上面はあらく,その高さはS$_1$の高さに等しい。質量$m$の小物体BとAの間の動摩擦係数を$\mu $とし,重力加速度を$g$とする。いま,Bを初速$v_0$で水平面S$_1$上から,Aの上面中央を直進させたところ,Aは運動をはじめ,ある時刻$t_0$以後,両物体の速さは等しくなった。\\
            \hakosyokika
            ~~BがA上に達した時刻を$t=0$とする。時刻$t_0$より以前の時刻$t$におけるBの速さは\Hako で,Aの速さは\Hako である。$t_0$は\Hako で,そのときの速さは\Hako である。また,BがA上を進んだ距離$\ell $は\Hako である。
        \end{mawarikomi}
