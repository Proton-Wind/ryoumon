\hakosyokika
\item
    \begin{mawarikomi}{150pt}{\begin{zahyou*}[ul=6mm](-0.3,7)(-1,7.5)
    \small
    \def\Fx{-1*(X+2.5)*(X-2.5)}
    \def\Gx{-1*(X-2.5)*(X-6.5)}
    \def\A{(0,0)}
    \def\AL{(\xmin,0)}
    \def\B{(0,6.25)}
    \def\C{(2.5,0)}
    \def\D{(6.5,0)}
    \def\DR{(\xmax,0)}
    \def\E{(0,-0.5)}
    \def\EL{(\xmin,-0.5)}
    \def\F{(6.5,-0.5)}
    \def\FR{(\xmax,-0.5)}
    \def\vvec{(2,0)}
    \Nuritubusi{\AL\EL\FR\DR\AL}
    \Drawline{\AL\DR}
    \Put\A(0pt,-10pt)[t]{A}
    \Put\C(0pt,-10pt)[t]{C}
    \Put\D(0pt,-10pt)[t]{D}
    \YGurafu\Fx{0}{2.5}
    \YGurafu\Gx{2.5}{6.5}
    \HenKo<henkotype=parallel,
    henkoH=0ex,
    yazirusi=b
    % henkosideb=0,
    % henkosidet=1.2
    >\A\B{$h$}
    {\thicklines
    \Kuromaru{\B}
    \Put\B{\Yasen\vvec}}
    \Put\B[w]{B}
    \Put\B(35pt,0pt)[l]{$v_0$}

\end{zahyou*}
}
        なめらかな水平面の点Aの真上,高さ$h$の点Bから,小球を初速$v_0$で水平方向に投げ出した。小球は水平面の点Cではね返り,次に落下した点をDとする。ここで小球と水平面との反発係数(はね返り係数)を$e$とする。重力加速度の大きさを$g$とし,問(2),(3)では,水平成分は右向きを正,鉛直成分は上向きを正とする。
        \begin{enumerate}
            \item 点Bから点Cに落下するまでの時間$t_1$と,AC間の距離を求めよ。
            \item 点Cに落下する直前の,速度の水平成分と鉛直成分をそれぞれ求めよ。
            \item 点Cではね返った直後の,速度の水平成分と鉛直成分をそれぞれ求めよ。
            \item CD間での最高点の高さ$H$を求めよ。
            \item 点Cから点Dに達するまでの時間$t_2$と,CD間の距離を求めよ。
        \end{enumerate}
    \end{mawarikomi}