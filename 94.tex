\hakosyokika
\item
    \begin{mawarikomi}(10pt,0){160pt}{%WinTpicVersion4.32a
{\unitlength 0.1in%
\begin{picture}(15.7185,37.0866)(4.1339,-37.3031)%
% VECTOR 2 0 3 0 Black White  
% 2 1045 1986 1045 195
% 
\special{pn 8}%
\special{pa 1029 1955}%
\special{pa 1029 192}%
\special{fp}%
\special{sh 1}%
\special{pa 1029 192}%
\special{pa 1009 258}%
\special{pa 1029 244}%
\special{pa 1048 258}%
\special{pa 1029 192}%
\special{fp}%
% LINE 2 1 3 0 Black White  
% 2 1045 1091 1941 1091
% 
\special{pn 8}%
\special{pa 1029 1074}%
\special{pa 1910 1074}%
\special{da 0.030}%
% CIRCLE 2 1 3 0 Black White  
% 4 1045 1091 1792 1091 1045 1837 1045 344
% 
\special{pn 8}%
\special{pn 8}%
\special{pa 1029 339}%
\special{pa 1049 339}%
\special{pa 1055 340}%
\special{pa 1056 340}%
\special{fp}%
\special{pa 1083 341}%
\special{pa 1090 342}%
\special{pa 1095 342}%
\special{pa 1109 344}%
\special{pa 1110 344}%
\special{fp}%
\special{pa 1137 346}%
\special{pa 1150 348}%
\special{pa 1156 349}%
\special{pa 1162 351}%
\special{pa 1164 351}%
\special{fp}%
\special{pa 1191 356}%
\special{pa 1196 358}%
\special{pa 1202 359}%
\special{pa 1209 361}%
\special{pa 1216 362}%
\special{pa 1217 362}%
\special{fp}%
\special{pa 1243 371}%
\special{pa 1248 372}%
\special{pa 1254 374}%
\special{pa 1261 376}%
\special{pa 1267 378}%
\special{pa 1269 379}%
\special{fp}%
\special{pa 1294 389}%
\special{pa 1298 390}%
\special{pa 1305 393}%
\special{pa 1311 395}%
\special{pa 1319 399}%
\special{fp}%
\special{pa 1345 410}%
\special{pa 1347 411}%
\special{pa 1353 414}%
\special{pa 1360 417}%
\special{pa 1369 421}%
\special{fp}%
\special{pa 1393 435}%
\special{pa 1396 437}%
\special{pa 1402 440}%
\special{pa 1407 444}%
\special{pa 1412 447}%
\special{pa 1415 449}%
\special{fp}%
\special{pa 1438 464}%
\special{pa 1441 466}%
\special{pa 1447 469}%
\special{pa 1452 472}%
\special{pa 1461 478}%
\special{fp}%
\special{pa 1483 496}%
\special{pa 1484 497}%
\special{pa 1490 501}%
\special{pa 1495 506}%
\special{pa 1500 510}%
\special{pa 1504 512}%
\special{fp}%
\special{pa 1524 531}%
\special{pa 1531 536}%
\special{pa 1535 541}%
\special{pa 1540 546}%
\special{pa 1544 549}%
\special{fp}%
\special{pa 1563 569}%
\special{pa 1569 575}%
\special{pa 1573 580}%
\special{pa 1578 585}%
\special{pa 1581 589}%
\special{fp}%
\special{pa 1599 610}%
\special{pa 1599 610}%
\special{pa 1603 616}%
\special{pa 1608 621}%
\special{pa 1612 626}%
\special{pa 1615 631}%
\special{fp}%
\special{pa 1631 653}%
\special{pa 1632 654}%
\special{pa 1640 664}%
\special{pa 1643 670}%
\special{pa 1647 675}%
\special{fp}%
\special{pa 1661 699}%
\special{pa 1661 699}%
\special{pa 1667 711}%
\special{pa 1671 717}%
\special{pa 1674 722}%
\special{pa 1674 722}%
\special{fp}%
\special{pa 1687 746}%
\special{pa 1690 752}%
\special{pa 1693 759}%
\special{pa 1699 770}%
\special{fp}%
\special{pa 1709 796}%
\special{pa 1712 801}%
\special{pa 1714 808}%
\special{pa 1717 814}%
\special{pa 1719 821}%
\special{fp}%
\special{pa 1727 847}%
\special{pa 1729 852}%
\special{pa 1731 859}%
\special{pa 1733 865}%
\special{pa 1735 872}%
\special{pa 1735 873}%
\special{fp}%
\special{pa 1742 899}%
\special{pa 1744 905}%
\special{pa 1745 910}%
\special{pa 1747 917}%
\special{pa 1748 924}%
\special{pa 1748 926}%
\special{fp}%
\special{pa 1753 953}%
\special{pa 1755 958}%
\special{pa 1757 970}%
\special{pa 1758 977}%
\special{pa 1758 979}%
\special{fp}%
\special{pa 1761 1006}%
\special{pa 1761 1011}%
\special{pa 1762 1018}%
\special{pa 1762 1025}%
\special{pa 1763 1031}%
\special{pa 1763 1033}%
\special{fp}%
\special{pa 1764 1061}%
\special{pa 1764 1088}%
\special{fp}%
\special{pa 1763 1115}%
\special{pa 1763 1118}%
\special{pa 1762 1125}%
\special{pa 1762 1132}%
\special{pa 1761 1142}%
\special{fp}%
\special{pa 1757 1169}%
\special{pa 1756 1179}%
\special{pa 1755 1185}%
\special{pa 1753 1196}%
\special{fp}%
\special{pa 1748 1223}%
\special{pa 1748 1225}%
\special{pa 1747 1232}%
\special{pa 1745 1238}%
\special{pa 1743 1245}%
\special{pa 1742 1249}%
\special{fp}%
\special{pa 1736 1276}%
\special{pa 1735 1278}%
\special{pa 1731 1290}%
\special{pa 1729 1297}%
\special{pa 1728 1301}%
\special{fp}%
\special{pa 1719 1327}%
\special{pa 1719 1329}%
\special{pa 1716 1335}%
\special{pa 1714 1342}%
\special{pa 1711 1347}%
\special{pa 1709 1352}%
\special{fp}%
\special{pa 1699 1378}%
\special{pa 1698 1379}%
\special{pa 1687 1402}%
\special{fp}%
\special{pa 1674 1426}%
\special{pa 1673 1427}%
\special{pa 1667 1439}%
\special{pa 1663 1444}%
\special{pa 1660 1449}%
\special{fp}%
\special{pa 1646 1473}%
\special{pa 1646 1473}%
\special{pa 1643 1478}%
\special{pa 1635 1490}%
\special{pa 1631 1495}%
\special{fp}%
\special{pa 1615 1517}%
\special{pa 1611 1523}%
\special{pa 1607 1528}%
\special{pa 1602 1533}%
\special{pa 1598 1538}%
\special{fp}%
\special{pa 1581 1559}%
\special{pa 1577 1564}%
\special{pa 1567 1574}%
\special{pa 1563 1579}%
\special{pa 1563 1579}%
\special{fp}%
\special{pa 1543 1599}%
\special{pa 1539 1603}%
\special{pa 1534 1607}%
\special{pa 1530 1612}%
\special{pa 1525 1617}%
\special{pa 1524 1618}%
\special{fp}%
\special{pa 1503 1635}%
\special{pa 1494 1643}%
\special{pa 1488 1648}%
\special{pa 1483 1652}%
\special{fp}%
\special{pa 1461 1669}%
\special{pa 1457 1672}%
\special{pa 1451 1676}%
\special{pa 1445 1679}%
\special{pa 1440 1683}%
\special{pa 1439 1684}%
\special{fp}%
\special{pa 1416 1699}%
\special{pa 1411 1702}%
\special{pa 1406 1705}%
\special{pa 1400 1709}%
\special{pa 1394 1712}%
\special{pa 1393 1713}%
\special{fp}%
\special{pa 1368 1725}%
\special{pa 1364 1728}%
\special{pa 1346 1737}%
\special{pa 1344 1738}%
\special{fp}%
\special{pa 1319 1749}%
\special{pa 1316 1751}%
\special{pa 1309 1753}%
\special{pa 1297 1759}%
\special{pa 1295 1760}%
\special{fp}%
\special{pa 1269 1769}%
\special{pa 1265 1770}%
\special{pa 1259 1772}%
\special{pa 1252 1775}%
\special{pa 1246 1777}%
\special{pa 1243 1778}%
\special{fp}%
\special{pa 1217 1784}%
\special{pa 1214 1785}%
\special{pa 1207 1787}%
\special{pa 1201 1788}%
\special{pa 1194 1790}%
\special{pa 1191 1790}%
\special{fp}%
\special{pa 1164 1796}%
\special{pa 1160 1797}%
\special{pa 1155 1798}%
\special{pa 1137 1801}%
\special{fp}%
\special{pa 1110 1805}%
\special{pa 1107 1805}%
\special{pa 1100 1805}%
\special{pa 1088 1807}%
\special{pa 1083 1807}%
\special{fp}%
\special{pa 1056 1809}%
\special{pa 1054 1809}%
\special{pa 1029 1809}%
\special{fp}%
% CIRCLE 1 0 3 0 Black White  
% 4 1045 1091 1792 1091 1493 1837 1493 344
% 
\special{pn 13}%
\special{ar 1029 1074 735 735 5.2526116 1.0299827}%
% LINE 2 1 3 0 Black White  
% 2 1045 1091 1418 449
% 
\special{pn 8}%
\special{pa 1029 1074}%
\special{pa 1396 442}%
\special{da 0.030}%
% LINE 2 1 3 0 Black White  
% 2 1045 1091 1418 1732
% 
\special{pn 8}%
\special{pa 1029 1074}%
\special{pa 1396 1705}%
\special{da 0.030}%
% VECTOR 2 0 3 0 Black White  
% 2 1045 1091 1717 792
% 
\special{pn 8}%
\special{pa 1029 1074}%
\special{pa 1690 780}%
\special{fp}%
\special{sh 1}%
\special{pa 1690 780}%
\special{pa 1622 788}%
\special{pa 1642 801}%
\special{pa 1638 824}%
\special{pa 1690 780}%
\special{fp}%
% CIRCLE 2 0 3 0 Black White  
% 4 1045 1091 1344 1091 1792 1091 1717 792
% 
\special{pn 8}%
\special{ar 1029 1074 294 294 5.8645468 6.2831853}%
% STR 2 0 3 0 Black White  
% 4 1366 991 1366 1066 2 0 0 0
% $\theta $
\put(13.4449,-10.4921){\makebox(0,0)[lb]{$\theta $}}%
% LINE 2 0 3 0 Black White  
% 2 1008 344 1083 344
% 
\special{pn 8}%
\special{pa 992 339}%
\special{pa 1066 339}%
\special{fp}%
% LINE 2 0 3 0 Black White  
% 2 1008 1837 1083 1837
% 
\special{pn 8}%
\special{pa 992 1808}%
\special{pa 1066 1808}%
\special{fp}%
% STR 2 0 3 0 Black White  
% 4 909 270 909 344 5 0 0 0
% 1.0
\put(8.9469,-3.3858){\makebox(0,0){1.0}}%
% STR 2 0 3 0 Black White  
% 4 859 1762 859 1837 5 0 0 0
% $-1.0$
\put(8.4547,-18.0807){\makebox(0,0){$-1.0$}}%
% BOX 2 0 3 0 Black White  
% 2 1049 994 1009 1194
% 
\special{pn 8}%
\special{pa 1032 978}%
\special{pa 993 978}%
\special{pa 993 1175}%
\special{pa 1032 1175}%
\special{pa 1032 978}%
\special{pa 993 978}%
\special{fp}%
% LINE 3 0 3 0 Black White  
% 16 1045 1145 1005 1185 1045 1175 1030 1190 1045 1115 1005 1155 1045 1085 1005 1125 1045 1055 1005 1095 1045 1025 1005 1065 1045 995 1005 1035 1020 990 1005 1005
% 
\special{pn 4}%
\special{pa 1029 1127}%
\special{pa 989 1166}%
\special{fp}%
\special{pa 1029 1156}%
\special{pa 1014 1171}%
\special{fp}%
\special{pa 1029 1097}%
\special{pa 989 1137}%
\special{fp}%
\special{pa 1029 1068}%
\special{pa 989 1107}%
\special{fp}%
\special{pa 1029 1038}%
\special{pa 989 1078}%
\special{fp}%
\special{pa 1029 1009}%
\special{pa 989 1048}%
\special{fp}%
\special{pa 1029 979}%
\special{pa 989 1019}%
\special{fp}%
\special{pa 1004 974}%
\special{pa 989 989}%
\special{fp}%
% LINE 3 0 3 0 Black White  
% 2 1009 994 809 794
% 
\special{pn 4}%
\special{pa 993 978}%
\special{pa 796 781}%
\special{fp}%
% STR 2 0 3 0 Black White  
% 4 420 740 420 790 2 0 0 0
% 回折格子
\put(4.1339,-7.7756){\makebox(0,0)[lb]{回折格子}}%
% VECTOR 1 0 3 0 Black White  
% 2 550 1090 985 1090
% 
\special{pn 13}%
\special{pa 541 1073}%
\special{pa 969 1073}%
\special{fp}%
\special{sh 1}%
\special{pa 969 1073}%
\special{pa 904 1053}%
\special{pa 917 1073}%
\special{pa 904 1093}%
\special{pa 969 1073}%
\special{fp}%
% STR 2 0 3 0 Black White  
% 4 505 1025 505 1075 2 0 0 0
% 単色光
\put(4.9705,-10.5807){\makebox(0,0)[lb]{単色光}}%
% STR 2 0 3 0 Black White  
% 4 1317 1985 1317 2035 2 0 0 0
% 図1
\put(12.9626,-20.0295){\makebox(0,0)[lb]{図1}}%
% STR 2 0 3 0 Black White  
% 4 1045 20 1045 95 5 0 0 0
% $x$\kern-4pt〔{\sf m}〕
\put(10.2854,-0.9350){\makebox(0,0){$x$\kern-4pt〔{\sf m}〕}}%
% VECTOR 2 0 3 0 Black White  
% 2 1042 3695 1042 2395
% 
\special{pn 8}%
\special{pa 1026 3637}%
\special{pa 1026 2357}%
\special{fp}%
\special{sh 1}%
\special{pa 1026 2357}%
\special{pa 1006 2423}%
\special{pa 1026 2409}%
\special{pa 1045 2423}%
\special{pa 1026 2357}%
\special{fp}%
% BOX 2 0 3 0 Black White  
% 2 1046 3394 1006 3594
% 
\special{pn 8}%
\special{pa 1030 3341}%
\special{pa 990 3341}%
\special{pa 990 3537}%
\special{pa 1030 3537}%
\special{pa 1030 3341}%
\special{pa 990 3341}%
\special{fp}%
% LINE 3 0 3 0 Black White  
% 16 1042 3545 1002 3585 1042 3575 1027 3590 1042 3515 1002 3555 1042 3485 1002 3525 1042 3455 1002 3495 1042 3425 1002 3465 1042 3395 1002 3435 1017 3390 1002 3405
% 
\special{pn 4}%
\special{pa 1026 3489}%
\special{pa 986 3529}%
\special{fp}%
\special{pa 1026 3519}%
\special{pa 1011 3533}%
\special{fp}%
\special{pa 1026 3460}%
\special{pa 986 3499}%
\special{fp}%
\special{pa 1026 3430}%
\special{pa 986 3469}%
\special{fp}%
\special{pa 1026 3401}%
\special{pa 986 3440}%
\special{fp}%
\special{pa 1026 3371}%
\special{pa 986 3410}%
\special{fp}%
\special{pa 1026 3342}%
\special{pa 986 3381}%
\special{fp}%
\special{pa 1001 3337}%
\special{pa 986 3351}%
\special{fp}%
% LINE 2 1 3 0 Black White  
% 2 1042 3486 1342 3486
% 
\special{pn 8}%
\special{pa 1026 3431}%
\special{pa 1321 3431}%
\special{da 0.030}%
% STR 2 0 3 0 Black White  
% 4 982 3725 982 3775 2 0 0 0
% ~
\put(9.6654,-37.1555){\makebox(0,0)[lb]{~}}%
% SPLINE 2 0 3 0 Black White  
% 4 1371 3430 1357 3459 1371 3501 1357 3530
% 
\special{pn 8}%
\special{pa 1349 3376}%
\special{pa 1336 3405}%
\special{pa 1346 3434}%
\special{pa 1344 3464}%
\special{pa 1336 3474}%
\special{fp}%
% CIRCLE 2 1 3 0 Black White  
% 4 1042 3485 2042 3485 2242 3185 1042 3185
% 
\special{pn 8}%
\special{pn 8}%
\special{pa 1026 2446}%
\special{pa 1053 2446}%
\special{fp}%
\special{pa 1080 2447}%
\special{pa 1082 2448}%
\special{pa 1094 2448}%
\special{pa 1096 2449}%
\special{pa 1106 2449}%
\special{fp}%
\special{pa 1132 2452}%
\special{pa 1136 2452}%
\special{pa 1139 2453}%
\special{pa 1144 2453}%
\special{pa 1147 2454}%
\special{pa 1153 2454}%
\special{pa 1156 2455}%
\special{pa 1159 2455}%
\special{fp}%
\special{pa 1185 2459}%
\special{pa 1186 2459}%
\special{pa 1189 2460}%
\special{pa 1191 2460}%
\special{pa 1194 2461}%
\special{pa 1197 2461}%
\special{pa 1200 2462}%
\special{pa 1203 2462}%
\special{pa 1206 2463}%
\special{pa 1208 2463}%
\special{pa 1211 2464}%
\special{fp}%
\special{pa 1237 2469}%
\special{pa 1241 2469}%
\special{pa 1247 2471}%
\special{pa 1250 2471}%
\special{pa 1252 2472}%
\special{pa 1255 2473}%
\special{pa 1258 2473}%
\special{pa 1261 2474}%
\special{pa 1263 2475}%
\special{fp}%
\special{pa 1289 2482}%
\special{pa 1293 2483}%
\special{pa 1296 2483}%
\special{pa 1299 2484}%
\special{pa 1301 2485}%
\special{pa 1310 2488}%
\special{pa 1312 2488}%
\special{pa 1314 2489}%
\special{fp}%
\special{pa 1340 2497}%
\special{pa 1340 2497}%
\special{pa 1344 2499}%
\special{pa 1347 2500}%
\special{pa 1349 2501}%
\special{pa 1358 2504}%
\special{pa 1360 2505}%
\special{pa 1365 2507}%
\special{fp}%
\special{pa 1390 2516}%
\special{pa 1390 2516}%
\special{pa 1392 2517}%
\special{pa 1398 2519}%
\special{pa 1400 2520}%
\special{pa 1406 2522}%
\special{pa 1407 2523}%
\special{pa 1410 2525}%
\special{pa 1413 2526}%
\special{pa 1414 2527}%
\special{fp}%
\special{pa 1439 2536}%
\special{pa 1441 2538}%
\special{pa 1447 2540}%
\special{pa 1449 2541}%
\special{pa 1452 2543}%
\special{pa 1454 2544}%
\special{pa 1460 2546}%
\special{pa 1462 2548}%
\special{pa 1463 2548}%
\special{fp}%
\special{pa 1486 2561}%
\special{pa 1487 2561}%
\special{pa 1489 2562}%
\special{pa 1492 2563}%
\special{pa 1494 2565}%
\special{pa 1497 2566}%
\special{pa 1499 2567}%
\special{pa 1502 2569}%
\special{pa 1504 2570}%
\special{pa 1507 2572}%
\special{pa 1509 2573}%
\special{pa 1510 2573}%
\special{fp}%
\special{pa 1533 2587}%
\special{pa 1533 2587}%
\special{pa 1536 2589}%
\special{pa 1538 2590}%
\special{pa 1540 2592}%
\special{pa 1543 2593}%
\special{pa 1545 2594}%
\special{pa 1548 2596}%
\special{pa 1550 2597}%
\special{pa 1553 2599}%
\special{pa 1555 2600}%
\special{pa 1555 2600}%
\special{fp}%
\special{pa 1578 2615}%
\special{pa 1581 2618}%
\special{pa 1584 2619}%
\special{pa 1586 2621}%
\special{pa 1588 2622}%
\special{pa 1591 2624}%
\special{pa 1593 2626}%
\special{pa 1595 2627}%
\special{pa 1598 2630}%
\special{fp}%
\special{pa 1620 2646}%
\special{pa 1622 2648}%
\special{pa 1625 2649}%
\special{pa 1631 2655}%
\special{pa 1634 2656}%
\special{pa 1636 2657}%
\special{pa 1640 2661}%
\special{pa 1641 2662}%
\special{fp}%
\special{pa 1662 2679}%
\special{pa 1662 2679}%
\special{pa 1670 2687}%
\special{pa 1672 2688}%
\special{pa 1675 2690}%
\special{pa 1682 2697}%
\special{fp}%
\special{pa 1702 2715}%
\special{pa 1706 2719}%
\special{pa 1708 2720}%
\special{pa 1712 2724}%
\special{pa 1715 2726}%
\special{pa 1719 2730}%
\special{pa 1720 2732}%
\special{pa 1721 2733}%
\special{fp}%
\special{pa 1739 2752}%
\special{pa 1748 2761}%
\special{pa 1749 2763}%
\special{pa 1751 2766}%
\special{pa 1757 2772}%
\special{fp}%
\special{pa 1775 2792}%
\special{pa 1776 2793}%
\special{pa 1780 2797}%
\special{pa 1782 2801}%
\special{pa 1784 2804}%
\special{pa 1792 2812}%
\special{pa 1792 2812}%
\special{fp}%
\special{pa 1808 2834}%
\special{pa 1809 2835}%
\special{pa 1811 2837}%
\special{pa 1813 2840}%
\special{pa 1814 2842}%
\special{pa 1820 2848}%
\special{pa 1821 2850}%
\special{pa 1823 2852}%
\special{pa 1825 2854}%
\special{fp}%
\special{pa 1840 2877}%
\special{pa 1841 2878}%
\special{pa 1843 2881}%
\special{pa 1845 2885}%
\special{pa 1846 2888}%
\special{pa 1848 2890}%
\special{pa 1850 2893}%
\special{pa 1851 2895}%
\special{pa 1853 2897}%
\special{pa 1854 2899}%
\special{fp}%
\special{pa 1868 2921}%
\special{pa 1869 2923}%
\special{pa 1871 2926}%
\special{pa 1872 2928}%
\special{pa 1874 2931}%
\special{pa 1875 2933}%
\special{pa 1877 2936}%
\special{pa 1878 2938}%
\special{pa 1880 2941}%
\special{pa 1881 2943}%
\special{pa 1881 2944}%
\special{fp}%
\special{pa 1894 2967}%
\special{pa 1895 2968}%
\special{pa 1897 2972}%
\special{pa 1899 2975}%
\special{pa 1900 2977}%
\special{pa 1901 2980}%
\special{pa 1903 2983}%
\special{pa 1904 2985}%
\special{pa 1905 2988}%
\special{pa 1906 2990}%
\special{pa 1906 2991}%
\special{fp}%
\special{pa 1918 3015}%
\special{pa 1919 3019}%
\special{pa 1920 3021}%
\special{pa 1922 3024}%
\special{pa 1923 3026}%
\special{pa 1924 3029}%
\special{pa 1926 3033}%
\special{pa 1928 3036}%
\special{pa 1929 3039}%
\special{pa 1929 3039}%
\special{fp}%
\special{pa 1940 3064}%
\special{pa 1941 3068}%
\special{pa 1942 3070}%
\special{pa 1944 3076}%
\special{pa 1945 3078}%
\special{pa 1948 3087}%
\special{pa 1949 3089}%
\special{pa 1949 3089}%
\special{fp}%
\special{pa 1959 3115}%
\special{pa 1959 3115}%
\special{pa 1959 3118}%
\special{pa 1960 3121}%
\special{pa 1961 3123}%
\special{pa 1964 3132}%
\special{pa 1965 3134}%
\special{pa 1966 3137}%
\special{pa 1966 3140}%
\special{fp}%
\special{pa 1974 3166}%
\special{pa 1974 3166}%
\special{pa 1974 3169}%
\special{pa 1976 3175}%
\special{pa 1977 3177}%
\special{pa 1977 3180}%
\special{pa 1979 3186}%
\special{pa 1979 3189}%
\special{pa 1980 3191}%
\special{fp}%
% CIRCLE 1 0 3 0 Black White  
% 4 1042 3485 2042 3485 2142 3185 1542 2185
% 
\special{pn 13}%
\special{ar 1026 3430 984 984 5.0795628 6.0169333}%
% LINE 2 1 3 0 Black White  
% 2 1042 2845 1807 2845
% 
\special{pn 8}%
\special{pa 1026 2800}%
\special{pa 1779 2800}%
\special{da 0.030}%
% LINE 2 1 3 0 Black White  
% 2 1042 3165 1987 3165
% 
\special{pn 8}%
\special{pa 1026 3115}%
\special{pa 1956 3115}%
\special{da 0.030}%
% LINE 2 1 3 0 Black White  
% 2 1042 3005 1917 3005
% 
\special{pn 8}%
\special{pa 1026 2958}%
\special{pa 1887 2958}%
\special{da 0.030}%
% STR 2 0 3 0 Black White  
% 4 882 3115 882 3165 5 0 0 0
% 0.32
\put(8.6811,-31.1516){\makebox(0,0){0.32}}%
% STR 2 0 3 0 Black White  
% 4 882 2955 882 3005 5 0 0 0
% 0.48
\put(8.6811,-29.5768){\makebox(0,0){0.48}}%
% STR 2 0 3 0 Black White  
% 4 882 2795 882 2845 5 0 0 0
% 0.64
\put(8.6811,-28.0020){\makebox(0,0){0.64}}%
% STR 2 0 3 0 Black White  
% 4 1952 3265 1952 3315 2 0 0 0
% ~
\put(19.2126,-32.6280){\makebox(0,0)[lb]{~}}%
% STR 2 0 3 0 Black White  
% 4 882 2435 882 2485 5 0 0 0
% 1.0
\put(8.6811,-24.4587){\makebox(0,0){1.0}}%
% STR 2 0 3 0 Black White  
% 4 1042 2220 1042 2295 5 0 0 0
% $x$\kern-4pt〔{\sf m}〕
\put(10.2559,-22.5886){\makebox(0,0){$x$\kern-4pt〔{\sf m}〕}}%
% STR 2 0 3 0 Black White  
% 4 1832 2780 1832 2830 2 0 0 0
% R
\put(18.0315,-27.8543){\makebox(0,0)[lb]{R}}%
% DOT 0 0 3 0 Black White  
% 3 1812 2845 1917 3005 1987 3165
% 
\special{pn 4}%
\special{sh 1}%
\special{ar 1783 2800 16 16 0 6.2831853}%
\special{sh 1}%
\special{ar 1887 2958 16 16 0 6.2831853}%
\special{sh 1}%
\special{ar 1956 3115 16 16 0 6.2831853}%
% STR 2 0 3 0 Black White  
% 4 1942 2945 1942 2995 2 0 0 0
% Q
\put(19.1142,-29.4783){\makebox(0,0)[lb]{Q}}%
% STR 2 0 3 0 Black White  
% 4 2017 3115 2017 3165 2 0 0 0
% P
\put(19.8524,-31.1516){\makebox(0,0)[lb]{P}}%
% LINE 3 0 3 0 Black White  
% 2 1900 425 1635 640
% 
\special{pn 4}%
\special{pa 1870 418}%
\special{pa 1609 630}%
\special{fp}%
% STR 2 0 3 0 Black White  
% 4 1620 370 1620 420 2 0 0 0
% スクリーン
\put(15.9449,-4.1339){\makebox(0,0)[lb]{スクリーン}}%
% STR 2 0 3 0 Black White  
% 4 1717 3370 1717 3420 2 0 0 0
% スクリーン
\put(16.8996,-33.6614){\makebox(0,0)[lb]{スクリーン}}%
% STR 2 0 3 0 Black White  
% 4 965 1090 965 1140 4 0 0 0
% O
\put(9.4980,-11.2205){\makebox(0,0)[rt]{O}}%
% STR 2 0 3 0 Black White  
% 4 1057 3450 1057 3500 1 0 0 0
% O
\put(10.4035,-34.4488){\makebox(0,0)[lt]{O}}%
% STR 2 0 3 0 Black White  
% 4 1317 3885 1317 3935 2 0 0 0
% 図2
\put(12.9626,-38.7303){\makebox(0,0)[lb]{図2}}%
\end{picture}}%
}
    格子定数$d$の回折格子に垂直に単色光を入射させ,入射光の進行方向と回折光の進行方向のなす角度を$\theta $
    として,円筒状のスクリーン上に現れる明線を$-60\Deg <\theta <60\Deg $の範囲で観測する。
    回折格子の位置を原点Oとして,入射光および円筒の中心軸に垂直な方向に$x$軸を定める。
        \begin{enumerate}
            \item $d=1.2\times 10^{-6}$\sftanni{m}の回折格子に,波長$\lambda =6.0\times 10^{-7}$\sftanni{m}光を入射させたとき,スクリーン上($-60\Deg <\theta <60\Deg $)に現れる明線の本数は何本か。
            \item 格子定数の分かっていない回折格子に取り替えた。この回折格子に,赤色の単色光と青色の単色光を同時に入射させたところ,スクリーンの$0\Deg < \theta <60\Deg $の範囲には,図2のP,Q,Rの位置にのみ明線が観測された。3本の明線のうち,青色の明線はどれか。次のうちから選べ。
            \begin{edaenumerate}<3>[m]
                \item Pのみ
                \item Qのみ
                \item Rのみ
                \item PとQ
                \item QとR
                \item PとR
            \end{edaenumerate}
            \item (2)において,赤色の波長を$\lambda_\mathrm{R}=6.8\times 10^{-7}$\sftanni{m}とする。格子定数を求めよ。
        \end{enumerate}
    \end{mawarikomi}