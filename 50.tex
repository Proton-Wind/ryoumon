\hakosyokika
\item
    \begin{mawarikomi}{150pt}{\begin{zahyou*}[ul=6mm](-3,3)(0,9)
    \small
    \def\A{(-1,6.5)}
    \def\B{(-1,1)}
    \def\C{(1,1)}
    \def\D{(1,6.5)}
    \def\E{(-3,5)}
    \def\F{(-3,0)}
    \def\G{(3,0)}
    \def\H{(3,5)}
    \def\P{(0,7.5)}
    \def\Q{(0.5,8)}
    \Nuritubusi{\E\F\G\H\E}
    \Drawline{\E\H}
    \Nuritubusi[0]{\A\B\C\D\A}
    \Drawline{\A\B\C\D\A}
    \Suisen\C\E\H\CU
    \HenKo<henkotype=parallel,
    henkoH=2ex,
    yazirusi=b,
    henkosideb=0,
    henkosidet=1.2>\C\CU{$d$}
    \HenKo<henkotype=parallel,
    henkoH=3ex,
    yazirusi=b,
    henkosideb=0,
    henkosidet=1.2>\A\B{$\ell $}
    \put(-0.3,4){$\rho _0$}
    {\fboxsep=0.5pt
    \put(-2.5,2){\colorbox{white}{$\rho $}}
    \put(-0.3,0.5){\colorbox{white}{$S$}}}
    \En*[1]\P{0.3}
    {\thicklines
    \Put\Q{\yasen(0,-1.3)}
    }
    \Put\Q[se]{$v_0$}
    % \Put\P[w]{くぎ}
    % \Put\Q(6pt,0pt)[l]{$m$}
    % \Put\L(6pt,5pt)[l]{床}
    % \Put\B[w]{天井}
    % \HenKo[0]\M\Q{$L$}
\end{zahyou*}
}
        底面積$S$で長さ$\ell $で一様な密度$\rho _0$の円柱が密度$\rho $の液体に浮かんでいる。この円柱と同じ質量の小球を真上から落とし,速さ$v_0$で弾性衝突させた。重力加速度の大きさを$g$とし,円柱の運動に伴う液体からの抵抗は無視でき,液面は一定の高さを保つものとする。また,円柱の上面が液面下に沈むことはないものとする。
        \begin{Enumerate}
            \item 円柱が静止しているときの,液面下の深さ$d$を求めよ。
            \item 小球と衝突直後の円柱の速さを求めよ。 
        \end{Enumerate}
        衝突後,小球は取り除かれ,円柱は単振動を始めた。
        \begin{Enumerate*}
            \item 静止した状態から円柱が下方に$x$だけ沈んでいるとき,円柱が受ける合力$f$を下向き正として求めよ。
            \item 円柱が達する液面下の深さの最大値$d_1$を求めよ。
            \item 円柱が静止位置より上に上がって初めて速さが0になるまでの時間$t$(衝突後の時間)を求めよ。
        \end{Enumerate*}
    \end{mawarikomi}