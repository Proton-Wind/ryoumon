\hakosyokika
\item
    \begin{mawarikomi}(20pt,0pt){150pt}{
        %%% C:/vpn/vpn/KeTCindy/fig/fig111_1.tex 
%%% Generator=fig111_1.cdy 
{\unitlength=1cm%
\begin{picture}%
(6,9.5)(-3,-6)%
\special{pn 8}%
%
\special{pa  -197  -984}\special{pa   197  -984}\special{pa   197   787}\special{pa  -197   787}%
\special{pa  -197  -984}%
\special{fp}%
\special{pn 32}%
\special{pa  -787  -984}\special{pa  -787   787}%
\special{fp}%
\special{pn 8}%
\special{pn 32}%
\special{pa   787  -984}\special{pa   787   787}%
\special{fp}%
\special{pn 8}%
\special{pa  -787   591}\special{pa  -984   591}\special{pa  -984  1575}\special{pa   984  1575}%
\special{pa   984   591}\special{pa   787   591}%
\special{fp}%
\special{pa   122   787}\special{pa   197   713}%
\special{fp}%
\special{pa   -17   787}\special{pa   197   574}%
\special{fp}%
\special{pa  -156   787}\special{pa   197   434}%
\special{fp}%
\special{pa  -197   689}\special{pa   197   295}%
\special{fp}%
\special{pa  -197   550}\special{pa   197   156}%
\special{fp}%
\special{pa  -197   411}\special{pa   197    17}%
\special{fp}%
\special{pa  -197   271}\special{pa   197  -122}%
\special{fp}%
\special{pa  -197   132}\special{pa   197  -262}%
\special{fp}%
\special{pa  -197    -7}\special{pa   197  -401}%
\special{fp}%
\special{pa  -197  -146}\special{pa   197  -540}%
\special{fp}%
\special{pa  -197  -285}\special{pa   197  -679}%
\special{fp}%
\special{pa  -197  -425}\special{pa   197  -818}%
\special{fp}%
\special{pa  -197  -564}\special{pa   197  -957}%
\special{fp}%
\special{pa  -197  -703}\special{pa    84  -984}%
\special{fp}%
\special{pa  -197  -842}\special{pa   -55  -984}%
\special{fp}%
\special{pa  -197  -981}\special{pa  -194  -984}%
\special{fp}%
\special{pa     0  1094}\special{pa   -14   874}%
\special{fp}%
\special{pa     0  1181}\special{pa     0  1094}%
\special{fp}%
\special{pa     0   787}\special{pa     0   874}%
\special{fp}%
{%
\color[cmyk]{0,0,0,0}%
\special{pa 15 1094}\special{pa 15 1093}\special{pa 14 1091}\special{pa 14 1089}\special{pa 13 1087}%
\special{pa 12 1086}\special{pa 11 1084}\special{pa 10 1083}\special{pa 8 1082}\special{pa 6 1081}%
\special{pa 5 1080}\special{pa 3 1080}\special{pa 1 1080}\special{pa -1 1080}\special{pa -3 1080}%
\special{pa -5 1080}\special{pa -6 1081}\special{pa -8 1082}\special{pa -10 1083}%
\special{pa -11 1084}\special{pa -12 1086}\special{pa -13 1087}\special{pa -14 1089}%
\special{pa -14 1091}\special{pa -15 1093}\special{pa -15 1094}\special{pa -15 1096}%
\special{pa -14 1098}\special{pa -14 1100}\special{pa -13 1102}\special{pa -12 1103}%
\special{pa -11 1105}\special{pa -10 1106}\special{pa -8 1107}\special{pa -6 1108}%
\special{pa -5 1109}\special{pa -3 1109}\special{pa -1 1109}\special{pa 1 1109}\special{pa 3 1109}%
\special{pa 5 1109}\special{pa 6 1108}\special{pa 8 1107}\special{pa 10 1106}\special{pa 11 1105}%
\special{pa 12 1103}\special{pa 13 1102}\special{pa 14 1100}\special{pa 14 1098}\special{pa 15 1096}%
\special{pa 15 1094}\special{pa 15 1094}\special{sh 1}\special{ip}%
}%
\special{pa    15  1094}\special{pa    15  1093}\special{pa    14  1091}\special{pa    14  1089}%
\special{pa    13  1087}\special{pa    12  1086}\special{pa    11  1084}\special{pa    10  1083}%
\special{pa     8  1082}\special{pa     6  1081}\special{pa     5  1080}\special{pa     3  1080}%
\special{pa     1  1080}\special{pa    -1  1080}\special{pa    -3  1080}\special{pa    -5  1080}%
\special{pa    -6  1081}\special{pa    -8  1082}\special{pa   -10  1083}\special{pa   -11  1084}%
\special{pa   -12  1086}\special{pa   -13  1087}\special{pa   -14  1089}\special{pa   -14  1091}%
\special{pa   -15  1093}\special{pa   -15  1094}\special{pa   -15  1096}\special{pa   -14  1098}%
\special{pa   -14  1100}\special{pa   -13  1102}\special{pa   -12  1103}\special{pa   -11  1105}%
\special{pa   -10  1106}\special{pa    -8  1107}\special{pa    -6  1108}\special{pa    -5  1109}%
\special{pa    -3  1109}\special{pa    -1  1109}\special{pa     1  1109}\special{pa     3  1109}%
\special{pa     5  1109}\special{pa     6  1108}\special{pa     8  1107}\special{pa    10  1106}%
\special{pa    11  1105}\special{pa    12  1103}\special{pa    13  1102}\special{pa    14  1100}%
\special{pa    14  1098}\special{pa    15  1096}\special{pa    15  1094}%
\special{fp}%
{%
\color[cmyk]{0,0,0,0}%
\special{pa 15 874}\special{pa 15 872}\special{pa 14 870}\special{pa 14 869}\special{pa 13 867}%
\special{pa 12 865}\special{pa 11 864}\special{pa 10 862}\special{pa 8 861}\special{pa 6 860}%
\special{pa 5 860}\special{pa 3 859}\special{pa 1 859}\special{pa -1 859}\special{pa -3 859}%
\special{pa -5 860}\special{pa -6 860}\special{pa -8 861}\special{pa -10 862}\special{pa -11 864}%
\special{pa -12 865}\special{pa -13 867}\special{pa -14 869}\special{pa -14 870}\special{pa -15 872}%
\special{pa -15 874}\special{pa -15 876}\special{pa -14 878}\special{pa -14 880}\special{pa -13 881}%
\special{pa -12 883}\special{pa -11 884}\special{pa -10 886}\special{pa -8 887}\special{pa -6 888}%
\special{pa -5 888}\special{pa -3 889}\special{pa -1 889}\special{pa 1 889}\special{pa 3 889}%
\special{pa 5 888}\special{pa 6 888}\special{pa 8 887}\special{pa 10 886}\special{pa 11 884}%
\special{pa 12 883}\special{pa 13 881}\special{pa 14 880}\special{pa 14 878}\special{pa 15 876}%
\special{pa 15 874}\special{pa 15 874}\special{sh 1}\special{ip}%
}%
\special{pa    15   874}\special{pa    15   872}\special{pa    14   870}\special{pa    14   869}%
\special{pa    13   867}\special{pa    12   865}\special{pa    11   864}\special{pa    10   862}%
\special{pa     8   861}\special{pa     6   860}\special{pa     5   860}\special{pa     3   859}%
\special{pa     1   859}\special{pa    -1   859}\special{pa    -3   859}\special{pa    -5   860}%
\special{pa    -6   860}\special{pa    -8   861}\special{pa   -10   862}\special{pa   -11   864}%
\special{pa   -12   865}\special{pa   -13   867}\special{pa   -14   869}\special{pa   -14   870}%
\special{pa   -15   872}\special{pa   -15   874}\special{pa   -15   876}\special{pa   -14   878}%
\special{pa   -14   880}\special{pa   -13   881}\special{pa   -12   883}\special{pa   -11   884}%
\special{pa   -10   886}\special{pa    -8   887}\special{pa    -6   888}\special{pa    -5   888}%
\special{pa    -3   889}\special{pa    -1   889}\special{pa     1   889}\special{pa     3   889}%
\special{pa     5   888}\special{pa     6   888}\special{pa     8   887}\special{pa    10   886}%
\special{pa    11   884}\special{pa    12   883}\special{pa    13   881}\special{pa    14   880}%
\special{pa    14   878}\special{pa    15   876}\special{pa    15   874}%
\special{fp}%
\special{pa  -138  1350}\special{pa   138  1350}%
\special{fp}%
\special{pn 16}%
\special{pa   -55  1406}\special{pa    55  1406}%
\special{fp}%
\special{pn 8}%
\special{pa     0  1575}\special{pa     0  1406}%
\special{fp}%
\special{pa     0  1181}\special{pa     0  1350}%
\special{fp}%
\settowidth{\Width}{図1}\setlength{\Width}{-0.5\Width}%
\settoheight{\Height}{図1}\settodepth{\Depth}{図1}\setlength{\Height}{\Depth}%
\put(  0.000, -5.300){\hspace*{\Width}\raisebox{\Height}{図1}}%
%
\settowidth{\Width}{$\mathrm{P_1}$}\setlength{\Width}{-0.5\Width}%
\settoheight{\Height}{$\mathrm{P_1}$}\settodepth{\Depth}{$\mathrm{P_1}$}\setlength{\Height}{\Depth}%
\put( -2.000,  2.700){\hspace*{\Width}\raisebox{\Height}{$\mathrm{P_1}$}}%
%
\settowidth{\Width}{$\mathrm{P_2}$}\setlength{\Width}{-0.5\Width}%
\settoheight{\Height}{$\mathrm{P_2}$}\settodepth{\Depth}{$\mathrm{P_2}$}\setlength{\Height}{\Depth}%
\put(  2.000,  2.700){\hspace*{\Width}\raisebox{\Height}{$\mathrm{P_2}$}}%
%
\settowidth{\Width}{$\mathrm{P_M}$}\setlength{\Width}{-0.5\Width}%
\settoheight{\Height}{$\mathrm{P_M}$}\settodepth{\Depth}{$\mathrm{P_M}$}\setlength{\Height}{\Depth}%
\put(  0.000,  2.700){\hspace*{\Width}\raisebox{\Height}{$\mathrm{P_M}$}}%
%
\settowidth{\Width}{S}\setlength{\Width}{-1\Width}%
\settoheight{\Height}{S}\settodepth{\Depth}{S}\setlength{\Height}{-0.5\Height}\setlength{\Depth}{0.5\Depth}\addtolength{\Height}{\Depth}%
\put( -0.200, -2.500){\hspace*{\Width}\raisebox{\Height}{S}}%
%
\settowidth{\Width}{$-Q_0$}\setlength{\Width}{0\Width}%
\settoheight{\Height}{$-Q_0$}\settodepth{\Depth}{$-Q_0$}\setlength{\Height}{-0.5\Height}\setlength{\Depth}{0.5\Depth}\addtolength{\Height}{\Depth}%
\put( -1.800,  0.500){\hspace*{\Width}\raisebox{\Height}{$-Q_0$}}%
%
\settowidth{\Width}{$V_0$}\setlength{\Width}{0\Width}%
\settoheight{\Height}{$V_0$}\settodepth{\Depth}{$V_0$}\setlength{\Height}{-0.5\Height}\setlength{\Depth}{0.5\Depth}\addtolength{\Height}{\Depth}%
\put(  0.560, -3.500){\hspace*{\Width}\raisebox{\Height}{$V_0$}}%
%
\special{pa -713 566}\special{pa -787 591}\special{pa -713 615}\special{pa -727 591}%
\special{pa -713 566}\special{pa -713 566}\special{sh 1}\special{ip}%
\special{pn 1}%
\special{pa  -713   566}\special{pa  -787   591}\special{pa  -713   615}\special{pa  -727   591}%
\special{pa  -713   566}%
\special{fp}%
\special{pn 8}%
\special{pa  -492   591}\special{pa  -727   591}%
\special{fp}%
\special{pa -272 615}\special{pa -197 591}\special{pa -272 566}\special{pa -257 591}%
\special{pa -272 615}\special{pa -272 615}\special{sh 1}\special{ip}%
\special{pn 1}%
\special{pa  -272   615}\special{pa  -197   591}\special{pa  -272   566}\special{pa  -257   591}%
\special{pa  -272   615}%
\special{fp}%
\special{pn 8}%
\special{pa  -492   591}\special{pa  -257   591}%
\special{fp}%
\special{pa 713 615}\special{pa 787 591}\special{pa 713 566}\special{pa 727 591}\special{pa 713 615}%
\special{pa 713 615}\special{sh 1}\special{ip}%
\special{pn 1}%
\special{pa   713   615}\special{pa   787   591}\special{pa   713   566}\special{pa   727   591}%
\special{pa   713   615}%
\special{fp}%
\special{pn 8}%
\special{pa   492   591}\special{pa   727   591}%
\special{fp}%
\special{pa 272 566}\special{pa 197 591}\special{pa 272 615}\special{pa 257 591}\special{pa 272 566}%
\special{pa 272 566}\special{sh 1}\special{ip}%
\special{pn 1}%
\special{pa   272   566}\special{pa   197   591}\special{pa   272   615}\special{pa   257   591}%
\special{pa   272   566}%
\special{fp}%
\special{pn 8}%
\special{pa   492   591}\special{pa   257   591}%
\special{fp}%
\settowidth{\Width}{$a$}\setlength{\Width}{-0.5\Width}%
\settoheight{\Height}{$a$}\settodepth{\Depth}{$a$}\setlength{\Height}{\Depth}%
\put( -1.250, -1.350){\hspace*{\Width}\raisebox{\Height}{$a$}}%
%
\settowidth{\Width}{$a$}\setlength{\Width}{-0.5\Width}%
\settoheight{\Height}{$a$}\settodepth{\Depth}{$a$}\setlength{\Height}{\Depth}%
\put(  1.250, -1.350){\hspace*{\Width}\raisebox{\Height}{$a$}}%
%
\special{pa 15 1575}\special{pa 15 1573}\special{pa 14 1571}\special{pa 14 1569}\special{pa 13 1568}%
\special{pa 12 1566}\special{pa 11 1565}\special{pa 10 1563}\special{pa 8 1562}\special{pa 6 1561}%
\special{pa 5 1561}\special{pa 3 1560}\special{pa 1 1560}\special{pa -1 1560}\special{pa -3 1560}%
\special{pa -5 1561}\special{pa -6 1561}\special{pa -8 1562}\special{pa -10 1563}%
\special{pa -11 1565}\special{pa -12 1566}\special{pa -13 1568}\special{pa -14 1569}%
\special{pa -14 1571}\special{pa -15 1573}\special{pa -15 1575}\special{pa -15 1577}%
\special{pa -14 1579}\special{pa -14 1580}\special{pa -13 1582}\special{pa -12 1584}%
\special{pa -11 1585}\special{pa -10 1586}\special{pa -8 1587}\special{pa -6 1588}%
\special{pa -5 1589}\special{pa -3 1589}\special{pa -1 1590}\special{pa 1 1590}\special{pa 3 1589}%
\special{pa 5 1589}\special{pa 6 1588}\special{pa 8 1587}\special{pa 10 1586}\special{pa 11 1585}%
\special{pa 12 1584}\special{pa 13 1582}\special{pa 14 1580}\special{pa 14 1579}\special{pa 15 1577}%
\special{pa 15 1575}\special{pa 15 1575}\special{sh 1}\special{ip}%
\special{pa    15  1575}\special{pa    15  1573}\special{pa    14  1571}\special{pa    14  1569}%
\special{pa    13  1568}\special{pa    12  1566}\special{pa    11  1565}\special{pa    10  1563}%
\special{pa     8  1562}\special{pa     6  1561}\special{pa     5  1561}\special{pa     3  1560}%
\special{pa     1  1560}\special{pa    -1  1560}\special{pa    -3  1560}\special{pa    -5  1561}%
\special{pa    -6  1561}\special{pa    -8  1562}\special{pa   -10  1563}\special{pa   -11  1565}%
\special{pa   -12  1566}\special{pa   -13  1568}\special{pa   -14  1569}\special{pa   -14  1571}%
\special{pa   -15  1573}\special{pa   -15  1575}\special{pa   -15  1577}\special{pa   -14  1579}%
\special{pa   -14  1580}\special{pa   -13  1582}\special{pa   -12  1584}\special{pa   -11  1585}%
\special{pa   -10  1586}\special{pa    -8  1587}\special{pa    -6  1588}\special{pa    -5  1589}%
\special{pa    -3  1589}\special{pa    -1  1590}\special{pa     1  1590}\special{pa     3  1589}%
\special{pa     5  1589}\special{pa     6  1588}\special{pa     8  1587}\special{pa    10  1586}%
\special{pa    11  1585}\special{pa    12  1584}\special{pa    13  1582}\special{pa    14  1580}%
\special{pa    14  1579}\special{pa    15  1577}\special{pa    15  1575}%
\special{fp}%
\special{pa     0  1575}\special{pa     0  1772}%
\special{fp}%
\special{pa   -98  1772}\special{pa    98  1772}%
\special{fp}%
\special{pa   -59  1811}\special{pa    59  1811}%
\special{fp}%
\special{pa   -20  1850}\special{pa    20  1850}%
\special{fp}%
\end{picture}}%
        %%% C:/vpn/vpn/KeTCindy/fig/fig111_2.tex 
%%% Generator=fig111_2.cdy 
{\unitlength=1cm%
\begin{picture}%
(6,9.5)(-3,-6)%
\special{pn 8}%
%
\special{pa     0  -984}\special{pa   394  -984}\special{pa   394   787}\special{pa     0   787}%
\special{pa     0  -984}%
\special{fp}%
\special{pn 32}%
\special{pa  -787  -984}\special{pa  -787   787}%
\special{fp}%
\special{pn 8}%
\special{pn 32}%
\special{pa   787  -984}\special{pa   787   787}%
\special{fp}%
\special{pn 8}%
\special{pa  -787   591}\special{pa  -984   591}\special{pa  -984  1575}\special{pa   984  1575}%
\special{pa   984   591}\special{pa   787   591}%
\special{fp}%
\special{pa   262   787}\special{pa   394   655}%
\special{fp}%
\special{pa   122   787}\special{pa   394   516}%
\special{fp}%
\special{pa     0   771}\special{pa   394   377}%
\special{fp}%
\special{pa     0   631}\special{pa   394   238}%
\special{fp}%
\special{pa     0   492}\special{pa   394    98}%
\special{fp}%
\special{pa     0   353}\special{pa   394   -41}%
\special{fp}%
\special{pa     0   214}\special{pa   394  -180}%
\special{fp}%
\special{pa     0    75}\special{pa   394  -319}%
\special{fp}%
\special{pa     0   -65}\special{pa   394  -458}%
\special{fp}%
\special{pa     0  -204}\special{pa   394  -598}%
\special{fp}%
\special{pa     0  -343}\special{pa   394  -737}%
\special{fp}%
\special{pa    -0  -482}\special{pa   394  -876}%
\special{fp}%
\special{pa     0  -621}\special{pa   363  -984}%
\special{fp}%
\special{pa     0  -761}\special{pa   224  -984}%
\special{fp}%
\special{pa     0  -900}\special{pa    84  -984}%
\special{fp}%
\special{pa    49  1083}\special{pa    36   863}%
\special{fp}%
\special{pa     0  1181}\special{pa    49  1083}%
\special{fp}%
\special{pa   197   787}\special{pa   148   886}%
\special{fp}%
{%
\color[cmyk]{0,0,0,0}%
\special{pa 64 1083}\special{pa 64 1081}\special{pa 64 1079}\special{pa 63 1077}\special{pa 62 1076}%
\special{pa 61 1074}\special{pa 60 1073}\special{pa 59 1071}\special{pa 57 1070}\special{pa 55 1069}%
\special{pa 54 1069}\special{pa 52 1068}\special{pa 50 1068}\special{pa 48 1068}\special{pa 46 1068}%
\special{pa 45 1069}\special{pa 43 1069}\special{pa 41 1070}\special{pa 40 1071}\special{pa 38 1073}%
\special{pa 37 1074}\special{pa 36 1076}\special{pa 35 1077}\special{pa 35 1079}\special{pa 34 1081}%
\special{pa 34 1083}\special{pa 34 1085}\special{pa 35 1087}\special{pa 35 1088}\special{pa 36 1090}%
\special{pa 37 1092}\special{pa 38 1093}\special{pa 40 1094}\special{pa 41 1095}\special{pa 43 1096}%
\special{pa 45 1097}\special{pa 46 1098}\special{pa 48 1098}\special{pa 50 1098}\special{pa 52 1098}%
\special{pa 54 1097}\special{pa 55 1096}\special{pa 57 1095}\special{pa 59 1094}\special{pa 60 1093}%
\special{pa 61 1092}\special{pa 62 1090}\special{pa 63 1088}\special{pa 64 1087}\special{pa 64 1085}%
\special{pa 64 1083}\special{pa 64 1083}\special{sh 1}\special{ip}%
}%
\special{pa    64  1083}\special{pa    64  1081}\special{pa    64  1079}\special{pa    63  1077}%
\special{pa    62  1076}\special{pa    61  1074}\special{pa    60  1073}\special{pa    59  1071}%
\special{pa    57  1070}\special{pa    55  1069}\special{pa    54  1069}\special{pa    52  1068}%
\special{pa    50  1068}\special{pa    48  1068}\special{pa    46  1068}\special{pa    45  1069}%
\special{pa    43  1069}\special{pa    41  1070}\special{pa    40  1071}\special{pa    38  1073}%
\special{pa    37  1074}\special{pa    36  1076}\special{pa    35  1077}\special{pa    35  1079}%
\special{pa    34  1081}\special{pa    34  1083}\special{pa    34  1085}\special{pa    35  1087}%
\special{pa    35  1088}\special{pa    36  1090}\special{pa    37  1092}\special{pa    38  1093}%
\special{pa    40  1094}\special{pa    41  1095}\special{pa    43  1096}\special{pa    45  1097}%
\special{pa    46  1098}\special{pa    48  1098}\special{pa    50  1098}\special{pa    52  1098}%
\special{pa    54  1097}\special{pa    55  1096}\special{pa    57  1095}\special{pa    59  1094}%
\special{pa    60  1093}\special{pa    61  1092}\special{pa    62  1090}\special{pa    63  1088}%
\special{pa    64  1087}\special{pa    64  1085}\special{pa    64  1083}%
\special{fp}%
{%
\color[cmyk]{0,0,0,0}%
\special{pa 163 886}\special{pa 163 884}\special{pa 162 882}\special{pa 162 880}\special{pa 161 878}%
\special{pa 160 877}\special{pa 159 875}\special{pa 157 874}\special{pa 156 873}\special{pa 154 872}%
\special{pa 152 871}\special{pa 151 871}\special{pa 149 871}\special{pa 147 871}\special{pa 145 871}%
\special{pa 143 871}\special{pa 141 872}\special{pa 140 873}\special{pa 138 874}\special{pa 137 875}%
\special{pa 136 877}\special{pa 135 878}\special{pa 134 880}\special{pa 133 882}\special{pa 133 884}%
\special{pa 133 886}\special{pa 133 888}\special{pa 133 889}\special{pa 134 891}\special{pa 135 893}%
\special{pa 136 894}\special{pa 137 896}\special{pa 138 897}\special{pa 140 898}\special{pa 141 899}%
\special{pa 143 900}\special{pa 145 900}\special{pa 147 901}\special{pa 149 901}\special{pa 151 900}%
\special{pa 152 900}\special{pa 154 899}\special{pa 156 898}\special{pa 157 897}\special{pa 159 896}%
\special{pa 160 894}\special{pa 161 893}\special{pa 162 891}\special{pa 162 889}\special{pa 163 888}%
\special{pa 163 886}\special{pa 163 886}\special{sh 1}\special{ip}%
}%
\special{pa   163   886}\special{pa   163   884}\special{pa   162   882}\special{pa   162   880}%
\special{pa   161   878}\special{pa   160   877}\special{pa   159   875}\special{pa   157   874}%
\special{pa   156   873}\special{pa   154   872}\special{pa   152   871}\special{pa   151   871}%
\special{pa   149   871}\special{pa   147   871}\special{pa   145   871}\special{pa   143   871}%
\special{pa   141   872}\special{pa   140   873}\special{pa   138   874}\special{pa   137   875}%
\special{pa   136   877}\special{pa   135   878}\special{pa   134   880}\special{pa   133   882}%
\special{pa   133   884}\special{pa   133   886}\special{pa   133   888}\special{pa   133   889}%
\special{pa   134   891}\special{pa   135   893}\special{pa   136   894}\special{pa   137   896}%
\special{pa   138   897}\special{pa   140   898}\special{pa   141   899}\special{pa   143   900}%
\special{pa   145   900}\special{pa   147   901}\special{pa   149   901}\special{pa   151   900}%
\special{pa   152   900}\special{pa   154   899}\special{pa   156   898}\special{pa   157   897}%
\special{pa   159   896}\special{pa   160   894}\special{pa   161   893}\special{pa   162   891}%
\special{pa   162   889}\special{pa   163   888}\special{pa   163   886}%
\special{fp}%
\special{pa  -138  1350}\special{pa   138  1350}%
\special{fp}%
\special{pn 16}%
\special{pa   -55  1406}\special{pa    55  1406}%
\special{fp}%
\special{pn 8}%
\special{pa     0  1575}\special{pa     0  1406}%
\special{fp}%
\special{pa     0  1181}\special{pa     0  1350}%
\special{fp}%
\settowidth{\Width}{図2}\setlength{\Width}{-0.5\Width}%
\settoheight{\Height}{図2}\settodepth{\Depth}{図2}\setlength{\Height}{\Depth}%
\put(  0.000, -5.300){\hspace*{\Width}\raisebox{\Height}{図2}}%
%
\settowidth{\Width}{$\mathrm{P_1}$}\setlength{\Width}{-0.5\Width}%
\settoheight{\Height}{$\mathrm{P_1}$}\settodepth{\Depth}{$\mathrm{P_1}$}\setlength{\Height}{\Depth}%
\put( -2.000,  2.700){\hspace*{\Width}\raisebox{\Height}{$\mathrm{P_1}$}}%
%
\settowidth{\Width}{$\mathrm{P_2}$}\setlength{\Width}{-0.5\Width}%
\settoheight{\Height}{$\mathrm{P_2}$}\settodepth{\Depth}{$\mathrm{P_2}$}\setlength{\Height}{\Depth}%
\put(  2.000,  2.700){\hspace*{\Width}\raisebox{\Height}{$\mathrm{P_2}$}}%
%
\settowidth{\Width}{$\mathrm{P_M}$}\setlength{\Width}{-0.5\Width}%
\settoheight{\Height}{$\mathrm{P_M}$}\settodepth{\Depth}{$\mathrm{P_M}$}\setlength{\Height}{\Depth}%
\put(  0.500,  2.700){\hspace*{\Width}\raisebox{\Height}{$\mathrm{P_M}$}}%
%
\settowidth{\Width}{S}\setlength{\Width}{-1\Width}%
\settoheight{\Height}{S}\settodepth{\Depth}{S}\setlength{\Height}{-0.5\Height}\setlength{\Depth}{0.5\Depth}\addtolength{\Height}{\Depth}%
\put(  0.000, -2.500){\hspace*{\Width}\raisebox{\Height}{S}}%
%
\settowidth{\Width}{$V_0$}\setlength{\Width}{0\Width}%
\settoheight{\Height}{$V_0$}\settodepth{\Depth}{$V_0$}\setlength{\Height}{-0.5\Height}\setlength{\Depth}{0.5\Depth}\addtolength{\Height}{\Depth}%
\put(  0.560, -3.500){\hspace*{\Width}\raisebox{\Height}{$V_0$}}%
%
\special{pa -713 566}\special{pa -787 591}\special{pa -713 615}\special{pa -727 591}%
\special{pa -713 566}\special{pa -713 566}\special{sh 1}\special{ip}%
\special{pn 1}%
\special{pa  -713   566}\special{pa  -787   591}\special{pa  -713   615}\special{pa  -727   591}%
\special{pa  -713   566}%
\special{fp}%
\special{pn 8}%
\special{pa  -394   591}\special{pa  -727   591}%
\special{fp}%
\special{pa -75 615}\special{pa 0 591}\special{pa -75 566}\special{pa -60 591}\special{pa -75 615}%
\special{pa -75 615}\special{sh 1}\special{ip}%
\special{pn 1}%
\special{pa   -75   615}\special{pa     0   591}\special{pa   -75   566}\special{pa   -60   591}%
\special{pa   -75   615}%
\special{fp}%
\special{pn 8}%
\special{pa  -394   591}\special{pa   -60   591}%
\special{fp}%
\special{pa 713 615}\special{pa 787 591}\special{pa 713 566}\special{pa 727 591}\special{pa 713 615}%
\special{pa 713 615}\special{sh 1}\special{ip}%
\special{pn 1}%
\special{pa   713   615}\special{pa   787   591}\special{pa   713   566}\special{pa   727   591}%
\special{pa   713   615}%
\special{fp}%
\special{pn 8}%
\special{pa   591   591}\special{pa   727   591}%
\special{fp}%
\special{pa 469 566}\special{pa 394 591}\special{pa 469 615}\special{pa 454 591}\special{pa 469 566}%
\special{pa 469 566}\special{sh 1}\special{ip}%
\special{pn 1}%
\special{pa   469   566}\special{pa   394   591}\special{pa   469   615}\special{pa   454   591}%
\special{pa   469   566}%
\special{fp}%
\special{pn 8}%
\special{pa   591   591}\special{pa   454   591}%
\special{fp}%
\settowidth{\Width}{$x$}\setlength{\Width}{-0.5\Width}%
\settoheight{\Height}{$x$}\settodepth{\Depth}{$x$}\setlength{\Height}{\Depth}%
\put( -1.000, -1.350){\hspace*{\Width}\raisebox{\Height}{$x$}}%
%
\settowidth{\Width}{$a-x$}\setlength{\Width}{-0.5\Width}%
\settoheight{\Height}{$a-x$}\settodepth{\Depth}{$a-x$}\setlength{\Height}{\Depth}%
\put(  1.500, -1.350){\hspace*{\Width}\raisebox{\Height}{$a-x$}}%
%
\special{pa 15 1575}\special{pa 15 1573}\special{pa 14 1571}\special{pa 14 1569}\special{pa 13 1568}%
\special{pa 12 1566}\special{pa 11 1565}\special{pa 10 1563}\special{pa 8 1562}\special{pa 6 1561}%
\special{pa 5 1561}\special{pa 3 1560}\special{pa 1 1560}\special{pa -1 1560}\special{pa -3 1560}%
\special{pa -5 1561}\special{pa -6 1561}\special{pa -8 1562}\special{pa -10 1563}%
\special{pa -11 1565}\special{pa -12 1566}\special{pa -13 1568}\special{pa -14 1569}%
\special{pa -14 1571}\special{pa -15 1573}\special{pa -15 1575}\special{pa -15 1577}%
\special{pa -14 1579}\special{pa -14 1580}\special{pa -13 1582}\special{pa -12 1584}%
\special{pa -11 1585}\special{pa -10 1586}\special{pa -8 1587}\special{pa -6 1588}%
\special{pa -5 1589}\special{pa -3 1589}\special{pa -1 1590}\special{pa 1 1590}\special{pa 3 1589}%
\special{pa 5 1589}\special{pa 6 1588}\special{pa 8 1587}\special{pa 10 1586}\special{pa 11 1585}%
\special{pa 12 1584}\special{pa 13 1582}\special{pa 14 1580}\special{pa 14 1579}\special{pa 15 1577}%
\special{pa 15 1575}\special{pa 15 1575}\special{sh 1}\special{ip}%
\special{pa    15  1575}\special{pa    15  1573}\special{pa    14  1571}\special{pa    14  1569}%
\special{pa    13  1568}\special{pa    12  1566}\special{pa    11  1565}\special{pa    10  1563}%
\special{pa     8  1562}\special{pa     6  1561}\special{pa     5  1561}\special{pa     3  1560}%
\special{pa     1  1560}\special{pa    -1  1560}\special{pa    -3  1560}\special{pa    -5  1561}%
\special{pa    -6  1561}\special{pa    -8  1562}\special{pa   -10  1563}\special{pa   -11  1565}%
\special{pa   -12  1566}\special{pa   -13  1568}\special{pa   -14  1569}\special{pa   -14  1571}%
\special{pa   -15  1573}\special{pa   -15  1575}\special{pa   -15  1577}\special{pa   -14  1579}%
\special{pa   -14  1580}\special{pa   -13  1582}\special{pa   -12  1584}\special{pa   -11  1585}%
\special{pa   -10  1586}\special{pa    -8  1587}\special{pa    -6  1588}\special{pa    -5  1589}%
\special{pa    -3  1589}\special{pa    -1  1590}\special{pa     1  1590}\special{pa     3  1589}%
\special{pa     5  1589}\special{pa     6  1588}\special{pa     8  1587}\special{pa    10  1586}%
\special{pa    11  1585}\special{pa    12  1584}\special{pa    13  1582}\special{pa    14  1580}%
\special{pa    14  1579}\special{pa    15  1577}\special{pa    15  1575}%
\special{fp}%
\special{pa     0  1575}\special{pa     0  1772}%
\special{fp}%
\special{pa   -98  1772}\special{pa    98  1772}%
\special{fp}%
\special{pa   -59  1811}\special{pa    59  1811}%
\special{fp}%
\special{pa   -20  1850}\special{pa    20  1850}%
\special{fp}%
\end{picture}}%
        }
        図1のように,面積の等しい3枚の導体板$\mathrm{P_1}$,$\mathrm{P_M}$,$\mathrm{P_2}$を平行に置き,
        間隔が共に$a$となるようにして固定する。$\mathrm{P_1}$と$\mathrm{P_2}$は接地し,スイッチS
        を閉じて$\mathrm{P_M}$の電位を$V_0$とする。このとき,$\mathrm{P_1}$の電荷$-Q_0$であった。
        \begin{Enumerate}
            \item $\mathrm{P_M}$全体の電荷を求めよ。
        \end{Enumerate}
        次に,スイッチを切り,$\mathrm{P_M}$を$\mathrm{P_2}$の方へ平行に$x(x<a)$だけ移動した(図2)。
        \begin{Enumerate*}
            \item $\mathrm{P_M}$の電位を求めよ。
            \item $\mathrm{P_1}$の電荷を求めよ。
        \end{Enumerate*}
    \end{mawarikomi}
