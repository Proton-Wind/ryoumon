\hakosyokika
\item 異なる位置にある2つの音源SおよびTを結ぶ直線状に観測者がいる。音源SおよびTから出る音の振動数200\sftanni{Hz},音速340\sftanni{m/s}であり,音源SおよびTは疎蜜で同位相の音を左右に送り出し,音は減衰しないものとする。
    \begin{enumerate}
        \item 音源SおよびTの右側にいた観測者が,音が強めあっていると観測できたとすれば,2つの音源の距離は最短で\Hako \tanni{m}である。
        \item 音源Sと音源Tとの間隔を$5.6$\sftanni{m}とし,SとTの間にいた観測者が音が強めあっていると観測できたとする。そのような位置は間隔が\Hako \tanni{m}をなしていくつかあるが,観測者がTにもっとも近い位置にいたとすれば,Tまでの距離は\Hako \tanni{m}である。そして,観測者がSに向かって$1.7$\tanni{m/s}の速さで歩くと,音の大きさが繰り返し変化して聞こえる。音が強めあっていると観測する回数は1秒当たり\Hako \tanni{回/s}である。
    \end{enumerate}