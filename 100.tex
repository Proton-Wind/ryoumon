\hakosyokika
\item
    \begin{mawarikomi}(20pt,0pt){160pt}{{\small
\begin{zahyou*}[ul=4mm](0,13)(0,8)
    \tenretu*{%
        A(4,2);
        B(1,2);
        E(4,7);
        F(4,6.5);
        G(8,6.5);
        H(8,7);
    }%
    \Landl\F\G\A{(1,1.732)}\D
    \Candk\D{4}\D{270}\P\Q
    \Nuritubusi*{\E\F\G\H\E}
    \Drawlines{\F\G;\A\D}
    \Hasen{\D\P}
    \Kakukigou<2>\A\D\P<Hankei=9mm>[s]{$30\Deg$}
    \Put\A(8pt,-5pt)[b]{$m$}
    \HenKo<henkotype=parallel,
    % henkoH=11ex,
    yazirusi=b,
    henkosideb=0,
    henkosidet=1.2>\B\A{$d$}
    \En*[0]\A{0.2}
    \En\A{0.2}
    \Put\A(-3pt,5pt)[b]{A}
    \En*[0]\B{0.2}
    \En\B{0.2}
    \Put\B(-3pt,5pt)[b]{B}


\end{zahyou*}}
}
        真空中で,質量$m$\tanni{kg}の小球Aに$-q$\tanni{C}の負電荷を与え,
        絶縁性の糸でつるす。今,正電荷$+2q$\tanni{C}をもつ小球BをAに水平方向から近づけると,BA間が$d$\tanni{m}のとき,
        糸は鉛直方向と$30\Deg $の角をなしてつり合った。重力加速度の大きさを$g$\tanni{m/s^2}とする。
        \begin{enumerate}
            \item 糸の張力は何\tanni{N}か。
            \item クーロンの法則の比例定数を$k$\tanni{N\cdot m^2/C^2}として,$q$を他の量で表せ。
            \item AとBは同じ金属球とする。A,B両球を接触させた後に,再び水平方向$d$\tanni{m}の距離に保つ。糸が鉛直方向となす角を$\theta $として,$\tan{\theta }$を数値で表せ。
        \end{enumerate}
    \end{mawarikomi}
