\hakosyokika
\item
    \begin{mawarikomi}(20pt,0pt){150pt}{
        %%% C:/vpn/vpn/KeTCindy/fig/fig108.tex 
%%% Generator=fig108.cdy 
{\unitlength=1cm%
\begin{picture}%
(5,4)(-2.5,-1.5)%
\special{pn 8}%
%
\special{pa  -157  -591}\special{pa   115  -748}%
\special{fp}%
\special{pa  -591  -591}\special{pa  -157  -591}%
\special{fp}%
\special{pa   591  -591}\special{pa   157  -591}%
\special{fp}%
{%
\color[cmyk]{0,0,0,0}%
\special{pa -143 -591}\special{pa -143 -592}\special{pa -143 -594}\special{pa -144 -596}%
\special{pa -144 -598}\special{pa -145 -599}\special{pa -147 -601}\special{pa -148 -602}%
\special{pa -149 -603}\special{pa -151 -604}\special{pa -153 -605}\special{pa -155 -605}%
\special{pa -157 -605}\special{pa -158 -605}\special{pa -160 -605}\special{pa -162 -605}%
\special{pa -164 -604}\special{pa -165 -603}\special{pa -167 -602}\special{pa -168 -601}%
\special{pa -170 -599}\special{pa -171 -598}\special{pa -171 -596}\special{pa -172 -594}%
\special{pa -172 -592}\special{pa -172 -591}\special{pa -172 -589}\special{pa -172 -587}%
\special{pa -171 -585}\special{pa -171 -583}\special{pa -170 -582}\special{pa -168 -580}%
\special{pa -167 -579}\special{pa -165 -578}\special{pa -164 -577}\special{pa -162 -576}%
\special{pa -160 -576}\special{pa -158 -576}\special{pa -157 -576}\special{pa -155 -576}%
\special{pa -153 -576}\special{pa -151 -577}\special{pa -149 -578}\special{pa -148 -579}%
\special{pa -147 -580}\special{pa -145 -582}\special{pa -144 -583}\special{pa -144 -585}%
\special{pa -143 -587}\special{pa -143 -589}\special{pa -143 -591}\special{pa -143 -591}%
\special{sh 1}\special{ip}%
}%
\special{pa  -143  -591}\special{pa  -143  -592}\special{pa  -143  -594}\special{pa  -144  -596}%
\special{pa  -144  -598}\special{pa  -145  -599}\special{pa  -147  -601}\special{pa  -148  -602}%
\special{pa  -149  -603}\special{pa  -151  -604}\special{pa  -153  -605}\special{pa  -155  -605}%
\special{pa  -157  -605}\special{pa  -158  -605}\special{pa  -160  -605}\special{pa  -162  -605}%
\special{pa  -164  -604}\special{pa  -165  -603}\special{pa  -167  -602}\special{pa  -168  -601}%
\special{pa  -170  -599}\special{pa  -171  -598}\special{pa  -171  -596}\special{pa  -172  -594}%
\special{pa  -172  -592}\special{pa  -172  -591}\special{pa  -172  -589}\special{pa  -172  -587}%
\special{pa  -171  -585}\special{pa  -171  -583}\special{pa  -170  -582}\special{pa  -168  -580}%
\special{pa  -167  -579}\special{pa  -165  -578}\special{pa  -164  -577}\special{pa  -162  -576}%
\special{pa  -160  -576}\special{pa  -158  -576}\special{pa  -157  -576}\special{pa  -155  -576}%
\special{pa  -153  -576}\special{pa  -151  -577}\special{pa  -149  -578}\special{pa  -148  -579}%
\special{pa  -147  -580}\special{pa  -145  -582}\special{pa  -144  -583}\special{pa  -144  -585}%
\special{pa  -143  -587}\special{pa  -143  -589}\special{pa  -143  -591}%
\special{fp}%
{%
\color[cmyk]{0,0,0,0}%
\special{pa 172 -591}\special{pa 172 -592}\special{pa 172 -594}\special{pa 171 -596}%
\special{pa 171 -598}\special{pa 170 -599}\special{pa 168 -601}\special{pa 167 -602}%
\special{pa 165 -603}\special{pa 164 -604}\special{pa 162 -605}\special{pa 160 -605}%
\special{pa 158 -605}\special{pa 157 -605}\special{pa 155 -605}\special{pa 153 -605}%
\special{pa 151 -604}\special{pa 149 -603}\special{pa 148 -602}\special{pa 147 -601}%
\special{pa 145 -599}\special{pa 144 -598}\special{pa 144 -596}\special{pa 143 -594}%
\special{pa 143 -592}\special{pa 143 -591}\special{pa 143 -589}\special{pa 143 -587}%
\special{pa 144 -585}\special{pa 144 -583}\special{pa 145 -582}\special{pa 147 -580}%
\special{pa 148 -579}\special{pa 149 -578}\special{pa 151 -577}\special{pa 153 -576}%
\special{pa 155 -576}\special{pa 157 -576}\special{pa 158 -576}\special{pa 160 -576}%
\special{pa 162 -576}\special{pa 164 -577}\special{pa 165 -578}\special{pa 167 -579}%
\special{pa 168 -580}\special{pa 170 -582}\special{pa 171 -583}\special{pa 171 -585}%
\special{pa 172 -587}\special{pa 172 -589}\special{pa 172 -591}\special{pa 172 -591}%
\special{sh 1}\special{ip}%
}%
\special{pa   172  -591}\special{pa   172  -592}\special{pa   172  -594}\special{pa   171  -596}%
\special{pa   171  -598}\special{pa   170  -599}\special{pa   168  -601}\special{pa   167  -602}%
\special{pa   165  -603}\special{pa   164  -604}\special{pa   162  -605}\special{pa   160  -605}%
\special{pa   158  -605}\special{pa   157  -605}\special{pa   155  -605}\special{pa   153  -605}%
\special{pa   151  -604}\special{pa   149  -603}\special{pa   148  -602}\special{pa   147  -601}%
\special{pa   145  -599}\special{pa   144  -598}\special{pa   144  -596}\special{pa   143  -594}%
\special{pa   143  -592}\special{pa   143  -591}\special{pa   143  -589}\special{pa   143  -587}%
\special{pa   144  -585}\special{pa   144  -583}\special{pa   145  -582}\special{pa   147  -580}%
\special{pa   148  -579}\special{pa   149  -578}\special{pa   151  -577}\special{pa   153  -576}%
\special{pa   155  -576}\special{pa   157  -576}\special{pa   158  -576}\special{pa   160  -576}%
\special{pa   162  -576}\special{pa   164  -577}\special{pa   165  -578}\special{pa   167  -579}%
\special{pa   168  -580}\special{pa   170  -582}\special{pa   171  -583}\special{pa   171  -585}%
\special{pa   172  -587}\special{pa   172  -589}\special{pa   172  -591}%
\special{fp}%
\special{pa   787  -138}\special{pa   394  -138}%
\special{fp}%
\special{pa   787  -256}\special{pa   394  -256}%
\special{fp}%
\special{pa   591  -591}\special{pa   591  -256}%
\special{fp}%
\special{pa   591   197}\special{pa   591  -138}%
\special{fp}%
\special{pa  -197   295}\special{pa   197   295}\special{pa   197    98}\special{pa  -197    98}%
\special{pa  -197   295}%
\special{fp}%
\special{pa   591   197}\special{pa   197   197}%
\special{fp}%
\special{pa  -591   197}\special{pa  -197   197}%
\special{fp}%
\special{pa  -728  -224}\special{pa  -453  -224}%
\special{fp}%
\special{pn 16}%
\special{pa  -646  -169}\special{pa  -535  -169}%
\special{fp}%
\special{pn 8}%
\special{pa  -591   197}\special{pa  -591  -169}%
\special{fp}%
\special{pa  -591  -591}\special{pa  -591  -224}%
\special{fp}%
\special{pa -576 197}\special{pa -576 195}\special{pa -576 193}\special{pa -577 191}%
\special{pa -577 190}\special{pa -578 188}\special{pa -580 187}\special{pa -581 185}%
\special{pa -583 184}\special{pa -584 183}\special{pa -586 183}\special{pa -588 182}%
\special{pa -590 182}\special{pa -591 182}\special{pa -593 182}\special{pa -595 183}%
\special{pa -597 183}\special{pa -599 184}\special{pa -600 185}\special{pa -601 187}%
\special{pa -603 188}\special{pa -604 190}\special{pa -604 191}\special{pa -605 193}%
\special{pa -605 195}\special{pa -606 197}\special{pa -605 199}\special{pa -605 201}%
\special{pa -604 202}\special{pa -604 204}\special{pa -603 206}\special{pa -601 207}%
\special{pa -600 208}\special{pa -599 209}\special{pa -597 210}\special{pa -595 211}%
\special{pa -593 212}\special{pa -591 212}\special{pa -590 212}\special{pa -588 212}%
\special{pa -586 211}\special{pa -584 210}\special{pa -583 209}\special{pa -581 208}%
\special{pa -580 207}\special{pa -578 206}\special{pa -577 204}\special{pa -577 202}%
\special{pa -576 201}\special{pa -576 199}\special{pa -576 197}\special{pa -576 197}%
\special{sh 1}\special{ip}%
\special{pa  -576   197}\special{pa  -576   195}\special{pa  -576   193}\special{pa  -577   191}%
\special{pa  -577   190}\special{pa  -578   188}\special{pa  -580   187}\special{pa  -581   185}%
\special{pa  -583   184}\special{pa  -584   183}\special{pa  -586   183}\special{pa  -588   182}%
\special{pa  -590   182}\special{pa  -591   182}\special{pa  -593   182}\special{pa  -595   183}%
\special{pa  -597   183}\special{pa  -599   184}\special{pa  -600   185}\special{pa  -601   187}%
\special{pa  -603   188}\special{pa  -604   190}\special{pa  -604   191}\special{pa  -605   193}%
\special{pa  -605   195}\special{pa  -606   197}\special{pa  -605   199}\special{pa  -605   201}%
\special{pa  -604   202}\special{pa  -604   204}\special{pa  -603   206}\special{pa  -601   207}%
\special{pa  -600   208}\special{pa  -599   209}\special{pa  -597   210}\special{pa  -595   211}%
\special{pa  -593   212}\special{pa  -591   212}\special{pa  -590   212}\special{pa  -588   212}%
\special{pa  -586   211}\special{pa  -584   210}\special{pa  -583   209}\special{pa  -581   208}%
\special{pa  -580   207}\special{pa  -578   206}\special{pa  -577   204}\special{pa  -577   202}%
\special{pa  -576   201}\special{pa  -576   199}\special{pa  -576   197}%
\special{fp}%
\special{pa  -591   197}\special{pa  -591   394}%
\special{fp}%
\special{pa  -689   394}\special{pa  -492   394}%
\special{fp}%
\special{pa  -650   433}\special{pa  -531   433}%
\special{fp}%
\special{pa  -610   472}\special{pa  -571   472}%
\special{fp}%
\special{pa 598 -591}\special{pa 598 -591}\special{pa 598 -592}\special{pa 598 -593}%
\special{pa 597 -594}\special{pa 597 -595}\special{pa 596 -596}\special{pa 595 -596}%
\special{pa 595 -597}\special{pa 594 -597}\special{pa 593 -598}\special{pa 592 -598}%
\special{pa 591 -598}\special{pa 590 -598}\special{pa 589 -598}\special{pa 588 -598}%
\special{pa 587 -597}\special{pa 587 -597}\special{pa 586 -596}\special{pa 585 -596}%
\special{pa 584 -595}\special{pa 584 -594}\special{pa 584 -593}\special{pa 583 -592}%
\special{pa 583 -591}\special{pa 583 -591}\special{pa 583 -590}\special{pa 583 -589}%
\special{pa 584 -588}\special{pa 584 -587}\special{pa 584 -586}\special{pa 585 -585}%
\special{pa 586 -585}\special{pa 587 -584}\special{pa 587 -584}\special{pa 588 -583}%
\special{pa 589 -583}\special{pa 590 -583}\special{pa 591 -583}\special{pa 592 -583}%
\special{pa 593 -583}\special{pa 594 -584}\special{pa 595 -584}\special{pa 595 -585}%
\special{pa 596 -585}\special{pa 597 -586}\special{pa 597 -587}\special{pa 598 -588}%
\special{pa 598 -589}\special{pa 598 -590}\special{pa 598 -591}\special{pa 598 -591}%
\special{sh 1}\special{ip}%
\special{pa   598  -591}\special{pa   598  -591}\special{pa   598  -592}\special{pa   598  -593}%
\special{pa   597  -594}\special{pa   597  -595}\special{pa   596  -596}\special{pa   595  -596}%
\special{pa   595  -597}\special{pa   594  -597}\special{pa   593  -598}\special{pa   592  -598}%
\special{pa   591  -598}\special{pa   590  -598}\special{pa   589  -598}\special{pa   588  -598}%
\special{pa   587  -597}\special{pa   587  -597}\special{pa   586  -596}\special{pa   585  -596}%
\special{pa   584  -595}\special{pa   584  -594}\special{pa   584  -593}\special{pa   583  -592}%
\special{pa   583  -591}\special{pa   583  -591}\special{pa   583  -590}\special{pa   583  -589}%
\special{pa   584  -588}\special{pa   584  -587}\special{pa   584  -586}\special{pa   585  -585}%
\special{pa   586  -585}\special{pa   587  -584}\special{pa   587  -584}\special{pa   588  -583}%
\special{pa   589  -583}\special{pa   590  -583}\special{pa   591  -583}\special{pa   592  -583}%
\special{pa   593  -583}\special{pa   594  -584}\special{pa   595  -584}\special{pa   595  -585}%
\special{pa   596  -585}\special{pa   597  -586}\special{pa   597  -587}\special{pa   598  -588}%
\special{pa   598  -589}\special{pa   598  -590}\special{pa   598  -591}%
\special{fp}%
\settowidth{\Width}{A}\setlength{\Width}{0\Width}%
\settoheight{\Height}{A}\settodepth{\Depth}{A}\setlength{\Height}{\Depth}%
\put(  1.600,  1.650){\hspace*{\Width}\raisebox{\Height}{A}}%
%
\settowidth{\Width}{G}\setlength{\Width}{-1\Width}%
\settoheight{\Height}{G}\settodepth{\Depth}{G}\setlength{\Height}{-\Height}%
\put( -1.600, -0.650){\hspace*{\Width}\raisebox{\Height}{G}}%
%
\settowidth{\Width}{S}\setlength{\Width}{-1\Width}%
\settoheight{\Height}{S}\settodepth{\Depth}{S}\setlength{\Height}{\Depth}%
\put( -0.300,  1.700){\hspace*{\Width}\raisebox{\Height}{S}}%
%
\settowidth{\Width}{C}\setlength{\Width}{0\Width}%
\settoheight{\Height}{C}\settodepth{\Depth}{C}\setlength{\Height}{\Depth}%
\put(  2.150,  0.650){\hspace*{\Width}\raisebox{\Height}{C}}%
%
\settowidth{\Width}{R}\setlength{\Width}{-0.5\Width}%
\settoheight{\Height}{R}\settodepth{\Depth}{R}\setlength{\Height}{\Depth}%
\put(  0.000, -0.100){\hspace*{\Width}\raisebox{\Height}{R}}%
%
\settowidth{\Width}{E}\setlength{\Width}{-1\Width}%
\settoheight{\Height}{E}\settodepth{\Depth}{E}\setlength{\Height}{\Depth}%
\put( -1.940,  0.650){\hspace*{\Width}\raisebox{\Height}{E}}%
%
\end{picture}}%
        }
        起電力が$V$で内部抵抗の無視できる電池E,電気容量が$C$の平行板コンデンサーC,
        抵抗値$R$の抵抗R,およびスイッチSを接続した回路がある。G点は接地されており,
        その電位は0である。はじめSは開いており,コンデンサーには電荷は蓄えられていない。
        \begin{enumerate}[(a)]
            \item まず,Sを閉じ,Cを充電する。Sを閉じた瞬間に抵抗$R$を流れる電流は\Hako である。
            \item Sを閉じてから十分時間が経ったとき,Cに蓄えられている静電エネルギーは\Hako である。また,この充電の過程で電池がした仕事は\Hako であり,抵抗Rで発生したジュール熱は\Hako である。
            \item 次に(b)の状態からSを開いた。最初Cの極板間隔は$d$であったが,極板を平行に保ったまま,ゆっくりと$2d$に広げた。このとき,A点の電位は\Hako である。また,極板を広げるのに必要な仕事は\Hako であり,極板間に働く静電気力の大きさ(一定と考えて良い)は\Hako と表される。
        \end{enumerate}
    \end{mawarikomi}
