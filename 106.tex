\hakosyokika
\item
    \begin{mawarikomi}(20pt,0pt){150pt}{
        %%% C:/vpn/vpn/KeTCindy/fig/fig106_1.tex 
%%% Generator=fig106_1.cdy 
{\unitlength=1cm%
\begin{picture}%
(4.5,4)(-3,-2)%
\special{pn 8}%
%
\special{pa     0  -394}\special{pa   394  -394}\special{pa   394  -354}\special{pa  -394  -354}%
\special{pa  -394  -394}\special{pa     0  -394}%
\special{fp}%
\special{pa     0   197}\special{pa   394   197}\special{pa   394   157}\special{pa  -394   157}%
\special{pa  -394   197}\special{pa     0   197}%
\special{fp}%
\special{pa -772 394}\special{pa -773 392}\special{pa -773 390}\special{pa -773 388}%
\special{pa -774 386}\special{pa -775 385}\special{pa -776 383}\special{pa -778 382}%
\special{pa -779 381}\special{pa -781 380}\special{pa -783 379}\special{pa -785 379}%
\special{pa -786 379}\special{pa -788 379}\special{pa -790 379}\special{pa -792 379}%
\special{pa -794 380}\special{pa -795 381}\special{pa -797 382}\special{pa -798 383}%
\special{pa -800 385}\special{pa -801 386}\special{pa -801 388}\special{pa -802 390}%
\special{pa -802 392}\special{pa -802 394}\special{pa -802 396}\special{pa -802 397}%
\special{pa -801 399}\special{pa -801 401}\special{pa -800 402}\special{pa -798 404}%
\special{pa -797 405}\special{pa -795 406}\special{pa -794 407}\special{pa -792 408}%
\special{pa -790 408}\special{pa -788 409}\special{pa -786 409}\special{pa -785 408}%
\special{pa -783 408}\special{pa -781 407}\special{pa -779 406}\special{pa -778 405}%
\special{pa -776 404}\special{pa -775 402}\special{pa -774 401}\special{pa -773 399}%
\special{pa -773 397}\special{pa -773 396}\special{pa -772 394}\special{pa -772 394}%
\special{sh 1}\special{ip}%
\special{pa  -772   394}\special{pa  -773   392}\special{pa  -773   390}\special{pa  -773   388}%
\special{pa  -774   386}\special{pa  -775   385}\special{pa  -776   383}\special{pa  -778   382}%
\special{pa  -779   381}\special{pa  -781   380}\special{pa  -783   379}\special{pa  -785   379}%
\special{pa  -786   379}\special{pa  -788   379}\special{pa  -790   379}\special{pa  -792   379}%
\special{pa  -794   380}\special{pa  -795   381}\special{pa  -797   382}\special{pa  -798   383}%
\special{pa  -800   385}\special{pa  -801   386}\special{pa  -801   388}\special{pa  -802   390}%
\special{pa  -802   392}\special{pa  -802   394}\special{pa  -802   396}\special{pa  -802   397}%
\special{pa  -801   399}\special{pa  -801   401}\special{pa  -800   402}\special{pa  -798   404}%
\special{pa  -797   405}\special{pa  -795   406}\special{pa  -794   407}\special{pa  -792   408}%
\special{pa  -790   408}\special{pa  -788   409}\special{pa  -786   409}\special{pa  -785   408}%
\special{pa  -783   408}\special{pa  -781   407}\special{pa  -779   406}\special{pa  -778   405}%
\special{pa  -776   404}\special{pa  -775   402}\special{pa  -774   401}\special{pa  -773   399}%
\special{pa  -773   397}\special{pa  -773   396}\special{pa  -772   394}%
\special{fp}%
\special{pa  -787   394}\special{pa  -787   591}%
\special{fp}%
\special{pa  -886   591}\special{pa  -689   591}%
\special{fp}%
\special{pa  -846   630}\special{pa  -728   630}%
\special{fp}%
\special{pa  -807   669}\special{pa  -768   669}%
\special{fp}%
\special{pa  -906  -122}\special{pa  -669  -122}%
\special{fp}%
\special{pn 16}%
\special{pa  -835   -75}\special{pa  -740   -75}%
\special{fp}%
\special{pn 8}%
\special{pa  -787   394}\special{pa  -787   -75}%
\special{fp}%
\special{pa  -787  -591}\special{pa  -787  -122}%
\special{fp}%
\special{pa     0  -591}\special{pa     0  -394}%
\special{fp}%
\special{pa     0   197}\special{pa     0   394}\special{pa  -787   394}%
\special{fp}%
\settowidth{\Width}{A}\setlength{\Width}{0\Width}%
\settoheight{\Height}{A}\settodepth{\Depth}{A}\setlength{\Height}{-0.5\Height}\setlength{\Depth}{0.5\Depth}\addtolength{\Height}{\Depth}%
\put(  1.220,  1.000){\hspace*{\Width}\raisebox{\Height}{A}}%
%
\settowidth{\Width}{B}\setlength{\Width}{0\Width}%
\settoheight{\Height}{B}\settodepth{\Depth}{B}\setlength{\Height}{-0.5\Height}\setlength{\Depth}{0.5\Depth}\addtolength{\Height}{\Depth}%
\put(  1.220, -0.500){\hspace*{\Width}\raisebox{\Height}{B}}%
%
\settowidth{\Width}{図1}\setlength{\Width}{0\Width}%
\settoheight{\Height}{図1}\settodepth{\Depth}{図1}\setlength{\Height}{-0.5\Height}\setlength{\Depth}{0.5\Depth}\addtolength{\Height}{\Depth}%
\put(  1.050, -1.500){\hspace*{\Width}\raisebox{\Height}{図1}}%
%
\special{pa  -504  -591}\special{pa  -313  -701}%
\special{fp}%
\special{pa  -787  -591}\special{pa  -504  -591}%
\special{fp}%
\special{pa     0  -591}\special{pa  -283  -591}%
\special{fp}%
{%
\color[cmyk]{0,0,0,0}%
\special{pa -489 -591}\special{pa -489 -592}\special{pa -489 -594}\special{pa -490 -596}%
\special{pa -491 -598}\special{pa -492 -599}\special{pa -493 -601}\special{pa -494 -602}%
\special{pa -496 -603}\special{pa -498 -604}\special{pa -499 -605}\special{pa -501 -605}%
\special{pa -503 -605}\special{pa -505 -605}\special{pa -507 -605}\special{pa -509 -605}%
\special{pa -510 -604}\special{pa -512 -603}\special{pa -513 -602}\special{pa -515 -601}%
\special{pa -516 -599}\special{pa -517 -598}\special{pa -518 -596}\special{pa -518 -594}%
\special{pa -519 -592}\special{pa -519 -591}\special{pa -519 -589}\special{pa -518 -587}%
\special{pa -518 -585}\special{pa -517 -583}\special{pa -516 -582}\special{pa -515 -580}%
\special{pa -513 -579}\special{pa -512 -578}\special{pa -510 -577}\special{pa -509 -576}%
\special{pa -507 -576}\special{pa -505 -576}\special{pa -503 -576}\special{pa -501 -576}%
\special{pa -499 -576}\special{pa -498 -577}\special{pa -496 -578}\special{pa -494 -579}%
\special{pa -493 -580}\special{pa -492 -582}\special{pa -491 -583}\special{pa -490 -585}%
\special{pa -489 -587}\special{pa -489 -589}\special{pa -489 -591}\special{pa -489 -591}%
\special{sh 1}\special{ip}%
}%
\special{pa  -489  -591}\special{pa  -489  -592}\special{pa  -489  -594}\special{pa  -490  -596}%
\special{pa  -491  -598}\special{pa  -492  -599}\special{pa  -493  -601}\special{pa  -494  -602}%
\special{pa  -496  -603}\special{pa  -498  -604}\special{pa  -499  -605}\special{pa  -501  -605}%
\special{pa  -503  -605}\special{pa  -505  -605}\special{pa  -507  -605}\special{pa  -509  -605}%
\special{pa  -510  -604}\special{pa  -512  -603}\special{pa  -513  -602}\special{pa  -515  -601}%
\special{pa  -516  -599}\special{pa  -517  -598}\special{pa  -518  -596}\special{pa  -518  -594}%
\special{pa  -519  -592}\special{pa  -519  -591}\special{pa  -519  -589}\special{pa  -518  -587}%
\special{pa  -518  -585}\special{pa  -517  -583}\special{pa  -516  -582}\special{pa  -515  -580}%
\special{pa  -513  -579}\special{pa  -512  -578}\special{pa  -510  -577}\special{pa  -509  -576}%
\special{pa  -507  -576}\special{pa  -505  -576}\special{pa  -503  -576}\special{pa  -501  -576}%
\special{pa  -499  -576}\special{pa  -498  -577}\special{pa  -496  -578}\special{pa  -494  -579}%
\special{pa  -493  -580}\special{pa  -492  -582}\special{pa  -491  -583}\special{pa  -490  -585}%
\special{pa  -489  -587}\special{pa  -489  -589}\special{pa  -489  -591}%
\special{fp}%
{%
\color[cmyk]{0,0,0,0}%
\special{pa -269 -591}\special{pa -269 -592}\special{pa -269 -594}\special{pa -270 -596}%
\special{pa -270 -598}\special{pa -271 -599}\special{pa -273 -601}\special{pa -274 -602}%
\special{pa -275 -603}\special{pa -277 -604}\special{pa -279 -605}\special{pa -281 -605}%
\special{pa -283 -605}\special{pa -284 -605}\special{pa -286 -605}\special{pa -288 -605}%
\special{pa -290 -604}\special{pa -291 -603}\special{pa -293 -602}\special{pa -294 -601}%
\special{pa -296 -599}\special{pa -297 -598}\special{pa -297 -596}\special{pa -298 -594}%
\special{pa -298 -592}\special{pa -298 -591}\special{pa -298 -589}\special{pa -298 -587}%
\special{pa -297 -585}\special{pa -297 -583}\special{pa -296 -582}\special{pa -294 -580}%
\special{pa -293 -579}\special{pa -291 -578}\special{pa -290 -577}\special{pa -288 -576}%
\special{pa -286 -576}\special{pa -284 -576}\special{pa -283 -576}\special{pa -281 -576}%
\special{pa -279 -576}\special{pa -277 -577}\special{pa -275 -578}\special{pa -274 -579}%
\special{pa -273 -580}\special{pa -271 -582}\special{pa -270 -583}\special{pa -270 -585}%
\special{pa -269 -587}\special{pa -269 -589}\special{pa -269 -591}\special{pa -269 -591}%
\special{sh 1}\special{ip}%
}%
\special{pa  -269  -591}\special{pa  -269  -592}\special{pa  -269  -594}\special{pa  -270  -596}%
\special{pa  -270  -598}\special{pa  -271  -599}\special{pa  -273  -601}\special{pa  -274  -602}%
\special{pa  -275  -603}\special{pa  -277  -604}\special{pa  -279  -605}\special{pa  -281  -605}%
\special{pa  -283  -605}\special{pa  -284  -605}\special{pa  -286  -605}\special{pa  -288  -605}%
\special{pa  -290  -604}\special{pa  -291  -603}\special{pa  -293  -602}\special{pa  -294  -601}%
\special{pa  -296  -599}\special{pa  -297  -598}\special{pa  -297  -596}\special{pa  -298  -594}%
\special{pa  -298  -592}\special{pa  -298  -591}\special{pa  -298  -589}\special{pa  -298  -587}%
\special{pa  -297  -585}\special{pa  -297  -583}\special{pa  -296  -582}\special{pa  -294  -580}%
\special{pa  -293  -579}\special{pa  -291  -578}\special{pa  -290  -577}\special{pa  -288  -576}%
\special{pa  -286  -576}\special{pa  -284  -576}\special{pa  -283  -576}\special{pa  -281  -576}%
\special{pa  -279  -576}\special{pa  -277  -577}\special{pa  -275  -578}\special{pa  -274  -579}%
\special{pa  -273  -580}\special{pa  -271  -582}\special{pa  -270  -583}\special{pa  -270  -585}%
\special{pa  -269  -587}\special{pa  -269  -589}\special{pa  -269  -591}%
\special{fp}%
\settowidth{\Width}{$V$}\setlength{\Width}{-1\Width}%
\settoheight{\Height}{$V$}\settodepth{\Depth}{$V$}\setlength{\Height}{-0.5\Height}\setlength{\Depth}{0.5\Depth}\addtolength{\Height}{\Depth}%
\put( -2.470,  0.250){\hspace*{\Width}\raisebox{\Height}{$V$}}%
%
\settowidth{\Width}{S}\setlength{\Width}{-1\Width}%
\settoheight{\Height}{S}\settodepth{\Depth}{S}\setlength{\Height}{\Depth}%
\put( -1.050,  1.750){\hspace*{\Width}\raisebox{\Height}{S}}%
%
\end{picture}}%
        %%% C:/vpn/vpn/KeTCindy/fig/fig106_2.tex 
%%% Generator=fig106_2.cdy 
{\unitlength=1cm%
\begin{picture}%
(4.5,4)(-3,-2)%
\special{pn 8}%
%
\special{pa     0  -394}\special{pa   394  -394}\special{pa   394  -354}\special{pa  -394  -354}%
\special{pa  -394  -394}\special{pa     0  -394}%
\special{fp}%
\special{pa     0   197}\special{pa   394   197}\special{pa   394   157}\special{pa  -394   157}%
\special{pa  -394   197}\special{pa     0   197}%
\special{fp}%
\special{pa   382  -216}\special{pa   382    20}\special{pa  -405    20}\special{pa  -405  -216}%
\special{pa   382  -216}%
\special{fp}%
\special{pa -772 394}\special{pa -773 392}\special{pa -773 390}\special{pa -773 388}%
\special{pa -774 386}\special{pa -775 385}\special{pa -776 383}\special{pa -778 382}%
\special{pa -779 381}\special{pa -781 380}\special{pa -783 379}\special{pa -785 379}%
\special{pa -786 379}\special{pa -788 379}\special{pa -790 379}\special{pa -792 379}%
\special{pa -794 380}\special{pa -795 381}\special{pa -797 382}\special{pa -798 383}%
\special{pa -800 385}\special{pa -801 386}\special{pa -801 388}\special{pa -802 390}%
\special{pa -802 392}\special{pa -802 394}\special{pa -802 396}\special{pa -802 397}%
\special{pa -801 399}\special{pa -801 401}\special{pa -800 402}\special{pa -798 404}%
\special{pa -797 405}\special{pa -795 406}\special{pa -794 407}\special{pa -792 408}%
\special{pa -790 408}\special{pa -788 409}\special{pa -786 409}\special{pa -785 408}%
\special{pa -783 408}\special{pa -781 407}\special{pa -779 406}\special{pa -778 405}%
\special{pa -776 404}\special{pa -775 402}\special{pa -774 401}\special{pa -773 399}%
\special{pa -773 397}\special{pa -773 396}\special{pa -772 394}\special{pa -772 394}%
\special{sh 1}\special{ip}%
\special{pa  -772   394}\special{pa  -773   392}\special{pa  -773   390}\special{pa  -773   388}%
\special{pa  -774   386}\special{pa  -775   385}\special{pa  -776   383}\special{pa  -778   382}%
\special{pa  -779   381}\special{pa  -781   380}\special{pa  -783   379}\special{pa  -785   379}%
\special{pa  -786   379}\special{pa  -788   379}\special{pa  -790   379}\special{pa  -792   379}%
\special{pa  -794   380}\special{pa  -795   381}\special{pa  -797   382}\special{pa  -798   383}%
\special{pa  -800   385}\special{pa  -801   386}\special{pa  -801   388}\special{pa  -802   390}%
\special{pa  -802   392}\special{pa  -802   394}\special{pa  -802   396}\special{pa  -802   397}%
\special{pa  -801   399}\special{pa  -801   401}\special{pa  -800   402}\special{pa  -798   404}%
\special{pa  -797   405}\special{pa  -795   406}\special{pa  -794   407}\special{pa  -792   408}%
\special{pa  -790   408}\special{pa  -788   409}\special{pa  -786   409}\special{pa  -785   408}%
\special{pa  -783   408}\special{pa  -781   407}\special{pa  -779   406}\special{pa  -778   405}%
\special{pa  -776   404}\special{pa  -775   402}\special{pa  -774   401}\special{pa  -773   399}%
\special{pa  -773   397}\special{pa  -773   396}\special{pa  -772   394}%
\special{fp}%
\special{pa  -787   394}\special{pa  -787   591}%
\special{fp}%
\special{pa  -886   591}\special{pa  -689   591}%
\special{fp}%
\special{pa  -846   630}\special{pa  -728   630}%
\special{fp}%
\special{pa  -807   669}\special{pa  -768   669}%
\special{fp}%
\special{pa  -906  -122}\special{pa  -669  -122}%
\special{fp}%
\special{pn 16}%
\special{pa  -835   -75}\special{pa  -740   -75}%
\special{fp}%
\special{pn 8}%
\special{pa  -787   394}\special{pa  -787   -75}%
\special{fp}%
\special{pa  -787  -591}\special{pa  -787  -122}%
\special{fp}%
\special{pa     0  -591}\special{pa     0  -394}%
\special{fp}%
\special{pa     0   197}\special{pa     0   394}\special{pa  -787   394}%
\special{fp}%
\settowidth{\Width}{A}\setlength{\Width}{0\Width}%
\settoheight{\Height}{A}\settodepth{\Depth}{A}\setlength{\Height}{-0.5\Height}\setlength{\Depth}{0.5\Depth}\addtolength{\Height}{\Depth}%
\put(  1.220,  1.000){\hspace*{\Width}\raisebox{\Height}{A}}%
%
\settowidth{\Width}{P}\setlength{\Width}{0\Width}%
\settoheight{\Height}{P}\settodepth{\Depth}{P}\setlength{\Height}{-0.5\Height}\setlength{\Depth}{0.5\Depth}\addtolength{\Height}{\Depth}%
\put(  1.220,  0.250){\hspace*{\Width}\raisebox{\Height}{P}}%
%
\settowidth{\Width}{B}\setlength{\Width}{0\Width}%
\settoheight{\Height}{B}\settodepth{\Depth}{B}\setlength{\Height}{-0.5\Height}\setlength{\Depth}{0.5\Depth}\addtolength{\Height}{\Depth}%
\put(  1.220, -0.500){\hspace*{\Width}\raisebox{\Height}{B}}%
%
\settowidth{\Width}{図2}\setlength{\Width}{0\Width}%
\settoheight{\Height}{図2}\settodepth{\Depth}{図2}\setlength{\Height}{-0.5\Height}\setlength{\Depth}{0.5\Depth}\addtolength{\Height}{\Depth}%
\put(  1.050, -1.500){\hspace*{\Width}\raisebox{\Height}{図2}}%
%
\special{pa  -504  -591}\special{pa  -284  -604}%
\special{fp}%
\special{pa  -787  -591}\special{pa  -504  -591}%
\special{fp}%
\special{pa     0  -591}\special{pa  -283  -591}%
\special{fp}%
{%
\color[cmyk]{0,0,0,0}%
\special{pa -489 -591}\special{pa -489 -592}\special{pa -489 -594}\special{pa -490 -596}%
\special{pa -491 -598}\special{pa -492 -599}\special{pa -493 -601}\special{pa -494 -602}%
\special{pa -496 -603}\special{pa -498 -604}\special{pa -499 -605}\special{pa -501 -605}%
\special{pa -503 -605}\special{pa -505 -605}\special{pa -507 -605}\special{pa -509 -605}%
\special{pa -510 -604}\special{pa -512 -603}\special{pa -513 -602}\special{pa -515 -601}%
\special{pa -516 -599}\special{pa -517 -598}\special{pa -518 -596}\special{pa -518 -594}%
\special{pa -519 -592}\special{pa -519 -591}\special{pa -519 -589}\special{pa -518 -587}%
\special{pa -518 -585}\special{pa -517 -583}\special{pa -516 -582}\special{pa -515 -580}%
\special{pa -513 -579}\special{pa -512 -578}\special{pa -510 -577}\special{pa -509 -576}%
\special{pa -507 -576}\special{pa -505 -576}\special{pa -503 -576}\special{pa -501 -576}%
\special{pa -499 -576}\special{pa -498 -577}\special{pa -496 -578}\special{pa -494 -579}%
\special{pa -493 -580}\special{pa -492 -582}\special{pa -491 -583}\special{pa -490 -585}%
\special{pa -489 -587}\special{pa -489 -589}\special{pa -489 -591}\special{pa -489 -591}%
\special{sh 1}\special{ip}%
}%
\special{pa  -489  -591}\special{pa  -489  -592}\special{pa  -489  -594}\special{pa  -490  -596}%
\special{pa  -491  -598}\special{pa  -492  -599}\special{pa  -493  -601}\special{pa  -494  -602}%
\special{pa  -496  -603}\special{pa  -498  -604}\special{pa  -499  -605}\special{pa  -501  -605}%
\special{pa  -503  -605}\special{pa  -505  -605}\special{pa  -507  -605}\special{pa  -509  -605}%
\special{pa  -510  -604}\special{pa  -512  -603}\special{pa  -513  -602}\special{pa  -515  -601}%
\special{pa  -516  -599}\special{pa  -517  -598}\special{pa  -518  -596}\special{pa  -518  -594}%
\special{pa  -519  -592}\special{pa  -519  -591}\special{pa  -519  -589}\special{pa  -518  -587}%
\special{pa  -518  -585}\special{pa  -517  -583}\special{pa  -516  -582}\special{pa  -515  -580}%
\special{pa  -513  -579}\special{pa  -512  -578}\special{pa  -510  -577}\special{pa  -509  -576}%
\special{pa  -507  -576}\special{pa  -505  -576}\special{pa  -503  -576}\special{pa  -501  -576}%
\special{pa  -499  -576}\special{pa  -498  -577}\special{pa  -496  -578}\special{pa  -494  -579}%
\special{pa  -493  -580}\special{pa  -492  -582}\special{pa  -491  -583}\special{pa  -490  -585}%
\special{pa  -489  -587}\special{pa  -489  -589}\special{pa  -489  -591}%
\special{fp}%
{%
\color[cmyk]{0,0,0,0}%
\special{pa -269 -591}\special{pa -269 -592}\special{pa -269 -594}\special{pa -270 -596}%
\special{pa -270 -598}\special{pa -271 -599}\special{pa -273 -601}\special{pa -274 -602}%
\special{pa -275 -603}\special{pa -277 -604}\special{pa -279 -605}\special{pa -281 -605}%
\special{pa -283 -605}\special{pa -284 -605}\special{pa -286 -605}\special{pa -288 -605}%
\special{pa -290 -604}\special{pa -291 -603}\special{pa -293 -602}\special{pa -294 -601}%
\special{pa -296 -599}\special{pa -297 -598}\special{pa -297 -596}\special{pa -298 -594}%
\special{pa -298 -592}\special{pa -298 -591}\special{pa -298 -589}\special{pa -298 -587}%
\special{pa -297 -585}\special{pa -297 -583}\special{pa -296 -582}\special{pa -294 -580}%
\special{pa -293 -579}\special{pa -291 -578}\special{pa -290 -577}\special{pa -288 -576}%
\special{pa -286 -576}\special{pa -284 -576}\special{pa -283 -576}\special{pa -281 -576}%
\special{pa -279 -576}\special{pa -277 -577}\special{pa -275 -578}\special{pa -274 -579}%
\special{pa -273 -580}\special{pa -271 -582}\special{pa -270 -583}\special{pa -270 -585}%
\special{pa -269 -587}\special{pa -269 -589}\special{pa -269 -591}\special{pa -269 -591}%
\special{sh 1}\special{ip}%
}%
\special{pa  -269  -591}\special{pa  -269  -592}\special{pa  -269  -594}\special{pa  -270  -596}%
\special{pa  -270  -598}\special{pa  -271  -599}\special{pa  -273  -601}\special{pa  -274  -602}%
\special{pa  -275  -603}\special{pa  -277  -604}\special{pa  -279  -605}\special{pa  -281  -605}%
\special{pa  -283  -605}\special{pa  -284  -605}\special{pa  -286  -605}\special{pa  -288  -605}%
\special{pa  -290  -604}\special{pa  -291  -603}\special{pa  -293  -602}\special{pa  -294  -601}%
\special{pa  -296  -599}\special{pa  -297  -598}\special{pa  -297  -596}\special{pa  -298  -594}%
\special{pa  -298  -592}\special{pa  -298  -591}\special{pa  -298  -589}\special{pa  -298  -587}%
\special{pa  -297  -585}\special{pa  -297  -583}\special{pa  -296  -582}\special{pa  -294  -580}%
\special{pa  -293  -579}\special{pa  -291  -578}\special{pa  -290  -577}\special{pa  -288  -576}%
\special{pa  -286  -576}\special{pa  -284  -576}\special{pa  -283  -576}\special{pa  -281  -576}%
\special{pa  -279  -576}\special{pa  -277  -577}\special{pa  -275  -578}\special{pa  -274  -579}%
\special{pa  -273  -580}\special{pa  -271  -582}\special{pa  -270  -583}\special{pa  -270  -585}%
\special{pa  -269  -587}\special{pa  -269  -589}\special{pa  -269  -591}%
\special{fp}%
\settowidth{\Width}{$V$}\setlength{\Width}{-1\Width}%
\settoheight{\Height}{$V$}\settodepth{\Depth}{$V$}\setlength{\Height}{-0.5\Height}\setlength{\Depth}{0.5\Depth}\addtolength{\Height}{\Depth}%
\put( -2.470,  0.250){\hspace*{\Width}\raisebox{\Height}{$V$}}%
%
\settowidth{\Width}{S}\setlength{\Width}{-1\Width}%
\settoheight{\Height}{S}\settodepth{\Depth}{S}\setlength{\Height}{\Depth}%
\put( -1.050,  1.750){\hspace*{\Width}\raisebox{\Height}{S}}%
%
\end{picture}}%
        }
        間隔$d$だけ離れた極板A,Bからなる電気容量$C$の平行板コンデンサー,起電力$V$の電池とスイッチSからなる図1のような回路がある。
        まず,スイッチSを閉じた。
        \begin{Enumerate}
            \item コンデンサーに蓄えられた電気量はいくらか。
        \end{Enumerate}
        次に,スイッチAは閉じたまま,厚さ$\bunsuu{d}{2}$の金属板Pを図2のように極板A,Bに平行に極板間の中央に挿入した。
        \begin{Enumerate*}
            \item このときの極板Aから極板Bまでの電位の変化の様子を極板Aからの距離を横軸としてグラフに描け。
            \item また,このときの極板Aに蓄えられた電気量はいくらか。
            \item さらに,スイッチSを開いた後,金属板Pを取り去った。このときの極板間の電位差$V'$はいくらか。
            \item Pを取り去るときに外力がした仕事$W$はいくらか。
        \end{Enumerate*}
    \end{mawarikomi}
