\item
    \begin{mawarikomi}{240pt}{
        \input{./fig/fig139.tex}
    }
質量 $m$\tanni{kg}、電荷 $-e$\tanni{C}、初速 $0$ の電子を電圧 $V_0$\tanni{V} で加速し、間隔 $d$\tanni{m}、長さ $l$\tanni{m}、極板間電圧 $V$\tanni{V} の平行極板間を通過させる。電子の入射方向に $x$ 軸をとり、極板の左端を原点 $\mathrm{O}$ とする。

極板は $x$ 軸に平行で、電子は極板間の一様な電場(電界)から力を受け、蛍光面に到達する。$y$ 軸は極板に垂直であり、蛍光面は $x$ 軸に垂直で $x=L$\tanni{m} の位置にある。

    \begin{enumerate}
        \item 平行極板間に入射するときの電子の速さ $v_0$ はいくらか。
        \item 極板間で電子が受ける力の大きさはいくらか。また、極板の右端 ($x=l$) における電子の $y$ 座標 $y_1$ を求めよ。$v_0$ を用いてよい(以下の問も同様)。
        \item 蛍光面上に到達したときの電子の $y$ 座標 $y_2$ を求めよ。
        \item 平行極板間の領域に一様な磁場(磁界)を加えることによって電子の軌道を $x$ 軸からそれないようにしたい。磁束密度 $B$ および磁場の向きをどのように選べばよいか。
    \end{enumerate}
\end{mawarikomi}
