\hakosyokika
\item 高さ$8$\sftanni{cm}のろうそくと,焦点距離$10$\sftanni{cm}の凸レンズ$\mathrm{L_1}$,焦点距離$10$\sftanni{cm}の凹レンズ$\mathrm{L_2}$がある。レンズの厚さは考えなくてよい。
        \begin{enumerate}
            \item 凸レンズ$\mathrm{L_1}$の前方$30$\sftanni{cm}の位置に光軸に垂直にろうそくを立てる。$\mathrm{L_1}$によって作られるろうそくの像の位置および像の大きさを求めよ。さらに,像が実像か虚像か,また正立か倒立かを答えよ。
            \item $\mathrm{L_1}$により倍率1の実像ができるとき,ろうそくから$\mathrm{L_1}$までの距離はいくらか。
            \item 凹レンズ$\mathrm{L_2}$の前方$30$\sftanni{cm}の位置にろうそくを立てる。像の位置および像の大きさを答えよ。さらに実像か,虚像か,また正立か倒立かを答えよ。
            \item $\mathrm{L_1}$と$\mathrm{L_2}$を$30$\sftanni{cm}離し,光軸を合わせる。$\mathrm{L_1}$の前方($\mathrm{L_2}$とは反対方向)$5$\sftanni{cm}の位置にろうそくを立てる。まず,$\mathrm{L_1}$だけによる像の位置を求めよ。次に$\mathrm{L_1}$,$\mathrm{L_2}$全体による像の位置と大きさを求めよ。
        \end{enumerate}