\hakosyokika
\item
    \begin{mawarikomi}{150pt}{\begin{zahyou*}[ul=5mm](-5,6)(-1,6)
    \small
    \def\O{(0,0)}
    \def\A{(0,4)}
    \def\AU{(0,4.2)}
    \def\L{(-4.5,0)}
    \def\R{(6,0)}
    \def\LD{(-4.5,-0.2)}
    \def\RD{(6,-0.2)}
    \def\Q{(5,0)}
    \def\Fx{1.2+2*X-X*X*0.48}
    \Kaiten\O\AU{-30}\P
    \Kaiten\O\AU{-50}\PP
    \Kaiten\O\A{-50}\PPP
    \Nuritubusi*{\L\LD\RD\R\L}
    \Drawline{\L\R}
    \Hasen{\O\A}
    \Hasen{\O\P}
    \Hasen{\O\PPP}
    \YGurafu(*)(0.05)\Fx{3.2}{4.7}
    \Kakukigou\P\O\A<Hankei=10mm>(0pt,4pt)[l]{$\theta $}
    \Kakukigou\PP\O\A<Hankei=5mm>(-1.8pt,4pt)[l]{$\theta _0$}
    \HenKo\A\O{$a$}
    \En*[1]\P{0.2}
    \En<hasen=[3][2.7]>\PP{0.2}
    \Enko\O{4}{0}{180}
    \Put\A[n]{A}
    \Put\P[ne]{P}
    \Put\O[s]{O}
    \Put\Q[s]{Q}
\end{zahyou*}
}
        次の空欄に適切な数式や数値を入れよ。重力加速度の大きさを$g$とし,答えに用いてよい文字は,$m$,$g$,$a$,$\theta $とする。\\
        ~~半径$a$のなめらかな半円柱が図のように水平面上に置かれている。質量$m$の小球を最高点Aに静かに置いたところ,小球は円柱面を滑り始めた。この小球がP点($\angle \mathrm{AOP}=\theta $)に達したときの速さは\Hako であり,小球が円柱面を押す力は\Hako である。\\
        ~~やがて小球は円柱面を離れる。このときの$\cos{\theta }$の値$\cos{\theta _0}$は\Hako であり,小球は速さ$v_0=\Hako $で円柱面を離れ,水平面上のQ点に落ちる。小球がQ点に到達するときの速さは$v_1=\Hako $となる。
    \end{mawarikomi}