%\documentclass[b5j,10pt]{jsarticle}
\documentclass[b5j,9.5pt]{jsbook}
\usepackage[noalphabet]{pxchfon}
\setminchofont{UDDIGIKYOKASHON-R.TTC} 
\setgothicfont{UDDIGIKYOKASHON-B.TTC} 
%\setminchofont{BIZ-UDMINCHOM.TTC} 
%\setgothicfont{BIZ-UDGOTHICR.TTC} 
\usepackage{okumacro}
\usepackage{amsmath,amsthm,amssymb,fancybox}
\usepackage{enumerate,multicol}
\usepackage{ascmac,itembbox,emath,hako,scrpage,ulinej,emathP,emathPp,emathMw,emathEy}
\usepackage[version=4]{mhchem}
%\usepackage[draft]{graphicx}
\usepackage{graphicx}
\usepackage{picins}
\pagestyle{empty}
\setlength{\textwidth}{162mm}
\setlength{\textheight}{230mm}
\setlength{\oddsidemargin}{-15.4mm}
\setlength{\evensidemargin}{-15.4mm}
\setlength{\topmargin}{-17.4mm}
%\setlength{\columnseprule}{0.4pt}
\renewcommand\labelenumi{\fbox{\bfseries{\sffamily{\theenumi}}}}
\renewcommand{\labelenumii}{(\arabic{enumii})}
\renewcommand{\labelenumiii}{\bfseries{\カタカナ{enumiii}.}}
\def\bunsuushisuu#1#2{\raisebox{0.7zh}{$#1 \over #2$}}
\def\santaku{{\bfseries ア}~{\bfseries ウ}}
\def\yontaku{{\bfseries ア}~{\bfseries エ}}
\def\gotaku{{\bfseries ア}~{\bfseries オ}}
\def\rokutaku{{\bfseries ア}~{\bfseries カ}}
\def\nanataku{{\bfseries ア}~{\bfseries キ}}
\def\sbou#1#2{$_{\sf{#1}}$\kern-0.2pt \ulinej{#2}}%
\def\tanni#1{$〔\mathrm{\sf #1}〕\kern -2pt$}%
\def\sftanni#1{$\kern 2pt{\mathrm{\sf #1}}$}
\def\gen#1#2#3{{\sf{\ce{_{#2}^{#1}#3}}}}
\begin{document}
\AtBeginDvi{\special{papersize=\the\paperwidth,\the\paperheight}}
%\setcounter{page}{1}
\newpagestyle{custom}{(0mm,0pt)%
{\hfill}
 {%
 \begin{minipage}{16.5cm}
 {\bf {\large 令和6年度一学期中間考査 1年物理基礎(専門科)}}\hfill T1A原田,T1BCD舟橋 R6.5.16〔木〕~2限\\
※定規使用可能
 \end{minipage}
}%
{\hfill}%
(\textwidth,0pt)}%
{(0mm,0pt)%
{\hfill}{\hfill}{\hfill}%
(0mm,0pt)}
\pagestyle{custom}
\begin{enumerate}
	\item 以下の問いに答えよ。
		\begin{enumerate}
			\item 有効数字に注意して,以下の計算をせよ。ただし,$\sqrt{3}=1.73205\cdots$とする。
				\begin{edaenumerate}<3>[m]
					\item $3.14\times 3.0$
					\item $\bunsuu{\sqrt{3}}{2.0} $
					\item $35.2+3.26$
				\end{edaenumerate}
				\hakosyokika
			\item 次の数値を\karaHako$\times 10^n$の形で表せ。$1\leqq \karaHako < 10$とし,有効数字は2桁とする。
				\begin{edaenumerate}<3>[m]
					\item 380000
					\item 0.0016
					\item 0.5000
				\end{edaenumerate}
			\item 25\sftanni{m/s}は何\sftanni{km/h}か。
			\item 流水の速さが$2.0$\sftanni{m/s}のまっすぐな川を静水時の速さが$5.0$\sftanni{m/s}の船が下流から上流へ向かって進んでいるときの,川岸で静止している人から見た船の速さ(速度の大きさ)は何\sftanni{m/s}か。
			\item 東西に通じる道路上を,次のように自転車A,Bが進むとき,自転車Aに対する自転車Bの相対速度はどの向きに何\sftanni{m/s}か。
				\begin{enumerate}[m]
					\item Aは東向きに$2.0$\sftanni{m/s}の速さ,Bは東向き$3.0$\sftanni{m/s}の速さ
					\item Aは西向きに$2.0$\sftanni{m/s}の速さ,Bは東向き$3.0$\sftanni{m/s}の速さ
				\end{enumerate}
			\item 次の各場合について,物体の平均加速度$\bar{a}$\tanni{m/s}を求めよ。
				\begin{enumerate}[m]
					\item 一直線上を正の向きに$5.0$\sftanni{m/s}で進む物体が,$2.0$秒後に正の向きに$9.0$\sftanni{m/s}になったとき。
					\item 一直線上を正の向きに$5.0$\sftanni{m/s}で進む物体が,$2.0$秒後に負の向きに$4.0$\sftanni{m/s}になったとき。
				\end{enumerate}
			\item $3.0$\sftanni{m/s}で動いてた物体が一定の加速度$1.5$\sftanni{m/s^2}で速さを増した。
				\begin{enumerate}[m]
					\item $2.0$秒後の物体の速さは何\sftanni{m/s}か。
					\item $2.0$秒後までに物体が進んだ距離は何\sftanni{m}か。
				\end{enumerate}
			\item $2.0$\sftanni{m/s}の速さで動いていた物体が,一定の加速度$5.0$\sftanni{m/s^2}で速さを増し,$6.0$\sftanni{m/s}の速さになった。この間に物体が進んだ距離は何\sftanni{m}か。
		\end{enumerate}
	\vfill
	\item
	\begin{mawarikomi}(30pt,0pt){150pt}{
		\begin{zahyou}[ul=3.5mm
			,yokozikukigou=$t$\tanni{s}
			,tatezikukigou=$x$\tanni{m}
			,yokozikuhaiti={[s]}
			,migiyohaku=.5
			](0,10)(0,10)%
			\zahyouMemori[g][n]<2>
			\def\Fx{1/8*X*X}
			\def\O{(2,0)}
			\def\A{(10,8)}
			\def\P{(4,2)}
			\def\L{(8,6)}
			\def\M{(10,5)}
			\Put\P[nw]{P}
			\kuromaru{\P}
			{\Thicklines
			\YGurafu\Fx{0}{9}
			\Hasen*{\O\A}
			}
			\xmemori<2.0>{2}
			\xmemori<4.0>{4}
			\xmemori<6.0>{6}
			\xmemori<8.0>{8}
			\ymemori<2.0>{2}
			\ymemori<4.0>{4}
			\ymemori<6.0>{6}
			\ymemori<8.0>{8}
%			\HenKo[0]\O\A{L}
			\PutStr\M{L}to\L
		\end{zahyou}
	}%
	図は,$x$軸上を運動する物体の位置$x$と経過時間$t$の関係をグラフに表したものである($x-t$図)。図の破線Lは点Pにおける接線である。
	\begin{enumerate}
		\item 時刻$4.0~8.0$秒の間の平均の速度は何\sftanni{m/s}か。
		\item 時刻$4.0$秒における瞬間の速さは何\sftanni{m/s}か。
	\end{enumerate}
	\end{mawarikomi}
\vfill
\item 東西に通じる道路上を自動車A,B,Cが進んでいる。自動車Aの速度は東向きに15\sftanni{m/s}である。以下の問いに答えよ。
\begin{enumerate}
	\item 自動車Bに対する自動車Aの相対速度は西向き8\sftanni{m/s}であった。自動車Bの速度はどちら向きに何\sftanni{m/s}か。
	\item 自動車Cに対する自動車Aの相対速度が東向き20\sftanni{m/s}であるとき,(1)の自動車Bに対する自動車Cの相対速度はどちら向きに何\sftanni{m/s}か。
\end{enumerate}
\vfill
\newpage
\item 速さ15.0\sftanni{m/s}で運動していた自動車が,一定の加速度で速さを増し,$2.0$秒後に$20.0$\sftanni{m/s}の速さになった。
	\begin{enumerate}
		\item このときの加速度の大きさを求めよ。
		\item 自動車が加速している間に進んだ距離を求めよ。
		\item こののち,自動車が急ブレーキをかけて,一定の加速度で減速し,$50$\sftanni{m}進んで静止した。運動の向きを正として,このときの加速度を求めよ。
	\end{enumerate}
\vfill
\item
	\begin{mawarikomi}(30pt,0pt){150pt}{
		\begin{zahyou}[ul=3.5mm
			,yokozikukigou=$t$\tanni{s}
			,tatezikukigou=$v$\tanni{m/s}
			,yokozikuhaiti={[s]}
			,migiyohaku=.5
			,yscale=0.5
			](0,9)(-6,15)%
			\def\Fx{-1.5*X+9}
			\def\tval{8}
			\YTen\Fx\tval\V
			{\Thicklines
			\YGurafu\Fx{0}{8}
			}
			\ymemori<9.0>{9}
			\Put\V[
				syaei=xy
				,xlabel=8.0
				,ylabel=-3.0
			]{}
		\end{zahyou}
		}
		図は,$x$軸上を等加速度直線運動している物体が,原点を時刻0\sftanni{s}に通過した後の,8.0秒間の速度と時間の関係を表す$v-t$図である。
		\begin{enumerate}
			\item 物体の加速度$a$\tanni{m/s^2}を求めよ。
			\item 物体が原点から最も遠ざかるときの時刻$t_1$\tanni{s}と,その位置$x_1$\tanni{m}を求めよ。
			\item 8.0秒後の物体の位置$x_2$\tanni{m}を求めよ。
			\item 8.0秒間に物体が移動した距離(道のり)$L$\tanni{m}を求めよ。
			\item 経過時間$t$\tanni{s}と物体の位置$x$\tanni{m}の関係をグラフに示せ。
		\end{enumerate}
	\end{mawarikomi}
\vfill
\item
	\begin{mawarikomi}(15pt,0pt){150pt}{
		\begin{zahyou}[ul=3.5mm
			,yokozikukigou=$t$\tanni{s}
			,tatezikukigou=$a$\tanni{m/s^2}
			,yokozikuhaiti={[s]}
			,tatezikuhaiti={[e]}
			,ueyohaku=0
			,migiyohaku=0.5
			,yscale=1.5
			](0,12)(-3,3)%
			\def\O{(0,2)}
			\def\A{(3,2)}
			\def\B{(3,0)}
			\def\D{(7,0)}
			\def\E{(7,-1.5)}
			\def\F{(11,-1.5)}
			{\Thicklines
			\Drawline{\O\A}
			\Drawline{\B\D}
			\Drawline{\E\F}
			}
			\Put\A[
				syaei=xy
				,xlabel=3.0
				,ylabel=2.0
			]{}
			\Put\E[
				syaei=xy
				,xlabel=7.0
				,ylabel=-1.5
			]{}
			\Put\F[
				syaei=x
				,xlabel=11.0
			]{}
		\end{zahyou}
		}
	右の図は,止まっていたエレベーターが上昇し,停止するまでの加速度$a$\tanni{m/s^2}の時間変化を表したグラフである。
		\begin{enumerate}
			\item 時刻$5.0$\tanni{s}のエレベーターの速さは何\sftanni{m/s}か。
			\item エレベーターの速度$v$\tanni{m/s}と時間$t$\tanni{s}との関係をグラフに示せ。
			\item $11.0$秒間にエレベーターが上昇した高さ$h$\tanni{m}を求めよ。
			\item エレベーターの上昇距離$x$\tanni{m}と時間$t$\tanni{s}との関係をグラフに示せ。
		\end{enumerate}
	\end{mawarikomi}
\vfill
\item
\begin{mawarikomi}(30pt,0pt){120pt}{
	\begin{zahyou}[ul=3.5mm
		,yokozikukigou=$t$\tanni{s}
		,tatezikukigou=$v$\tanni{m/s}
		,yokozikuhaiti={[s]}
		,tatezikuhaiti={[w]}
		,ueyohaku=2
		,migiyohaku=0.5
		,yscale=0.4
		,xscale=1.8
		](0,5)(0,25)%
		\def\Ba{(0,21)}
		\def\Bb{(2,15)}
		\def\Bc{(4,9)}
		\def\Aa{(0,6)}
		\def\Ab{(2,6)}
		\def\Ac{(4,9)}
		\def\Aq{(3,5)}
		\def\Bq{(3,20)}
		{\Thicklines
		\Drawline{\Ba\Bb\Bc}
		\Drawline{\Aa\Ab\Ac}
		}
		\ymemori<21.0>{21}
		\ymemori<6.0>{6}
		\Put\Bb[
			syaei=xy
			,xlabel=2.0
			,ylabel=15.0
		]{}
		\Put\Bc[
			syaei=x
			,xlabel=4.0
		]{}
		\PutStr\Aq{A}to\Ab
		\PutStr\Bq{B}to\Bb
	\end{zahyou}
	}
	直線状の道路を速さ$21.0$\sftanni{m/s}で走っていた自動車Bの運転手は,前方に同じ方向に走る自動車Aを発見し,ブレーキをかけて一定の加速度で減速し始めた。ブレーキをかけた瞬間を時刻$t=0$\sftanni{s}とすると,Bは$t=2.0$\sftanni{s}に速さ$15.0$\sftanni{m/s}になった。\\
	~~一方,速さ$6.0$\sftanni{m/s}の等速で進んでいたAは,$t=2.0$\sftanni{s}の瞬間からアクセルを踏んで,一定の加速度で加速し始めた。その結果,$t=4.0$\sftanni{s}のとき,車間間隔は最も短くなって,$3.0$\sftanni{m}となり,衝突をまぬがれた。右図は,自動車A,Bの$v-t$図である。A,Bの進行方向を正として以下の問いに答えよ。
	\begin{enumerate}
		\item 自動車Bの加速度$a_\mathrm{B}$\tanni{m/s^2}を求めよ。
		\item 時刻$t=2.0~4.0$\sftanni{s}の自動車Aの加速度$a_\mathrm{A}$\tanni{m/s^2}を求めよ。
		\item 時刻$t=2.0$\sftanni{s}の瞬間のAとBの車間距離$L $\tanni{m}を求めよ。
		\item 時刻$t=0$\sftanni{s}の瞬間のAとBの車間距離$L_0 $\tanni{m}を求めよ。
	\end{enumerate}

	\end{mawarikomi}
\end{enumerate}
	% 〔計算用紙〕
\vfill
\end{document}