\hakosyokika
\item
    \begin{mawarikomi}{150pt}{\begin{zahyou*}[ul=5mm](-6,5)(-1,7)
    \small
    \def\O{(2,2.5)}
    \def\C{(2,0)}
    \def\B{(0,0)}
    \def\Fx{X*X/6}
    \funcval\Fx{-5}\yi
    \calcval{\yi+0.22}\yii
    \def\M{(-5,\yii)}
    \def\MD{(-5,0)}
    \YGurafu\Fx{-6}{0}
    \Drawline{\MD\C}
    \Kaiten\O\C{60}\P
    \Kaiten\O\C{240}\D
    \Hasen{\P\O\C}
    \Enko\O{2.5}{270}{150}
    \Kakukigou\C\O\P(-0.8pt,-5pt)[l]{$\theta $}
    {\KuromaruHankei{1.8pt}
    \Kuromaru\M}
    \HenKo\O\C{$r$}
    \HenKo<henkotype=parallel,
    % henkoH=11ex,
    yazirusi=b,
    henkosideb=0,
    henkosidet=1.2>\M\MD{$h$}
    \Put\B[s]{B}
    \Put\C[s]{C}
    \Put\O[n]{O}
    \Put\M[ne]{$m$}
    \Put\D[sw]{D}
    \Put\P[se]{P}
    \put(-6,4.7){A}
\end{zahyou*}
}
        右の図で,BC間は水平面で,AB間の曲面やCD間の円筒面となめらかにつながっている。円筒面の半径は$r$で中心軸はOである。いま,曲面上で水平面から$h$の高さの位置から質量$m$の小球を静かに放す。摩擦はなく,重力加速度の大きさを$g$とする。
        \begin{enumerate}
            \item 水平面BC上での小球の速さを求めよ。
            \item 点Cを通る直前に小球が受ける垂直抗力$N_1$と,通った直後に受ける垂直抗力$N_2$を求めよ。
            \item 図の点P($\angle \mathrm{COP}=\theta $)での速さ$v$と垂直抗力$N$を求めよ。
            \item 小球が円筒面に沿って,点Dに達するのに必要な高さ$h$の最小値$h_0$を求め,$r$を用いて表せ。
            \item $h=2r$のときには,小球は途中で円筒面を離れる。離れる点での$\cos{\theta }$の値を求めよ。
        \end{enumerate}
    \end{mawarikomi}