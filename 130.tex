\hakosyokika
\item
    \begin{mawarikomi}(10pt,0pt){120pt}{
        %WinTpicVersion4.32a
{\unitlength 0.1in%
\begin{picture}(12.0000,13.4800)(6.0000,-16.0000)%
% CIRCLE 1 0 3 0 Black White  
% 4 1200 1000 1800 1000 1100 400 1200 400
% 
\special{pn 13}%
\special{ar 1200 1000 600 600 4.7123890 4.5472403}%
% STR 2 0 3 0 Black White  
% 4 1075 645 1075 695 5 0 0 0
% $R$
\put(10.7500,-6.9500){\makebox(0,0){$R$}}%
% LINE 2 0 3 0 Black White  
% 2 1200 1010 1200 410
% 
\special{pn 8}%
\special{pa 1200 1010}%
\special{pa 1200 410}%
\special{fp}%
% DOT 0 0 3 0 Black White  
% 2 1200 400 1100 410
% 
\special{pn 4}%
\special{sh 1}%
\special{ar 1200 400 16 16 0 6.2831853}%
\special{sh 1}%
\special{ar 1100 410 16 16 0 6.2831853}%
% DOT 0 0 3 0 Black White  
% 1 1202 1001
% 
\special{pn 4}%
\special{sh 1}%
\special{ar 1202 1001 16 16 0 6.2831853}%
% CIRCLE 3 0 3 0 Black White  
% 4 1726 1415 1742 745 1742 745 1199 1000
% 
\special{pn 4}%
\special{ar 1726 1415 670 670 3.8086504 4.7362650}%
% LINE 1 0 3 0 Black White  
% 2 1202 1001 1775 729
% 
\special{pn 13}%
\special{pa 1202 1001}%
\special{pa 1775 729}%
\special{fp}%
% STR 2 0 3 0 Black White  
% 4 1415 725 1415 775 5 0 1 0
% $a$
\put(14.1500,-7.7500){\makebox(0,0){{\colorbox[named]{White}{$a$}}}}%
% BOX 2 0 2 0 Black White  
% 2 1175 610 1225 810
% 
\special{pn 0}%
\special{sh 0}%
\special{pa 1175 610}%
\special{pa 1225 610}%
\special{pa 1225 810}%
\special{pa 1175 810}%
\special{pa 1175 610}%
\special{ip}%
\special{pn 8}%
\special{pa 1175 610}%
\special{pa 1225 610}%
\special{pa 1225 810}%
\special{pa 1175 810}%
\special{pa 1175 610}%
\special{pa 1225 610}%
\special{fp}%
% STR 2 0 3 0 Black White  
% 4 1230 275 1230 325 5 0 0 0
% P
\put(12.3000,-3.2500){\makebox(0,0){P}}%
% STR 2 0 3 0 Black White  
% 4 1095 280 1095 330 5 0 0 0
% S
\put(10.9500,-3.3000){\makebox(0,0){S}}%
% STR 2 0 3 0 Black White  
% 4 1770 580 1770 630 5 0 0 0
% Q
\put(17.7000,-6.3000){\makebox(0,0){Q}}%
% CIRCLE 2 0 3 0 Black White  
% 4 1200 1005 1330 1005 1530 1105 1430 405
% 
\special{pn 8}%
\special{ar 1200 1005 130 130 5.0784455 0.2942346}%
% SARROW 2 0 3 1 Black White  
% 2 1328 1028 1324 1043
% 
\special{pn 8}%
\special{pa 1328 1028}%
\special{pa 1324 1043}%
\special{fp}%
\special{sh 1}%
\special{pa 1324 1043}%
\special{pa 1361 984}%
\special{pa 1338 991}%
\special{pa 1322 973}%
\special{pa 1324 1043}%
\special{fp}%
% STR 2 0 3 0 Black White  
% 4 1370 1040 1370 1090 2 0 0 0
% $\omega$
\put(13.7000,-10.9000){\makebox(0,0)[lb]{$\omega$}}%
% STR 2 0 3 0 Black White  
% 4 1195 955 1195 1005 4 0 0 0
% O
\put(11.9500,-10.0500){\makebox(0,0)[rt]{O}}%
% STR 2 0 3 0 Black White  
% 4 1295 1195 1295 1245 5 0 0 0
% $\otimes$
\put(12.9500,-12.4500){\makebox(0,0){$\otimes$}}%
% STR 2 0 3 0 Black White  
% 4 1195 1340 1195 1390 5 0 0 0
% $B$
\put(11.9500,-13.9000){\makebox(0,0){$B$}}%
% STR 2 0 3 0 Black White  
% 4 1595 860 1595 910 5 0 0 0
% \small{棒}
\put(15.9500,-9.1000){\makebox(0,0){\small{棒}}}%
\end{picture}}%

    }
    細い導線で作った半径$a$\tanni{m}の円形レール(S,P間は切れている)があり,このレール面の中心Oとレールの点Pとの間には$R$\tanni{\Omega }の抵抗が接続されている。さらに,中心Oとレールの間には,レールに接しながら回転できる導体の棒OQが橋渡ししてあり,この棒は一定の角速度$\omega $\tanni{rad/s}で回転している。レール面には,それに垂直に磁束密度$B$\tanni{T}の一様な磁場が紙面の表から裏向きに加わっている。
        \begin{enumerate}
            \item コイルOPQを貫く磁束は$\varDelta t$\tanni{s}間にどれだけ増加するか。
            \item 抵抗$R$\tanni{\Omega }の両端に発生する電位差$V$\tanni{V}を求めよ。また,抵抗を流れる電流の向きはO$\rightarrow $PかそれともP$\rightarrow$Oか。
            \item 抵抗$R$\tanni{\Omega }で消費する電力はいくらか。
            \item 棒OQが磁場から受ける力はいくらか。その向きは回転と同方向か,逆方向か。
            \item 棒OQを一定の角速度$\omega $\tanni{rad/s}で回転させるために必要な外力の仕事率$P$\tanni{W}はいくらか。
        \end{enumerate}
    \end{mawarikomi}