\hakosyokika
\item
    \begin{mawarikomi}(20pt,0){120pt}{\begin{zahyou*}[ul=5mm](0,9)(0,10)
    \small
    \def\O{(0,0)}
    \def\A{(0,9)}
    \def\B{(1,9)}
    \def\C{(1,1)}
    \def\D{(7,1)}
    \def\E{(7,9)}
    \def\F{(8,9)}
    \def\G{(8,0)}
    \def\H{(1,7)}
    \def\I{(7,7)}
    \def\J{(7,6)}
    \def\K{(1,6)}
    \Nuritubusi*{\H\I\J\K\H}
    \Drawline{\O\A\B\C\D\E\F\G\O}
    \Drawline{\H\I}
    \Drawline{\K\J}
    \HenKo\C\K{$h$}
    \put(3,8){$M$}
    \put(5,5){$S$}
\end{zahyou*}
}
        なめらかに動く質量$M$\tanni{kg}~,断面積$S$\tanni{m^2}のピストン付きの容器がある。容器は断熱材でできているが,加熱器より熱を加えることができる。この容器に$n$モルの理想気体を入れたところ,ピストンの高さは$h$\tanni{m}であった(状態A)。大気圧を$P_0$\tanni{Pa}~,重力加速度の大きさを$g$\tanni{m/s^2}~,気体定数を$R$\tanni{J/(mol\cdot K)}とする。
        \begin{Enumerate}
            \item 状態Aでの気体の圧力$P$\tanni{Pa}と温度$T$\tanni{K}を求めよ。
        \end{Enumerate}
        次に,気体をゆっくりと加熱したところ,気体の温度は$T'$\tanni{K}となった(状態B)
        \begin{Enumerate*}
            \item 状態Bでのピストンの高さ$h'$\tanni{m}を$h$,$T$,$T'$で表せ。
            \item 状態AからBまで変化する間に気体がする仕事$W$\tanni{J}を$n$,$T$,$T'$,$R$を用いて表せ。また,その間に気体に与えた熱量を$Q$\tanni{J}として,内部エネルギーの変化$\varDelta U$\tanni{J}を$Q$と$W$で表せ。
            \item (3)の結果を用いて,定積モル比熱$C_V $\tanni{J/mol\cdot K}と定圧モル比熱$C_P$\tanni{J/mol\cdot K}との間に成り立つ関係式を求めよ。
        \end{Enumerate*}
    \end{mawarikomi}