\documentclass[a4paper, 10pt, dvipdfmx]{bxjsarticle}
\usepackage{amsmath}
\usepackage{graphicx}
\usepackage{enumitem}
\usepackage{geometry}
\geometry{a4paper, margin=2cm}

\begin{document}

\pagenumbering{gobble} % ページ番号を非表示にする

% --- OCRからのテキスト ---

% =================================================================
% 1ページ目
% =================================================================
\section*{1ページ目}

\noindent
\textbf{90}

\section*{6 交流}

\subsection*{電圧と電流}
\begin{tabular}{|l|l|l|l|}
\hline
 & \textbf{抵抗} & \textbf{コイル} & \textbf{コンデンサー} \\ \hline
 & R [$\Omega$] & L [H] & C [F] \\ \hline
 & V = RI & $V = \omega LI$ & $V = \frac{1}{\omega C}I$ \\ \hline
\textbf{電圧の位相と} & 電圧に対して & 電圧に対して & 電圧に対して \\
\textbf{電流の位相} & 位相は同じ & 電流は遅れる & 電流は進む \\ \hline
\end{tabular}

\begin{flushleft}
※ V, Iは共に実効値、または共に最大値。$\omega$は角周波数 \\
実効値 = $\frac{\text{最大値}}{\sqrt{2}}$ \\
※ 消費電力は抵抗でのみ生じ $RI^2 = V_eI_e$ (V, Iは実効値)
\end{flushleft}

\subsection*{電気振動}
\begin{flushleft}
周期 $T = 2\pi\sqrt{LC}$ \\
静電エネルギー $\frac{1}{2}CV^2$ + 磁気エネルギー $\frac{1}{2}Li^2$ = 一定
\end{flushleft}

\bigskip

\noindent
\textbf{電磁気 91}

\paragraph{135} 空欄に入る数値を、解答群から選べ。同じものを繰り返し選んでもよい。
発電所で発電された交流の電気は、変圧器(トランス)により電圧を高くして、送電線を通して送られる。たとえば、電圧を10倍にするには変圧器の1次コイルの巻数に対して2次コイルの巻数を \textbf{(1)} 倍にすればよい。このとき周波数は \textbf{(2)} 倍になる。発電所から同じ電力を送るとき、送電線に送り出す電圧(送電電圧)を10倍にすると、送電線を流れる電流は \textbf{(3)} 倍になる。この結果、送電線の抵抗によって熱として失われる電力は \textbf{(4)} 倍になる。ただし、送電線の抵抗は変化しないものとする。

\begin{enumerate}[label=(\arabic*)]
    \item $\frac{1}{100}$
    \item $\frac{1}{10}$
    \item $\frac{1}{\sqrt{10}}$
    \item $1$
    \item $\sqrt{10}$
    \item $10$
    \item $100$
\end{enumerate}

\paragraph{136} 図1のように、抵抗値Rの抵抗、電気容量Cのコンデンサーおよび自己インダクタンスLのコイルを直列に接続し、交流電源につないだ回路がある。オシロスコープで抵抗の両端の電圧を観測したところ、図2のような周期T,最大値$V_0$の正弦曲線であった。
% 図1と図2の画像プレースホルダー
\begin{center}
    \fbox{図1の画像} \quad \fbox{図2の画像}
\end{center}
(1) 交流の角周波数を求めよ。\\
以下、(5)以外はTの代わりに$\omega$を用いて答えよ。
(2) 抵抗に流れる電流を時刻tの関数として表せ。また実効値を求めよ。
(3) この直列回路での消費電力(平均電力)を求めよ。
(4) コンデンサーにかかる電圧の実効値を求めよ。また、電圧$v_C$を時刻tの関数として表せ。
(5) 図2で、コンデンサーにかかる電圧が0になる時刻tを$0 \le t \le T$の範囲で求めよ。
(6) コイルにかかる電圧の実効値を求めよ。また、電圧$v_L$を時刻tの関数として表せ。
(7) 電源電圧の最大値Vを求めよ。また、ab間の電圧の最大値$V_{ab}$を求めよ。

\paragraph{137} 電池(起電力V), 抵抗(抵抗値R), コンデンサー(容量C), コイル(自己インダクタンスL), スイッチS1, S2からなる回路があり、最初S1, S2は開いている。
% 図の画像プレースホルダー
\begin{center}
    \fbox{137の回路図}
\end{center}

\clearpage

% =================================================================
% 2ページ目
% =================================================================
\section*{2ページ目}
\noindent
\textbf{92}

池やコイルなどの内部抵抗は無視する。
(1) S1を閉じる。
\begin{enumerate}[label=(\alph*)]
    \item 閉じた直後に抵抗に流れる電流$I_0$を求めよ。
    \item 電流が$i$ ($0 \le i \le I_0$)になったとき、コンデンサーに蓄えられた電気量qを求めよ。
    \item 十分時間が経過した後、コンデンサーに蓄えられる電気量Qを求めよ。
\end{enumerate}
(2) S1を閉じて十分時間が経過した後、S1を開き、次にS2を閉じる。
\begin{enumerate}[label=(\alph*)]
    \item 回路を流れる振動電流の最大値を求めよ。
    \item S2を閉じた直後からのiの時間変化を図示せよ。ただし、iは時計回りの向きを正とする。
    \item S2を閉じてから、コンデンサーの下側極板Bの電荷が正で最大となるまでにかかる時間を求めよ。
\end{enumerate}

\paragraph{138*} 電気容量Cのコンデンサー、自己インダクタンスLのコイル、抵抗値Rの抵抗および起電力Vの電池を図のように接続した。初めスイッチSを開いておく。R以外の抵抗はないものとする。
% 図の画像プレースホルダー
\begin{center}
    \fbox{138の回路図}
\end{center}
(1) Sを閉じた直後に電池を流れる電流$I_0$を求めよ。
(2) Sを閉じてから十分に時間がたったとき、コイルを流れる電流Iを求めよ。また、このときのコンデンサーの電気量を求めよ。
(3) 次にSを開いた。コイルを流れる電流が最初に0になるまでの時間を求めよ。
(4) その後のコンデンサーの電位差の最大値$V_m$を求めよ。

\bigskip

\noindent
\textbf{電磁気 93}

\section*{7 電磁場内の荷電粒子}

\subsection*{一様電場内}
放物運動 \\
静電気力 $F=qE$
% 図の画像プレースホルダー
\begin{center}
    \fbox{一様電場内の荷電粒子の図}
\end{center}

\subsection*{一様磁場内}
等速円運動 \\
ローレンツ力 $f=qvB$が向心力 \\
※ 磁場方向は等速運動
% 図の画像プレースホルダー
\begin{center}
    \fbox{一様磁場内の荷電粒子の図}
\end{center}

\paragraph{139} 質量m[kg], 電荷-e[C], 初速0の電子を電圧$V_0$[V]で加速し、間隔d[m], 長さl[m], 極板間電圧V[V]の平行極板間を通過させる。電子の入射方向にx軸をとり、極板の左端を原点Oとする。
% 図の画像プレースホルダー
\begin{center}
    \fbox{139の装置の図}
\end{center}
極板はx軸に平行で、電子は極板間の一様な電場(電界)から力を受け、蛍光面上に到達する。y軸は極板に垂直であり、蛍光面はx軸に垂直でx=L[m]の位置にある。
(1) 平行極板間に入射するときの電子の速さ$v_0$はいくらか。
(2) 極板間で電子が受ける力の大きさはいくらか。また、極板の右端(x=l)における電子のy座標$y_1$を求めよ。$v_0$を用いてよい(以下の問も同様)。
(3) 蛍光面上に到達したときの電子のy座標$y_2$を求めよ。
(4) 平行極板間の領域に一様な磁場(磁界)を加えることによって電子の軌道をx軸からそれないようにしたい。磁束密度Bおよび磁場の向きをどのように選べばよいか。

\clearpage

% =================================================================
% 3ページ目
% =================================================================
\section*{3ページ目}

\noindent
\textbf{94}

\paragraph{140} z軸の正の方向に磁束密度がBの一様な磁界がかかっている。質量がmで電荷がq(>0)の荷電粒子を、原点Oからyz面内でy軸から角度$\theta$の方向に一定速度vで打ち出した。重力の影響は無視する。
% 図の画像プレースホルダー
\begin{center}
    \fbox{140の座標系の図}
\end{center}
(1) y軸の正の方向($\theta=0$)に打ち出した場合、荷電粒子は等速円運動をする。この等速円運動の中心点の座標($x_0, y_0, z_0$)を求めよ。また、1周するのに要する時間はいくらか。
(2) z軸の正の方向($\theta=\pi/2$)に打ち出した場合、この荷電粒子はどのような運動をするか説明せよ。
(3) y軸との角度$\theta$ ($0<\theta<\pi/2$)の方向に打ち出した場合について、
\begin{enumerate}[label=(\alph*)]
    \item 荷電粒子はどのような運動をするか、説明せよ。
    \item 原点Oから荷電粒子が打ち出されてから、次に初めてz軸と交わるまでの時間を求めよ。また、この交点をPとするとき、OP間の距離はいくらか。
\end{enumerate}

\paragraph{141} 次の(1)~(5)には式を、(a)~(e)には適当な語句を入れよ。
直方体のn型半導体があり, x, y, z方向の長さをそれぞれa, b, cとする。また、半導体は単位体積あたりn個の電子をもつ。図のようにy軸の正の向きに強さIの一様な電流が流れている。
% 図の画像プレースホルダー
\begin{center}
    \fbox{141の半導体の図}
\end{center}
電子の電荷の大きさをe、平均の速さをvとすると、電流Iは \textbf{(1)} と表される。
いま、z軸の正の向きに磁束密度Bの一様な磁場を加えた。電子はやはり平均の速さvで運動しているとすると、大きさ \textbf{(2)} の力をx軸の \textbf{(a)} の向きに受ける。この力は \textbf{(b)} とよばれる。その結果、電子がx軸方向で移動するため、Mに対してNの電位は \textbf{(c)} なり、MN間には電場が発生する。やがて半導体内の電子に対して磁場による

\bigskip

\noindent
\textbf{電磁気 95}

力と電場による力がつりあうことになる。この状態での電場の強さは \textbf{(3)} と表される。したがって、MN間の電位差Vは \textbf{(4)} と表され、Iを用いるとV= \textbf{(5)} と表される。
次に、n型半導体のかわりにp型半導体で同様な実験を行った。p型では \textbf{(d)} が電流のにない手となるので、Mに対してNの電位は \textbf{(e)} なる。

\paragraph{142} \underline{\hspace{2cm}}に語句または式を記し、問いに答えよ。
電気量には最小の単位があり、全ての電気量はその整数倍になっている。この最小単位を電気素量といい、これは \textbf{(ア)} のもっている電気量の大きさに等しい。ミリカンは、図1のような装置に霧吹きから油滴を吹き込み、間隔d[m]の平行な極板A, Bの間を上下する油滴を顕微鏡で観察し、電気素量e[C]を測定した。密度$\rho$[kg/m$^3$], 半径r[m]の球形の油滴の運動を考える。重力加速度をg[m/s$^2$]とし、空気の浮力は無視する。
% 図1, 図2の画像プレースホルダー
\begin{center}
    \fbox{142の図1} \quad \fbox{142の図2}
\end{center}
油滴は極板間に電場がないときは、重力と空気の抵抗力を受けて、鉛直下向きに一定の速さ(終端速度)$v_1$[m/s]で落下する。空気の抵抗力はrと$v_1$の積に比例するので、比例定数をkとすると、この抵抗力と重力のつり合いの式は \textbf{(イ)} と書ける。
油滴は一般に帯電している。その電気量をq[C]とする。Aに対するBの電位をV[V](V>0)とすると、油滴は図2に示すように、鉛直上向きに一定の速さ$v_2$[m/s]で上昇した。このときのつり合いの式は \textbf{(ウ)} となる。
(イ)と(ウ)よりqは$v_1$, $v_2$, d, r, k, Vを用いて, q= \textbf{(エ)} と表される。

\end{document}