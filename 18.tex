\item
\begin{mawarikomi}{150pt}{%WinTpicVersion4.32a
{\unitlength 0.1in%
\begin{picture}(23.3800,35.7200)(2.7000,-41.0000)%
% LINE 2 0 3 0 Black White  
% 2 400 1400 2200 600
% 
\special{pn 8}%
\special{pa 400 1400}%
\special{pa 2200 600}%
\special{fp}%
% POLYGON 2 0 3 0 Black White  
% 5 1940 710 1858 528 1310 772 1392 955 1940 710
% 
\special{pn 8}%
\special{pa 1940 710}%
\special{pa 1858 528}%
\special{pa 1310 772}%
\special{pa 1392 955}%
\special{pa 1940 710}%
\special{pa 1858 528}%
\special{fp}%
% LINE 0 0 3 1 Black White  
% 2 1585 650 1422 285
% 
\special{pn 20}%
\special{pa 1585 650}%
\special{pa 1422 285}%
\special{fp}%
% VECTOR 2 0 3 0 Black White  
% 2 1148 694 748 884
% 
\special{pn 8}%
\special{pa 1148 694}%
\special{pa 748 884}%
\special{fp}%
\special{sh 1}%
\special{pa 748 884}%
\special{pa 817 873}%
\special{pa 796 861}%
\special{pa 800 837}%
\special{pa 748 884}%
\special{fp}%
% LINE 2 1 3 0 Black White  
% 2 408 1404 1808 1404
% 
\special{pn 8}%
\special{pa 408 1404}%
\special{pa 1808 1404}%
\special{da 0.015}%
% CIRCLE 2 0 3 0 Black White  
% 4 408 1404 808 1404 2208 1404 2208 604
% 
\special{pn 8}%
\special{ar 408 1404 400 400 5.8649610 6.2831853}%
% STR 2 0 3 0 Black White  
% 4 858 1254 858 1354 2 0 0 0
% $\theta $
\put(8.5800,-13.5400){\makebox(0,0)[lb]{$\theta $}}%
% STR 2 0 3 0 Black White  
% 4 1608 1465 1608 1565 5 0 0 0
% {\bf 図1}
\put(16.0800,-15.6500){\makebox(0,0){{\bf 図1}}}%
% VECTOR 2 0 3 0 Black White  
% 2 408 3795 408 1995
% 
\special{pn 8}%
\special{pa 408 3795}%
\special{pa 408 1995}%
\special{fp}%
\special{sh 1}%
\special{pa 408 1995}%
\special{pa 388 2062}%
\special{pa 408 2048}%
\special{pa 428 2062}%
\special{pa 408 1995}%
\special{fp}%
% VECTOR 2 0 3 0 Black White  
% 2 408 3795 2608 3795
% 
\special{pn 8}%
\special{pa 408 3795}%
\special{pa 2608 3795}%
\special{fp}%
\special{sh 1}%
\special{pa 2608 3795}%
\special{pa 2541 3775}%
\special{pa 2555 3795}%
\special{pa 2541 3815}%
\special{pa 2608 3795}%
\special{fp}%
% LINE 2 0 3 0 Black White  
% 2 808 3795 808 3845
% 
\special{pn 8}%
\special{pa 808 3795}%
\special{pa 808 3845}%
\special{fp}%
% LINE 2 0 3 0 Black White  
% 2 1200 3791 1200 3841
% 
\special{pn 8}%
\special{pa 1200 3791}%
\special{pa 1200 3841}%
\special{fp}%
% LINE 2 0 3 0 Black White  
% 2 1600 3791 1600 3841
% 
\special{pn 8}%
\special{pa 1600 3791}%
\special{pa 1600 3841}%
\special{fp}%
% LINE 2 0 3 0 Black White  
% 2 2000 3791 2000 3841
% 
\special{pn 8}%
\special{pa 2000 3791}%
\special{pa 2000 3841}%
\special{fp}%
% LINE 2 0 3 0 Black White  
% 2 2400 3791 2400 3841
% 
\special{pn 8}%
\special{pa 2400 3791}%
\special{pa 2400 3841}%
\special{fp}%
% LINE 2 0 3 0 Black White  
% 2 401 3395 351 3395
% 
\special{pn 8}%
\special{pa 401 3395}%
\special{pa 351 3395}%
\special{fp}%
% LINE 2 0 3 0 Black White  
% 2 400 2991 350 2991
% 
\special{pn 8}%
\special{pa 400 2991}%
\special{pa 350 2991}%
\special{fp}%
% LINE 2 0 3 0 Black White  
% 2 400 2591 350 2591
% 
\special{pn 8}%
\special{pa 400 2591}%
\special{pa 350 2591}%
\special{fp}%
% LINE 2 0 3 0 Black White  
% 2 400 2191 350 2191
% 
\special{pn 8}%
\special{pa 400 2191}%
\special{pa 350 2191}%
\special{fp}%
% LINE 2 1 3 0 Black White  
% 4 400 2591 400 2591 800 2591 800 2591
% 
\special{pn 8}%
\special{pa 400 2591}%
\special{pa 400 2591}%
\special{da 0.015}%
\special{pa 800 2591}%
\special{pa 800 2591}%
\special{da 0.015}%
% LINE 2 1 3 0 Black White  
% 4 400 2591 800 2591 800 2591 800 3791
% 
\special{pn 8}%
\special{pa 400 2591}%
\special{pa 800 2591}%
\special{da 0.015}%
\special{pa 800 2591}%
\special{pa 800 3791}%
\special{da 0.015}%
% LINE 2 1 3 0 Black White  
% 2 400 2191 2600 2191
% 
\special{pn 8}%
\special{pa 400 2191}%
\special{pa 2600 2191}%
\special{da 0.015}%
% SPLINE 1 0 3 0 Black White  
% 4 400 3791 1200 2391 2200 2191 2600 2191
% 
\special{pn 13}%
\special{pa 400 3791}%
\special{pa 444 3655}%
\special{pa 454 3621}%
\special{pa 466 3587}%
\special{pa 488 3519}%
\special{pa 499 3486}%
\special{pa 510 3452}%
\special{pa 534 3386}%
\special{pa 545 3353}%
\special{pa 557 3320}%
\special{pa 569 3288}%
\special{pa 582 3255}%
\special{pa 594 3223}%
\special{pa 607 3191}%
\special{pa 646 3098}%
\special{pa 660 3067}%
\special{pa 702 2977}%
\special{pa 747 2890}%
\special{pa 763 2862}%
\special{pa 795 2808}%
\special{pa 812 2781}%
\special{pa 829 2755}%
\special{pa 847 2729}%
\special{pa 883 2679}%
\special{pa 921 2631}%
\special{pa 941 2608}%
\special{pa 981 2564}%
\special{pa 1002 2543}%
\special{pa 1024 2522}%
\special{pa 1046 2502}%
\special{pa 1092 2464}%
\special{pa 1115 2446}%
\special{pa 1140 2429}%
\special{pa 1190 2397}%
\special{pa 1216 2382}%
\special{pa 1270 2354}%
\special{pa 1297 2341}%
\special{pa 1326 2329}%
\special{pa 1354 2318}%
\special{pa 1383 2307}%
\special{pa 1413 2296}%
\special{pa 1473 2278}%
\special{pa 1504 2269}%
\special{pa 1535 2262}%
\special{pa 1567 2254}%
\special{pa 1598 2247}%
\special{pa 1662 2235}%
\special{pa 1728 2225}%
\special{pa 1760 2220}%
\special{pa 1826 2212}%
\special{pa 1860 2209}%
\special{pa 1926 2203}%
\special{pa 1959 2201}%
\special{pa 1993 2199}%
\special{pa 2059 2195}%
\special{pa 2093 2194}%
\special{pa 2192 2191}%
\special{pa 2224 2191}%
\special{pa 2257 2190}%
\special{pa 2321 2190}%
\special{pa 2353 2189}%
\special{pa 2385 2189}%
\special{pa 2417 2190}%
\special{pa 2544 2190}%
\special{pa 2576 2191}%
\special{pa 2600 2191}%
\special{fp}%
% LINE 2 1 3 0 Black White  
% 4 400 3791 800 2591 800 2591 1000 1991
% 
\special{pn 8}%
\special{pa 400 3791}%
\special{pa 800 2591}%
\special{da 0.015}%
\special{pa 800 2591}%
\special{pa 1000 1991}%
\special{da 0.015}%
% STR 2 0 3 0 Black White  
% 4 300 3291 300 3391 5 0 0 0
% 1
\put(3.0000,-33.9100){\makebox(0,0){1}}%
% STR 2 0 3 0 Black White  
% 4 300 2891 300 2991 5 0 0 0
% 2
\put(3.0000,-29.9100){\makebox(0,0){2}}%
% STR 2 0 3 0 Black White  
% 4 300 2491 300 2591 5 0 0 0
% 3
\put(3.0000,-25.9100){\makebox(0,0){3}}%
% STR 2 0 3 0 Black White  
% 4 300 2091 300 2191 5 0 0 0
% 4
\put(3.0000,-21.9100){\makebox(0,0){4}}%
% STR 2 0 3 0 Black White  
% 4 300 1841 300 1941 2 0 0 0
% $v$\kern-4pt{\sf 〔m/s〕}
\put(3.0000,-19.4100){\makebox(0,0)[lb]{$v$\kern-4pt{\sf 〔m/s〕}}}%
% STR 2 0 3 0 Black White  
% 4 2480 3931 2480 4031 2 0 0 0
% $t$\kern-4pt{\sf 〔s〕}
\put(24.8000,-40.3100){\makebox(0,0)[lb]{$t$\kern-4pt{\sf 〔s〕}}}%
% STR 2 0 3 0 Black White  
% 4 408 3835 408 3935 5 0 0 0
% 0
\put(4.0800,-39.3500){\makebox(0,0){0}}%
% STR 2 0 3 0 Black White  
% 4 800 3831 800 3931 5 0 0 0
% 1
\put(8.0000,-39.3100){\makebox(0,0){1}}%
% STR 2 0 3 0 Black White  
% 4 1200 3831 1200 3931 5 0 0 0
% 2
\put(12.0000,-39.3100){\makebox(0,0){2}}%
% STR 2 0 3 0 Black White  
% 4 1600 3831 1600 3931 5 0 0 0
% 3
\put(16.0000,-39.3100){\makebox(0,0){3}}%
% STR 2 0 3 0 Black White  
% 4 2000 3831 2000 3931 5 0 0 0
% 4
\put(20.0000,-39.3100){\makebox(0,0){4}}%
% STR 2 0 3 0 Black White  
% 4 2400 3831 2400 3931 5 0 0 0
% 5
\put(24.0000,-39.3100){\makebox(0,0){5}}%
% STR 2 0 3 0 Black White  
% 4 1608 4065 1608 4165 5 0 0 0
% {\bf 図2}
\put(16.0800,-41.6500){\makebox(0,0){{\bf 図2}}}%
\end{picture}}%
}
    傾角$\theta $の斜面上を{\bf 図1}のようなT型の物体がすべる運動を考える。物体の質量を $M$,動摩擦係数を$\mu $,重力加速度を$g$とする。速さが$v$のとき,空気の抵抗力$kv$がはたらくものとする。
    \begin{Enumerate}
        \item 運動中の物体に作用する力の名称とその向きを,矢印で図の上に示せ。
        \item 物体が速さ$v$,加速度$a$で運動しているときの運動方程式を記せ。
        \item しばらくして,等速度運動になった場合の速さ$v$を求めよ。
    \end{Enumerate}
    $M=2.0$\tanni{kg},$\theta =30\Deg$のとき,{\bf 図2}の曲線のような結果が得られた。\\ なお,{\bf 図2}の斜めの破線は,時刻$t=0$のときの接線とし,$g=10$\tanni{m/s^2}とする。
    \begin{Enumerate*}
        \item 動摩擦係数$\mu $を求めよ。
        \item 空気の抵抗力の係数$k$を求めよ。
    \end{Enumerate*}
\end{mawarikomi}
