\hakosyokika
\item
    \begin{mawarikomi}(10pt,0){160pt}{%WinTpicVersion4.32a
{\unitlength 0.1in%
\begin{picture}(23.6220,15.7480)(5.9055,-19.6850)%
% CIRCLE 1 0 3 0 Black White  
% 4 1400 1200 2200 1200 2200 1200 2200 1200
% 
\special{pn 13}%
\special{ar 1378 1181 787 787 0.0000000 6.2831853}%
% DOT 0 0 3 0 Black White  
% 1 1400 1200
% 
\special{pn 4}%
\special{sh 1}%
\special{ar 1378 1181 16 16 0 6.2831853}%
% STR 2 0 3 0 Black White  
% 4 1400 1010 1400 1110 5 0 0 0
% O
\put(13.7795,-10.9252){\makebox(0,0){O}}%
% CIRCLE 1 0 3 0 Black White  
% 4 1400 1200 2200 1200 800 400 600 400
% 
\special{pn 13}%
\special{ar 1378 1181 787 787 3.9269908 4.0688879}%
% SARROW 1 0 3 1 Black White  
% 2 845 624 834 634
% 
\special{pn 13}%
\special{pa 832 614}%
\special{pa 821 624}%
\special{fp}%
\special{sh 1}%
\special{pa 821 624}%
\special{pa 883 594}%
\special{pa 859 589}%
\special{pa 856 565}%
\special{pa 821 624}%
\special{fp}%
% LINE 2 1 3 0 Black White  
% 2 1400 1200 3000 1200
% 
\special{pn 8}%
\special{pa 1378 1181}%
\special{pa 2953 1181}%
\special{da 0.030}%
% DOT 0 0 3 0 Black White  
% 1 2800 1200
% 
\special{pn 4}%
\special{sh 1}%
\special{ar 2756 1181 16 16 0 6.2831853}%
% STR 2 0 3 0 Black White  
% 4 2800 1000 2800 1100 5 0 0 0
% P
\put(27.5591,-10.8268){\makebox(0,0){P}}%
\end{picture}}%
}
    自動車に振動数$f_0$\tanni{Hz}のサイレンを乗せ,点Oを中心とする半径$r$\tanni{m}の円周上を,
    一定の速さ$v$\tanni{m/s}で左まわりに走らせた。円の外側の点Pに人が立ち,この音を聞くこととし,
    音速を$V$\tanni{m/s}とする。
        \begin{enumerate}
            \item 図の円周上で,最大振動数$f_\mathrm{H}$\tanni{Hz}の音が発せられた点にAを,振動数$f_0$\tanni{Hz}の音が発せられた点にBを,最小振動数$f_\mathrm{L}$\tanni{Hz}の音が発せられた点にCを,それぞれ記入せよ。
            \item $v$と$f_0$を,$f_\mathrm{H}$,$f_\mathrm{L}$,$V$を用いてそれぞれ表せ。
            \item $f_\mathrm{H}=525$\tanni{Hz},$f_\mathrm{L}=495$\tanni{Hz},$V=340$\tanni{m/s}であるとき,$525$\sftanni{Hz}の音を聞く周期が$9.42$\sftanni{s}であった。$v$と$r$はいくらか。
            \item (3)において,距離OPが$2r$\tanni{m}に等しいとき,
                \begin{enumerate}
                    \item $f_\mathrm{H}$\tanni{Hz}の音を聞いて,次に$f_\mathrm{L}$\tanni{Hz}の音を聞くまでには,どれだけの時間がかかるか。
                    \item $f_0$\tanni{Hz}の音を聞いて,次に$f_\mathrm{L}$\tanni{Hz}の音を聞き,再び$f_0$\tanni{Hz}の音を聞くまでには,どれだけの時間がかかるか。
                \end{enumerate}
        \end{enumerate}
    \end{mawarikomi}