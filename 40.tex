\hakosyokika
\item
    \begin{mawarikomi}{150pt}{\begin{zahyou*}[ul=5mm](-5,5)(0,6)
    \small
    \def\O{(0,0)}
    \def\OU{(0,3.05)}
    \def\OUU{(0,4.8)}
    \def\A{(-4,4.73)}
    \def\B{(4,4.73)}
    \def\C{(-3,4.73)}
    \def\D{(3,4.73)}
    \def\P{(-1,2.4)}
    \Drawline{\B\O\A}
    \Daen\OUU{4}{0.8}
    \Put\OU{\Daenko<hasen=[0.5][0.5]>{2.5}{0.8}{0}{180}}
    \Put\OU{\Daenko{2.5}{0.7}{180}{360}}
    \hasen(0,0)(0,2.5)
    \HenKo<henkotype=parallel,
    henkoH=11ex,
    yazirusi=b,
    henkosideb=0.8,
    henkosidet=1.2>\O\OU{$h$}
    \drawline(0,0)(3,0)
    \Kuromaru\P
    \Put\OU{\kaitenkigou<
                        hazimekaku=100,
                        owarikaku=200,
                        tyouhankei=8mm,
                        tanhankei=2mm>[90]}
    \Kakukigou\OU\O\A(-3pt,5pt)[l]{$\theta $}
\end{zahyou*}
}
        図のように軸が鉛直で半頂角$\theta $の円錐のなめらかな内面に沿って,質量$m$\tanni{kg}の小球が高さ$h$\tanni{m}の位置で等速円運動している。重力加速度の大きさを$g$\tanni{m/s^2}とする。
        \begin{enumerate}
            \item 小球の速さ$v$\tanni{m/s}を$\theta $,$m$,$h$,$g$のうち必要なものを用いて表せ。
            \item 小球が円錐面から受ける垂直抗力の大きさ$N$\tanni{N}を$m$,$\theta $,$g$を用いて表せ。
            \item 円運動の周期$T$\tanni{s}を$\theta $,$h$,$g$を用いて表せ。
            \item 円錐面が一定の大きさの加速度$\alpha $\tanni{m/s^2}で上昇しているとき,同じ高さ$h$\tanni{m}で等速円運動させるためには,円錐面に対する小球の速さ$v'$\tanni{m/s}をいくらにすればよいか。$h$,$g$,$\alpha $を用いて表せ。
        \end{enumerate}
    \end{mawarikomi}