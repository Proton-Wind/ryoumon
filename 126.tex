\hakosyokika
\item
    \begin{mawarikomi}(10pt,0pt){120pt}{
        %WinTpicVersion4.32a
{\unitlength 0.1in%
\begin{picture}(22.0000,14.2300)(7.1200,-24.6700)%
% BOX 2 5 1 0 Black Black  
% 2 1817 1107 2912 2467
% 
\special{pn 0}%
\special{sh 0.200}%
\special{pa 1817 1107}%
\special{pa 2912 1107}%
\special{pa 2912 2467}%
\special{pa 1817 2467}%
\special{pa 1817 1107}%
\special{ip}%
\special{pn 8}%
\special{pa 1817 1107}%
\special{pa 2912 1107}%
\special{pa 2912 2467}%
\special{pa 1817 2467}%
\special{pa 1817 1107}%
\special{ip}%
% BOX 1 0 3 0 Black White  
% 2 918 1431 1466 1979
% 
\special{pn 13}%
\special{pa 918 1431}%
\special{pa 1466 1431}%
\special{pa 1466 1979}%
\special{pa 918 1979}%
\special{pa 918 1431}%
\special{pa 1466 1431}%
\special{fp}%
% CIRCLE 2 0 3 0 Black White  
% 4 1192 1979 1466 1431 1466 1431 918 1431
% 
\special{pn 8}%
\special{ar 1192 1979 613 613 4.2487414 5.1760366}%
% CIRCLE 2 0 3 0 Black White  
% 4 1466 1705 918 1431 918 1431 918 1979
% 
\special{pn 8}%
\special{ar 1466 1705 613 613 2.6779450 3.6052403}%
% STR 2 0 3 0 Black White  
% 4 1192 1274 1192 1342 5 0 1 0
% $a$
\put(11.9200,-13.4200){\makebox(0,0){{\colorbox[named]{White}{$a$}}}}%
% STR 2 0 3 0 Black White  
% 4 802 1637 802 1705 5 0 1 0
% $b$
\put(8.0200,-17.0500){\makebox(0,0){{\colorbox[named]{White}{$b$}}}}%
% VECTOR 2 0 3 0 Black White  
% 2 1493 1705 1767 1705
% 
\special{pn 8}%
\special{pa 1493 1705}%
\special{pa 1767 1705}%
\special{fp}%
\special{sh 1}%
\special{pa 1767 1705}%
\special{pa 1700 1685}%
\special{pa 1714 1705}%
\special{pa 1700 1725}%
\special{pa 1767 1705}%
\special{fp}%
% STR 2 0 3 0 Black White  
% 4 1616 1545 1616 1613 5 0 1 0
% $v$
\put(16.1600,-16.1300){\makebox(0,0){{\colorbox[named]{White}{$v$}}}}%
% LINE 2 0 3 0 Black White  
% 2 1815 1096 1815 2465
% 
\special{pn 8}%
\special{pa 1815 1096}%
\special{pa 1815 2465}%
\special{fp}%
% LINE 2 0 3 0 Black White  
% 2 2910 2465 2910 1096
% 
\special{pn 8}%
\special{pa 2910 2465}%
\special{pa 2910 1096}%
\special{fp}%
% STR 2 0 3 0 Black Black  
% 4 2362 1712 2362 1780 5 0 0 0
% $\otimes$
\put(23.6200,-17.8000){\makebox(0,0){$\otimes$}}%
% STR 2 0 3 0 Black Black  
% 4 2424 1589 2424 1657 2 0 0 0
% $B$
\put(24.2400,-16.5700){\makebox(0,0)[lb]{$B$}}%
% VECTOR 2 0 3 0 Black Black  
% 4 2362 1109 1815 1109 2362 1109 2910 1109
% 
\special{pn 8}%
\special{pa 2362 1109}%
\special{pa 1815 1109}%
\special{fp}%
\special{sh 1}%
\special{pa 1815 1109}%
\special{pa 1882 1129}%
\special{pa 1868 1109}%
\special{pa 1882 1089}%
\special{pa 1815 1109}%
\special{fp}%
\special{pa 2362 1109}%
\special{pa 2910 1109}%
\special{fp}%
\special{sh 1}%
\special{pa 2910 1109}%
\special{pa 2843 1089}%
\special{pa 2857 1109}%
\special{pa 2843 1129}%
\special{pa 2910 1109}%
\special{fp}%
% STR 2 0 3 0 Black Black  
% 4 2362 1041 2362 1109 5 0 1 0
% $2a$
\put(23.6200,-11.0900){\makebox(0,0){{\colorbox[named]{White}{$2a$}}}}%
% STR 2 0 3 0 Black White  
% 4 860 1290 860 1390 3 0 0 0
% A
\put(8.6000,-13.9000){\makebox(0,0)[rb]{A}}%
% STR 2 0 3 0 Black White  
% 4 1490 1290 1490 1390 2 0 0 0
% B
\put(14.9000,-13.9000){\makebox(0,0)[lb]{B}}%
% STR 2 0 3 0 Black White  
% 4 1490 1910 1490 2010 1 0 0 0
% C
\put(14.9000,-20.1000){\makebox(0,0)[lt]{C}}%
% STR 2 0 3 0 Black White  
% 4 860 1910 860 2010 4 0 0 0
% D
\put(8.6000,-20.1000){\makebox(0,0)[rt]{D}}%
\end{picture}}%
\\
        \\
        %WinTpicVersion4.32a
{\unitlength 0.1in%
\begin{picture}(20.8563,19.7047)(3.8189,-25.6102)%
% VECTOR 2 0 3 0 Black White  
% 2 507 2602 507 602
% 
\special{pn 8}%
\special{pa 499 2561}%
\special{pa 499 593}%
\special{fp}%
\special{sh 1}%
\special{pa 499 593}%
\special{pa 479 658}%
\special{pa 499 645}%
\special{pa 519 658}%
\special{pa 499 593}%
\special{fp}%
% LINE 2 1 3 0 Black White  
% 2 507 1302 2507 1302
% 
\special{pn 8}%
\special{pa 499 1281}%
\special{pa 2468 1281}%
\special{da 0.030}%
% LINE 2 1 3 0 Black White  
% 2 505 1000 2505 1000
% 
\special{pn 8}%
\special{pa 497 984}%
\special{pa 2466 984}%
\special{da 0.030}%
% LINE 2 1 3 0 Black White  
% 2 505 700 2505 700
% 
\special{pn 8}%
\special{pa 497 689}%
\special{pa 2466 689}%
\special{da 0.030}%
% LINE 2 1 3 0 Black White  
% 2 505 1900 2505 1900
% 
\special{pn 8}%
\special{pa 497 1870}%
\special{pa 2466 1870}%
\special{da 0.030}%
% LINE 2 1 3 0 Black White  
% 2 505 2200 2505 2200
% 
\special{pn 8}%
\special{pa 497 2165}%
\special{pa 2466 2165}%
\special{da 0.030}%
% LINE 2 1 3 0 Black White  
% 2 505 2500 2505 2500
% 
\special{pn 8}%
\special{pa 497 2461}%
\special{pa 2466 2461}%
\special{da 0.030}%
% LINE 2 1 3 0 Black White  
% 2 807 602 807 2602
% 
\special{pn 8}%
\special{pa 794 593}%
\special{pa 794 2561}%
\special{da 0.030}%
% LINE 2 1 3 0 Black White  
% 2 1105 600 1105 2600
% 
\special{pn 8}%
\special{pa 1088 591}%
\special{pa 1088 2559}%
\special{da 0.030}%
% LINE 2 1 3 0 Black White  
% 2 1405 600 1405 2600
% 
\special{pn 8}%
\special{pa 1383 591}%
\special{pa 1383 2559}%
\special{da 0.030}%
% LINE 2 1 3 0 Black White  
% 2 1705 600 1705 2600
% 
\special{pn 8}%
\special{pa 1678 591}%
\special{pa 1678 2559}%
\special{da 0.030}%
% LINE 2 1 3 0 Black White  
% 2 2005 600 2005 2600
% 
\special{pn 8}%
\special{pa 1973 591}%
\special{pa 1973 2559}%
\special{da 0.030}%
% LINE 2 1 3 0 Black White  
% 2 2305 600 2305 2600
% 
\special{pn 8}%
\special{pa 2269 591}%
\special{pa 2269 2559}%
\special{da 0.030}%
% VECTOR 2 0 3 0 Black White  
% 2 505 1600 2505 1600
% 
\special{pn 8}%
\special{pa 497 1575}%
\special{pa 2466 1575}%
\special{fp}%
\special{sh 1}%
\special{pa 2466 1575}%
\special{pa 2400 1555}%
\special{pa 2413 1575}%
\special{pa 2400 1594}%
\special{pa 2466 1575}%
\special{fp}%
% STR 2 0 3 0 Black White  
% 4 2500 1650 2500 1700 5 0 0 0
% $t$
\put(24.6063,-16.7323){\makebox(0,0){$t$}}%
% STR 2 0 3 0 Black White  
% 4 420 1550 420 1600 5 0 0 0
% 0
\put(4.1339,-15.7480){\makebox(0,0){0}}%
% LINE 2 0 3 0 Black White  
% 2 1105 1600 1105 1550
% 
\special{pn 8}%
\special{pa 1088 1575}%
\special{pa 1088 1526}%
\special{fp}%
% LINE 2 0 3 0 Black White  
% 2 1705 1600 1705 1550
% 
\special{pn 8}%
\special{pa 1678 1575}%
\special{pa 1678 1526}%
\special{fp}%
% LINE 2 0 3 0 Black White  
% 2 2305 1600 2305 1550
% 
\special{pn 8}%
\special{pa 2269 1575}%
\special{pa 2269 1526}%
\special{fp}%
% STR 2 0 3 0 Black White  
% 4 1105 1750 1105 1800 5 0 1 0
% $\bunsuu{a}{v}$
\put(10.8760,-17.7165){\makebox(0,0){{\colorbox[named]{White}{$\bunsuu{a}{v}$}}}}%
% STR 2 0 3 0 Black White  
% 4 1705 1750 1705 1800 5 0 1 0
% $\bunsuu{2a}{v}$
\put(16.7815,-17.7165){\makebox(0,0){{\colorbox[named]{White}{$\bunsuu{2a}{v}$}}}}%
% STR 2 0 3 0 Black White  
% 4 2305 1750 2305 1800 5 0 1 0
% $\bunsuu{3a}{v}$
\put(22.6870,-17.7165){\makebox(0,0){{\colorbox[named]{White}{$\bunsuu{3a}{v}$}}}}%
\end{picture}}%
\\
        \\
        %WinTpicVersion4.32a
{\unitlength 0.1in%
\begin{picture}(20.8563,19.7047)(3.8189,-25.6102)%
% VECTOR 2 0 3 0 Black White  
% 2 507 2602 507 602
% 
\special{pn 8}%
\special{pa 499 2561}%
\special{pa 499 593}%
\special{fp}%
\special{sh 1}%
\special{pa 499 593}%
\special{pa 479 658}%
\special{pa 499 645}%
\special{pa 519 658}%
\special{pa 499 593}%
\special{fp}%
% LINE 2 1 3 0 Black White  
% 2 507 1302 2507 1302
% 
\special{pn 8}%
\special{pa 499 1281}%
\special{pa 2468 1281}%
\special{da 0.030}%
% LINE 2 1 3 0 Black White  
% 2 505 1000 2505 1000
% 
\special{pn 8}%
\special{pa 497 984}%
\special{pa 2466 984}%
\special{da 0.030}%
% LINE 2 1 3 0 Black White  
% 2 505 700 2505 700
% 
\special{pn 8}%
\special{pa 497 689}%
\special{pa 2466 689}%
\special{da 0.030}%
% LINE 2 1 3 0 Black White  
% 2 505 1900 2505 1900
% 
\special{pn 8}%
\special{pa 497 1870}%
\special{pa 2466 1870}%
\special{da 0.030}%
% LINE 2 1 3 0 Black White  
% 2 505 2200 2505 2200
% 
\special{pn 8}%
\special{pa 497 2165}%
\special{pa 2466 2165}%
\special{da 0.030}%
% LINE 2 1 3 0 Black White  
% 2 505 2500 2505 2500
% 
\special{pn 8}%
\special{pa 497 2461}%
\special{pa 2466 2461}%
\special{da 0.030}%
% LINE 2 1 3 0 Black White  
% 2 807 602 807 2602
% 
\special{pn 8}%
\special{pa 794 593}%
\special{pa 794 2561}%
\special{da 0.030}%
% LINE 2 1 3 0 Black White  
% 2 1105 600 1105 2600
% 
\special{pn 8}%
\special{pa 1088 591}%
\special{pa 1088 2559}%
\special{da 0.030}%
% LINE 2 1 3 0 Black White  
% 2 1405 600 1405 2600
% 
\special{pn 8}%
\special{pa 1383 591}%
\special{pa 1383 2559}%
\special{da 0.030}%
% LINE 2 1 3 0 Black White  
% 2 1705 600 1705 2600
% 
\special{pn 8}%
\special{pa 1678 591}%
\special{pa 1678 2559}%
\special{da 0.030}%
% LINE 2 1 3 0 Black White  
% 2 2005 600 2005 2600
% 
\special{pn 8}%
\special{pa 1973 591}%
\special{pa 1973 2559}%
\special{da 0.030}%
% LINE 2 1 3 0 Black White  
% 2 2305 600 2305 2600
% 
\special{pn 8}%
\special{pa 2269 591}%
\special{pa 2269 2559}%
\special{da 0.030}%
% VECTOR 2 0 3 0 Black White  
% 2 505 1600 2505 1600
% 
\special{pn 8}%
\special{pa 497 1575}%
\special{pa 2466 1575}%
\special{fp}%
\special{sh 1}%
\special{pa 2466 1575}%
\special{pa 2400 1555}%
\special{pa 2413 1575}%
\special{pa 2400 1594}%
\special{pa 2466 1575}%
\special{fp}%
% STR 2 0 3 0 Black White  
% 4 2500 1650 2500 1700 5 0 0 0
% $t$
\put(24.6063,-16.7323){\makebox(0,0){$t$}}%
% STR 2 0 3 0 Black White  
% 4 420 1550 420 1600 5 0 0 0
% 0
\put(4.1339,-15.7480){\makebox(0,0){0}}%
% LINE 2 0 3 0 Black White  
% 2 1105 1600 1105 1550
% 
\special{pn 8}%
\special{pa 1088 1575}%
\special{pa 1088 1526}%
\special{fp}%
% LINE 2 0 3 0 Black White  
% 2 1705 1600 1705 1550
% 
\special{pn 8}%
\special{pa 1678 1575}%
\special{pa 1678 1526}%
\special{fp}%
% LINE 2 0 3 0 Black White  
% 2 2305 1600 2305 1550
% 
\special{pn 8}%
\special{pa 2269 1575}%
\special{pa 2269 1526}%
\special{fp}%
% STR 2 0 3 0 Black White  
% 4 1105 1750 1105 1800 5 0 1 0
% $\bunsuu{a}{v}$
\put(10.8760,-17.7165){\makebox(0,0){{\colorbox[named]{White}{$\bunsuu{a}{v}$}}}}%
% STR 2 0 3 0 Black White  
% 4 1705 1750 1705 1800 5 0 1 0
% $\bunsuu{2a}{v}$
\put(16.7815,-17.7165){\makebox(0,0){{\colorbox[named]{White}{$\bunsuu{2a}{v}$}}}}%
% STR 2 0 3 0 Black White  
% 4 2305 1750 2305 1800 5 0 1 0
% $\bunsuu{3a}{v}$
\put(22.6870,-17.7165){\makebox(0,0){{\colorbox[named]{White}{$\bunsuu{3a}{v}$}}}}%
\end{picture}}%
\\

    }
    辺の長さ$a$\tanni{m}と$b$\tanni{m}の長方形のコイルABCDがあり(以下,Pとよぶ),その全抵抗は$R$\tanni{\Omega }である。灰色で示した幅$2a$\tanni{m}の領域には,紙面に垂直に表から裏へ向かう方向に,磁束密度$B$\tanni{T}の一様な磁場が加えられている。Pを磁場に垂直な方向に一定の速度$v$\tanni{m/s}で動かす。時間を$t$\tanni{s}で表し,辺BCが磁場領域に達したときを$t=0$\tanni{s}とする。
        \begin{enumerate}
            \item $0\leqq t \leqq \bunsuu{a}{v}$において,Pに誘導される起電力の大きさは\Hako \tanni{V}であるから,Pを流れる電流の強さは\Hako \tanni{A}で,Pは磁場から\Hako の向きに\Hako \tanni{N}の力を受けている。
            \item $0\leqq t\leqq \bunsuu{3a}{v}$において,Pを流れる電流と$t$の関係を図に示せ。最大値と最小値も示すこと。ただし,時計回りを電流の正の向きとする。
            \item $0\leqq t\leqq \bunsuu{3a}{v}$において,Pに加えている外力と$t$との関係を図に示せ。最大値と最小値も示すこと。ただし,Pの速度の向きを力の正の向きとする。
            \item Pが磁場領域を完全に通過し終える間に,発生した全熱エネルギーを求めよ。
        \end{enumerate}
    \end{mawarikomi}