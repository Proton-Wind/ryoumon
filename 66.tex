\hakosyokika
\item
    \begin{mawarikomi}(20pt,0){150pt}{{\small
\begin{zahyou*}[ul=4mm](0,15)(0,7)
    \def\A{(0,0)}
    \def\B{(6,0)}
    \def\C{(6,3.25)}
    \def\D{(9,3.25)}
    \def\E{(9,3.75)}
    \def\F{(6,3.75)}
    \def\G{(6,7)}
    \def\H{(0,7)}
    \def\I{(2,0)}
    \def\J{(3,0)}
    \def\K{(3,7)}
    \def\L{(2,7)}
    \def\M{(0,4)}
    \def\N{(2,4)}
    \def\O{(0,3)}
    \def\P{(2,3)}
    \def\Q{(12,3.5)}
    \def\R{(7.5,3.5)}
    \En*[0]\Q{3.05}
    \Nuritubusi[0]{\C\D\E\F\C}
    \Drawlines{\A\B\C\D;\E\F\G\H;\I\L;\J\K;\M\N;\O\P}
    \Nuritubusi*{\I\J\K\L\I}
    \En*[0]\R{0.5}
    \Put\Q(0pt,-50pt)[b]{B}
    \Put\I(0pt,-10pt)[b]{A}
    \Put\R(0,-5pt)[b]{弁}
    % \Put\B(0,-15pt)[b]{$3p$}
    % \Put\A(0,10pt)[b]{1モル}
    % \Put\B(0,10pt)[b]{2モル}
    % \Put\K(-3pt,0pt)[b]{K}
\end{zahyou*}}
}
        なめらかに動くピストンとシリンダーからなる容器Aと,容積$V$の容器Bがあり,その間はごく細い管とこれを開閉できる弁で連結されている。器材は熱を伝えない材料でできている。容器のAに単原子分子の理想気体を入れ,体積$V$,圧力$p$,絶対温度$T$の状態でピストンは固定され,弁は閉じられている。また,容器Bには同種の理想気体が圧力$2p$,絶対温度$T$で封じ込められている。気体定数を$R$とする。
        \begin{enumerate}
            \item 容器A内には何モルの機体が入っているか。また,気体の内部エネルギー$U$はいくらか。
            \item ピストンを動かし容器Aの体積を$\bunsuu{V}{8}$に圧縮する。A内の気体の圧力と温度はいくらになるか。ただし,この断熱変化では気体の圧力$p$と体積$V$の間には,$pV^{\bunsuu{5}{3}}=一定$という関係がある。
            \item 上の状態でピストンを固定したまま,弁を開けて十分長い時間放置する。そのときの容器内の気体の温度および圧力はいくらになるか。
        \end{enumerate}
    \end{mawarikomi}