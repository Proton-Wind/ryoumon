\hakosyokika
\item
    \begin{mawarikomi}{150pt}{\begin{zahyou*}[ul=6mm](-5,3)(-1,9)
    \small
    \def\A{(-1,9)}
    \def\B{(-1,8.5)}
    \def\C{(1,8.5)}
    \def\D{(1,9)}
    \def\E{(-4,0)}
    \def\F{(-4,-0.5)}
    \def\G{(2,-0.5)}
    \def\H{(2,0)}
    \def\M{(0,8.5)}
    \def\L{(0,0)}
    \def\P{(0,4.5)}
    \def\Q{(-3,2.5)}
    \Hasen{\M\L}
    \Nuritubusi{\A\B\C\D\A}
    \Nuritubusi{\E\F\G\H\E}
    \Drawline{\B\C}
    \Drawline{\E\H}
    \Drawline{\M\Q}
    \Kuromaru{\P}
    \En*[1]\Q{0.2}
    \Suisen\Q\E\H\QD
    \Suisen\P\B\C\PU
    \HenKo<henkotype=parallel,
    henkoH=2ex,
    yazirusi=b,
    henkosideb=0,
    henkosidet=1.2>\Q\QD{$h_0$}
    \HenKo<henkotype=parallel,
    henkoH=2ex,
    yazirusi=b,
    henkosideb=0,
    henkosidet=1.2>\P\PU{$d$}
    \HenKo<henkotype=parallel,
    henkoH=5ex,
    yazirusi=b,
    henkosideb=0,
    henkosidet=1.2>\L\M{$H$}
    \Put\P[w]{くぎ}
    \Put\Q(6pt,0pt)[l]{$m$}
    \Put\L(6pt,5pt)[l]{床}
    \Put\B[w]{天井}
    \HenKo[0]\M\Q{$L$}
	% \def\kakudo{-30}
	% \def\kakudo{-30}
	% \calcval{\kakudo *2*$pi/360}\TH
	% \def\O{(0,0)}
    % \def\A{(-4.35,0.4)}
    % \def\B{(-8,0.4)}
    % \def\AL{(-4.75,0)}
    % \def\BR{(-7.6,0)}
    % \def\C{(-4.75,1.4)}
    % \def\CU{(-4.75,1.8)}
    % \def\D{(-7,1.6)}
    % \def\E{(-0.8,1.6)}
	% \def\G{(-9.5,0)}
    % \def\J{(0,1)}
	% \def\Dy{-0.1}
	% \calcval{\Dy*(1/tan(\TH))}\Dx
	% \Kaiten\O\C{\kakudo}\CC
	% \Kaiten\O\CU{\kakudo}\CCU
	% \Kaiten\O\D{\kakudo}\DD
	% \Kaiten\O\E{\kakudo}\EE
	% \Kaiten\O\A{\kakudo}\AA
	% \Kaiten\O\B{\kakudo}\BB
	% \Kaiten\O\AL{\kakudo}\AAL
	% \Kaiten\O\BR{\kakudo}\BBR
	% \Kaiten\O\G{\kakudo}\GG
	% \Kaiten\O\J{\kakudo}\JJ
	% \def\Fx{(-0.18*(cos(T)+T/3.5))*cos(\TH)-(-0.3*sin(T))*sin(\TH)+\Dx}
	% \def\Fy{(-0.18*(cos(T)+T/3.5))*sin(\TH)+(-0.3*sin(T))*cos(\TH)+\Dy+0.5}
    % \hasen(-9,0)(1,0)
    % \Drawline{\CC\CCU}
    % \Arrowline{\DD\EE}
    % \Put\CCU{O}
    % \Put\EE{$x$}
    % \Put\AA(3pt,10pt)[l]{$M$}
    % \Put\BB(-1pt,10pt)[r]{$m$}
    % \Put\AAL[sw]{A}
    % \Put\BBR[sw]{B}
	% \BGurafu\Fx\Fy{$pi}{26*$\pi}
    % \En**\AA{0.4}
    % \En\AA{0.4}
    % \En**\BB{0.4}
    % \En\BB{0.4}
	% \Drawline{\JJ\O\GG}
	% \Kakukigou\GG\O\G<hankei=1.5>(-2pt,0.8pt)[r]{30\Deg}
    % \HenKo<henkotype=parallel,
    % henkoH=4ex,
    % yazirusi=b,
    % henkosideb=0,
    % henkosidet=1.1>\AAL\BBR{$d$}
\end{zahyou*}
}
        天井から長さ$L$の糸で質量$m$のおもりをつるし,支点から真下$d$の位置に細くてなめらかなくぎを固定する。おもりを水平な床から高さ$h_0$の位置で静かに放す。天井の高さは$H$で,糸はゆるむことがなく,重力加速度の大きさを$g$とする。
        \begin{enumerate}
            \item 糸がくぎにひっかかった後,最初に静止したときのおもりの床からの高さを求めよ。また,おもりの速さの最大値を求めよ。
            \item おもりが運動を始めてから,最初の位置に戻ってくるまでの時間を求めよ。ただし,おもりの振れ幅は十分小さいものとする。
            \item 同じ実験を,鉛直上向きに一定の加速度$a$で上昇するエレベーター内で行うときについて,(1),(2)の問いに答えよ。
        \end{enumerate}
    \end{mawarikomi}