\hakosyokika
\item
    \begin{mawarikomi}{150pt}{\begin{zahyou*}[ul=7mm](-4.5,4.5)(-3.5,3.5)
    \small
    \def\OL{(0,0)}
    \def\OM{(-1,0)}
    \def\O{(-2,0)}
    \def\A{(-4,0)}
    \def\B{(4,0)}
    \En*\O{1}
    \En\O{2}
    \Daen\OL{4}{3.5}
    \Hasen{\A\B}
    \HenKo\OM\O{$R$}
    \HenKo\A\O{$2R$}
    \HenKo\O\B{$6R$}
    \Put\A[w]{A}
    \Put\B[e]{B}
    \Put\O[s]{O}
    \Put\O(0,-30pt)[b]{地球}
    {\thicklines
    \changeArrowHeadSize[15]<0.33>{2}
    \Put\A{\yasen(0,1.8)}
    \Put\B{\yasen(0,-0.6)}
    \put(-3,0){\yasen(-0.7,0)}
    }
    \put(-3.5,0.2){$v_0$}
    \put(-4.3,1.7){$v$}
    \put(4.1,-0.6){$V$}
\end{zahyou*}
}
        地表から鉛直上方へ質量$m$\tanni{kg}の小物体を打ち上げる。\\
        地球は半径$R$\tanni{m},質量$M$\tanni{kg}の一様な球で,万有引力定数を$G$\tanni{N\cdot m^2/kg^2}とする。
        \begin{enumerate}
            \item 物体の速度が地球の中心Oから$2R$の距離にある点Aで0になるためには,初速$v_0$\tanni{m/s}をどれだけにすればよいか。
            \item 物体が点Aで静止した瞬間,物体にOAに垂直な方向の速度$v$\tanni{m/s}を与える。物体がOを中心とする等速円運動をするためには,$v$はどれだけにすればよいか。また円運動の周期$T_0$\tanni{s}を求めよ。
            \item 点Aでは物体に与える速さ$v$が問(2)で求められた値からずれると,物体の軌道は楕円となる。物体がABを長軸とする楕円軌道を描くためには,$v$をどれだけにすればよいか。以下の手順で求めよ。
            \begin{enumerate}
                \item 点Aと点Bにおける面積速度に注目して,点Bにおける速さ$V$\tanni{m/s}を$v$を用いて表せ。
                \item 速さ$v$を求め,$M$,$R$,$G$を用いて表せ。
                \item この楕円軌道の周期$T$\tanni{s}を$T_0$を用いて表せ。
            \end{enumerate}
        \end{enumerate}
    \end{mawarikomi}