\hakosyokika
\item 油膜が水面に広がっていて,空気中での波長が$6.0\times 10^{-7}$\sftanni{m}の光がこの油膜へ垂直に入射している。空気,水,および油膜の屈折率はそれぞれ1.0,1.3,1.5とし,空気中の光速を$3.0\times 10^8$\sftanni{m/s}とする。
    \begin{enumerate}
            \item 油膜中での光の速さと波長はいくらか。
            \item 油膜の表面と裏面で反射した光が干渉によって強めあう膜の最小の厚さはいくらか。
            \item 油膜の厚さを前問から厚くしていった場合,次に強めあう膜の厚さはいくらか。
            \item 波長$6.0\times 10^{-7}$\sftanni{m}の光では強めあい,波長$4.5\times 10^{-7}$\sftanni{m}の光では弱めあう膜の最小の厚さはいくらか。
    \end{enumerate}
