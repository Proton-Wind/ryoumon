\hakosyokika
\item
    \begin{mawarikomi}(20pt,0pt){150pt}{%%% C:/vpn/vpn/KeTCindy/fig/fig105.tex 
%%% Generator=fig105.cdy 
{\unitlength=1cm%
\begin{picture}%
(6,5)(-3,-2)%
\special{pn 8}%
%
\special{pa   591  -236}\special{pa   984  -236}%
\special{fp}%
\special{pn 16}%
\special{pa   709  -157}\special{pa   866  -157}%
\special{fp}%
\special{pn 8}%
\special{pa   787   394}\special{pa   787  -157}%
\special{fp}%
\special{pa   787  -787}\special{pa   787  -236}%
\special{fp}%
\special{pa  -157  -787}\special{pa   115  -945}%
\special{fp}%
\special{pa  -787  -787}\special{pa  -157  -787}%
\special{fp}%
\special{pa   787  -787}\special{pa   157  -787}%
\special{fp}%
\special{pa  -591  -138}\special{pa  -984  -138}%
\special{fp}%
\special{pa  -591  -256}\special{pa  -984  -256}%
\special{fp}%
\special{pa  -787  -787}\special{pa  -787  -256}%
\special{fp}%
\special{pa  -787   394}\special{pa  -787  -138}%
\special{fp}%
\special{pa  -787   394}\special{pa   787   394}%
\special{fp}%
\settowidth{\Width}{$V_0$}\setlength{\Width}{-1\Width}%
\settoheight{\Height}{$V_0$}\settodepth{\Depth}{$V_0$}\setlength{\Height}{\Depth}%
\put(  1.250,  0.550){\hspace*{\Width}\raisebox{\Height}{$V_0$}}%
%
\settowidth{\Width}{S}\setlength{\Width}{0\Width}%
\settoheight{\Height}{S}\settodepth{\Depth}{S}\setlength{\Height}{\Depth}%
\put(  0.350,  2.200){\hspace*{\Width}\raisebox{\Height}{S}}%
%
\settowidth{\Width}{金属板}\setlength{\Width}{0\Width}%
\settoheight{\Height}{金属板}\settodepth{\Depth}{金属板}\setlength{\Height}{\Depth}%
\put( -1.700,  0.850){\hspace*{\Width}\raisebox{\Height}{金属板}}%
%
\settowidth{\Width}{$d$}\setlength{\Width}{0\Width}%
\settoheight{\Height}{$d$}\settodepth{\Depth}{$d$}\setlength{\Height}{-0.5\Height}\setlength{\Depth}{0.5\Depth}\addtolength{\Height}{\Depth}%
\put( -1.240,  0.500){\hspace*{\Width}\raisebox{\Height}{$d$}}%
%
\settowidth{\Width}{電池}\setlength{\Width}{0\Width}%
\settoheight{\Height}{電池}\settodepth{\Depth}{電池}\setlength{\Height}{-\Height}%
\put(  1.250,  0.250){\hspace*{\Width}\raisebox{\Height}{電池}}%
%
\special{pa -551 -209}\special{pa -566 -254}\special{pa -580 -209}\special{pa -566 -218}%
\special{pa -551 -209}\special{pa -551 -209}\special{sh 1}\special{ip}%
\special{pn 1}%
\special{pa  -551  -209}\special{pa  -566  -254}\special{pa  -580  -209}\special{pa  -566  -218}%
\special{pa  -551  -209}%
\special{fp}%
\special{pn 8}%
\special{pa  -566  -196}\special{pa  -566  -218}%
\special{fp}%
\special{pa -581 -187}\special{pa -568 -142}\special{pa -551 -186}\special{pa -566 -178}%
\special{pa -581 -187}\special{pa -581 -187}\special{sh 1}\special{ip}%
\special{pn 1}%
\special{pa  -581  -187}\special{pa  -568  -142}\special{pa  -551  -186}\special{pa  -566  -178}%
\special{pa  -581  -187}%
\special{fp}%
\special{pn 8}%
\special{pa  -566  -196}\special{pa  -566  -178}%
\special{fp}%
\special{pa -772 394}\special{pa -773 392}\special{pa -773 390}\special{pa -773 388}%
\special{pa -774 386}\special{pa -775 385}\special{pa -776 383}\special{pa -778 382}%
\special{pa -779 381}\special{pa -781 380}\special{pa -783 379}\special{pa -785 379}%
\special{pa -786 379}\special{pa -788 379}\special{pa -790 379}\special{pa -792 379}%
\special{pa -794 380}\special{pa -795 381}\special{pa -797 382}\special{pa -798 383}%
\special{pa -800 385}\special{pa -801 386}\special{pa -801 388}\special{pa -802 390}%
\special{pa -802 392}\special{pa -802 394}\special{pa -802 396}\special{pa -802 397}%
\special{pa -801 399}\special{pa -801 401}\special{pa -800 402}\special{pa -798 404}%
\special{pa -797 405}\special{pa -795 406}\special{pa -794 407}\special{pa -792 408}%
\special{pa -790 408}\special{pa -788 409}\special{pa -786 409}\special{pa -785 408}%
\special{pa -783 408}\special{pa -781 407}\special{pa -779 406}\special{pa -778 405}%
\special{pa -776 404}\special{pa -775 402}\special{pa -774 401}\special{pa -773 399}%
\special{pa -773 397}\special{pa -773 396}\special{pa -772 394}\special{pa -772 394}%
\special{sh 1}\special{ip}%
\special{pa  -772   394}\special{pa  -773   392}\special{pa  -773   390}\special{pa  -773   388}%
\special{pa  -774   386}\special{pa  -775   385}\special{pa  -776   383}\special{pa  -778   382}%
\special{pa  -779   381}\special{pa  -781   380}\special{pa  -783   379}\special{pa  -785   379}%
\special{pa  -786   379}\special{pa  -788   379}\special{pa  -790   379}\special{pa  -792   379}%
\special{pa  -794   380}\special{pa  -795   381}\special{pa  -797   382}\special{pa  -798   383}%
\special{pa  -800   385}\special{pa  -801   386}\special{pa  -801   388}\special{pa  -802   390}%
\special{pa  -802   392}\special{pa  -802   394}\special{pa  -802   396}\special{pa  -802   397}%
\special{pa  -801   399}\special{pa  -801   401}\special{pa  -800   402}\special{pa  -798   404}%
\special{pa  -797   405}\special{pa  -795   406}\special{pa  -794   407}\special{pa  -792   408}%
\special{pa  -790   408}\special{pa  -788   409}\special{pa  -786   409}\special{pa  -785   408}%
\special{pa  -783   408}\special{pa  -781   407}\special{pa  -779   406}\special{pa  -778   405}%
\special{pa  -776   404}\special{pa  -775   402}\special{pa  -774   401}\special{pa  -773   399}%
\special{pa  -773   397}\special{pa  -773   396}\special{pa  -772   394}%
\special{fp}%
\special{pa  -787   394}\special{pa  -787   591}%
\special{fp}%
\special{pa  -886   591}\special{pa  -689   591}%
\special{fp}%
\special{pa  -846   630}\special{pa  -728   630}%
\special{fp}%
\special{pa  -807   669}\special{pa  -768   669}%
\special{fp}%
\end{picture}}%}
        共に面積$S$\tanni{m^2}の2枚の金属板を距離$d$\tanni{m}だけ離して平行板コンデンサーをつくった。
        このコンデンサーに起電力$V_0$\tanni{V}の電池とスイッチSをつなぎ,Sを閉じてから十分に時間がたった
        (以下,これをはじめの状態とする)。真空の誘電率を$\varepsilon _0$\tanni{F/m}とする。
        \begin{enumerate}
            \item コンデンサーの電気量$Q_0$,極板間の電場の強さ$E_0$,静電エネルギー$U_0$はそれぞいくらか。
            \item スイッチSを閉じたまま,コンデンサーの極板間隔を$2d$に広げた。コンデンサーの電気量と電場はそれぞれ何倍になるか。
            \item はじめの状態に戻し,スイッチSを開き,極板間隔を$2d$に広げた。極板間の電場,電位差,静電エネルギーはそれぞれ何倍になるか。
            \item (3)に続いて,極板と同型で,厚さ$d$,比誘電率2の誘電体を極板間に入れた。極板間の電位差$V_1$を$V_0$で表せ。
        \end{enumerate}
    \end{mawarikomi}
