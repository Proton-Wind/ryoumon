\hakosyokika
\item
    \begin{mawarikomi}(20pt,0pt){150pt}{\input{./fig/fig105.tex}}
        共に面積$S$\tanni{m^2}の2枚の金属板を距離$d$\tanni{m}だけ離して平行板コンデンサーをつくった。
        このコンデンサーに起電力$V_0$\tanni{V}の電池とスイッチSをつなぎ,Sを閉じてから十分に時間がたった
        (以下,これをはじめの状態とする)。真空の誘電率を$\varepsilon _0$\tanni{F/m}とする。
        \begin{enumerate}
            \item コンデンサーの電気量$Q_0$,極板間の電場の強さ$E_0$,静電エネルギー$U_0$はそれぞいくらか。
            \item スイッチSを閉じたまま,コンデンサーの極板間隔を$2d$に広げた。コンデンサーの電気量と電場はそれぞれ何倍になるか。
            \item はじめの状態に戻し,スイッチSを開き,極板間隔を$2d$に広げた。極板間の電場,電位差,静電エネルギーはそれぞれ何倍になるか。
            \item (3)に続いて,極板と同型で,厚さ$d$,比誘電率2の誘電体を極板間に入れた。極板間の電位差$V_1$を$V_0$で表せ。
        \end{enumerate}
    \end{mawarikomi}
