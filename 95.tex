\hakosyokika
\item
    \begin{mawarikomi}[10](140pt,0){260pt}{%WinTpicVersion4.32a
{\unitlength 0.1in%
\begin{picture}(14.6161,12.9921)(40.3543,-16.5354)%
% LINE 2 0 3 0 Black White  
% 8 4100 360 4100 690 4100 690 5585 1680 5585 1680 5585 1350 5585 1350 4100 360
% 
\special{pn 8}%
\special{pa 4035 354}%
\special{pa 4035 679}%
\special{fp}%
\special{pa 4035 679}%
\special{pa 5497 1654}%
\special{fp}%
\special{pa 5497 1654}%
\special{pa 5497 1329}%
\special{fp}%
\special{pa 5497 1329}%
\special{pa 4035 354}%
\special{fp}%
% DOT 0 0 3 1 Black White  
% 1 4842 1016
% 
\special{pn 4}%
\special{sh 1}%
\special{ar 4766 1000 16 16 0 6.2831853}%
% DOT 0 0 3 2 Black White  
% 1 5007 1127
% 
\special{pn 4}%
\special{sh 1}%
\special{ar 4928 1109 16 16 0 6.2831853}%
% DOT 0 0 3 3 Black White  
% 1 5172 1234
% 
\special{pn 4}%
\special{sh 1}%
\special{ar 5091 1215 16 16 0 6.2831853}%
% DOT 0 0 3 4 Black White  
% 1 5337 1346
% 
\special{pn 4}%
\special{sh 1}%
\special{ar 5253 1325 16 16 0 6.2831853}%
% DOT 0 0 3 5 Black White  
% 1 5502 1453
% 
\special{pn 4}%
\special{sh 1}%
\special{ar 5415 1430 16 16 0 6.2831853}%
% DOT 0 0 3 6 Black White  
% 1 4182 583
% 
\special{pn 4}%
\special{sh 1}%
\special{ar 4116 574 16 16 0 6.2831853}%
% DOT 0 0 3 7 Black White  
% 1 4347 690
% 
\special{pn 4}%
\special{sh 1}%
\special{ar 4279 679 16 16 0 6.2831853}%
% DOT 0 0 3 8 Black White  
% 1 4512 801
% 
\special{pn 4}%
\special{sh 1}%
\special{ar 4441 788 16 16 0 6.2831853}%
% DOT 0 0 3 9 Black White  
% 1 4677 909
% 
\special{pn 4}%
\special{sh 1}%
\special{ar 4603 895 16 16 0 6.2831853}%
% LINE 1 0 3 10 Black White  
% 2 4842 1020 2801 1762
% 
\special{pn 13}%
\special{pa 4766 1004}%
\special{pa 2757 1734}%
\special{fp}%
% VECTOR 1 0 3 11 Black White  
% 2 2801 1762 3956 1342
% 
\special{pn 13}%
\special{pa 2757 1734}%
\special{pa 3894 1321}%
\special{fp}%
\special{sh 1}%
\special{pa 3894 1321}%
\special{pa 3826 1325}%
\special{pa 3844 1339}%
\special{pa 3839 1362}%
\special{pa 3894 1321}%
\special{fp}%
% POLYGON 2 0 3 12 Black White  
% 6 3296 1457 3296 1585 3378 1668 3378 1540 3378 1540 3296 1457
% 
\special{pn 8}%
\special{pa 3244 1434}%
\special{pa 3244 1560}%
\special{pa 3325 1642}%
\special{pa 3325 1516}%
\special{pa 3244 1434}%
\special{pa 3244 1560}%
\special{fp}%
% LINE 3 0 3 13 Black White  
% 10 3308 1597 3308 1470 3325 1614 3325 1486 3341 1630 3341 1503 3358 1647 3358 1519 3374 1663 3374 1536
% 
\special{pn 4}%
\special{pa 3256 1572}%
\special{pa 3256 1447}%
\special{fp}%
\special{pa 3273 1589}%
\special{pa 3273 1463}%
\special{fp}%
\special{pa 3288 1604}%
\special{pa 3288 1479}%
\special{fp}%
\special{pa 3305 1621}%
\special{pa 3305 1495}%
\special{fp}%
\special{pa 3321 1637}%
\special{pa 3321 1512}%
\special{fp}%
% LINE 2 1 3 14 Black White  
% 2 3337 1647 3337 1894
% 
\special{pn 8}%
\special{pa 3284 1621}%
\special{pa 3284 1864}%
\special{da 0.030}%
% LINE 2 1 3 15 Black White  
% 2 4842 739 4842 1317
% 
\special{pn 8}%
\special{pa 4766 727}%
\special{pa 4766 1296}%
\special{da 0.030}%
% LINE 2 1 3 16 Black White  
% 2 5337 1069 5337 1346
% 
\special{pn 8}%
\special{pa 5253 1052}%
\special{pa 5253 1325}%
\special{da 0.030}%
% VECTOR 2 0 3 17 Black White  
% 2 3935 1548 3337 1767
% 
\special{pn 8}%
\special{pa 3873 1524}%
\special{pa 3284 1739}%
\special{fp}%
\special{sh 1}%
\special{pa 3284 1739}%
\special{pa 3352 1735}%
\special{pa 3334 1721}%
\special{pa 3340 1698}%
\special{pa 3284 1739}%
\special{fp}%
% VECTOR 2 0 3 18 Black White  
% 2 3935 1552 4842 1222
% 
\special{pn 8}%
\special{pa 3873 1528}%
\special{pa 4766 1203}%
\special{fp}%
\special{sh 1}%
\special{pa 4766 1203}%
\special{pa 4698 1207}%
\special{pa 4717 1220}%
\special{pa 4711 1244}%
\special{pa 4766 1203}%
\special{fp}%
% STR 2 0 3 19 Black White  
% 4 3939 1507 3939 1548 5 0 1 0
% $L$
\put(38.7697,-15.2362){\makebox(0,0){{\colorbox[named]{White}{$L$}}}}%
% STR 2 0 3 20 Black White  
% 4 3168 1408 3168 1449 2 0 0 0
% 回折格子
\put(31.1811,-14.2618){\makebox(0,0)[lb]{回折格子}}%
% ELLIPSE 2 0 3 21 Black White  
% 4 2450 1894 2487 1812 2487 1812 2487 1812
% 
\special{pn 8}%
\special{ar 2411 1864 36 81 0.0000000 6.2831853}%
% LINE 2 0 3 22 Black White  
% 2 2458 1975 2762 1865
% 
\special{pn 8}%
\special{pa 2419 1944}%
\special{pa 2719 1836}%
\special{fp}%
% LINE 2 0 3 23 Black White  
% 2 2453 1812 2757 1701
% 
\special{pn 8}%
\special{pa 2414 1783}%
\special{pa 2714 1674}%
\special{fp}%
% ELLIPSE 2 0 3 24 Black White  
% 4 2760 1782 2797 1700 2764 1898 2764 1667
% 
\special{pn 8}%
\special{ar 2717 1754 36 81 4.7891609 1.4940244}%
% STR 2 0 3 25 Black White  
% 4 2112 1441 2112 1482 2 0 0 0
% レーザー光源
\put(20.7874,-14.5866){\makebox(0,0)[lb]{レーザー光源}}%
% STR 2 0 3 26 Black White  
% 4 2112 1555 2112 1596 2 0 0 0
% または白色光源
\put(20.7874,-15.7087){\makebox(0,0)[lb]{または白色光源}}%
% VECTOR 2 0 3 27 Black White  
% 4 5082 933 4842 777 5090 942 5337 1107
% 
\special{pn 8}%
\special{pa 5002 918}%
\special{pa 4766 765}%
\special{fp}%
\special{sh 1}%
\special{pa 4766 765}%
\special{pa 4810 817}%
\special{pa 4810 793}%
\special{pa 4832 784}%
\special{pa 4766 765}%
\special{fp}%
\special{pa 5010 927}%
\special{pa 5253 1090}%
\special{fp}%
\special{sh 1}%
\special{pa 5253 1090}%
\special{pa 5210 1036}%
\special{pa 5210 1060}%
\special{pa 5187 1070}%
\special{pa 5253 1090}%
\special{fp}%
% STR 2 0 3 28 Black White  
% 4 5123 843 5123 884 5 0 1 0
% $x_m$
\put(50.4232,-8.7008){\makebox(0,0){{\colorbox[named]{White}{$x_m$}}}}%
\end{picture}}%
}
    図のように,格子定数$d$\tanni{m}の回折格子に光源から光を垂直に入射し,回折格子から距離$L$だけ離れているところにスクリーンを配置して以下の実験を行った。
        \begin{enumerate}
            \item 光源として波長$\lambda $\tanni{m}のレーザー光を用いたところ,スクリーン上に明るい点の列が観測された。中心の明るい点から測って,$m$番目の明るい点までの距離を$x_m$\tanni{m}とし,$x_m$が$L$\tanni{m}に比べて十分小さいとした場合,$x_m$を$\lambda $,$d$,$L$,$m$を用いて表せ。ただし,中心の明るい点を$m=0$とし,微小角$\theta $に対して$\sin{\theta }\kinzi \tan{\theta }$の近似を用いてよい。また,このとき明るい点の間隔$\varDelta x$\tanni{m}を求めよ。
            \item 前問において,回折格子のすじが$1$\sftanni{mm}あたり100本あり,$L$が1.00\sftanni{m}のとき,$m=3$の明るい点までの距離$x_3$が19.0\sftanni{cm}と測定された。レーザー光の波長$\lambda $\tanni{nm}を求めよ。
            \item 光源としてレーザー光のかわりに可視光領域の白色光を用いると,スクリーン上ではどのような像が見られるか。$m=0$と$m=1$の明るい点について簡単に説明せよ。
            \item 可視光の波長範囲は$380$\sftanni{nm}~$770$\sftanni{nm}である。$m=1$のときの$x_1$の広がる範囲\tanni{cm}を求めよ。
        \end{enumerate}
    \end{mawarikomi}