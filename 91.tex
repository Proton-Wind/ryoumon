\hakosyokika
\item
    \begin{mawarikomi}(10pt,0){160pt}{%WinTpicVersion4.32a
{\unitlength 0.1in%
\begin{picture}(17.7165,19.1043)(3.9370,-30.2362)%
% CIRCLE 2 0 3 0 Black White  
% 4 700 2135 1000 2135 1000 2135 1000 2135
% 
\special{pn 8}%
\special{ar 689 2101 295 295 0.0000000 6.2831853}%
% DOT 0 0 3 0 Black White  
% 1 700 2135
% 
\special{pn 4}%
\special{sh 1}%
\special{ar 689 2101 16 16 0 6.2831853}%
% CIRCLE 2 0 3 0 Black White  
% 4 700 2135 1300 2135 400 3035 400 1535
% 
\special{pn 8}%
\special{ar 689 2101 591 591 4.2487414 1.8925469}%
% CIRCLE 2 0 3 0 Black White  
% 4 700 2135 1600 2135 400 3335 400 1235
% 
\special{pn 8}%
\special{ar 689 2101 886 886 4.3906384 1.8157750}%
% CIRCLE 2 0 3 0 Black White  
% 4 700 2135 1900 2135 1900 3635 1300 1235
% 
\special{pn 8}%
\special{ar 689 2101 1181 1181 5.3003916 0.8960554}%
% DOT 0 0 3 0 Black White  
% 1 1900 2135
% 
\special{pn 4}%
\special{sh 1}%
\special{ar 1870 2101 16 16 0 6.2831853}%
% CIRCLE 2 0 3 0 Black White  
% 4 1900 2129 1600 2129 1600 2129 1600 2129
% 
\special{pn 8}%
\special{ar 1870 2095 295 295 0.0000000 6.2831853}%
% CIRCLE 2 0 3 0 Black White  
% 4 1900 2129 1300 2129 2200 1529 2200 3029
% 
\special{pn 8}%
\special{ar 1870 2095 591 591 1.2490458 5.1760366}%
% CIRCLE 2 0 3 0 Black White  
% 4 1900 2129 1000 2129 2200 1229 2200 3329
% 
\special{pn 8}%
\special{ar 1870 2095 886 886 1.3258177 5.0341395}%
% CIRCLE 2 0 3 0 Black White  
% 4 1900 2129 700 2129 1300 1229 700 3629
% 
\special{pn 8}%
\special{ar 1870 2095 1181 1181 2.2455373 4.1243864}%
% DOT 0 1 3 0 Black White  
% 1 1201 2693
% 
\special{pn 4}%
\special{sh 1}%
\special{ar 1182 2651 16 16 0 6.2831853}%
% DOT 0 2 3 0 Black White  
% 1 1075 1490
% 
\special{pn 4}%
\special{sh 1}%
\special{ar 1058 1467 16 16 0 6.2831853}%
% STR 2 0 3 0 Black White  
% 4 1075 1538 1075 1568 5 0 0 0
% P$_3$
\put(10.5807,-15.4331){\makebox(0,0){P$_3$}}%
% STR 2 0 3 0 Black White  
% 4 580 2105 580 2135 5 0 0 0
% $A$
\put(5.7087,-21.0138){\makebox(0,0){$A$}}%
% STR 2 0 3 0 Black White  
% 4 2020 2105 2020 2135 5 0 0 0
% $B$
\put(19.8819,-21.0138){\makebox(0,0){$B$}}%
% STR 2 0 3 0 Black White  
% 4 1255 2618 1255 2648 5 0 0 0
% P$_2$
\put(12.3524,-26.0630){\makebox(0,0){P$_2$}}%
% DOT 0 0 3 0 Black White  
% 1 1747 1550
% 
\special{pn 4}%
\special{sh 1}%
\special{ar 1719 1526 16 16 0 6.2831853}%
% STR 2 0 3 0 Black White  
% 4 1801 1385 1801 1415 5 0 0 0
% P$_1$
\put(17.7264,-13.9272){\makebox(0,0){P$_1$}}%
\end{picture}}%
}
    水面上で$d$\tanni{m}離れた2点A,Bを周期$T$\tanni{s}で振動させ,2つの波をつくった。図は,波源A,Bから出る波の,ある時刻での山の位置を描いたものである。
        \begin{enumerate}
            \item この波の波長$\lambda $,および波の速さ$v$はいくらか。
            \item 図中の点$\mathrm{P_1}$,$\mathrm{P_2}$,$\mathrm{P_3}$はそれぞれ強めあいの位置か,弱めあいの位置か。
            \item 点A,Bから各点までの距離の差$\mathrm{AP_1-BP_1}$,$\mathrm{AP_2-BP_2}$,$\mathrm{AP_3-BP_3}$を波長$\lambda $で表せ。
            \item 線分ABを横切り,波源Aの最も近くを通る強めあいの線を図に描き入れよ。
            \item AとBを逆位相で振動させると,AB間には何本の弱めあいの線が現れるか(波源は除く)。
        \end{enumerate}
    \end{mawarikomi}