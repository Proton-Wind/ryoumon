\hakosyokika
\item
    \begin{mawarikomi}(10pt,0){160pt}{\input{./fig/fig091.tex}}
    水面上で$d$\tanni{m}離れた2点A,Bを周期$T$\tanni{s}で振動させ,2つの波をつくった。図は,波源A,Bから出る波の,ある時刻での山の位置を描いたものである。
        \begin{enumerate}
            \item この波の波長$\lambda $,および波の速さ$v$はいくらか。
            \item 図中の点$\mathrm{P_1}$,$\mathrm{P_2}$,$\mathrm{P_3}$はそれぞれ強めあいの位置か,弱めあいの位置か。
            \item 点A,Bから各点までの距離の差$\mathrm{AP_1-BP_1}$,$\mathrm{AP_2-BP_2}$,$\mathrm{AP_3-BP_3}$を波長$\lambda $で表せ。
            \item 線分ABを横切り,波源Aの最も近くを通る強めあいの線を図に描き入れよ。
            \item AとBを逆位相で振動させると,AB間には何本の弱めあいの線が現れるか(波源は除く)。
        \end{enumerate}
    \end{mawarikomi}