\hakosyokika
\item
    \begin{mawarikomi}(20pt,0){150pt}{\begin{zahyou*}[ul=5mm](0,11)(0,12)
    \small
    \tenretu*{
        A(0,2);
        B(5,2);
        C(5,10);
        D(4,10);
        E(4,3);
        F(1,3);
        G(1,10);
        H(0,10);
        I(1,6);
        J(4,6);
        K(4,7);
        L(1,7);
        M(1.2,3);
        N(1.2,6);
        O(1.8,1);
        P(3.2,1);
        Q(3.2,3.8);
        R(1.8,3.8);
        S(2.2,3.7);
        T(2.8,3.7);
        U(2.8,3.9);
        V(2.2,3.9);
        AA(6,2);
        BB(7,2);
        CC(7,9);
        DD(10,9);
        EE(10,2);
        FF(11,2);
        GG(11,10);
        HH(6,10);
        II(10,5);
        JJ(7,5);
        KK(7,4);
        LL(10,4);
        MM(9.8,9);
        NN(9.8,5);
        OO(9.2,11);
        PP(7.8,11);
        QQ(7.8,8.2);
        RR(9.2,8.2);
        SS(8.8,8.3);
        TT(8.2,8.3);
        UU(8.2,8.1);
        VV(8.8,8.1);
    }
    \Drawlines{\A\B\C\D\E\F\G\H\A;\I\J;\K\L}
    \Drawlines{\P\Q\R\O}
    \Nuritubusi*{\I\J\K\L}
    \Nuritubusi[0]{\S\T\U\V\S}
    \Drawlines{\S\T\U\V\S}
    \kuromaru{\O;\P}
    \HenKo<henkotype=parallel
    ,henkoH=0.5ex
    ,yazirusi=b>{\M}{\N}{$\ell $}
    \Drawlines{\AA\BB\CC\DD\EE\FF\GG\HH\AA;\II\JJ;\KK\LL}
    \Drawlines{\PP\QQ\RR\OO}
    \Nuritubusi*{\II\JJ\KK\LL}
    \Nuritubusi[0]{\SS\TT\UU\VV\SS}
    \Drawlines{\SS\TT\UU\VV\SS}
    \kuromaru{\OO;\PP}
    \HenKo<henkotype=parallel
    ,henkoH=1.6ex
    ,yazirusi=b>{\MM}{\NN}{$\bunsuu{4}{3}\ell $}

\end{zahyou*}
}
        なめらかに動く質量$M$\tanni{kg}のピストンを備えた断面積$S$\tanni{m^2}の容器がある。これらは断熱材で作られていて,ヒーターに電流を流すことにより,容器内の機体を加熱することができる。ヒーターの体積,熱容量は小さく,無視できる。容器は鉛直に保たれていて,内部には単原子分子の理想気体が$n$\tanni{mol}入っている。気体定数を$R$\tanni{J/(mol\cdot K)},大気圧を$P_0$\tanni{Pa},重力加速度の大きさを$g$\tanni{m/s^2}とする。
        \begin{enumerate}
            \item 最初,ヒーターに電流を流さない状態では,図1のように,ピストンの下面は容器の底から距離$\ell $\tanni{m}の位置にあった。このときの気体の温度はどれだけか。
            \item 次に,ヒーターで加熱したら,ピストンは最初の位置より$\bunsuu{1}{2}\ell $上昇した。気体の温度は(1)の何倍になっているか。また,ヒーターで発生したジュール熱はどれだけか。
            \item (1)の状態で,容器の上下を反対にして鉛直にし,気体の温度を(1)の温度と同じに保ったら,図2のように,ピストンの上面は容器の底から$\bunsuu{4}{3}\ell $\tanni{m}の位置で静止した。ピストン$M$を他の量で表せ。
            \item この状態で,ヒーターにより,(2)におけるジュール熱の$\bunsuu{1}{2}$だけの熱を加えたら,ピストンの上面は容器の底からどれだけの距離のところで静止するか。
        \end{enumerate}
    \end{mawarikomi}