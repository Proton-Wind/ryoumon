\hakosyokika
\item
    \begin{mawarikomi}(20pt,0pt){160pt}{%WinTpicVersion4.32a
{\unitlength 0.1in%
\begin{picture}(21.8500,12.0000)(3.2000,-14.0000)%
% BOX 2 0 3 0 Black White  
% 2 400 1000 2000 1200
% 
\special{pn 8}%
\special{pa 400 1000}%
\special{pa 2000 1000}%
\special{pa 2000 1200}%
\special{pa 400 1200}%
\special{pa 400 1000}%
\special{pa 2000 1000}%
\special{fp}%
% BOX 2 0 0 0 Black Black  
% 2 2000 1000 2400 800
% 
\special{pn 0}%
\special{sh 0.400}%
\special{pa 2000 1000}%
\special{pa 2400 1000}%
\special{pa 2400 800}%
\special{pa 2000 800}%
\special{pa 2000 1000}%
\special{ip}%
\special{pn 8}%
\special{pa 2000 1000}%
\special{pa 2400 1000}%
\special{pa 2400 800}%
\special{pa 2000 800}%
\special{pa 2000 1000}%
\special{pa 2400 1000}%
\special{fp}%
% POLYGON 2 0 3 0 Black White  
% 5 376 801 400 1000 2018 801 1994 603 376 801
% 
\special{pn 8}%
\special{pa 376 801}%
\special{pa 400 1000}%
\special{pa 2018 801}%
\special{pa 1994 603}%
\special{pa 376 801}%
\special{pa 400 1000}%
\special{fp}%
% LINE 2 1 3 0 Black White  
% 4 400 1200 400 1400 2000 1400 2000 1200
% 
\special{pn 8}%
\special{pa 400 1200}%
\special{pa 400 1400}%
\special{da 0.015}%
\special{pa 2000 1400}%
\special{pa 2000 1200}%
\special{da 0.015}%
% VECTOR 2 0 3 0 Black White  
% 4 1200 1300 400 1300 1200 1300 2000 1300
% 
\special{pn 8}%
\special{pa 1200 1300}%
\special{pa 400 1300}%
\special{fp}%
\special{sh 1}%
\special{pa 400 1300}%
\special{pa 467 1320}%
\special{pa 453 1300}%
\special{pa 467 1280}%
\special{pa 400 1300}%
\special{fp}%
\special{pa 1200 1300}%
\special{pa 2000 1300}%
\special{fp}%
\special{sh 1}%
\special{pa 2000 1300}%
\special{pa 1933 1280}%
\special{pa 1947 1300}%
\special{pa 1933 1320}%
\special{pa 2000 1300}%
\special{fp}%
% STR 2 0 3 0 Black White  
% 4 1200 1200 1200 1300 5 0 1 0
% $L$
\put(12.0000,-13.0000){\makebox(0,0){{\colorbox[named]{White}{$L$}}}}%
% STR 2 0 3 0 Black White  
% 4 1200 700 1200 800 5 0 0 0
% A
\put(12.0000,-8.0000){\makebox(0,0){A}}%
% STR 2 0 3 0 Black White  
% 4 1200 1000 1200 1100 5 0 0 0
% B
\put(12.0000,-11.0000){\makebox(0,0){B}}%
% VECTOR 2 0 3 0 Black White  
% 2 2470 900 2470 800
% 
\special{pn 8}%
\special{pa 2470 900}%
\special{pa 2470 800}%
\special{fp}%
\special{sh 1}%
\special{pa 2470 800}%
\special{pa 2450 867}%
\special{pa 2470 853}%
\special{pa 2490 867}%
\special{pa 2470 800}%
\special{fp}%
% VECTOR 2 0 3 0 Black White  
% 2 2470 900 2470 1000
% 
\special{pn 8}%
\special{pa 2470 900}%
\special{pa 2470 1000}%
\special{fp}%
\special{sh 1}%
\special{pa 2470 1000}%
\special{pa 2490 933}%
\special{pa 2470 947}%
\special{pa 2450 933}%
\special{pa 2470 1000}%
\special{fp}%
% STR 2 0 3 0 Black White  
% 4 2600 800 2600 900 5 0 0 0
% $D$
\put(26.0000,-9.0000){\makebox(0,0){$D$}}%
% STR 2 0 3 0 Black White  
% 4 2080 660 2080 760 2 0 0 0
% アルミ箔
\put(20.8000,-7.6000){\makebox(0,0)[lb]{アルミ箔}}%
% VECTOR 2 0 3 0 Black White  
% 6 610 200 610 600 1210 200 1210 600 1810 200 1810 600
% 
\special{pn 8}%
\special{pa 610 200}%
\special{pa 610 600}%
\special{fp}%
\special{sh 1}%
\special{pa 610 600}%
\special{pa 630 533}%
\special{pa 610 547}%
\special{pa 590 533}%
\special{pa 610 600}%
\special{fp}%
\special{pa 1210 200}%
\special{pa 1210 600}%
\special{fp}%
\special{sh 1}%
\special{pa 1210 600}%
\special{pa 1230 533}%
\special{pa 1210 547}%
\special{pa 1190 533}%
\special{pa 1210 600}%
\special{fp}%
\special{pa 1810 200}%
\special{pa 1810 600}%
\special{fp}%
\special{sh 1}%
\special{pa 1810 600}%
\special{pa 1830 533}%
\special{pa 1810 547}%
\special{pa 1790 533}%
\special{pa 1810 600}%
\special{fp}%
% STR 2 0 3 0 Black White  
% 4 410 300 410 400 5 0 0 0
% 光
\put(4.1000,-4.0000){\makebox(0,0){光}}%
% STR 2 0 3 0 Black White  
% 4 1060 300 1060 400 5 0 0 0
% $\lambda $
\put(10.6000,-4.0000){\makebox(0,0){$\lambda $}}%
% STR 2 0 3 0 Black White  
% 4 400 900 400 1000 4 0 0 0
% O
\put(4.0000,-10.0000){\makebox(0,0)[rt]{O}}%
\end{picture}}%
}
    2枚の平板ガラスA,Bの一端Oから$L=0.10$\sftanni{m}離れたところにアルミ箔をはさむ。
    真上から波長$\lambda = 5.9\times 10^{-7}$\sftanni{m}の光を当てて,上から見ると干渉縞が見えた。空気の屈折率を1とする。
        \begin{enumerate}
            \item O点の縞は明線になるか,暗線になるか。それともそのいずれでもないかを答えよ。
            \item 隣り合う明線の間隔$\varDelta x$が$\varDelta x=2.0$\sftanni{mm}のとき,はさんだアルミ箔の厚さ$D$\tanni{m}を求めよ。
            \item 光の方向と反対側(ガラス板B側)から干渉縞を観察する。上から見る場合と比べて,干渉縞はどう変わるか,簡潔に述べよ。
            \item 2枚のガラス板の間を屈折率$n$の水で満たす。空気中と比べて明線の間隔は何倍になるか。
        \end{enumerate}
    \end{mawarikomi}