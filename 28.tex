\item
    \begin{mawarikomi}{120pt}{\begin{zahyou*}[ul=4mm](-1,10)(-2,5)
	\def\O{(0,0)}
	\def\X{(10,0)}
	\def\OD{(0,-0.5)}
	\def\XD{(10,-0.5)}
	\def\B{(1,0)}
	\def\BL{(0.8,0)}
	\def\A{(8,0)}
	\def\P{(8,0.25)}
	\def\Q{(1,0.25)}
	\def\Pvec{(0,3)}
	\def\PL{(2.8,2.5)}
	\def\QR{(8,2.5)}
	\def\Qvec{(2.5,2.8)}
	\def\M{(5,-1)}
	\Candl\Q{1}\BL{\Qvec}\R\S
	\Put\Q{\Yasen\Qvec}
	\Put\P{\Yasen\Pvec}
	\En*[0]\Q{0.25}
	\En*[1]\P{0.25}
	\En\Q{0.25}
	\En\P{0.25}
	\Put\Q(-4pt,12pt)[t]{B}
	\Put\P(4pt,12pt)[t]{A}
	\Put\PL(0,0)[rb]{$V$}
	\Put\QR(3pt,0)[lb]{$v$}
	\Put\M[t]{地面}
	\Nuritubusi[0.25]{\O\OD\XD\X}
	\Drawline{\O\X}
	\Kakukigou\X\BL\S<hankei=1>(2pt,3pt)[l]{$\alpha $}
\end{zahyou*}
}
        水平な地面上のP点から質量$m$の小物体Aを鉛直に打ち上げ,同時にQ点から質量$M$の小球Bを打ち出す。Bの打ち上げ角度$\alpha $は変化させることができる。Aの打ち上げの初速を$v$,Bの初速を$V(>v)$とし,重力加速度の大きさを$g$とする。
        \begin{enumerate}
            \item AとBが衝突しない場合,Aの打ち上げから着地までの時間を求めよ。
            \item BをAに衝突させるには,角度$\alpha $をいくらにすべきか。$\sin{\alpha }$を求めよ。
            \item Aが最高点に達したときに衝突が起こるようにしたい。そのためにはPQ間の距離$\ell $をいくらにすればよいか。$\alpha $を用いずに表せ(以下,同様)。
            \item AとBが最も高い位置で衝突し,両者は合体した。合体直後の速度の水平成分と鉛直成分の大きさはそれぞれいくらか。
            \item AとBは合体した後,地面に落下した。P点から落下点までの距離$x$を求めよ。
        \end{enumerate}
    \end{mawarikomi}