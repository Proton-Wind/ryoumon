\hakosyokika
\item
    \begin{mawarikomi}{180pt}{\begin{zahyou*}[ul=5mm](-1,15)(-1,10)
    \drawline(0,0.5)(0,2)(10,2)(10,0.5)(0,0.5)    
    \def\P{(5,2)}
    \def\PD{(5,1.5)}
    \def\Q{(0,2)}
    \def\QD{(0,1.5)}
    \def\TL{(1.5,0.5)}
    \def\TR{(9,0.5)}
    \def\C{(10,6.5)}
    \def\B{(14,6.5)}
    \def\D{(10,2.8)}
    \def\E{(8.5,2.8)}
    \def\F{(8.5,2)}
    \def\G{(10,2)}
    \Kaiten\C\B{-30}\BD
    \Kaiten\C\B{-90}\CD
    \small
    \Nuritubusi{\D\E\F\G\D}
    \Drawlines{\D\E\F\G\D}
    \Put\P[n]{P}
    \Put\E[nw]{A}
    \Put\PD[se]{$M$}
    \Put\D[ne]{$m$}
    \Put\B(-11pt,-6pt)[l]{B}
    \put(14,7){$\bunsuu{1}{8}m$}
    \put(12,6.8){$\ell $}
    \put(9.5,4.5){$\ell $}
    \put(2.5,2.5){$\mu $}
    \En*[0]\TL{0.5}
    \En*[1]\TL{0.15}
    \En\TL{0.5}
    \En*[0]\TR{0.5}
    \En*[1]\TR{0.15}
    \En\TR{0.5}
    \Nuritubusi*{\P\Q\QD\PD\P}
    \Drawlines{\P\Q\QD\PD\P}
    \Drawline{\C\B}
    \Hasen{\C\BD}
    \Hasen{\C\CD}
    \En*\B{0.2}
    \En\B{0.2}
    \Enko<hasen=[0.4][0.5]>\C{4}{270}{360}
    \Kakukigou\CD\C\BD(0pt,-5pt)[l]{60\Deg}

	\drawline(-1,0)(14,0)
	% \drawline(10,3)(10,0)
\end{zahyou*}
}
        なめらかな水平面上に,質量$M$の台車を静止させてある。台車の表面は水平でP点より右側がなめらかで,左側には摩擦がある。
        台車の右側には質量$m$の小物体Aが置いてあり,その鉛直上方の点から長さ$\ell $の軽い糸で質量$\bunsuu{1}{8}m$の小球Bをつり下げる。
        摩擦面とAとの間の動摩擦係数を$\mu $,重力加速度の大きさを$g$とする。
        \begin{enumerate}
            \item 糸が水平になる位置でBを静かに放し,Aと衝突させたら,Bははね返って糸が鉛直と60\Deg の角度をなす位置まで戻った。衝突後のAの速さ$v_0$を求めよ。(以下の問いには$v_0$を用いてもよい)
            \item 動き出したAはやがて台車に対して止まった。このときの台車の速さ$V$を求めよ。
            \item Aが動き出してから,台車に対して止まるまでに,Aと台車の物体系から失われた力学的エネルギー$E$を求めよ。
            \item 摩擦のある面上において,Aが台車に対して滑った距離$d$を求めよ。
        \end{enumerate}
    \end{mawarikomi}