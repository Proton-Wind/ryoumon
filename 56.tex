\hakosyokika
\item
    \begin{mawarikomi}(20pt,0){170pt}{\begin{zahyou*}[ul=6mm](-2,8)(-2,8)
    \small
    \def\A{(0,5)}
    \def\B{(0,0)}
    \def\C{(6.5,0)}
    \def\D{(6.5,5)}
    \def\E{(6,5)}
    \def\F{(6,0.5)}
    \def\G{(0.5,0.5)}
    \def\H{(0.5,5)}
    \def\I{(0.5,4.5)}
    \def\J{(6,4.5)}
    \def\K{(5,6)}
    \def\L{(5,1.5)}
    \def\M{(5.5,1.5)}
    \def\N{(5.5,6)}
    \def\O{(1.5,1.9)}
    \def\P{(1.5,1.7)}
    \def\Q{(2.8,1.7)}
    \def\R{(2.8,1.9)}
    \Nuritubusi{\A\B\C\D\E\F\G\H\A}
    \Nuritubusi{\H\I\J\E\H}
    \Drawline{\A\B\C\D\E\F\G\H\A}
    \drawline(2.8,1.8)(3,1.8)(3,7.5)(2.4,7.5)
    \drawline(1.5,1.8)(1.3,1.8)(1.3,6.4)(1.1,6.8)
    \drawline(1.3,6.8)(1.3,7.5)(2.1,7.5)
    \Nuritubusi{\O\P\Q\R\O}
    \Drawline{\O\P\Q\R\O}
    \Drawline{\H\E}
    \Drawline{\I\J}
    \drawline(0.5,3.5)(6,3.5)
    \Nuritubusi[0]{\K\L\M\N\K}
    \Drawline{\K\L\M\N\K}
    {\Thicklines
    \drawline(0.5,4)(0.5,0.5)(6,0.5)(6,4)
    \drawline(5.25,4.2)(5.25,1.7)
    }
    {\thicklines
    \Daen{(2.5,1)}{1.5}{0.2}
    \drawline(4,1)(4,7)
    \drawline(2.1,7.7)(2.1,7.3)
    }
    \drawline(2.2,7.9)(2.2,7.1)
    \put(1.5,8){電源}
    \put(3,7.3){かきまぜ棒}
    \put(4.5,6.3){温度計}
    \put(4,2.5){水}
    \put(3,6.5){導}
    \put(3,6){線}
    \Put\O(-4pt,5pt)[l]{抵抗線}
    \put(3,-1){図1}
    \Put[kaiten=90]\H(2mm,3mm)[lb]{スイッチ}
    \put(6.7,5){断}
    \put(6.7,4.5){熱}
    \put(6.7,4){容}
    \put(6.7,3.5){器}
    \drawline(6,4.2)(6.7,4.4)
    \put(-1,4){銅}
    \put(-1,3.5){製}
    \put(-1,3){容}
    \put(-1,2.5){器}
    \drawline(0.35,3.2)(-0.5,4)
\end{zahyou*}
\\\begin{zahyou}[ul=6mm
    ,xscale=0.017
    ,yscale=0.4
    ,Hidariyohaku=30pt
    ,Sitayohaku=20pt
    ,yokozikukigou=時間\tanni{秒}
    ,tatezikukigou=温度\tanni{℃}
    ,yokozikuhaiti={(0,-10pt)[rt]}](0,500)(0,22)
    \small
    \def\Fx{X/50+10}
    \def\Gx{X/100+10}
    \tenretu*<perl>{
        A(100,100/100+10);
        B(200,200/100+10);
        C(300,300/100+10);
        D(400,400/100+10);
        E(100,100/50+10);
        F(200,200/50+10);
        G(300,300/50+10);
        H(400,400/50+10);
    }
    \kuromaru{\A;\B;\C;\D;\E;\F;\G;\H}
    \zahyouMemori[g]<dx=100,dy=2>
    \YGurafu\Fx{0}{500}
    \YGurafu\Gx{0}{500}
    \Put\D[se]{実験2}
    \Put\H[nw]{実験1}
    \put(200,-4){図2}
\end{zahyou}
}
        断熱容器内に質量250\sftanni{g}の薄い銅製容器を入れた水熱量計を用いて以下の実験を行った。
        \begin{description}
            \item[実験1:]温度10℃の銅製容器内に,10℃の水を100\sftanni{g}入れ,スイッチを閉じて消費電力10.0\sftanni{W}で抵抗線を加熱し,かきまぜ棒で水をかき混ぜながら水温を測定した。加熱時間を水温の関係を図2に示す。
            \item[実験2:]10℃の銅製容器内に,10℃の水200\sftanni{g}入れ,スイッチを閉じて消費電力9.0\sftanni{W}の抵抗線を加熱し,{\bf 実験1}と同様の測定をした(図2) 
            \item[実験3:]10℃の銅製容器内に,10℃の水を200\sftanni{g}入れた後,80℃に熱した100\sftanni{g}の金属球を水中に沈めた。かきまぜ棒を使用し,十分時間がたったときの水温は17℃であった。 
        \end{description}
        以下の問に有効数字2桁で答えよ。ただし,断熱容器によって外部との熱の出入りはなく,抵抗線で消費された電力は,水と容器の温度上昇に全て使われたものとする。
        \begin{enumerate}
            \item 銅製容器と水の合計の熱容量を,{\bf 実験1},{\bf 2}についてそれぞれ求めよ。
            \item {\bf 実験1}と{\bf 実験2}の結果から水と銅の比熱をそれぞれ求めよ。
            \item {\bf 実験1}~{\bf 3}の結果から{\bf 実験3}で使用した金属の比熱を求めよ。
            \item 水熱量計の断熱容器をはずして,{\bf 実験3}と同様の実験を行った。このとき,室温は25℃で,他の実験条件は{\bf 実験3}と同じであった。この実験の結果の水温は17℃より高いか,低いか。また,外部との熱の出入りがないと仮定して得られる金属球の比熱は,{\bf 実験3}の値より大きいか,小さいか。
        \end{enumerate}
    \end{mawarikomi}