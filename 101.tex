\hakosyokika
\item
    \begin{mawarikomi}(20pt,0pt){280pt}{%%% C:/vpn/vpn/KeTCindy/fig/fig101.tex 
%%% Generator=fig101.cdy 
{\unitlength=1cm%
\begin{picture}%
(10,8)(-5,-4)%
\special{pn 8}%
%
\special{pa -75 24}\special{pa 0 0}\special{pa -75 -24}\special{pa -60 0}\special{pa -75 24}%
\special{pa -75 24}\special{sh 1}\special{ip}%
\special{pn 1}%
\special{pa   -75    24}\special{pa     0    -0}\special{pa   -75   -24}\special{pa   -60    -0}%
\special{pa   -75    24}%
\special{fp}%
\special{pn 8}%
\special{pa  -669    -0}\special{pa   669    -0}%
\special{fp}%
\special{pa 1394 -24}\special{pa 1319 0}\special{pa 1394 24}\special{pa 1379 0}\special{pa 1394 -24}%
\special{pa 1394 -24}\special{sh 1}\special{ip}%
\special{pn 1}%
\special{pa  1394   -24}\special{pa  1319    -0}\special{pa  1394    24}\special{pa  1379    -0}%
\special{pa  1394   -24}%
\special{fp}%
\special{pn 8}%
\special{pa  1850    -0}\special{pa   787    -0}%
\special{fp}%
\special{pa -1362 -24}\special{pa -1437 0}\special{pa -1362 24}\special{pa -1377 0}%
\special{pa -1362 -24}\special{pa -1362 -24}\special{sh 1}\special{ip}%
\special{pn 1}%
\special{pa -1362   -24}\special{pa -1437    -0}\special{pa -1362    24}\special{pa -1377    -0}%
\special{pa -1362   -24}%
\special{fp}%
\special{pn 8}%
\special{pa  -906    -0}\special{pa -1969    -0}%
\special{fp}%
\special{pa  -787    -0}\special{pa  -735   -76}\special{pa  -682  -147}\special{pa  -630  -213}%
\special{pa  -577  -273}\special{pa  -525  -328}\special{pa  -472  -378}\special{pa  -420  -423}%
\special{pa  -367  -462}\special{pa  -315  -496}\special{pa  -262  -525}\special{pa  -210  -549}%
\special{pa  -157  -567}\special{pa  -105  -580}\special{pa   -52  -588}\special{pa     0  -591}%
\special{pa    52  -588}\special{pa   105  -580}\special{pa   157  -567}\special{pa   210  -549}%
\special{pa   262  -525}\special{pa   315  -496}\special{pa   367  -462}\special{pa   420  -423}%
\special{pa   472  -378}\special{pa   525  -328}\special{pa   577  -273}\special{pa   630  -213}%
\special{pa   682  -147}\special{pa   735   -76}\special{pa   787    -0}%
\special{fp}%
\special{pa  -787    -0}\special{pa  -734   -38}\special{pa  -680   -73}\special{pa  -627  -106}%
\special{pa  -573  -136}\special{pa  -520  -164}\special{pa  -467  -189}\special{pa  -413  -211}%
\special{pa  -360  -231}\special{pa  -307  -248}\special{pa  -254  -262}\special{pa  -201  -274}%
\special{pa  -149  -283}\special{pa   -96  -290}\special{pa   -43  -294}\special{pa     9  -295}%
\special{pa    62  -294}\special{pa   114  -290}\special{pa   166  -283}\special{pa   219  -274}%
\special{pa   271  -262}\special{pa   323  -248}\special{pa   375  -231}\special{pa   427  -211}%
\special{pa   478  -189}\special{pa   530  -164}\special{pa   582  -136}\special{pa   633  -106}%
\special{pa   685   -73}\special{pa   736   -38}\special{pa   787    -0}%
\special{fp}%
\special{pa  -787    -0}\special{pa  -735    76}\special{pa  -682   147}\special{pa  -630   213}%
\special{pa  -577   273}\special{pa  -525   328}\special{pa  -472   378}\special{pa  -420   423}%
\special{pa  -367   462}\special{pa  -315   496}\special{pa  -262   525}\special{pa  -210   549}%
\special{pa  -157   567}\special{pa  -105   580}\special{pa   -52   588}\special{pa     0   591}%
\special{pa    52   588}\special{pa   105   580}\special{pa   157   567}\special{pa   210   549}%
\special{pa   262   525}\special{pa   315   496}\special{pa   367   462}\special{pa   420   423}%
\special{pa   472   378}\special{pa   525   328}\special{pa   577   273}\special{pa   630   213}%
\special{pa   682   147}\special{pa   735    76}\special{pa   787    -0}%
\special{fp}%
\special{pa  -787    -0}\special{pa  -735    37}\special{pa  -682    71}\special{pa  -630   103}%
\special{pa  -577   132}\special{pa  -525   159}\special{pa  -472   183}\special{pa  -420   204}%
\special{pa  -367   223}\special{pa  -315   240}\special{pa  -262   254}\special{pa  -210   265}%
\special{pa  -157   274}\special{pa  -105   281}\special{pa   -52   284}\special{pa     0   286}%
\special{pa    52   284}\special{pa   105   281}\special{pa   157   274}\special{pa   210   265}%
\special{pa   262   254}\special{pa   315   240}\special{pa   367   223}\special{pa   420   204}%
\special{pa   472   183}\special{pa   525   159}\special{pa   577   132}\special{pa   630   103}%
\special{pa   682    71}\special{pa   735    37}\special{pa   787    -0}%
\special{fp}%
\special{pa   787    -0}\special{pa   839    26}\special{pa   891    52}\special{pa   941    79}%
\special{pa   990   105}\special{pa  1039   131}\special{pa  1087   157}\special{pa  1133   184}%
\special{pa  1179   210}\special{pa  1224   236}\special{pa  1269   262}\special{pa  1312   289}%
\special{pa  1354   315}\special{pa  1396   341}\special{pa  1437   367}\special{pa  1476   394}%
\special{pa  1515   420}\special{pa  1553   446}\special{pa  1591   472}\special{pa  1627   499}%
\special{pa  1662   525}\special{pa  1697   551}\special{pa  1731   577}\special{pa  1763   604}%
\special{pa  1795   630}\special{pa  1826   656}\special{pa  1857   682}\special{pa  1886   709}%
\special{pa  1914   735}\special{pa  1942   761}\special{pa  1969   787}%
\special{fp}%
\special{pa   787    -0}\special{pa   839   -26}\special{pa   891   -52}\special{pa   941   -79}%
\special{pa   990  -105}\special{pa  1039  -131}\special{pa  1087  -157}\special{pa  1133  -184}%
\special{pa  1179  -210}\special{pa  1224  -236}\special{pa  1269  -262}\special{pa  1312  -289}%
\special{pa  1354  -315}\special{pa  1396  -341}\special{pa  1437  -367}\special{pa  1476  -394}%
\special{pa  1515  -420}\special{pa  1553  -446}\special{pa  1591  -472}\special{pa  1627  -499}%
\special{pa  1662  -525}\special{pa  1697  -551}\special{pa  1731  -577}\special{pa  1763  -604}%
\special{pa  1795  -630}\special{pa  1826  -656}\special{pa  1857  -682}\special{pa  1886  -709}%
\special{pa  1914  -735}\special{pa  1942  -761}\special{pa  1969  -787}%
\special{fp}%
\special{pa   787    -0}\special{pa   827   -41}\special{pa   865   -82}\special{pa   902  -123}%
\special{pa   938  -164}\special{pa   972  -204}\special{pa  1004  -245}\special{pa  1036  -286}%
\special{pa  1066  -326}\special{pa  1094  -366}\special{pa  1122  -406}\special{pa  1148  -446}%
\special{pa  1172  -486}\special{pa  1196  -526}\special{pa  1217  -566}\special{pa  1238  -606}%
\special{pa  1257  -645}\special{pa  1275  -685}\special{pa  1291  -724}\special{pa  1306  -763}%
\special{pa  1320  -803}\special{pa  1332  -842}\special{pa  1343  -881}\special{pa  1353  -919}%
\special{pa  1361  -958}\special{pa  1368  -997}\special{pa  1374 -1035}\special{pa  1378 -1074}%
\special{pa  1380 -1112}\special{pa  1382 -1150}\special{pa  1382 -1188}%
\special{fp}%
\special{pa   787    -0}\special{pa   825    39}\special{pa   861    79}\special{pa   895   118}%
\special{pa   929   157}\special{pa   961   197}\special{pa   993   236}\special{pa  1023   275}%
\special{pa  1051   314}\special{pa  1079   354}\special{pa  1105   393}\special{pa  1131   432}%
\special{pa  1155   472}\special{pa  1177   511}\special{pa  1199   550}\special{pa  1220   590}%
\special{pa  1239   629}\special{pa  1257   668}\special{pa  1274   708}\special{pa  1289   747}%
\special{pa  1304   786}\special{pa  1317   825}\special{pa  1329   865}\special{pa  1340   904}%
\special{pa  1349   943}\special{pa  1358   983}\special{pa  1365  1022}\special{pa  1371  1061}%
\special{pa  1376  1101}\special{pa  1380  1140}\special{pa  1382  1179}%
\special{fp}%
\special{pa  -787    -0}\special{pa  -839   -26}\special{pa  -891   -52}\special{pa  -941   -79}%
\special{pa  -990  -105}\special{pa -1039  -131}\special{pa -1087  -157}\special{pa -1133  -184}%
\special{pa -1179  -210}\special{pa -1224  -236}\special{pa -1269  -262}\special{pa -1312  -289}%
\special{pa -1354  -315}\special{pa -1396  -341}\special{pa -1437  -367}\special{pa -1476  -394}%
\special{pa -1515  -420}\special{pa -1553  -446}\special{pa -1591  -472}\special{pa -1627  -499}%
\special{pa -1662  -525}\special{pa -1697  -551}\special{pa -1731  -577}\special{pa -1763  -604}%
\special{pa -1795  -630}\special{pa -1826  -656}\special{pa -1857  -682}\special{pa -1886  -709}%
\special{pa -1914  -735}\special{pa -1942  -761}\special{pa -1969  -787}%
\special{fp}%
\special{pa  -787    -0}\special{pa  -826   -40}\special{pa  -863   -81}\special{pa  -899  -121}%
\special{pa  -933  -161}\special{pa  -966  -202}\special{pa  -998  -242}\special{pa -1028  -282}%
\special{pa -1057  -322}\special{pa -1085  -362}\special{pa -1112  -401}\special{pa -1137  -441}%
\special{pa -1161  -481}\special{pa -1183  -520}\special{pa -1204  -560}\special{pa -1224  -599}%
\special{pa -1243  -638}\special{pa -1260  -677}\special{pa -1276  -716}\special{pa -1290  -755}%
\special{pa -1304  -794}\special{pa -1316  -833}\special{pa -1326  -872}\special{pa -1335  -911}%
\special{pa -1343  -949}\special{pa -1350  -988}\special{pa -1355 -1026}\special{pa -1359 -1065}%
\special{pa -1362 -1103}\special{pa -1363 -1141}\special{pa -1364 -1179}%
\special{fp}%
\special{pa  -787    -0}\special{pa  -839    26}\special{pa  -891    52}\special{pa  -941    79}%
\special{pa  -990   105}\special{pa -1039   131}\special{pa -1087   157}\special{pa -1133   184}%
\special{pa -1179   210}\special{pa -1224   236}\special{pa -1269   262}\special{pa -1312   289}%
\special{pa -1354   315}\special{pa -1396   341}\special{pa -1437   367}\special{pa -1476   394}%
\special{pa -1515   420}\special{pa -1553   446}\special{pa -1591   472}\special{pa -1627   499}%
\special{pa -1662   525}\special{pa -1697   551}\special{pa -1731   577}\special{pa -1763   604}%
\special{pa -1795   630}\special{pa -1826   656}\special{pa -1857   682}\special{pa -1886   709}%
\special{pa -1914   735}\special{pa -1942   761}\special{pa -1969   787}%
\special{fp}%
\special{pa  -787    -0}\special{pa  -826    39}\special{pa  -863    79}\special{pa  -899   118}%
\special{pa  -933   157}\special{pa  -966   197}\special{pa  -998   236}\special{pa -1029   275}%
\special{pa -1058   314}\special{pa -1086   354}\special{pa -1113   393}\special{pa -1138   432}%
\special{pa -1162   472}\special{pa -1185   511}\special{pa -1206   550}\special{pa -1226   590}%
\special{pa -1245   629}\special{pa -1263   668}\special{pa -1279   708}\special{pa -1294   747}%
\special{pa -1308   786}\special{pa -1320   825}\special{pa -1331   865}\special{pa -1341   904}%
\special{pa -1349   943}\special{pa -1356   983}\special{pa -1362  1022}\special{pa -1367  1061}%
\special{pa -1370  1101}\special{pa -1372  1140}\special{pa -1373  1179}%
\special{fp}%
\special{pa   787    -0}\special{pa   800   -53}\special{pa   810  -105}\special{pa   819  -156}%
\special{pa   826  -207}\special{pa   832  -258}\special{pa   835  -308}\special{pa   837  -358}%
\special{pa   837  -407}\special{pa   836  -455}\special{pa   832  -504}\special{pa   827  -551}%
\special{pa   820  -598}\special{pa   811  -645}\special{pa   801  -691}\special{pa   789  -737}%
\special{pa   775  -782}\special{pa   759  -827}\special{pa   742  -871}\special{pa   723  -915}%
\special{pa   702  -958}\special{pa   679 -1001}\special{pa   655 -1043}\special{pa   628 -1085}%
\special{pa   600 -1126}\special{pa   571 -1167}\special{pa   539 -1207}\special{pa   506 -1247}%
\special{pa   471 -1286}\special{pa   435 -1325}\special{pa   396 -1364}%
\special{fp}%
\special{pa   787    -0}\special{pa   800    53}\special{pa   810   105}\special{pa   819   156}%
\special{pa   826   207}\special{pa   832   258}\special{pa   835   308}\special{pa   837   358}%
\special{pa   837   407}\special{pa   836   455}\special{pa   832   504}\special{pa   827   551}%
\special{pa   820   598}\special{pa   811   645}\special{pa   801   691}\special{pa   789   737}%
\special{pa   775   782}\special{pa   759   827}\special{pa   742   871}\special{pa   723   915}%
\special{pa   702   958}\special{pa   679  1001}\special{pa   655  1043}\special{pa   628  1085}%
\special{pa   600  1126}\special{pa   571  1167}\special{pa   539  1207}\special{pa   506  1247}%
\special{pa   471  1286}\special{pa   435  1325}\special{pa   396  1364}%
\special{fp}%
\special{pa  -787    -0}\special{pa  -800    53}\special{pa  -810   105}\special{pa  -819   156}%
\special{pa  -826   207}\special{pa  -832   258}\special{pa  -835   308}\special{pa  -837   358}%
\special{pa  -837   407}\special{pa  -836   455}\special{pa  -832   504}\special{pa  -827   551}%
\special{pa  -820   598}\special{pa  -811   645}\special{pa  -801   691}\special{pa  -789   737}%
\special{pa  -775   782}\special{pa  -759   827}\special{pa  -742   871}\special{pa  -723   915}%
\special{pa  -702   958}\special{pa  -679  1001}\special{pa  -655  1043}\special{pa  -628  1085}%
\special{pa  -600  1126}\special{pa  -571  1167}\special{pa  -539  1207}\special{pa  -506  1247}%
\special{pa  -471  1286}\special{pa  -435  1325}\special{pa  -396  1364}%
\special{fp}%
\special{pa  -787    -0}\special{pa  -800   -53}\special{pa  -810  -105}\special{pa  -819  -156}%
\special{pa  -826  -207}\special{pa  -832  -258}\special{pa  -835  -308}\special{pa  -837  -358}%
\special{pa  -837  -407}\special{pa  -836  -455}\special{pa  -832  -504}\special{pa  -827  -551}%
\special{pa  -820  -598}\special{pa  -811  -645}\special{pa  -801  -691}\special{pa  -789  -737}%
\special{pa  -775  -782}\special{pa  -759  -827}\special{pa  -742  -871}\special{pa  -723  -915}%
\special{pa  -702  -958}\special{pa  -679 -1001}\special{pa  -655 -1043}\special{pa  -628 -1085}%
\special{pa  -600 -1126}\special{pa  -571 -1167}\special{pa  -539 -1207}\special{pa  -506 -1247}%
\special{pa  -471 -1286}\special{pa  -435 -1325}\special{pa  -396 -1364}%
\special{fp}%
\special{pa -73 -560}\special{pa 0 -591}\special{pa -77 -609}\special{pa -58 -586}%
\special{pa -73 -560}\special{pa -73 -560}\special{sh 1}\special{ip}%
\special{pn 1}%
\special{pa   -73  -560}\special{pa     0  -591}\special{pa   -77  -609}\special{pa   -58  -586}%
\special{pa   -73  -560}%
\special{fp}%
\special{pn 8}%
\special{pa -63 -268}\special{pa 11 -295}\special{pa -65 -316}\special{pa -48 -293}%
\special{pa -63 -268}\special{pa -63 -268}\special{sh 1}\special{ip}%
\special{pn 1}%
\special{pa   -63  -268}\special{pa    11  -295}\special{pa   -65  -316}\special{pa   -48  -293}%
\special{pa   -63  -268}%
\special{fp}%
\special{pn 8}%
\special{pa -77 609}\special{pa 0 591}\special{pa -73 560}\special{pa -58 586}\special{pa -77 609}%
\special{pa -77 609}\special{sh 1}\special{ip}%
\special{pn 1}%
\special{pa   -77   609}\special{pa     0   591}\special{pa   -73   560}\special{pa   -58   586}%
\special{pa   -77   609}%
\special{fp}%
\special{pn 8}%
\special{pa -76 307}\special{pa 0 286}\special{pa -74 258}\special{pa -59 283}\special{pa -76 307}%
\special{pa -76 307}\special{sh 1}\special{ip}%
\special{pn 1}%
\special{pa   -76   307}\special{pa     0   286}\special{pa   -74   258}\special{pa   -59   283}%
\special{pa   -76   307}%
\special{fp}%
\special{pn 8}%
\special{pa 1619 463}\special{pa 1543 439}\special{pa 1590 502}\special{pa 1592 474}%
\special{pa 1619 463}\special{pa 1619 463}\special{sh 1}\special{ip}%
\special{pn 1}%
\special{pa  1619   463}\special{pa  1543   439}\special{pa  1590   502}\special{pa  1592   474}%
\special{pa  1619   463}%
\special{fp}%
\special{pn 8}%
\special{pa 1590 -502}\special{pa 1543 -439}\special{pa 1619 -463}\special{pa 1592 -474}%
\special{pa 1590 -502}\special{pa 1590 -502}\special{sh 1}\special{ip}%
\special{pn 1}%
\special{pa  1590  -502}\special{pa  1543  -439}\special{pa  1619  -463}\special{pa  1592  -474}%
\special{pa  1590  -502}%
\special{fp}%
\special{pn 8}%
\special{pa 1275 -751}\special{pa 1269 -672}\special{pa 1320 -732}\special{pa 1292 -726}%
\special{pa 1275 -751}\special{pa 1275 -751}\special{sh 1}\special{ip}%
\special{pn 1}%
\special{pa  1275  -751}\special{pa  1269  -672}\special{pa  1320  -732}\special{pa  1292  -726}%
\special{pa  1275  -751}%
\special{fp}%
\special{pn 8}%
\special{pa 1296 699}\special{pa 1244 640}\special{pa 1252 719}\special{pa 1267 694}%
\special{pa 1296 699}\special{pa 1296 699}\special{sh 1}\special{ip}%
\special{pn 1}%
\special{pa  1296   699}\special{pa  1244   640}\special{pa  1252   719}\special{pa  1267   694}%
\special{pa  1296   699}%
\special{fp}%
\special{pn 8}%
\special{pa -1468 -417}\special{pa -1543 -439}\special{pa -1495 -377}\special{pa -1494 -406}%
\special{pa -1468 -417}\special{pa -1468 -417}\special{sh 1}\special{ip}%
\special{pn 1}%
\special{pa -1468  -417}\special{pa -1543  -439}\special{pa -1495  -377}\special{pa -1494  -406}%
\special{pa -1468  -417}%
\special{fp}%
\special{pn 8}%
\special{pa -1199 -603}\special{pa -1253 -661}\special{pa -1243 -583}\special{pa -1228 -608}%
\special{pa -1199 -603}\special{pa -1199 -603}\special{sh 1}\special{ip}%
\special{pn 1}%
\special{pa -1199  -603}\special{pa -1253  -661}\special{pa -1243  -583}\special{pa -1228  -608}%
\special{pa -1199  -603}%
\special{fp}%
\special{pn 8}%
\special{pa -1495 377}\special{pa -1543 439}\special{pa -1468 417}\special{pa -1494 406}%
\special{pa -1495 377}\special{pa -1495 377}\special{sh 1}\special{ip}%
\special{pn 1}%
\special{pa -1495   377}\special{pa -1543   439}\special{pa -1468   417}\special{pa -1494   406}%
\special{pa -1495   377}%
\special{fp}%
\special{pn 8}%
\special{pa -1241 566}\special{pa -1252 644}\special{pa -1198 587}\special{pa -1227 591}%
\special{pa -1241 566}\special{pa -1241 566}\special{sh 1}\special{ip}%
\special{pn 1}%
\special{pa -1241   566}\special{pa -1252   644}\special{pa -1198   587}\special{pa -1227   591}%
\special{pa -1241   566}%
\special{fp}%
\special{pn 8}%
\special{pa 743 -801}\special{pa 789 -737}\special{pa 789 -816}\special{pa 770 -794}%
\special{pa 743 -801}\special{pa 743 -801}\special{sh 1}\special{ip}%
\special{pn 1}%
\special{pa   743  -801}\special{pa   789  -737}\special{pa   789  -816}\special{pa   770  -794}%
\special{pa   743  -801}%
\special{fp}%
\special{pn 8}%
\special{pa 789 816}\special{pa 789 737}\special{pa 743 801}\special{pa 770 794}\special{pa 789 816}%
\special{pa 789 816}\special{sh 1}\special{ip}%
\special{pn 1}%
\special{pa   789   816}\special{pa   789   737}\special{pa   743   801}\special{pa   770   794}%
\special{pa   789   816}%
\special{fp}%
\special{pn 8}%
\special{pa -831 670}\special{pa -789 737}\special{pa -783 658}\special{pa -803 680}%
\special{pa -831 670}\special{pa -831 670}\special{sh 1}\special{ip}%
\special{pn 1}%
\special{pa  -831   670}\special{pa  -789   737}\special{pa  -783   658}\special{pa  -803   680}%
\special{pa  -831   670}%
\special{fp}%
\special{pn 8}%
\special{pa -783 -658}\special{pa -789 -737}\special{pa -831 -670}\special{pa -803 -680}%
\special{pa -783 -658}\special{pa -783 -658}\special{sh 1}\special{ip}%
\special{pn 1}%
\special{pa  -783  -658}\special{pa  -789  -737}\special{pa  -831  -670}\special{pa  -803  -680}%
\special{pa  -783  -658}%
\special{fp}%
\special{pn 8}%
{%
\color[cmyk]{0,0,0,0}%
\special{pa -669 0}\special{pa -670 -15}\special{pa -673 -29}\special{pa -678 -43}%
\special{pa -684 -57}\special{pa -692 -69}\special{pa -701 -81}\special{pa -712 -91}%
\special{pa -724 -100}\special{pa -737 -107}\special{pa -751 -112}\special{pa -765 -116}%
\special{pa -780 -118}\special{pa -795 -118}\special{pa -810 -116}\special{pa -824 -112}%
\special{pa -838 -107}\special{pa -851 -100}\special{pa -863 -91}\special{pa -874 -81}%
\special{pa -883 -69}\special{pa -891 -57}\special{pa -897 -43}\special{pa -902 -29}%
\special{pa -905 -15}\special{pa -906 0}\special{pa -905 15}\special{pa -902 29}\special{pa -897 43}%
\special{pa -891 57}\special{pa -883 69}\special{pa -874 81}\special{pa -863 91}\special{pa -851 100}%
\special{pa -838 107}\special{pa -824 112}\special{pa -810 116}\special{pa -795 118}%
\special{pa -780 118}\special{pa -765 116}\special{pa -751 112}\special{pa -737 107}%
\special{pa -724 100}\special{pa -712 91}\special{pa -701 81}\special{pa -692 69}%
\special{pa -684 57}\special{pa -678 43}\special{pa -673 29}\special{pa -670 15}\special{pa -669 0}%
\special{pa -669 0}\special{sh 1}\special{ip}%
}%
\special{pa  -669    -0}\special{pa  -670   -15}\special{pa  -673   -29}\special{pa  -678   -43}%
\special{pa  -684   -57}\special{pa  -692   -69}\special{pa  -701   -81}\special{pa  -712   -91}%
\special{pa  -724  -100}\special{pa  -737  -107}\special{pa  -751  -112}\special{pa  -765  -116}%
\special{pa  -780  -118}\special{pa  -795  -118}\special{pa  -810  -116}\special{pa  -824  -112}%
\special{pa  -838  -107}\special{pa  -851  -100}\special{pa  -863   -91}\special{pa  -874   -81}%
\special{pa  -883   -69}\special{pa  -891   -57}\special{pa  -897   -43}\special{pa  -902   -29}%
\special{pa  -905   -15}\special{pa  -906     0}\special{pa  -905    15}\special{pa  -902    29}%
\special{pa  -897    43}\special{pa  -891    57}\special{pa  -883    69}\special{pa  -874    81}%
\special{pa  -863    91}\special{pa  -851   100}\special{pa  -838   107}\special{pa  -824   112}%
\special{pa  -810   116}\special{pa  -795   118}\special{pa  -780   118}\special{pa  -765   116}%
\special{pa  -751   112}\special{pa  -737   107}\special{pa  -724   100}\special{pa  -712    91}%
\special{pa  -701    81}\special{pa  -692    69}\special{pa  -684    57}\special{pa  -678    43}%
\special{pa  -673    29}\special{pa  -670    15}\special{pa  -669     0}%
\special{fp}%
{%
\color[cmyk]{0,0,0,0}%
\special{pa 906 0}\special{pa 905 -15}\special{pa 902 -29}\special{pa 897 -43}\special{pa 891 -57}%
\special{pa 883 -69}\special{pa 874 -81}\special{pa 863 -91}\special{pa 851 -100}%
\special{pa 838 -107}\special{pa 824 -112}\special{pa 810 -116}\special{pa 795 -118}%
\special{pa 780 -118}\special{pa 765 -116}\special{pa 751 -112}\special{pa 737 -107}%
\special{pa 724 -100}\special{pa 712 -91}\special{pa 701 -81}\special{pa 692 -69}%
\special{pa 684 -57}\special{pa 678 -43}\special{pa 673 -29}\special{pa 670 -15}\special{pa 669 0}%
\special{pa 670 15}\special{pa 673 29}\special{pa 678 43}\special{pa 684 57}\special{pa 692 69}%
\special{pa 701 81}\special{pa 712 91}\special{pa 724 100}\special{pa 737 107}\special{pa 751 112}%
\special{pa 765 116}\special{pa 780 118}\special{pa 795 118}\special{pa 810 116}\special{pa 824 112}%
\special{pa 838 107}\special{pa 851 100}\special{pa 863 91}\special{pa 874 81}\special{pa 883 69}%
\special{pa 891 57}\special{pa 897 43}\special{pa 902 29}\special{pa 905 15}\special{pa 906 0}%
\special{pa 906 0}\special{sh 1}\special{ip}%
}%
\special{pa   906    -0}\special{pa   905   -15}\special{pa   902   -29}\special{pa   897   -43}%
\special{pa   891   -57}\special{pa   883   -69}\special{pa   874   -81}\special{pa   863   -91}%
\special{pa   851  -100}\special{pa   838  -107}\special{pa   824  -112}\special{pa   810  -116}%
\special{pa   795  -118}\special{pa   780  -118}\special{pa   765  -116}\special{pa   751  -112}%
\special{pa   737  -107}\special{pa   724  -100}\special{pa   712   -91}\special{pa   701   -81}%
\special{pa   692   -69}\special{pa   684   -57}\special{pa   678   -43}\special{pa   673   -29}%
\special{pa   670   -15}\special{pa   669     0}\special{pa   670    15}\special{pa   673    29}%
\special{pa   678    43}\special{pa   684    57}\special{pa   692    69}\special{pa   701    81}%
\special{pa   712    91}\special{pa   724   100}\special{pa   737   107}\special{pa   751   112}%
\special{pa   765   116}\special{pa   780   118}\special{pa   795   118}\special{pa   810   116}%
\special{pa   824   112}\special{pa   838   107}\special{pa   851   100}\special{pa   863    91}%
\special{pa   874    81}\special{pa   883    69}\special{pa   891    57}\special{pa   897    43}%
\special{pa   902    29}\special{pa   905    15}\special{pa   906     0}%
\special{fp}%
\special{pa 0 -1575}\special{pa 0 -1536}\special{fp}\special{pa 0 -1497}\special{pa 0 -1458}\special{fp}%
\special{pa 0 -1419}\special{pa 0 -1380}\special{fp}\special{pa 0 -1341}\special{pa 0 -1303}\special{fp}%
\special{pa 0 -1264}\special{pa 0 -1225}\special{fp}\special{pa 0 -1186}\special{pa 0 -1147}\special{fp}%
\special{pa 0 -1108}\special{pa 0 -1069}\special{fp}\special{pa 0 -1030}\special{pa 0 -992}\special{fp}%
\special{pa 0 -953}\special{pa 0 -914}\special{fp}\special{pa 0 -875}\special{pa 0 -836}\special{fp}%
\special{pa 0 -797}\special{pa 0 -758}\special{fp}\special{pa 0 -719}\special{pa 0 -680}\special{fp}%
\special{pa 0 -642}\special{pa 0 -603}\special{fp}\special{pa 0 -564}\special{pa 0 -525}\special{fp}%
\special{pa 0 -486}\special{pa 0 -447}\special{fp}\special{pa 0 -408}\special{pa 0 -369}\special{fp}%
\special{pa 0 -331}\special{pa 0 -292}\special{fp}\special{pa 0 -253}\special{pa 0 -214}\special{fp}%
\special{pa 0 -175}\special{pa 0 -136}\special{fp}\special{pa 0 -97}\special{pa 0 -58}\special{fp}%
\special{pa 0 -19}\special{pa 0 19}\special{fp}\special{pa 0 58}\special{pa 0 97}\special{fp}%
\special{pa 0 136}\special{pa 0 175}\special{fp}\special{pa 0 214}\special{pa 0 253}\special{fp}%
\special{pa 0 292}\special{pa 0 331}\special{fp}\special{pa 0 369}\special{pa 0 408}\special{fp}%
\special{pa 0 447}\special{pa 0 486}\special{fp}\special{pa 0 525}\special{pa 0 564}\special{fp}%
\special{pa 0 603}\special{pa 0 642}\special{fp}\special{pa 0 680}\special{pa 0 719}\special{fp}%
\special{pa 0 758}\special{pa 0 797}\special{fp}\special{pa 0 836}\special{pa 0 875}\special{fp}%
\special{pa 0 914}\special{pa 0 953}\special{fp}\special{pa 0 992}\special{pa 0 1030}\special{fp}%
\special{pa 0 1069}\special{pa 0 1108}\special{fp}\special{pa 0 1147}\special{pa 0 1186}\special{fp}%
\special{pa 0 1225}\special{pa 0 1264}\special{fp}\special{pa 0 1303}\special{pa 0 1341}\special{fp}%
\special{pa 0 1380}\special{pa 0 1419}\special{fp}\special{pa 0 1458}\special{pa 0 1497}\special{fp}%
\special{pa 0 1536}\special{pa 0 1575}\special{fp}%
%
\settowidth{\Width}{P}\setlength{\Width}{-0.5\Width}%
\settoheight{\Height}{P}\settodepth{\Depth}{P}\setlength{\Height}{-0.5\Height}\setlength{\Depth}{0.5\Depth}\addtolength{\Height}{\Depth}%
\put( -2.000,  0.000){\hspace*{\Width}\raisebox{\Height}{P}}%
%
\settowidth{\Width}{Q}\setlength{\Width}{-0.5\Width}%
\settoheight{\Height}{Q}\settodepth{\Depth}{Q}\setlength{\Height}{-0.5\Height}\setlength{\Depth}{0.5\Depth}\addtolength{\Height}{\Depth}%
\put(  2.000,  0.000){\hspace*{\Width}\raisebox{\Height}{Q}}%
%
\settowidth{\Width}{A}\setlength{\Width}{0\Width}%
\settoheight{\Height}{A}\settodepth{\Depth}{A}\setlength{\Height}{-0.5\Height}\setlength{\Depth}{0.5\Depth}\addtolength{\Height}{\Depth}%
\put(  0.150,  3.000){\hspace*{\Width}\raisebox{\Height}{A}}%
%
\settowidth{\Width}{B}\setlength{\Width}{-1\Width}%
\settoheight{\Height}{B}\settodepth{\Depth}{B}\setlength{\Height}{\Depth}%
\put(  1.020,  0.730){\hspace*{\Width}\raisebox{\Height}{B}}%
%
\settowidth{\Width}{D}\setlength{\Width}{-1\Width}%
\settoheight{\Height}{D}\settodepth{\Depth}{D}\setlength{\Height}{-0.5\Height}\setlength{\Depth}{0.5\Depth}\addtolength{\Height}{\Depth}%
\put( -3.490,  2.120){\hspace*{\Width}\raisebox{\Height}{D}}%
%
\settowidth{\Width}{C}\setlength{\Width}{-0.5\Width}%
\settoheight{\Height}{C}\settodepth{\Depth}{C}\setlength{\Height}{\Depth}%
\put( -3.000,  0.150){\hspace*{\Width}\raisebox{\Height}{C}}%
%
\settowidth{\Width}{O}\setlength{\Width}{-1\Width}%
\settoheight{\Height}{O}\settodepth{\Depth}{O}\setlength{\Height}{-\Height}%
\put( -0.150, -0.150){\hspace*{\Width}\raisebox{\Height}{O}}%
%
\special{pa 22 -1181}\special{pa 22 -1184}\special{pa 22 -1187}\special{pa 21 -1189}%
\special{pa 20 -1192}\special{pa 18 -1194}\special{pa 16 -1196}\special{pa 14 -1198}%
\special{pa 12 -1200}\special{pa 10 -1201}\special{pa 7 -1202}\special{pa 4 -1203}%
\special{pa 1 -1203}\special{pa -1 -1203}\special{pa -4 -1203}\special{pa -7 -1202}%
\special{pa -10 -1201}\special{pa -12 -1200}\special{pa -14 -1198}\special{pa -16 -1196}%
\special{pa -18 -1194}\special{pa -20 -1192}\special{pa -21 -1189}\special{pa -22 -1187}%
\special{pa -22 -1184}\special{pa -22 -1181}\special{pa -22 -1178}\special{pa -22 -1176}%
\special{pa -21 -1173}\special{pa -20 -1170}\special{pa -18 -1168}\special{pa -16 -1166}%
\special{pa -14 -1164}\special{pa -12 -1162}\special{pa -10 -1161}\special{pa -7 -1160}%
\special{pa -4 -1159}\special{pa -1 -1159}\special{pa 1 -1159}\special{pa 4 -1159}%
\special{pa 7 -1160}\special{pa 10 -1161}\special{pa 12 -1162}\special{pa 14 -1164}%
\special{pa 16 -1166}\special{pa 18 -1168}\special{pa 20 -1170}\special{pa 21 -1173}%
\special{pa 22 -1176}\special{pa 22 -1178}\special{pa 22 -1181}\special{pa 22 -1181}%
\special{sh 1}\special{ip}%
\special{pa    22 -1181}\special{pa    22 -1184}\special{pa    22 -1187}\special{pa    21 -1189}%
\special{pa    20 -1192}\special{pa    18 -1194}\special{pa    16 -1196}\special{pa    14 -1198}%
\special{pa    12 -1200}\special{pa    10 -1201}\special{pa     7 -1202}\special{pa     4 -1203}%
\special{pa     1 -1203}\special{pa    -1 -1203}\special{pa    -4 -1203}\special{pa    -7 -1202}%
\special{pa   -10 -1201}\special{pa   -12 -1200}\special{pa   -14 -1198}\special{pa   -16 -1196}%
\special{pa   -18 -1194}\special{pa   -20 -1192}\special{pa   -21 -1189}\special{pa   -22 -1187}%
\special{pa   -22 -1184}\special{pa   -22 -1181}\special{pa   -22 -1178}\special{pa   -22 -1176}%
\special{pa   -21 -1173}\special{pa   -20 -1170}\special{pa   -18 -1168}\special{pa   -16 -1166}%
\special{pa   -14 -1164}\special{pa   -12 -1162}\special{pa   -10 -1161}\special{pa    -7 -1160}%
\special{pa    -4 -1159}\special{pa    -1 -1159}\special{pa     1 -1159}\special{pa     4 -1159}%
\special{pa     7 -1160}\special{pa    10 -1161}\special{pa    12 -1162}\special{pa    14 -1164}%
\special{pa    16 -1166}\special{pa    18 -1168}\special{pa    20 -1170}\special{pa    21 -1173}%
\special{pa    22 -1176}\special{pa    22 -1178}\special{pa    22 -1181}%
\special{fp}%
\special{pa 483 -267}\special{pa 483 -270}\special{pa 482 -273}\special{pa 482 -275}%
\special{pa 480 -278}\special{pa 479 -280}\special{pa 477 -283}\special{pa 475 -284}%
\special{pa 473 -286}\special{pa 470 -287}\special{pa 468 -289}\special{pa 465 -289}%
\special{pa 462 -290}\special{pa 459 -290}\special{pa 456 -289}\special{pa 454 -289}%
\special{pa 451 -287}\special{pa 449 -286}\special{pa 446 -284}\special{pa 444 -283}%
\special{pa 442 -280}\special{pa 441 -278}\special{pa 440 -275}\special{pa 439 -273}%
\special{pa 438 -270}\special{pa 438 -267}\special{pa 438 -264}\special{pa 439 -262}%
\special{pa 440 -259}\special{pa 441 -256}\special{pa 442 -254}\special{pa 444 -252}%
\special{pa 446 -250}\special{pa 449 -248}\special{pa 451 -247}\special{pa 454 -246}%
\special{pa 456 -245}\special{pa 459 -245}\special{pa 462 -245}\special{pa 465 -245}%
\special{pa 468 -246}\special{pa 470 -247}\special{pa 473 -248}\special{pa 475 -250}%
\special{pa 477 -252}\special{pa 479 -254}\special{pa 480 -256}\special{pa 482 -259}%
\special{pa 482 -262}\special{pa 483 -264}\special{pa 483 -267}\special{pa 483 -267}%
\special{sh 1}\special{ip}%
\special{pa   483  -267}\special{pa   483  -270}\special{pa   482  -273}\special{pa   482  -275}%
\special{pa   480  -278}\special{pa   479  -280}\special{pa   477  -283}\special{pa   475  -284}%
\special{pa   473  -286}\special{pa   470  -287}\special{pa   468  -289}\special{pa   465  -289}%
\special{pa   462  -290}\special{pa   459  -290}\special{pa   456  -289}\special{pa   454  -289}%
\special{pa   451  -287}\special{pa   449  -286}\special{pa   446  -284}\special{pa   444  -283}%
\special{pa   442  -280}\special{pa   441  -278}\special{pa   440  -275}\special{pa   439  -273}%
\special{pa   438  -270}\special{pa   438  -267}\special{pa   438  -264}\special{pa   439  -262}%
\special{pa   440  -259}\special{pa   441  -256}\special{pa   442  -254}\special{pa   444  -252}%
\special{pa   446  -250}\special{pa   449  -248}\special{pa   451  -247}\special{pa   454  -246}%
\special{pa   456  -245}\special{pa   459  -245}\special{pa   462  -245}\special{pa   465  -245}%
\special{pa   468  -246}\special{pa   470  -247}\special{pa   473  -248}\special{pa   475  -250}%
\special{pa   477  -252}\special{pa   479  -254}\special{pa   480  -256}\special{pa   482  -259}%
\special{pa   482  -262}\special{pa   483  -264}\special{pa   483  -267}%
\special{fp}%
\special{pa -1293 -834}\special{pa -1293 -837}\special{pa -1293 -840}\special{pa -1294 -842}%
\special{pa -1295 -845}\special{pa -1297 -847}\special{pa -1299 -849}\special{pa -1301 -851}%
\special{pa -1303 -853}\special{pa -1305 -854}\special{pa -1308 -855}\special{pa -1311 -856}%
\special{pa -1314 -857}\special{pa -1316 -857}\special{pa -1319 -856}\special{pa -1322 -855}%
\special{pa -1325 -854}\special{pa -1327 -853}\special{pa -1329 -851}\special{pa -1331 -849}%
\special{pa -1333 -847}\special{pa -1335 -845}\special{pa -1336 -842}\special{pa -1337 -840}%
\special{pa -1337 -837}\special{pa -1337 -834}\special{pa -1337 -831}\special{pa -1337 -829}%
\special{pa -1336 -826}\special{pa -1335 -823}\special{pa -1333 -821}\special{pa -1331 -819}%
\special{pa -1329 -817}\special{pa -1327 -815}\special{pa -1325 -814}\special{pa -1322 -813}%
\special{pa -1319 -812}\special{pa -1316 -812}\special{pa -1314 -812}\special{pa -1311 -812}%
\special{pa -1308 -813}\special{pa -1305 -814}\special{pa -1303 -815}\special{pa -1301 -817}%
\special{pa -1299 -819}\special{pa -1297 -821}\special{pa -1295 -823}\special{pa -1294 -826}%
\special{pa -1293 -829}\special{pa -1293 -831}\special{pa -1293 -834}\special{pa -1293 -834}%
\special{sh 1}\special{ip}%
\special{pa -1293  -834}\special{pa -1293  -837}\special{pa -1293  -840}\special{pa -1294  -842}%
\special{pa -1295  -845}\special{pa -1297  -847}\special{pa -1299  -849}\special{pa -1301  -851}%
\special{pa -1303  -853}\special{pa -1305  -854}\special{pa -1308  -855}\special{pa -1311  -856}%
\special{pa -1314  -857}\special{pa -1316  -857}\special{pa -1319  -856}\special{pa -1322  -855}%
\special{pa -1325  -854}\special{pa -1327  -853}\special{pa -1329  -851}\special{pa -1331  -849}%
\special{pa -1333  -847}\special{pa -1335  -845}\special{pa -1336  -842}\special{pa -1337  -840}%
\special{pa -1337  -837}\special{pa -1337  -834}\special{pa -1337  -831}\special{pa -1337  -829}%
\special{pa -1336  -826}\special{pa -1335  -823}\special{pa -1333  -821}\special{pa -1331  -819}%
\special{pa -1329  -817}\special{pa -1327  -815}\special{pa -1325  -814}\special{pa -1322  -813}%
\special{pa -1319  -812}\special{pa -1316  -812}\special{pa -1314  -812}\special{pa -1311  -812}%
\special{pa -1308  -813}\special{pa -1305  -814}\special{pa -1303  -815}\special{pa -1301  -817}%
\special{pa -1299  -819}\special{pa -1297  -821}\special{pa -1295  -823}\special{pa -1294  -826}%
\special{pa -1293  -829}\special{pa -1293  -831}\special{pa -1293  -834}%
\special{fp}%
\special{pa -1159 0}\special{pa -1159 -3}\special{pa -1159 -6}\special{pa -1160 -8}%
\special{pa -1161 -11}\special{pa -1163 -13}\special{pa -1165 -15}\special{pa -1167 -17}%
\special{pa -1169 -19}\special{pa -1172 -20}\special{pa -1174 -21}\special{pa -1177 -22}%
\special{pa -1180 -22}\special{pa -1183 -22}\special{pa -1185 -22}\special{pa -1188 -21}%
\special{pa -1191 -20}\special{pa -1193 -19}\special{pa -1195 -17}\special{pa -1197 -15}%
\special{pa -1199 -13}\special{pa -1201 -11}\special{pa -1202 -8}\special{pa -1203 -6}%
\special{pa -1203 -3}\special{pa -1204 0}\special{pa -1203 3}\special{pa -1203 6}%
\special{pa -1202 8}\special{pa -1201 11}\special{pa -1199 13}\special{pa -1197 15}%
\special{pa -1195 17}\special{pa -1193 19}\special{pa -1191 20}\special{pa -1188 21}%
\special{pa -1185 22}\special{pa -1183 22}\special{pa -1180 22}\special{pa -1177 22}%
\special{pa -1174 21}\special{pa -1172 20}\special{pa -1169 19}\special{pa -1167 17}%
\special{pa -1165 15}\special{pa -1163 13}\special{pa -1161 11}\special{pa -1160 8}%
\special{pa -1159 6}\special{pa -1159 3}\special{pa -1159 0}\special{pa -1159 0}\special{sh 1}\special{ip}%
\special{pa -1159    -0}\special{pa -1159    -3}\special{pa -1159    -6}\special{pa -1160    -8}%
\special{pa -1161   -11}\special{pa -1163   -13}\special{pa -1165   -15}\special{pa -1167   -17}%
\special{pa -1169   -19}\special{pa -1172   -20}\special{pa -1174   -21}\special{pa -1177   -22}%
\special{pa -1180   -22}\special{pa -1183   -22}\special{pa -1185   -22}\special{pa -1188   -21}%
\special{pa -1191   -20}\special{pa -1193   -19}\special{pa -1195   -17}\special{pa -1197   -15}%
\special{pa -1199   -13}\special{pa -1201   -11}\special{pa -1202    -8}\special{pa -1203    -6}%
\special{pa -1203    -3}\special{pa -1204     0}\special{pa -1203     3}\special{pa -1203     6}%
\special{pa -1202     8}\special{pa -1201    11}\special{pa -1199    13}\special{pa -1197    15}%
\special{pa -1195    17}\special{pa -1193    19}\special{pa -1191    20}\special{pa -1188    21}%
\special{pa -1185    22}\special{pa -1183    22}\special{pa -1180    22}\special{pa -1177    22}%
\special{pa -1174    21}\special{pa -1172    20}\special{pa -1169    19}\special{pa -1167    17}%
\special{pa -1165    15}\special{pa -1163    13}\special{pa -1161    11}\special{pa -1160     8}%
\special{pa -1159     6}\special{pa -1159     3}\special{pa -1159     0}%
\special{fp}%
\special{pa 22 0}\special{pa 22 -3}\special{pa 22 -6}\special{pa 21 -8}\special{pa 20 -11}%
\special{pa 18 -13}\special{pa 16 -15}\special{pa 14 -17}\special{pa 12 -19}\special{pa 10 -20}%
\special{pa 7 -21}\special{pa 4 -22}\special{pa 1 -22}\special{pa -1 -22}\special{pa -4 -22}%
\special{pa -7 -21}\special{pa -10 -20}\special{pa -12 -19}\special{pa -14 -17}\special{pa -16 -15}%
\special{pa -18 -13}\special{pa -20 -11}\special{pa -21 -8}\special{pa -22 -6}\special{pa -22 -3}%
\special{pa -22 0}\special{pa -22 3}\special{pa -22 6}\special{pa -21 8}\special{pa -20 11}%
\special{pa -18 13}\special{pa -16 15}\special{pa -14 17}\special{pa -12 19}\special{pa -10 20}%
\special{pa -7 21}\special{pa -4 22}\special{pa -1 22}\special{pa 1 22}\special{pa 4 22}%
\special{pa 7 21}\special{pa 10 20}\special{pa 12 19}\special{pa 14 17}\special{pa 16 15}%
\special{pa 18 13}\special{pa 20 11}\special{pa 21 8}\special{pa 22 6}\special{pa 22 3}%
\special{pa 22 0}\special{pa 22 0}\special{sh 1}\special{ip}%
\special{pa    22    -0}\special{pa    22    -3}\special{pa    22    -6}\special{pa    21    -8}%
\special{pa    20   -11}\special{pa    18   -13}\special{pa    16   -15}\special{pa    14   -17}%
\special{pa    12   -19}\special{pa    10   -20}\special{pa     7   -21}\special{pa     4   -22}%
\special{pa     1   -22}\special{pa    -1   -22}\special{pa    -4   -22}\special{pa    -7   -21}%
\special{pa   -10   -20}\special{pa   -12   -19}\special{pa   -14   -17}\special{pa   -16   -15}%
\special{pa   -18   -13}\special{pa   -20   -11}\special{pa   -21    -8}\special{pa   -22    -6}%
\special{pa   -22    -3}\special{pa   -22     0}\special{pa   -22     3}\special{pa   -22     6}%
\special{pa   -21     8}\special{pa   -20    11}\special{pa   -18    13}\special{pa   -16    15}%
\special{pa   -14    17}\special{pa   -12    19}\special{pa   -10    20}\special{pa    -7    21}%
\special{pa    -4    22}\special{pa    -1    22}\special{pa     1    22}\special{pa     4    22}%
\special{pa     7    21}\special{pa    10    20}\special{pa    12    19}\special{pa    14    17}%
\special{pa    16    15}\special{pa    18    13}\special{pa    20    11}\special{pa    21     8}%
\special{pa    22     6}\special{pa    22     3}\special{pa    22     0}%
\special{fp}%
\end{picture}}%}
           $r$\tanni{m}離れた2点P,Qにそれぞれ$q_1$\tanni{C},
           $q_2$\tanni{C}の小帯電球が固定して置かれている。
           まわりの電場の様子を一平面上で調べたら,電気力線は図のようになった。それらは直線PQおよび
           その垂直2等分線AOに対して対称になっている。
        \begin{enumerate}
            \item $q_1$の符号は\Hako で,$q_2$の符号は\Hako である。そして,$|q_1|$と$|q_2|$の比の値は\Hako である。
            また,帯電体が点Aで受ける力の大きさは,点Bの場合より\Hako 。
            \item 図の点CとDを通る等電位線をそれぞれ図示せよ。
            \item 点O,A,B,C,Dの電位をそれぞれ$V_\mathrm{O}$,$V_\mathrm{A}$,$V_\mathrm{B}$,$V_\mathrm{C}$,$V_\mathrm{D}$\tanni{V}として,
            大小関係を$V_\mathrm{O}<V_\mathrm{A}=V_\mathrm{B}<\cdots $のように表せ。
            \item 正の電荷$q_0$\tanni{C}をA→B→C→D→Aの順にゆっくり移動させるとき,外力のする仕事が正となる区間はどこか。
            また,全体での仕事はいくらか。
        \end{enumerate}
    \end{mawarikomi}
