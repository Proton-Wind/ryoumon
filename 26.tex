\hakosyokika
\item
    \begin{mawarikomi}{150pt}{\begin{zahyou*}[ul=4mm](-10,2)(-10,1)
	% \drawline(-1,-1)(-1,13)(13,13)(13,-1)(-1,-1)
	\def\kakudo{-60}
%	\calcval{\kakudo *2*3.14159265/360}\TH
	\def\O{(0,0)}
	\def\B{(0,-9)}
	\def\F{(2,0)}
	\def\G{(2,0.5)}
	\def\H{(-2,0.5)}
	\def\I{(-2,0)}
	\Kaiten\O\B{\kakudo}\A
	\HenKo<henkoH=2.5ex>\B\O{$\ell $}
	\HenKo<henkoH=2.5ex>\O\A{$\ell $}
	\Nuritubusi*{\F\G\H\I\F}
	\Drawline{\O\B}
	\Drawline{\O\A}
	\Drawline{\I\F}
	\En*\B{0.4}
	\En\B{0.4}
	\En*[0]\A{0.4}
	\En\A{0.4}
	\Kakukigou\A\O\B<hankei=1.5>(-7pt,-8pt)[l]{60\Deg}
	\Put\B(5pt,0pt)[l]{B}
	\Put\O(8pt,-3pt)[t]{O}
	\Put\A(0pt,8pt)[b]{A}
\end{zahyou*}
}
        質量$m$の小球Aと$2m$の小球Bがあり,それぞれ長さ$\ell $の糸で天井の点Oからつるされている。Bを鉛直線に沿って静止させ,Aを糸が鉛直線から60\Deg 傾いた位置に持ち上げて,静かに放したところ,最下点でBに衝突した\\
        ~~AとBの衝突が完全弾性衝突のとき,衝突直後のBの速さは,重力加速度
        重力加速度の大きさを$g$とすると\Hako である。\\
        ~~AとBの衝突の直後にAが最下点でそのまま静止して,Bのみが運動する場合がある。
        このとき,AとBのはね返り係数は\Hako であり,Bは最下点より\Hako の高さまで上昇する。\\
        ~~次に,小球Aを取り去り,鉛直線に沿って静止させた小球Bに弾丸Cを水平に打ち込んだところ,BはCと一体になって運動を始めた。衝突直後の速さが衝突直前のCの速さの$\bunsuu{1}{5}$になったとすると,Cの質量は\Hako である。また,この衝突で失われた力学的エネルギーは,衝突直前のCの運動エネルギーの\Hako 倍である。
    \end{mawarikomi}