\hakosyokika
\item
    \begin{mawarikomi}(20pt,0pt){150pt}{
        %%% C:/vpn/KeTCindy/fig/fig124.tex 
%%% Generator=fig124.cdy 
{\unitlength=1cm%
\begin{picture}%
(5,5)(-2.5,-2.5)%
\special{pn 8}%
%
\special{pa   492   394}\special{pa   -98   394}\special{pa   -98  -197}\special{pa   492  -197}%
\special{pa   492   394}%
\special{fp}%
\special{pa  -886  -886}\special{pa  -886   886}%
\special{fp}%
\special{pa   886  -886}\special{pa   886   886}%
\special{fp}%
\settowidth{\Width}{Y}\setlength{\Width}{-0.5\Width}%
\settoheight{\Height}{Y}\settodepth{\Depth}{Y}\setlength{\Height}{\Depth}%
\put( -2.250,  2.400){\hspace*{\Width}\raisebox{\Height}{Y}}%
%
\settowidth{\Width}{X}\setlength{\Width}{-0.5\Width}%
\settoheight{\Height}{X}\settodepth{\Depth}{X}\setlength{\Height}{-\Height}%
\put( -2.250, -2.400){\hspace*{\Width}\raisebox{\Height}{X}}%
%
\settowidth{\Width}{N}\setlength{\Width}{-0.5\Width}%
\settoheight{\Height}{N}\settodepth{\Depth}{N}\setlength{\Height}{\Depth}%
\put(  2.250,  2.400){\hspace*{\Width}\raisebox{\Height}{N}}%
%
\settowidth{\Width}{M}\setlength{\Width}{-0.5\Width}%
\settoheight{\Height}{M}\settodepth{\Depth}{M}\setlength{\Height}{-\Height}%
\put(  2.250, -2.400){\hspace*{\Width}\raisebox{\Height}{M}}%
%
\settowidth{\Width}{C}\setlength{\Width}{0\Width}%
\settoheight{\Height}{C}\settodepth{\Depth}{C}\setlength{\Height}{-\Height}%
\put(  1.300, -1.050){\hspace*{\Width}\raisebox{\Height}{C}}%
%
\settowidth{\Width}{D}\setlength{\Width}{-1\Width}%
\settoheight{\Height}{D}\settodepth{\Depth}{D}\setlength{\Height}{-\Height}%
\put( -0.300, -1.050){\hspace*{\Width}\raisebox{\Height}{D}}%
%
\settowidth{\Width}{A}\setlength{\Width}{-1\Width}%
\settoheight{\Height}{A}\settodepth{\Depth}{A}\setlength{\Height}{\Depth}%
\put( -0.300,  0.550){\hspace*{\Width}\raisebox{\Height}{A}}%
%
\settowidth{\Width}{B}\setlength{\Width}{0\Width}%
\settoheight{\Height}{B}\settodepth{\Depth}{B}\setlength{\Height}{\Depth}%
\put(  1.300,  0.550){\hspace*{\Width}\raisebox{\Height}{B}}%
%
\settowidth{\Width}{コイル}\setlength{\Width}{-0.5\Width}%
\settoheight{\Height}{コイル}\settodepth{\Depth}{コイル}\setlength{\Height}{\Depth}%
\put(  0.500,  0.650){\hspace*{\Width}\raisebox{\Height}{コイル}}%
%
\special{pa 492 394}\special{pa 492 429}\special{fp}\special{pa 492 465}\special{pa 492 501}\special{fp}%
\special{pa 492 537}\special{pa 492 573}\special{fp}\special{pa 492 608}\special{pa 492 644}\special{fp}%
\special{pa 492 680}\special{pa 492 716}\special{fp}\special{pa 492 752}\special{pa 492 787}\special{fp}%
%
%
\special{pa -98 394}\special{pa -98 429}\special{fp}\special{pa -98 465}\special{pa -98 501}\special{fp}%
\special{pa -98 537}\special{pa -98 573}\special{fp}\special{pa -98 608}\special{pa -98 644}\special{fp}%
\special{pa -98 680}\special{pa -98 716}\special{fp}\special{pa -98 752}\special{pa -98 787}\special{fp}%
%
%
\special{pa -833 672}\special{pa -886 689}\special{pa -833 706}\special{pa -844 689}%
\special{pa -833 672}\special{pa -833 672}\special{sh 1}\special{ip}%
\special{pn 1}%
\special{pa  -833   672}\special{pa  -886   689}\special{pa  -833   706}\special{pa  -844   689}%
\special{pa  -833   672}%
\special{fp}%
\special{pn 8}%
\special{pa  -492   689}\special{pa  -844   689}%
\special{fp}%
\special{pa -151 706}\special{pa -98 689}\special{pa -151 672}\special{pa -140 689}%
\special{pa -151 706}\special{pa -151 706}\special{sh 1}\special{ip}%
\special{pn 1}%
\special{pa  -151   706}\special{pa   -98   689}\special{pa  -151   672}\special{pa  -140   689}%
\special{pa  -151   706}%
\special{fp}%
\special{pn 8}%
\special{pa  -492   689}\special{pa  -140   689}%
\special{fp}%
{%
\color[cmyk]{0,0,0,0}%
\special{pa -433 689}\special{pa -434 682}\special{pa -435 674}\special{pa -437 667}%
\special{pa -440 661}\special{pa -444 654}\special{pa -449 649}\special{pa -454 643}%
\special{pa -460 639}\special{pa -467 636}\special{pa -474 633}\special{pa -481 631}%
\special{pa -488 630}\special{pa -496 630}\special{pa -503 631}\special{pa -510 633}%
\special{pa -517 636}\special{pa -524 639}\special{pa -530 643}\special{pa -535 649}%
\special{pa -540 654}\special{pa -544 661}\special{pa -547 667}\special{pa -549 674}%
\special{pa -551 682}\special{pa -551 689}\special{pa -551 696}\special{pa -549 704}%
\special{pa -547 711}\special{pa -544 717}\special{pa -540 724}\special{pa -535 729}%
\special{pa -530 734}\special{pa -524 739}\special{pa -517 742}\special{pa -510 745}%
\special{pa -503 747}\special{pa -496 748}\special{pa -488 748}\special{pa -481 747}%
\special{pa -474 745}\special{pa -467 742}\special{pa -460 739}\special{pa -454 734}%
\special{pa -449 729}\special{pa -444 724}\special{pa -440 717}\special{pa -437 711}%
\special{pa -435 704}\special{pa -434 696}\special{pa -433 689}\special{pa -433 689}%
\special{sh 1}\special{ip}%
}%
\settowidth{\Width}{$b$}\setlength{\Width}{-0.5\Width}%
\settoheight{\Height}{$b$}\settodepth{\Depth}{$b$}\setlength{\Height}{-0.5\Height}\setlength{\Depth}{0.5\Depth}\addtolength{\Height}{\Depth}%
\put( -1.250, -1.750){\hspace*{\Width}\raisebox{\Height}{$b$}}%
%
\special{pa -46 672}\special{pa -98 689}\special{pa -46 706}\special{pa -56 689}\special{pa -46 672}%
\special{pa -46 672}\special{sh 1}\special{ip}%
\special{pn 1}%
\special{pa   -46   672}\special{pa   -98   689}\special{pa   -46   706}\special{pa   -56   689}%
\special{pa   -46   672}%
\special{fp}%
\special{pn 8}%
\special{pa   197   689}\special{pa   -56   689}%
\special{fp}%
\special{pa 440 706}\special{pa 492 689}\special{pa 440 672}\special{pa 450 689}\special{pa 440 706}%
\special{pa 440 706}\special{sh 1}\special{ip}%
\special{pn 1}%
\special{pa   440   706}\special{pa   492   689}\special{pa   440   672}\special{pa   450   689}%
\special{pa   440   706}%
\special{fp}%
\special{pn 8}%
\special{pa   197   689}\special{pa   450   689}%
\special{fp}%
{%
\color[cmyk]{0,0,0,0}%
\special{pa 256 689}\special{pa 255 682}\special{pa 254 674}\special{pa 252 667}\special{pa 249 661}%
\special{pa 245 654}\special{pa 240 649}\special{pa 234 643}\special{pa 228 639}\special{pa 222 636}%
\special{pa 215 633}\special{pa 208 631}\special{pa 201 630}\special{pa 193 630}\special{pa 186 631}%
\special{pa 179 633}\special{pa 172 636}\special{pa 165 639}\special{pa 159 643}\special{pa 154 649}%
\special{pa 149 654}\special{pa 145 661}\special{pa 142 667}\special{pa 140 674}\special{pa 138 682}%
\special{pa 138 689}\special{pa 138 696}\special{pa 140 704}\special{pa 142 711}\special{pa 145 717}%
\special{pa 149 724}\special{pa 154 729}\special{pa 159 734}\special{pa 165 739}\special{pa 172 742}%
\special{pa 179 745}\special{pa 186 747}\special{pa 193 748}\special{pa 201 748}\special{pa 208 747}%
\special{pa 215 745}\special{pa 222 742}\special{pa 228 739}\special{pa 234 734}\special{pa 240 729}%
\special{pa 245 724}\special{pa 249 717}\special{pa 252 711}\special{pa 254 704}\special{pa 255 696}%
\special{pa 256 689}\special{pa 256 689}\special{sh 1}\special{ip}%
}%
\settowidth{\Width}{$a$}\setlength{\Width}{-0.5\Width}%
\settoheight{\Height}{$a$}\settodepth{\Depth}{$a$}\setlength{\Height}{-0.5\Height}\setlength{\Depth}{0.5\Depth}\addtolength{\Height}{\Depth}%
\put(  0.500, -1.750){\hspace*{\Width}\raisebox{\Height}{$a$}}%
%
\special{pa 545 672}\special{pa 492 689}\special{pa 545 706}\special{pa 534 689}\special{pa 545 672}%
\special{pa 545 672}\special{sh 1}\special{ip}%
\special{pn 1}%
\special{pa   545   672}\special{pa   492   689}\special{pa   545   706}\special{pa   534   689}%
\special{pa   545   672}%
\special{fp}%
\special{pn 8}%
\special{pa   689   689}\special{pa   534   689}%
\special{fp}%
\special{pa 833 706}\special{pa 886 689}\special{pa 833 672}\special{pa 844 689}\special{pa 833 706}%
\special{pa 833 706}\special{sh 1}\special{ip}%
\special{pn 1}%
\special{pa   833   706}\special{pa   886   689}\special{pa   833   672}\special{pa   844   689}%
\special{pa   833   706}%
\special{fp}%
\special{pn 8}%
\special{pa   689   689}\special{pa   844   689}%
\special{fp}%
{%
\color[cmyk]{0,0,0,0}%
\special{pa 748 689}\special{pa 748 682}\special{pa 746 674}\special{pa 744 667}\special{pa 741 661}%
\special{pa 737 654}\special{pa 732 649}\special{pa 727 643}\special{pa 721 639}\special{pa 714 636}%
\special{pa 707 633}\special{pa 700 631}\special{pa 693 630}\special{pa 685 630}\special{pa 678 631}%
\special{pa 671 633}\special{pa 664 636}\special{pa 657 639}\special{pa 651 643}\special{pa 646 649}%
\special{pa 641 654}\special{pa 637 661}\special{pa 634 667}\special{pa 632 674}\special{pa 630 682}%
\special{pa 630 689}\special{pa 630 696}\special{pa 632 704}\special{pa 634 711}\special{pa 637 717}%
\special{pa 641 724}\special{pa 646 729}\special{pa 651 734}\special{pa 657 739}\special{pa 664 742}%
\special{pa 671 745}\special{pa 678 747}\special{pa 685 748}\special{pa 693 748}\special{pa 700 747}%
\special{pa 707 745}\special{pa 714 742}\special{pa 721 739}\special{pa 727 734}\special{pa 732 729}%
\special{pa 737 724}\special{pa 741 717}\special{pa 744 711}\special{pa 746 704}\special{pa 748 696}%
\special{pa 748 689}\special{pa 748 689}\special{sh 1}\special{ip}%
}%
\settowidth{\Width}{$c$}\setlength{\Width}{-0.5\Width}%
\settoheight{\Height}{$c$}\settodepth{\Depth}{$c$}\setlength{\Height}{-0.5\Height}\setlength{\Depth}{0.5\Depth}\addtolength{\Height}{\Depth}%
\put(  1.750, -1.750){\hspace*{\Width}\raisebox{\Height}{$c$}}%
%
\end{picture}}%
    }
    十分に長い導線XYと導線MNが平行に固定されている。その間に1辺の長さが$a$の正方形のコイルABCDを置く。辺ADはXYに平行で,ADとXYの距離は$b$,BCとMNの距離は$c$であり,周囲の透磁率を$\mu $とする。\\
    ~~まず,導線XYに強さ$I_1$の電流をXからYの向きに流した。辺AD上での磁場の強さは\Hako となる。次にこの位置の磁場の強さを0にするために,導線MNに強さ$I_2$の電流を\Hako の向きに流す。$I_2$は$I_1$の\Hako 倍である。また,辺BC上での磁場の強さは\Hako $\times I_1$となる。\\
    ~~この状態でコイルに強さ$i$の電流を反時計回りに流すと,コイルは電流$I_1$,$I_2$による磁場から力を受ける。コイル全体が受ける力の大きさは,\Hako $\times I_1$となり,向きはコイルが導線\Hako に近づく向きとなる。\\
    ~~そこで,導線MNに流す電流の向きを変え,コイルの全体にはたらく力が0となるように電流を調節した。そのときのMNの電流の強さは$I_1$の\Hako 倍である。
    \end{mawarikomi}