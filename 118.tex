\hakosyokika
\item
    \begin{mawarikomi}(20pt,0pt){150pt}{
        %%% C:/vpn/vpn/KeTCindy/fig/fig118_1.tex 
%%% Generator=fig118_1.cdy 
{\unitlength=1.5cm%
\begin{picture}%
(3,2.75)(0.25,-0.5)%
\special{pn 8}%
%
{%
\color[cmyk]{0,0,0,0.5}%
\special{pa 443 -1154}\special{pa 443 -617}\special{pa 1624 -617}\special{pa 1624 -1154}%
\special{pa 443 -1154}\special{pa 443 -1154}\special{sh 1}\special{ip}%
}%
\special{pa   443 -1154}\special{pa   443  -617}\special{pa  1624  -617}\special{pa  1624 -1154}%
\special{pa   443 -1154}%
\special{fp}%
{%
\color[cmyk]{0,0,0,0.3}%
\special{pa 597 -495}\special{pa 597 -96}\special{pa 1470 -96}\special{pa 1470 -495}%
\special{pa 597 -495}\special{pa 597 -495}\special{sh 1}\special{ip}%
}%
\special{pa   709  -148}\special{pa   709  -443}%
\special{fp}%
\special{pn 16}%
\special{pa   768  -236}\special{pa   768  -354}%
\special{fp}%
\special{pn 8}%
\special{pa  1033  -295}\special{pa   768  -295}%
\special{fp}%
\special{pa   443  -295}\special{pa   709  -295}%
\special{fp}%
\special{pa  1440  -368}\special{pa  1145  -369}\special{pa  1145  -221}\special{pa  1440  -221}%
\special{pa  1440  -368}%
\special{fp}%
\special{pa  1033  -295}\special{pa  1145  -295}%
\special{fp}%
\special{pa  1551  -295}\special{pa  1440  -295}%
\special{fp}%
\special{pa  1403  -960}\special{pa  1107  -960}\special{pa  1107  -812}\special{pa  1403  -812}%
\special{pa  1403  -960}%
\special{fp}%
\special{pa  1033  -886}\special{pa  1107  -886}%
\special{fp}%
\special{pa  1476  -886}\special{pa  1403  -886}%
\special{fp}%
\special{pa 454 -295}\special{pa 454 -297}\special{pa 454 -298}\special{pa 453 -299}%
\special{pa 453 -301}\special{pa 452 -302}\special{pa 451 -303}\special{pa 450 -304}%
\special{pa 449 -305}\special{pa 448 -305}\special{pa 446 -306}\special{pa 445 -306}%
\special{pa 444 -306}\special{pa 442 -306}\special{pa 441 -306}\special{pa 439 -306}%
\special{pa 438 -305}\special{pa 437 -305}\special{pa 436 -304}\special{pa 435 -303}%
\special{pa 434 -302}\special{pa 433 -301}\special{pa 432 -299}\special{pa 432 -298}%
\special{pa 432 -297}\special{pa 432 -295}\special{pa 432 -294}\special{pa 432 -292}%
\special{pa 432 -291}\special{pa 433 -290}\special{pa 434 -289}\special{pa 435 -288}%
\special{pa 436 -287}\special{pa 437 -286}\special{pa 438 -285}\special{pa 439 -285}%
\special{pa 441 -284}\special{pa 442 -284}\special{pa 444 -284}\special{pa 445 -284}%
\special{pa 446 -285}\special{pa 448 -285}\special{pa 449 -286}\special{pa 450 -287}%
\special{pa 451 -288}\special{pa 452 -289}\special{pa 453 -290}\special{pa 453 -291}%
\special{pa 454 -292}\special{pa 454 -294}\special{pa 454 -295}\special{pa 454 -295}%
\special{sh 1}\special{ip}%
\special{pa   454  -295}\special{pa   454  -297}\special{pa   454  -298}\special{pa   453  -299}%
\special{pa   453  -301}\special{pa   452  -302}\special{pa   451  -303}\special{pa   450  -304}%
\special{pa   449  -305}\special{pa   448  -305}\special{pa   446  -306}\special{pa   445  -306}%
\special{pa   444  -306}\special{pa   442  -306}\special{pa   441  -306}\special{pa   439  -306}%
\special{pa   438  -305}\special{pa   437  -305}\special{pa   436  -304}\special{pa   435  -303}%
\special{pa   434  -302}\special{pa   433  -301}\special{pa   432  -299}\special{pa   432  -298}%
\special{pa   432  -297}\special{pa   432  -295}\special{pa   432  -294}\special{pa   432  -292}%
\special{pa   432  -291}\special{pa   433  -290}\special{pa   434  -289}\special{pa   435  -288}%
\special{pa   436  -287}\special{pa   437  -286}\special{pa   438  -285}\special{pa   439  -285}%
\special{pa   441  -284}\special{pa   442  -284}\special{pa   444  -284}\special{pa   445  -284}%
\special{pa   446  -285}\special{pa   448  -285}\special{pa   449  -286}\special{pa   450  -287}%
\special{pa   451  -288}\special{pa   452  -289}\special{pa   453  -290}\special{pa   453  -291}%
\special{pa   454  -292}\special{pa   454  -294}\special{pa   454  -295}%
\special{fp}%
\special{pa 1562 -295}\special{pa 1562 -296}\special{pa 1562 -297}\special{pa 1561 -299}%
\special{pa 1561 -300}\special{pa 1560 -301}\special{pa 1559 -302}\special{pa 1558 -303}%
\special{pa 1557 -304}\special{pa 1556 -305}\special{pa 1555 -305}\special{pa 1553 -306}%
\special{pa 1552 -306}\special{pa 1550 -306}\special{pa 1549 -306}\special{pa 1548 -305}%
\special{pa 1546 -305}\special{pa 1545 -304}\special{pa 1544 -303}\special{pa 1543 -302}%
\special{pa 1542 -301}\special{pa 1541 -300}\special{pa 1541 -299}\special{pa 1540 -297}%
\special{pa 1540 -296}\special{pa 1540 -295}\special{pa 1540 -293}\special{pa 1540 -292}%
\special{pa 1541 -290}\special{pa 1541 -289}\special{pa 1542 -288}\special{pa 1543 -287}%
\special{pa 1544 -286}\special{pa 1545 -285}\special{pa 1546 -284}\special{pa 1548 -284}%
\special{pa 1549 -283}\special{pa 1550 -283}\special{pa 1552 -283}\special{pa 1553 -283}%
\special{pa 1555 -284}\special{pa 1556 -284}\special{pa 1557 -285}\special{pa 1558 -286}%
\special{pa 1559 -287}\special{pa 1560 -288}\special{pa 1561 -289}\special{pa 1561 -290}%
\special{pa 1562 -292}\special{pa 1562 -293}\special{pa 1562 -295}\special{pa 1562 -295}%
\special{sh 1}\special{ip}%
\special{pa  1562  -295}\special{pa  1562  -296}\special{pa  1562  -297}\special{pa  1561  -299}%
\special{pa  1561  -300}\special{pa  1560  -301}\special{pa  1559  -302}\special{pa  1558  -303}%
\special{pa  1557  -304}\special{pa  1556  -305}\special{pa  1555  -305}\special{pa  1553  -306}%
\special{pa  1552  -306}\special{pa  1550  -306}\special{pa  1549  -306}\special{pa  1548  -305}%
\special{pa  1546  -305}\special{pa  1545  -304}\special{pa  1544  -303}\special{pa  1543  -302}%
\special{pa  1542  -301}\special{pa  1541  -300}\special{pa  1541  -299}\special{pa  1540  -297}%
\special{pa  1540  -296}\special{pa  1540  -295}\special{pa  1540  -293}\special{pa  1540  -292}%
\special{pa  1541  -290}\special{pa  1541  -289}\special{pa  1542  -288}\special{pa  1543  -287}%
\special{pa  1544  -286}\special{pa  1545  -285}\special{pa  1546  -284}\special{pa  1548  -284}%
\special{pa  1549  -283}\special{pa  1550  -283}\special{pa  1552  -283}\special{pa  1553  -283}%
\special{pa  1555  -284}\special{pa  1556  -284}\special{pa  1557  -285}\special{pa  1558  -286}%
\special{pa  1559  -287}\special{pa  1560  -288}\special{pa  1561  -289}\special{pa  1561  -290}%
\special{pa  1562  -292}\special{pa  1562  -293}\special{pa  1562  -295}%
\special{fp}%
\settowidth{\Width}{図1}\setlength{\Width}{-0.5\Width}%
\settoheight{\Height}{図1}\settodepth{\Depth}{図1}\setlength{\Height}{-0.5\Height}\setlength{\Depth}{0.5\Depth}\addtolength{\Height}{\Depth}%
\put(  1.750, -0.370){\hspace*{\Width}\raisebox{\Height}{図1}}%
%
\settowidth{\Width}{電池D}\setlength{\Width}{-0.5\Width}%
\settoheight{\Height}{電池D}\settodepth{\Depth}{電池D}\setlength{\Height}{-0.5\Height}\setlength{\Depth}{0.5\Depth}\addtolength{\Height}{\Depth}%
\put(  1.750,  0.000){\hspace*{\Width}\raisebox{\Height}{電池D}}%
%
\settowidth{\Width}{電圧計V}\setlength{\Width}{-0.5\Width}%
\settoheight{\Height}{電圧計V}\settodepth{\Depth}{電圧計V}\setlength{\Height}{-0.5\Height}\setlength{\Depth}{0.5\Depth}\addtolength{\Height}{\Depth}%
\put(  1.750,  2.100){\hspace*{\Width}\raisebox{\Height}{電圧計V}}%
%
\settowidth{\Width}{a}\setlength{\Width}{-0.5\Width}%
\settoheight{\Height}{a}\settodepth{\Depth}{a}\setlength{\Height}{-\Height}%
\put(  0.750,  0.433){\hspace*{\Width}\raisebox{\Height}{a}}%
%
\settowidth{\Width}{b}\setlength{\Width}{-0.5\Width}%
\settoheight{\Height}{b}\settodepth{\Depth}{b}\setlength{\Height}{-\Height}%
\put(  2.630,  0.433){\hspace*{\Width}\raisebox{\Height}{b}}%
%
\special{pa   443  -295}\special{pa   295  -295}\special{pa   295  -886}\special{pa   443  -886}%
\special{fp}%
\settowidth{\Width}{$r$}\setlength{\Width}{-0.5\Width}%
\settoheight{\Height}{$r$}\settodepth{\Depth}{$r$}\setlength{\Height}{-\Height}%
\put(  2.190,  0.317){\hspace*{\Width}\raisebox{\Height}{$r$}}%
%
\settowidth{\Width}{$r_\mathrm{V}$}\setlength{\Width}{-0.5\Width}%
\settoheight{\Height}{$r_\mathrm{V}$}\settodepth{\Depth}{$r_\mathrm{V}$}\setlength{\Height}{-\Height}%
\put(  2.120,  1.317){\hspace*{\Width}\raisebox{\Height}{$r_\mathrm{V}$}}%
%
\settowidth{\Width}{$E$}\setlength{\Width}{0\Width}%
\settoheight{\Height}{$E$}\settodepth{\Depth}{$E$}\setlength{\Height}{\Depth}%
\put(  1.283,  0.283){\hspace*{\Width}\raisebox{\Height}{$E$}}%
%
\special{pa  1551  -295}\special{pa  1772  -295}\special{pa  1772  -886}\special{pa  1476  -886}%
\special{fp}%
\special{pa   880  -886}\special{pa   879  -904}\special{pa   875  -921}\special{pa   870  -938}%
\special{pa   862  -954}\special{pa   853  -969}\special{pa   842  -983}\special{pa   829  -995}%
\special{pa   814 -1005}\special{pa   799 -1014}\special{pa   782 -1021}\special{pa   765 -1025}%
\special{pa   747 -1027}\special{pa   729 -1027}\special{pa   712 -1025}\special{pa   694 -1021}%
\special{pa   678 -1014}\special{pa   662 -1005}\special{pa   648  -995}\special{pa   635  -983}%
\special{pa   624  -969}\special{pa   614  -954}\special{pa   606  -938}\special{pa   601  -921}%
\special{pa   598  -904}\special{pa   596  -886}\special{pa   598  -868}\special{pa   601  -851}%
\special{pa   606  -834}\special{pa   614  -818}\special{pa   624  -803}\special{pa   635  -789}%
\special{pa   648  -777}\special{pa   662  -766}\special{pa   678  -758}\special{pa   694  -751}%
\special{pa   712  -747}\special{pa   729  -744}\special{pa   747  -744}\special{pa   765  -747}%
\special{pa   782  -751}\special{pa   799  -758}\special{pa   814  -766}\special{pa   829  -777}%
\special{pa   842  -789}\special{pa   853  -803}\special{pa   862  -818}\special{pa   870  -834}%
\special{pa   875  -851}\special{pa   879  -868}\special{pa   880  -886}%
\special{fp}%
\special{pa   443  -886}\special{pa   596  -886}%
\special{fp}%
\special{pa  1033  -886}\special{pa   880  -886}%
\special{fp}%
\settowidth{\Width}{}\setlength{\Width}{-0.5\Width}%
\settoheight{\Height}{}\settodepth{\Depth}{}\setlength{\Height}{-0.5\Height}\setlength{\Depth}{0.5\Depth}\addtolength{\Height}{\Depth}%
\put(  1.250,  1.500){\hspace*{\Width}\raisebox{\Height}{}}%
%
\special{pa 687 -961}\special{pa 638 -986}\special{pa 663 -937}\special{pa 667 -957}%
\special{pa 687 -961}\special{pa 687 -961}\special{sh 1}\special{ip}%
\special{pn 1}%
\special{pa   687  -961}\special{pa   638  -986}\special{pa   663  -937}\special{pa   667  -957}%
\special{pa   687  -961}%
\special{fp}%
\special{pn 8}%
\special{pa   839  -785}\special{pa   667  -957}%
\special{fp}%
\end{picture}}%
        %%% C:/vpn/vpn/KeTCindy/fig/fig118_2.tex 
%%% Generator=fig118_2.cdy 
{\unitlength=1.5cm%
\begin{picture}%
(3,3.75)(0.25,-1)%
\special{pn 8}%
%
{%
\color[cmyk]{0,0,0,0.3}%
\special{pa 413 -118}\special{pa 413 118}\special{pa 827 118}\special{pa 827 -118}%
\special{pa 413 -118}\special{pa 413 -118}\special{sh 1}\special{ip}%
}%
\special{pa   455    89}\special{pa   455   -89}%
\special{fp}%
\special{pn 16}%
\special{pa   490    35}\special{pa   490   -35}%
\special{fp}%
\special{pn 8}%
\special{pa   591    -0}\special{pa   490    -0}%
\special{fp}%
\special{pa   354    -0}\special{pa   455    -0}%
\special{fp}%
\special{pa   809  -148}\special{pa   809  -325}%
\special{fp}%
\special{pn 16}%
\special{pa   844  -201}\special{pa   844  -272}%
\special{fp}%
\special{pn 8}%
\special{pa  1358  -236}\special{pa   844  -236}%
\special{fp}%
\special{pa   295  -236}\special{pa   809  -236}%
\special{fp}%
\special{pa   957 -1093}\special{pa   957 -1270}%
\special{fp}%
\special{pn 16}%
\special{pa   992 -1146}\special{pa   992 -1217}%
\special{fp}%
\special{pn 8}%
\special{pa  1654 -1181}\special{pa   992 -1181}%
\special{fp}%
\special{pa   295 -1181}\special{pa   957 -1181}%
\special{fp}%
\special{pa   797   -30}\special{pa   679   -30}\special{pa   679    30}\special{pa   797    30}%
\special{pa   797   -30}%
\special{fp}%
\special{pa   591    -0}\special{pa   679    -0}%
\special{fp}%
\special{pa   886    -0}\special{pa   797    -0}%
\special{fp}%
\special{pa  1624  -531}\special{pa  1623  -543}\special{pa  1621  -554}\special{pa  1618  -564}%
\special{pa  1613  -574}\special{pa  1607  -584}\special{pa  1600  -592}\special{pa  1592  -600}%
\special{pa  1583  -606}\special{pa  1573  -612}\special{pa  1563  -616}\special{pa  1552  -619}%
\special{pa  1541  -620}\special{pa  1530  -620}\special{pa  1519  -619}\special{pa  1508  -616}%
\special{pa  1498  -612}\special{pa  1488  -606}\special{pa  1479  -600}\special{pa  1471  -592}%
\special{pa  1464  -584}\special{pa  1458  -574}\special{pa  1453  -564}\special{pa  1450  -554}%
\special{pa  1448  -543}\special{pa  1447  -531}\special{pa  1448  -520}\special{pa  1450  -509}%
\special{pa  1453  -499}\special{pa  1458  -489}\special{pa  1464  -479}\special{pa  1471  -471}%
\special{pa  1479  -463}\special{pa  1488  -457}\special{pa  1498  -451}\special{pa  1508  -447}%
\special{pa  1519  -444}\special{pa  1530  -443}\special{pa  1541  -443}\special{pa  1552  -444}%
\special{pa  1563  -447}\special{pa  1573  -451}\special{pa  1583  -457}\special{pa  1592  -463}%
\special{pa  1600  -471}\special{pa  1607  -479}\special{pa  1613  -489}\special{pa  1618  -499}%
\special{pa  1621  -509}\special{pa  1623  -520}\special{pa  1624  -531}%
\special{fp}%
\special{pa  1535  -413}\special{pa  1535  -443}%
\special{fp}%
\special{pa  1535  -650}\special{pa  1535  -620}%
\special{fp}%
\settowidth{\Width}{G}\setlength{\Width}{-0.5\Width}%
\settoheight{\Height}{G}\settodepth{\Depth}{G}\setlength{\Height}{-0.5\Height}\setlength{\Depth}{0.5\Depth}\addtolength{\Height}{\Depth}%
\put(  2.600,  0.900){\hspace*{\Width}\raisebox{\Height}{G}}%
%
\settowidth{\Width}{C}\setlength{\Width}{-0.5\Width}%
\settoheight{\Height}{C}\settodepth{\Depth}{C}\setlength{\Height}{\Depth}%
\put(  2.200,  1.433){\hspace*{\Width}\raisebox{\Height}{C}}%
%
\settowidth{\Width}{A}\setlength{\Width}{-1\Width}%
\settoheight{\Height}{A}\settodepth{\Depth}{A}\setlength{\Height}{-0.5\Height}\setlength{\Depth}{0.5\Depth}\addtolength{\Height}{\Depth}%
\put(  0.467,  1.400){\hspace*{\Width}\raisebox{\Height}{A}}%
%
\settowidth{\Width}{B}\setlength{\Width}{0\Width}%
\settoheight{\Height}{B}\settodepth{\Depth}{B}\setlength{\Height}{-0.5\Height}\setlength{\Depth}{0.5\Depth}\addtolength{\Height}{\Depth}%
\put(  2.833,  1.400){\hspace*{\Width}\raisebox{\Height}{B}}%
%
\settowidth{\Width}{\maru{1}}\setlength{\Width}{-0.5\Width}%
\settoheight{\Height}{\maru{1}}\settodepth{\Depth}{\maru{1}}\setlength{\Height}{\Depth}%
\put(  2.300,  0.467){\hspace*{\Width}\raisebox{\Height}{\maru{1}}}%
%
\settowidth{\Width}{\maru{2}}\setlength{\Width}{-0.5\Width}%
\settoheight{\Height}{\maru{2}}\settodepth{\Depth}{\maru{2}}\setlength{\Height}{-\Height}%
\put(  2.300, -0.067){\hspace*{\Width}\raisebox{\Height}{\maru{2}}}%
%
\settowidth{\Width}{電池D}\setlength{\Width}{-0.5\Width}%
\settoheight{\Height}{電池D}\settodepth{\Depth}{電池D}\setlength{\Height}{-\Height}%
\put(  1.000, -0.433){\hspace*{\Width}\raisebox{\Height}{電池D}}%
%
\settowidth{\Width}{図2}\setlength{\Width}{-0.5\Width}%
\settoheight{\Height}{図2}\settodepth{\Depth}{図2}\setlength{\Height}{-0.5\Height}\setlength{\Depth}{0.5\Depth}\addtolength{\Height}{\Depth}%
\put(  1.750, -0.750){\hspace*{\Width}\raisebox{\Height}{図2}}%
%
\settowidth{\Width}{$\mathrm{E_0}$}\setlength{\Width}{-0.5\Width}%
\settoheight{\Height}{$\mathrm{E_0}$}\settodepth{\Depth}{$\mathrm{E_0}$}\setlength{\Height}{\Depth}%
\put(  1.650,  2.190){\hspace*{\Width}\raisebox{\Height}{$\mathrm{E_0}$}}%
%
\settowidth{\Width}{$\mathrm{E_S}$}\setlength{\Width}{-0.5\Width}%
\settoheight{\Height}{$\mathrm{E_S}$}\settodepth{\Depth}{$\mathrm{E_S}$}\setlength{\Height}{\Depth}%
\put(  1.400,  0.590){\hspace*{\Width}\raisebox{\Height}{$\mathrm{E_S}$}}%
%
\settowidth{\Width}{$E$}\setlength{\Width}{-0.5\Width}%
\settoheight{\Height}{$E$}\settodepth{\Depth}{$E$}\setlength{\Height}{-\Height}%
\put(  0.800, -0.177){\hspace*{\Width}\raisebox{\Height}{$E$}}%
%
\settowidth{\Width}{$r$}\setlength{\Width}{-0.5\Width}%
\settoheight{\Height}{$r$}\settodepth{\Depth}{$r$}\setlength{\Height}{-\Height}%
\put(  1.250, -0.100){\hspace*{\Width}\raisebox{\Height}{$r$}}%
%
\special{pn 24}%
\special{pa   295  -827}\special{pa  1654  -827}%
\special{fp}%
\special{pn 8}%
\special{pa  1358    -0}\special{pa   886    -0}%
\special{fp}%
\special{pa   354    -0}\special{pa   295    -0}\special{pa   295 -1181}%
\special{fp}%
\special{pa  1654 -1181}\special{pa  1654  -827}%
\special{fp}%
\special{pa  1476  -118}\special{pa  1535  -118}%
\special{fp}%
\special{pa  1535  -650}\special{pa  1535  -709}\special{pa  1299  -709}%
\special{fp}%
\special{pa 956 0}\special{pa 956 -1}\special{pa 956 -3}\special{pa 955 -4}\special{pa 955 -5}%
\special{pa 954 -7}\special{pa 953 -8}\special{pa 952 -9}\special{pa 951 -9}\special{pa 950 -10}%
\special{pa 948 -11}\special{pa 947 -11}\special{pa 946 -11}\special{pa 944 -11}\special{pa 943 -11}%
\special{pa 941 -11}\special{pa 940 -10}\special{pa 939 -9}\special{pa 938 -9}\special{pa 937 -8}%
\special{pa 936 -7}\special{pa 935 -5}\special{pa 934 -4}\special{pa 934 -3}\special{pa 934 -1}%
\special{pa 934 0}\special{pa 934 1}\special{pa 934 3}\special{pa 934 4}\special{pa 935 5}%
\special{pa 936 7}\special{pa 937 8}\special{pa 938 9}\special{pa 939 9}\special{pa 940 10}%
\special{pa 941 11}\special{pa 943 11}\special{pa 944 11}\special{pa 946 11}\special{pa 947 11}%
\special{pa 948 11}\special{pa 950 10}\special{pa 951 9}\special{pa 952 9}\special{pa 953 8}%
\special{pa 954 7}\special{pa 955 5}\special{pa 955 4}\special{pa 956 3}\special{pa 956 1}%
\special{pa 956 0}\special{pa 956 0}\special{sh 1}\special{ip}%
\special{pa   956    -0}\special{pa   956    -1}\special{pa   956    -3}\special{pa   955    -4}%
\special{pa   955    -5}\special{pa   954    -7}\special{pa   953    -8}\special{pa   952    -9}%
\special{pa   951    -9}\special{pa   950   -10}\special{pa   948   -11}\special{pa   947   -11}%
\special{pa   946   -11}\special{pa   944   -11}\special{pa   943   -11}\special{pa   941   -11}%
\special{pa   940   -10}\special{pa   939    -9}\special{pa   938    -9}\special{pa   937    -8}%
\special{pa   936    -7}\special{pa   935    -5}\special{pa   934    -4}\special{pa   934    -3}%
\special{pa   934    -1}\special{pa   934     0}\special{pa   934     1}\special{pa   934     3}%
\special{pa   934     4}\special{pa   935     5}\special{pa   936     7}\special{pa   937     8}%
\special{pa   938     9}\special{pa   939     9}\special{pa   940    10}\special{pa   941    11}%
\special{pa   943    11}\special{pa   944    11}\special{pa   946    11}\special{pa   947    11}%
\special{pa   948    11}\special{pa   950    10}\special{pa   951     9}\special{pa   952     9}%
\special{pa   953     8}\special{pa   954     7}\special{pa   955     5}\special{pa   955     4}%
\special{pa   956     3}\special{pa   956     1}\special{pa   956     0}%
\special{fp}%
\special{pa 366 0}\special{pa 365 -1}\special{pa 365 -3}\special{pa 365 -4}\special{pa 364 -5}%
\special{pa 363 -7}\special{pa 363 -8}\special{pa 361 -9}\special{pa 360 -9}\special{pa 359 -10}%
\special{pa 358 -11}\special{pa 356 -11}\special{pa 355 -11}\special{pa 354 -11}\special{pa 352 -11}%
\special{pa 351 -11}\special{pa 350 -10}\special{pa 348 -9}\special{pa 347 -9}\special{pa 346 -8}%
\special{pa 345 -7}\special{pa 344 -5}\special{pa 344 -4}\special{pa 343 -3}\special{pa 343 -1}%
\special{pa 343 0}\special{pa 343 1}\special{pa 343 3}\special{pa 344 4}\special{pa 344 5}%
\special{pa 345 7}\special{pa 346 8}\special{pa 347 9}\special{pa 348 9}\special{pa 350 10}%
\special{pa 351 11}\special{pa 352 11}\special{pa 354 11}\special{pa 355 11}\special{pa 356 11}%
\special{pa 358 11}\special{pa 359 10}\special{pa 360 9}\special{pa 361 9}\special{pa 363 8}%
\special{pa 363 7}\special{pa 364 5}\special{pa 365 4}\special{pa 365 3}\special{pa 365 1}%
\special{pa 366 0}\special{pa 366 0}\special{sh 1}\special{ip}%
\special{pa   366    -0}\special{pa   365    -1}\special{pa   365    -3}\special{pa   365    -4}%
\special{pa   364    -5}\special{pa   363    -7}\special{pa   363    -8}\special{pa   361    -9}%
\special{pa   360    -9}\special{pa   359   -10}\special{pa   358   -11}\special{pa   356   -11}%
\special{pa   355   -11}\special{pa   354   -11}\special{pa   352   -11}\special{pa   351   -11}%
\special{pa   350   -10}\special{pa   348    -9}\special{pa   347    -9}\special{pa   346    -8}%
\special{pa   345    -7}\special{pa   344    -5}\special{pa   344    -4}\special{pa   343    -3}%
\special{pa   343    -1}\special{pa   343     0}\special{pa   343     1}\special{pa   343     3}%
\special{pa   344     4}\special{pa   344     5}\special{pa   345     7}\special{pa   346     8}%
\special{pa   347     9}\special{pa   348     9}\special{pa   350    10}\special{pa   351    11}%
\special{pa   352    11}\special{pa   354    11}\special{pa   355    11}\special{pa   356    11}%
\special{pa   358    11}\special{pa   359    10}\special{pa   360     9}\special{pa   361     9}%
\special{pa   363     8}\special{pa   363     7}\special{pa   364     5}\special{pa   365     4}%
\special{pa   365     3}\special{pa   365     1}\special{pa   366     0}%
\special{fp}%
\special{pa 306 -236}\special{pa 306 -238}\special{pa 306 -239}\special{pa 306 -240}%
\special{pa 305 -242}\special{pa 304 -243}\special{pa 303 -244}\special{pa 302 -245}%
\special{pa 301 -246}\special{pa 300 -246}\special{pa 299 -247}\special{pa 297 -247}%
\special{pa 296 -247}\special{pa 295 -247}\special{pa 293 -247}\special{pa 292 -247}%
\special{pa 290 -246}\special{pa 289 -246}\special{pa 288 -245}\special{pa 287 -244}%
\special{pa 286 -243}\special{pa 285 -242}\special{pa 285 -240}\special{pa 284 -239}%
\special{pa 284 -238}\special{pa 284 -236}\special{pa 284 -235}\special{pa 284 -233}%
\special{pa 285 -232}\special{pa 285 -231}\special{pa 286 -230}\special{pa 287 -229}%
\special{pa 288 -228}\special{pa 289 -227}\special{pa 290 -226}\special{pa 292 -226}%
\special{pa 293 -225}\special{pa 295 -225}\special{pa 296 -225}\special{pa 297 -225}%
\special{pa 299 -226}\special{pa 300 -226}\special{pa 301 -227}\special{pa 302 -228}%
\special{pa 303 -229}\special{pa 304 -230}\special{pa 305 -231}\special{pa 306 -232}%
\special{pa 306 -233}\special{pa 306 -235}\special{pa 306 -236}\special{pa 306 -236}%
\special{sh 1}\special{ip}%
\special{pa   306  -236}\special{pa   306  -238}\special{pa   306  -239}\special{pa   306  -240}%
\special{pa   305  -242}\special{pa   304  -243}\special{pa   303  -244}\special{pa   302  -245}%
\special{pa   301  -246}\special{pa   300  -246}\special{pa   299  -247}\special{pa   297  -247}%
\special{pa   296  -247}\special{pa   295  -247}\special{pa   293  -247}\special{pa   292  -247}%
\special{pa   290  -246}\special{pa   289  -246}\special{pa   288  -245}\special{pa   287  -244}%
\special{pa   286  -243}\special{pa   285  -242}\special{pa   285  -240}\special{pa   284  -239}%
\special{pa   284  -238}\special{pa   284  -236}\special{pa   284  -235}\special{pa   284  -233}%
\special{pa   285  -232}\special{pa   285  -231}\special{pa   286  -230}\special{pa   287  -229}%
\special{pa   288  -228}\special{pa   289  -227}\special{pa   290  -226}\special{pa   292  -226}%
\special{pa   293  -225}\special{pa   295  -225}\special{pa   296  -225}\special{pa   297  -225}%
\special{pa   299  -226}\special{pa   300  -226}\special{pa   301  -227}\special{pa   302  -228}%
\special{pa   303  -229}\special{pa   304  -230}\special{pa   305  -231}\special{pa   306  -232}%
\special{pa   306  -233}\special{pa   306  -235}\special{pa   306  -236}%
\special{fp}%
{%
\color[cmyk]{0,0,0,0}%
\special{pa 1381 0}\special{pa 1381 -3}\special{pa 1380 -6}\special{pa 1379 -8}\special{pa 1378 -11}%
\special{pa 1376 -13}\special{pa 1375 -15}\special{pa 1373 -17}\special{pa 1370 -19}%
\special{pa 1368 -20}\special{pa 1365 -21}\special{pa 1362 -22}\special{pa 1360 -22}%
\special{pa 1357 -22}\special{pa 1354 -22}\special{pa 1351 -21}\special{pa 1349 -20}%
\special{pa 1346 -19}\special{pa 1344 -17}\special{pa 1342 -15}\special{pa 1340 -13}%
\special{pa 1339 -11}\special{pa 1337 -8}\special{pa 1337 -6}\special{pa 1336 -3}%
\special{pa 1336 0}\special{pa 1336 3}\special{pa 1337 6}\special{pa 1337 8}\special{pa 1339 11}%
\special{pa 1340 13}\special{pa 1342 15}\special{pa 1344 17}\special{pa 1346 19}\special{pa 1349 20}%
\special{pa 1351 21}\special{pa 1354 22}\special{pa 1357 22}\special{pa 1360 22}\special{pa 1362 22}%
\special{pa 1365 21}\special{pa 1368 20}\special{pa 1370 19}\special{pa 1373 17}\special{pa 1375 15}%
\special{pa 1376 13}\special{pa 1378 11}\special{pa 1379 8}\special{pa 1380 6}\special{pa 1381 3}%
\special{pa 1381 0}\special{pa 1381 0}\special{sh 1}\special{ip}%
}%
\special{pa  1381    -0}\special{pa  1381    -3}\special{pa  1380    -6}\special{pa  1379    -8}%
\special{pa  1378   -11}\special{pa  1376   -13}\special{pa  1375   -15}\special{pa  1373   -17}%
\special{pa  1370   -19}\special{pa  1368   -20}\special{pa  1365   -21}\special{pa  1362   -22}%
\special{pa  1360   -22}\special{pa  1357   -22}\special{pa  1354   -22}\special{pa  1351   -21}%
\special{pa  1349   -20}\special{pa  1346   -19}\special{pa  1344   -17}\special{pa  1342   -15}%
\special{pa  1340   -13}\special{pa  1339   -11}\special{pa  1337    -8}\special{pa  1337    -6}%
\special{pa  1336    -3}\special{pa  1336     0}\special{pa  1336     3}\special{pa  1337     6}%
\special{pa  1337     8}\special{pa  1339    11}\special{pa  1340    13}\special{pa  1342    15}%
\special{pa  1344    17}\special{pa  1346    19}\special{pa  1349    20}\special{pa  1351    21}%
\special{pa  1354    22}\special{pa  1357    22}\special{pa  1360    22}\special{pa  1362    22}%
\special{pa  1365    21}\special{pa  1368    20}\special{pa  1370    19}\special{pa  1373    17}%
\special{pa  1375    15}\special{pa  1376    13}\special{pa  1378    11}\special{pa  1379     8}%
\special{pa  1380     6}\special{pa  1381     3}\special{pa  1381     0}%
\special{fp}%
\special{pa 306 -236}\special{pa 306 -238}\special{pa 306 -239}\special{pa 306 -240}%
\special{pa 305 -242}\special{pa 304 -243}\special{pa 303 -244}\special{pa 302 -245}%
\special{pa 301 -246}\special{pa 300 -246}\special{pa 299 -247}\special{pa 297 -247}%
\special{pa 296 -247}\special{pa 295 -247}\special{pa 293 -247}\special{pa 292 -247}%
\special{pa 290 -246}\special{pa 289 -246}\special{pa 288 -245}\special{pa 287 -244}%
\special{pa 286 -243}\special{pa 285 -242}\special{pa 285 -240}\special{pa 284 -239}%
\special{pa 284 -238}\special{pa 284 -236}\special{pa 284 -235}\special{pa 284 -233}%
\special{pa 285 -232}\special{pa 285 -231}\special{pa 286 -230}\special{pa 287 -229}%
\special{pa 288 -228}\special{pa 289 -227}\special{pa 290 -226}\special{pa 292 -226}%
\special{pa 293 -225}\special{pa 295 -225}\special{pa 296 -225}\special{pa 297 -225}%
\special{pa 299 -226}\special{pa 300 -226}\special{pa 301 -227}\special{pa 302 -228}%
\special{pa 303 -229}\special{pa 304 -230}\special{pa 305 -231}\special{pa 306 -232}%
\special{pa 306 -233}\special{pa 306 -235}\special{pa 306 -236}\special{pa 306 -236}%
\special{sh 1}\special{ip}%
\special{pa   306  -236}\special{pa   306  -238}\special{pa   306  -239}\special{pa   306  -240}%
\special{pa   305  -242}\special{pa   304  -243}\special{pa   303  -244}\special{pa   302  -245}%
\special{pa   301  -246}\special{pa   300  -246}\special{pa   299  -247}\special{pa   297  -247}%
\special{pa   296  -247}\special{pa   295  -247}\special{pa   293  -247}\special{pa   292  -247}%
\special{pa   290  -246}\special{pa   289  -246}\special{pa   288  -245}\special{pa   287  -244}%
\special{pa   286  -243}\special{pa   285  -242}\special{pa   285  -240}\special{pa   284  -239}%
\special{pa   284  -238}\special{pa   284  -236}\special{pa   284  -235}\special{pa   284  -233}%
\special{pa   285  -232}\special{pa   285  -231}\special{pa   286  -230}\special{pa   287  -229}%
\special{pa   288  -228}\special{pa   289  -227}\special{pa   290  -226}\special{pa   292  -226}%
\special{pa   293  -225}\special{pa   295  -225}\special{pa   296  -225}\special{pa   297  -225}%
\special{pa   299  -226}\special{pa   300  -226}\special{pa   301  -227}\special{pa   302  -228}%
\special{pa   303  -229}\special{pa   304  -230}\special{pa   305  -231}\special{pa   306  -232}%
\special{pa   306  -233}\special{pa   306  -235}\special{pa   306  -236}%
\special{fp}%
\special{pa 1316 -774}\special{pa 1299 -827}\special{pa 1282 -774}\special{pa 1299 -785}%
\special{pa 1316 -774}\special{pa 1316 -774}\special{sh 1}\special{ip}%
\special{pn 1}%
\special{pa  1316  -774}\special{pa  1299  -827}\special{pa  1282  -774}\special{pa  1299  -785}%
\special{pa  1316  -774}%
\special{fp}%
\special{pn 8}%
\special{pa  1299  -709}\special{pa  1299  -785}%
\special{fp}%
\special{pa  1535  -337}\special{pa  1606  -214}%
\special{fp}%
\special{pa  1535  -413}\special{pa  1535  -337}%
\special{fp}%
\special{pa  1535  -118}\special{pa  1535  -195}%
\special{fp}%
{%
\color[cmyk]{0,0,0,0}%
\special{pa 1558 -337}\special{pa 1558 -339}\special{pa 1557 -342}\special{pa 1556 -345}%
\special{pa 1555 -347}\special{pa 1554 -350}\special{pa 1552 -352}\special{pa 1550 -354}%
\special{pa 1547 -356}\special{pa 1545 -357}\special{pa 1542 -358}\special{pa 1540 -359}%
\special{pa 1537 -359}\special{pa 1534 -359}\special{pa 1531 -359}\special{pa 1528 -358}%
\special{pa 1526 -357}\special{pa 1523 -356}\special{pa 1521 -354}\special{pa 1519 -352}%
\special{pa 1517 -350}\special{pa 1516 -347}\special{pa 1515 -345}\special{pa 1514 -342}%
\special{pa 1513 -339}\special{pa 1513 -337}\special{pa 1513 -334}\special{pa 1514 -331}%
\special{pa 1515 -328}\special{pa 1516 -326}\special{pa 1517 -323}\special{pa 1519 -321}%
\special{pa 1521 -319}\special{pa 1523 -318}\special{pa 1526 -316}\special{pa 1528 -315}%
\special{pa 1531 -315}\special{pa 1534 -314}\special{pa 1537 -314}\special{pa 1540 -315}%
\special{pa 1542 -315}\special{pa 1545 -316}\special{pa 1547 -318}\special{pa 1550 -319}%
\special{pa 1552 -321}\special{pa 1554 -323}\special{pa 1555 -326}\special{pa 1556 -328}%
\special{pa 1557 -331}\special{pa 1558 -334}\special{pa 1558 -337}\special{pa 1558 -337}%
\special{sh 1}\special{ip}%
}%
\special{pa  1558  -337}\special{pa  1558  -339}\special{pa  1557  -342}\special{pa  1556  -345}%
\special{pa  1555  -347}\special{pa  1554  -350}\special{pa  1552  -352}\special{pa  1550  -354}%
\special{pa  1547  -356}\special{pa  1545  -357}\special{pa  1542  -358}\special{pa  1540  -359}%
\special{pa  1537  -359}\special{pa  1534  -359}\special{pa  1531  -359}\special{pa  1528  -358}%
\special{pa  1526  -357}\special{pa  1523  -356}\special{pa  1521  -354}\special{pa  1519  -352}%
\special{pa  1517  -350}\special{pa  1516  -347}\special{pa  1515  -345}\special{pa  1514  -342}%
\special{pa  1513  -339}\special{pa  1513  -337}\special{pa  1513  -334}\special{pa  1514  -331}%
\special{pa  1515  -328}\special{pa  1516  -326}\special{pa  1517  -323}\special{pa  1519  -321}%
\special{pa  1521  -319}\special{pa  1523  -318}\special{pa  1526  -316}\special{pa  1528  -315}%
\special{pa  1531  -315}\special{pa  1534  -314}\special{pa  1537  -314}\special{pa  1540  -315}%
\special{pa  1542  -315}\special{pa  1545  -316}\special{pa  1547  -318}\special{pa  1550  -319}%
\special{pa  1552  -321}\special{pa  1554  -323}\special{pa  1555  -326}\special{pa  1556  -328}%
\special{pa  1557  -331}\special{pa  1558  -334}\special{pa  1558  -337}%
\special{fp}%
{%
\color[cmyk]{0,0,0,0}%
\special{pa 1558 -195}\special{pa 1558 -198}\special{pa 1557 -200}\special{pa 1556 -203}%
\special{pa 1555 -206}\special{pa 1554 -208}\special{pa 1552 -210}\special{pa 1550 -212}%
\special{pa 1547 -214}\special{pa 1545 -215}\special{pa 1542 -216}\special{pa 1540 -217}%
\special{pa 1537 -217}\special{pa 1534 -217}\special{pa 1531 -217}\special{pa 1528 -216}%
\special{pa 1526 -215}\special{pa 1523 -214}\special{pa 1521 -212}\special{pa 1519 -210}%
\special{pa 1517 -208}\special{pa 1516 -206}\special{pa 1515 -203}\special{pa 1514 -200}%
\special{pa 1513 -198}\special{pa 1513 -195}\special{pa 1513 -192}\special{pa 1514 -189}%
\special{pa 1515 -187}\special{pa 1516 -184}\special{pa 1517 -182}\special{pa 1519 -180}%
\special{pa 1521 -178}\special{pa 1523 -176}\special{pa 1526 -175}\special{pa 1528 -174}%
\special{pa 1531 -173}\special{pa 1534 -172}\special{pa 1537 -172}\special{pa 1540 -173}%
\special{pa 1542 -174}\special{pa 1545 -175}\special{pa 1547 -176}\special{pa 1550 -178}%
\special{pa 1552 -180}\special{pa 1554 -182}\special{pa 1555 -184}\special{pa 1556 -187}%
\special{pa 1557 -189}\special{pa 1558 -192}\special{pa 1558 -195}\special{pa 1558 -195}%
\special{sh 1}\special{ip}%
}%
\special{pa  1558  -195}\special{pa  1558  -198}\special{pa  1557  -200}\special{pa  1556  -203}%
\special{pa  1555  -206}\special{pa  1554  -208}\special{pa  1552  -210}\special{pa  1550  -212}%
\special{pa  1547  -214}\special{pa  1545  -215}\special{pa  1542  -216}\special{pa  1540  -217}%
\special{pa  1537  -217}\special{pa  1534  -217}\special{pa  1531  -217}\special{pa  1528  -216}%
\special{pa  1526  -215}\special{pa  1523  -214}\special{pa  1521  -212}\special{pa  1519  -210}%
\special{pa  1517  -208}\special{pa  1516  -206}\special{pa  1515  -203}\special{pa  1514  -200}%
\special{pa  1513  -198}\special{pa  1513  -195}\special{pa  1513  -192}\special{pa  1514  -189}%
\special{pa  1515  -187}\special{pa  1516  -184}\special{pa  1517  -182}\special{pa  1519  -180}%
\special{pa  1521  -178}\special{pa  1523  -176}\special{pa  1526  -175}\special{pa  1528  -174}%
\special{pa  1531  -173}\special{pa  1534  -172}\special{pa  1537  -172}\special{pa  1540  -173}%
\special{pa  1542  -174}\special{pa  1545  -175}\special{pa  1547  -176}\special{pa  1550  -178}%
\special{pa  1552  -180}\special{pa  1554  -182}\special{pa  1555  -184}\special{pa  1556  -187}%
\special{pa  1557  -189}\special{pa  1558  -192}\special{pa  1558  -195}%
\special{fp}%
\special{pa  1467  -127}\special{pa  1361  -221}%
\special{fp}%
\special{pa  1476  -118}\special{pa  1467  -127}%
\special{fp}%
\special{pa  1358  -236}\special{pa  1367  -227}%
\special{fp}%
{%
\color[cmyk]{0,0,0,0}%
\special{pa 1490 -127}\special{pa 1490 -130}\special{pa 1489 -133}\special{pa 1488 -135}%
\special{pa 1487 -138}\special{pa 1486 -140}\special{pa 1484 -142}\special{pa 1482 -144}%
\special{pa 1479 -146}\special{pa 1477 -147}\special{pa 1474 -148}\special{pa 1472 -149}%
\special{pa 1469 -149}\special{pa 1466 -149}\special{pa 1463 -149}\special{pa 1460 -148}%
\special{pa 1458 -147}\special{pa 1455 -146}\special{pa 1453 -144}\special{pa 1451 -142}%
\special{pa 1449 -140}\special{pa 1448 -138}\special{pa 1447 -135}\special{pa 1446 -133}%
\special{pa 1445 -130}\special{pa 1445 -127}\special{pa 1445 -124}\special{pa 1446 -121}%
\special{pa 1447 -119}\special{pa 1448 -116}\special{pa 1449 -114}\special{pa 1451 -112}%
\special{pa 1453 -110}\special{pa 1455 -108}\special{pa 1458 -107}\special{pa 1460 -106}%
\special{pa 1463 -105}\special{pa 1466 -105}\special{pa 1469 -105}\special{pa 1472 -105}%
\special{pa 1474 -106}\special{pa 1477 -107}\special{pa 1479 -108}\special{pa 1482 -110}%
\special{pa 1484 -112}\special{pa 1486 -114}\special{pa 1487 -116}\special{pa 1488 -119}%
\special{pa 1489 -121}\special{pa 1490 -124}\special{pa 1490 -127}\special{pa 1490 -127}%
\special{sh 1}\special{ip}%
}%
\special{pa  1490  -127}\special{pa  1490  -130}\special{pa  1489  -133}\special{pa  1488  -135}%
\special{pa  1487  -138}\special{pa  1486  -140}\special{pa  1484  -142}\special{pa  1482  -144}%
\special{pa  1479  -146}\special{pa  1477  -147}\special{pa  1474  -148}\special{pa  1472  -149}%
\special{pa  1469  -149}\special{pa  1466  -149}\special{pa  1463  -149}\special{pa  1460  -148}%
\special{pa  1458  -147}\special{pa  1455  -146}\special{pa  1453  -144}\special{pa  1451  -142}%
\special{pa  1449  -140}\special{pa  1448  -138}\special{pa  1447  -135}\special{pa  1446  -133}%
\special{pa  1445  -130}\special{pa  1445  -127}\special{pa  1445  -124}\special{pa  1446  -121}%
\special{pa  1447  -119}\special{pa  1448  -116}\special{pa  1449  -114}\special{pa  1451  -112}%
\special{pa  1453  -110}\special{pa  1455  -108}\special{pa  1458  -107}\special{pa  1460  -106}%
\special{pa  1463  -105}\special{pa  1466  -105}\special{pa  1469  -105}\special{pa  1472  -105}%
\special{pa  1474  -106}\special{pa  1477  -107}\special{pa  1479  -108}\special{pa  1482  -110}%
\special{pa  1484  -112}\special{pa  1486  -114}\special{pa  1487  -116}\special{pa  1488  -119}%
\special{pa  1489  -121}\special{pa  1490  -124}\special{pa  1490  -127}%
\special{fp}%
{%
\color[cmyk]{0,0,0,0}%
\special{pa 1390 -227}\special{pa 1389 -230}\special{pa 1389 -233}\special{pa 1388 -236}%
\special{pa 1387 -238}\special{pa 1385 -240}\special{pa 1384 -243}\special{pa 1382 -245}%
\special{pa 1379 -246}\special{pa 1377 -248}\special{pa 1374 -249}\special{pa 1371 -249}%
\special{pa 1369 -250}\special{pa 1366 -250}\special{pa 1363 -249}\special{pa 1360 -249}%
\special{pa 1358 -248}\special{pa 1355 -246}\special{pa 1353 -245}\special{pa 1351 -243}%
\special{pa 1349 -240}\special{pa 1348 -238}\special{pa 1346 -236}\special{pa 1345 -233}%
\special{pa 1345 -230}\special{pa 1345 -227}\special{pa 1345 -224}\special{pa 1345 -222}%
\special{pa 1346 -219}\special{pa 1348 -216}\special{pa 1349 -214}\special{pa 1351 -212}%
\special{pa 1353 -210}\special{pa 1355 -208}\special{pa 1358 -207}\special{pa 1360 -206}%
\special{pa 1363 -205}\special{pa 1366 -205}\special{pa 1369 -205}\special{pa 1371 -205}%
\special{pa 1374 -206}\special{pa 1377 -207}\special{pa 1379 -208}\special{pa 1382 -210}%
\special{pa 1384 -212}\special{pa 1385 -214}\special{pa 1387 -216}\special{pa 1388 -219}%
\special{pa 1389 -222}\special{pa 1389 -224}\special{pa 1390 -227}\special{pa 1390 -227}%
\special{sh 1}\special{ip}%
}%
\special{pa  1390  -227}\special{pa  1389  -230}\special{pa  1389  -233}\special{pa  1388  -236}%
\special{pa  1387  -238}\special{pa  1385  -240}\special{pa  1384  -243}\special{pa  1382  -245}%
\special{pa  1379  -246}\special{pa  1377  -248}\special{pa  1374  -249}\special{pa  1371  -249}%
\special{pa  1369  -250}\special{pa  1366  -250}\special{pa  1363  -249}\special{pa  1360  -249}%
\special{pa  1358  -248}\special{pa  1355  -246}\special{pa  1353  -245}\special{pa  1351  -243}%
\special{pa  1349  -240}\special{pa  1348  -238}\special{pa  1346  -236}\special{pa  1345  -233}%
\special{pa  1345  -230}\special{pa  1345  -227}\special{pa  1345  -224}\special{pa  1345  -222}%
\special{pa  1346  -219}\special{pa  1348  -216}\special{pa  1349  -214}\special{pa  1351  -212}%
\special{pa  1353  -210}\special{pa  1355  -208}\special{pa  1358  -207}\special{pa  1360  -206}%
\special{pa  1363  -205}\special{pa  1366  -205}\special{pa  1369  -205}\special{pa  1371  -205}%
\special{pa  1374  -206}\special{pa  1377  -207}\special{pa  1379  -208}\special{pa  1382  -210}%
\special{pa  1384  -212}\special{pa  1385  -214}\special{pa  1387  -216}\special{pa  1388  -219}%
\special{pa  1389  -222}\special{pa  1389  -224}\special{pa  1390  -227}%
\special{fp}%
\settowidth{\Width}{$\mathrm{S_1}$}\setlength{\Width}{0\Width}%
\settoheight{\Height}{$\mathrm{S_1}$}\settodepth{\Depth}{$\mathrm{S_1}$}\setlength{\Height}{-0.5\Height}\setlength{\Depth}{0.5\Depth}\addtolength{\Height}{\Depth}%
\put(  2.767,  0.450){\hspace*{\Width}\raisebox{\Height}{$\mathrm{S_1}$}}%
%
\settowidth{\Width}{$\mathrm{S_2}$}\setlength{\Width}{0\Width}%
\settoheight{\Height}{$\mathrm{S_2}$}\settodepth{\Depth}{$\mathrm{S_2}$}\setlength{\Height}{-\Height}%
\put(  2.467,  0.100){\hspace*{\Width}\raisebox{\Height}{$\mathrm{S_2}$}}%
%
\end{picture}}%
    }
        図1のように,起電力$E$\tanni{V},内部抵抗$r$\tanni{\Omega }の電池Dに内部抵抗$r_\mathrm{V}$\tanni{\Omega }の電圧計Vを接続した。このときVが示す値は$E$ではなく,\Hako \tanni{V}である。\\
        ~~そこで,電池Dを,図2のように電池$\mathrm{E_0}$,抵抗線AB,検流計G,既知の起電力$E_\mathrm{S}$\tanni{V}をもつ標準電池$\mathrm{E_S}$およびスイッチ$\mathrm{S_1}$,$\mathrm{S_2}$を組み合わせた回路に接続した。ABは太さが一様で,接点Cの位置は調整でき,AC間の抵抗値が読み取れるようになっている。$\mathrm{S_1}$を開いたとき,ABに流れる電流を$I$\tanni{A}とする。$\mathrm{S_2}$を\maru{1}に入れた状態でGに電流が流れないようにCの位置を調整したときのAC間の抵抗値$R_\mathrm{S}$は$R_\mathrm{S}=$\Hako \tanni{\Omega }となる。\\
        ~~次に$\mathrm{S_2}$を\maru{2}に入れ,再びGに電流が流れないようにCの位置を調整したとき,AC間の抵抗値を$R$\tanni{\Omega }とすると,$E$は既知の$E_\mathrm{S}$,$R_\mathrm{S}$,$R$を用いて$E=$\Hako \tanni{V}と表すことができる。 
    \end{mawarikomi}