\item
    \begin{mawarikomi}{120pt}{
        %WinTpicVersion4.32a
{\unitlength 0.1in%
\begin{picture}(13.2500,14.4800)(2.7500,-32.0000)%
% VECTOR 2 0 3 0 Black White  
% 6 800 2800 800 1900 800 2800 400 3200 800 2800 1600 2800
% 
\special{pn 8}%
\special{pa 800 2800}%
\special{pa 800 1900}%
\special{fp}%
\special{sh 1}%
\special{pa 800 1900}%
\special{pa 780 1967}%
\special{pa 800 1953}%
\special{pa 820 1967}%
\special{pa 800 1900}%
\special{fp}%
\special{pa 800 2800}%
\special{pa 400 3200}%
\special{fp}%
\special{sh 1}%
\special{pa 400 3200}%
\special{pa 461 3167}%
\special{pa 438 3162}%
\special{pa 433 3139}%
\special{pa 400 3200}%
\special{fp}%
\special{pa 800 2800}%
\special{pa 1600 2800}%
\special{fp}%
\special{sh 1}%
\special{pa 1600 2800}%
\special{pa 1533 2780}%
\special{pa 1547 2800}%
\special{pa 1533 2820}%
\special{pa 1600 2800}%
\special{fp}%
% VECTOR 1 0 3 0 Black White  
% 2 800 2800 1300 2500
% 
\special{pn 13}%
\special{pa 800 2800}%
\special{pa 1300 2500}%
\special{fp}%
\special{sh 1}%
\special{pa 1300 2500}%
\special{pa 1233 2517}%
\special{pa 1254 2527}%
\special{pa 1253 2551}%
\special{pa 1300 2500}%
\special{fp}%
% CIRCLE 2 0 3 0 Black White  
% 4 800 2800 1000 2800 1200 2800 1300 2500
% 
\special{pn 8}%
\special{ar 800 2800 200 200 5.7427658 6.2831853}%
% STR 2 0 3 0 Black White  
% 4 1015 2730 1015 2780 2 0 0 0
% $\theta $
\put(10.1500,-27.8000){\makebox(0,0)[lb]{$\theta $}}%
% LINE 2 1 3 0 Black White  
% 4 1300 2500 1300 2800 1300 2500 800 2500
% 
\special{pn 8}%
\special{pa 1300 2500}%
\special{pa 1300 2800}%
\special{da 0.015}%
\special{pa 1300 2500}%
\special{pa 800 2500}%
\special{da 0.015}%
% STR 2 0 3 0 Black White  
% 4 980 2565 980 2615 5 0 0 0
% $v$
\put(9.8000,-26.1500){\makebox(0,0){$v$}}%
% VECTOR 2 0 3 0 Black White  
% 2 700 2300 700 2100
% 
\special{pn 8}%
\special{pa 700 2300}%
\special{pa 700 2100}%
\special{fp}%
\special{sh 1}%
\special{pa 700 2100}%
\special{pa 680 2167}%
\special{pa 700 2153}%
\special{pa 720 2167}%
\special{pa 700 2100}%
\special{fp}%
% VECTOR 2 0 3 0 Black White  
% 2 900 2300 900 2100
% 
\special{pn 8}%
\special{pa 900 2300}%
\special{pa 900 2100}%
\special{fp}%
\special{sh 1}%
\special{pa 900 2100}%
\special{pa 880 2167}%
\special{pa 900 2153}%
\special{pa 920 2167}%
\special{pa 900 2100}%
\special{fp}%
% VECTOR 2 0 3 0 Black White  
% 2 1100 2300 1100 2100
% 
\special{pn 8}%
\special{pa 1100 2300}%
\special{pa 1100 2100}%
\special{fp}%
\special{sh 1}%
\special{pa 1100 2100}%
\special{pa 1080 2167}%
\special{pa 1100 2153}%
\special{pa 1120 2167}%
\special{pa 1100 2100}%
\special{fp}%
% VECTOR 2 0 3 0 Black White  
% 2 1300 2300 1300 2100
% 
\special{pn 8}%
\special{pa 1300 2300}%
\special{pa 1300 2100}%
\special{fp}%
\special{sh 1}%
\special{pa 1300 2100}%
\special{pa 1280 2167}%
\special{pa 1300 2153}%
\special{pa 1320 2167}%
\special{pa 1300 2100}%
\special{fp}%
% STR 2 0 3 0 Black White  
% 4 1165 2150 1165 2200 5 0 0 0
% $B$
\put(11.6500,-22.0000){\makebox(0,0){$B$}}%
% STR 2 0 3 0 Black White  
% 4 815 2765 815 2815 1 0 0 0
% O
\put(8.1500,-28.1500){\makebox(0,0)[lt]{O}}%
% STR 2 0 3 0 Black White  
% 4 1670 2750 1670 2800 5 0 0 0
% $y$
\put(16.7000,-28.0000){\makebox(0,0){$y$}}%
% STR 2 0 3 0 Black White  
% 4 370 3175 370 3225 5 0 0 0
% $x$
\put(3.7000,-32.2500){\makebox(0,0){$x$}}%
% STR 2 0 3 0 Black White  
% 4 800 1775 800 1825 5 0 0 0
% $z$
\put(8.0000,-18.2500){\makebox(0,0){$z$}}%
\end{picture}}%

    }
$z$ 軸の正の方向に磁束密度 $B$ の一様な磁界がかかっている。質量が $m$ で電荷が $q(>0)$ の荷電粒子を、原点 $\mathrm{O}$ から $yz$ 面内で $y$ 軸から角度 $\theta$ の方向に一定速度 $v$ で打ち出した。重力の影響は無視する。

    \begin{enumerate}
        \item $y$ 軸の正の方向 ($\theta=0$) に打ち出した場合、荷電粒子は等速円運動をする。この等速円運動の中心点の座標 $(x_0, y_0, z_0)$ を求めよ。また、1周するのに要する時間はいくらか。
        \item $z$ 軸の正の方向 ($\theta=\bunsuu{\pi}{2}$) に打ち出した場合、この荷電粒子はどのような運動をするか説明せよ。
        \item $y$ 軸との角度 ($0<\theta<\bunsuu{\pi}{2}$) の方向に打ち出した場合について、
        \begin{enumerate}
            \item 荷電粒子はどのような運動をするか、説明せよ。
            \item 原点 $\mathrm{O}$ から荷電粒子が打ち出されてから、次に初めて $z$ 軸と交わるまでの時間を求めよ。また、この交点を $\mathrm{P}$ とするとき、$\mathrm{OP}$ 間の距離はいくらか。
        \end{enumerate}
    \end{enumerate}
\end{mawarikomi}
