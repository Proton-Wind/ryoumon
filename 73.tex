\hakosyokika
\item
    \begin{mawarikomi}(10pt,0){210pt}{{\small
\begin{zahyou*}[ul=5mm](0,12)(-1,1)
    \def\O{(0,0)}
    \def\A{(1,0)}
    \def\M{(4,0)}
    \def\B{(10,0)}
    \def\C{(12,0)}
    \ArrowLine\O\C
    \kuromaru{\A;\B}
    \Siromaru{\M}
    \Put\C[e]{$x$}
    \Put\A[s]{A}
    \Put\M[s]{M}
    \Put\B[s]{B}
    \HenKo<
    henkotype=parallel
    ,yazirusi=b
    ,henkosideb=0
    ,henkosidet=1.5
    ,henkoH=3ex
    >\B\A{250\sftanni{m}}
\end{zahyou*}}
}
        波源AとB,観測器Mが,ある媒質中の$x$軸上に置かれている。AとBは250\sftanni{m}離れており,それぞれ振幅$3.0$\sftanni{m},波長$16$\sftanni{m}の波を,互いに向かって送り出している。Mは$x$軸上波源AとBとの間で自由に動くことができ,その位置での波の振幅を観測する。AとBは同位相とし,波の減衰は無視する。
        \begin{enumerate}
            \item Mを静止させ,Aからの波だけを観測したところ,連続する2つの山の時間間隔は$4.0$\sftanni{s}であった。波の速さは何\sftanni{m/s}か。
            \item Mを正の向きに速さ$2.0$\sftanni{m/s}で動かしながらAからの波を観測した。このとき,連続する2つの山を観測する間隔は何\sftanni{s}か。
            \item Mを静止させ,AとBの2つの波の合成波を観測したところ,振幅が最大となる位置が複数あった。その最大振幅は何\sftanni{m}か。また,AとBの間で合成波の振幅が最大となる位置は何箇所あるか。
            \item Aから正の方向に75\sftanni{m}離れた位置に,自由端反射をする反射板Rを$x$軸に垂直に置いた。AR間で合成波を観測したところ,振幅が0となる点が複数あった。この内,Aに最も近い点はAから何\sftanni{m}離れているか。
        \end{enumerate}
    \end{mawarikomi}