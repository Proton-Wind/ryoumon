\hakosyokika
\item
    \begin{mawarikomi}(20pt,0pt){160pt}{%WinTpicVersion4.32a
{\unitlength 0.1in%
\begin{picture}(22.0000,19.5500)(6.0000,-26.0000)%
% CIRCLE 2 0 3 0 Black White  
% 4 1600 800 1600 2200 600 1800 2600 1800
% 
\special{pn 8}%
\special{ar 1600 800 1400 1400 0.7853982 2.3561945}%
% LINE 2 1 3 0 Black White  
% 4 1600 800 1600 2200 1600 800 2310 2000
% 
\special{pn 8}%
\special{pa 1600 800}%
\special{pa 1600 2200}%
\special{da 0.015}%
\special{pa 1600 800}%
\special{pa 2310 2000}%
\special{da 0.015}%
% STR 2 0 3 0 Black White  
% 4 1600 610 1600 710 5 0 0 0
% O
\put(16.0000,-7.1000){\makebox(0,0){O}}%
% STR 2 0 3 0 Black White  
% 4 2000 1120 2000 1220 5 0 0 0
% $R$
\put(20.0000,-12.2000){\makebox(0,0){$R$}}%
% STR 2 0 3 0 Black White  
% 4 1600 2210 1600 2310 5 0 0 0
% C
\put(16.0000,-23.1000){\makebox(0,0){C}}%
% BOX 2 0 3 0 Black White  
% 2 600 2200 2600 2400
% 
\special{pn 8}%
\special{pa 600 2200}%
\special{pa 2600 2200}%
\special{pa 2600 2400}%
\special{pa 600 2400}%
\special{pa 600 2200}%
\special{pa 2600 2200}%
\special{fp}%
% LINE 2 0 3 0 Black White  
% 6 600 1800 600 1700 600 1700 2600 1700 2600 1700 2600 1800
% 
\special{pn 8}%
\special{pa 600 1800}%
\special{pa 600 1700}%
\special{fp}%
\special{pa 600 1700}%
\special{pa 2600 1700}%
\special{fp}%
\special{pa 2600 1700}%
\special{pa 2600 1800}%
\special{fp}%
% LINE 2 2 3 0 Black White  
% 4 1600 2000 2800 2000 2800 2200 1600 2200
% 
\special{pn 8}%
\special{pa 1600 2000}%
\special{pa 2800 2000}%
\special{dt 0.025}%
\special{pa 2800 2200}%
\special{pa 1600 2200}%
\special{dt 0.025}%
% STR 2 0 3 0 Black White  
% 4 2700 2000 2700 2100 5 0 0 0
% $d$
\put(27.0000,-21.0000){\makebox(0,0){$d$}}%
% VECTOR 2 0 3 0 Black White  
% 4 2700 1800 2700 2000 2700 2400 2700 2200
% 
\special{pn 8}%
\special{pa 2700 1800}%
\special{pa 2700 2000}%
\special{fp}%
\special{sh 1}%
\special{pa 2700 2000}%
\special{pa 2720 1933}%
\special{pa 2700 1947}%
\special{pa 2680 1933}%
\special{pa 2700 2000}%
\special{fp}%
\special{pa 2700 2400}%
\special{pa 2700 2200}%
\special{fp}%
\special{sh 1}%
\special{pa 2700 2200}%
\special{pa 2680 2267}%
\special{pa 2700 2253}%
\special{pa 2720 2267}%
\special{pa 2700 2200}%
\special{fp}%
% LINE 2 2 3 0 Black White  
% 2 2300 2000 2300 2600
% 
\special{pn 8}%
\special{pa 2300 2000}%
\special{pa 2300 2600}%
\special{dt 0.025}%
% LINE 2 2 3 0 Black White  
% 2 1600 2400 1600 2600
% 
\special{pn 8}%
\special{pa 1600 2400}%
\special{pa 1600 2600}%
\special{dt 0.025}%
% VECTOR 2 0 3 0 Black White  
% 4 2000 2500 1600 2500 2000 2500 2300 2500
% 
\special{pn 8}%
\special{pa 2000 2500}%
\special{pa 1600 2500}%
\special{fp}%
\special{sh 1}%
\special{pa 1600 2500}%
\special{pa 1667 2520}%
\special{pa 1653 2500}%
\special{pa 1667 2480}%
\special{pa 1600 2500}%
\special{fp}%
\special{pa 2000 2500}%
\special{pa 2300 2500}%
\special{fp}%
\special{sh 1}%
\special{pa 2300 2500}%
\special{pa 2233 2480}%
\special{pa 2247 2500}%
\special{pa 2233 2520}%
\special{pa 2300 2500}%
\special{fp}%
% STR 2 0 3 0 Black White  
% 4 1970 2400 1970 2500 5 0 1 0
% $r$
\put(19.7000,-25.0000){\makebox(0,0){{\colorbox[named]{White}{$r$}}}}%
\end{picture}}%
}
    平面ガラスの板の上に,大きい曲率半径$R$をもつ平凸レンズをのせ,上から波長$\lambda $の単色光をあてて上から見ると,レンズとガラス板の接点Cを中心とする明暗の輪が同心円状に並んでいるのが見える(ニュートンリング)。
        \begin{Enumerate}
            \item 輪の半径を$r$とする。その位置での空気層の厚さ$d$を$R$,$r$を用いて表せ。ただし,$d$は$R$に比べて十分小さいとする。
            \item 平凸レンズの中心部は明るく見えるか,暗く見えるか。また,青色の光と赤色の光では,輪の半径はどちらが大きいか。
        \end{Enumerate}
    $\lambda = 540$\sftanni{nm}の光を用いたところ,中心から3番目の明輪が$r=3.0$\sftanni{mm}の位置に見えた。
        \begin{Enumerate*}
            \item 平凸レンズの曲率半径$R$\sftanni{m}を求めよ。
            \item 平凸レンズと平面ガラスの間に,ある液体を満たして,今度はガラスの下から単色光をあててレンズの上から見るとする。この場合,ニュートンリングはどのように見えるか簡潔に述べよ。
        \end{Enumerate*}
    \end{mawarikomi}
